\chapter{生物地球化学循环预热加速}\label{生物地球化学循环预热加速}
%\addcontentsline{toc}{chapter}{生物地球化学循环预热加速}

%\begin{生物地球化学循环预热加速}
生物地球化学循环预热是陆地生态系统碳氮循环模拟必不可少的初始化过程,是众多模式比较计划都采用的标准初始化流程。
生物地球化学循环预热通过重复使用同样的气候强迫场和大气二氧化碳浓度数据,使陆地生态系统碳氮储量达到平衡。
通常来说,不同模型间平衡态碳氮储量的差异往往大于历史时期由于$\rm CO_2$浓度上升引起的碳氮储量变化的不确定性。
因此,生物地球化学循环预热过程十分重要。但由于高纬度地区的低温气候造成土壤分解过慢,
碳氮预热达到平衡态需要上千年的模拟时间,如何加速生物地球化学循环预热过程十分关键。
CoLM生物地球化学循环模块运用半解析预热方法~\citep{xia2012semi}。从碳氮平衡方程入手,求解平衡态的碳氮库大小。


\section{植被生物地球化学循环预热加速}
CoLM植被碳氮循环可以表示为21个碳库和22个氮库的碳氮平衡方程组~\citep{lu2020full},
可以写成矩阵形式:
\begin{equation}
\frac{d C_{veg}}{dt}=B I_{Cin}+\left(A_{p h c}(t) K_{p h c}+A_{gmc}(t) K_{gmc}\right) C_{veg}(t)
\end{equation}
\begin{equation}
\frac{d N_{\text {veg}}}{dt}=B I_{Nin}+\left(A_{phn}(t) K_{phn}+A_{gmn}(t) K_{gmn}\right) N_{\text {veg }}(t)
\end{equation}
$C_{veg}$ 和$N_{veg}$ 是植被碳氮库的状态变量 (\unit{g.C.m^{-2}} 和 \unit{g.N.m^{-2}}),是长度分别为21和22的列向量,
具体内容见章节~\ref{植被碳氮库结构} 对植被碳氮库的详细介绍。$I_{Cin}$ 和$I_{Nin}$ 分别是植被的碳氮输入(\unit{g.C.m^{-2}.s^{-1}} 和 \unit{g.N.m^{-2}.s^{-1}}),
是标量。其中,碳输入来源于净第一性生产力,氮输入来源于生物固氮和植被被动氮吸收。
$B$是分配系数向量,代表植被碳氮输入分配到每个植被碳氮库的比例。  
$K_{phc}$和$K_{gmc}$是21$\times$21的对角矩阵,分别代表植被每个库因为物候过程和自然死亡过程产生的碳周转速率(\unit{s^{-1}})。
$K_{phn}$和$K_{gmn}$是22$\times$22的对角矩阵,分别代表植被每个库因为物候过程和自然死亡过程产生的氮周转速率(\unit{s^{-1}}):
\begin{equation}
K_{p h c}=\left(\begin{array}{ccc}k_{p 1} & \cdots & 0 \\ \vdots & \ddots & \vdots \\ 0 & \cdots & k_{p 21}\end{array}\right)
\end{equation}
\begin{equation}
K_{gmc}=\left(\begin{array}{ccc}k_{g 1} & \cdots & 0 \\ \vdots & \ddots & \vdots \\ 0 & \cdots & k_{n 21}\end{array}\right)
\end{equation}
\begin{equation}
K_{phn}=\left(\begin{array}{ccc}k_{p 1} & \cdots & 0 \\ \vdots & \ddots & \vdots \\ 0 & \cdots & k_{p 22}\end{array}\right)
\end{equation}
\begin{equation}
K_{gmn}=\left(\begin{array}{ccc}k_{g 1} & \cdots & 0 \\ \vdots & \ddots & \vdots \\ 0 & \cdots & k_{n 22}\end{array}\right)
\end{equation}
$A_{phc}$和$A_{phn}$分别是碳氮传输系数矩阵,代表不同植被碳氮库之间的传输比例。

\afterpage{%\clearpage %
\begin{landscape}
\enlargethispage{10pt}
\begin{equation}
  A_{phc}=\left(\begin{array}{rcccccccccccccccccc}
      -1 & 0 & a_{1,3} & 0 & 0 & 0 & 0 & 0 & 0 & 0 & 0 & 0 & 0 & 0 & 0 & 0 & 0 & 0 & 0 \\
       0 & -1 & 0 & 0 & 0 & 0 & 0 & 0 & 0 & 0 & 0 & 0 & 0 & 0 & 0 & 0 & 0 & 0 & 0 \\ 
       0 & a_{3,2} & -1 & 0 & 0 & 0 & 0 & 0 & 0 & 0 & 0 & 0 & 0 & 0 & 0 & 0 & 0 & 0 & 0 \\
       0 & 0 & 0 & -1 & 0 & a_{4,6} & 0 & 0 & 0 & 0 & 0 & 0 & 0 & 0 & 0 & 0 & 0 & 0 & 0 \\ 
       0 & 0 & 0 & 0 & -1 & 0 & 0 & 0 & 0 & 0 & 0 & 0 & 0 & 0 & 0 & 0 & 0 & 0 & 0 \\ 
       0 & 0 & 0 & 0 & a_{6,5} & -1 & 0 & 0 & 0 & 0 & 0 & 0 & 0 & 0 & 0 & 0 & 0 & 0 & 0 \\
       0 & 0 & 0 & 0 & 0 & 0 & -1 & 0 & a_{7,9} & 0 & 0 & 0 & 0 & 0 & 0 & 0 & 0 & 0 & 0 \\ 
       0 & 0 & 0 & 0 & 0 & 0 & 0 & -1 & 0 & 0 & 0 & 0 & 0 & 0 & 0 & 0 & 0 & 0 & 0 \\
       0 & 0 & 0 & 0 & 0 & 0 & 0 & a_{9,8} & -1 & 0 & 0 & 0 & 0 & 0 & 0 & 0 & 0 & 0 & 0 \\
       0 & 0 & 0 & 0 & 0 & 0 & a_{10,7} & 0 & 0 & -1 & 0 & a_{10,12} & 0 & 0 & 0 & 0 & 0 & 0 & 0 \\
       0 & 0 & 0 & 0 & 0 & 0 & 0 & 0 & 0 & 0 & -1 & 0 & 0 & 0 & 0 & 0 & 0 & 0 &0 \\
       0 & 0 & 0 & 0 & 0 & 0 & 0 & 0 & 0 & 0 & a_{12,11}&-1 & 0 & 0 & 0 & 0 & 0 & 0 & 0\\
       0 & 0 & 0 & 0 & 0 & 0 & 0 & 0 & 0 & 0 & 0 & 0 & -1 & 0 & a_{13,15} & 0 & 0 & 0 & 0\\
       0 & 0 & 0 & 0 & 0 & 0 & 0 & 0 & 0 & 0 & 0 & 0 & 0 & -1 & 0 & 0 & 0 & 0 & 0 \\
       0 & 0 & 0 & 0 & 0 & 0 & 0 & 0 & 0 & 0 & 0 & 0 & 0 & a_{15,14}&-1 & 0 & 0 & 0 & 0\\
       0 & 0 & 0 & 0 & 0 & 0 & 0 & 0 & 0 & 0 & 0 & 0 & 0 & a_{16,13} & 0 & 0 & -1 & 0 & a_{16,18} \\
       0 & 0 & 0 & 0 & 0 & 0 & 0 & 0 & 0 & 0 & 0 & 0 & 0 & 0 & 0 & 0 & 0&-1 & 0\\
       0 & 0 & 0 & 0 & 0 & 0 & 0 & 0 & 0 & 0 & 0 & 0 & 0 & 0 & 0 & 0 & 0 &  a_{18,17} & -1\end{array}\right)
  \end{equation}
\end{landscape}
}

%\afterpage{%\clearpage %
\begin{landscape}
\enlargethispage{55pt}
  \begin{equation}
    A_{phn}=\left(\begin{array}{rcccccccccccccccccc}
      -1 & 0 & a_{1,3} & 0 & 0 & 0 & 0 & 0 & 0 & 0 & 0 & 0 & 0 & 0 & 0 & 0 & 0 & 0 & a_{1,19} \\ 
      0 & -1 & 0 & 0 & 0 & 0 & 0 & 0 & 0 & 0 & 0 & 0 & 0 & 0 & 0 & 0 & 0 & 0 & a_{2,19} \\ 
      0 & a_{3,2} & -1 & 0 & 0 & 0 & 0 & 0 & 0 & 0 & 0 & 0 & 0 & 0 & 0 & 0 & 0 & 0 & a_{3,19} \\ 
      0 & 0 & 0 & -1 & 0 & a_{4,6} & 0 & 0 & 0 & 0 & 0 & 0 & 0 & 0 & 0 & 0 & 0 & 0 & a_{4,19} \\ 
      0 & 0 & 0 & 0 & -1 & 0 & 0 & 0 & 0 & 0 & 0 & 0 & 0 & 0 & 0 & 0 & 0 & 0 & a_{5,19} \\ 
      0 & 0 & 0 & 0 & a_{6,5} & -1 & 0 & 0 & 0 & 0 & 0 & 0 & 0 & 0 & 0 & 0 & 0 & 0 & a_{6,19} \\
      0 & 0 & 0 & 0 & 0 & 0 & -1 & 0 & a_{7,9} & 0 & 0 & 0 & 0 & 0 & 0 & 0 & 0 & 0 & a_{7,19} \\
      0 & 0 & 0 & 0 & 0 & 0 & 0 & -1 & 0 & 0 & 0 & 0 & 0 & 0 & 0 & 0 & 0 & 0 & a_{8,19} \\ 
      0 & 0 & 0 & 0 & 0 & 0 & 0 & a_{9,8} & -1 & 0 & 0 & 0 & 0 & 0 & 0 & 0 & 0 & 0 & a_{9,19} \\
      0 & 0 & 0 & 0 & 0 & 0 & a_{10,7} & 0 & 0 & -1 & 0 & a_{10,12} & 0 & 0 & 0 & 0 & 0 & 0 & a_{10,19} \\
      0 & 0 & 0 & 0 & 0 & 0 & 0 & 0 & 0 & 0 & -1 & 0 & 0 & 0 & 0 & 0 & 0 & 0 & a_{11,19} \\ 
      0 & 0 & 0 & 0 & 0 & 0 & 0 & 0 & 0 & 0 & a_{12,11}&-1 & 0 & 0 & 0 & 0 & 0 & 0 & a_{12,19} \\ 
      0 & 0 & 0 & 0 & 0 & 0 & 0 & 0 & 0 & 0 & 0 & 0 & -1 & 0 & a_{13,15} & 0 & 0 & 0 & a_{13,19} \\ 
      0 & 0 & 0 & 0 & 0 & 0 & 0 & 0 & 0 & 0 & 0 & 0 & 0 & -1 & 0 & 0 & 0 & 0 & a_{14,19} \\ 
      0 & 0 & 0 & 0 & 0 & 0 & 0 & 0 & 0 & 0 & 0 & 0 & 0 & a_{15,14}&-1 & 0 & 0 & 0 & a_{15,19} \\
      0 & 0 & 0 & 0 & 0 & 0 & 0 & 0 & 0 & 0 & 0 & 0 & a_{16,13} & 0 & 0 & -1 & 0 & a_{16,18}& a_{16,19} \\ 
      0 & 0 & 0 & 0 & 0 & 0 & 0 & 0 & 0 & 0 & 0 & 0 & 0 & 0 & 0 & 0 & -1 & 0 & a_{17,19} \\ 
      0 & 0 & 0 & 0 & 0 & 0 & 0 & 0 & 0 & 0 & 0 & 0 & 0 & 0 & 0 & 0 & a_{18,17}&-1 & a_{18,19} \\ 
      a_{19,1} & 0 & 0 & a_{19,4} & 0 & 0 & a_{19,7} & 0 & 0 & a_{19,10} & 0 & 0 & 0 & 0 & 0 & 0 & 0 & 0 & -1
    \end{array}\right)
    \end{equation}
  \end{landscape}
%}

    \begin{equation}
      A_{gmc}=\left(\begin{array}{rrrrrrrrrrrrrrrrrrrrrrrrrrrrrr}
        -1 & 0 & 0 & 0 & 0 & 0 & 0 & 0 & 0 & 0 & 0 & 0 & 0 & 0 & 0 & 0 & 0 & 0 \\ 
        0 & -1 & 0 & 0 & 0 & 0 & 0 & 0 & 0 & 0 & 0 & 0 & 0 & 0 & 0 & 0 & 0 & 0 \\ 
        0 & 0 & -1 & 0 & 0 & 0 & 0 & 0 & 0 & 0 & 0 & 0 & 0 & 0 & 0 & 0 & 0 & 0 \\
        0 & 0 & 0 & -1 & 0 & 0 & 0 & 0 & 0 & 0 & 0 & 0 & 0 & 0 & 0 & 0 & 0 & 0 \\ 
        0 & 0 & 0 & 0 & -1 & 0 & 0 & 0 & 0 & 0 & 0 & 0 & 0 & 0 & 0 & 0 & 0 & 0 \\ 
        0 & 0 & 0 & 0 & 0 & -1 & 0 & 0 & 0 & 0 & 0 & 0 & 0 & 0 & 0 & 0 & 0 & 0 \\ 
        0 & 0 & 0 & 0 & 0 & 0 & -1 & 0 & 0 & 0 & 0 & 0 & 0 & 0 & 0 & 0 & 0 & 0 \\ 
        0 & 0 & 0 & 0 & 0 & 0 & 0 & -1 & 0 & 0 & 0 & 0 & 0 & 0 & 0 & 0 & 0 & 0 \\
        0 & 0 & 0 & 0 & 0 & 0 & 0 & 0 & -1 & 0 & 0 & 0 & 0 & 0 & 0 & 0 & 0 & 0 \\
        0 & 0 & 0 & 0 & 0 & 0 & 0 & 0 & 0 & -1 & 0 & 0 & 0 & 0 & 0 & 0 & 0 & 0 \\ 
        0 & 0 & 0 & 0 & 0 & 0 & 0 & 0 & 0 & 0 & -1 & 0 & 0 & 0 & 0 & 0 & 0 & 0 \\ 
        0 & 0 & 0 & 0 & 0 & 0 & 0 & 0 & 0 & 0 & 0 & -1 & 0 & 0 & 0 & 0 & 0 & 0 \\ 
        0 & 0 & 0 & 0 & 0 & 0 & 0 & 0 & 0 & 0 & 0 & 0 & -1 & 0 & 0 & 0 & 0 & 0 \\ 
        0 & 0 & 0 & 0 & 0 & 0 & 0 & 0 & 0 & 0 & 0 & 0 & 0 & -1 & 0 & 0 & 0 & 0 \\ 
        0 & 0 & 0 & 0 & 0 & 0 & 0 & 0 & 0 & 0 & 0 & 0 & 0 & 0 & -1 & 0 & 0 & 0 \\ 
        0 & 0 & 0 & 0 & 0 & 0 & 0 & 0 & 0 & 0 & 0 & 0 & 0 & 0 & 0 & -1 & 0 & 0 \\ 
        0 & 0 & 0 & 0 & 0 & 0 & 0 & 0 & 0 & 0 & 0 & 0 & 0 & 0 & 0 & 0 & -1 & 0 \\ 
        0 & 0 & 0 & 0 & 0 & 0 & 0 & 0 & 0 & 0 & 0 & 0 & 0 & 0 & 0 & 0 & 0  & -1
      \end{array}\right)
      \end{equation}
      
      \begin{equation}
        A_{gmn}=\left(\begin{array}{rrrrrrrrrrrrrrrrrrrrrrrrrrrrrr}
          -1 & 0 & 0 & 0 & 0 & 0 & 0 & 0 & 0 & 0 & 0 & 0 & 0 & 0 & 0 & 0 & 0 & 0 \\
           0 & -1 & 0 & 0 & 0 & 0 & 0 & 0 & 0 & 0 & 0 & 0 & 0 & 0 & 0 & 0 & 0 & 0 \\ 
           0 & 0 & -1 & 0 & 0 & 0 & 0 & 0 & 0 & 0 & 0 & 0 & 0 & 0 & 0 & 0 & 0 & 0 \\ 
           0 & 0 & 0 & -1 & 0 & 0 & 0 & 0 & 0 & 0 & 0 & 0 & 0 & 0 & 0 & 0 & 0 & 0 \\ 
           0 & 0 & 0 & 0 & -1 & 0 & 0 & 0 & 0 & 0 & 0 & 0 & 0 & 0 & 0 & 0 & 0 & 0 \\
           0 & 0 & 0 & 0 & 0 & -1 & 0 & 0 & 0 & 0 & 0 & 0 & 0 & 0 & 0 & 0 & 0 & 0 \\ 
           0 & 0 & 0 & 0 & 0 & 0 & -1 & 0 & 0 & 0 & 0 & 0 & 0 & 0 & 0 & 0 & 0 & 0 \\ 
           0 & 0 & 0 & 0 & 0 & 0 & 0 & -1 & 0 & 0 & 0 & 0 & 0 & 0 & 0 & 0 & 0 & 0 \\ 
           0 & 0 & 0 & 0 & 0 & 0 & 0 & 0 & -1 & 0 & 0 & 0 & 0 & 0 & 0 & 0 & 0 & 0 \\ 
           0 & 0 & 0 & 0 & 0 & 0 & 0 & 0 & 0 & -1 & 0 & 0 & 0 & 0 & 0 & 0 & 0 & 0 \\ 
           0 & 0 & 0 & 0 & 0 & 0 & 0 & 0 & 0 & 0 & -1 & 0 & 0 & 0 & 0 & 0 & 0 & 0 \\ 
           0 & 0 & 0 & 0 & 0 & 0 & 0 & 0 & 0 & 0 & 0 & -1 & 0 & 0 & 0 & 0 & 0 & 0 \\ 
           0 & 0 & 0 & 0 & 0 & 0 & 0 & 0 & 0 & 0 & 0 & 0 & -1 & 0 & 0 & 0 & 0 & 0 \\ 
           0 & 0 & 0 & 0 & 0 & 0 & 0 & 0 & 0 & 0 & 0 & 0 & 0 & -1 & 0 & 0 & 0 & 0 \\ 
           0 & 0 & 0 & 0 & 0 & 0 & 0 & 0 & 0 & 0 & 0 & 0 & 0 & 0 & -1 & 0 & 0 & 0 \\
           0 & 0 & 0 & 0 & 0 & 0 & 0 & 0 & 0 & 0 & 0 & 0 & 0 & 0 & 0 & -1 & 0 & 0 \\ 
           0 & 0 & 0 & 0 & 0 & 0 & 0 & 0 & 0 & 0 & 0 & 0 & 0 & 0 & 0 & 0 & -1 &0 \\
           0 & 0 & 0 & 0 & 0 & 0 & 0 & 0 & 0 & 0 & 0 & 0 & 0 & 0 & 0 & 0 & 0 & -1\end{array}\right)
        \end{equation}
其中,矩阵的非对角元素 $a_{(i,j)}$, 代表从$j$到$i$的碳氮传输在$j$库总周转掉的碳氮中的比例。


当碳氮平衡方程中$(dC_{veg})/dt=0$和 $(dN_{veg})/dt=0$,则碳氮处于平衡状态,所以可以解得植被碳氮的平衡状态($C_{(veg,ss)}$, $N_{(veg,ss)}$)是:
\begin{equation}
C_{veg, s s}=\left(A_{p h c}(t) K_{p h c}+A_{gmc}(t) K_{gmc}\right)^{-1} B I_{{Cin }}
\end{equation}
\begin{equation}
N_{veg, s s}=\left(A_{phn}(t) K_{phn}+A_{gmn}(t) K_{gmn}\right)^{-1} B I_{Nin}
\end{equation}
由于生物地球化学循环运用循环的气象强迫场数据,所以平衡状态需要累加计算一个气象循环的平均值。


\section{土壤凋落物生物地球化学循环预热加速}
CoLM土壤凋落物的碳氮循环可以表示为7个碳库和7个氮库的碳氮平衡方程组~\citep{lu2020full},其中每个库在土壤垂直方向分为10层,可以写成矩阵形式:
\begin{equation}
\frac{d C_{\text {soil }}}{dt}=I_{C \text { soil }}+\left(A_{h c} \xi(t) K_{h}+V(t)\right) C_{\text {soil }}(t)
\end{equation}
\begin{equation}
\frac{d N_{\text {soil }}}{dt}=I_{N s o i l}+\left(A_{h n} \xi(t) K_{h}+V(t)\right) N_{\text {soil }}(t)
\end{equation}
$C_{soil}$和$N_{soil}$是CoLM土壤凋落物有机碳氮库的状态变量(\unit{g.C.m^{-3}} 和 \unit{g.N.m^{-3}}), 是向量。
$I_{Csoil}$和$I_{Nsoil}$同样是长度为70的向量,代表进入土壤碳氮库中不同分层的凋落物输入。
 $A_{hc}$ 和$A_{hn}$代表土壤碳氮在同一层内不同库之间的传输系数。$V$土壤碳氮在不同层之间的传输系数:
\begin{equation}
A_{h c} \text { or } A_{h n}=\left(\begin{array}{ccccccc}
  A_{11} & 0 & 0 & 0 & 0 & 0 & 0 \\ 
  0 & A_{22} & 0 & 0 & 0 & 0 & 0 \\
  A_{31} & 0 & A_{33} & 0 & 0 & 0 & 0 \\
  A_{41} & 0 & 0 & A_{44} & 0 & 0 & 0 \\
  0 & A_{52} & A_{53} & 0 & A_{55} & A_{56} & A_{57} \\
  0 & 0 & 0 & A_{64} & A_{65} & A_{66} & 0 \\ 
  0 & 0 & 0 & 0 & A_{75} & A_{76} & A_{77}\end{array}\right)
\end{equation}
其中每个子矩阵 $A_{ij}$ 都是10$\times$10的对角矩阵:  

\begin{equation}
A_{i j}=\left\{\begin{array}{ccc}{\left[\begin{array}{ccc}-1 & \cdots & 0 \\ \vdots & \ddots & \vdots \\ 0 & \cdots & -1\end{array}\right]}   &i=j;\ C\ and\ N\ cycles \\
 {\left[\begin{array}{ccc}\left(1-r f_{j}\right) T_{i j} & \cdots & \\ \vdots & \ddots & \vdots \\ 0 & \cdots & \left(1-r f_{j}\right) T_{i j}\end{array}\right]}   &\ i\neq j;\ C\ cycle\ \left(A_{hc}\right)\  \\ 
 {\left[\begin{array}{cccc}\left(1-r f_{j}\right) T_{i j} \frac{C N_{j}(1)}{C N_{i}(1)} & \cdots & 0 \\ \vdots & \ddots & \vdots \\ 0 & & \cdots & \left(1-r f_{j}\right) T_{i j} \frac{C N_{j}(10)}{C N_{i}(10)}\end{array}\right]}& i\neq j;\ N\ cycle\left(A_{hn}\right) \end{array}\right.
\end{equation}


对角线上的子矩阵 $A_{ii}$是负的单位矩阵。非对角子矩阵$A_{ij}$代表10层土壤中的传输系数的差异。
$r_{ij}$, $T_{ij}$,$CN_{j(z)}$以及环境控制因子$\xi$在第~\ref{土壤凋落物生物地球化学循环过程} 章都有过详细介绍。

周转速率矩阵$K_h$是对角矩阵,代表土壤分解过程在不同土壤分层引起的碳氮周转:
\begin{equation}
K_{h}=\left(\begin{array}{ccc}k_{1} & \cdots & 0 \\ \vdots & \ddots & \vdots \\ 0 & \cdots & k_{n}\end{array}\right)
\end{equation}
垂直传输系数矩阵V是由6个同样的三对角矩阵组成$v$:
\begin{equation}
V(t)=\left(\begin{array}{ccccccc}v & 0 & 0 & 0 & 0 & 0 & 0 \\ 0 & v & 0 & 0 & 0 & 0 & 0 \\ 0 & 0 & v & 0 & 0 & 0 & 0 \\ 0 & 0 & 0 & 0 & 0 & 0 & 0 \\ 0 & 0 & 0 & 0 & v & 0 & 0 \\ 0 & 0 & 0 & 0 & 0 & v & 0 \\ 0 & 0 & 0 & 0 & 0 & 0 & v\end{array}\right)
\end{equation}
其中CoLM的粗木质残体库不存在土壤的垂直传输,所以其对应的矩阵$v$是0。
其余库对应的矩阵都是三对角矩阵,意味着碳氮的垂直传输只发生在相邻的两层土壤之间:
\begin{equation}
\begin{array}{c}v=\operatorname{diag}\left(d z_{1}, d z_{2}, \ldots,  d z_{20}\right)^{-1} \\ 
  \left(\begin{array}{ccccccc}g_{1} & -g_{1} & 0 & & 0 & 0 & 0 \\
     -h_{2} & h_{2}+g_{2} & -g_{2} & \ldots & 0 & 0 & 0 \\ 0 & -h_{3} & h_{3}+g_{3} & & 0 & 0 & 0 \\
      & \vdots & & \ddots & & \vdots & \\ 0 & 0 & 0 & & h_{18}+g_{18} & -g_{18} & 0 \\
     0 & 0 & 0 & \ldots & -h_{19} & h_{19}+g_{19} & -g_{19} \\ 0 & 0 & 0 & & 0 & -h_{10} & h_{20}\end{array}\right)\end{array}
\end{equation}
其中$dz$代表每层土壤的厚度,$g$和$h$是土壤碳氮的上行和下行垂直传输系数。

当碳氮平衡方程中$(dC_{soil})/dt=0$和 $(dN_{soil})/dt=0$,则碳氮处于平衡状态,
所以可以解得植被碳氮的平衡状态($C_{(soil,ss)}$, $N_{(soil,ss)}$)是:
%
\begin{equation}
C_{soil, s s}=\left(A_{h c} \xi(t) K_{h}+V(t)\right)^{-1} I_{C s o i l}
\end{equation}
\begin{equation}
N_{soil, s s}=\left(A_{h n} \xi(t) K_{h}+V(t)\right)^{-1} I_{N s o i l}
\end{equation}
同样的,由于生物地球化学循环运用循环的气象强迫场数据,
所以土壤碳氮的平衡状态也需要累加计算一个气象循环的平均值。 









