\chapter{降水与地表的能量交换}
%\addcontentsline{toc}{chapter}{降水与地表的能量交换}

%\begin{降水与地表的能量交换}
雨水与地表/植被之间存在温度差异,当降水发生时,它们之间可发生能量交换(雨水感热)。
\citet{wei2014impact} 表明,虽然在气候尺度上此能量交换的量级很小,但它可通过改变大气环流影响不同地区的气候特征,
故在气候模式中不可被忽略。下面给出雨水感热的计算方案。

\section{雨水温度}\label{雨水温度}
已有研究(如\citet{anderson1998moored})表明,
雨水温度非常接近于大气湿球温度,故这里雨水温度由湿球温度近似。根据湿球温度定义有如下关系:
\begin{equation}
C_{p a}\left(T_{w b}-T\right)=\lambda_{v}\left(r-r_{s a t}^{T_{w b}}\right)
\end{equation}
其中$C_{pa}$表示干空气的比热容(J/kg/K),$T_{wb}$表示大气湿球温度(K),
$T$表示大气环境温度(K),$r$表示混合比,$r_{sat}^{T_{wb}}$表示$T_{wb}$温度下的饱和混合比,
$\lambda_v$表示液态水的蒸发潜热(J/kg)。于是,取$T_{wb}$的初始值$T_{wb}^{\left(0\right)}=T$,
$T_{wb}$可通过如下过程进行迭代求解:
\begin{enumerate}
    \item 由章节 \ref{饱和水汽压(比湿)及其随温度的变化} 计算$T_{wb}^{\left(n\right)}$温度下的饱和比湿$q_{sat}^{T_{wb}^{\left(n\right)}}$;
    \item 通过混合比和比湿的换算关系($r=\frac{q}{1-q}$),计算混合比$r$与$r_{sat}^{T_{wb}^{\left(n\right)}}$;
    \item 由上述关系更新湿球温度:$T_{wb}^{\left(n\right)\ast}=T+\frac{\lambda_v}{C_{pa}}\left(r-r_{sat}^{T_{wb}^{\left(n\right)}}\right)$;
    \item 取更新前后湿球温度的平均值作为新一步的湿球温度:$T_{wb}^{\left(n+1\right)}=\left(T_{wb}^{\left(n\right)}+T_{wb}^{\left(n\right)\ast}\right)/2.0$。
\end{enumerate}
将上述过程迭代6次($n=0$,$\ldots$,5),作为最终的湿球温度$T_{wb}$。


当$T>T_f+2.5$K时,降水完全为液态水,可取雨水温度$T_p=T_{wb}$,否则$T_p$需根据降水的固态与液态比例进行调整。
根据“Snow Hydrology”(1956)中的实验结果,液态水占总降水的比例$f_{pl}$可按如下方式定义:
当$T_f+2.0K<T\le T_f+2.5K$时,$f_{pl}=0.4$ \\
当$T_f<T\le T_f+2.0K$时,$f_{pl}=-54.632+0.2T$ \\
当$T\le T_f$时,$f_{pl}=0$ \\
于是,当$T\le T_f+2.5K$且$f_{pl}>0$时,$T_p=T_f-\sqrt{\frac{1}{f_{pl}}-1}/100$,
否则当降水完全为固态时,$T_p=min{\left(T_f,T_{wb}\right)}$。


\section{植被/地表的雨水感热}\label{植被地表的雨水感热}
当地表有植被覆盖时,植被叶片与雨水的能量交换 ($\rm W/m^2$) 分别计算如下:
\begin{equation}
H_{p r c v}=C_{p l} q_{p l}\left(T_{p}-T_{v}\right)+C_{p i} q_{p i}\left(T_{p}-T_{v}\right)
\end{equation}

其中$C_{pl}$与$C_pi$分别表示液态水与固态水的比热容(J/kg/K),
$q_{pl}$与$q_{pi}$分别表示植被冠层对液态水和固态水的截流率(mm H2O/s)(计算见章节\ref{植被冠层截留})。


地表与雨水的能量交换 ($\rm W/m^2$) 为:
\begin{equation}
H_{p r c g}=C_{p l} p_{l}\left(T_{p}-T_{g}\right)+C_{p i} p_{i}\left(T_{p}-T_{g}\right)
\end{equation}
其中$p_l$与$p_i$分别表示落到地面的液态水与固态水降水率(mm H2O/s),
当有植被覆盖时,$p_l$与$p_i$为直接穿透植被冠层的降水率与沿叶茎流出冠层的降水率之和(计算见章节\ref{植被冠层截留})。

