\chapter{降水与地表的能量交换}
%\addcontentsline{toc}{chapter}{降水与地表的能量交换}
降水是陆面与大气之间保证水循环以及维持能量平衡的重要过程,CoLM模式通过多个物理参数化方案考虑降水相态分配、水凝物密度计算、雨水与植被/地表之间的能量交换等物理过程。

\begin{mymdframed}{代码}
本部分的对应代码为\texttt{MOD\_GroundTemperature.F90}、\texttt{MOD\_RainSnowTemp.F90}和\\
\texttt{MOD\_WetBulb.F90}。
\end{mymdframed}

\section{降水相态}\label{相态分配}
在CoLM中共有四种基于温、湿参量近似雨水温度来分配降水相态的方法,具体如表~\ref{tab:相态区分方案}~所示。

\begin{table}[]
\caption{相态区分方案}
\label{tab:相态区分方案}
\begin{tabular}[h]{p{1cm}p{5cm}p{10cm}}
\toprule
选项 & 注释(*代表的是默认方案) & 参考文献 \\\midrule
I & 湿球温度方案      & \citet{Wang-etal_19WetBulb}   \\
II*  & CLM5方案      & Ben Andre et al., 2020, CLM5 Documentation      \\
III  & PSY方案     & \citet{harder2013estimating}     \\
IV  & 单阈值方案     & \citet{us1956snow}     \\    \bottomrule

\end{tabular}
\end{table}

\subsection{湿球温度方案}\label{湿球温度方案}
已有研究(如~\citet{anderson1998moored})表明,
雨水温度非常接近于大气湿球温度,故该方案中雨水温度由湿球温度近似,根据湿球温度定义有如下关系:
\begin{equation}
\label{eq:湿球温度定义}
C_{p a}\left(T_{w b}-T\right)=\lambda_{v}\left(r-r_{s a t}^{T_{w b}}\right)
\end{equation}
其中 $C_{pa}$ 表示干空气的比热容 (\unit{J.kg^{-1}.K^{-1}}),$T_{wb}$ 表示大气湿球温度 (K),
$T$ 表示大气环境温度 (K),$r$ 表示混合比,$r_{sat}^{T_{wb}}$ 表示 $T_{wb}$ 温度下的饱和混合比,
$\lambda_v$ 表示液态水的蒸发潜热 (\unit{J.kg^{-1}})。于是,取 $T_{wb}$ 的初始值 $T_{wb}^{\left(0\right)}=T$,
$T_{wb}$ 可通过如下过程进行迭代求解:
\begin{enumerate}
    \item 由附录~\ref{饱和水汽压(比湿)及其随温度的变化} 计算$T_{wb}^{\left(n\right)}$温度下的饱和比湿$q_{sat}^{T_{wb}^{\left(n\right)}}$;
    \item 通过混合比和比湿的换算关系($r=\frac{q}{1-q}$),计算混合比$r$与$r_{sat}^{T_{wb}^{\left(n\right)}}$;
    \item 由上述关系更新湿球温度:$T_{wb}^{\left(n\right)\ast}=T+\frac{\lambda_v}{C_{pa}}\left(r-r_{sat}^{T_{wb}^{\left(n\right)}}\right)$;
    \item 取更新前后湿球温度的平均值作为新一步的湿球温度:$T_{wb}^{\left(n+1\right)}=\left(T_{wb}^{\left(n\right)}+T_{wb}^{\left(n\right)\ast}\right)/2.0$。
\end{enumerate}
将上述过程迭代6次($n=0$, $\ldots$, 5),作为最终的湿球温度$T_{wb}$。本方案将湿球温度视为雨水温度,通过雨水温度界定液态降水占比,具体如下:

\begin{equation}
f_{pl}= \begin{cases}
1, & \text{当 }\ 3\leqslant T_{wb} - T_{frz} \text{ 时} \\
1 - \frac{1}{1 + 5e-5\exp{(T_{wb} - T_{frz}+4)}} , & \text{当 }\ -2\leqslant T_{wb} - T_{frz} < 3 \text{ 时} \\
0, & \text{当 }\ T_{wb} - T_{frz} < -2 \text{ 时}
\end{cases}
\end{equation}
%
其中$T_{frz}$ = 273.15 K.


\subsection{CLM5方案}
CLM5方案假定大气环境温度为雨水温度,其区别于其他方法的地方在于方案会判断下垫面类型是否为冰川,并对冰川下垫面类型做单独考虑,实施方案具体如下:

当地表水体为冰面时(land water type = 3, land ice):
\begin{equation}
Allsnow_{tc} = T_{frz} - 2,\quad Allrain_{tc} = T_{frz}
\end{equation}

其他下垫面情况:
\begin{equation}
Allsnow_{tc} = T_{frz},\quad Allrain_{tc} = T_{frz} +2
\end{equation}

于是基于$Allsnow_{tc}$和$Allrain_{tc}$,液态降水占比为:
\begin{equation*}
f_{pl}= \begin{cases}
1, & \text{当 }\ Allrain_{tc}\leqslant forc_t \text{ 时}\\
(forc_t - Allsnow_{tc})/2, & \text{当 }\ Allsnow_{tc}\leqslant forc_t < Allrain_{tc} \text{ 时} \\
0, & \text{当 }\ forc_t < Allsnow_{tc} \text{ 时}
\end{cases}
\end{equation*}


\subsection{PSY方案}
PSY (Psychrometric Energy Balance Method)方案假设:
\begin{enumerate}
    \item 在无限接近水凝物(液态水或冰晶)的表面有一理想气团,其温度与水凝物一致,且满足Tetents饱和水汽压经验公式;
    \item 理想气团与环境大气之间满足质量守恒以及能量守恒且两者都满足理想气体方程;
    \item 水凝物的凝结(凝华)潜热以及空气导热率与环境温度(\textcelsius)满足一定函数关系。
\end{enumerate}

综上,方案基于假设,计算出水凝物表面空气温度以代表水凝物温度(\textcelsius),具体如下:
\begin{equation}
L\frac{dm}{dt}=4\pi C\lambda_t(T_i - T)
\end{equation}
%
\begin{equation}
\frac{dm}{dt}=4\pi CD_v(\rho - \rho_i)
\end{equation}

简化后:
\begin{equation}
\label{eq:水凝物表面空气温度定义}
T_i = T + \frac{D_v}{\lambda_{t}}L(\rho - \rho_i)
\end{equation}
%
其中$L$为凝结或凝华潜热,$C$为水凝物形状因子,$\lambda_t$为空气导热率因子,$T_i$为水凝物表面空气温度,$T$为环境温度,$D_v$为水汽扩散系数,$\rho$为环境空气密度,$\rho_i$为水凝物表面空气密度,以上方程可通过牛顿迭代求解,得出水凝物表面空气温度,相较于方程\ref{eq:湿球温度定义}对湿球温度的等焓假设,方程\ref{eq:水凝物表面空气温度定义}计算水凝物温度时并未做出类似假设,两者的物理基础不同,计算方程不能相互转化。

本方案计算液态降水占比如下:
\begin{equation*}
f_{pl}= \begin{cases}
1, & \text{当 }\ 5\leqslant T_i \text{ 时}\\
\frac{1}{1 + 2.50286\times 0.125006^{T_{i}}}, & \text{当 }\ -5\leqslant T_i < 5 \text{ 时} \\
0, & \text{当 }\ T_i < -5 \text{ 时}
\end{cases}
\end{equation*}
%
上式的Sigmoid函数拟合有别于湿球温度方案。

PSY方案的拟合数据来源于降水量:
\begin{equation}
f_{pl} = \frac{Rainfall(mm)}{Rainfall(mm) + Snowfall(mm)}
\end{equation}
%
而湿球温度方案的拟合数据源于降水事件数:
\begin{equation}
f_{pl} = \frac{Rain_{event}}{Rain_{event} + Snow_{event}}
\end{equation}


\subsection{单阈值方案}
单阈值方案就是设定一个关键温度以界定液态降水的占比,假设认为液态降水占比随温度升高逐渐线性增加,在2 \textcelsius 时占40\%,当环境温度高于2 \textcelsius 后全转变为液态降水,过程具体如下:
\begin{equation*}
f_{pl}= \begin{cases}
1, & \text{当 }\ T_{frz}+2\leqslant forc_t \text{ 时}\\
\max(0,-54.632 + 0.2\times forc_t), & \text{当 }\ forc_t < T_{frz} + 2 \text{ 时} \\
\end{cases}
\end{equation*}


\section{降水强度}
CoLM在物理过程计算时使用的降水强度由大气驱动数据或大气模式的降水模拟结果直接提供。由于CoLM中的部分物理过程(如冠层截留过程)需区分大尺度降水和对流性降水,而大气驱动数据往往只提供总降水强度,因此在CoLM中,暂时假设大气驱动数据提供的降水有2/3的比例为大尺度降水,1/3的比例为对流性降水。当与大气模式耦合时,大气模式往往可以直接提供大尺度降水和对流性降水的降水强度模拟结果。


\section{降水温度}
CoLM中使用的雨水温度可由湿球温度近似,计算方案在第~\ref{湿球温度方案} 节给出。同时,雨水温度也需根据降水相态进行调整,具体方案为:当降水全部为液态水时,雨水温度取湿球温度和冻结点温度$T_{frz}$之间的较大值;当液态降水比例$f_{pl}$介于0到1之间时,雨水温度计算公式为:$$T_p=T_{frz}-\sqrt{\frac{1}{f_{pl}}-1}/100.0$$
当降水全部为固态水时,雨水温度取湿球温度和冻结点温度$T_{frz}$之间的较小值。


\section{新雪密度}
雪的密度影响到地表的雪深、雪的堆积等一系列物理过程,是一个很重要的参量,在CoLM中采用CLM5的方案计算新雪密度,具体过程如下:
\begin{equation*}
Bifall= \begin{cases}
50 + 1.7\times 17^{1.5}, & \text{当 }\ 2\leqslant forc_t - T_{frz} \text{ 时}\\
50 + 1.7(forc_t - T_{frz}+15)^{1.5}, & \text{当 }\ -15\leqslant forc_t - T_{frz} < 2 \text{ 时} \\
-3.8328(forc_t - T_{frz}) - 0.0333(forc_t - T_{frz})^2, & \text{当 }\ -57.55\leqslant forc_t - T_{frz} < -15 \text{ 时} \\
110.2898, & \text{当 }\ forc_t - T_{frz} < -57.55 \text{ 时}\\
\end{cases}
\end{equation*}

当风速达到一定阈值时,雪密度在考虑温度后还需要继续考虑风速:
\begin{equation}
forc_{wind} = \sqrt{(u^2 + v^2)}
\end{equation}

当 $forc_{wind}$ > 0.1 时
\begin{equation}
Bifall_{new} = Bifall_{old} + 266.81{\left(\frac{1 + \tanh (\frac{forc_{wind}}{5})}{2}\right)}^{8.8}
\end{equation}


\section{降水感热}\label{植被地表的雨水感热}
雨水与地表/植被之间存在温度差异,当降水发生时,它们之间可发生能量交换(降水感热)。
\citet{wei2014impact} 表明,虽然在气候尺度上此能量交换的量级很小,但它可通过改变大气环流影响不同地区的气候特征,
故在气候模式中不可被忽略。感热计算中默认湿球温度为雨水温度,下面给出雨水感热的计算方案。

当地表有植被覆盖时,植被叶片与雨水的能量交换 (\unit{W.m^{-2}}) 分别计算如下:
\begin{equation}
H_{prcv}=C_{pl} q_{pl}\left(T_{p}-T_{v}\right)+C_{pi} q_{pi}\left(T_{p}-T_{v}\right)
\end{equation}
其中$C_{pl}$与$C_{pi}$分别表示液态水与固态水的比热容(\unit{J.kg^{-1}.K^{-1}}),
$q_{pl}$与$q_{pi}$分别表示植被冠层对液态水和固态水的截流率(\unit{mm.H_2O.s^{-1}})(计算见第~\ref{植被冠层截留} 章)。


地表与雨水的能量交换 (\unit{W.m^{-2}}) 为:
\begin{equation}
H_{p r c g}=C_{p l} p_{l}\left(T_{p}-T_{g}\right)+C_{p i} p_{i}\left(T_{p}-T_{g}\right)
\end{equation}
其中$p_l$与$p_i$分别表示落到地面的液态水与固态水降水率(\unit{mm.H_2O.s^{-1}}),
当有植被覆盖时,$p_l$与$p_i$为直接穿透植被冠层的降水率与沿叶茎流出冠层的降水率之和(计算见第~\ref{植被冠层截留} 章)。










