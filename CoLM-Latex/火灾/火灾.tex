\chapter{火灾}\label{ch:火灾}
CoLM 里采用 Li 火灾参数化方案~\citep{LiF2012,LiF2013,LiF2017,LiF2019}。它包括四个部分:热带密闭林区由砍伐引起的火灾、农业火、泥炭火、以及农田和热带密闭林以外的非泥炭火。其中,前三部分是简单的经验统计方程;而第四部分是中等复杂程度的过程模式,是该火灾方案的主体。燃烧面积由天气气候、植被特征(组成和生物量)、人类活动决定。在燃烧面积计算完成后,我们估计火灾影响。

\section{农田和热带密闭林区以外的非泥炭火}
格点燃烧面积 $A_b$ (\unit{km^2.s^{-1}})是着火数$N_f$ (\unit{count.s^{-1}})以及每个火的平均蔓延面积$a$ (\unit{km^2.count^{-1}})的乘积:
\begin{equation}
A_b = N_f a
\end{equation}

\subsection{火发生}
格点的着火数是格点点火数 $N_i$ ($\text{count}~\text{s}^{-1}$),燃料供给率 $f_b$, 燃料可燃性 $f_m$,以及未被人为抑制的比例 $f_{se,o}$ 的乘积:  
\begin{equation}
N_f = N_i f_b f_m f_{se,o} 
\end{equation}
%
公式中的点火数为:
\begin{equation}
N_{i}=\left(I_{n}+I_{a}\right) A_{g}
\end{equation}
%
其中 $I_n$ (\unit{count.s^{-1}.km^{-2}})和 $I_a$ (\unit{count.s^{-1}.km^{-2}})为格点内的闪电和人为点火频率,$A_g$ (\unit{km^2})为格点面积。

闪电引起的点火频率为: 
\begin{equation}
I_{n}=\gamma \psi I_{l}
\end{equation}
其中,$\gamma$ 是云-地闪电的点火效率;$\psi=1/\left[(5.16+2.16\cos(3\min(60,\lambda))\right]$ 是云-地闪电频率占总闪电频率的比率,随纬度 $\lambda$ 而变化, $I_{l}$ 为闪电频率 (\unit{count.s^{-1}.km^{-2}})。

而人为点火频率为:  
\begin{equation}
I_{a}=\frac{\alpha D_{P} k\left(D_{P}\right)}{n}
\end{equation}
其中,$\alpha=0.01$ (\unit{count.person^{-1}.mon^{-1}})是每月人为潜在点火数, $D_P$ ($\text{person}~\text{km}^{-2}$)是人口密度, $k\left(D_{P}\right)=6.8D_P^{-0.6}$为人为点火比率,$n$ 是每月的秒数。

燃料供给率为:
\begin{equation} 
f_b = \begin{cases} 
0 & \text{if } B_{ag} < B_{\text{low}}\\
\frac{B_{ag}-B_{low}}{B_{up}-B_{low}} & \text{if } B_{low} \leqslant B_{ag} \leqslant B_{up}\\
1 & \text{if } B_{ag} > B_{up}
\end{cases}
\end{equation}
其中, $B_{ag}$ 为地上生物量, $B_{low}$ 和 $B_{up}$ 为阈值。

燃料可燃性为: 
\begin{equation}
f_m = f_{RH} f_{\beta},\quad T_{17cm}>T_f
\end{equation}
其中, $f_{RH}$ 和 $f_{\beta}$ 分别是相对湿度$RH$和根区土壤湿度因$\beta$的函数, $T_{17cm}$和$T_{f}$ 分别为表层17 cm的土壤温度和冰点温度。
%
\begin{equation}
f_{RH} = (1-\omega)l_{RH0}+\omega l_{RH_{30d}}
\end{equation}
其中的权重因子是地上生物量的权重 $$\omega=\max\left[0, \min\left(1, \frac{B_{ag} - 2500}{2500}\right)\right]$$
$$l_{RH_0}=1-\max\left[0, \min\left(1, \frac{RH_{0} - 30}{80-30}\right)\right]$$
和
$$l_{RH_{30d}}=1-\max\left[0.75, \min\left(1, \frac{RH_{30d}}{90}\right)\right]$$
分别是当前时步相对湿度和过去30天滑动平均相对湿度的函数。
\begin{equation} 
f_b = \begin{cases} 
1 & \text{if } \beta < \beta_{low}\\
\frac{\beta_{up}-\beta}{\beta_{up}-\beta_{low}} & \text{if } \beta_{low} \leqslant \beta \leqslant \beta_{up}\\
0 & \text{if } \beta \geqslant \beta_{up}
\end{cases}
\end{equation}
中的$\beta_{low}=0.85$和$\beta_{up}=0.98$。


对于人烟稀少的地区($P \leqslant 0.1$ person km$^{-2}$),我们假设人为火抑制可以忽略,即 $f_{se,o} = 1.0$;对于其它地区, 
\begin{equation}
f_{se,o} = f_d f_e  
\end{equation}
其中的 $f_d$ 及 $f_e$ 为社会经济条件对火发生的影响:
\begin{equation}
f_d = 0.01 + 0.98\exp(-0.025 D_P)  
\end{equation}
对于草和灌木:  
\begin{equation} 
f_e = 0.1 + 0.9\exp\left[-\pi \left(\frac{\text{GDP}}{8} \right)^{0.5} \right] 
\end{equation}
%
对于树:
%
\begin{equation}
f_e = \begin{cases}
0.39 & \text{if GDP} > 20\\  
0.79 & \text{if } 8 < \text{GDP} \leqslant 20\\
1.0 & \text{if GDP} \leqslant 8  
\end{cases}
\end{equation}
%
其中GDP为人均GDP (\unit{kUS\$(1995).capita^{-1}})。


\subsection{火蔓延}
每个火的平均蔓延面积为: 
\begin{equation}
a = a^* F_{se}
\end{equation}
%
其中, $a^*$ 为每个火的潜在蔓延面积 ($\text{km}^2~\text{count}^{-1}$), $F_{se}$ 为社会经济条件的影响。

每个火的潜在蔓延面积为:   
%
\begin{equation}
a^{*}=\pi \frac{l}{2} \frac{\omega}{2} \times 10^{-6}=\frac{\pi u_{p}^{2} \tau^{2}}{4 L_{B}}\left(1+\frac{1}{H_{B}}\right)^{2} \times 10^{-6}
\end{equation}
%
其中,$t = 1$(d) $= 3600\times24$(s)为未考虑人为火抑制的潜在平均火持续时间; $L_B$ 和 $H_B$为火蔓延椭圆区域的长宽比和焦点距离长轴两端的距离比:  
\begin{equation}
L_{B}=1.0+10.0\left[1-\exp (-0.06 W)\right],
\end{equation}
\begin{equation}
H_{B}=\frac{u_{p}}{u_{b}}=\frac{L_{B}+\left(L_{B}^{2}-1\right)^{0.5}}{L_{B}-\left(L_{B}^{2}-1\right)^{0.5}};
\end{equation}
$u_p$ (km s$^{-1}$)为下风区的蔓延速率:  
\begin{equation}
u_{p}=u_{\max } C_{m} g(W)
\end{equation}
%
其中, $W$ 为风速, $u_{max}$ 为下风区火最大蔓延速率, $C_{m}=\sqrt{f_{m}}$为燃料湿度对火蔓延的影响因子,风速对下风区火蔓延速率的影响$g(W)$为:  
\begin{equation}
g(W)=0.05 \times \frac{2 L_{B}}{1+\frac{1}{H_{B}}}。
\end{equation}
类似火发生部分,当 $D_P \leqslant 0.1$ 时,$F_{se} = 1.0$;当 $D_P > 0.1$
%
\begin{equation}
F_{se} = F_d F_e
\end{equation}  
%
其中, $F_d$ 和 $F_e$ 分别为社会和经济因子对火蔓延的影响。对于草和灌木:  
\begin{equation}
F_{d}=0.2+0.8 \times \exp \left[-\pi\left(\frac{D_{p}}{450}\right)^{0.5}\right]
\end{equation}
和  
\begin{equation}
F_{e}=0.2+0.8 \times \exp \left(-\pi \frac{\text{GDP}}{7}\right)。
\end{equation}
对于树:  
\begin{equation}
F_{d}=0.4+0.6 \times \exp \left(-\pi \frac{D_{p}}{125}\right)
\end{equation}
和  
\begin{equation}
F_{e}=\begin{cases}
0.62 & \text{GDP}>20 \\
0.83 & 8< \text{GDP} \leqslant 20 \\
1 & \text{GDP} \leqslant 8
\end{cases}
\end{equation}


\subsection{农业火}

农田上发生的火灾燃烧面积为:  
\begin{equation}
A_{b}=a_{1} f_{s e} f_{t} f_{c r o p} A_{g}
\end{equation}
%
其中,$a_{1} =4.4\times10^{-7}$ (\unit{s^{-1}})是根据 GFED3 全球农田发生燃烧面积占全球总燃烧面积比例率定的全球常熟; $f_{se}$ 反映社会经济的影响; $f_t$ 为描述的农业火发生时刻;$f_{crop}$ 为格点里农田面积比例。


社会经济因子反映的是人口密度越高和经济越发达的地区,用秸秆焚烧的方法处理农业残余的比例越低:  
\begin{equation}
f_{se} = f_d f_e 
\end{equation}  
其中,  
\begin{equation}
f_{d}=0.04+0.96 \times \exp \left[-\pi\left(\frac{D_{p}}{350}\right)^{0.5}\right]
\end{equation}
和  
\begin{equation}
f_{e}=0.01+0.99 \times \exp \left(-\pi \frac{\text{GDP}}{10}\right)
\end{equation}
分别为人口密度和人均 GDP 的函数。


\section{热带密闭林区由人类砍伐引起的火灾}  
热带密闭林区(树覆盖率高于 60\%)由人类砍伐引起的火灾即包括发生在砍伐区的火灾(deforestation fires)也包括不受控制蔓延到砍伐区以外的(degradation fires)。其燃烧面积由下列公式计算:  
\begin{equation}
A_{b}=b f_{l u} f_{c l i, d} f_{b} A_{g}
\end{equation}
%
其中,$b =3.8\times10^{-7}$ (\unit{s^{-1}})是根据 GFED3 亚马逊雨燃烧面积率定的全球常数, $f_{lu}$ 为砍伐率影响因子, $f_{cli,d}$ 为气候影响因子。  

砍伐率影响因子: 
\begin{equation}
f_{lu} = \max(0.0005, 0.19D - 0.001)
\end{equation}  
是砍伐率 ($D$,\unit{yr^{-1}})的函数。砍伐率由模式输入的土地利用及土地覆盖变化数据计算得来。
气候影响因子
\begin{equation}
\begin{aligned}
f_{c l i, d}=\max\left[0, \min \left(1, \frac{b_{2}-P_{60 d}}{b_{2}}\right)\right]^{0.5}& \times\max\left[0, \min \left(1, \frac{b_{3}-P_{10 d}}{b_{3}}\right)\right]^{0.5}\\
& \times\max\left[0, \min \left(1, \frac{0.25-P}{0.25}\right)\right]^{0.5}
\end{aligned}
\end{equation}
是当前时步降水量 $P$(\unit{mm.d^{-1}}) 和此前 60 天滑动平均降水$P_{60 d}$ (\unit{mm.d^{-1}})及 10 天滑动平均降水$P_{60 d}$ (\unit{mm.d^{-1}})的函数。 $b_2 = b_3$, 对于热带常绿树取 $4.0$ \unit{mm.d^{-1}},对于热带落叶树取 $1.8$ \unit{mm.d^{-1}}.


\section{泥炭火}  

泥炭火燃烧面积为:   
\begin{equation}
A_{b}=c f_{c l i, p} f_{peat}\left(1-f_{sat}\right) A_{g}
\end{equation}
其 中 , $c$ 是 全 球 常 数 , 对 于 热 带 泥 炭 火 $c=4.7\times10^{-7}$ (\unit{s^{-1}}) 及 对 于 寒 带 泥 炭 火 $c=2.5\times10^{-9}$ (\unit{s^{-1}}) 是基于 GFED3 的泥炭火燃烧面积估算而来; $f_{cli,p}$ 为气候影响因子, $f_{peat}$ 为格点内泥炭地面积比例; $f_{sat}$ 为水位高于地表的面积比例。  

对于热带泥炭火,气候影响因子为:  
\begin{equation}
f_{c l i, p}=\max \left[0, \min \left(1, \frac{4-P_{60 d}}{4}\right)\right]^{2}
\end{equation}
而对于寒带泥炭火,气候影响因子为:  
\begin{equation}
f_{c l i, p}=\exp \left(-\pi \frac{\theta_{17 c m}}{0.3}\right) \times \max \left[0, \min \left(1, \frac{T_{17 c m}-T_{f}}{10}\right)\right]
\end{equation}
其中, $\theta_{\text{17cm}}$ 为表层 17 cm 土壤湿度。

\section{火灾影响}
\subsection{燃烧引起的碳排放}  

对于过火区,第 $j$ 个 PFT 生物量燃烧引起的碳排放${\phi}_j$(\unit{g.C.s^{-1}.m^{-2}})为:  
%
\begin{equation}
\phi_{j}=\frac{A_{b, j}}{A_{g}} C_{j} \times C C_{j}
\end{equation}
%
其中, $A_{b,j}/A_g$ 为燃烧面积率(\unit{s^{-1}}); $
C_{j}=\left(C_{leaf}, C_{stem}, C_{root}, C_{t s}\right)
$是碳库(\unit{g.C.m^{-2}}),包括叶、茎、根、储存库的碳库; $
C C_{j}=\left(C C_{leaf}, C C_{stem}, C C_{root}, C C_{t s}\right)_{j}
$是燃烧完全因子 (表~\ref{tab:burning_factors})。 此外,我们假设 50\%和 28\%的 litter 和 coarse woody debris (CWD)的碳库被燃烧后排放到大气。  


对于热带泥炭火,泥炭燃烧引起的碳排放(\unit{g.C.s^{-1}.m^{-2}})是泥炭火的燃烧面积率(\unit{s^{-1}})、土壤有机碳(\unit{g.C.m^{-2}})、0.18 的乘积;对于寒带泥炭火,泥炭燃烧引起的碳排放是泥炭火的燃烧面积率(\unit{s^{-1}})与2200 (\unit{g.C.m^{-2}})的乘积。以上常数由已有泥炭火研究数值结果推导而来。

对于热带密闭林砍伐引起的火灾,我们假设燃烧先发生在砍伐区,砍伐区燃烧2遍后才会蔓延到砍伐区以外。对于第一部分,我们按照比例 $\min \left(\frac{A_{b}}{2 A_{g}}, D\right)$ 从土地类型发生变化引起的碳排放里扣除。对于第二部分引起的燃烧面积 $\max(0, A_b-2DA_g)$,按照其它火灾去计算火灾的影响。


\subsection{火灾引起的植物组织死亡}

对于燃烧后剩余的碳库部分,即
\begin{equation}
\begin{aligned}
C_{j 1}^{\prime}=\big(& C_{\text {leaf}}\left(1-C C_{\text {leaf}}\right), C_{\text {livestem}}\left(1-C C_{\text {stem}}\right), \\
&C_{\text {deadstem}}\left(1-C C_{\text {stem}}\right), C_{\text {root}}\left(1-C C_{\text {root}}\right), C_{t s}\left(1-C C_{t s}\right)\big)_{j}
\end{aligned}
\end{equation}
%
考虑火灾引起植物组织死亡为:
\begin{equation}
\Psi_{j 1}=\frac{A_{b, j}}{f_{j} A_{g}} C_{j 1}^{\prime} \times M_{j 1}
\end{equation}
%
其中, $
M_{j 1}=\left(M_{\text{leaf}}, M_{\text{livestem}}, M_{\text{deadstem}}, M_{\text {root}}, M_{\text {ts}}\right)_{j}
$ 是植物组织死亡因子,相应的碳由植物碳库转移到 litter 或 CWD。此外,还有一部分 livestem 的碳库会转移到 deadstem:
\begin{equation}
\Psi_{j 2}=\frac{A_{b, j}}{f_{j} A_{g}} C_{\text {livestem }}\left(1-C C_{\text {stem }}\right) M_{\text {livestem}, 2}
\end{equation}
%
其中, $M_{\text{livestem},2}$ 为这部分 livestem 的死亡因子(表~\ref{tab:burning_factors})。


\subsection{火灾引起的痕量气体和气溶胶排放}

对于第$j$个PFT和第$x$种痕量气体和气溶胶,其排放量 (\unit{g.species.s^{-1}})为:
\begin{equation}
E_{x, j}=E F_{x, j} \frac{\phi_{j}}{[C]}
\end{equation}
其中 $\phi_{j}$ (\unit{g.C.s^{-1}.m^{-2}})为第 $j$ 个 PFT 的火灾引起的碳排放,$E F_{x, j}$为排放因子~\citep{LiF2019},为单位转化常数。


针叶树、其它寒带和温带树木、热带地区树木、灌木和草,燃烧引起的火灾最高排放高度分别为:4.3、3、2.5、2、1 km。

\begin{landscape}
\begin{table}[htbp]
\caption{燃烧完全因子及火灾引起的植物组织死亡因子}
\label{tab:burning_factors}
\begin{tabular}{lcccccccccc}
\toprule
PFT              & $CC_{leaf}$ & $CC_{stem}$ & $CC_{root}$ & $CC_{ts}$ & $M_{leaf}$ & $M_{livestem,1}$ & $M_{deadstem}$ & $M_{root}$ & $M_{ts}$  & $M_{livestem,2}$ \\ \midrule
\multicolumn{11}{l}{\textbf{Tree:}}                                                                                        \\
NET Temperate    & 0.80   & 0.27   & 0.00   & 0.45 & 0.80  & 0.13        & 0.13      & 0.13  & 0.45 & 0.32        \\
NET Boreal       & 0.80   & 0.30   & 0.00   & 0.50 & 0.80  & 0.15        & 0.15      & 0.15  & 0.50 & 0.35        \\
NDT Boreal       & 0.80   & 0.30   & 0.00   & 0.50 & 0.80  & 0.15        & 0.15      & 0.15  & 0.50 & 0.35        \\
BET Tropical     & 0.80   & 0.27   & 0.00   & 0.45 & 0.80  & 0.13        & 0.13      & 0.13  & 0.45 & 0.32        \\
BET Temperate    & 0.80   & 0.27   & 0.00   & 0.45 & 0.80  & 0.13        & 0.13      & 0.13  & 0.45 & 0.32        \\
BDT Tropical     & 0.80   & 0.27   & 0.00   & 0.45 & 0.80  & 0.10        & 0.10      & 0.10  & 0.35 & 0.25        \\
BDT Temperate    & 0.80   & 0.27   & 0.00   & 0.45 & 0.80  & 0.10        & 0.10      & 0.10  & 0.35 & 0.25        \\
BDT Boreal       & 0.80   & 0.27   & 0.00   & 0.45 & 0.80  & 0.13        & 0.13      & 0.13  & 0.45 & 0.32        \\\hline
\multicolumn{11}{l}{\textbf{Shrub:}}                                                                                       \\
BES Temperate    & 0.80   & 0.35   & 0.00   & 0.55 & 0.80  & 0.17        & 0.17      & 0.17  & 0.55 & 0.38        \\
BDS Temperate    & 0.80   & 0.35   & 0.00   & 0.55 & 0.80  & 0.17        & 0.17      & 0.17  & 0.55 & 0.38        \\
BDS Boreal       & 0.80   & 0.35   & 0.00   & 0.55 & 0.80  & 0.17        & 0.17      & 0.17  & 0.55 & 0.38        \\\hline
\multicolumn{11}{l}{\textbf{Grass:}}                                                                                       \\
C3  arctic grass & 0.80   & -----  & 0.00   & 0.80 & 0.80  & -----       & -----     & 0.20  & 0.80 & -----       \\
C3  grass        & 0.80   & -----  & 0.00   & 0.80 & 0.80  & -----       & -----     & 0.20  & 0.80 & -----       \\
C4  grass        & 0.80   & -----  & 0.00   & 0.80 & 0.80  & -----       & -----     & 0.20  & 0.80 & -----       \\\hline
\multicolumn{11}{l}{\textbf{Crop:}}                                                                                        \\
Crop1            & 0.80   & -----  & 0.00   & 0.80 & 0.80  & -----       & -----     & 0.20  & 0.80 & -----       \\
Crop2            & 0.80   & -----  & 0.00   & 0.80 & 0.80  & -----       & -----     & 0.20  & 0.80 & -----       \\ \bottomrule
\end{tabular}
\end{table}
\end{landscape}