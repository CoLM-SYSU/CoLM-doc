\chapter{火灾}\label{ch:火灾}
\begin{mymdframed}{代码}
  本章对应代码源文件位于\texttt{main/BGC/}目录下。
\end{mymdframed}

CoLM 里采用 Li 火灾参数化方案~\citep{LiF2012,LiF2013,LiF2017,LiF2019}。它包括四个部分:热带密闭林区由砍伐引起的火灾、农业火、泥炭火、以及农田和热带密闭林以外的非泥炭火。其中,前三部分是简单的经验统计方程;而第四部分是中等复杂程度的过程模式,是该火灾方案的主体。燃烧面积由天气气候、植被特征(组成和生物量)、人类活动决定。在燃烧面积计算完成后,我们估计火灾影响。

\section{农田和热带密闭林区以外的非泥炭火}
格点燃烧面积 $A_{\mathrm {b}} $ (\unit{km^2.s^{-1}})是着火数$N_{\mathrm {f}} $ (\unit{count.s^{-1}})以及每个火的平均蔓延面积$a$ (\unit{km^2.count^{-1}})的乘积:
\begin{equation}
  A_{\mathrm {b}}  = N_{\mathrm {f}}  a
\end{equation}

\subsection{火发生}
格点的着火数是格点点火数 $N_{\mathrm {i}} $ ($\text{count}~\text{s}^{-1}$),燃料供给率 $f_{\mathrm {b}} $, 燃料可燃性 $f_{\mathrm {m}} $,以及未被人为抑制的比例 $f_{\mathrm{se,o}}$ 的乘积:
\begin{equation}
  N_{\mathrm {f}}  = N_{\mathrm {i}}  f_{\mathrm {b}}  f_{\mathrm {m}}  f_{\mathrm{se,o}}
\end{equation}
%
公式中的点火数为:
\begin{equation}
  N_{\mathrm{i}}=\left(I_{\mathrm{n}}+I_{\mathrm{a}}\right) A_{\mathrm{g}}
\end{equation}
%
其中 $I_{\mathrm {n}} $ (\unit{count.s^{-1}.km^{-2}})和 $I_{\mathrm {a}} $ (\unit{count.s^{-1}.km^{-2}})为格点内的闪电和人为点火频率,$A_{\mathrm {g}} $ (\unit{km^2})为格点面积。

闪电引起的点火频率为:
\begin{equation}
  I_{\mathrm{n}}=\gamma \psi I_{\mathrm{l}}
\end{equation}
其中,$\gamma$ 是云-地闪电的点火效率;$\psi=1/\left[(5.16+2.16\cos(3\min(60,\lambda))\right]$ 是云-地闪电频率占总闪电频率的比率,随纬度 $\lambda$ 而变化, $I_{\mathrm{l}}$ 为闪电频率 (\unit{count.s^{-1}.km^{-2}})。

而人为点火频率为:
\begin{equation}
I_{\mathrm{a}}=\frac{\alpha D_{\mathrm{P}} k\left(D_{\mathrm{P}}\right)}{n}
\end{equation}
其中,$\alpha=0.01$ (\unit{count.person^{-1}.mon^{-1}})是每月人为潜在点火数, $D_{\mathrm {P}} $ ($\text{person}~\text{km}^{-2}$)是人口密度, $k\left(D_{\mathrm{P}}\right)=6.8D_{\mathrm {P}} ^{-0.6}$为人为点火比率,$n$ 是每月的秒数。

燃料供给率为:
\begin{equation}
  f_{\mathrm {b}}  = \begin{cases}
    0 & \text{if } B_{\mathrm{ag}} < B_{\text{low}}\\
    \frac{B_{\mathrm{ag}}-B_{\mathrm{low}}}{B_{\mathrm{up}}-B_{\mathrm{low}}} & \text{if } B_{\mathrm{low}} \leqslant B_{\mathrm{ag}} \leqslant B_{\mathrm{up}}\\
    1 & \text{if } B_{\mathrm{ag}} > B_{\mathrm{up}}
  \end{cases}
\end{equation}
其中, $B_{\mathrm{ag}}$ 为地上生物量, $B_{\mathrm{low}}$ 和 $B_{\mathrm{up}}$ 为阈值。

燃料可燃性为:
\begin{equation}
  f_{\mathrm {m}}  = f_{\mathrm{RH}} f_{\beta},\quad T_{17cm}>T_{\mathrm {frz}}
\end{equation}
其中, $f_{\mathrm{RH}}$ 和 $f_{\beta}$ 分别是相对湿度${\mathrm {RH}}$和根区土壤湿度因$\beta$的函数, $T_{17cm}$和$T_{\mathrm{frz}}$ 分别为表层17 cm的土壤温度和冰点温度。
%
\begin{equation}
  f_{\mathrm{RH}} = (1-\omega)l_{\mathrm{RH0}}+\omega l_{\mathrm{RH_{\rm 30d}}}
\end{equation}
其中的权重因子是地上生物量的权重 $$\omega=\max\left[0, \min\left(1, \frac{B_{\mathrm{ag}} - 2500}{2500}\right)\right]$$
$$l_{\mathrm{RH_0}}=1-\max\left[0, \min\left(1, \frac{RH_{\mathrm{0}} - 30}{80-30}\right)\right]$$
和
$$l_{\mathrm{RH_{\rm 30d}}}=1-\max\left[0.75, \min\left(1, \frac{RH_{\rm 30d}}{90}\right)\right]$$
分别是当前时步相对湿度和过去30天滑动平均相对湿度的函数。
\begin{equation}
  f_{\mathrm {b}}  = \begin{cases}
    1 & \text{if } \beta < \beta_{\mathrm{low}}\\
    \frac{\beta_{\mathrm{up}}-\beta}{\beta_{\mathrm{up}}-\beta_{\mathrm{low}}} & \text{if } \beta_{\mathrm{low}} \leqslant \beta \leqslant \beta_{\mathrm{up}}\\
    0 & \text{if } \beta \geqslant \beta_{\mathrm{up}}
  \end{cases}
\end{equation}
中的$\beta_{\mathrm{low}}=0.85$和$\beta_{\mathrm{up}}=0.98$。


对于人烟稀少的地区($P \leqslant 0.1$ person km$^{-2}$),我们假设人为火抑制可以忽略,即 $f_{\mathrm{se,o}} = 1.0$;对于其它地区,
\begin{equation}
  f_{\mathrm{se,o}} = f_{\mathrm {d}}  f_{\mathrm {e}}
\end{equation}
其中的 $f_{\mathrm {d}} $ 及 $f_{\mathrm {e}} $ 为社会经济条件对火发生的影响:
\begin{equation}
  f_{\mathrm {d}}  = 0.01 + 0.98\exp(-0.025 D_{\mathrm {P}} )
\end{equation}
对于草和灌木:
\begin{equation}
  f_{\mathrm {e}}  = 0.1 + 0.9\exp\left[-\pi \left(\frac{\text{GDP}}{8} \right)^{0.5} \right]
\end{equation}
%
对于树:
%
\begin{equation}
  f_{\mathrm {e}}  = \begin{cases}
    0.39 & \text{if GDP} > 20\\
    0.79 & \text{if } 8 < \text{GDP} \leqslant 20\\
    1.0 & \text{if GDP} \leqslant 8
  \end{cases}
\end{equation}
%
其中GDP为人均GDP (\unit{kUS\$(1995).capita^{-1}})。


\subsection{火蔓延}
每个火的平均蔓延面积为:
\begin{equation}
  a = a^* F_{\mathrm{se}}
\end{equation}
%
其中, $a^*$ 为每个火的潜在蔓延面积 ($\text{km}^2~\text{count}^{-1}$), $F_{\mathrm{se}}$ 为社会经济条件的影响。

每个火的潜在蔓延面积为:
%
\begin{equation}
  a^{*}=\pi \frac{l}{2} \frac{\omega}{2} \times 10^{-6}=\frac{\pi u_{\mathrm{p}}^{2} \tau^{2}}{4 L_{\mathrm{B}}}\left(1+\frac{1}{H_{\mathrm{B}}}\right)^{2} \times 10^{-6}
\end{equation}
%
其中,$t = 1$(d) $= 3600\times24$(s)为未考虑人为火抑制的潜在平均火持续时间; $L_{\mathrm {B}} $ 和 $H_{\mathrm {B}} $为火蔓延椭圆区域的长宽比和焦点距离长轴两端的距离比:
\begin{equation}
  L_{\mathrm{B}}=1.0+10.0\left[1-\exp (-0.06 W)\right],
\end{equation}
\begin{equation}
  H_{\mathrm{B}}=\frac{u_{\mathrm{p}}}{u_{\mathrm{b}}}=\frac{L_{\mathrm{B}}+\left(L_{\mathrm{B}}^{2}-1\right)^{0.5}}{L_{\mathrm{B}}-\left(L_{\mathrm{B}}^{2}-1\right)^{0.5}};
\end{equation}
$u_{\mathrm {p}} $ (km s$^{-1}$)为下风区的蔓延速率:
\begin{equation}
  u_{\mathrm{p}}=u_{\max } C_{\mathrm{m}} g(W)
\end{equation}
%
其中, $W$ 为风速, $u_{\mathrm{max}}$ 为下风区火最大蔓延速率, $C_{\mathrm{m}}=\sqrt{f_{\mathrm{m}}}$为燃料湿度对火蔓延的影响因子,风速对下风区火蔓延速率的影响$g(W)$为:
\begin{equation}
  g(W)=0.05 \times \frac{2 L_{\mathrm{B}}}{1+\frac{1}{H_{\mathrm{B}}}}。
\end{equation}
类似火发生部分,当 $D_{\mathrm {P}}  \leqslant 0.1$ 时,$F_{\mathrm{se}} = 1.0$;当 $D_{\mathrm {P}}  > 0.1$
%
\begin{equation}
  F_{\mathrm{se}} = F_{\mathrm {d}}  F_{\mathrm {e}}
\end{equation}
%
其中, $F_{\mathrm {d}} $ 和 $F_{\mathrm {e}} $ 分别为社会和经济因子对火蔓延的影响。对于草和灌木:
\begin{equation}
  F_{\mathrm{d}}=0.2+0.8 \times \exp \left[-\pi\left(\frac{D_{\mathrm{p}}}{450}\right)^{0.5}\right]
\end{equation}
和
\begin{equation}
  F_{\mathrm{e}}=0.2+0.8 \times \exp \left(-\pi \frac{\text{GDP}}{7}\right)。
\end{equation}
对于树:
\begin{equation}
  F_{\mathrm{d}}=0.4+0.6 \times \exp \left(-\pi \frac{D_{\mathrm{p}}}{125}\right)
\end{equation}
和
\begin{equation}
  F_{\mathrm{e}}=\begin{cases}
    0.62 & \text{GDP}>20 \\
    0.83 & 8< \text{GDP} \leqslant 20 \\
    1 & \text{GDP} \leqslant 8
  \end{cases}
\end{equation}


\subsection{农业火}

农田上发生的火灾燃烧面积为:
\begin{equation}
  A_{\mathrm{b}}=a_{1} f_{\mathrm{s e}} f_{\mathrm{t}} f_{\mathrm{c r o p}} A_{\mathrm{g}}
\end{equation}
%
其中,$a_{1} =4.4\times10^{-7}$ (\unit{s^{-1}})是根据 GFED3 全球农田发生燃烧面积占全球总燃烧面积比例率定的全球常熟; $f_{\mathrm{se}}$ 反映社会经济的影响; $f_{\mathrm {t}} $ 为描述的农业火发生时刻;$f_{\mathrm{crop}}$ 为格点里农田面积比例。


社会经济因子反映的是人口密度越高和经济越发达的地区,用秸秆焚烧的方法处理农业残余的比例越低:
\begin{equation}
  f_{\mathrm{se}} = f_{\mathrm {d}}  f_{\mathrm {e}}
\end{equation}
其中,
\begin{equation}
  f_{\mathrm{d}}=0.04+0.96 \times \exp \left[-\pi\left(\frac{D_{\mathrm{p}}}{350}\right)^{0.5}\right]
\end{equation}
和
\begin{equation}
  f_{\mathrm{e}}=0.01+0.99 \times \exp \left(-\pi \frac{\text{GDP}}{10}\right)
\end{equation}
分别为人口密度和人均 GDP 的函数。


\section{热带密闭林区由人类砍伐引起的火灾}
热带密闭林区(树覆盖率高于 60\%)由人类砍伐引起的火灾即包括发生在砍伐区的火灾(deforestation fires)也包括不受控制蔓延到砍伐区以外的(degradation fires)。其燃烧面积由下列公式计算:
\begin{equation}
  A_{\mathrm{b}}=b f_{\mathrm{l u}} f_{\mathrm{c l i, d}} f_{\mathrm{b}} A_{\mathrm{g}}
\end{equation}
%
其中,$b =3.8\times10^{-7}$ (\unit{s^{-1}})是根据 GFED3 亚马逊雨燃烧面积率定的全球常数, $f_{\mathrm{lu}}$ 为砍伐率影响因子, $f_{\mathrm{cli,d}}$ 为气候影响因子。

砍伐率影响因子:
\begin{equation}
  f_{\mathrm{lu}} = \max(0.0005, 0.19D - 0.001)
\end{equation}
是砍伐率 ($D$,\unit{yr^{-1}})的函数。砍伐率由模式输入的土地利用及土地覆盖变化数据计算得来。
气候影响因子
\begin{equation}
  \begin{aligned}
    f_{\mathrm{c l i, d}}=\max\left[0, \min \left(1, \frac{b_{2}-p_{\rm 60d}}{b_{2}}\right)\right]^{0.5}& \times\max\left[0, \min \left(1, \frac{b_{3}-p_{\rm 10d}}{b_{3}}\right)\right]^{0.5}\\
    & \times\max\left[0, \min \left(1, \frac{0.25-P}{0.25}\right)\right]^{0.5}
  \end{aligned}
\end{equation}
是当前时步降水量 $P$(\unit{mm.d^{-1}}) 和此前 60 天滑动平均降水$p_{\rm 60d}$ (\unit{mm.d^{-1}})及 10 天滑动平均降水$p_{\rm 60d}$ (\unit{mm.d^{-1}})的函数。 $b_2 = b_3$, 对于热带常绿树取 $4.0$ \unit{mm.d^{-1}},对于热带落叶树取 $1.8$ \unit{mm.d^{-1}}.


\section{泥炭火}

泥炭火燃烧面积为:
\begin{equation}
  A_{\mathrm{b}}=c f_{\mathrm{c l i, p}} f_{\mathrm{peat}}\left(1-f_{\mathrm{sat}}\right) A_{\mathrm{g}}
\end{equation}
其 中 , $c$ 是 全 球 常 数 , 对 于 热 带 泥 炭 火 $c=4.7\times10^{-7}$ (\unit{s^{-1}}) 及 对 于 寒 带 泥 炭 火 $c=2.5\times10^{-9}$ (\unit{s^{-1}}) 是基于 GFED3 的泥炭火燃烧面积估算而来; $f_{\mathrm{cli,p}}$ 为气候影响因子, $f_{\mathrm{peat}}$ 为格点内泥炭地面积比例; $f_{\mathrm{sat}}$ 为水位高于地表的面积比例。

对于热带泥炭火,气候影响因子为:
\begin{equation}
  f_{\mathrm{c l i, p}}=\max \left[0, \min \left(1, \frac{4-p_{\rm 60d}}{4}\right)\right]^{2}
\end{equation}
而对于寒带泥炭火,气候影响因子为:
\begin{equation}
  f_{\mathrm{c l i, p}}=\exp \left(-\pi \frac{\theta_{17 c m}}{0.3}\right) \times \max \left[0, \min \left(1, \frac{T_{17 c m}-T_{\mathrm{frz}}}{10}\right)\right]
\end{equation}
其中, $\theta_{\text{17cm}}$ 为表层 17 cm 土壤湿度。

\section{火灾影响}
\subsection{燃烧引起的碳排放}

对于过火区,第 $j$ 个 PFT 生物量燃烧引起的碳排放${\phi}_{j} $(\unit{g.C.s^{-1}.m^{-2}})为:
%
\begin{equation}
  \phi_{j}=\frac{A_{\mathrm{b},j}}{A_{\mathrm{g}}} C_{j} \times C C_{j}
\end{equation}
%
其中, $A_{\mathrm{b},j}/A_{\mathrm {g}} $ 为燃烧面积率(\unit{s^{-1}}); $
C_{j}=\left(C_{\mathrm{leaf}}, C_{\mathrm{stem}}, C_{\mathrm{root}}, C_{\mathrm{t s}}\right)
$是碳库(\unit{g.C.m^{-2}}),包括叶、茎、根、储存库的碳库; $
C C_{j}=\left(C C_{\mathrm{leaf}}, C C_{\mathrm{stem}}, C C_{\mathrm{root}}, C C_{\mathrm{t s}}\right)_{j}
$是燃烧完全因子 (表~\ref{tab:burning_factors})。 此外,我们假设 50\%和 28\%的 litter 和 coarse woody debris (CWD)的碳库被燃烧后排放到大气。


对于热带泥炭火,泥炭燃烧引起的碳排放(\unit{g.C.s^{-1}.m^{-2}})是泥炭火的燃烧面积率(\unit{s^{-1}})、土壤有机碳(\unit{g.C.m^{-2}})、0.18 的乘积;对于寒带泥炭火,泥炭燃烧引起的碳排放是泥炭火的燃烧面积率(\unit{s^{-1}})与2200 (\unit{g.C.m^{-2}})的乘积。以上常数由已有泥炭火研究数值结果推导而来。

对于热带密闭林砍伐引起的火灾,我们假设燃烧先发生在砍伐区,砍伐区燃烧2遍后才会蔓延到砍伐区以外。对于第一部分,我们按照比例 $\min \left(\frac{A_{\mathrm{b}}}{2 A_{\mathrm{g}}}, D\right)$ 从土地类型发生变化引起的碳排放里扣除。对于第二部分引起的燃烧面积 $\max(0, A_{\mathrm {b}} -2DA_{\mathrm {g}} )$,按照其它火灾去计算火灾的影响。


\subsection{火灾引起的植物组织死亡}

对于燃烧后剩余的碳库部分,即
\begin{equation}
  \begin{aligned}
    C_{j 1}^{\prime}=\big(& C_{\text {leaf}}\left(1-C C_{\text {leaf}}\right), C_{\text {livestem}}\left(1-C C_{\text {stem}}\right), \\
    &C_{\text {deadstem}}\left(1-C C_{\text {stem}}\right), C_{\text {root}}\left(1-C C_{\text {root}}\right), C_{\mathrm{t s}}\left(1-C C_{\mathrm{t s}}\right)\big)_{j}
  \end{aligned}
\end{equation}
%
考虑火灾引起植物组织死亡为:
\begin{equation}
  \Psi_{j 1}=\frac{A_{\mathrm{b},j}}{f_{j} A_{\mathrm{g}}} C_{j 1}^{\prime} \times M_{j 1}
\end{equation}
%
其中, $
M_{j 1}=\left(M_{\text{leaf}}, M_{\text{livestem}}, M_{\text{deadstem}}, M_{\text {root}}, M_{\text {ts}}\right)_{j}
$ 是植物组织死亡因子,相应的碳由植物碳库转移到 litter 或 CWD。此外,还有一部分 livestem 的碳库会转移到 deadstem:
\begin{equation}
  \Psi_{j 2}=\frac{A_{\mathrm{b},j}}{f_{j} A_{\mathrm{g}}} C_{\text {livestem }}\left(1-C C_{\text {stem }}\right) M_{\text {livestem}, 2}
\end{equation}
%
其中, $M_{\text{livestem},2}$ 为这部分 livestem 的死亡因子(表~\ref{tab:burning_factors})。


\subsection{火灾引起的痕量气体和气溶胶排放}

对于第$j$个PFT和第$x$种痕量气体和气溶胶,其排放量 (\unit{g.species.s^{-1}})为:
\begin{equation}
  E_{x,j}=E F_{x,j} \frac{\phi_{j}}{[C]}
\end{equation}
其中 $\phi_{j}$ (\unit{g.C.s^{-1}.m^{-2}})为第 $j$ 个 PFT 的火灾引起的碳排放,$E F_{x,j}$为排放因子~\citep{LiF2019},为单位转化常数。


针叶树、其它寒带和温带树木、热带地区树木、灌木和草,燃烧引起的火灾最高排放高度分别为:4.3、3、2.5、2、1 km。

\begin{landscape}
  \begin{table}[htbp]
    \caption{燃烧完全因子及火灾引起的植物组织死亡因子}
    \label{tab:burning_factors}
    \begin{tabular}{lcccccccccc}
      \toprule
      PFT                                   & $CC_{\mathrm{leaf}}$ & $CC_{\mathrm{stem}}$ & $CC_{\mathrm{root}}$ & $CC_{\mathrm{ts}}$ & $M_{\mathrm{leaf}}$ & $M_{\mathrm{livestem,1}}$ & $M_{\mathrm{deadstem}}$ & $M_{\mathrm{root}}$ & $M_{\mathrm{ts}}$ & $M_{\mathrm{livestem,2}}$ \\ \midrule
      \multicolumn{11}{l}{\textbf{Tree:}}  \\
      NET Temperate                         & 0.80                 & 0.27                 & 0.00                 & 0.45               & 0.80                & 0.13                      & 0.13                    & 0.13                & 0.45              & 0.32                      \\
      NET Boreal                            & 0.80                 & 0.30                 & 0.00                 & 0.50               & 0.80                & 0.15                      & 0.15                    & 0.15                & 0.50              & 0.35                      \\
      NDT Boreal                            & 0.80                 & 0.30                 & 0.00                 & 0.50               & 0.80                & 0.15                      & 0.15                    & 0.15                & 0.50              & 0.35                      \\
      BET Tropical                          & 0.80                 & 0.27                 & 0.00                 & 0.45               & 0.80                & 0.13                      & 0.13                    & 0.13                & 0.45              & 0.32                      \\
      BET Temperate                         & 0.80                 & 0.27                 & 0.00                 & 0.45               & 0.80                & 0.13                      & 0.13                    & 0.13                & 0.45              & 0.32                      \\
      BDT Tropical                          & 0.80                 & 0.27                 & 0.00                 & 0.45               & 0.80                & 0.10                      & 0.10                    & 0.10                & 0.35              & 0.25                      \\
      BDT Temperate                         & 0.80                 & 0.27                 & 0.00                 & 0.45               & 0.80                & 0.10                      & 0.10                    & 0.10                & 0.35              & 0.25                      \\
      BDT Boreal                            & 0.80                 & 0.27                 & 0.00                 & 0.45               & 0.80                & 0.13                      & 0.13                    & 0.13                & 0.45              & 0.32                      \\ \hline
      \multicolumn{11}{l}{\textbf{Shrub:}} \\
      BES Temperate                         & 0.80                 & 0.35                 & 0.00                 & 0.55               & 0.80                & 0.17                      & 0.17                    & 0.17                & 0.55              & 0.38                      \\
      BDS Temperate                         & 0.80                 & 0.35                 & 0.00                 & 0.55               & 0.80                & 0.17                      & 0.17                    & 0.17                & 0.55              & 0.38                      \\
      BDS Boreal                            & 0.80                 & 0.35                 & 0.00                 & 0.55               & 0.80                & 0.17                      & 0.17                    & 0.17                & 0.55              & 0.38                      \\ \hline
      \multicolumn{11}{l}{\textbf{Grass:}} \\
      C3  arctic grass                      & 0.80                 & -----                & 0.00                 & 0.80               & 0.80                & -----                     & -----                   & 0.20                & 0.80              & -----                     \\
      C3  grass                             & 0.80                 & -----                & 0.00                 & 0.80               & 0.80                & -----                     & -----                   & 0.20                & 0.80              & -----                     \\
      C4  grass                             & 0.80                 & -----                & 0.00                 & 0.80               & 0.80                & -----                     & -----                   & 0.20                & 0.80              & -----                     \\ \hline
      \multicolumn{11}{l}{\textbf{Crop:}}  \\
      Crop1                                 & 0.80                 & -----                & 0.00                 & 0.80               & 0.80                & -----                     & -----                   & 0.20                & 0.80              & -----                     \\
      Crop2                                 & 0.80                 & -----                & 0.00                 & 0.80               & 0.80                & -----                     & -----                   & 0.20                & 0.80              & -----                     \\ \bottomrule
    \end{tabular}
  \end{table}
\end{landscape}
