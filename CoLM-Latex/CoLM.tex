% !TEX encoding = UTF-8 Unicode
\documentclass[a4paper,12pt,twoside]{report}
\usepackage[left=2.5cm,right=2.5cm,top=3cm,bottom=3cm]{geometry}
\usepackage[dvipsnames]{xcolor}
\usepackage[UTF8,heading=true,scheme=chinese]{ctex}
\ctexset{
  contentsname=目录,
  listtablename=表格,
  listfigurename=插图,
  part={
    pagestyle = empty
  },
  chapter={
    name={第~,~章},
    number = \arabic{chapter},
    numberformat = \color{blue}\zihao{1} %,
%   afterskip = -0.6 ex plus 0.2ex minus 0.2ex,
  },
  appendix={
    name={附录~},
    % numbering=false,
    number = {\Alph{chapter}}
  }
}

\usepackage[nottoc]{tocbibind} % used to add bibliography to toc with correct bookmark and page number

%\usepackage[T1]{fontenc}
\newcommand{\arial}[1]{{\fontencoding{T1}\usefont{T1}{phv}{m}{n}{#1}}}
\newcommand{\courier}[1]{{\fontencoding{T1}\usefont{T1}{pcr}{m}{n}{#1}}}
\newcommand{\Times}[1]{{\fontencoding{T1}\usefont{T1}{ptm}{m}{n}{#1}}}
\newcommand{\Rom}[1]{{\fontencoding{T1}\usefont{T1}{cmr}{m}{n}{#1}}}
%\usepackage{wasysym}

%\usepackage{xeCJK}
\usepackage{float}
%\usepackage{afterpage}
\usepackage{graphicx}
\usepackage{verbatim}
\usepackage{latexsym}
\usepackage{mathchars}
\usepackage{amsmath}
\usepackage{setspace}
\usepackage[square]{natbib}
\usepackage{rotating}
\usepackage[labelfont=bf]{caption}
\usepackage{subcaption}
\usepackage{multirow}
\usepackage{tabularx}
\usepackage{booktabs}
\usepackage{etoolbox}
\usepackage{fancyhdr}
\usepackage{pifont}
\usepackage[textsize=footnotesize,tickmarkheight=0.2cm]{todonotes}
\usepackage{suffix}

\makeatletter
\patchcmd{\NAT@test}{\else\NAT@nm}{\else\NAT@nmfmt{\NAT@nm}}{}{}
\let\NAT@up\itshape

% Add an English Table of Contents
\newcommand\engcontentsname{Contents}
\newcommand\tableofengcontents{%
    \if@twocolumn
      \@restonecoltrue\onecolumn
    \else
      \@restonecolfalse
    \fi
    \chapter*{\engcontentsname
        \@mkboth{%
           \MakeUppercase\engcontentsname}{\MakeUppercase\engcontentsname}}%
    \@starttoc{toe}% !!!!Define a new contents!!!!
    \if@restonecol\twocolumn\fi
    }
\newcommand\addengcontents[2]{%
    \addcontentsline{toe}{#1}{\protect\numberline{\csname the#1\endcsname}#2}}
%\makeatother

% redefine \part in order to put both Chinese and English part names together
\renewcommand\part{%
   \CTEX@part@break
   \CTEX@setthispagestyle{part}%
   \if@twocolumn
     \onecolumn
     \@tempswatrue
   \else
     \@tempswafalse
   \fi
   \CTEX@setheadingskip \CTEX@part@beforeskip
    \ifodd \CTEX@part@fixskip \CTEX@fixtopskip \fi
    \vspace*{\CTEX@headingskip}%
    \secdef\@part\@spart
}

\def\@part[#1]#2#3{%
    \ifnum \c@secnumdepth >-2\relax
       \ifodd \CTEX@part@numbering
         \CTEX@ifnametrue
         \refstepcounter{part}%
       \else
         \CTEX@ifnamefalse
         \CTEX@makeanchor{part*}%
       \fi 
    \else
       \CTEX@ifnamefalse
       \CTEX@makeanchor{part*}%
    \fi
    \CTEX@gettitle{#1}%
    \CTEX@addtocline{part}{#1}%
    \partmark{#1}%
    \begingroup
       \CTEX@heading@format@initial
       \CTEX@part@format{%
          \CTEX@headinghang{part}%
          {\CTEXifname{\CTEX@partname\CTEX@part@aftername}{}}%
          \CTEX@part@titleformat{#2}%
          \CTEX@part@aftertitle}\par\par
     \endgroup
     \vspace{6ex}
     \begingroup
       \ctexset{
          part={
               name = {Part~},
               number = \Roman{part},
               numberformat = \color{gray}\zihao{1},
               format+ = \color{gray}%
          }
        }
        \addengcontents{part}{#3}
        \CTEX@heading@format@initial
        \CTEX@part@format{%
          \CTEX@headinghang{part}%
           {\CTEXifname{\CTEX@partname\CTEX@part@aftername}{}}%
            \CTEX@part@titleformat{#3}%
            \CTEX@part@aftertitle}\par
     \endgroup
     \@endpart
}

\def\@spart#1#2{%
    \CTEX@ifnamefalse
    \CTEX@makeanchor@spart{part*}%
    \CTEX@gettitle{#1}%
    \begingroup
       \CTEX@heading@format@initial
       \CTEX@part@format{%
         \CTEX@headinghang{part}{}%
         \CTEX@part@titleformat{#1}%
         \CTEX@part@aftertitle}\par\par
    \endgroup
    \vspace{6ex}
    \begingroup
       \ctexset{
          part={
               name = {Part~},
               number = \Roman{part},
               numberformat = \color{gray}\zihao{1},
               format+ = \color{gray}%
          }
        }
        \CTEX@heading@format@initial
        \CTEX@part@format{%
          \CTEX@headinghang{part}{}%
          \CTEX@part@titleformat{#2}%
          \CTEX@part@aftertitle}\par
     \endgroup
    \@endpart
}

\newcommand\epart[1]{%
  \addtocounter{part}{-1}
  \begingroup
%    \let\cleardoublepage\relax
    \renewcommand\addcontentsline[3]{}
    \ctexset{
      part={
        name = {Part~},
        number = \Roman{part},
        numberformat = \color{gray}\zihao{1},
        format+ = \color{gray}%,
%        beforeskip = -2.5ex plus 0.2ex minus 0.2ex
%        beforeskip = -50pt
%        afterskip = 12.0ex plus 0.2ex minus 0.2ex
      }
    }
    \part{#1}
   \endgroup
   \addengcontents{part}{#1}
}
\WithSuffix\newcommand\epart*[1]{%
  \begingroup
    \let\cleardoublepage\relax
    \ctexset{
      part={
        number = \Roman{\thepart},
%        name = {Chapter~},
%        numberformat = \color{gray}\zihao{1},
        format+ = \color{gray}%,
%        beforeskip = -2.5ex plus 0.2ex minus 0.2ex
%        afterskip = 12.0ex plus 0.2ex minus 0.2ex
      }
    }
    \part*{#1}
   \endgroup
   \addcontentsline{toe}{part}{#1}
}

\newcommand\echapter[1]{%
  \addtocounter{chapter}{-1}
  \begingroup
     \let\clearpage\relax
     \renewcommand{\chaptermark}[1]{}
%    \let\markleft\relax
%    \let\markright\relax
%    \let\markboth\relax
    \renewcommand\addcontentsline[3]{}
    \ctexset{
      chapter={
        name = {Chapter\space},
        numberformat = \color{gray}\zihao{1},
        format+ = \color{gray},
%        beforeskip = -2.5ex plus 0.2ex minus 0.2ex,
        beforeskip = -55pt
%        afterskip = 12.0ex plus 0.2ex minus 0.2ex
      }%,
%      appendix={
%        name = {Appendix\space},
%        number = {\Alph{chapter}}
%      }
    }
    \chapter{#1}
   \endgroup
   \addengcontents{chapter}{#1}
}

\WithSuffix\newcommand\echapter*[1]{%
  \begingroup
    \let\clearpage\relax
    \renewcommand{\chaptermark}[1]{}
    \ctexset{
      chapter={
        format+ = \color{gray},
%        beforeskip = -2.5ex plus 0.2ex minus 0.2ex,
        beforeskip = -55pt
%        afterskip = 12.0ex plus 0.2ex minus 0.2ex
      },
      appendix={
        name = {Appendix\space},
        number = {\Alph{chapter}}
      }
    }
    \chapter*{#1}
   \endgroup
   \addcontentsline{toe}{chapter}{#1}
}

% new command to change appendix chapater name
\newcommand\eappendixchapter[1]{%
  \addtocounter{chapter}{-1}
  \begingroup
     \let\clearpage\relax
     \renewcommand{\chaptermark}[1]{}
     \renewcommand\addcontentsline[3]{}
     \ctexset{
       chapter={
        name = {Appendix\space},
        numberformat = \color{gray}\zihao{1},
        format+ = \color{gray},
        beforeskip = -55pt
%        afterskip = 12.0ex plus 0.2ex minus 0.2ex
      }
    }
    \chapter{#1}
   \endgroup
   \addengcontents{chapter}{#1}
}

\newcommand\esection[1]{%
  \addtocounter{section}{-1}
  \begingroup
    \renewcommand\addcontentsline[3]{}
    \renewcommand{\sectionmark}[1]{}
    \ctexset{
      section={
        format+ = \color{gray},
        beforeskip = -50pt
%        afterskip = 2.0ex plus 0.2ex minus 0.2ex
      }
    }
    \newcommand\sectionprelude{%
      \vspace{-4.5ex}
    }
    \newcommand\sectionpostlude{%
      \vspace{0ex}
    }
    \sectionprelude\section{#1}\sectionpostlude
  \endgroup
  \addengcontents{section}{#1}
}

\newcommand\esubsection[1]{%
  \addtocounter{subsection}{-1}
  \begingroup
    \renewcommand\addcontentsline[3]{}
    \ctexset{
      subsection={
        format+ = \color{gray},
        beforeskip = -30pt
%        afterskip = 2.0ex plus 0.2ex minus 0.2ex
      }
    }
    \newcommand\subsectionprelude{%
      \vspace{-4ex}
    }
    \newcommand\subsectionpostlude{%
      \vspace{0ex}
    }
    \subsectionprelude\subsection{#1}\subsectionpostlude
  \endgroup
  \addengcontents{subsection}{#1}
}

% Hack the definition of \listoffigures and \listoftables to add an entry to the English table of contents
\newcommand\elistchapter[1]{%
  \begingroup
     \let\clearpage\relax
     %\renewcommand{\chaptermark}[1]{}
     \renewcommand{\chaptermark}[1]{}
     \renewcommand\addcontentsline[3]{}
     \ctexset{
       chapter={
        format+ = \color{gray},
        beforeskip = -45pt
      }
    }
    \chapter*{#1}
   \endgroup
   \addcontentsline{toe}{chapter}{#1}
}

\renewcommand{\contentsname}{目~录}

\renewcommand\listoffigures{%
%    \let\origaddvspace\addvspace
    \renewcommand{\addvspace}[1]{}
    \if@twocolumn
      \@restonecoltrue\onecolumn
    \else
      \@restonecolfalse
    \fi
    \chapter*{插~图}%
    \addcontentsline{toc}{chapter}{\listfigurename}
    \@mkboth{\MakeUppercase\listfigurename}%
              {\MakeUppercase\listfigurename}%
    \elistchapter{List of Figures}
    \@starttoc{lof}%
    \if@restonecol\twocolumn\fi
%    \renewcommand{\addvspace}[1]{\origaddvspace{#1}}
}

\renewcommand\listoftables{%
%    \let\origaddvspace\addvspace
    \renewcommand{\addvspace}[1]{}
    \if@twocolumn
      \@restonecoltrue\onecolumn
    \else
      \@restonecolfalse
    \fi
    \chapter*{表~格}%
    \addcontentsline{toc}{chapter}{\listtablename}
      \@mkboth{%
          \MakeUppercase\listtablename}%
         {\MakeUppercase\listtablename}%
    \elistchapter{List of Tables}
    \@starttoc{lot}%
    \if@restonecol\twocolumn\fi
%    \renewcommand{\addvspace}[1]{\origaddvspace{#1}}
}


% Let natbib handle the specific bibliogrphy format
\renewenvironment{thebibliography}[1]{%
 \bibsection
 \echapter*{Bibliography}
 \parindent\z@
 \bibpreamble
 \bibfont
 \list{\@biblabel{\the\c@NAT@ctr}}{\@bibsetup{#1}\global\c@NAT@ctr\z@}%
 \ifNAT@openbib
   \renewcommand\newblock{\par}%
 \else
   \renewcommand\newblock{\hskip .11em \@plus.33em \@minus.07em}%
 \fi
 \sloppy\clubpenalty4000\widowpenalty4000
 \sfcode`\.\@m
 \let\NAT@bibitem@first@sw\@firstoftwo
    \let\citeN\cite \let\shortcite\cite
    \let\citeasnoun\cite
}{%
 \bibitem@fin
 \bibpostamble
 \def\@noitemerr{%
  \PackageWarning{natbib}{Empty `thebibliography' environment}%
 }%
 \endlist
 \bibcleanup
}%

\makeatother

%\usepackage[none]{hyphenat}
\fancypagestyle{fancy}{%
  \fancyhf{}
  \fancyhead[RO]{\textnormal{\kaishu\nouppercase\rightorleftmark}}
  \fancyhead[LE]{\textnormal{\kaishu\nouppercase\leftmark}}
% \fancyhead[RE,LO]{--\ \thepage\ --}
% \fancyhead[CO]{\textnormal{\kaishu\nouppercase\rightmark}}
% \fancyhead[CE]{\textnormal{\kaishu\nouppercase\leftmark}}
  \fancyfoot[C]{--\ \thepage\ --}
  \renewcommand{\headrulewidth}{0.4pt}
}

\makeatletter
\newcommand{\rightorleftmark}{%
  \begingroup\protected@edef\x{\rightmark}%
  \ifx\x\@empty
    \endgroup\leftmark
  \else
    \endgroup\rightmark
\fi}
\makeatother

%Making the pagestyle of part and chapter page fancier
\fancypagestyle{plain}{%
  \fancyhf{}
  \cfoot{--\ \thepage\ --}
  \renewcommand{\headrulewidth}{0pt}
}
%\pagestyle{fancy}
% \sectionmark 的重定义需要在 \pagestyle 之后生效
%\renewcommand\sectionmark[1]{%
%    \markboth{\CTEXifname{\CTEXthesection}{}#1}}
%\renewcommand\chaptermark[1]{%
%    \markright{\CTEXifname{\CTEXthechapter}{}#1}}
\setlength{\headheight}{15pt}
\addtolength{\topmargin}{-2.5pt}
\setlength{\marginparwidth}{2.25cm}

%\input{blocked.sty}
%\input{uhead.sty}
\input{boxit.sty}
\pagestyle{empty}

\makeatletter  %to avoid error messages generated by "\@". Makes Latex treat "@" like a letter

%\linespread{1.5}
\def\submitdate#1{\gdef\@submitdate{#1}}


\def\maketitle{
  \begin{titlepage}{
    %\linespread{1.5}
     \vskip 1.5in
    \Large \bf \@title \par
  }
  \vskip 0.3in
  \par
  {\Large \@author}
  \vskip 4in
  \end{titlepage}
}

\def\titlepage{
  \newpage
  \centering
  \linespread{1}
  \normalsize
  \vbox to \vsize\bgroup\vbox to 9in\bgroup
}
\def\endtitlepage{
  \par
  \kern 0pt
  \egroup
  \vss
  \egroup
  \newpage
  \thispagestyle{empty}
  \null\vfill
  %\centerline{\textcopyright~2007-2022~ Land-Atmosphere Interaction Research Group at Sun Yat-sen University}
  \centerline{\textcopyright~2003--2024~   戴永久陆面模式研发团队}
  \vskip 3ex
  \vfill
  \cleardoublepage
}

\def\abstract{
  \begin{center}{
    \huge \bf Abstract}
  \end{center}
  \small
  %\def\baselinestretch{1.5}
  \linespread{1.5}
  \normalsize
}
\def\endabstract{
  \par
}

\newenvironment{acknowledgements}{
  \cleardoublepage
  \begin{center}{
    \huge \bf Acknowledgements}
  \end{center}
  \small
  \linespread{1.5}
  \normalsize
}{\cleardoublepage}
\def\endacknowledgements{
  \par
}

\newenvironment{dedication}
  {\clearpage           % we want a new page
   \thispagestyle{empty}% no header and footer
   \vspace*{\stretch{1}}% some space at the top 
   \itshape             % the text is in italics
   \center              
  }
  {\par % end the paragraph
   \vspace{\stretch{3}} % space at bottom is three times that at the top
   \clearpage           % finish off the page
  }

\def\preface{%
    \pagenumbering{roman}%
    \pagestyle{plain}%
    %\doublespacing
}

\def\body{%
    \cleardoublepage
    %\pagestyle{uheadings}
    \pagestyle{fancy}
    \tableofcontents   % 中文目录
    \tableofengcontents % 英文目录
    %\pagestyle{plain}
    %\cleardoublepage
    %\pagestyle{uheadings}
    \pagestyle{fancy}
    \listoftables
    %\pagestyle{plain}
    %\cleardoublepage
    %\pagestyle{uheadings}
    \pagestyle{fancy}
    \listoffigures
    %\pagestyle{plain}
    %\cleardoublepage
    \clearpage
    %\pagestyle{uheadings}
    \pagestyle{fancy}
    \pagenumbering{arabic}
    \setlength{\parskip}{2ex plus 0.5ex minus 0.2ex}
    \setlength{\parindent}{2em}
    %\doublespacing
}

\makeatother  %to avoid error messages generated by "\@". Makes Latex treat "@" like a letter

%\addtolength{\parskip}{8pt}

\usepackage[framemethod=default]{mdframed}

\global\mdfdefinestyle{exampledefault}{%
  linecolor=lightgray,linewidth=1pt,%
  leftmargin=1cm,rightmargin=1cm,%
}

\newenvironment{mymdframed}[1]{%
  \mdfsetup{%
    frametitle={\colorbox{lightgray}{\kaishu\,#1\,}},
    frametitleaboveskip=-0.5\ht\strutbox,
    frametitlealignment=\raggedright
  }%
  \begin{mdframed}[style=exampledefault]
}{\end{mdframed}}

\newcommand{\EngTitlePage}{%
   \clearpage
   \thispagestyle{empty}
   \vspace*{1in}
   \begin{center}
   \fontsize{22.5}{20}\selectfont {\bf Description of the Common Land Model\\[2ex] (CoLM 2024)}
   \end{center}
   \vskip 1in
   \textbf{\Large Coordinating Lead Authors:}\\
   {\large Hua Yuan, Yongjiu Dai} \\[3ex]

   \noindent\textbf{\Large Lead Authors:}\\
   {\large Fang Li, Lu Li, Xianxiang Li, Shaofeng Liu, Xingjie Lu, Nan Wei, Zhongwang Wei, Xiaodong Zeng, Shulei Zhang, Shupeng Zhang} \\[3ex]

   \noindent\textbf{\Large Contributing Authors:}\\
   {\large Wenzong Dong, Hanwen Fan, Shuyang Guo, Zulong Huang, Hongbin Liang, Wanyi Lin, Zhuo Liu, Jiahao Shi, Aobo Tan, Xionghui Xu}
}

%\setCJKfamilyfont{jamspmi}{MS PMincho}
%\setCJKfamilyfont{jaIPAmi}{IPAMincho}
%\newcommand\JapMSPMi{\CJKfamily{jamspmi}\CJKnospace}
%\newcommand\JapIPAMi{\CJKfamily{jaIPAmi}\CJKnospace}

\newcommand{\ipc}{{\sf ipc}}

\newcommand{\Prob}{\bbbp}
\newcommand{\Real}{\bbbr}
\newcommand{\real}{\Real}
\newcommand{\Int}{\bbbz}
\newcommand{\Nat}{\bbbn}

\newcommand{\NN}{{\sf I\kern-0.14emN}}   % Natural numbers
\newcommand{\ZZ}{{\sf Z\kern-0.45emZ}}   % Integers
\newcommand{\QQQ}{{\sf C\kern-0.48emQ}}   % Rational numbers
\newcommand{\RR}{{\sf I\kern-0.14emR}}   % Real numbers
\newcommand{\KK}{{\cal K}}
\newcommand{\OO}{{\cal O}}
\newcommand{\AAA}{{\bf A}}
\newcommand{\HH}{{\bf H}}
\newcommand{\II}{{\bf I}}
\newcommand{\LL}{{\bf L}}
\newcommand{\PP}{{\bf P}}
\newcommand{\PPprime}{{\bf P'}}
\newcommand{\QQ}{{\bf Q}}
\newcommand{\UU}{{\bf U}}
\newcommand{\UUprime}{{\bf U'}}
\newcommand{\zzero}{{\bf 0}}
\newcommand{\ppi}{\mbox{\boldmath $\pi$}}
\newcommand{\aalph}{\mbox{\boldmath $\alpha$}}
\newcommand{\bb}{{\bf b}}
\newcommand{\ee}{{\bf e}}
\newcommand{\mmu}{\mbox{\boldmath $\mu$}}
\newcommand{\vv}{{\bf v}}
\newcommand{\xx}{{\bf x}}
\newcommand{\yy}{{\bf y}}
\newcommand{\zz}{{\bf z}}
\newcommand{\oomeg}{\mbox{\boldmath $\omega$}}
\newcommand{\res}{{\bf res}}
\newcommand{\cchi}{{\mbox{\raisebox{.4ex}{$\chi$}}}}
%\newcommand{\cchi}{{\cal X}}
%\newcommand{\cchi}{\mbox{\Large $\chi$}}

% Logical operators and symbols
\newcommand{\imply}{\Rightarrow}
\newcommand{\bimply}{\Leftrightarrow}
\newcommand{\union}{\cup}
\newcommand{\intersect}{\cap}
\newcommand{\boolor}{\vee}
\newcommand{\booland}{\wedge}
\newcommand{\boolimply}{\imply}
\newcommand{\boolbimply}{\bimply}
\newcommand{\boolnot}{\neg}
\newcommand{\boolsat}{\!\models}
\newcommand{\boolnsat}{\!\not\models}


\newcommand{\op}[1]{\mathrm{#1}}
\newcommand{\s}[1]{\ensuremath{\mathcal #1}}

% Properly styled differentiation and integration operators
\newcommand{\diff}[1]{\mathrm{\frac{d}{d\mathit{#1}}}}
\newcommand{\diffII}[1]{\mathrm{\frac{d^2}{d\mathit{#1}^2}}}
\newcommand{\intg}[4]{\int_{#3}^{#4} #1 \, \mathrm{d}#2}
\newcommand{\intgd}[4]{\int\!\!\!\!\int_{#4} #1 \, \mathrm{d}#2 \, \mathrm{d}#3}

% Large () brackets on different lines of an eqnarray environment
\newcommand{\Leftbrace}[1]{\left(\raisebox{0mm}[#1][#1]{}\right.}
\newcommand{\Rightbrace}[1]{\left.\raisebox{0mm}[#1][#1]{}\right)}

% Funky symobols for footnotes
\newcommand{\symbolfootnote}{\renewcommand{\thefootnote}{\fnsymbol{footnote}}}
% now add \symbolfootnote to the beginning of the document...

\newcommand{\normallinespacing}{\renewcommand{\baselinestretch}{1.5} \normalsize}
\newcommand{\mediumlinespacing}{\renewcommand{\baselinestretch}{1.2} \normalsize}
\newcommand{\narrowlinespacing}{\renewcommand{\baselinestretch}{1.0} \normalsize}
\newcommand{\bump}{\noalign{\vspace*{\doublerulesep}}}
\newcommand{\cell}{\multicolumn{1}{}{}}
\newcommand{\spann}{\mbox{span}}
\newcommand{\diagg}{\mbox{diag}}
\newcommand{\modd}{\mbox{mod}}
\newcommand{\minn}{\mbox{min}}
\newcommand{\andd}{\mbox{and}}
\newcommand{\forr}{\mbox{for}}
\newcommand{\EE}{\mbox{E}}

\newcommand{\deff}{\stackrel{\mathrm{def}}{=}}
\newcommand{\syncc}{~\stackrel{\textstyle \rhd\kern-0.57em\lhd}{\scriptstyle L}~}

\def\coop{\mbox{\large $\rhd\!\!\!\lhd$}}
\newcommand{\sync}[1]{\raisebox{-1.0ex}{$\;\stackrel{\coop}{\scriptscriptstyle
#1}\,$}}

\newtheorem{definition}{Definition}[chapter]
\newtheorem{theorem}{Theorem}[chapter]

\newcommand{\Figref}[1]{Figure~\ref{#1}}
\newcommand{\fig}[3]{
  \begin{figure}[!ht]
    \begin{center}
      \scalebox{#3}{\includegraphics{figs/#1.ps}}
      \vspace{-0.1in}
      \caption[ ]{\label{#1} #2}
    \end{center}
  \end{figure}
}

\newcommand{\figtwo}[8]{
  \begin{figure}
    \parbox[b]{#4 \textwidth}{
      \begin{center}
        \scalebox{#3}{\includegraphics{figs/#1.ps}}
        \vspace{-0.1in}
        \caption{\label{#1}#2}
      \end{center}
    }
    \hfill
    \parbox[b]{#8 \textwidth}{
      \begin{center}
        \scalebox{#7}{\includegraphics{figs/#5.ps}}
        \vspace{-0.1in}
        \caption{\label{#5}#6}
      \end{center}
    }
  \end{figure}
}

\setlength{\abovedisplayskip}{8pt}
\setlength{\belowdisplayskip}{8pt}
\setlength{\abovedisplayshortskip}{8pt}
\setlength{\belowdisplayshortskip}{8pt}

\setlength{\parskip}{0em}
\setcounter{secnumdepth}{3}
\usepackage{enumitem}
\usepackage{wasysym}
\usepackage{textcomp}
\usepackage{rotating}
\usepackage{amssymb}
\usepackage{afterpage} % avoid blank page before landscape environment
\usepackage{pdflscape}
\usepackage{xr}
\usepackage{color}
\usepackage{textcomp}
\usepackage{threeparttable}
\usepackage{array}
\usepackage{longtable} % for 'longtable' environment
\usepackage{threeparttablex} % for 'ThreePartTable' environment
\usepackage{makecell}
\raggedbottom

\usepackage{siunitx}
%\PassOptionsToPackage{hyphens}{url}
\usepackage[%hidelinks
citecolor=Blue,linkcolor=Blue,urlcolor=Blue,colorlinks=True,breaklinks=true]{hyperref}
\usepackage{xurl}
\usepackage{setspace}
\usepackage{doi}
\setstretch{1.5}
\usepackage{bookmark}

\begin{document}
%\include{uhead.sty}

\title{\huge {\bf 通用陆面模式 2024}\\[3em]
%\vspace{2mm}
  \fontsize {22}{24}
  \bf{ Description of The Common Land Model (CoLM2024)}\\[3in]
  \fontsize {20}{23}%\\[3in]
% \vskip 3in
}

\author{%
  \large{ 中山大学 }\\[0.1in]
  {\bf 大气科学学院}\\[1in]
% \vskip 1in
  \upshape
  \large%\\[0.5in]
% \vskip 0.5in
  2024 年 09 月%
}

%\normallinespacing
\maketitle
\preface
%\input{acknowledgements/acknowledgements}
%
\chapter*{List of symbols Variables}
%\section*{}

List of symbols Variables:
$ET$ \ \ \ \  蒸散发, mm/s \\
...

%\input{dedication/dedication}
%\input{quotes/quotes}

%\listoftodos[待核查/修改列表]

\body
\chapter*{前~言}
\addcontentsline{toc}{chapter}{前言}
\markboth{前言}{前言}
\echapter*{Preface}
%\addengcontents{chapter}{Foreword}

地球是作为一个系统在运行着。地球系统将大气、陆地(含陆地生物圈)、海洋(含海洋生物圈、海冰)、冰冻圈(大陆冰原)和岩石圈视为一个整体,由一系列相互作用过程(包括系统各组分之间的相互作用,物理、化学和生物三大基本过程的相互作用,以及人类与地球的相互作用)联系起来的复杂非线性的多重耦合系统。

地球系统模式(ESMs)是描述这一耦合系统的数学物理模式,旨在定量理解控制整个地球系统相互作用着的物理、化学和生物过程,人类活动影响与反馈,作为一个复杂、自适应性系统的结构、功能和运行机制。ESMs是地球系统预测的基本工具或唯一手段。

与传统物理气候系统模式相比,ESMs还包括元素(碳:氮:磷等)循环、陆地和海洋生态系统及生物地球化学过程、大气化学过程、以及自然和人为干扰。ESMs已超越气候模式对大气、陆地和海洋状态的物理描述,将气候预测扩展到更全面的地球系统预测,包括生态系统、水文水资源、人类活动(农林牧业、水利、城市和交通等)的影响与反馈等。

陆面过程模式作为数值天气预报模式/气候模式/ESMs的重要分系统模式。自20世纪60年代末70年代初始,已由简单“水桶”水分平衡和简单能量平衡陆面参数化方案,到当前的包含陆面物理、水文、植被生理生态、生物地球化学和人类活动的陆面过程模式,陆面过程模式的内涵和外延已全面扩充,内容已得到极大的丰富和发展。

一方面,它的一些重要的关键过程数学建模问题,例如,“冠层/积雪/城市辐射传输过程”,“水分相变过程”,“土壤水力学过程-可变饱和土壤水运动算法”,“产汇流水文过程:降水-冠层截流-土壤表面径流-坡面流-侧向流与基流(地下水流)-湿地湖泊河汇流-河道径流-人工水库-调水(包括地下水)-灌溉与排涝-工业与生活用水”,“植被光合作用与气孔导度关系,光合作用-水分关系”,“碳循环:包括总初级生产、自养呼吸、落叶、异养呼吸和野火等组成部分-植物和土壤碳储存池-其他生物地球化学通量包括沙尘、野火化学排放、生物挥发性有机化合物、活性氮循环以及湿地甲烷排放”,“通过考虑植被类型、叶面积、叶片上气孔以及碳和氮库来表示陆地生态系统”,以及“从叶片到植物冠层、从生态系统到景观再到生物群落的空间尺度问题;对当前环境条件的即时生理响应(如气孔导度、光合作用和呼吸,季节性落叶现象以及枯萎),在数十年和数百年时间尺度上的生态系统结构和生物地理对自然干扰、人为干扰和气候变化响应的时间尺度问题”等等,已逐步揭示清楚和数学建模。

另一方面,开展了一系列大规模的现场实验,例如,FIFE、BOREAS、LBA和中国青藏高原、黑河流域等项目,运用涡度相关技术精确测量生物圈与大气之间碳、水和能量的循环过程的通量观测网络FLUXNET等,为我们提供了深入理解陆面水、能量和生态系统功能的基础,提供了验证陆-气方案中所涉及的过程模型和尺度假设,这些实验还推动了将卫星数据解译为模式所需的全球地表参数集的发展。基于实地调研普查和卫星观测反演研制了全球高分辨率地形高程、土地覆盖/土地利用、土壤属性、植被属性、农业耕作、城市结构和功能、河流湖泊和冰川冻土等巨量数据集。得益于能严格解释辐射并将其解译为有用生物物理量的技术和方法的进步,从而使卫星遥感能提供全球陆地表面参数场最可行、一致和准确方法和数据集。卫星遥感数据已扮演至关重要的角色,成为可获取和可用于以改进和验证陆面模式的全球数据。正因为数据的不断丰富,极大推动陆面模式的发展,实现从不可能到可能。

再一方面,它和其它学科的交叉、渗透,包括气象学、水文学、植物生理生态学、生物地球化学、农学、卫星遥感应用、全球变化、应用数学、高性能计算等,它集成不同学科的优秀成果,提供了深入整合领域和丰富研究工具和方法,已为陆面过程建模工作注入勃勃生机和崭新内容。

我们于上世纪90年代初,开展研制服务于大气环流模式(AGCM)的陆面过程模式,1994年完成中国科学院大气物理研究所陆面过程模式(IAP LSM version 1994)研制,并实现与IAP-AGCM的耦合运行。1997年末启动研制通用陆面模式(The Common Land Model, CoLM),至此,已迭代了四个版本(即,CLM 1.0, CoLM 2004, CoLM 2014, CoLM 2024),十年发布一个版本。通用“Common”的最初旨意是为气候模式科学家们提供一个集成研究平台。CoLM 2024实现了真正字面意义上的“通用”。它可广泛应用于数值天气预报/气候预测、水文水资源、生态环境、城市、农林牧等行业的科学研究和精细化业务,适用于多尺度(\textasciitilde1米 至\textasciitilde100公里)应用。

地球系统模式开启了地球系统预测新时代,其中陆面模式的内涵和外延的全面扩充,大大拓展了ESMs的功能。

\textit{1、在数值天气/气候模拟研究中的作用}

天气/气候系统模式,将陆地视为一个物质和能量流动的物理系统,通过多种物理化学生命过程在不同空间时间尺度上影响天气/气候。从天气/气候模式工作者的角度来看,陆面过程模式在陆-气物质和能量通量的计算中起着基础性作用,尽可能提供准确的陆-气通量。陆-气间物质和能量通量对大气温度、水汽、降水、云、辐射以及环流等大气过程和状态产生重要影响。

陆-气垂向之间,水汽从陆面输送到大气底层。随后,大气中的水汽通过大尺度风和垂直扩散或对流进一步混合并向空间扩散。在水汽浓度超过饱和的气柱层内形成云和降水。这种过量的水汽要么来自大尺度上升运动,要么由湿对流产生。地表通量变化也可以对边界层云产生反馈作用,与深对流云一起,会改变地表能量通量。此外,地表加热的空间结构受到地形分布的制约,从而形成积云对流集中的区域。由于云凝结核种类的差异,陆地上云的形成过程可能与海洋上的云有很大不同。

不同尺度的陆地非连续性景观之间(海-陆、湖-陆、山坡-山谷、山地-平原、城-乡、灌溉-非灌溉农田等等)动力和热力差异会引发不同尺度的大气运动。不同尺度的环流如海风、湖风、山谷风等中小尺度环流,乃至大陆尺度或全球尺度大气环流。

陆地属性、状态及其变化,对大气在局地、区域和全球尺度上的直接和间接影响,已有大量的论述,粗糙度、反照率、土壤湿度、植被、积雪等对天气/气候的直接和间接,局地到全球的影响已有充分的研究。不同尺度的陆地属性、状态及其变化,包括自然因素(极端天气/气候、水文、生态、地质事件导致地表景观和属性的变化:洪涝、高温热浪、火灾、荒漠化、沼泽化、山崩地裂等)和人为因素(人类为了生存和美好生活行动:农林牧、城市建设/管理/生产生活、交通、水文水资源工程、绿水青山或生态环境工程等等)导致的改变,对不同尺度(时间:局地至全球,空间:瞬时至百年)大气过程的影响(由陆面模式输入到大气,对不同尺度大气过程的影响),反馈(由大气模式输入陆面,导致陆面的变化),及循环反馈直至平衡或跃级到新的平衡态(例如,高温、干旱、多雨、低温冰冻等天气/气候极端事件成为常态等),以及演变过程中的突变或极端事件。

正因为精细化的陆面模式,为诸多方面的陆-气相互作用研究提供了科学方法和工具。

\textit{2、在水文/水资源管理模拟研究中的作用}

基于物理原理的水文过程模式是深入理解、监测和预测水循环的重要工具。陆面模式水文过程建模集成了近年来水文学领域的最新进展,水文学家深度参与了模式的研发。陆面模式精细描述了对植被冠层截留过程、积雪和土壤水文过程(垂向、侧向流动和地下水),产、汇流过程,湿地、湖泊、水库、河道径流过程,水文连通性及多尺度空间变异性,人类活动等关键水文过程,以及全面的水文模式基准验证(流域或关键带测站数据)与系统性地严格评估。

陆面模式适用不同空间尺度(\textasciitilde1米 至\textasciitilde100公里)水文过程模拟。其中,基于流域水文响应单元的水文过程建模,明确解决了次流域尺度的地形结构方面存在问题,显著改善了次流域尺度上的水分、能量及生物地球化学过程与通量模拟,实现在空间和水文学过程要素之间进行综合,提升模式的物理真实性和普适性,这些努力已超越传统水文学中采用的方法。

由于地域与过程的复杂性、广泛的模型需求,以及众多建模群体等,导致当前水文领域模型数量和种类繁多。过多水文模型,可能会导致研究资源浪费和低效率。基于服务于天气/气候/地球系统模式的陆面模式这一系统集成平台,开展水文模型研发、机理研究与业务应用等,将极大推动水文过程建模。

\textit{3、在生态系统和农业生产与管理模拟中的作用}

气候模式迈向地球系统模式(ESMs)的得益于陆面分系统模式的重大进展。正因为此,ESMs可以模拟陆地生态系统以及生物地球化学循环,为相关于气候过程的生态研究提供了一个通用框架,包括:物候、生长季节长度和群落组成,水利用效率和生产力,野火、昆虫、极端事件和人类活动引起的干扰等;以及脆弱性分析、影响和适应性分析,气候变化缓解措施等。ESMs使我们可以超越传统对大气状态的物理描述,转而可关注与社会相关的问题。例如,火灾风险、栖息地丧失、水资源可用性以及作物、木材产量等问题;干旱、热浪、不稳定且强烈的降水分布、风暴和洪水,$\rm CO_2$、$\rm O_3$,土壤侵蚀、土壤盐碱化,用水需求和可用性等影响农业生产力及稳定性和农民生计问题。ESMs不仅提供了评估未来气候变化影响的方法,还能确定气候变化对生物圈和农业生产与管理的影响,以及生态系统固碳能力和农业适应气候变化策略有效性的评估与预估提供基础理论与应用平台。

本报告是一系列介绍了通用陆面模式2024版本(The Common Land Model version 2024, CoLM 2024)的第一册,为用户提供了 CoLM 的全面详尽的描述,包括其所有的陆面物理、化学、水文、生态和人类活动等过程的数学建模的科学和算法方法。此技术说明覆盖了所有在CoLM版本2024之前发布的版本,并将随着新版本发布及功能添加进行更新。

本报告由八部分组成,共分三十章。第一部分为引言,介绍通用陆面模式CoLM的由来及其版本发展历史,并简要介绍CoLM 2024的新特性。第二部分为模式构架与基础数据,讲明模型的总体结构与计算框架,特别是本版本新增加的非结构网格、流域单元网格以及植被功能性次网格和植物群落次网格结构;另外,对模型所需的地表覆盖、土壤、植被、水文、城市等数据进行简要介绍。第三部分为地表通量方案,包括地表辐射通量计算、地表湍流通量计算、光合作用与气孔导度计算、植被水力模式以及降水与地表的能量交换。第四部分介绍植被冠层、积雪和土壤温度计算方案。第五部分为植被冠层、积雪和土壤水分计算方案,包括植被冠层雨水截留过程以及积雪、土壤水分垂直运动过程。第六部分为水文过程,包括耦合径流模型的产流、汇流过程及侧向流的模拟;对湖泊模式、冰川模式和湿地模式进行介绍。第七部分为生物地球化学循环过程,说明植被和土壤碳库结构,植被和土壤凋落物的生物地球化学循环过程及其预热加速功能,并描述火灾模块。第八部分与人类活动相关,涵盖对城市模块、作物模块、水库模块和土地利用土地覆盖变化模块的描述。

CoLM由中山大学大气科学学院戴永久研究团队开发和维护。CoLM属于开源系统,我们欢迎任何个人或实体均可在任何目的下无需支付任何费用下载和使用。最新的模式程序和文档可以通过以下网址获取:~\url{https://github.com/CoLM-SYSU/CoLM202X}(源代码),\url{https://github.com/CoLM-SYSU/CoLM-doc}(科学与技术报告)和~\url{https://github.com/CoLM-SYSU/CoLM-UsersGuide}(用户手册)。

\chapter *{致谢 (Acknowledgments)}

作者感谢自2004年以来对CoLM的发展做出重大贡献的人员。他们是:

曾庆存、Robert E. Dickinson、陈海山、梁信忠、曾旭斌、周黎明、牛国跃, …

纪多颖、张倩、朱司光、张香香,以及其他所有参与人员。

\part{引言}{Introduction}
%\epart{Introduction}
\chapter{引言}

陆地是天气/气候/地球系统的重要组成部分,其物理、化学、生物过程深刻影响着陆地与大气、陆地与海洋之间的能量和物质的交换。陆—气、陆—海界面是人类活动的主要场所,随着人类社会的发展,人类活动导致的地球陆面状况变化,深刻影响了陆—气、陆—海之间物质与能量交换、区域气候和生态环境的变化。这些变化已对自然和人类产生了巨大的影响。准确描述陆面物理、化学、生物过程,准确计算陆面状态以及陆—气、陆—海界面的物质和能量交换通量,对天气/气候数值预报预测业务,以及充分理解全球变化所带来的水安全、粮食安全、生态环境恶化等问题的形成机制,制定相应的对策,具有重要的科学和社会意义。

陆面过程是指发生在陆地表层的所有物理、化学、生物过程,及其与大气、海洋的相互作用过程。陆面过程模式是指定量描述这些过程以及研究人类活动与环境相互作用的数学物理模式,并可通过计算机实现仿真,是数值天气/气候/地球系统模式的不可或缺的重要组成部分,是陆面过程机理以及人类活动与全球变化关系研究的重要手段,也是全球气象-水文-生态的精细化预报的核心技术。

本报告目的是详细描述通用陆面模式(CoLM 版本2024)的陆面物理、化学、水文、生态和人类活动等过程的数学建模。本报告与CoLM 2024用户指南一起为用户提供了CoLM的全面详尽的科学描述,和简单易用的操作指南。


\section{模式历史 (Model History and Overview)}

\subsection{通用陆面模式由来 (Inception of CoLM)}
从20世纪60年代的简单水桶(bucket-type)陆面过程模式到21世纪初的土壤-植被-大气相互作用过程模式,在期间40年里已经发展出数十个陆面过程模式来计算这些通量。国际陆面过程模式比较计划(PILPS)表明,即使使用相同的大气强迫数据和相似的陆面参数,不同的陆面模式仍然给出显著不同的地表通量和土壤湿度,部分原因是由于参与模式中个别过程的公式和编码架构的差异~\citep{Henderson-Sellers_95}。另一方面,大多数陆面模式共享许多通用组件,这表明需要开发一个具有模块化结构的可通用的陆面模式,以促进对新问题的探索,减少对过去工作的重复,并共享不同群体新的研究成果。

如果一个陆面模式可以由大多数陆面模式中使用的组件构建,并且以科学群体(Land Modeling Community)可以接受的方式构建,那么对陆面建模感兴趣但缺乏必要资源或专业知识的个人团体可以专注于新的方面,而不必过度重复过去的努力。此外,通过一个公用平台,本着“开放源代码”代码开发的精神,所有用户可以共享由各自提供的改进和精细化。

在1996年2月在美国国家大气研究中心气候系统模式(NCAR CSM)陆面工作组(LMWG)的研讨会上,Robert E. Dickinson提出:让一个更为广泛的科学家共同体为NCAR CSM提供一个陆面模式框架,并提出开发的软件的初始构想。他认为一个真正的通用陆面模式的目标不太可能在现有的机构中实现,但提供一个原型将是有用的,这个原型可以证明至少有一些群体是被激励的,并且有能力联合起来产生一个共享的模式。

戴永久通过数年的努力完全独立研发了中国科学院大气物理研究所陆面过程模式(IAP LSM,1994版,IAP94),融会贯通国际主流陆面过程模式(BATS、SiB2、SNTHERM等)和代码,具有杰出的建模能力和编程技术。IAP94基于多孔介质流体动力学和热力学原理,提出了陆面过程模式的统一框架,包括基本方程组、本构方程及定解条件,成功解决了陆面水分相态变化、冠层湍流、计算稳定性等模式研制中的难点问题,并在PILPS模式比较中体现了良好的性能,得到同行们的充分认可。

戴永久适逢其时于1997年8月加盟Robert E. Dickinson团队,推进项目的实施。1997年11月Robert E. Dickinson为主席的CSM LMWG确定戴永久作为项目首席(Lead Scientist),启动研制通用陆面模式原型(proto-type)模式,并建议模式命名“通用陆面模式(The Common Land Model, CLM)”及简写(CLM)。

最初的原型模式编程规范和建模侧重于评估三个陆面模式的最佳特征。它们是:用于Community气候模式(CCM3)和CCSM初始版本的NCAR LSM~\citep{bonan1996land},中国科学院大气物理研究所陆面模式(IAP94)~\citep{Dai-Zeng_97_IAP94},以及与CCM2一起使用的生物圈-大气交换方案(BATS)~\citep{dickinson1993biosphere}。在1998年2月LMWG会议上,戴永久汇报了审查这些代码的内容及编程规范的情况。这次会议正式确定:研制的原型模式不仅被视为NCAR CCSM陆面部分的下一代模式,而且被视为需要对其当前陆面模式进行广泛修订的其他美国小组使用的下一代候选模式。

为此,成立了一个专门的科学指导委员会,审查戴永久提出的设计规范,并促进进一步的发展。指导委员会成员的选择是为了在模式研制中提供科学指导和专业知识,他们来自美国大学 (Robert Dickinson、曾旭斌、Jay Famiglietti、Jon Foley)和美国政府实验室(Paul Dirmeyer、Paul Houser) 的科学家。委员会由CSM LMWG主席Robert Dickinson(亚利桑那大学)和Gordon Bonan (NCAR)担任,最初的成员是: Robert Dickinson、戴永久、曾旭斌(亚利桑那大学团队),Paul Dirmeyer(海洋-陆地-大气研究中心),Jay Famiglietti(德克萨斯大学奥斯汀分校),Jon Foley(威斯康星大学)和Paul Houser (GSFC)。Paul Dirmeyer和Paul Houser就如何以更好地满足其机构要求的方式改进规则提供了特别实质性的指导。曾旭斌负责群体间的协调和监督项目实施。

通过一年多的努力完成了原型模式研制。原型模式不仅集成了国际陆面研究的优秀成果,并在诸多方面做出重大改进和新的发展,特别,在水分相态变化、次网格参数化、复杂问题分解、计算稳定性、模式整体协调性等关键问题做出重大创新。1999年3月由Mike Bosilovich、Paul Dirmeyer和Paul Houser对CLM原型模式进行了检查。此后,开始了一个广泛的代码测试阶段。Keith Oleson, 戴永久, Adam Schlosser和Paul Houser在1999年6月的CCSM Workshop LMWG会议上介绍了离线测试的初步结果。

不同研究团队开展对原型模式全面评估测试,时间持续一年有余。评估团队有:戴永久、Robert Dickinson、曾旭斌和杨宗良组成的亚利桑那大学团队,Gordon Bonan和Keith Oleson的NCAR团队,Paul Dirmeyer和Adam Schlosser的COLA (Center for Ocean-Land-Atmosphere Studies)团队,Paul Houser和Mike Bosilovich的GSFC (Goddard Space Flight Center, NASA)团队,Scott Denning和Ian Baker的科罗拉多州立大学团队。1999年11月在COLA的CLM研讨会上每个团队汇报了独立的测试结果,充分展示了CLM原型模式优秀的性能。用于验证的站点数据包括所有PILPS站点(Cabauw、Valdai、Red-Arkansas河流域)~\citep{Henderson-Sellers_1993_PILPS}和其他站点数据(FIFE~\citep{Sellers88FIFE}、BOREAS~\citep{Sellers95BOREAS}、HAPEX-MOBILHY~\citep{Andre1986hapex}、ABRACOS~\citep{Gash96ABRACOS}、Sonoran Desert~\citep{Unland1996surface},以及全球土壤湿度项目(GSWP)数据~\citep{Dirmeyer1999global})。

在2000年6月召开的CCSM Workshop LMWG会议上,戴永久正式汇报了通用陆面模式与CCM3的初步耦合结果,表明CLM可以成功地与气候模式耦合。与NCAR LSM和观测结果的比较表明,径流的季节性、夏季偏冷的显著减少和雪深有了重大改善。但是,注意到与径流和反照率有关的一些缺陷,随后予以解决。杨宗良和Ian Baker汇报了CLM在模拟雪和土壤温度方面的改进。
戴永久具体实施和组织完成了CLM的研发、CSM的耦合、全面的性能评估、完整的技术报告和用户指南~\citep{Dai2001CoLM}。CLM由戴永久通过2003年BAMS封面论文向公众发布~\citep{dai2003common}。CLM与CCM的耦合评估由曾旭斌总结发表~\citep{zeng2002coupling}。戴永久参与了NCAR Community Atmosphere Model (CAM 3.0)技术报告编写~\citep{Collins2004CAM}。

在此基础上,NCAR团队改土地覆盖/土地利用的biome分类体系(BATS分类)为植被功能(PFT)分类体系,新增了全球陆面基础数据集,河道径流和生物地球化学的模式,以及代码架构。为了与CCSM中Community相匹配,名称改The Common Land Model为The Community Land Model,简写仍然为CLM。Community-CLM技术报告于2004年通过NCAR Tech Note~\citep{Oleson2004CLM}正式发布,提供免费自由下载的模式系统。Community-CLM与CCM耦合评估由Gordon Bonan总结发表~\citep{Bonan2002CLM}。


戴永久于2002年6月加盟北京师范大学,2003年6月全职回到北京,希望能在Common-CLM的基础上,发展我们自主可控的陆面模拟研究平台,得到了Robert Dickinson的鼓励和全面的建议,并建议中国版CLM继承The Common Land Model这一名称,为了区别NCAR主导的Community–CLM,改Common-CLM的简写为CoLM。



\subsection{通用陆面模式 2004 版本 (CoLM 2004)}
在CoLM原型版本基础上,在四个方面做了重大改进:

\begin{enumerate}[label={\arabic*)}]
    \item 陆面属性数据集

    全球 \ang{;;30}$\times$\ang{;;30} 分辨率的格点数据集,包括:
    \begin{enumerate}[label={\alph*)}]
        \item 全球高程数据集 (\url{http://webgis.wr.usgs.gov/globalgis/gtopo30/});
        \item USGS土地覆盖/土地利用数据,24个类型,数据集来源于1992年4月至1993年3月获得的1公里AVHRR数据 (\url{http://edc2.usgs.gov/glcc/});
        \item 土壤质地由FAO~\citep{GlobalSoilData2000}和STATSGO~\citep{Miller1998conterminous}合并数据,土壤剖面(0--30 cm)和(30--100 cm)两层,16类土壤质地,土壤水力参数(孔隙度、饱和水势、持结曲线斜率、饱和导水率)从查找表~\citep{cosby1984statistical}获取。
    \end{enumerate}
    
    \item 植被冠层双大叶模式~\citep{dai2004two}
    
    双大叶模式成功解决了植被阴叶与阳叶的光能吸收、光合-气孔导度、二氧化碳和水汽通量、叶面温度等本构方程的构建及数值适定算法等难题,为深入研究植被光合-蒸腾作用及其相关的生态与环境过程提供了关键理论与技术。
    
    \item 平板海洋模式

    为天气/气候模式应用提供一个选择,配置了简单海洋-海冰模式(平板海洋模式,Slab Ocean Model),即允许在一个简化的一维海洋-海冰模式之上运行完整的大气模式。基于输入的海温、海冰范围和厚度数据,应用CoLM常通量层Monin-Obukhov相似理论理论计算海-气动量、热量和水汽通量,基于海洋表面反照率计算向上太阳辐射通量,基于SST计算向上长波辐射。海冰的温度用厚度和热性能固定的“冰”层来表示,隐式求解热扩散方程预报温度。平板海洋模式仅适用于海洋动力学作用最小的情况下的海-气相互作用过程问题。CoLM-平板海洋模式的基础模型为NCAR CESM 平板海洋模式 (\url{https://www.cesm.ucar.edu/models/simple/slab-ocean-model})。
    
    \item 模式程序结构

    对模式程序做了全面改写和结构调整,并将代码规整到四个子目录:
    \begin{enumerate}[label={\alph*)}]
        \item \texttt{/mksrfdata/}:基础数据(\ang{;;30}$\times$\ang{;;30}分辨率)读入,及升尺度和属性参数转化函数(算法);
        \item \texttt{/mkinidata/}:初始状态变量读入及尺度转换;
        \item \texttt{/main/}:主程序;
        \item \texttt{/run/}:脚本(script)文件实施编译和运算。
    \end{enumerate}
    另外,构建了并行计算版本。
\end{enumerate}

CoLM2004模式源程序、基础数据集、科学技术报告和用户指南发布网站: \url{http://globalchange.bnu.edu.cn/research/models}。


\subsection{通用陆面模式 2014 版本 (CoLM 2014)}

\begin{enumerate}[label={\arabic*)}]
    \item 陆面属性数据集
    \begin{enumerate}[label={\alph*)}]
    \item 土地覆盖/土地利用数据。用美国国家雪/冰资料中心(Natinal Snow and Ice Data Center:\url{http://www.glims.org/RGI/}; \url{http://glims.colorado.edu/glacierdata/})专业中心数据产品替换USGS永久性积雪和冰盖(permanent snow and ice)数据。用(\url{http://www.wwfus.org/science/data.cfm})永久性湖泊(permanent lakes)和永久性湿地(permanent wetlands)专业部门制作的数据替换USGS永久性湖泊和永久性湿地数据。用全球城市面积数据(\url{http://sage.wisc.edu/urbanenvironment.html})专业部门制作的数据替换USGS城市(urban and built-up)数据。
    \item 中国和全球土壤属性数据集。起始于2005年,全面收集和整理了土壤剖面物理与化学属性数据、土壤分类图,并作精细质量控制,建立了逾3万个土壤剖面的土壤属性数据库。建立了与模式相匹配、参数最全、分辨率最高的中国和全球土壤属性数据集~\citep{shangguan2013china,shangguan2014global},中国和全球土壤水力学和热力学参数集。建立了多土壤属性转换函数集,模式土壤水力和热力参数取土壤属性转换函数集的计算中位值~\citep{dai2013development}。
    \item 全球植被叶面积指数数据。全球植被LAI是陆面和气候模拟非常重要的变量。NASA发布的LAI数据存在时空不连续性和不一致性问题,极大限制了其高分辨率资料的应用。发展了相互协调的滤波算法和订正方法,解决了原数据存在的问题,建立了可直接为模式所用的、高分辨率和长序列的全球LAI数据集~\citep{yuan2011reprocessing}。
    \end{enumerate}
    \item 湖泊模式

    在CLM 4.5~\citep{oleson2013technical}湖泊模式~\citep{subin2012improved}的基础上,我们在降水感热能量传递、湖水冻融、湖冰积雪、湍流通量、辐射传输和温度计算等方程作了实质性改进~\citep{戴永久2018通用陆面模式}。
    
    \item 河道径流模式(CaMa-Flood Version 3.6.2)

    耦合了由日本东京大学Dai Yamazaki研制的大尺度分布式汇流模型  (Catchment-based Macro-scale Floodplain, CaMa-Flood)~\citep{yamazaki2011physically}。CaMa-Flood对河道径流和洪泛过程有精细的描述,特别重要的是,研制了非常精细的全球高分辨率数据集。

    \item 模式程序结构

    在模式程序及结构作了全面的改进,更为好用、易用。
\end{enumerate}

模式源程序、基础数据集、科学技术报告和用户指南发布网站: \url{http://globalchange.bnu.edu.cn/research/models}。


\section[通用陆面模式 2024 版本(CoLM 2024)]{通用陆面模式 2024 版本(CoLM 2024) (Overview of CoLM 2024)}
当今用于数值天气/气候/地球系统模式的陆面过程模式研究需特别强调向多时空尺度、系统集成的方向发展,全球性与区域性、宏观与微观、生态系统过程等的结合,多源观测和数据同化相结合的方向发展;特别强调学科研究与国家需求、经济和社会可持续发展以及政策决策紧密结合,使陆面过程模式研究不断向深度和广度发展。一个有效的陆面模拟系统能够为我们识别气候变化与人类活动对环境变化的影响、探索水文气象灾害成因并且进行预报与预警、优化水资源配置、保护粮食安全和生态环境提供科学支撑。

我们旨在研发我国具备国际先进地位的新一代陆面过程模式。主要研究目标包括:建立包含人类活动和生态系统过程的高分辨率(100米级)全球陆面过程模式;建成全球分辨率为亚公里级且与模式相匹配的陆面基础数据集和验证资料数据集;开发适用于模式不同分辨率应用的尺度转换方法,实现与中国科学院地球系统模式CAS-ESM、中国气象局数值天气预报系统模式GRAPES、天气研究与预报模式气候版(CWRF)、国家气候中心气候系统模式BCC-CSM等的多尺度耦合应用。最为重要的目标是:创建涉及更广泛科学群体的科学研究范式,更直接地让非科学用户参与模式开发,促进知识协同生产。

以通用陆面模式CoLM 2014为基础,针对地表能量、水文、生物地球化学循环和人为扰动等过程对其改进和完善,研发新一代高分辨率全球陆面过程模式。主要研究内容包括:研制包含人类活动和生态系统过程的高分辨率全球陆面过程模式;基于理论研究在精细尺度下创新或改进陆面模式;研制与全球陆面过程模式相匹配的高分辨率数据集;发展陆面过程模式和基础数据的尺度转换方法,提高陆面过程模式的时空分辨率和模拟准确性;实现陆面过程模式与天气/气候/地球系统模式的耦合应用。

CoLM 2024主要新内容(New Features of CoLM 2024):

\begin{enumerate}[label={\arabic*)}]
    \item 网格结构、PFT和PC次网格

    除经纬度网格外,CoLM2024版引入了两种新的网格结构:非结构网格和流域单元网格。非结构网格可以基于多个水平陆面分布特征,自动识别不同区域所需的网格分辨率,生成包括三角形网格和六边形网格在内的无规则拓扑关系网格结构。流域单元网格重点考虑了地形对陆面过程的影响,建立了流域单元-高度带单元的多级网格结构。
    
    \item 基础数据集
    \item 辐射(三维冠层辐射、SNICAR …)
    \item 湍流通量
    \item 冠层降水截留和植被水力学

    冠层截流过程和植被水力过程是生态水文过程中决定陆气水交换的关键元素,CoLM2024版本引入了多冠层截留方案,并增加了以达西定律为基础的植被水力过程。新引入的冠层截留参数化方案考虑了不同的物理过程,为探明多因子共同作用下冠层截留的演化规律、驱动机制和发展趋势提供了有效手段,有助于提升冠层阻拦降水分配的精度,改变水分在大气、土壤和径流之间的比例。植被水力过程描述了大气-植被-土壤连通体的物理根本,机理地刻画了植被水分胁迫随植物水势的变化关系,替代了过去土壤水势决定植物水分胁迫的经验关系,改进了陆气水分交换对环境变化的响应模拟。

    \item 植被冠层温度
    
    \item 水文过程(CaMa-Flood new version;可变饱和土壤水运动、地下水、动力学模式、河网径流、流域单元网格的侧向水分运动 …)

    CoLM2024版对水文过程的模拟方案进行了全面的更新。在垂直方向,发展了新的可变饱和流土壤水运动算法,它可将地表积水、非饱和带及饱和带中的水分运动和交换进行统一求解,具有更高的数值精度和更好的算法稳定性。在水平方向,考虑了地形对水分重分配的作用,对三种网格结构分别更新或引入了侧向流模拟方案:1)对经纬度网格,更新CaMa-Flood至最新版本,并完成了CaMa-Flood的模块化,实现了CaMa-Flood与CoLM的双向耦合;2)对流域单元网格,发展了基于浅水波方程的坡面流和河道径流计算方法,建立了多尺度的地下水侧向流计算方案;3)对非结构网格,同时支持CaMa-Flood的汇流方案以及对流域单元网格中所发展的坡面流和河道径流计算方法。对水文过程的更新使得CoLM2024版可用于多种尺度的陆面过程的模拟,尤其是与人类活动密切相关的公里尺度。
    
    \item 生物地球化学过程

    生物地球化学过程是CoLM模型为了拓展其在气候变化生态学领域的应用而引入的关键增量。它刻画气候变化研究的最关键问题之一,即陆地生态系统碳隔离的气候变化响应。它包括植被自养呼吸,植被碳分配,植被物候,植被死亡,植被氮吸收,植被氮重利用,植被死亡,土壤植被氮竞争,土壤碳氮垂直混合过程,土壤无机氮过程和土壤有机碳氮分解等过程。同时CoLM生物地球化学模块引入了基于矩阵运算的半解析加速预热方案,可以提高生物地球化学循环过程预热效率700\%。

    \item 城市
    \item 作物 

    研制全球格点作物模式GPAM1,在CoLM里实现了对水稻、小麦、玉米、大豆等粮食作物生长发育关键过程和产量的模拟,以及作物对环境气候变化和农田管理的响应及生物、物理、化学动态反馈模拟。
    
    \item 水库

    水库建设和管理是人类最重要的用水活动之一。CoLM2024版本以Cama-Flood为基础研制了水库模块,估算了与高精度河网数据相匹配的水库特征数据,引入了多种水库调度规则参数化方案,采用了包含水库影响的河道汇流参数化方案。水库模块的加入有效提升了模式对人类活动影响下河道径流和洪水风险的模拟精度。
    
    \item 土地利用土地覆盖变化 
    \item 火灾

    使用Li全球火灾参数化方案代替GlobFIRM方案,模拟火发生、火蔓延、及火影响过程。Li的基本方程和框架可以解决GlobFIRM模拟的全球燃烧面积不足观测一半的大问题,并实现火灾季节变化模拟;实现了人类活动影响火灾的模拟;实现了基于卫星和场观测资料对关键参数的率定;实现了火灾引起的痕量气体和气溶胶排放模拟。

    \item 臭氧生态胁迫

    基于收集整理的4210组实验数建立的臭氧生态胁迫新参数化方案,能成功再现观测到的各种植被类型的光合和气孔导度对臭氧污染的线性和非线性响应,并间接地使全球模式再现植物应对臭氧胁迫的适应性变化和植物品种转化模拟,各方面性能远优于已有方案。
    
    \item 尺度转换

    大气强迫降尺度是CoLM模型为实现高分辨率精细模拟的新增模块。它通过高分辨率高程数据(及衍生的地形因子)对输入大气强迫数据在次网格上进行修正,从而在较低计算代价下刻画大气强迫的精细特征。CoLM2024版本降尺度模块主要分为三个部分:1)基于地形差对气温、湿度、气压、长波辐射调整方案;2)基于地形因子的风速与短波调整参数化方案、3)基于自动化机器学习(AutoML)的降水调整方案,并通过MPI实现了机器学习模型与CoLM的耦合。降尺度模块的引入有效提升了模式的高分辨率模拟精度。
    
    \item 模式程序结构
\end{enumerate}


\section{致谢 (Acknowledgments)}
作者感谢CoLM工作组外部成员自2004年以来对CoLM的发展做出的重大贡献。他们是:

曾庆存、Robert E. Dickinson、陈海山、梁信忠、曾旭斌、周黎明、牛国跃, …

纪多颖、张倩、朱司光、张香香、…


\part{模式构架与基础数据}{Model Architecture and Fundamental Datasets}
%\epart{Model Architecture and Fundamental Datasets}
\chapter{模式构架}\label{模式构架}
%\addcontentsline{toc}{chapter}{模式结构}
\section{网格构建}\label{网格构建}
\begin{mymdframed}{代码}
本节对应的网格类型可在\texttt{/include/define.h}中进行选择。
\end{mymdframed}

通用陆面模式(CoLM,the Common Land Model)首先对陆地表面进行剖分,将模拟区域按一定规则划分成单元(Element),模拟分辨率决定了单元平均面积的大小。CoLM中包含三种单元划分规则:1)经纬度网格(图~\ref{fig:经纬度网格});2)非结构网格(三角形/六边形)(图~\ref{fig:六边形单元网格},~\ref{fig:三角形单元网格});3)流域单元(Catchment)(图~\ref{fig:流域单元网格})。除默认的三种网格外,CoLM实际可使用任意形状单元的网格。

{
\begin{figure}[htbp]
\centering
\includegraphics[width=0.8\textwidth]{Figures/模式构架/网格-格点.jpg}
\caption{CoLM中可使用的网格(1):经纬度单元}
\label{fig:经纬度网格}
\end{figure}
}
{
\begin{figure}[htbp]
\centering
\includegraphics[width=0.8\textwidth]{Figures/模式构架/网格-六边形.jpg}
\caption{CoLM中可使用的网格(2):六边形单元}
\label{fig:六边形单元网格}
\end{figure}
}
 {
\begin{figure}[htbp]
\centering
\includegraphics[width=0.8\textwidth]{Figures/模式构架/网格-三角形.jpg}
\caption{CoLM中可使用的网格(3):三角形单元}
\label{fig:三角形单元网格}
\end{figure}
}
 {
\begin{figure}[htbp]
\centering
\includegraphics[width=0.8\textwidth]{Figures/模式构架/网格-流域.jpg}
\caption{CoLM中可使用的网格(4):流域单元网格}
\label{fig:流域单元网格}
\end{figure}
}

\subsection{经纬度网格}\label{经纬度网格}
经纬度网格通过规定经度和纬度分割线的位置来建立,单元的边界由东西两段经线和南北两段纬线组成。经纬度网格的分辨率通常使用经纬度分割线的间距来表达,例如,分辨率为0.5\textdegree 表示相邻两条纬度分割线和经度分割线的距离均为0.5\textdegree。经度分割线和纬度分割线分别可以是不等间距的,当间距不等时,需从外部数据读入分割线的位置来定义模拟区域的经纬度网格。

\subsection{非结构网格}\label{非结构网格}
在原有的经纬度网格基础上,CoLM 新开发了一个非结构化网格构建工具,它可以基于多个水平分布特征,自动识别不同区域所需的网格分辨率,生成包括三角形网格和多边形网格(以六边形网格为主)在内的无规则拓扑关系网格。具体而言,非结构网格的分辨率取决于所指定的一个或者多个目标的分布特征(例如高程、坡度、土地利用、植被类型等)。该工具可在目标变化梯度较大的地区采用高分辨率,在变化梯度较小的地区采用低分辨率。因此基于非结构化网格的多分辨率模拟保留了全局模型的整体结构,同时支持局部区域的高分辨率模拟。

在非结构网格的构建过程中,该工具首先基于 Delaunay 三角网等值线生成算法构建初始网格,即将经纬度网格数据转化插值,生成铺盖整个球面的三角形网格数据;接着根据所选取的目标进行一次或者多次细化与网格结构调整;再依次连接具有相同顶点的五至七个三角形重心生成多边形网格,最后输出全球区域的三角形或多边形可变分辨率网格。总之,非结构网格具有灵活性强、节点和单元的分布可控性好、能较好地控制网格的大小和节点的密度等优点。模式运行流程如图~\ref{fig:非结构化网格CoLM总体运行流程图} 所示:
 {
\begin{figure}[htbp]
\centering
\includegraphics[width=0.8\textwidth]{Figures/模式构架/非结构化网格CoLM总体运行流程图.png}
\caption{非结构化网格CoLM总体运行流程图}
\label{fig:非结构化网格CoLM总体运行流程图}
\end{figure}
}

该工具基于非结构化一致性三角-六边形构建算法 \citep{fatichi2020soil,walko2008ocean,walko_direct_2011},以准均匀的全局三角形(Delaunay)的网格化方法为理论依据。其中非结构网格构造流程如图~\ref{fig:非结构化网格生成流程图} 所示,具体介绍如下:
{
\begin{figure}[htbp]
\centering
\includegraphics[width=\textwidth]{Figures/模式构架/非结构化网格生成流程图.png}
\caption{非结构化网格生成流程图}
\label{fig:非结构化网格生成流程图}
\end{figure}
}

{
\begin{table}[htbp]
\centering
\caption{细化阈值文件}
\label{tab:细化阈值文件}
\begin{tabular}{@{}ll@{}}
\toprule
变量名            & 具体描述           \\\midrule
num\_landtypes & 网格包含的土地类型数量    \\
f\_mainland    & 网格主导土地类型       \\
max\_iter      & 三角形网格最大细化迭代次数  \\
lai            & 叶面积指数          \\
slope          & 坡度             \\
k\_s           & 饱和导水率          \\
k\_solids      & 土壤固体导热系数       \\
tkdry          & 土壤导热系数(干燥)     \\
tksatf         & 土壤导热系数(冻结、饱和)  \\
tksatu         & 土壤导热系数(未冻结、饱和) \\\bottomrule
\end{tabular}
\end{table}
}

{
\begin{figure}[htbp]
\centering
\includegraphics[width=\textwidth]{Figures/模式构架/三角形网格简易细化.png}
\caption{三角形网格简易细化}
\label{fig:三角形网格简易细化}
\end{figure}
}

\begin{enumerate}
\item 构建初始网格

在此过程中,首先将一个正二十面体投射到地球表面,它的12个顶点中有两个位于南北两极,其余的位于$\pm\arctan(1/2)$ 纬度。正二十面体网格的二维投影基于0\textdegree 经线所在经线圈与与0\textdegree 维度所在纬线圈轴对称。这有利于在计算包含关系时简化运算步骤。同时每个区域被划分为面积基本相同大小的三角形分块,以满足非结构网格初步细化和提升并行化计算效率的需要。

\item 初步细化

在此步骤中,网格生成工具能够将每个球面三角形面细分为NXP $\times$ NXP更小的三角形,其中NXP可设置为任意正整数,代表所选的网格分辨率。NXP定义为两个相邻多边形网格的水平网格间距(网格单元面积的平方根),网格的分辨率大约等于7200公里/NXP。因此,如果试图在网格上获得100公里的水平网格间距,应将NXP设置为“72”。这种网格配置采用基于二十面体改进的六边形或三角形网格单元的非结构化网格,从而避免了传统经纬度网格的南北极奇点。该网格配置为局部网格细化提供了一种完全无缝的自然适应性,不需要额外特殊的网格嵌套算法。

\item 弹簧动态调整

考虑到地球球面对三角形网格实际边长的影响,通过初步细化生成的三角形并不等同于等边三角形。\citet{tomita2002optimization}描述了一种称为“Spring Dynamics”的计算方法,用于在准均匀的全局网格上调整三角形网格单元的形状和大小。该方法代表了一种物理模拟,其中网格中的每个边都像弹簧一样在其顶点间施加吸引力或相反的力,该弹力取决于其自身的长度、平衡长度和弹簧系数。该研究表明,将平衡长度设置为使弹簧松弛在数值上稳定的最大可能值,即接近整个网格的平均边长,会产生最小的网格单元尺寸的空间变化,并且调整过程能显著提高模型的数值精度。

\item 计算包含关系

CoLM非结构网格的空间细分基于三角形网格与阈值数据(部分地表数据,如表~\ref{tab:细化阈值文件})进行。一般来说,三角形网格的分辨率远低于阈值数据集。这些高分辨率网格(以下简称像素)构成了非结构化网格。在进行网格细化之前,需要计算两者间的包含关系,并得到每个网格对应像素的索引,以便识别像素对非结构网格的归属。最后,根据这些像素的信息,用户可以了解特定网格内的空间异质性,并决定是否需要对其进行细化。这个过程将在细化过程中的每次迭代中执行。

\item 阈值文件计算


\citet{walko_direct_2011} 中的细化面积和程度是通过分配一系列点的经纬度坐标来确定的(例如,(lat = 30.1, lon = 120.5); \dots)加上影响半径(例如,(R = 1km); \dots)。与之不同,CoLM的网格细化工具能够根据各种指标确定精细化的区域。关于这些指标的详细信息可以在表~\ref{tab:细化阈值文件} 中找到。该工具允许对一个或多个网格进行细化,并约束到用户定义的阈值。换言之,CoLM 可以根据任意数量的重要特征对网格进行细化,例如 $LAI$ 的标准差,网格内土地利用类型数量等。图~\ref{fig:多阈值非结构网格细化} 展示了基于多种不同地表特征进行细化得到的全球陆面多边形网格。

{
\begin{figure}[htbp]
\centering
\includegraphics[width=\textwidth]{Figures/模式构架/多阈值非结构网格细化.png}
\caption{多阈值非结构网格细化示例}
\label{fig:多阈值非结构网格细化}
\end{figure}
}

\item 阈值细化

如图~\ref{fig:非结构化网格生成流程图} 所示,本步骤主要实现三角形网格的阈值细化。利用生成的初始三角形网格包含关系,可计算三角形网格下各阈值数组,再根据阈值标记网格是否需要细化。如图~\ref{fig:三角形网格简易细化} 所示,通过将三角形网格“分成四格”,即可提高三角形网格的分辨率。然后,将新生成的网格信息添加到原始数组中,并标记被细化的网格。在计算新生成网格的包含关系时,只需要遍历被细化三角形最小外围矩形内的经纬度网格,可以大大减少计算时间。当迭代次数超过阈值或所有三角形网格满足阈值时,阈值细化终止,程序将输出最终生成的带有包含信息的三角形网格信息数据(见表~\ref{tab:细化阈值文件})。图~\ref{fig:根据LAI对非结构网格进行多重细化} 展示了基于叶面积指数特征进行多层级细化所生成的多边形网格结构。

{
\begin{figure}[htbp]
\centering
\includegraphics[width=0.8\textwidth]{Figures/模式构架/根据LAI对非结构网格进行多重细化.png}
\caption{根据LAI对非结构网格进行多重细化}
\label{fig:根据LAI对非结构网格进行多重细化}
\end{figure}
}

\item 清除悬挂点\\
该步骤将网格细化生成的附着在原始三角形边缘上的点称为悬挂点。为了保证每个三角形只与三个相邻三角形共享边,我们针对两种情况设置了不同的悬挂点消除方案。如图~\ref{fig:非结构化网格生成流程图} 顶部所示,对于细化边缘较平坦的区域,可以连接悬挂点与其相邻三角形的顶点。而对于一些凹陷区域,如图~\ref{fig:非结构化网格生成流程图} 底部所示,由于多个悬挂点位置相邻,直接连接悬挂点与其相邻三角形的顶点会使顶点被相邻的八条边共享,导致生成多边形网格时产生八边形,因此我们采用了特殊的细化方法来避免这一问题。此外,在消除悬挂点时,如果两个细化区域的边界相遇,也容易产生八边形。综上所述,在消除悬挂点前,需要对初步细化后的网格进行如下预处理:
\begin{enumerate}
\item 遍历所有未细化的三角网格。当其相邻的3个三角形网格中有1个以上被细化时,该三角形网格需要被细化。
\item 在三角形网格中找到并记录凹陷区域(即图~\ref{fig:非结构化网格生成流程图} 底部细化区域)。当两个凹陷区域有一个公共三角形网格时,细化组成它们的三个三角形网格
\item 循环遍历所有三角形网格顶点,计算剔除悬挂点后该点将添加的邻边数,当增加后其邻边数大于七时,细化其所有邻接三角形网格。在预处理过程中,三个进程依次进入迭代。当三个过程同时没有进行网格细化时,视为预处理完成。预处理完成后,可以进行消除悬挂点操作(即图~\ref{fig:非结构化网格生成流程图}),并得到可用于构建多边形网格的三角形网格。
\end{enumerate}
\item 网格调整和多边形网格生成\\
在消除悬挂点的操作过程中,消除凹陷区域会生成许多钝角三角形。而多边形网格由三角形网格重心连接而成,因此过多的钝角三角形会导致生成的部分多边形与正多边形偏差较大,进而降低网格的各向同性,最终影响网格在CoLM等水文模型中的应用。为了在不过度影响原有网格结构的情况下对网格进行校正,本文根据钝角三角形角度与边长计算经纬度位移分量,并对其进行迭代调整。当不存在钝角三角形或迭代次数超过阈值时,调整结束。在每次迭代过程中,通过三角形网格重心构造多边形网格,并计算三角形网格与正三角形、多边形网格与正多边形角度的标准差,这是衡量非结构网格自身各向同性的标准之一。在每次迭代过程中,通过三角网格的重心构造多边形网格,并计算三角形网格与正三角形、多边形网格与正多边形的标准差,这是衡量网格自身物理结构质量的标准之一。迭代完成后,通过三角网格重心所构造处的最后一个多边形网格即为最终结果。
\item 输出网格文件\\
在上述步骤完成之后,可输出两种类型的网格。一是最后一次迭代生成的三角形网格,二是根据前者重心连接构造的多边形网格。在 MPI 版本中,无需将地表、大气等初始数据的存储方式由经纬度网格转换为非结构化网格,就可以实现 CoLM 中地表水文过程在非结构化网格上的模拟。
\end{enumerate}


\subsection{流域单元网格}\label{流域单元网格}
流域单元为面积大致相同的集水区域(图~\ref{fig:流域单元网格})。流域单元考虑了网格单元之间的水文关联以及单元内部的水文结构,基于流域单元网格,可建立对产流和汇流动力学过程的模拟方案。此外,流域单元及其次级单元(高度带单元)的划分主要依赖高程数据,在陆面过程受地形影响较为显著的区域,采用流域单元网格可显著减少网格内陆表变量的异质性,提高模式模拟的精度。

CoLM中的流域单元网格为三级结构:第一级为流域单元,第二级为高度带单元,第三级为次网格单元(图~\ref{fig:流域单元示意图})。其中,流域单元和高度带单元使用水文学数据生成,次网格单元的生成方法与其他网格中的次网格单元相同(见~\ref{次网格}~节)。

{
\begin{figure}[htbp]
\centering
\includegraphics[width=\textwidth]{Figures/模式构架/流域单元示意图.jpg}
\caption{CoLM中的流域单元网格划分示意图}
\label{fig:流域单元示意图}
\end{figure}
}

流域单元(第一级)为面积大致相同的集水区域(图~\ref{fig:流域单元示意图}(2))。流域单元网格的分辨率表示为一个单元的面积阈值(即河道集水区域的面积阈值,记为Cat-size)。流域单元的划分使用高分辨率的“上游集水面积”(UPA)和“水流方向”格点数据(其中的格点以下称为像素点),分为两个步骤。
第一步,提取研究区域的河道像素点并将其分段。此处的河道像素点定义为上游集水面积大于或等于Cat-size的像素点,未必为真实的河道。提取完成后,根据以下三个规则从下游至上游将河道分段:1.一段河道内没有河道分支;2.局部集水面积(定义为该河段的总集水面积减去上游河道的集水面积)不超过Cat-size;3.河道长度小于Cat-size的平方根。其中,规则1的目的是排除汇流点在流域单元内部的情况,以更合理地进行汇流过程的模拟;规则3的目的是避免流域单元较为狭长,以减少单元内部陆表变量的变化。
第二步,根据河道分段将研究区域划分为流域单元。一个流域单元定义为一段河道的局部集水区域。对每个像素点,可根据水流方向数据向下找到其所在的流域单元。

高度带单元(第二级)为主要依赖高程进行划分的流域单元内部子区域(图~\ref{fig:流域单元示意图}(3))。因流域面积较大时,一段河道的上下游会有落差,CoLM中不直接使用高程数据对流域单元进行划分,而是使用像素点的排水高度。排水高度定义为一个像素点和它流入的河道点的高度差(Height above nearest drainage, HAND),可根据高程数据和像素点与河道的汇流关系计算得出。此外,为了基于高度带单元进行产流动力学过程的模拟,CoLM对高度带单元的划分进行了进一步的约束,划分后的高度带单元需满足:1)连通性;2)每个高度带单元都有唯一的下游单元。对高度带单元进行划分的算法见图~\ref{fig:高度带单元算法流程}。

{
\begin{figure}[htbp]
\centering
\includegraphics[width=0.8\textwidth]{Figures/模式构架/高度带单元划分算法.jpg}
\caption{高度带单元划分算法流程图}
\label{fig:高度带单元算法流程}
\end{figure}
}

图~\ref{fig:高度带单元约束示意}~显示了对高度带单元的划分进行上述约束的作用。仅使用HAND数据进行高度带单元的划分时,会产生空间上不连通的单元(图~\ref{fig:高度带单元约束示意}b~中的6号和7号单元),一个高度带单元的不同部分在地理位置上可能相隔较远,增加连通性约束后,可避免这种情形。

{
\begin{figure}[htbp]
\centering
\includegraphics[width=\textwidth]{Figures/模式构架/高度带单元的约束示意图.jpg}
\caption[对高度带单元进行约束的作用]{对高度带单元进行约束的作用:a)排水高度(HAND);b)仅根据排水高度划分高度带单元;c)具有连通性的高度带单元;d)满足连通性和具有唯一下游单元的高度带单元。上图中的箭头表示水流的方向}
\label{fig:高度带单元约束示意}
\end{figure}
}

在流域单元网格中,每个流域单元关联了一段河道。目前未考虑河道向下游分叉的情况,因此,除入海口和内陆洼地单元外,每个流域单元都有唯一的下游单元,整个模拟区域的流域单元在水文上都是连通的(见图~\ref{fig:湖泊划分}~左)。

自然连通的湖泊设置为独立的单元,不受面积阈值的限制。生成流域单元网格时,首先使用HydroLAKES数据\citep{messager2016nc}将区域内的湖泊标识出来,再将模拟区域内的其余部分划分为流域单元。为了避免单个湖泊的面积过大,使用CVT算法\citep{du1999siam}对湖泊进行了进一步的划分(见图~\ref{fig:湖泊划分}~右)。
{
\begin{figure}[htbp]
\centering
\includegraphics[width=\textwidth]{Figures/模式构架/湖泊划分.jpg}
\caption[流域单元网格中的湖泊]{流域单元网格中的湖泊。左图为模拟区域(淮河流域)的范围、高程以及河道和湖泊网络。右图表示采用CVT算法将湖泊区域进一步划分为更小的单元}
\label{fig:湖泊划分}
\end{figure}
}



\section{次网格结构}\label{次网格}
\begin{mymdframed}{代码}
本节对应的次网格类型可在\texttt{/include/define.h}中进行选择。
\end{mymdframed}
\subsection{次网格结构概述}
考虑到地表下垫面覆盖的异质性,CoLM对模式网格单元进一步划分次网格。
次网格是CoLM计算模拟的基本结构单元,通常称为patch(斑块)。
Patch是通过使用高分辨率的精细化网格数据,根据网格单元内部地表覆盖类型、植被功能类型、叶面积指数、土壤属性和地形等分布特征,按一定方式进行划分并聚合而来。Patch是用于模拟网格单元内部不同下垫面覆盖过程(对地表异质性考虑),同时也起到了减少计算量作用。


由patch可以组成按经纬度网格、流域单元网格(章节~\ref{流域单元网格})及非结构(章节~\ref{非结构网格})网格。
在patch尺度计算得到的通量按其所在网格覆盖比例进行面积加权平均,作为地表网格通量模拟结果。
地表状态或预报变量一般情况下亦是如此,在patch尺度计算得到的结果按照面积加权平均后作为模式结果输出。

{
\begin{figure}[htbp]
\centering
\includegraphics[width=\textwidth]{Figures/模式构架/CoLM次网格结构示意图-v2.jpg}
\caption[CoLM次网格结构示意图]{CoLM次网格结构示意图。次网格植被(Sub-grid vegetation)结构提供三种可选方式:LCT, PFT以及PC;当作物模式打开时,每种作物当成一种独立的PFT (CFT)进行模拟;当城市模式打开时,次网格城市(Sub-grid urban)结构提供两种可选方式:NCAR 3种城市分类和LCZ 10种分类}
\label{fig:次网格结构示意图}
\end{figure}
}


CoLM非海洋patch从大类上分为五类:植被(含裸土)、城市、湿地、冰川和水体,如图~\ref{fig:次网格结构示意图} 所示。在模式中的编号依次为0--4。其中植被和城市patch可根据不同类型进一步细分。

\subsection{植被次网格结构}\label{sec:植被次网格}
植被patch可以分为自然植被和作物。自然植被patch采用三种次网格植被结构进行表征(图~\ref{fig:植被次网格方案}):1) 地表覆盖类型---LCT (Land Cover Type);2) 植被功能型---PFT (Plant Function Type)和3) 植物群落---PC (Plant Community)。以上三种方式,模式运行时只能选择其中一种。

{
\begin{figure}[htbp]
\centering
\includegraphics[width=0.55\textwidth]{Figures/模式构架/植被次网格方案示意图_v4.jpg}
\caption[CoLM植被次网格方案示意图]{CoLM植被次网格方案示意图。图中a (LCT方案)、b (PFT方案)中虚线框蒙版表示其包含的植被结构水平均一;图c (PC方案)与图a的差别在于缺少虚线框所示蒙版,即对次网格中的植被结构进行显式表达}
\label{fig:植被次网格方案}
\end{figure}
}

LCT方案为CoLM2014版原有方案(图~\ref{fig:植被次网格方案}a所示),即将某一地表覆盖类型可能包含的多种功能型植被当成混合(水平均一)植被进行模拟,其设置的相关参数可视为等效参数,图~\ref{fig:植被次网格方案}a中虚线框所示蒙版表示其中所包含植被结构水平均一。例如热带大草原,虽然可能包含树和草,但仍视为一种植被,因此相应的植被参数只有一套。

PFT是目前陆面模式(如CLM和JULES)常采用的次网格表征方式,也是本版本CoLM新添加的方式(图~\ref{fig:植被次网格方案}b所示)。
PFT方案是将每一个细网格地表覆盖类型进行拆解,得到其PFT的组成种类和各自面积占比,并将其聚合到模式网格中。同样,每种PFT植被结构也假设为水平均一,在图~\ref{fig:植被次网格方案}b中用虚线蒙版表示。本版本CoLM PFT方案类似于CLM PFT方案,即模式格点中的所有PFT作为1个植被patch,共享土壤水热等环境,但各PFT的辐射和通量等过程计算相对独立。

{
\begin{figure}[htbp]
\centering
\includegraphics[width=0.95\textwidth]{Figures/模式构架/植物群落示意图.png}
\caption[CoLM植物群落(PC)次网格概念示意图]{CoLM植物群落(PC)次网格概念示意图。同一植物群落中的PFT环境共享,资源竞争。包括辐射、风速、水热及营养元素等}
\label{fig:植物群落示意图}
\end{figure}
}

PC方案保留LCT方案中模拟对象,即地表覆盖类型,同时还进一步对LCT中的地表覆盖植被类型进行PFT细分(图~\ref{fig:植被次网格方案}c所示)。
不同于LCT方案把某一类型地表覆盖的所有植被视为混合植被,PC方案对某一地表覆盖类型所组成的PFT进行显式表达计算,所有PFT在同一次网格patch中共享辐射、风、水热及营养元素等环境,同时辐射和通量计算等过程PFT之间相互影响、同时求解,即考虑了PFT之间的共存与竞争(如图~\ref{fig:植物群落示意图} 所示)。这一方案类似于植物群落的概念,故命名为PC (Plant Community)方案。

LCT方案所依赖的地表覆盖类型数据可以直接由USGS (章节~\ref{USGS地表覆盖数据})或者MODIS-IGBP地表覆盖数据获取(章节~\ref{IGBP地表覆盖数据})。PFT和PC方案所需要的植被结构及属性数据由MODIS-IGBP地表覆盖数据加以其他辅助数据制作而成(章节~\ref{PFTPC数据及其依赖数据})。

对于作物patch,当作物模式未打开时,作物被当成一种特殊的自然植被进行模拟;当打开作物模式时,每个模式格点根据包含的作物分类及组成比例(外部数据读取)分别建立相互独立的patch进行模拟计算,方式类似PFT方案,但不同之处在于每种作物的土壤计算保持独立,不共享土壤水热等环境。

\subsection{城市次网格结构}
\begin{mymdframed}{代码}
本节对应的城市次网格类型可在\texttt{MOD\_Namelist.F90}中进行选择。
\end{mymdframed}

城市patch与作物类似,当城市模式未打开时,城市被当成一种地表覆盖进行模拟,即薄板城市模型(Slab Urban);
当城市模式打开时,每个模式格点根据包含的城市类型和组成比例(根据几何参数)分别建立相互独立的patch进行模拟计算。

{
\begin{figure}[htbp]
\centering
\includegraphics[width=0.95\textwidth]{Figures/模式构架/CoLM城市次网格示意图.jpg}
\caption[CoLM城市模式次网格结构示意图]{CoLM城市模式次网格结构示意图}
\label{fig:城市次网格}
\end{figure}
}

目前城市分类提供两种方式(图~\ref{fig:城市次网格}):
\begin{enumerate}
    \item 根据城市密度分为高建筑-TB、高密度-HD和中密度-MD 3类(基于NCAR CLMU城市模式);
    \item 根据城市局地气候区(LCZ,Local Climate Zone)分为10类。
\end{enumerate}
每种城市类型由屋顶、(地面)不透水面、透水面、阴/阳面墙、植被和水体组成。
城市分类及其相关属性数据主要从外部文件读取(章节~\ref{城市数据})。

\section{植被、土壤和积雪的垂直分层}\label{土壤和积雪的垂直分层}

{
\begin{figure}[htbp]
\centering
\includegraphics[width=0.9\textwidth]{Figures/模式构架/CoLM模式概念图.jpg}
\caption[CoLM植被、土壤和积雪垂直分层示意图]{CoLM植被、土壤和积雪垂直分层示意图}
\label{fig:CoLM垂直分层}
\end{figure}
}

CoLM植被、土壤和积雪的垂直分层如图~\ref{fig:CoLM垂直分层} 所示。

CoLM可根据次网格类型(章节~\ref{sec:植被次网格})将植被按照单层或最多三层进行模拟。对于单层植被(LCT和PFT植被次网格),采用双大叶模型进行模拟\citep{dai2004two};对于多层植被(PC植被次网格),每层植被的各PFT采用三维植被模型(章节~\ref{三维植被辐射传输模型} 和~\ref{三维植被湍流}),并结合双大叶模型进行模拟。

CoLM默认将土壤分为10层,每一层土壤的中心深度$z_i$~(m)定义为:
\begin{equation}
z_{i} = f_{s}\left\{ \exp{\left\lbrack 0.5(i - 0.5) \right\rbrack} - 1 \right\}
\end{equation}
其中尺度因子$f_s=0.025$. 因为土壤水热梯度在接近土壤与大气的交界面时往往较大,采用指数定义可以保证土壤接近表面时得到更细的分层。每一层土壤的厚度$\Delta z_i$~(m)计算为:
\begin{equation}
\Delta z_{i}=\left\{\begin{array}{ll}0.5\left(z_{1}+z_{2}\right) & i=1 \\
0.5\left(z_{i+1}-z_{i-1}\right) & i=2, \ldots, 9 \\ 
z_{10}-z_{9} & i=10\end{array}\right.
\end{equation}
相邻两层土壤交界处的深度$z_{h,i}$ (m)可计算为:
\begin{equation}
z_{h, i}=\left\{\begin{array}{ll}0.5\left(z_{i}+z_{i+1}\right) & i=1, \ldots, 9 \\
z_{10}+0.5 \Delta z_{10} & i=10\end{array}\right.
\end{equation}
CoLM中每层土壤中心的深度及每层下边界的深度见表~\ref{table:土壤分层}。

\begin{table}[b]
\caption{CoLM中的土壤分层(单位:米)} \label{table:土壤分层}
\centering \renewcommand{\arraystretch}{1.2} \footnotesize
\begin{tabular}{ccccccccccc}
\toprule
层 & 1 & 2 & 3 & 4 & 5 & 6 & 7 & 8 & 9 & 10 \\
\midrule
中心位置 & 0.0071 & 0.0279 & 0.0623 & 0.1189 & 0.2122 & 0.3661 & 0.6198 & 1.0380 & 1.7276 & 2.8646 \\
下边界位置 & 0.0175 & 0.0451 & 0.0906 & 0.1655 & 0.2891 & 0.4929 & 0.8289 & 1.3828 & 2.2961 & 3.4331 \\
\bottomrule
\end{tabular} 
\end{table}


雪盖位于土壤之上,可根据其厚度划分为至多5层。为与土壤层编号一致,
这里与土壤表层相邻的雪层记为第0层,逐渐向上依次记为第 -1 层直至至多为第 -4 层。
记$snl$为划分的雪层总层数的相反数,则最上层雪层即为第$snl+1$层。记雪层的厚度为$z_{sno}$,当$z_{sno}<0.01$时,这时积雪较少,不单独划分为层,$snl=0$,当$z_{sno}\geqslant 0.01$时,积雪分层的方案见图~\ref{fig:积雪分层}。

{
\begin{figure}[htbp]
\centering
\includegraphics[width=0.8\textwidth]{Figures/模式构架/积雪分层.jpg}
\caption[CoLM积雪分层方案]{CoLM积雪分层方案。当$z_{sno}<0.01$时不形成积雪层}
\label{fig:积雪分层}
\end{figure}
}
    
$z_{h,0}=0$为雪盖底层与土壤表层交界处的高度,$\Delta z_{i}$为第$i$层积雪的厚度。将交界面以上的高度定义为负值,则每一层雪的中心高度$z_i$~(m)与相邻两层雪交界处的高度$z_{h,i}$~(m)计算为:
\begin{equation}
\begin{aligned}
z_{i} &= z_{h, i}-0.5 \Delta z_{i} \quad i=0, \ldots, snl+1 \\ 
z_{h, i} &= z_{h, i+1}-\Delta z_{i+1}  \quad i=-1, \ldots, snl
\end{aligned}
\end{equation}


\section{计算框架}\label{计算框架}

{
\begin{figure}[htbp]
\centering
\includegraphics[width=\textwidth]{Figures/模式构架/CoLM计算框图_v5.png}
\caption{CoLM计算框图}
\label{fig:CoLM计算框图}
\end{figure}
}

CoLM主程序计算流程主要包括模式数据的读取,时间步循环,次网格循环,以及模式数据输出(图~\ref{fig:CoLM计算框图})。涉及的过程主要在\texttt{main}文件夹下\texttt{CoLM.F90},\texttt{CoLMDRIVER.F90}及 \allowbreak \texttt{CoLMMAIN.F90}
\allowbreak (\texttt{URBAN\allowbreak /Urban\allowbreak \_CoLMMAIN.F90},当城市模式打开时)文件中进行调用。

\textbf{模式数据读取}大概可分为驱动数据和地表数据。其中驱动数据可以通过在离线模式下读取大气强迫数据,包括温度、风速、辐射、降雨等必要变量,以及气溶胶、氮沉降等可选变量,具体可参见表~\ref{tab:陆面模式所需的大气状态变量}。目前可用于驱动CoLM离线运行的大气驱动数据集详见表~\ref{tab:可用于驱动CoLM离线运行的大气驱动数据集}。在读取过程中可根据需要进行适当数据尺度转换(章节~\ref{大气强迫降尺度});如时间尺度不匹配,可进行简单时间维插值,详见\hyperlink{驱动数据时间尺度插值}{驱动数据时间尺度插值};另外,驱动数据可以通过与大气模式耦合来获得(章节~\ref{CoLM与大气模式耦合})。

\textbf{模式数据输出}可通过既定的网格在用户设定的时间间隔内进行变量聚合(次网格聚合到网格,章节~\ref{次网格}),作为模式运行结果输出(history file)或者对耦合大气模式的反馈,具体可参见表~\ref{tab:大气模式所需的陆面模式输出变量}。网格输出类型包括经纬度网格、流域单元网格(章节~\ref{流域单元网格})及非结构网格(章节~\ref{非结构网格}),同时也可以根据用户需求进行网格转换。地表数据涉及的内容繁多,根据模式运行所选择的功能选项按需读取,涉及的数据详见章节~\ref{基础数据} \nameref{基础数据}部分。

\textbf{时间步循环}主要包括次网格循环(涉及的过程在\texttt{CoLMDRIVER.F90}和\texttt{CoLMMAIN.F90\allowbreak /Urban\allowbreak \_CoLMMAIN.F90}进行调用)以及可选模块(侧向流、河道径流、数据同化和土地利用土地覆盖变化等模块,在\texttt{CoLM.F90}进行调用)的运行(图~\ref{fig:CoLM计算框图} \textbf{时间步循环}所示)。

\textbf{次网格循环}沿用了原CoLM运行方式,对5种非海洋patch进行循环计算,即植被(含裸土)、城市、湿地、冰川和水体。如果打开海洋模块,则在循环中加入对海洋热力过程的模拟(图~\ref{fig:CoLM计算框图} \textbf{次网格循环}所示)。图中\textbf{次网格循环}所列过程为其简要列表,具体物理过程将从\nameref{part:flux}、\nameref{part:temp}、\nameref{part:SPC}、\nameref{part:hydro}、\nameref{part:BGC}、\nameref{part:human}、\nameref{part:scaling}七大部分各相关章节进行阐述。

\section{陆气耦合}\label{陆气耦合}
\subsection{陆面模式与大气模式的数据交换}
陆面模式的运行需要当前时刻的大气状态作为驱动。当陆面模式离线运行时(offline),大气状态可由观测数据、再分析数据或大气模式模拟结果数据直接提供;当陆面模式置于天气/气候/地球系统模式耦合运行时(online),大气状态可由大气模式通过耦合器或程序调用接口实时传递给陆面模式。陆面模式接收到当前时刻的大气状态后,首先结合上一时刻的植被、雪盖或土壤的状态和地表特征(如地表反照率、空气动力学阻抗等)计算当前时刻的陆气湍流交换通量、辐射通量、进入地表的能量和水分通量等,然后基于这些通量条件计算当前时刻的植被、雪盖或土壤的状态变量和地表特征量。在耦合运行时,这些地表通量和状态变量实时返回给大气,为大气模式进行下一时刻的计算提供下边界条件。陆面模式所需的大气变量和大气模式所需的陆面变量详见表~\ref{tab:陆面模式所需的大气状态变量} 和表~\ref{tab:大气模式所需的陆面模式输出变量}。

关于参考高度(reference height) (见表~\ref{tab:陆面模式所需的大气状态变量}),即大气驱动变量所在的高度,需要做一点特殊说明。参考高度的设置通常比较随意,一般认为只要不超过近地层即可。在陆面模式离线运行试验中,参考高度通常为10 m到50 m的高度;在陆气耦合试验中,参考高度通常设为大气模式的最底层。但本团队的最新研究\citep{liu2023referenceheight} 表明,把参考高度设置在近地层顶附近,可以显著提高地表湍流通量的模拟精度。考虑1 km的典型对流边界层高度,建议参考高度设置为100 m左右。

{
\begin{table}[htbp]
\centering
\caption{陆面模式所需的大气状态变量}
\label{tab:陆面模式所需的大气状态变量}
\begin{threeparttable}
\begin{tabular}{lcc}
\toprule
大气状态变量               & 变量名           & 单位           \\  \midrule
大气风速参考高度             & $z_{atm,m}$   & m            \\
大气温度参考高度             & $z_{atm,h}$   & m            \\
大气比湿参考高度             & $z_{atm,w}$   & m            \\
位于$z_{atm,m}$高度的纬向风速 & $u_{atm}$     & \unit{m.s^{-1}}   \\
位于$z_{atm,m}$高度的经向风速 & $v_{atm}$     & \unit{m.s^{-1}}   \\
位于$z_{atm,h}$高度的大气温度 & $T_{atm}$     & K            \\
位于$z_{atm,w}$高度的大气比湿 & $q_{atm}$     & \unit{kg.kg^{-1}} \\
近地面气压                & $P_{atm}$     & Pa           \\
近地面下行长波辐射            & $L ^\downarrow$ & \unit{W.m^{-2}}   \\
近地面下行短波辐射            & $S ^\downarrow$ & \unit{W.m^{-2}}   \\
降水                   & $p$           & \unit{mm.s^{-1}}      \\
二氧化碳浓度               & $c_a$         & ppmv         \\
臭氧浓度                 & $c_o$         & \unit{mol.mol^{-1}}  \\
大气气溶胶沉降速率        & $D_{sp}$      & \unit{kg.m^{-2}.s^{-1}}  \\
氮沉降速率                & $N_{dep}$     & \unit{g(N).m^{-2}.yr^{-1}}   \\
闪电频率                 & $I_l$         & \unit{flash.km^{-2}.hr^{-1}} \\ \bottomrule    
\end{tabular}
\begin{tablenotes}
\footnotesize
\item[1] 根据气溶胶种类和亲水性,气溶胶沉降速率可按照14种不同的气溶胶给出,其中沙尘气溶胶可分为8种(视为4种不同气溶胶颗粒大小的干气溶胶或湿气溶胶),黑碳气溶胶分为3种(干亲水性气溶胶、湿亲水性气溶胶、干疏水性气溶胶),有机碳气溶胶分为3种(干亲水性气溶胶、湿亲水性气溶胶、干疏水性气溶胶)。气溶胶沉降主要用于积雪、冰盖和气溶胶辐射模型 (SNICAR),影响积雪反照率和积雪内部辐射传输过程的计算。 
\item[2] 氮沉降速率用于生物地球化学循环模型(BGC),表征无机氮(主要由氮氧化物 $\mathrm{NO_y}$ 和氮氢化物 $\mathrm{NH_x}$ 组成)在陆地表面的沉降通量。
\item[3] 臭氧浓度用于模拟其对气孔导度等植被生理过程的影响,闪电频率用于火灾模式。
\end{tablenotes}
\end{threeparttable}
\end{table}
}
{
\begin{table}[htbp]
\centering
\caption{大气模式所需的陆面模式输出变量}
\label{tab:大气模式所需的陆面模式输出变量}
\begin{threeparttable}
\begin{tabular}{lcc}
\toprule
陆面模式输出变量    & 变量名                            & 单位      \\ \midrule
潜热通量        & $\lambda E$ & \unit{W.m^{-2}}    \\
感热通量        & $H$                    & \unit{W.m^{-2}}    \\
水汽通量        & $E$                    & \unit{mm.s^{-1}}    \\
纬向动量通量      & $\tau_{x}$      & \unit{kg.m^{-1}.s^{-2}} \\
经向动量通量      & $\tau_{y}$      & \unit{kg.m.s^{-2}} \\
地表出射长波辐射通量  & $L ^\uparrow$    & \unit{W.m^{-2}}    \\
直射光可见光波段反照率 & $\alpha_{vis,dir}$             & -       \\
直射光近红外波段反照率 & $\alpha_{nir,dir}$             & -       \\
漫射光可见光波段反照率 & $\alpha_{vis,dif}$             & -       \\
漫射光近红外波段反照率 & $\alpha_{nir,dif}$             & -       \\
地表辐射温度      & $T_{rad}$                      & K       \\
近地面2 m温度     & $T_{2m}$                       & K       \\
近地面2 m比湿     & $q_{2m}$                       & \unit{kg.kg^{-1}}   \\
近地面10 m风速    & $u_{10m}$                      & \unit{m.s^{-1}}     \\
雪水当量        & $W_{sno}$                      & mm      \\
空气动力学阻抗     & $r_{am}$                       & \unit{s.m^{-1}}     \\
摩擦速度        & $u_\ast$                       & \unit{m.s^{-1}}     \\
净生态系统碳交换通量  &   NEE                      & \unit{g.C.m^{-2}.s^{-1}} \\
氧化亚氮浓度      & $\mathrm{N_2O}$               & \unit{g.N.m^{-2}.s^{-1}}\\
\bottomrule         
\end{tabular}
\begin{tablenotes}
\footnotesize
\item[1] $\lambda$ 表示蒸发潜热(\unit{J.kg^{-1}}),$\lambda$ 根据地表水份是否冻结取为蒸发潜热或升华潜热。
\end{tablenotes}
\end{threeparttable}
\end{table}
}


\subsection{离线运行CoLM可使用的大气驱动数据集}\label{离线大气驱动数据集}
CoLM支持多套大气驱动数据集的使用,包括全球/区域格点数据和单点数据。其中,全球/区域格点数据见表~\ref{tab:可用于驱动CoLM离线运行的大气驱动数据集}。用户也可使用其他数据,只需参照任意一种目前支持的数据集制作成对应的数据格式,并参照该数据对应的namelist提供相应的信息即可使用。特别地,当运行区域或单点模式时,大气驱动数据可选择任意一套覆盖该区域或该点的数据,模式可自动从大气驱动数据中提取出该区域或该点的数据信息来驱动CoLM。

\hypertarget{驱动数据时间尺度插值}{CoLM默认的积分时间步长为1800秒}。当大气驱动数据的时间分辨率不满足这一积分步长时,相邻数据之间将进行插值以保持大气驱动数据的时间分辨率与积分步长相同。其中,温度、气压、比湿、风速、长波辐射通量默认采用线性插值获得,降水速率采用临近时刻插值获得,短波辐射通量采用基于太阳高度角的余弦值加权平均获得。短波辐射通量的插值具体操作为:对于插值时刻$t_M$,短波辐射通量$S^{\downarrow}$($t_M$)计算为
\begin{equation}\label{t_M}
S^{\downarrow}\left(t_{M}\right)=\left\{\begin{array}{ll}\frac{ \mu\left(t_{M}\right)}{\overline {\sum_{i=1}^{\frac{\Delta t_{FD}}{\Delta t_{M}}} \mu\left(t_{M_{i}}\right)}}S^{\downarrow}\left(t_{F D}\right)  & \text { 当 }\ \mu\left(t_{M}\right)>0.001 \\
0 & \text { 当 }\ \mu\left(t_{M}\right) \leqslant 0.001\end{array}\right.
\end{equation}
其中,$\Delta t_{FD}$代表驱动数据的时间间隔(例如,对于3-hourly数据$\Delta t_{FD}$=10800秒),$\Delta t_{M}$代表模式积分步长,$S^{\downarrow}(t_{FD})$ 代表比$t_M$时刻早的最接近$t_M$时刻的短波辐射驱动数据,$\mu\left(t_M\right)$代表$t_M$时刻的太阳高度角的余弦值,$\overline{
\sum_{i=1}^{\frac{\Delta t_{F D}}{\Delta t_{M}}} \mu\left(t_{M_{i}}\right)}$代表落在驱动数据时间间隔内的所有模式积分时刻的太阳高度角余弦值的平均值。以上插值方式可根据原数据特征及用户需求在大气强迫namelist配置文件中进行修改。

当大气强迫只有总的入射太阳辐射时,CoLM采用源于Sib2的经验方法对其进行直射/漫射,可见光/近红外分解。根据网格所在经纬度和Julian
day计算太阳高度角余弦值\(\mu\),由此计算云覆盖比例:
%
\begin{equation}
f_{cloud} = \min\left( 1,\max\left( 0.58,\frac{1160\mu - S^{\downarrow}}{963\mu} \right) \right)
\end{equation}
%
入射漫射辐射比例首先计算为:
%
\begin{equation}
f_{dif} = \min\left( 1,\max\left( 0,\frac{0.0604}{\mu - 0.0223} + 0.0683 \right) \right)
\end{equation}
%
然后根据云覆盖比例调整为:
%
\begin{equation}
f_{dif} = f_{dif} + \left( 1 - f_{dif} \right)f_{cloud}
\end{equation}
%
太阳辐射可见光波段占比计算为:
%
\begin{equation}
f_{vis} = \frac{500 - {464f}_{cloud}}{\left( 580 - {499f}_{cloud} \right) + \left( 580 - {464f}_{cloud} \right)}
\end{equation}
%
最终分波段的直射/漫射入射太阳辐射计算为:
%
\begin{equation}
\begin{aligned}
S_{vis,dir}^{\downarrow} &= S^{\downarrow}\left( 1 - f_{dif} \right)f_{vis}\\
%
S_{nir,dir}^{\downarrow} &= S^{\downarrow}\left( 1 - f_{dif} \right)\left( 1 - f_{vis} \right)\\
%
S_{vis,dif}^{\downarrow} &= S^{\downarrow}f_{dif}f_{vis}\\
%
S_{nir,dif}^{\downarrow} &= S^{\downarrow}f_{dif}\left( 1 - f_{vis} \right)
\end{aligned}
\end{equation}
%

对于QIAN强迫驱动太阳辐射,CoLM采用CLM4.5方案,可见光和近红外分配系数$f_{vis}=f_{nir}=0.5$。对于以上两个波段,太阳辐射被进一步分为直射光与漫射光。直射光在可见光波段的比例系数为
\begin{equation}\label{R_vis}
f_{vis,dir}=a_{0}+a_{1} f_{vis} S^{\downarrow}+a_{2}(f_{vis} S^{\downarrow})^{2}+a_{3}(f_{vis} S^{\downarrow})^{3} \quad 0.01 \leqslant f_{vis,dir} \leqslant 0.99
\end{equation}
直射光在近红外波段的比例系数为
\begin{equation}
f_{nir,dir}=b_{0}+b_{1}(1-f_{nir}) S^{\downarrow}+b_{2}((1-f_{nir}) S^{\downarrow})^{2}+b_{3}((1-f_{nir}) S^{\downarrow})^{3} \quad 0.01 \leqslant f_{nir,dir} \leqslant 0.99
\end{equation}
其中$a_0$=0.17639, $a_1$=0.0038, $a_2=-9.0039\times{10}^{-8}$, $a_3=8.1351\times10^{-9}$, $b_0=0.29548$, $b_1=0.00504$, $b_2=-1.4957\times10^{-5}$, $b_3=1.4881\times10^{-8}$。这些系数来自对 NCAR-CAM 模式模拟结果的拟合。
相应分波段的直射/漫射入射太阳辐射计算为:
%
\begin{equation}
\begin{aligned}
S_{vis,dir}^{\downarrow} &= S^{\downarrow}f_{vis,dir} f_{vis}\\
%
S_{nir,dir}^{\downarrow} &= S^{\downarrow}f_{nir,dir} f_{nir} \\
%
S_{vis,dif}^{\downarrow} &= S^{\downarrow}\left ( 1 - f_{vis,dir} \right ) f_{vis}\\
%
S_{nir,dif}^{\downarrow} &= S^{\downarrow}\left ( 1 - f_{nir,dir} \right ) f_{nir}
\end{aligned}
\end{equation}

当大气驱动数据未提供下行长波辐射通量时,模式将根据大气温度$T_{atm}$、比湿$q_{atm}$和水汽压$e_{atm}$计算获得~\citep{idso1981SetEquationsFull},计算方案为
\begin{equation}\label{L_downarrow}
L ^\downarrow=\left[0.70+5.95 \times 10^{-5} \times 0.01 e_{a t m} \exp \left(\frac{1500}{T_{a t m}}\right)\right] \sigma T_{a t m}^{4}
\end{equation}
其中大气水汽压计算为$e_{a t m}=\frac{P_{a t m} q_{a t m}}{0.622+0.378 q_{a t m}}$,$\sigma$为Stefan-Boltzmann常数。

未来情景下的大气驱动数据可采用CMIP6气候模式的模拟输出结果。当历史时期与未来情景共同模拟时,未来时期CMIP6气候模式的模拟结果往往不能直接使用,因为当模拟历史时期使用的大气驱动数据来自观测或再分析数据时,其均值、时空变率、数据拼接时刻的取值与CMIP6气候模式的模拟结果往往存在较大差异。这里可使用异常值叠加方式来融合历史与未来的大气驱动数据。举例说明:当模拟1850--2099年的陆面状态时,1850--2014年的历史时期模拟可采用GSWP3大气驱动数据,2015--2099年的未来时期模拟可采用CMIP6气候模式的模拟结果,两套数据的融合方式如下:首先,将2015--2100年各驱动变量的逐月数据去除气候态的季节循环,得到带有某种社会经济排放情景下的各驱动变量的异常值序列,此序列可表征未来气候变化趋势的影响;其次,将历史时期有限时间段的各大气驱动变量反复叠加到未来时期各驱动变量的异常值序列(例如将2010-2014年的大气驱动数据反复叠加到2015--2019、2020--2024、…、2095--2099年的异常值序列),得到可使用的未来时期的大气驱动数据。此种数据融合方式既得到了未来时期的气候变化趋势,又保留了历史时期大气驱动数据的年际变率,缩小了历史时期与未来时期数据拼接时刻的数据阶跃程度。对于不同的大气驱动变量,历史时期数据与未来时期异常值序列的叠加方式不同:对于温度、气压、比湿、风速等大气状态变量,叠加方式采用加法叠加($S_{new}=S_{old}+k_{anomaly}$),对于降水、下行短波辐射通量、下行长波辐射通量,叠加方式采用乘法叠加($F_{new}=F_{old}\times k_{anomaly}$)。


最后,针对单点模拟,CoLM已经收集整理测试的站点信息见表~\ref{tab:CoLM离线运行已经测试的站点列表}。

\subsection{通过耦合器实现CoLM与大气模式的耦合模拟}\label{CoLM与大气模式耦合}
陆面是天气/气候/地球系统的重要组成部分,其物理、化学、生物过程深刻影响着陆地与大气、陆地与海洋之间的能量与物质交换~\citep{Dai2020}。因此,在天气/气候/地球系统模式的耦合框架下运行陆面过程模式,既可刻画在特定条件下陆面系统内部复杂的多圈层多尺度过程,又可将陆面系统变化反馈给大气和海洋系统,充分刻画地球系统各分量之间紧密的相互作用。

地球系统模式通过耦合器连接大气、海洋、陆面、海冰等多个分量模式,形成一个包含地球系统各圈层演化和相互作用过程的完整模式软件。耦合器负责多个分量模式之间的数据交换、不同网格之间数据的插值、以及物质能量等的守恒性诊断,并控制整个地球系统模式的积分模拟~\citep{tangyanli_2015, chenyiran_2017}。耦合器的“可插拔式”框架更有利于促进地球系统模式的模块化和高效并行化。现有的耦合器主要有美国国家大气研究中心(National Center For Atmospheric Research, NCAR)开发的CPL (Coupler)耦合器、美国普林斯顿大学地球物理流体动力学实验室(Geophysical Fluid Dynamics Laboratory)开发的FMS (Flexible Modelling System)、美国的耦合工具库ESMF(Earth System Modelling Framework)、法国欧洲气候模拟和全球变化研究中心开发的OASIS等,其中美国NCAR的CPL和法国的OASIS应用最为广泛。近年来,国内自主研发的耦合器得到了显著发展,其中由清华大学研制的耦合器C-Coupler取得了一系列自主创新,完成了三个版本的研制~\citep{LiuLi_2023},并逐渐成长为国际首个面向地球系统数值预报的一体化软件平台,已稳定应用于国内多家业务和科研单位。目前,CoLM已经通过CPL7实现与中国科学院地球系统模式CAS-ESM和区域气候模式CWRF的耦合,使得CoLM成为其中的陆面分量模式。其他使用CPL7的地球系统模式可通过类似流程实现与CoLM的耦合。CoLM将在后续尝试通过国产耦合器C-Coupler实现与中国气象局全球数值天气预报系统CMA-GFS等更多国内业务预报模式的耦合,使得CoLM能够更好地服务于我国天气气候研究与业务预报,有效提升天气气候模拟与预测精度。本节将简要介绍CoLM与CPL7的基本耦合原理,CoLM与CAS-ESM和CWRF的耦合流程见附录。

CPL7作为目前使用最为广泛的耦合器之一,其功能不再仅局限于地球系统模式中不同分量模式的数据连通器,它可直接作为模式的顶层驱动,控制各个分量模式运行的先后次序。CPL7作为总控程序控制CAS-ESM的运行流程可参见图~\ref{fig:CAS-ESM的运行流程},其中每个分量模式均需进行初始化,之后从耦合器读入需要的数据、运行模式、将必要的运行结果发送回耦合器,以此流程逐步积分直至结束。CPL7通过标准代码接口调用各个分量模式的初始化、运行和结束子程序,并在分量模式需要进行数据交换时调用耦合器函数(网格插值、数据发送与接收、重排与融合、通量计算等)实现分量模式之间的相互作用。分量模式之间的并发关系可根据模式计算量和用户需求来指定。总体上,CPL7实现总控调度和各分量模式之间的耦合主要基于三个方面的设计:并行机制、网格映射和时间管理。下面对这三方面逐一进行简要介绍。
{
\begin{figure}[htbp]
\centering
\includegraphics{Figures/模式构架/CAS-ESM的运行流程.png}
\caption[CPL7作为总控程序控制CAS-ESM的运行流程]{CPL7作为总控程序控制CAS-ESM的运行流程(引自何卷雄关于“地球系统数值模拟装置项目地球系统模式数值模拟系统集成模块分系统培训”PPT)}
\label{fig:CAS-ESM的运行流程}
\end{figure}
}

首先,在模式并行机制方面,CPL7除了将地球系统模式整体划分为一个完整的通讯域外,每个分量模式自身以及分量模式与耦合器各形成两个子通讯域,见图~\ref{fig:CPL7并行机制示意图}。每个分量模式自身的通讯域用于模式内部的数据交换,它使得分量模式各自的并行算法得以保持,不受其他分量模式和耦合环境的影响。每个分量模式与耦合器形成的通讯域用于模式与耦合器的数据交换,各分量模式只能从耦合器获取运行时所需的数据,并将其他分量模式所需的数据直接发送给耦合器,这种由耦合器充当数据传输枢纽的方式使得数据传递变得易于管理,有效避免了以往各分量模式之间进行数据传递时由于数据分辨率和所在处理器不同造成的混乱。基于此,当CoLM与CPL7进行耦合时,CoLM可直接接收CPL7提前划分的用于陆面模式独立运行的通讯域,并通过CPL7的模型耦合工具MCT(The Model Coupling Toolkit)提供的数据结构Attrvect实现与CPL7的数据交换。

{
\begin{figure}[htbp]
\centering
\includegraphics{Figures/模式构架/CPL7并行机制示意图.png}
\caption[CPL7并行机制示意图]{CPL7并行机制示意图。对于有5个组件(大气、陆面、海洋、海冰和耦合器)组成的地球系统模式,共划分为10个通讯域:5个组件各自形成通讯域、4个分量模式分别与耦合器形成通讯域、以及5个组件整体形成一个通讯域(引自何卷雄关于“地球系统数值模拟装置项目地球系统模式数值模拟系统集成模块分系统培训”PPT)}
\label{fig:CPL7并行机制示意图}
\end{figure}
}

其次,在网格映射方面,CPL7实现网格映射主要依赖模型耦合工具。目前,CPL7支持的模型耦合工具有MCT和地球系统模拟框架软件ESMF,默认情况下使用MCT。MCT提供的数据结构见图~\ref{fig:MCT的基本数据结构}。其中,GlobalSegmentMap主要描述各分量模式局部网格在全局网格上的映射,存储的信息包括分量模式的ID号、有多少个数据段和全局网格、每个数据段的网格在全局中的起始编号和长度、以及每个数据段所属的处理器编号,通过以上信息可获得各分量模式在每个处理器上处理的局部网格与全局网格的对应关系,此对应关系用于分量模式与耦合器进行数据交换时,数据在耦合器中的重新排列。GeneralGrid存储了各分量模式的物理网格信息,包含了每个处理器的网格坐标、面积、海陆标识、及其在全球网格的编号等。AttrVect是GeneralGrid数据结构中的一部分,它是一个二维数组,用于存储各分量模式在每个处理器的所有格点上需要进行交换的物理量数据,并与耦合器进行通信。GlobalSegmentMap和GeneralGrid是分量模式与耦合器进行数据交换的最基本的数据结构。当CoLM与CPL7进行耦合时,需要对GlobalSegmentMap和GeneralGrid进行相应信息的赋值,才能实现二者的信息交换。

{
\begin{figure}[htbp]
\centering
\includegraphics{Figures/模式构架/MCT的基本数据结构.png}
\caption{模型耦合工具MCT使用的基本数据结构}
\label{fig:MCT的基本数据结构}
\end{figure}
}

当耦合器获取来自各分量模式的数据以及各分量模式的局部网格在全局网格上的映射关系和所属处理器信息后,耦合器将在内部实现模式间的数据迁移。数据迁移的底层数据类型是MCT中的Router,Router为数据在不同处理器之间的迁移做好设置准备。模式之间的完整数据迁移一般分为两种情况。一种是进行信息交换的两个模式其网格划分/分辨率相同(如大气模式与陆面模式),此时只需要使用Router的上级数据类型Rearranger,即可完成对数据从局部网格到全局网格的重新排列,并实现数据从源模式处理器到耦合器处理器再到目标模式处理器的迁移。另一种是进行信息交换的两个模式其网格划分/分辨率不同(如大气模式与海洋模式),此时需要使用Rearranger的上级数据类型SparseMatrixPlus,该数据类型在完成数据从源模式处理器到耦合器处理器的迁移后,对数据从源模式网格到目标模式网格进行插值,然后再将数据重新排列并向目标模式处理器进行迁移。在CPL7中,网格的插值会通过一个矩阵乘来实现:$y=Tx$,其中$x$是物理场在源网格的分布,$y$是物理场在目标网格的分布,插值权重$T$是个稀疏矩阵,MCT提供了数据类型SparseMatrix来专门存储这个稀疏矩阵,并采用属性向量AttrVect来存储其非0元素的行列编号。插值权重$T$的取值主要取决于插值算法。常用的插值算法有四种:(1)双线性插值,此算法不保证物理量守恒,但很好的保持了物理量的空间分布和梯度,适用于标量插值;(2)面积加权(patch)平均,此算法可保证质量和能量守恒;(3)二阶面积加权平均,此算法在保证物理量守恒外,还可在一定程度上保持物理量的空间分布和梯度,适用于通量插值;(4)临近点插值。插值权重$T$由基于以上插值算法制作的外部映射文件读入(映射文件可通过NCL或Python中自带的函数制作完成)。得到$T$后,整个数据迁移过程即可封装在SparseMatrixPlus数据类型中完成。作为示例,图~\ref{fig:大气接收数据}~和图~\ref{fig:陆面接收数据}~分别展示了大气模式和陆面模式所接收的数据其数据迁移的完整流程。对于大气模式,其所需要的下垫面边界条件来自陆面、海洋和海冰模式,各源模式首先将大气模式所需的数据迁移至耦合器处理器并进行重排(通过图~\ref{fig:大气接收数据}~各蓝色通讯域Rearranger实现),然后在耦合器内部将数据从各源模式网格插值到大气模式网格,并将所有数据进行融合(通过图~\ref{fig:大气接收数据}~红色通讯域中的Map和Merge实现),最后将融合到一起的覆盖全部大气网格的数据进行重排并迁移至大气模式处理器(通过图~\ref{fig:大气接收数据}~棕色通讯域Rearranger实现)。对于陆面模式,由于默认情形下其与大气模式保持相同分辨率的网格,因此其所需要的大气状态变量只需要从大气模式处理器经过重排迁移至耦合器处理器,再经过重排迁移至陆面模式处理器即可(通过图~\ref{fig:陆面接收数据}~各通讯域Rearranger实现)。以上两个示例与上述介绍的SparseMatrixPlus和Rearranger数据类型实现的功能相对应。

{
\begin{figure}[htbp]
\centering
\includegraphics{Figures/模式构架/大气接收数据.png}
\caption[大气模式接收数据的数据迁移基本流程]{大气模式接收数据的数据迁移基本流程(修改自何卷雄关于“地球系统数值模拟装置项目地球系统模式数值模拟系统集成模块分系统培训”PPT)}
\label{fig:大气接收数据}
\end{figure}
}

{
\begin{figure}[htbp]
\centering
\includegraphics{Figures/模式构架/陆面接收数据.png}
\caption[陆面模式接收数据的数据迁移基本流程]{陆面模式接收数据的数据迁移基本流程(修改自何卷雄关于“地球系统数值模拟装置项目地球系统模式数值模拟系统集成模块分系统培训”PPT)}
\label{fig:陆面接收数据}
\end{figure}
}

最后,在时间管理方面,CPL7的时间管理一般采用esmf\_wrf\_timemgr软件。它包含了一系列的ESMF时间管理接口,各个分量模式和顶层驱动的时间管理都采用ESMF时间管理库来控制。基于ESMF时间管理库,模式中通常包含由顶层驱动管理的主时钟和控制分量模式的组件时钟以及由各分量模式自行管理的内部时钟。模式运行之前需要对所有时钟进行初始化,包括对各分量模式起止时间、重启时间、时间步长、耦合时间、当前步数、输入输出时间、以及运行闹钟等信息进行设置。当顶层驱动将主时钟和分量模式组件时钟初始化后,各分量模式会根据组件时钟来初始化模式内部的时钟。主时钟的步长是各个组件时钟的最小时间步长,并同时记录各个分量模式运行的闹钟信息。当模式向前推进一个主时钟步长后,将检查各个分量模式是否到达闹钟时间,并通知到达闹钟时间的分量模式开始运行,同时更新下一次闹钟时间。以此过程循环直至模式整体积分结束。

以上即是CPL7实现总控调度和各分量模式耦合主要涉及的三个方面。事实上,为方便用户接入新的分量模式,CPL7为各个分量模式提供了公共程序调用接口,存储于\texttt{*\_comp\_mct.F90}文件中。公共接口主要由三个子程序组成:Init(分量模式初始化)、run(分量模式运行)和final(分量模式积分结束)。在三个子程序内部,同样提供了可实现上述分量模式与CPL7完成耦合的内部接口,如Setgsmap和domain格点分布和映射函数、import和export分量模式从(向)耦合器输入(出)数据函数、read\_srfrest和write\_srfrest读(写)重启动文件函数等。用户可参考已有的\texttt{*\_comp\_mct.F90}文件对相应的接口程序进行修改,从而实现新分量模式与CPL7的耦合。CoLM与CPL7的耦合即是借助\texttt{lnd\_comp\_mct.F90}中的公共程序调用接口实现。CoLM与中国科学院地球系统模式CAS-ESM和区域气候模式CWRF的大致耦合流程将在附录中给出。

\chapter{基础数据}\label{基础数据}
%\addcontentsline{toc}{chapter}{地表输入数据}


\section{高程数据}
高程数据来自MERIT DEM数据集 \citep{yamazaki2017high}。该数据集是通过从已有的遥感DEM (SRTM3 v2.1和AW3D-30m v1) 中去除多种误差 (绝对偏差、条纹噪声、散斑噪声和树高偏差) 而开发的,它的分辨率为3秒 (赤道处约90米) ,覆盖了90N-60S之间的陆地区域。

\section{地表覆盖数据}\label{地表覆盖数据}
\subsection{USGS地表覆盖数据}\label{USGS地表覆盖数据}
USGS地表覆盖数据来自美国地质调查局(USGS),数据全称为Global Land Cover Characterization (GLCC) database Version 2.0 
(\url{https://www.usgs.gov/centers/eros/science/usgs-eros-archive-land-cover-products-global-land-cover-characterization-glcc})。
数据为全球 \ang{;;30} (约1公里) 分辨率,经纬度网格,单时次数据,划分类型如表~\ref{tab:USGS覆盖类型} 所示。
% Please add the following required packages to your document preamble:
% \usepackage{booktabs}
\begin{table}[]
\centering
\caption{USGS覆盖类型}
\label{tab:USGS覆盖类型}
\begin{tabular}{@{}ll@{}}
\toprule
编号 & USGS覆盖类型     \\ \midrule
1  & 城市           \\
2  & 干旱农田与牧场      \\
3  & 灌溉农田与牧场      \\
4  & 干旱/灌溉混合农田与牧场 \\
5  & 农田草地过渡带      \\
6  & 农田林地过渡带      \\
7  & 草地           \\
8  & 灌木地          \\
9  & 草地灌木地混合带     \\
10 & 稀疏草原         \\
11 & 落叶阔叶林        \\
12 & 落叶针叶林        \\
13 & 常绿阔叶林        \\
14 & 常绿针叶林        \\
15 & 混合森林         \\
16 & 内陆水体         \\
17 & 草本湿地         \\
18 & 森林湿地         \\
19 & 贫瘠稀疏植被       \\
20 & 草本苔原         \\
21 & 森林苔原         \\
22 & 混合苔原         \\
23 & 裸土苔原         \\
24 & 雪盖或冰川        \\ \bottomrule
\end{tabular}
\end{table}


由于USGS地表覆盖数据为单时次且年代稍远(主要数据源自AVHRR 1992-1993年间),因此辅以其他特定类型地表覆盖数据对USGS地表覆盖进行更新。
包括全球1公里水体和湿地数据(Global Lakes and Wetlands Database: Lakes and Wetlands Grid (Level 3))~\citep{lehner2004development}、
全球1公里冰川数据~\citep{RGIConsortium2017}、全球1公里城市覆盖数据 (MODIS)~\citep{schneider2009new} 和全球1公里高程数据(USGS)。
替换方式是将以上辅助数据中水体、冰川和城市网格直接对USGS地表数据进行替换。USGS高程数据用于判断海洋,即将高程低于0米的地方设置为海洋格点。
如采用USGS地表覆盖类型进行模拟 (LCT方案),其不同类型相关参数请参考附录~\ref{USGS地表覆盖类型相关参数}。

\subsection{MODIS IGBP地表覆盖数据}\label{IGBP地表覆盖数据}
IGBP地表覆盖数据来自美国宇航局(NASA)Moderate Resolution Imaging Spectroradiometer 
(MODIS)MCD12Q1 Version 6(The Terra and Aqua combined MODIS Land Cover Type data product)
的数据(\url{https://lpdaac.usgs.gov/products/mcd12q1v006/}),该数据(LC\_Type1)是Terra和Aqua两种卫星数据结合产生 
 \citep{Friedl2019}。数据为全球 \ang{;;15}(约500 m)分辨率,经纬度网格(对原始数据做了投影转换),每年一幅数据,时间范围从2001至今,分类如表~\ref{tab:IGBP覆盖类型} 所示。

\begin{table}[]
\centering
\caption{IGBP覆盖类型}
\label{tab:IGBP覆盖类型}
\begin{tabular}{@{}ll@{}}
\toprule
编号 & IGBP覆盖类型     \\ \midrule
1  & 常绿针叶林           \\
2  & 常绿阔叶林      \\
3  & 落叶针叶林     \\
4  & 落叶阔叶林 \\
5  & 混合林     \\
6  & 郁闭灌丛      \\
7  & 稀疏灌丛           \\
8  & 稀疏大草原(木本为主)         \\
9  & 稀疏大草原     \\
10 & 草地         \\
11 & 永久性湿地        \\
12 & 耕地        \\
13 & 城市        \\
14 & 耕地和自然植被混合带        \\
15 & 积雪和冰川        \\
16 & 裸土或稀疏植被覆盖       \\ 
17 & 水体           \\ \bottomrule
\end{tabular}
\end{table}

\subsection{ESA CCI地表覆盖数据}\label{ESA地表覆盖数据}
ESA CCI地表覆盖产品是由欧洲宇航局(ESA)的气候变化计划(CCI)提供(\url{http://maps.elie.ucl.ac.be}),全球覆盖,空间分辨率300 m,时间覆盖1992--2020年。采用联合国粮食及农业组织地表覆盖分类系统(UNLCCS),共分为36个土地覆盖类别(包括22种一级分类,14种二级分类),如表~\ref{tab:ESA覆盖类型} 所示。

\begin{table}[]
\centering
\caption[ESA CCI地表覆盖类型]{ESA CCI地表覆盖类型。正体编号表示联合国粮食及农业组织地表覆盖分类系统22种一级分类,斜体编号表示14种二级分类}
\label{tab:ESA覆盖类型}
\begin{tabular}{lll}
\toprule
\multicolumn{2}{c}{编号} & ESA CCI地表覆盖类型     \\ \midrule
10& & Cropland, rainfed\\
& \textit{11}& Herbaceous cover\\
& \textit{12}& Tree or shrub cover\\
20&& Cropland, irrigated or post-filooding\\
30&& Mosaic cropland (> 50 \%) nat. veg. (tree, shrub, herb.) (< 50 \% )\\
40&& Mosaic nat. veg. (tree, shrub, herb.) (> 50 \%)/cropland (< 50 \%) \\
50&& Tree cover, broadleaf, evergreen, closed to open (> 15 \%)          \\
60&& Tree cover, broadleaf, deciduous, closed to open (> 15 \%)        \\
& \textit{61}& Tree cover, broadleaf, deciduous, closed (> 40 \%)    \\
& \textit{62}& Tree cover, broadleaf, deciduous, open (15-40 \%)         \\
70&& Tree cover, needleleaf, evergreen, closed to open (> 15 \%)        \\
& \textit{71}& Tree cover, needleleaf, evergreen, closed (> 40 \%)        \\
& \textit{72}& Tree cover, needleleaf, evergreen, open (15-40 \%)       \\
80&& Tree cover, needleleaf, deciduous, closed to open (> 15 \%)        \\
& \textit{81}& Tree cover, needleleaf, deciduous, closed (> 40 \%)       \\
& \textit{82}& Tree cover, needleleaf, deciduous, open (15-40 \%)      \\ 
90&& Tree cover, mixed leaf type (broadleaf and needleleaf)          \\ 
 100&& Mosaic tree and shrub (> 50 \%) herbaceous cover (< 50 \%) \\
 110&& Mosaic herbaceous cover (> 50 \%) /tree and shrub (< 50 \%) \\
 120&& Shrubland\\
 & \textit{121}& Shrubland evergreen\\
 & \textit{122}& Shrubland deciduous\\
 130 && Grassland\\
 140 && Lichens and mosses\\
 150 && Sparse vegetation (tree, shrub, herbaceous cover) (< 15 \%)\\
 & \textit{152}& Sparse shrub (< 15 \%)\\
 & \textit{153}& Sparse herbaceous cover (< 15 \%) \\
 160&& Tree cover, flooded, fresh or brackish water\\
 170&& Tree cover, flooded, saline water\\
 180&& Shrub/herbaceous cover, flooded, fresh/saline/brackish water\\
 190&& Urban areas\\
 200&& Bare areas\\
 & \textit{201}& Consolidated bare areas\\
 & \textit{202}& Unconsolidated bare areas\\
 210 && Water bodies\\
 220 && Permanent snow and ice\\ \bottomrule
\end{tabular}
\end{table}

\subsection{GLC FCS30地表覆盖数据}\label{GLC FCS30地表覆盖数据}
USGS、MODIS、ESA CCI地表覆盖产品空间分辨率主要为公里级或者次公里级,近年来,地表覆盖产品分辨率已经达到了30m。GLC FSC30地表覆盖产品是由\cite{zhang2023glc_fcs30d}提供的全球地表覆盖数据,该数据空间分辨率为30m,在GEE平台通过结合多时相Landsat图像和局部自适应随机森林模型开发生成,采用了气候变化倡议土地覆盖数据集(CCI LC)分类系统,包含16个一级土地覆盖类型以及35个细分土地覆盖类型,对土地覆盖的描述更加全面。此外,GLC FCS30为提高湿地以及城市两种地表覆盖的准确性,对上述两类地表覆盖进行了单独处理,湿地和城市均采用了独立的数据。GLC FCS30产品覆盖1985-2022年,其中1985-2000年数据为每五年更新一次,2000-2022则逐年更新。由于目前模式基础植被分类采用IGBP方案,因此需要对GLC FCS30进行重分类处理,从而使每种地表覆盖都分配对应的IGBP分类编号,两种分类的对应关系如表~\ref{tab:GLC FCS30与IGBP分类对应关系} 所示。
%
% \begin{table}[]
% \fontsize{9.5}{9.5}\selectfont
% \renewcommand\arraystretch{1.85}
% \centering
% \caption{GLC FCS30与IGBP分类对应关系}
% \label{tab:GLC FCS30与IGBP对应分类}
% \begin{tabular}{c|c|c}
% \hline
% IGBP编号 & IGBP分类 & GLC FCS30细分类 \\
% \hline
% \multirow{2}{*}{1} & \multirow{2}{*}{常绿针叶林} & Closed evergreen needleleaved forest \\ 
% \cline{3-3}& & Open evergreen needleleaved forest \\
% \hline
% \multirow{2}{*}{2} & \multirow{2}{*}{常绿阔叶林} & Closed evergreen broadleaved forest \\ 
% \cline{3-3}& & Open evergreen broadleaved forest \\
% \hline
% \multirow{2}{*}{3} & \multirow{2}{*}{落叶针叶林} & Closed deciduous needleleaved forest \\ 
% \cline{3-3}& & Open deciduous needleleaved forest \\
% \hline
% \multirow{2}{*}{4} & \multirow{2}{*}{落叶阔叶林} & Closed deciduous broadleaved forest \\ 
% \cline{3-3}& & Open deciduous broadleaved forest \\
% \hline
% \multirow{2}{*}{5} & \multirow{2}{*}{混合林} & Closed mixed-leaf forest \\ 
% \cline{3-3}& & Open mixed-leaf forest \\
% \hline
% \multirow{3}{*}{6} & \multirow{3}{*}{郁闭灌丛} & Shrubland \\ 
% \cline{3-3}& & Evergreen shrubland \\
% \cline{3-3}& & Deciduous shrubland \\
% \hline
% 7 & 稀疏灌丛 & Sparse shrubland \\
% \hline
% 8 & 稀疏大草原(木本为主) & Sparse vegetation \\
% \hline
% 9 & 稀疏大草原 & Sparse herbaceous cover \\
% \hline
% \multirow{2}{*}{10} & \multirow{2}{*}{草地} & Grassland \\ 
% \cline{3-3}& & Lichens and mosses \\
% \hline
% \multirow{7}{*}{11} & \multirow{7}{*}{湿地} & Swamp \\ 
% \cline{3-3}& & Marsh \\
% \cline{3-3}& & Flooded flat \\
% \cline{3-3}& & Saline \\
% \cline{3-3}& & Mangrove \\
% \cline{3-3}& & Salt marsh \\
% \cline{3-3}& & Tidal flat \\
% \hline
% \multirow{2}{*}{12} & \multirow{2}{*}{耕地} & Rainfed cropland \\ 
% \cline{3-3}& & Irrigated cropland \\
% \hline
% 13 & 城市 & Impervious surface \\
% \hline
% \multirow{2}{*}{14} & \multirow{2}{*}{耕地和自然植被混合带} & Herbaceous cover cropland \\ 
% \cline{3-3}& & Tree or shrub cover cropland \\
% \hline
% 15 & 积雪和冰川 & Permanent snow and ice \\
% \hline
% \multirow{3}{*}{16} & \multirow{3}{*}{裸土或稀疏植被覆盖} & Bare areas \\ 
% \cline{3-3}& & Consolidated bare areas \\
% \cline{3-3}& & Unconsolidated bare areas \\
% \hline
% 17 & 水体 & Water body \\
% \hline
% \end{tabular}
% \end{table}
%
\begin{table}[]
\fontsize{9.5}{9.5}\selectfont
\renewcommand\arraystretch{1.85}
\centering
\caption{GLC FCS30与IGBP分类对应关系}
\label{tab:GLC FCS30与IGBP分类对应关系}
\begin{tabular}{cccc}
\toprule
LC ID & GLC FCS30分类 & 对应IGBP分类 & IGBP编号 \\
%
\midrule
 10 & Rainfed cropland & \multirow{2}{*}{耕地} & \multirow{2}{*}{12} \\
  20 & Irrigated cropland & & \\
%
\cline{1-4}  \textit{11} & Herbaceous cover cropland & \multirow{2}{*}{耕地和自然植被混合带} & \multirow{2}{*}{14} \\
  \textit{12} & Tree or shrub cover cropland & & \\
%
\cline{1-4}  \textit{51} & Closed evergreen broadleaved forest & \multirow{2}{*}{常绿阔叶林} & \multirow{2}{*}{2} \\
  \textit{52} & Open evergreen broadleaved forest & & \\
%
\cline{1-4}  \textit{61} & Closed deciduous broadleaved forest & \multirow{2}{*}{落叶阔叶林} & \multirow{2}{*}{4} \\
  \textit{62} & Open deciduous broadleaved forest & & \\
%
\cline{1-4}  \textit{71} & Closed evergreen needleleaved forest & \multirow{2}{*}{常绿针叶林} & \multirow{2}{*}{1} \\
  \textit{72} & Open evergreen needleleaved forest & & \\
%
\cline{1-4}  \textit{81} & Closed deciduous needleleaved forest & \multirow{2}{*}{落叶针叶林} & \multirow{2}{*}{3} \\
  \textit{82} & Open deciduous needleleaved forest & & \\
%
\cline{1-4}  \textit{91} & Closed mixed-leaf forest & \multirow{2}{*}{混合林} & \multirow{2}{*}{5} \\
  \textit{92} & Open mixed-leaf forest & & \\
%
\cline{1-4} 120 & Shrubland & \multirow{3}{*}{郁闭灌丛} & \multirow{3}{*}{6} \\
 \textit{121} & Evergreen shrubland & & \\
 \textit{122} & Deciduous shrubland & & \\
%
\cline{1-4} 130 & Grassland & \multirow{2}{*}{草地} & \multirow{2}{*}{10} \\
 140 & Lichens and mosses & & \\
%
\cline{1-4} 150 & Sparse vegetation & 稀疏大草原(木本为主) & 8 \\
%
\cline{1-4} \textit{152} & Sparse shrubland & 稀疏灌丛 & 7 \\
%
\cline{1-4} \textit{153} & Sparse herbaceous cover & 稀疏大草原 & 9 \\
%
\cline{1-4} \textit{181} & Swamp & \multirow{7}{*}{湿地} & \multirow{7}{*}{11} \\
 \textit{182} & Marsh & & \\
 \textit{183} & Flooded flat & & \\
 \textit{184} & Saline & & \\
 \textit{185} & Mangrove & & \\
 \textit{186} & Salt marsh & & \\
 \textit{187} & Tidal flat & & \\
%
\cline{1-4} 190 & Impervious surface & 城市 & 13 \\
%
\cline{1-4} 200 & Bare areas & \multirow{3}{*}{裸土或稀疏植被覆盖} & \multirow{3}{*}{16} \\
 \textit{201} & Consolidated bare areas & & \\
 \textit{202} & Unconsolidated bare areas & & \\
%
\cline{1-4} 210 & Water body & 水体 & 17 \\
%
\cline{1-4} 220 & Permanent snow and ice & 积雪和冰川 & 15 \\
%
\bottomrule
\end{tabular}
\end{table}

\section{土壤数据}\label{土壤数据}
\subsection{土壤基础属性}\label{土壤基础属性}
土壤基础属性数据采用Global Soil Dataset for Earth System Modeling(GSDE,\url{http://globalchange.bnu.edu.cn/research/soilw})~\citep{shangguan2014global}
和SoilGrids(\url{https://www.soilgrids.org/})\citep{poggio2021soilgrids} 两套数据。GSDE采用土壤连接法融合了世界土壤图和多个区域级或国家级的土壤数据库,基于土壤剖面和土壤类型图生成。其空间分辨率为1公里。其垂直层次为8层 (0--0.045, 0.045--0.091, 0.091--0.166, 0.166--0.289, 0.289--0.493, 0.493--0.829, 0.829--1.383, 1.383--2.296 m),这一分层方案与CoLM一致(第一层对应CoLM第一和第二层)。SoilGrids基于土壤剖面和数百个环境协变量图层采用基于机器学习的数字土壤制图方法生成,空间分辨率为250 m。其垂直层次为6层(0--0.05,0.05--0.15,0.15--0.3,0.3--0.6,0.6--1,1--2 m)。对两个数据按CoLM的土壤分层方案采用深度加权方法进行了标准化处理。其中第十层无数据,用第九层填补。

\subsection{基岩深度}\label{基岩深度}

基岩深度数据来自Global Depth to Bedrock Dataset for Earth System Modeling (GDBDESM,\url{http://globalchange.bnu.edu.cn/research/dtb.jsp}, \citet{shangguan2017mapping})。该数据集基于土壤剖面、地质钻孔和环境协变量图层采用机器学习的数字土壤制图方法生成,空间分辨率为1公里。


\section{植被结构及属性数据}\label{植被结构及属性数据}
\subsection{叶面积指数数据}\label{叶面积指数数据}
叶面积指数数据来自重处理的MODIS LAI数据~\citep{yuan2011reprocessing,lin2023ReprocessedMODISVersion}。目前模式提供MODIS第5版(\url{http://globalchange.bnu.edu.cn/research/lai})和第6.1版数据(\url{http://globalchange.bnu.edu.cn/research/laiv061})。其中MODIS第5版数据 (MOD15A2) 为全球 \ang{;;30} (1公里) 分辨率,经纬度网格,时间覆盖2000--2016年,时间分辨率为每8天。MODIS第6.1版数据 (MCD15A2H),空间分辨率为全球 \ang{;;15} (约500米),经纬度网格,时间覆盖从2000年至今,时间分辨率每8天或每月。


\subsection{树高数据}\label{树高数据}
树高数据来自2005年Geoscience Laser Altimeter System (GLAS) aboard ICESat 
(Ice, Cloud, and land Elevation Satellite)卫星数据~\citep{simard2011mapping},
全球 \ang{;;30} (1公里) 分辨率,经纬度网格,单时次数据。

\subsection{PFT/PC数据及其制作所依赖数据}\label{PFTPC数据及其依赖数据}
PFT/PC植被属性相关数据主要包括:1)网格所包含PFT种类;2)网格内各PFT占比;3)网格内各PFT LAI/SAI值;4)网格内各PFT高度。

模式如采用PFT或PC植被次网格方式运行,则不能像LCT方案一样直接进行地表覆盖类型聚合(图~\ref{fig:次网格聚合方案} LCT方案所示),
需要对每一个细网格地表覆盖类型进行拆解,得到其网格内PFT的组成种类和各自面积占比。
PFT方案则根据细网格计算得到的PFT种类及占比来聚合到模式网格(图~\ref{fig:次网格聚合方案} PFT方案)。
PC聚合方案同LCT,但是PC跟PFT方案一样,对每种LCT内的PFT进行表征 (包括类型和组成比例,图~\ref{fig:次网格聚合方案} PC方案)。
PFT和PC方案采用同样的植被功能型分类,如表~\ref{tab:PFT分类} 所示。
% Please add the following required packages to your document preamble:
% \usepackage{booktabs}
\begin{table}[htbp]
\centering
\caption{植被功能型-PFT分类}
\label{tab:PFT分类}
\begin{tabular}{ll}
\toprule
\multicolumn{1}{l}{编号} & \multicolumn{1}{l}{植被功能型} \\ \midrule
0                      & 裸土                        \\
1                      & 温带常绿针叶树                   \\
2                      & 北方常绿针叶树                   \\
3                      & 北方落叶针叶树                   \\
4                      & 热带常绿阔叶树                   \\
5                      & 温带常绿阔叶树                   \\
6                      & 热带落叶阔叶树                   \\
7                      & 温带落叶阔叶树                   \\
8                      & 北方落叶阔叶树                   \\
9                      & 常绿阔叶灌木                    \\
10                     & 温带落叶阔叶灌木                  \\
11                     & 北方落叶阔叶灌木                  \\
12                     & 极地C3草                     \\
13                     & C3草                       \\
14                     & C4草                       \\
15                     & C3作物                      \\ \bottomrule
\end{tabular}
\end{table}

{
\begin{figure}[htbp]
\centering
\includegraphics[width=0.9\textwidth]{Figures/基础数据/次网格聚合方案.png}
\caption[CoLM三种植被次网格聚合方案 (LCT、PC和PFT) 示意图]{CoLM三种植被次网格聚合方案 (LCT、PC和PFT) 示意图。具体聚合过程见章节~\ref{sec:次网格聚合}}
\label{fig:次网格聚合方案}
\end{figure}
}

由于PFT和PC方案都需要对细网格地表覆盖数据进行PFT分解,而地表覆盖原数据并不包含PFT植被结构属性信息,需要借助辅助数据进行计算,其数据来源如表~\ref{tab:网格划分辅助数据} 所列。

% Please add the following required packages to your document preamble:
% \usepackage{booktabs}
% \usepackage[table,xcdraw]{xcolor}
% If you use beamer only pass "xcolor=table" option, i.e. \documentclass[xcolor=table]{beamer}
\begin{landscape}
\begin{table}[htbp]
%\begin{sidewaystable}[]
\centering
\caption{用于PFT和PC次网格划分辅助数据}
\label{tab:网格划分辅助数据}

\begin{tabular}[h]{p{3cm}p{6cm}p{2cm}p{2cm}p{5cm}}
\toprule
数据名称      & 描述        & 分辨率   & 坐标投影     & 参考文献           \\ \midrule
MODIS Land Cover Type & The Terra and Aqua combined MODIS Land Cover Type (LC\_Type1) data product, Version 6 (MCD12Q1) & 500 m & Sinusoidal        & \citet{Friedl2019}                \\\midrule
ESA CCI-LC land cover type         & ESA CCI land cover type products                                                                & 300 m & Latitude-Longitude & \url{http://maps.elie.ucl.ac.be} \\\midrule
MODIS VCF      & MODIS Vegetation Continuous Fields product, Version 6 (MOD44B)                                           & 250 m    & Sinusoidal         & \citet{DiMiceli2015}                        \\\midrule
AVHRR VCF& AVHRR Tree Cover (evergreen, deciduous and broadleaf, needleleaf) Continuous Fields                      & 1 km  & Latitude-Longitude & \citet{defries2000new}                         \\\midrule
Köppen-Geiger climate classification        & Present Köppen-Geiger climate classification maps at 1-km resolution                            & 1 km  & Latitude-Longitude & \citet{beck2018}     \\\midrule
Reprocessed MODIS LAI                 & Reprocessed MODIS leaf area index produces, Version 6                                           & 500 m & Sinusoidal         & \citet{yuan2011reprocessing,lin2023ReprocessedMODISVersion}     \\\midrule
WorldClim Version2                    & Worldclim 2: New 1-km spatial resolution climate surfaces for global land areas                          & 1 km  & Latitude-Longitude & \citet{fick2017worldclim}   \\\midrule
Canopy height                         & Global 1 km forest canopy height map                                                            & 1 km  & Latitude-Longitude & \citet{simard2011mapping}             \\ \bottomrule
\end{tabular}
%\end{sidewaystable}
\end{table}
\end{landscape}

ESA CCI PFT v2.0.8全球植被功能类型数据是由欧洲宇航局 (ESA) 的气候变化计划 (CCI) 提供  (\url{https://doi.org/10.5285/26a0f46c95ee4c29b5c650b129aab788}),空间分辨率300 m,时间覆盖1992--2020年。该产品将提供了300米网格广泛特征的ESA CCI地表覆盖数据进行细化,借助现有的高分辨率数据集(主要有树冠高度、树冠覆盖、城市、水体数据),描述了共14种PFT在网格内的覆盖度,包括:常绿阔叶树、落叶阔叶树、常绿针叶树、落叶针叶树、常绿阔叶灌木、落叶阔叶灌木、常绿针叶灌木、落叶针叶灌木、天然草、草本作物、城市、水体、裸地、冰川和雪盖。

如采用PFT/PC方案进行植被次网格模拟,其不同类型相关参数请参考附录~\ref{植被功能型PFT相关参数} 和附录~\ref{植物群落PC次网格PFT相关参数}。PFT/PC植被结构数据制作流程及植被属性尺度转换详见章节~\ref{植被尺度转换}。


\section{水文数据}\label{水文数据}

\subsection{水文学基础数据}
水文学基础数据主要用于流域单元网格,CoLM中使用高程、水流方向、集水面积和湖泊范围等变量。

水流方向和集水面积数据来自MERIT Hydro~\citep{yamazaki2019merit}。 MERIT Hydro为全球格点数据,分辨率为3弧秒(赤道处约90米),由高程数据(MERIT DEM)和水体数据集(G1WBM、GSWO和OpenStreetMap)融合生成。MERIT Hydro使用算法识别和分辨实际的内陆流域和高程数据中误差引起的伪洼地,可近于自动地提取河网。除自动算法外,也进行了一定量的人工检查和编辑。评估结果表明,MERIT Hydro构建的水文学数据在集水面积和流域形状方面与其他经过质量控制的河网数据集显示出良好的一致性。

\subsection{河网}
CaMa-Flood水文水动力模拟所需的相关次网格地形参数是由90米超高精细分辨率的水文基础数据集 MERIT-Hydro ~\citep{yamazaki2019merit},
水文调整高程DEM数据 ~\citep{yamazaki2017high,yamazaki2012analysis},以及河宽数据~\citep{yamazaki2014development}; 
使用 \citet{yamazaki2009deriving} 开发的升尺度模式 Flexible Location of Waterways (FLOW) 计算得出。目前CaMa-Flood已经制备了01min(弧分)、03min、05min、06min、15min、30min和60min下运行所需的河网基础数据。相关数据以纯``二进制''格式 ($nx\times ny$)存储于 \texttt{map/} 目录下。默认的数据存储顺序是从180 \textdegree W到180 \textdegree E,从 90 \textdegree N 到 90 \textdegree S;数据的字节顺序是`little endian'。具体而言,除了需要陆面模式计算得出的产流量 (runoff) 之外,CaMa-Flood模块还需要包括流域面积 ($A_s$),河道长度 ($L$),河道宽度 ($W$) 及深度 ($B$) 以及用于计算平均漫滩水深的漫滩高程剖面等相关的次网格地形数据。除了河道横截面的宽度和深度数据以外,其他地形相关参数均可由 MERIT-hydro 直接计算得出。以下以全球~0.25\textdegree ~的河网为例进行详细介绍。


在目录\texttt{map/glb\_15min}下包含了全球 0.25\textdegree 分辨率的网格-矢量-混合 (grid-vector-hybrid) 河网数据 (图~\ref{fig:流域单元分布图});具体文件内容如表~\ref{tab:河网图及地形参数文件列表} 和表~\ref{tab:河网图及地形参数文件列表2} 所示。 河网地图的维度信息,包括东西方向网格数目$nx$,南北方向网格数目$ny$,泛滥平原层数 ($nfpl$),
网格大小 ($gsize$)和区域边界 (西、东、南、北),记录在 \texttt{params.txt} 之中。\texttt{nextxy.bin} 文件包含了2个参数 ($nextx$和$nexty$),
分别记录了每个网格的下游单元的相对位置,
其中与海洋接口的河口标记为-9,内陆河流终点标记为-10,海洋 (未定义) 标记为 -9999。

% Please add the following required packages to your document preamble:
% \usepackage{booktabs}
\begin{table}[htbp]
\centering
\caption{河网图及地形参数文件列表}
\label{tab:河网图及地形参数文件列表}
    \begin{tabular}[h]{p{3.5cm}p{1.5cm}p{1.5cm}p{5cm}p{1cm}p{1cm}}  %{@{}cccccc@{}}
    \toprule
    File              & Variable & Symbol                        & Description                                  & Unit    & Format  \\ \midrule
    \texttt{params.txt}        & -        & -                             & Map parameters                     & -         & text    \\
    \texttt{nextxy.bin}        & \texttt{nextx}    & $jx$                         & Downstream X (rec=1)         & -         & integer \\
                                       & \texttt{nexty}    & $jy$                         & Downstream Y (rec=2)            & -        & integer \\
    \texttt{downxy.bin}      & \texttt{downx}   & $dx$                        & Relative DownstreamX (rec=1)   & -   & integer \\
                                       & \texttt{downy}   & $dy$                      & Relative Downstream Y (rec=2)   & -    & integer \\
    \texttt{ctmare.bin}       & \texttt{ctmare}   & $Ac$                     & Unit-catchment Area                     & $\rm m^2$   & real    \\
    \texttt{elevtn.bin}        & \texttt{elevtn}    & $Z$                        & Base Elevation                         & m       & real    \\
    \texttt{rivlen.bin}         & \texttt{rivlen}    & $L$                         & Channel Length                        & m       & real    \\
    \texttt{rivhgt.bin}         & \texttt{rivhgt}   & $B$                         & Channel Depth                          & m       & real    \\
    \texttt{rivwth.bin}        & \texttt{rivwth}   & $W$                        & Channel Width                           & m       & real    \\
    \texttt{rivwth\_gwdlr.bin} & \texttt{rivwth}   & $W$                    & Combined Width (recommended)    & m       & real    \\
    \texttt{nxtdst.bin}        & \texttt{nxtdst}   & $X$                             & Downstream Distance                          & m       & real    \\
    \texttt{fldhgt.bin}        & \texttt{fldhgt}   & $D_f$                            & Floodplain Elevation profile (rec=1$\sim$10) & m       & real    \\
    \texttt{rivman.bin}        & \texttt{rivman}   & -                             & Manning’s Roughness                          & -       & real    \\
    \texttt{bifori.txt}        & -        & -                             & Bifurcation Channel Original Data            & -       & text    \\
    \texttt{bifprm.txt}        & -        & -                             & Bifurcation channel parameters               & -       & text    \\ \bottomrule
    \end{tabular}
\end{table}


    % Please add the following required packages to your document preamble:
% \usepackage{booktabs}
\begin{table}[htbp]
\centering
\caption{河网图及地形参数文件列表续}
\label{tab:河网图及地形参数文件列表2}
    \begin{tabular}[h]{p{3.5cm}p{1.5cm}p{1.5cm}p{5cm}p{1cm}p{1cm}} %{@{}cccccc@{}}
    \toprule
    File             & Variable & Symbol                             & Description                                          & Unit     & Format  \\ \midrule
    \texttt{grdare.bin}       & \texttt{grdare}   & -                                  & Rectangular grid area (optional)                     & \unit{m^2}       & real    \\
    \texttt{nxtdst\_grid.bin} &          &                                    &                                                      &          &         \\
    \texttt{rivlen\_grid.bin} & \texttt{nxtdst}   & $X$                                  & Downstream Distance (grid center)                    & m        & real    \\
                                        & \texttt{rivlen}    & $L$                            & Channel Length (grid center)       &    m                             & real        \\
    \texttt{inpmat.bin}       & \texttt{inpx}     & -                                  & Corresponding input grid X (rec=1) & -        & integer \\
                                       & \texttt{inpy}     & -                               & Corresponding input grid Y (rec=2)    & -        & integer  \\
    \texttt{ctmare.bin}       & \texttt{inpa}     & $A_{ij}$                            & Area of input grid XY (rec=3)              & \unit{m^2}     & real    \\
    \texttt{diminfo.txt}      & -        & -                                  & Dimension information                         & text     &         \\
    \texttt{lsmask.bin}       & -        & -                                  & Land ID of corresponding hi resolution area Basin ID & -        & integer \\
    \texttt{basin.bin}        & -        & -                                  & Basin ID                                             & -        & integer \\
    \texttt{bsncol.bin}       & -        & -                                  & Basin Color Pattern for Visualization                & -        & integer \\
    \texttt{lonlat.bin}       & \texttt{lon}      & -                                  & Longitude, catchment outlet (rec=1)     & \textdegree      & real    \\
                                     & \texttt{lat}       & -                                 & Latitude, catchment outlet (rec=2)        & \textdegree       & real    \\
    \texttt{uparea.bin}       & \texttt{uparea}   & -                           & Upstream Drainage Area                       & \unit{m^2}        & real    \\ 
    \texttt{outclm.bin}       & \texttt{outclm}   & -                           & Averaged Discharge (for channel parameters)                       & \unit{m^3 s^-1}        & real    \\  \bottomrule
    \end{tabular}
\end{table}

{
\begin{figure}[htbp]
\centering
\includegraphics[width=1.0\textwidth]{Figures/陆地表面的水分循环/CaMa-Flood流域单元分布图.png}
\caption[CaMa-Flood流域单元分布图]{CaMa-Flood所使用的流域单元分布图。每个流域单元的出口用白色圆点标记。红色线条连接不同流域单元,箭头指向下游流域单元。蓝色代表真实河道(源自于90米MERIT-Hydro数据)}
\label{fig:流域单元分布图}
\end{figure}
}

假设任意网格的洪水事件是从高程低的区域到高程高的区域发生淹没,
任意流域单元(图~\ref{fig:基于流域单元的漫滩高程剖面函数}) 的漫滩面积可以由漫滩水深$D_f$ (m) 
以及累积分布函数 (CDF) 的经验函数生成 (图~\ref{tab:河网图及地形参数文件列表2})。
每个网格的CDF函数的第10百分位数的10个值 (图~\ref{tab:河网图及地形参数文件列表2},红色圆点)分别存储在 \texttt{fldhgt.bin} 的十个参数之中
 (见表~\ref{tab:河网图及地形参数文件列表} 和表~\ref{tab:河网图及地形参数文件列表2}),文件格式与上述的 \texttt{nextxy.bin} 相同。
例如,\texttt{fldhgt.bin} 的第三个参数表示流域单元 30\% 的面积被淹没时的该流域单元的平均洪水深度 (m)。

{
\begin{figure}[htbp]
\centering
\includegraphics[width=1.0\textwidth]{Figures/陆地表面的水分循环/基于流域单元的漫滩高程剖面函数.png}
\caption{基于流域单元的漫滩高程剖面函数。}
\label{fig:基于流域单元的漫滩高程剖面函数}
\end{figure}
}


河道断面参数(河道长度$W$ (m)和河道深度 $B$ (m))是水动力模拟的主要参数之一。在CaMa-Flood中的该参数是流域产流量的气候态的经验函数,由以下公式推导得出:
\begin{equation}
W=\max \left(W_{min},\ c_{w} * Q_{ave}{}^{p_{w}}+W_{0}\right)
\end{equation}
\begin{equation}
B=\max \left(B_{min}, c_{H} * Q_{ave}{}^{p_{H}}+B_{0}\right)
\end{equation}
其中$Q_{ave}$是由陆面模式计算得出的产流量 (total runoff) (\unit{m^3.s^{-1}})。
河道宽度计算所使用的经验系数 $c_w$,$p_w$,$W_0$,$W_{min}$ 分别设置为 2.5,0.6,0.0 以及5.0;
河道深度计算所使用的经验系数 $c_H$,$p_H$,$B_0$ 和 $B_{min}$ 分别设置为 0.1,0.5,0.0 以及1.0。
值得注意的是这些经验系数存在着极大的不确定性,在使用之时需要对其进行对应的率定和调参。
除此之外,基于卫星观测的河道宽度数据 (\texttt{width.bin},如图~\ref{fig:基于卫星数据生成的河道宽度示意图}) 也可由MERIT-hydro水文地形数据中获取。
CaMa-Flood推荐使用通过 \texttt{set\_gwdlr.F90} 子程序生成组合宽度参数 (经验公式和卫星数据的组合) 数据。
相关地图的制备详见 \citet{yamazaki2014regional, yamazaki2014development}。

{
\begin{figure}[htbp]
\centering
\includegraphics[width=1.0\textwidth]{Figures/陆地表面的水分循环/基于卫星数据生成的河道宽度示意图.png}
\caption{基于卫星数据生成的河道宽度示意图}
\label{fig:基于卫星数据生成的河道宽度示意图}
\end{figure}
}

如图~\ref{fig:流域单元分布图} 所示,由于 CaMa-Flood 是基于流域单元进行计算的,而陆面模式是基于规则的经纬度网格,
流域单元在当前陆面模式分辨率 (经纬度网格) 投影下呈现不规则的形状。
因此在流域边缘有可能出现一个网格分属于不同流域,以及多个经纬度网格单元同属于一个流域单元的情况。
这就需要强制插值分割这些陆面模式经纬度网格生成的产流量到不同的流域。
指定流域单元$i$从陆面模式网格获得的水量由如下公式进行计算:
\begin{equation}
F_{i}=\sum_{N} A_{i, j} R_{j}
\end{equation}
其中$F_i$为进入流域单元$i$在单位时间 [\unit{m^3.s^{-1}}]的输入水量,$A_{i, j}$ 是陆面模式的经纬度网格单元$j$在属于指定流域单元的面积大小,
 $N$是指属于指定流域单元陆面模式的经纬度网格的数量。该映射信息被存储在 \texttt{test\_***.bin} (\texttt{***} 是网格分辨率
 ,例如 \texttt{test\_1deg.bin})。同时,还需准备一个文本文件 (比如说 \texttt{diminfo\_test-1deg.txt}) 用来指定模拟的维度
  (比如说区域,分辨率,流域单元数量,输入经纬度网格数量,输入的映射文件名 \texttt{test\_***.bin})。

在 \texttt{map} 目录中,CaMa-Flood 还准备了一些用于生成映射矩阵和漫滩高程剖面曲线所需的高分辨率数据 (即 \texttt{3sec},\texttt{15sec}和\texttt{1min})。高分辨率数据被划分为\texttt{\$(TILE).\$(VAR).bin},每个TILE的区域记录在 \texttt{location.txt} 文件之中;
具体提供的参数 \texttt{\$(VAR)} 如表~\ref{tab:河网图及地形参数文件列表2} 所示。其中 \texttt{\$(TILE).flwdir.bin} 和
\texttt{\$(TILE).downxy.bin} 分别以基于 D8 格式和 downstreamXY 格式描述了河流流向;\texttt{\$(TILE).catmzy.bin} 表示流域单元内的网格 (ix, iy) 
在高分辨率经纬度网格 (iXX, iYY)上的映射;\texttt{\$(TILE).catmzz.bin} 表示每个网格对应的漫滩层;\texttt{\$(TILE).flddif.bin} 表示每个网格堤坝平均高度 [m],
用于对粗分辨率的漫滩水深进行降尺度;\texttt{\$(TILE).visual.bin} 用于高分辨率流域边界可视化,其中数值上体现为海洋 = 0, 
陆地 (未调整) = 1, 陆地 (调整并在 CaMa 中使用) = 2, 通常网格 = 3,流域边界 = 5, 河道 = 10, 内陆流域出口 = 20, 与海洋连接的河口 = 25。
具体文件列表见表~\ref{tab:河网各数据文件1}。

% Please add the following required packages to your document preamble:
% \usepackage{booktabs}
\begin{table}[htbp]
    \centering
    \caption{河网各数据文件}
    \label{tab:河网各数据文件1}
    \begin{tabular}[h]{p{4cm}p{1.5cm}p{1.5cm}p{4cm}p{1cm}p{2cm}} %{@{}cccccc@{}} %
    \toprule
    File                & Variable    & Symbol                        & Description                   & Unit      & Format \\ \midrule
    \texttt{location.txt}        & -        & -                             & Hi-res domain info            & -         & text   \\
    \texttt{\$(TILE).catmxy.bin} & \texttt{catmx}    & $iXX$         & catchment (iXX,jYY) rec=1     & - & int*2byte       \\
                                               & \texttt{catmy}    & $jYY$      & catchment (iXX,jYY) rec=2     & - & int*2byte      \\
    \texttt{\$(TILE).catmzz.bin} & \texttt{catmz}    & -                 & floodplain layer              & -  & int*1byte \\
    \texttt{\$(TILE).flwdir.bin} & \texttt{dir}      & -                        & flow direction (D8)         & -   & int*1byte  \\
    \texttt{\$(TILE).downxy.bin} & \texttt{downx}    & $dx$          & relative downstream x (rec=1) & - & int*2byte   \\
                                                & \texttt{downy}    & $dy$       & relative downstream y (rec=2) & - & int*2byte   \\
    \texttt{\$(TILE).elevtn.bin}    & -        & -                             & elevation                     & m                             & real    \\
    \texttt{\$(TILE).flddif.bin}      & -        & -                             & height above river channel    & m                             & real   \\
    \texttt{\$(TILE).rivwth.bin} & -        & -                                & river channel width           & m                             & real   \\
    \texttt{\$(TILE).grdare.bin} & -        & -                               & pixel area                    & \unit{m^2}                            & real   \\
    \texttt{\$(TILE).uparea.bin} & -       & -                               & upstream drainage area        & \unit{m^2}       & real   \\
    \texttt{\$(TILE).visual.bin}  & -        & -                               & Catchment visualization        & -                                                        & int*1byte       \\ \bottomrule
    \end{tabular}
\end{table}


\subsection{水库}
模式中水库调度模块所用的水库基础数据主要包括:全球水库基本属性数据(GRanD)、全球水库水面面积数据(GRSAD)和全球水库几何数据(ReGeom),如表~\ref{tab:水库模块所用的水库基础数据} 所示,根据这些数据可以进一步得到:
\begin{enumerate}
\item 与不同分辨率河网数据相匹配的水库位置信息;
\item 水库调度模拟所需的特征流量和特征库容数据;
\item 与水库相对应的蓄水网格信息;
\end{enumerate}

\begin{table}[htbp]
    \centering
    \caption{水库模块所用的水库基础数据}
    \label{tab:水库模块所用的水库基础数据}
    \begin{tabular}{ccccc}
    \toprule
    数据集 & 数据描述 & 水库数量 & 时间分辨率 & 数据源 \\ \midrule
    GRanD  & \text{\makecell{全球水库\\基本属性数据}} & \text{\makecell{7320 - v1.3\\6862 - v1.1}} & /    & \text{\makecell{https://doi.org/\\10.7927/H4N877QK\\(\citep{lehner2011high})}} \\
    GRSAD  & \text{\makecell{全球水库\\水面面积数据}} & 6817 (GRanD v1.1) & \text{\makecell{1984-2015\\月尺度}} & \text{\makecell{https://doi.org/\\10.18738/T8/DF80WG\\(\citep{zhao2018automatic})}}  \\
    ReGeom & \text{\makecell{全球水库\\几何数据}} & 6868 (GRanD v1.1)  & / & \text{\makecell{https://doi.org/\\110.5281/zenodo.1322884\\(\citep{yigzaw2018new})}}  \\
    \bottomrule
    \end{tabular}
    \end{table}
    

\subsubsection{全球水库基本属性数据}
全球水库基本属性数据来自Global Reservoir and Dam Database (GranD; \citet{lehner2011high}),该数据集目前提供v1.1和v1.3两个版本。v1.1版本包含了全球范围内6862个水库及其关联大坝(图~\ref{fig:GranD-v1.1所包含水库的位置和主要用途})的基本属性数据,水库的累积库容为6197 km$^{3}$,水库数量较多、库容较大的区域主要集中在中国、美国、加拿大、印度和欧洲。v1.3版本包含的水库在v1.1版本的基础上增加到7320个,累积库容增加到6863.5 \unit{km^3}。GranD提供了水库及其关联大坝的地理位置和基本属性信息(对于其包含的大部分水库)。其中,水库模块(章节 \ref{水库模式})所用属性主要包括水库编码(\texttt{GRAND\_ID})、水库名称(\texttt{DAM\_NAME})、经度(\texttt{LONG\_DD})、纬度(\texttt{LAT\_DD})、水库总库容(\texttt{CAP\_MCM})、汇流面积(\texttt{CATCH\_SKM})、主要用途(\texttt{MAIN\_USE})、建坝年份(\texttt{YEAR})等。

{
\begin{figure}[htbp]
\centering
\includegraphics{Figures/基础数据/GranD-v1.1所包含水库的位置和主要用途.png}
\caption{GranD-v1.1所包含水库的位置和主要用途}
\label{fig:GranD-v1.1所包含水库的位置和主要用途}
\end{figure}
}


\subsubsection{全球水库水面面积数据}
全球水库水面面积数据来自Global Reservoir Surface Area Dataset(GRSAD),该数据集提供了全球6871个水库1984年至2015年的逐月水库面积变化数据(与GranD水库匹配)。该数据是基于Global Surface Water Dataset(GSWD)数据修正而得。\citet{pekel2016high} 使用1984$\sim$2015年300万张Landsat卫星图像创建了全球地表水数据集GSWD。但由于云、云阴影等因素的影响,利用Landsat对全球范围内单个水库进行长期水面变化分析会存在较大偏差。GRSAD修复了由于云层、云层阴影、地形阴影和扫描线校正器故障导致的像素污染,使每个时间序列中有效图像的数量平均提高了81\%~\citep{zhao2018automatic}。基于水库面积长期动态变化数据,可估算水库的防洪库容面积、正常库容面积和死库容面积。


\subsubsection{全球水库形状数据}

全球水库形状数据来源于Global Reservoir Geometry Database(ReGeom),该数据集包含了全球6824个水库的库容-面积-深度数据(与GranD水库匹配)。\citet{yigzaw2018new} 采用一种优化算法来确定每个水库的库容-面积-深度关系,该算法从五种可能的规则几何形状中迭代选择出最佳几何形状,以最小化水库总库容和表面积的估算误差。ReGeom数据集适用范围广泛,尤其适用于改进全球水文或地球系统模型中对水库动态的表达,特别是在水库信息有限或不精确的地区。数据集中包含了所有水库的近似几何数据(\texttt{ReGeomData\_WOW\_V1.xlsx})和每个水库的库容--面积--深度数据(\texttt{*.csv})。通过每个水库的防洪库容面积、正常库容面积和死库容面积,查找相应的库容--面积--深度数据,即可得到对应的防洪库容、正常库容和死库容。


\subsection{湖泊}
HydroLAKES提供了全球所有面积大于10公顷的湖泊或水库岸线多边形数据\citep{messager2016nc}。原HydroLAKE数据为矢量数据,在CoLM中,将其转化为了与MERIT Hydro相同的3弧秒分辨率的格点数据进行使用。

湖泊深度数据来自Lake-depth data set Version 2.0~\citep{kourzeneva2012global},数据为全球\ang{;;30} (约1公里)分辨率,经纬度网格,单时次数据。


\section{城市数据}\label{城市数据}
城市模式所需基础数据主要包含城市覆盖、城市类型、建筑形态、辐射与热力属性、生态数据以及人类活动相关数据。由于城市内部强异质性且近年来城市化迅速发展,为体现更为精细化的城市内部空间属性特征和其时空变化,城市模式基础数据在选取时主要遵循高分辨率、多时次的原则,通过以下两方面完成城市基础城市数据制作:

\begin{enumerate}
\item 集成目前已有的高分辨率数据。搜集目前已经发布的高分辨率数据,通过一定处理使其适用于城市模式;
\item 自主研制部分高分辨率(主要为城市LAI/SAI)。由于部分数据的缺失,通过多源数据发展适用于城市模式的基础数据;
\end{enumerate}
目前城市模式所用基础数据如表~\ref{tab:城市数据分类及来源} 所示。
{
\begin{landscape}
\begin{table}[htbp]
    \centering
    \caption{CoLM城市模式基础数据列表}
    \label{tab:城市数据分类及来源}
    \begin{tabular}{ccccc>{\centering\arraybackslash}p{5.5cm}}
    \toprule
    数据类型 & 数据名称 & 分辨率 & 时间范围 & 参考文献\\ \midrule
    城市覆盖 & MODIS Land Cover Type & 500m & 2000--2020 & \citet{Friedl2019}  \\
    \hline
    \multirow{2}*{城市类型}  & NCAR Urban Density Class & 1km & 单时次 & \citet{jackson2010parameterization} \\
                    & LCZ Class  & 100m & 单时次 & \citet{demuzere2022global} \\
    \hline
    \multirow{4}*{建筑形态数据}  & \makecell{建筑高度\\建筑比例} & 1km & 单时次 & \citet{li2022global} \\
                    % & 透水面/不透水面比例 & / & / & 待补充 \\
                    % & 街谷高宽比 & / & / & 待补充 \\
                    % & 屋顶/墙面厚度 & / & / & 待补充 \\
                    & 透水面/不透水面占比 & \multirow{3}*{/} &  \multirow{3}*{/} & \multirow{3}*{\makecell{\citet{oleson2020parameterization}\\\citet{stewart2014evaluation}}} \\
                    & 街谷高宽比 \\
                    & 屋顶/墙体厚度  \\
    \hline
    \multirow{4}*{建筑辐射与热力数据} & 屋顶/墙体/透水面/不透水面反照率 & \multirow{4}*{/} &  \multirow{4}*{/} & \multirow{4}*{\makecell{\citet{oleson2020parameterization}\\\citet{stewart2014evaluation}}} \\
                                   & 屋顶/墙体/透水面/不透水面发射率 \\
                                   & 屋顶/墙体/不透水面比热容 \\
                                   & 屋顶/墙体/不透水面热导率 \\
                                   % & 屋顶/墙面/透水面/不透水面发射率 & / & / & 待补充 \\
                                   % & 屋顶/墙面/不透水面比热容 & / & / & 待补充 \\
                                   % & 屋顶/墙面/不透水面热导率 & / & /& 待补充 \\
    \hline
    \multirow{4}*{城市生态数据} & 城市树LAI & 500m & 2000--2020 & / \\
                              & 城市树高 & 10m & 单时次 & \citet{lang2023high}  \\
                              & 城市树覆盖度 & 30m & 2000,2005,2010,2015 & \citet{townshend2016gfcc} \\
                              & 城市水体覆盖度 & 30m & 2000,2010,2020 & \citet{陈军2017} \\
    \hline
    \multirow{2}*{城市人类活动数据} & LandScan人口密度 & 1km & 2000--2020 & \citet{dobson2000landscan} \\
                                 & LUCY国家分类 & 5km & / & \citet{allen2011} \\
    \bottomrule
    \end{tabular}
\end{table}
\end{landscape}
}
\subsection{城市覆盖}\label{城市覆盖}
城市地表覆盖数据采用MODIS-IGBP地表覆盖数据(章节~\ref{IGBP地表覆盖数据})中的城市覆盖类型作为判别依据。
模式利用该数据集区分不同年份网格城市覆盖类型,获得城市在网格中的覆盖度,反应不同年份的城市覆盖变化。


\subsection{城市类型}\label{城市类型}
模式对城市覆盖地区不同城市类型进行细分。城市类型分类数据目前采取两种方案:
\begin{enumerate}
    \item 传统的3类城市——高建筑街区 (Tall Building Distinct, TBD)、高密度(High Density, HD) 和中密度 (Medium Density, MD);
    \item 10类Local Climate Zone (LCZ 1-10)分类,如图~\ref{fig:LCZ分类图示}  所示。
\end{enumerate}

{
\begin{figure}[htbp]
\centering
\includegraphics[width=\textwidth]{Figures/基础数据/LCZ分类图示.jpg}
\caption{LCZ分类示意图~\citep{demuzere2020combining,stewart2012local}}
\label{fig:LCZ分类图示}
\end{figure}
}

\subsubsection{传统城市分类}\label{传统城市分类}
模式对于传统的城市类型分类与NCAR CLM--Urban Model~\citep{oleson2020parameterization} 相同,
根据 \citet{jackson2010parameterization} 的城市分类数据对全球城市种类进行分类。该数据将城市地表覆盖进一步划分为TBD、HD和MD三种类型,其中,TBD类城市指至少在1平方公里内,建筑高度全部大于或等于10层楼高且透水面比例较小(5$\sim$15\%);HD类城市建筑高度为3$\sim$10层,透水面比例通常为5$\sim$25\%,这些地区通常为商业、住宅或工业区;MD类城市建筑高度通常为1$\sim$3层,透水面比例约为20$\sim$60\%。原数据由1km重采样到500m后,将城市类型赋予对应格点的MODIS城市地表。
\subsubsection{LCZ分类}\label{LCZ分类}
LCZ数据目前使用 \citet{demuzere2022global} 发布的全球LCZ分类数据,该数据空间分辨率为100米,通过向轻量级随机森林模型输入大量标记训练区域和地球观测图像来生成,并对150个选定的功能城市地区采用自助交叉验证和专题基准测试评估了其质量。将原始数据由100m聚合到500m,并把占比最高的LCZ作为500m网格的LCZ类型,进而将LCZ类型赋予对应格点的MODIS城市地表。


\subsection{建筑形态数据}\label{建筑形态数据}
\subsubsection{建筑高度}\label{建筑高度}
建筑高度采用\citet{li2022global}发布的数据,该数据利用随机森林模型使用世界各地的参考数据进行训练得到。为了增加代表性,\citet{li2022global}对参考数据进行了人工分类和补充。该数据分辨率为1km,单时次数据。
\subsubsection{建筑比例}\label{建筑比例}
建筑比例同样采用\citet{li2022global}发布的数据,相比于其他陆面模型使用的根据区域或者LCZ类型查找表赋值的方法,该数据精度较高,能更加真实的还原城市内部建筑的分布以及异质性。

\subsubsection{透水面占比}\label{透水面占比}
对于传统城市分类,城市的透水面占比通过33个国家/地区分类以及3种城市类型查找表获得,国家/地区分类由~\citet{oleson2020parameterization}数据得到,其全球分类如图~\ref{fig:地区分类} 所示。
LCZ同样根据LCZ类型查找表,形态参数设置参考~\citet{stewart2014evaluation},如表~\ref{tab:lcz局地气候区建筑属性参数} 所示。

\subsubsection{街谷高宽比}\label{街谷高宽比}
同透水面占比,传统城市分类通过33个国家/地区分类以及3种城市类型查找表获得,LCZ分类则根据LCZ类型查找表设置(附录表~\ref{tab:lcz局地气候区建筑属性参数})。

\subsubsection{屋顶/墙体厚度}\label{屋顶/墙体厚度}
屋顶/墙体厚度同样基于查找表,传统城市分类通过33个国家/地区分类以及3种城市类型查找表获得,LCZ分类则根据LCZ类型查找表设置(附录表~\ref{tab:lcz局地气候区建筑属性参数})。

{
\begin{figure}[htbp]
\centering
\includegraphics[width=.7\paperwidth]{Figures/基础数据/地区分类.jpg}
\caption{地区分类示意图}
\label{fig:地区分类}
\end{figure}
}

\subsection{建筑辐射及热力数据}\label{建筑辐射及热力数据}
对于传统城市分类,建筑辐射及热力参数目前根据\citet{oleson2020parameterization}基于\citet{jackson2010parameterization}的建筑数据开发的NCAR城市工具
(Toolbox for Human-Earth SystemIntegration \& Scaling (THESIS) toolset, \url{http://www.cgd.ucar.edu/iam/projects/thesis/thesis-urbanproperties-tool.html})得到,该工具通过不同地区建筑材料构成为全球33个分区及3类城市生成建筑辐射及热力参数表。
LCZ分类建筑辐射参数参考~\citet{stewart2014evaluation},热力参数则参考WRF设置,如附录表~\ref{tab:lcz局地气候区建筑属性参数} 所示。

\subsubsection{建筑反照率}\label{建筑反照率}
对于传统城市分类,反照率(包括屋顶、墙面、不透水面和透水面)通过33个国家/地区分类以及3种城市类型查找表获得;LCZ分类则根据LCZ类型查找表(附录表~\ref{tab:lcz局地气候区建筑属性参数})。

\subsubsection{建筑发射率}\label{建筑发射率}
反射率设置同反照率,包括屋顶、墙面、不透水面和透水面,对于传统城市分类,建筑反照率通过33个国家/地区分类以及3种城市类型查找表获得;LCZ分类则根据LCZ类型查找表(附录表~\ref{tab:lcz局地气候区建筑属性参数})。

\subsubsection{建筑比热容}\label{建筑比热容}
建筑比热容(包括屋顶、墙壁和不透水面)同样基于查找表,对于传统城市分类,建筑反照率通过33个国家/地区分类以及3种城市类型查找表获得;LCZ分类则根据LCZ类型查找表(附录表~\ref{tab:lcz局地气候区建筑属性参数})。

\subsubsection{建筑热导率}\label{建筑热导率}
建筑热导率同建筑比热容,包括屋顶、墙壁和不透水面,对于传统城市分类,建筑反照率通过33个国家/地区分类以及3种城市类型查找表获得;LCZ分类则根据LCZ类型查找表(附录表~\ref{tab:lcz局地气候区建筑属性参数})。

\subsection{城市生态数据}\label{城市生态数据}
现有的城市模式数据集主要关注城市建筑形态和物理性质,但是缺少关于植被(树)和水体(即城市生态功能)的描述,LCZ分类虽然有植被描述但通常是一个范围值。由于城市只占全球陆地的1\%左右,相比于自然地表,在大尺度上城市占比仍然比较小,
因此在补充城市内部的植被覆盖数据时,需要尽量选择高分辨率的数据。否则,如果分辨率过低,城市中的植被等信息很可能无法识别。

\subsubsection{城市LAI/SAI数据}\label{城市LAISAI数据}
由于目前全球LAI数据大部分为光学遥感数据,分辨率为500 m甚至1 km(比如MODIS LAI),然而在城市中,由于城市中建筑物的遮掩以及混合像元的干扰,这种分辨率下卫星观测的城市LAI存在一定问题,但是作为描述植被的重要参数,LAI数据是不可或缺的。因此为了补充该数据,城市模式中树木的LAI采取插值方法计算得到,即根据格点内其它PFTs(只考虑树,即1-9类PFT)的LAI/SAI生成,具体方法为通过城市树高与PFTs树高比值插值并加权平均计算城市LAI/SAI,方法如下:
\begin{equation}
LAI_{urb}=\sum_{i=1}^{9} LAI_{PFT_{i}} \cdot \frac{HTOP_{urb}}{HTOP_{PFT_{i}}} \cdot PCT_{PFT_{i}}
\end{equation}
\begin{equation}
SAI_{urb}=\sum_{i=1}^{9} SAI_{PFT_{i}} \cdot \frac{HTOP_{urb}}{HTOP_{PFT_{i}}} \cdot PCT_{PFT_{i}}
\end{equation}
其中$LAI_{PFT_{i}}$($SAI_{PFT_{i}}$)为PFT的LAI(SAI),$HTOP_{urb}$和 $HTOP_{PFT_{i}}$分别为城市树高和PFT树高,$PCT_{PFT_{i}}$为PFT占比。$HTOP_{urb}$分辨率为500 m,PFT占比、LAI/SAI以及PFT树高均为0.5\textdegree{}分辨率,通过插值计算后便得到500m分辨率下LAI/SAI数据。

\subsubsection{城市树高}\label{城市树高数据}
CoLM自然植被树高采用~\citet{simard2011mapping}发布的全球树高数据,该数据虽然应用广泛,但是分辨率较低(1 km),无法满足城市模拟的需求,近年来,关于树高的高分辨率数据发展迅速,\citet{potapov2021mapping}利用GEDI卫星数据反演的树高数据具有较高的空间分辨率(30m,),但是由于卫星运行轨道限制,该数据目前只发布了$51.6°N$ $\sim$ $51.6°S$范围内的数据。\citet{lang2023high}则开发了一种从Sentinel-2光学图像中检索冠层高度的深度学习模型,该模型可以有效降低卫星观测树高偏低的问题,\citet{lang2023high}利用该模型并结合GEDI 30m树高数据生成了全球10m树高地图。相比于GEDI数据,该数据更全(全球),同时改善了GEDI偏低的问题,并且分别率更高(10m),因此将该数据集作为城市树高基础数据。

\subsubsection{城市树覆盖度}
GFCC Tree Cover (\url{https://lpdaac.usgs.gov/products/gfcc30tcv003/}) 提供了
2000年、2005年、2010年和2015年四个年代的全球树木覆盖度信息,分辨率为30 m,可用于了解森林变化。
该数据的投影方式为UTM,与陆面模式使用的等经纬度网格不同,需对数据进行投影转换,由UTM投影转换为经纬度投影 ($\sim$0.00025\textdegree)。
经过投影转换后,通过对全球数据每个区域的文件进行读取,并统计500 m 
 ($\sim$0.0041667\textdegree)分辨率下30 m高分辨网格植被总占比,
求平均得到500 m分辨率下城市格点的树覆盖数据作为城市模式的基础输入数据。

\subsubsection{城市水体覆盖度}\label{城市水体覆盖度}
城市水体覆盖度数据由GlobeLand30得到,2014年GlobeLand30发布2000和2010版,自然资源部于2017年启动对该数据的更新,目前,GlobeLand30 2020版已发布 (\url{http://www.globallandcover.com/})。
除极地地区外,GlobeLand30采用了UTM投影,因此需要进行投影转换,其数据覆盖如图~\ref{fig:GlobeLand30数据覆盖示意图} 所示(除2020年外,其他年份不包含极地地区数据)。
该数据时空分辨率高,且水体精度最高,达到了92.09\%~\citep{陈军2017},因此投影转换后计算得到的水体覆盖度作为城市模式的原始数据。

{
\begin{figure}[htbp]
\centering
\includegraphics{Figures/基础数据/GlobeLand30数据覆盖示意图.png}
\captionsetup{justification=centering}
\caption[GlobeLand30数据覆盖示意图]{GlobeLand30数据覆盖示意图, 摘自GlobeLand30官网(\url{http://www.globallandcover.com/})}
\label{fig:GlobeLand30数据覆盖示意图}
\end{figure}
}


\subsection{城市人类活动数据}\label{城市人类活动数据}

\subsubsection{人口密度数据}\label{人口密度数据}
人口密度数据目前使用由美国能源部橡树岭国家实验室(ORNL)发布的LandScan数据~\citep{dobson2000landscan},该数据采用地理信息系统与遥感影相结合的创新方法,在1km网格分辨率范围内,获取24小时内平均人口分布状况,LandScan是常用的全球人口动态统计分析数据集之一,该数据空间分辨率为1km,时间覆盖2000--2022年。

\subsubsection{LUCY国家分类}\label{LUCY国家分类}
LUCY国家分类数据来自~\citet{allen2011},根据该数据分配不同国家的汽车拥有量以及不同时刻的交通流量,用于交通热模拟,该数据分辨率为5km,单时次数据。


\section{作物数据}

\subsection{作物种类分布数据}\label{作物种类分布数据}
64种作物相对分布数据来自CTSM 5.2陆表数据集。然后按此比例,放入CoLM作物总的覆盖率中,得到每种作物随时间变化的分布。

\subsection{施肥数据}\label{施肥数据}
格点工业肥施肥率 (\unit{m^2.(g\ N)^{-1}.yr^{-1}}) 数据来自 LUH2 数据~\citep{hurtt2011harmonization},是每年变化的。\citet{lawrence2016land} 将其转换为每种作物类型的施肥率。

\subsection{播种时间数据}\label{播种时间数据}
播种日数据采用 GGCMI phase 3 全球作物播种日数据集~\citep{jagermeyr2021climate}。为当前多年平均的播种日数据,无时间变化,空间分辨率为 0.5\textdegree。

\subsection{灌溉方式数据}\label{灌溉方式数据}
灌溉模拟所需的灌溉方式数据来源于~\citet{yao2022Irrigation}全球灌溉方式地图数据,数据中共包括了32种灌溉作物的灌溉方式分布,每种作物在数据网格中仅考虑一种灌溉方式(滴灌、喷灌、漫灌)。~\citet{jagermeyr2015irrigation}采用决策树算法对AQUASTAT数据\citep{fao2014aquastat}进行了重处理,该算法根据不同作物灌溉面积、土壤类型、作物特性、社会经济等要素,划分了全球14种CFT的灌溉方式 (每种作物在网格上均可有3种不同的灌溉方式),分辨率为0.5\textdegree。\citet{yao2022Irrigation} 则进一步将该数据中14种CFT与CLM5中的32种灌溉作物CFT进行匹配,并对数据进行重采样,得到新的全球灌溉方式数据。

\section{生物地球化学数据}

\subsection{氮沉降数据}\label{氮沉降数据}
氮沉降是陆地生态系统的重要氮输入之一。近几十年来,人类活动造成大气氮沉降增加对生态系统生物地球化学循环产生了重要影响。由于区域地表氮沉降数据尚无法利用卫星直接观测得到,目前生物地球化学循环模式更多地利用大气模式或地球系统模式模拟得出。CoLM生物地球化学循环模块的氮沉降输入是$\mathrm{NO_y}$和$\mathrm{NH_x}$的总和,数据来源于CESM-WACCM的CMIP6历史时期和未来SSP情景模拟的集合平均。氮沉降历史时期数据包含1849年到2013年的时空变化,氮沉降未来时期数据包含1849--2101年的时空变化,未来情景包括SSP126,SSP245,SSP370和SSP585四种情景的预测模拟结果。数据的空间分辨率为0.9375\textdegree\ $\times$ 1.25\textdegree,数据在月尺度上刻画大气氮沉降的季节和年际变化。


\subsection{土壤含氧量数据}\label{土壤含氧量数据}



\subsection{闪电频率数据}\label{闪电频率数据}



\subsection{国民生产总值数据}\label{国民生产总值数据}



\subsection{泥炭地比例数据}\label{泥炭地比例数据}



\subsection{火灾峰值月份数据}\label{火灾峰值月份数据}



\subsection{地表臭氧浓度数据}\label{地表臭氧浓度数据}



%\end{地表输入数据}

\part{地表通量方案}{Surface Flux Schemes}\label{part:flux}
%\epart{Surface Flux Schemes}
\chapter{辐射通量方案}
%\addcontentsline{toc}{chapter}{辐射通量方案}
%\begin{辐射通量方案}

\begin{mymdframed}{代码}
章节~\ref{sec:土壤反照率} -- \ref{sec:改进的二流近似植被辐射传输模型} 对应的代码文件为\texttt{MOD\_Albedo.F90}。
\end{mymdframed}


\section{土壤反照率}\label{sec:土壤反照率}
土壤反照率采用陆面模式BATS方案 \citep{dickinson1986biosphere,dickinson1993biosphere},利用土壤颜色和土壤表层含水量计算得到,公式如下:
\begin{equation}\label{eq:soil_albedo1}
\alpha_{soi,vis,dir}=\alpha_{soi,vis,dif}=\min\left(\alpha_{sat, vis}+\max\left(0,0.11-0.40 \theta_{l}\right), \alpha_{dry, vis}\right)
\end{equation}
%
\begin{equation}
\alpha_{soi,nir,dir}=\alpha_{soi,nir,dif}=\min\left(\alpha_{sat, nir}+\max\left(0,0.11-0.40 \theta_{l}\right), \alpha_{dry, nir}\right)
\end{equation}
其中$\alpha$表示反照率,下标$soi$表示土壤,$vis$和$nir$分别表示可见光和近红外波段,$dir$($dif$)表示直射光(漫射光),$\alpha_{sat}$和$\alpha_{dry}$是根据土壤颜色分类,通过查找表得到的饱和($sat$)和干($dry$)土壤的反照率数值,$\theta_{l}$为表层(即第一层)土壤体积含水量。目前土壤颜色数据采用CLM 4.5,全球20种颜色分类数据,包括可见光波段和近红外波段。漫射光与直射光相应波段反射率查找表数值一致。以上土壤反照率的计算适用于植被(土壤)、湿地次网格,当城市模式未打开时,同样用于城市地区。


\section{永久性冰川反照率}\label{sec:永久性冰川反照率}
对于永久性冰川覆盖地表,简单将可见光波段反照率设置为0.8,近红外波段设置为0.55,直射光与漫射光一致,即:
\begin{equation}
\alpha_{ice,vis,dir}=\alpha_{ice,vis,dif}=0.8
\end{equation}
%
\begin{equation}
\alpha_{ice,nir,dir}=\alpha_{ice,nir,dif}=0.55
\end{equation}

\section{内陆水体反照率}
对于液态水和直射入射辐射,反照率为太阳天顶角余弦值$\mu$的函数,计算为$\frac{0.05}{\mu+0.15}$;漫射入射辐射时,设置为常数0.1,可见光与近红外波段一致,即:
%
\begin{equation}
\alpha_{wat,vis,dir}=\alpha_{wat,nir,dir}= \frac {0.05}{\mu+0.15}
\end{equation}
%
\begin{equation}
\alpha_{wat,vis,dif}=\alpha_{wat,nir,dif}=0.1
\end{equation}

当水体结冰,即水体表面温度小于273.16 K时,可见光波段反照率设置为0.6,近红外波段设置为0.4,直射光与漫射光一致:
%
\begin{equation}
\alpha_{wat,vis,dir}=\alpha_{wat,vis,dif}=0.6
\end{equation}
%
\begin{equation}
\alpha_{wat,nir,dir}=\alpha_{wat,nir,dif}=0.4
\end{equation}

\section{积雪反照率}

CoLM提供两种积雪反照率方案可供选择:1) BATS积雪反照率方案(章节~\ref{BATS积雪反照率});2) SNICAR积雪反照率方案(章节~\ref{SNICAR积雪反照率})。

\subsection{BATS积雪反照率方案}\label{BATS积雪反照率}

CoLM2014以及之前版本采用 \citet{dickinson1986biosphere} BATS方案,当不打开SNICAR积雪反照率模块时采用,主要考虑太阳高度角和雪龄 (snow age) 两个主要参数,计算如下:
\begin{equation}
\alpha_{sno, vis, dir}=\alpha_{sno, vis, dif}+0.4 f(\mu)\left(1-\alpha_{sno, vis, dif}\right)
\end{equation}
%
\begin{equation}
\alpha_{sno,nir,dir}=\alpha_{sno, nir, dif}+0.4 f(\mu)\left(1-\alpha_{sno,nir,dif}\right)
\end{equation} 
%
\begin{equation}\label{alpha_sno_vis_dif}
\alpha_{sno,vis,dif}=0.85\left(1-0.2 F_{age}\right)
\end{equation}
%
\begin{equation}\label{alpha_sno_nir_dif}
\alpha_{sno,nir,dif}=0.65\left(1-0.5 F_{age}\right)
\end{equation}
其中下标$sno$表示雪盖地表,$f(\mu)$是一用来表征雪盖反照率与太阳高度角关系的参数化函数:
\begin{equation}\label{fmu}
f(\mu)=\max\left[0,\left(\frac{1.5}{1+4 \mu}-0.5\right)\right]
\end{equation}
公式~\eqref{alpha_sno_vis_dif}、\eqref{alpha_sno_nir_dif} 中0.85和0.65分别表示可见光和近红外波段新雪反照率,
$F_{age}$表示雪龄影响因子,使得雪的反照率随着雪盖中雪晶颗粒的增长和灰尘等杂质的积累而降低,其计算为:
\begin{equation}
F_{a g e}=\frac{\tau_{sno}}{1+\tau_{sno}}
\end{equation}
$\tau_{sno}$表示无量纲雪龄,是一预报变量,它随时间的变化函数为:
\begin{equation}
\Delta \tau_{sno}=1 \times 10^{-6}\left(r_{1}+r_{2}+r_{3}\right) \Delta t
\end{equation}
其中$\Delta t$是积分时间步长(s),
\begin{align}
r_{1} &= \exp \left[5000\left(\frac{1}{273.16}-\frac{1}{T_{g}}\right)\right] \\
%\end{equation}
%\begin{equation}
r_{2} &= r_{1}^{10} \leqslant 1 \\
%\end{equation}
%\begin{equation}
r_{3} &= 0.3
\end{align}
上式中,$r_1$表示雪盖中水汽的扩散后凝华导致的雪晶颗粒增长对反照率的影响,$r_2$表示融雪对反照率的影响,$r_3$表示灰尘或杂质的影响[注: BATS在南极$r_3$设置为0.01]。
为了考虑新的降雪影响,在$\tau_{sno}$的计算中还包含计算时间步长内的降雪量,表示为:
\begin{equation}
\tau_{sno}^{(N+1)}=\left(\tau_{sno}^{(N)}+\Delta \tau_{sno}\right)\left(1-100 \Delta p_{sno}\right)
\end{equation}
其中$\Delta p_{sno}$为在$t^{(N+1)}$-$t^{(N)}$时间内的降雪量 (m),在模式中计算为雪水当量的变化量,$N$为积分的时间步数。


对于部分雪覆盖的地表,其反照率为雪覆盖度($f_{sno}$,参见章节~\ref{无植被覆盖地表湍流通量的计算方案}) 的加权函数,表示为:
\begin{equation}
\alpha_{g}=\left(1-f_{sno}\right) \alpha_{g} + f_{sno} \alpha_{sno}
\end{equation}
其中等式右边$\alpha_{g}$表示土壤、冰川或水体地表覆盖时的反照率,等式左边$\alpha_{g}$表示考虑积雪覆盖订正后的地表反照率。对直射光/漫射光,可见光/近红外波段均按上式计算。

\subsection{SNICAR积雪反照率方案}\label{SNICAR积雪反照率}
\begin{mymdframed}{代码}
本节对应的代码文件为\texttt{MOD\_SnowSnicar.F90}。
\end{mymdframed}

CoLM另一种积雪反照率及其对太阳辐射吸收的模拟,采用雪、冰和气溶胶辐射SNICAR (the Snow, Ice, and Aerosol Radiative) 模型 \citep{flanner2021SNICARADv3CommunityTool}。该模型利用大气沉积气溶胶(黑炭、矿物粉尘、有机碳)、雪的有效粒径、太阳天顶角(\(\mu_{0}\))以及雪层所覆盖下垫面的反照率,采用\citet{toon1989RapidCalculationRadiative}的二流辐射传输方案来计算积雪反照率和垂直吸收廓线(图~\ref{fig:SNICAR模型流程图})。

{
\begin{figure}[htbp]
\centering
\includegraphics[width=1\columnwidth]{Figures/辐射过程及辐射通量计算/SNICAR模型计算流程图.png}
\caption{SNICAR积雪反照率计算流程图\citep{he2020SnowAlbedoRadiative}}
\label{fig:SNICAR模型流程图}
\end{figure}
}

二流近似方案(two-stream
solution)需要使用到每个雪层和光谱波段的以下光学属性:光学厚度($\tau$),
单次散射反照率($\omega$),
以及散射不对称性因子($g$)。SNICAR中,体散射光学属性是每种成分$k$的加权函数,针对每个雪层和光谱波段进行计算。

\begin{equation}
\tau = \sum_{1}^{k}\tau_{k}
\end{equation}
%
\begin{equation}
\omega = \frac{\sum_{1}^{k}\omega_{k}\tau_{k}}{\sum_{1}^{k}\tau_{k}}
\end{equation}
%
\begin{equation}
g = \frac{\sum_{1}^{k}{g_{k}\omega}_{k}\tau_{k}}{\sum_{1}^{k}\omega_{k}\tau_{k}}
\end{equation}

对于每种成分(冰粒、两种黑碳、两种有机碳、四种沙尘),$\tau$、$g$和质量消光截面$\psi$(\unit{m^2.kg^{-1}})使用米氏散射近似进行计算。每种成分的光学厚度取决于各自的质量消光截面和各层质量,从而可以得到:
\begin{equation}
\tau_{k} = \psi_{k}\omega_{k}
\end{equation}

\subsubsection{积雪的光学属性}
冰粒在五个波段的光学属性为离线获得,并存储在相应的检索表中。米氏散射属性先在较高的光谱分辨率下计算,再根据入射太阳通量\(I^{\downarrow}(\lambda)\)对五个波段进行加权。波长区间在$\lambda_{1}-\lambda_{2}$的质量消光截面(\(\overline{\psi}\))表达式为:

\begin{equation}
\overline{\psi} = \frac{\int_{\lambda_{2}}^{\lambda_{1}}{\psi(\lambda)I^{\downarrow}(\lambda)d\lambda\ }}{\int_{\lambda_{2}}^{\lambda_{1}}{I^{\downarrow}(\lambda)d\lambda\ }}
\end{equation}

对于半无限厚的雪盖,宽波段的单次散射反照率(\(\overline{\omega}\))由漫射反照率加权得到:
\begin{equation}
\overline{\omega} = \frac{\int_{\lambda_{2}}^{\lambda_{1}}{\omega(\lambda){I^{\downarrow}(\lambda)\alpha}_{sno}(\lambda)d\lambda\ }}{\int_{\lambda_{2}}^{\lambda_{1}}{I^{\downarrow}(\lambda)\alpha_{sno}(\lambda)d\lambda\ }}
\end{equation}

积雪反照率十分依赖于冰粒的单次散射反照率,包含额外的反照率权重可以提高五个波段反照率方案的准确性\citep{flanner2007PresentdayClimateForcing}。

积雪在五种波段的单次散射反照率、质量消光截面以及不对称参数由表~\ref{tab:积雪单次散射反照率} 给出,它们均基于米氏散射近似理论计算得到。不同大气沉积气溶胶的单次散射反照率、质量消光截面以及不对称参数取决于波段。而对于冰粒,其光学属性取决于入射辐射类型(直射或漫射辐射)、波段以及有效粒径,其中有效粒径考虑积雪的老化过程计算得到(见章节~\ref{积雪老化})。

\begin{table}[htbp]
\centering
\caption{雪中杂质和冰粒的单次散射反照率$\omega$}
\label{tab:积雪单次散射反照率}
\begin{tabular}{llllll}
\toprule
种类 & 波段一 & 波段二 & 波段三 & 波段四 & 波段五 \\ \midrule
亲水黑碳 & 0.516 & 0.516 & 0.516 & 0.516 & 0.516 \\
疏水黑碳 & 0.288 & 0.288 & 0.288 & 0.288 & 0.288 \\
亲水有机碳 & 0.997 & 0.997 & 0.997 & 0.997 & 0.997 \\
疏水有机碳 & 0.963 & 0.963 & 0.963 & 0.963 & 0.963 \\
沙尘一 & 0.979 & 0.979 & 0.979 & 0.979 & 0.979 \\
沙尘二 & 0.944 & 0.944 & 0.944 & 0.944 & 0.944 \\
沙尘三 & 0.904 & 0.904 & 0.904 & 0.904 & 0.904 \\
沙尘四 & 0.850 & 0.850 & 0.850 & 0.850 & 0.850 \\
冰($r_{e}=30$ $\mu\rm{m}$) & 0.9999 & 0.9999 & 0.9999 & 0.9999 &
0.9999 \\
冰($r_{e}=1500$ $\mu\rm{m}$) & 0.9998 & 0.9998 & 0.9998 & 0.9998 &
0.9998 \\ \bottomrule
\end{tabular}
\end{table}

\begin{table}[htbp]
\centering
\caption{雪中杂质和冰粒的质量消光系数$\tau$}
\label{tab:积雪消光系数}
\begin{tabular}{llllll}
\toprule
种类 & 波段一 & 波段二 & 波段三 & 波段四 & 波段五 \\ \midrule
亲水黑碳 & 25369 & 25369 & 25369 & 25369 & 25369 \\
疏水黑碳 & 11398 & 11398 & 11398 & 11398 & 11398 \\
亲水有机碳 & 37774 & 37774 & 37774 & 37774 & 37774 \\
疏水有机碳 & 3289 & 3289 & 3289 & 3289 & 3289 \\
沙尘一 & 2687 & 2687 & 2687 & 2687 & 2687 \\
沙尘二 & 841 & 841 & 841 & 841 & 841 \\
沙尘三 & 388 & 388 & 388 & 388 & 388 \\
沙尘四 & 197 & 197 & 197 & 197 & 197 \\
冰($r_{e}=30$ $\mu\rm{m}$) & 55.7 & 55.7 & 55.7 & 55.7 & 55.7 \\
冰($r_{e}=1500$ $\mu\rm{m}$) & 1.09 & 1.09 & 1.09 & 1.09 & 1.09 \\ \bottomrule
\end{tabular}
\end{table}

\begin{table}[htbp]
\centering
\caption{雪中杂质和冰粒的不对称参数$g$}
\label{tab:积雪不对称参数}
\begin{tabular}{llllll}
\toprule
种类 & 波段一 & 波段二 & 波段三 & 波段四 & 波段五 \\ \midrule
亲水黑碳 & 0.52 & 0.52 & 0.52 & 0.52 & 0.52 \\
疏水黑碳 & 0.35 & 0.35 & 0.35 & 0.35 & 0.35 \\
亲水有机碳 & 0.77 & 0.77 & 0.77 & 0.77 & 0.77 \\
疏水有机碳 & 0.62 & 0.62 & 0.62 & 0.62 & 0.62 \\
沙尘一 & 0.69 & 0.69 & 0.69 & 0.69 & 0.69 \\
沙尘二 & 0.70 & 0.70 & 0.70 & 0.70 & 0.70 \\
沙尘三 & 0.79 & 0.79 & 0.79 & 0.79 & 0.79 \\
沙尘四 & 0.83 & 0.83 & 0.83 & 0.83 & 0.83 \\
冰($r_{e}=30$ $\mu\rm{m}$) & 0.88 & 0.88 & 0.88 & 0.88 & 0.88 \\
冰($r_{e}=1500$ $\mu\rm{m}$) & 0.89 & 0.89 & 0.89 & 0.89 & 0.89\\ \bottomrule
\end{tabular}
\end{table}

\subsubsection{积雪老化过程}\label{积雪老化}

雪的老化表现为冰粒有效粒径($r_e$)的变化。已有研究表明,使用由更复杂形状组成的冰介质表面积与体积比(或比表面积)的球体会在模拟半球通量中产生相对较小误差(例如~\citet{grenfell1999RepresentationNonsphericalIce})。有效半径是球形粒子集合的表面积加权平均半径,与比表面积(SSA)直接相关,表达式为$r_e=3/(\rho_{ice}\text{SSA})$,其中$\rho_{ice}$是冰的密度。$r_e$是一个简单实用的度量指标,将积雪微观物理状态与干雪辐射特性联系起来。

雪变湿的过程也会导致反照率快速变化。液态水的存在会导致周围的冰粒迅速变得粗糙(例如~\citet{brun1989InvestigationWetSnowMetamorphism}),液态水往往会重新冻结成较大的冰块,使积雪变暗。小液滴的存在本身不会地使积雪显著变暗,因为冰和水的折射率在整个太阳光谱中非常接近。然而,积水会大大减少每单位质量的折射事件的发生,从而使积雪变暗。目前还没有考虑到这种影响。

每个时间步长发生的有效晶粒度的净变化在每个雪层中可以概括为干雪变质作用($dr_{e,dry}$)、液态水导致的变质作用($dr_{e,wet}$)、液体水的再冻结和新降雪的加入引起的变化。每个雪层的质量被划分为上一个时间步长遗留下来的雪($f_{old}$)、新降雪($f_{new}$)和重新冻结的液态水($f_{rfz}$),使得雪的$r_e$在每个时间步长${t}$更新为:
\begin{equation}
r_{e}(t) = \lbrack r_{e}(t - 1) + {dr}_{e,dry} + {dr}_{e,wet}\rbrack f_{old} + r_{e,0}f_{new} + r_{e,rfz}f_{rfz}
\end{equation}

新降雪($r_{e,0}$)的有效粒径仅基于其与温度的简单线性关系。低于$-30$摄氏度时,有效粒径的最小值被强制取为54.5 \unit{\mu{m}}(对应60 \unit{m^{2}.kg^{-1}});大于0摄氏度时,有效粒径的最大值被强制取为204.5 \unit{\mu{m}}。有效粒径在$-30$和0摄氏度之间为线性变化。重新冻结的液态水($r_{e,rfz}$)的有效粒径设置为1000 \unit{\mu{m}}。

干雪的老化基于~\citet{flanner2006LinkingSnowpackMicrophysics}描述的微观物理模型。该模型模拟了不同尺寸和颗粒间距的冰晶集合之间的扩散水蒸汽通量。根据雪温度、温度梯度、密度和初始颗粒大小分布的任何组合,预测比表面积和有效半径。高温、较大的温度梯度以及较小的雪密度最有利于加速雪的老化,而在低温下,无论温度梯度和密度如何,雪的老化都会缓慢进行。由于该模型目前计算成本太高,无法纳入气候模型,我们将参数曲线拟合为大范围雪况下的输出模型,并将这些参数应用于模型中。参数方程的函数形式为:
\begin{equation}
\frac{{dr}_{e,dry}}{dt} = \left( \frac{{dr}_{e}}{dt} \right)_{0}\left( \frac{\eta}{r_{e} - r_{e,0} + \eta} \right)^{1/\kappa}
\end{equation}

参数$\left( \frac{{dr}_{e}}{dt} \right)_{0}$,$\eta$和$\kappa$以交互方式从具有与积雪温度、温度梯度和密度相对应的查找表中检索。该查找表所涵盖的温度范围为223--273 K、温度梯度范围为0至300 \unit{K.m^{-1}},密度范围为50至400 \unit{kg.m^{-3}}。使用中间层温度在每个雪层n的中点来计算温度梯度($T_n$)和雪层厚度($dz_n$):
%
\begin{equation}
\left( \frac{dT}{dz} \right)_{n} = \frac{1}{{dz}_{n}}abs\left\lbrack \frac{T_{n - 1}{dz}_{n} + T_{n}{dz}_{n - 1}}{{dz}_{n} + {dz}_{n - 1}} + \frac{T_{n + 1}{dz}_{n} + T_{n}{dz}_{n + 1}}{{dz}_{n} + {dz}_{n + 1}} \right\rbrack
\end{equation}
%
对于底部雪层($n=0$),$T_{n+1}$为表层土壤的温度;对于表层雪,假设$T_{n-1}=T_n$。

液态水对加强的变质作用的贡献基于~\citet{brun1989InvestigationWetSnowMetamorphism}中的参数方程,其测量了不同液态水含量下的粒径增长率,这种关系取决于液态水质量分数$f_{liq}$:
%
\begin{equation}
\frac{{dr}_{e}}{dt} = \frac{10^{18}C_{1}f_{liq}^{3}}{4\pi r_{e}^{2}}
\end{equation}
%
其中$C_{1}$为$4.22×10^{-13}$,$f_{liq}=w_{liq}/(w_{liq}+w_{ice})$。

在雪质量大于零但尚未定义雪层的情况下,$r_e$设置为$r_{e,0}$,当雪层被合并或分割时,$r_e$根据其他状态变量的计算,计算为两层的质量加权平均值。计算得到的$r_e$的范围限制在30--1500 \unit{\mu{m}}之间。

\subsubsection{SNICAR模型二流近似求解}

对于多层积雪辐射传输二流近似求解,SNICAR模型提供了两种可供选择的方案:SNICAR\_RT \citep{dang2019IntercomparisonImprovementTwostream}与SNICAR\_AD\_RT方案\citep{flanner2021SNICARADv3CommunityTool}。

1、SNICAR\_RT

SNICAR\_RT在二流近似中使用了三对角矩阵方案(tri-diagonal
solution, \citet{toon1989RapidCalculationRadiative}),在每层界面计算向上和向下的辐射通量,从中可以推导出净辐射吸收、每层吸收和表面反照率,三对角矩阵方案包含了二流近似方案的中间项。SNICAR\_RT对可见光波段使用的是Eddington近似\citep{wiscombe1980ModelSpectralAlbedo},对近红外波段使用Hemispheric mean \citep{toon1989RapidCalculationRadiative},这样选择是因为Eddington近似方案适用于高强度散射的介质。

二流近似从水平均一介质辐射传输的一般方程所导出:
%
\begin{equation}
\mu\frac{\partial {I}}{\partial\tau}(\tau,\mu,\phi){= I(\tau,}\mu,\Phi) - \frac{\omega}{{4}{\pi}}\int_{{0}}^{{2}\pi}{\int_{- 1}^{1}{P\left( \mu,\mu^{'},\Phi,\Phi^{'} \right)}}I\left( \tau,\mu^{'},\Phi^{'} \right){d}\mu^{'}d\Phi^{'}{- S}\left( \tau,\mu,\Phi \right)
\end{equation}
%
其中\(\Phi\)是方位角,\(\mu\)是天顶角的余弦,\(\omega\)是单次散射反照率。等式右侧的三项分别为:光学厚度为\(\tau\)时的强度、多次散射引起的内源项和外部源函数$S$。对于太阳波长的纯外部光源,$S$为:
%
\begin{equation}
S = \frac{\omega}{4}F_{s}P\left( \mu,{- \mu}_{0},\Phi,\Phi_{0} \right)\exp\left( -\frac{\mathbf{\tau}}{\mu_{0}} \right)
\end{equation}

\citet{meador1980TwostreamApproximationsRadiative}指出,二流近似的表示式可以写成以下形式:
\begin{equation}
\frac{dF_{n}^{+}(\tau)}{d\tau_{n}} = \gamma_{1n}F_{n}^{+} - \gamma_{2n}F_{n}^{-} - S_{n}^{+}
\end{equation}
\begin{equation}
\frac{dF_{n}^{-}(\tau)}{d\tau_{n}} = \gamma_{2n}F_{n}^{+} - \gamma_{1n}F_{n}^{-} + S_{n}^{+}
\end{equation}
其中,\(\gamma_{1}\)和\(\gamma_{2}\)是取决于二流近似方程的特定形式的系数。我们通过用$n$(层数)标记通量来得到多层的解决方案。

\begin{table}[htbp]
\centering
\caption{SNICAR中包含的二流近似系数取值}
\label{tab:SNICAR二流近似系数}
\begin{tabular}{llll}
\toprule
方法 & $\gamma_{1}$ & $\gamma_{2}$ & $\gamma_{3}$ \\ \midrule
Eddington &
\(\frac{\mathbf{1}}{\mathbf{4}}\left\lbrack \mathbf{7 - (4 + 3}g^{*})\omega^{*} \right\rbrack\)
&
\(- \frac{\mathbf{1}}{\mathbf{4}}\left\lbrack \mathbf{1 - (4 - 3}g^{*})\omega^{*} \right\rbrack\)
& \(\frac{\mathbf{1}}{\mathbf{4}}\mathbf{(2 - 3}g^{*}\mu_{0})\) \\
Hemispheric mean & \(\mathbf{2 - (1 +}g^{*})\omega^{*}\) &
\(\mathbf{(1 -}g^{*})\omega^{*}\) &
\(\frac{\mathbf{1}}{\mathbf{2}}\left\lbrack \mathbf{1 -}\sqrt{\mathbf{3}}g^{*}\mu_{0} \right\rbrack\) \\
Quadrature &
\(\frac{\sqrt{\mathbf{3}}}{\mathbf{2}}\left\lbrack \mathbf{2 - (1 +}g^{*})\omega^{*} \right\rbrack\)
& \(\frac{\sqrt{\mathbf{3}}}{\mathbf{2}}\mathbf{(1 -}g^{*})\omega^{*}\)
&
\(\frac{\mathbf{1}}{\mathbf{2}}\left\lbrack \mathbf{1 -}\sqrt{\mathbf{3}}g^{*}\mu_{0} \right\rbrack\) \\ \bottomrule
\end{tabular}
\end{table}

对于直射入射太阳辐射:
\begin{equation}
S^{+} = \gamma_{3}\pi F_{s}\omega_{0}\exp\left( - \left( \tau_{c} + \tau \right)/\mu_{0} \right)
\end{equation}
\begin{equation}
S^{-} = \gamma_{4}\pi F_{s}\omega_{0}\exp\left( - \left( \tau_{c} + \tau \right)/\mu_{0} \right)
\end{equation}
\begin{equation}
\frac{dF^{\uparrow}(\tau)}{d\tau} = \gamma_{1}F^{\uparrow}(\tau) - \gamma_{2}F^{\downarrow}(\tau) - \gamma_{3}\widetilde{\omega}F_{\odot}exp( - \frac{\tau}{\mu_{0}})
\end{equation}
\begin{equation}
\frac{dF^{\downarrow}(\tau)}{d\tau} = \gamma_{1}F^{\uparrow}(\tau) - \gamma_{1}F^{\downarrow}(\tau) + (1 - \gamma_{3})\widetilde{\omega}F_{\odot}exp( - \frac{\tau}{\mu_{0}})
\end{equation}
其中\(\pi F_{s}\)是入射太阳通量,\(\mu_{0}\)是太阳辐射的入射方向,以上两式可以得到二流近似的通解。求解系数的关键步骤是计算相函数。然而,简单的近似方法,比如Eddington近似,通常难以处理高度非对称的相函数(例如冰晶相函数)。一种常见的解决方案是通过使用$\delta$函数来近似相函数的强前向散射峰,并对相函数进行二流展开,从而得到广泛使用的``$\delta$函数调整"方法\citep{joseph1976DeltaEddingtonApproximationRadiative}。$\delta$函数调整能够提高在相函数高度非对称情况下的不同二流近似的准确性,表达式如下:
\begin{equation}
\tau^{*} = (1 - \omega g^{2})\tau
\end{equation}
\begin{equation}
\omega^{*} = \frac{(1 - g^{2})\omega}{1 - \omega g^{2}}
\end{equation}
\begin{equation}
g^{*} = \frac{g}{1 + g}
\end{equation}

\citet{meador1980TwostreamApproximationsRadiative}的研究指出,在第$n$层、光学厚度为\(\tau\)时,向上和向下的辐射通量可以表示为:
\begin{equation}
F_{n}^{+} = k_{1n}\exp\left( \Lambda_{n}\mathbf{\tau} \right) + \Gamma_{n}k_{2n}\exp\left( {- \Lambda}_{n}\mathbf{\tau} \right) + C_{n}^{+}(\mathbf{\tau)}
\end{equation}
\begin{equation}
F_{n}^{-} = \Gamma_{n}k_{1n}\exp\left( \Lambda_{n}\mathbf{\tau} \right) + k_{2n}\exp\left( {- \Lambda}_{n}\mathbf{\tau} \right) + C_{n}^{-}(\mathbf{\tau)}
\end{equation}
%
\(\Lambda_{n}\)、\(\Gamma_{n}\)和\(C_{n}\)分别为由二流方法、入射太阳通量和太阳高度角确定的已知系数,\(k_{1n}\)和\(k_{2n}\)为由边界条件确定的未知系数。
\begin{equation}
\Lambda = \sqrt{\gamma_{1} - \gamma_{2}}
\end{equation}
\begin{equation}
\Gamma = \frac{\gamma_{2}}{\gamma_{1} + \Lambda} = \frac{\gamma_{1} - \Lambda}{\gamma_{2}}
\end{equation}

对于太阳直射辐射,我们有:
\begin{equation}
C^{+}(\mathbf{\tau}) = \frac{\omega F_{0}\left\lbrack \left( \gamma_{1} - 1/\mu_{0} \right)/\gamma_{3} + \gamma_{2}\gamma_{4} \right\rbrack e^{- \left( \mathbf{\tau}_{c} + \mathbf{\tau} \right)/\mu_{0}}}{\Lambda^{2} - 1/\mu_{0}^{2}}
\end{equation}
\begin{equation}
C^{-}(\mathbf{\tau}) = \frac{\omega F_{0}\left\lbrack \left( \gamma_{1} + 1/\mu_{0} \right)/\gamma_{4} + \gamma_{2}\gamma_{3} \right\rbrack e^{- \left( \mathbf{\tau}_{c} + \mathbf{\tau} \right)/\mu_{0}}}{\Lambda^{2} - 1/\mu_{0}^{2}}
\end{equation}

当没有直射入射辐射时,\(C^{+}(\mathbf{\tau})\)和\(C^{-}(\mathbf{\tau})\)为零。

2、SNICAR\_AD\_RT

SNICAR\_AD\_RT使用了Delta-Eddington
Adding-Doubling方案。在$\delta$函数调整后,再应用二流近似(Eddington),将得到的$\tau^*$,$\omega^*$和$g^*$计算得到如下的各个中间量:
\begin{equation}
\Gamma = \sqrt{3(1 - \omega^{*})(1 - \omega^{*}g^{*})}
\end{equation}
\begin{equation}
u = \frac{3}{2}\left( \frac{1 - \omega^{*}g^{*}}{\Gamma} \right)
\end{equation}
\begin{equation}
N = {(u + 1)}^{2}e^{\Gamma\tau^{*}} - {(u - 1)}^{2}e^{- \Gamma\tau^{*}}
\end{equation}
\begin{equation}
\eta = \frac{3}{4}\omega^{*}\mu_{0}\left( \frac{1 + g^{*}(1 - \omega^{*})}{1 - \Gamma^{2}\mu_{0}^{2}} \right)
\end{equation}
\begin{equation}
\Upsilon = \frac{1}{2}\omega^{*}\left( \frac{1 + {3g}^{*}(1 - \omega^{*})\mu_{0}^{2}}{1 - \Gamma^{2}\mu_{0}^{2}} \right)
\end{equation}

每层对直接辐射的反射率($R$)和透射率($T$)分别可以表示为:
\begin{equation}
R\left( \mu_{0} \right) = \frac{1}{N}(\eta + \Upsilon)\left( u^{2} - 1 \right)\left( e^{\Gamma\tau^{*}} - e^{- \Gamma\tau^{*}} \right) + (\eta - \Upsilon)\left\lbrack \frac{4u}{N}e^{\frac{- \tau^{*}}{\mu_{0}}} - 1 \right\rbrack
\end{equation}
\begin{equation}
T\left( \mu_{0} \right) = (\eta + \Upsilon)\frac{4u}{N} + \left\lbrack \frac{1}{N}(\eta - \Upsilon)\left( u^{2} - 1 \right)\left( e^{\Gamma\tau^{*}} - e^{- \Gamma\tau^{*}} \right) - \eta - \Upsilon + 1 \right\rbrack e^{\frac{- \tau^{*}}{\mu_{0}}}
\end{equation}

对于漫射辐射的反射和透射率(\(\overline{R}\)和\(\overline{T}\))是根据\citet{wiscombe1980ModelSpectralAlbedo}的假设,即在各向同性入射情况下,通过对直接辐射的反射率进行积分计算得到的。
%
\begin{equation}
\overline{R}\left( \mu_{0} \right) = 2\int_{0}^{+ 1}{\mu R(\mu)d\mu}
\end{equation}
\begin{equation}
\overline{T}\left( \mu_{0} \right) = 2\int_{0}^{+ 1}{\mu T(\mu)d\mu}
\end{equation}
%
对于反射率($R$)和透射率($T$)的数值求解是由8个高斯角积分得到。

在考虑层与层多次散射时,假设一旦辐射被散射,它就是漫射和各向同性的。对于位于上层的任意层1(具有辐射特性\(R_{1}\left( \mu_{0} \right)\)、\(T_{1}\left( \mu_{0} \right)\)、\({\overline{R}}_{1}\)、\({\overline{T}}_{1}\)),其下方是层2(具有辐射特性\(R_{2}\left( \mu_{0} \right)\)、\(T_{2}\left( \mu_{0} \right)\)、\({\overline{R}}_{2}\)、\({\overline{T}}_{2}\)),从上方入射的直射和漫射辐射的反射率、透射率计算为\citep{briegleb2007delta}:
\begin{equation}
R_{12}\left( \mu_{0} \right) = R_{1}\left( \mu_{0} \right) + \frac{\left\lbrack \left( T_{1}\left( \mu_{0} \right) - e^{- \tau^{*}/\mu_{0}} \right){\overline{R}}_{2} + e^{- \tau^{*}/\mu_{0}}R_{2}\left( \mu_{0} \right) \right\rbrack{\overline{T}}_{1}}{1 - {\overline{R}}_{1}{\overline{R}}_{2}}
\end{equation}
\begin{equation}
T_{12}\left( \mu_{0} \right) = {e^{- \tau^{*}/\mu_{0}}T}_{2}\left( \mu_{0} \right) + \frac{\left\lbrack \left( T_{1}\left( \mu_{0} \right) - e^{- \tau^{*}/\mu_{0}} \right) + e^{- \tau^{*}/\mu_{0}}R_{2}\left( \mu_{0} \right){\overline{R}}_{1} \right\rbrack{\overline{T}}_{2}}{1 - {\overline{R}}_{1}{\overline{R}}_{2}}
\end{equation}
\begin{equation}
{\overline{R}}_{12} = {\overline{R}}_{1} + \frac{{\overline{T}}_{1}{\overline{R}}_{2}{\overline{T}}_{1}}{1 - {\overline{R}}_{1}{\overline{R}}_{2}}
\end{equation}
\begin{equation}
{\overline{T}}_{12} = \frac{{\overline{T}}_{1}{\overline{T}}_{2}}{1 - {\overline{R}}_{1}{\overline{R}}_{2}}
\end{equation}

从下方照射时的透射率是相同的,而反射率假定为从上方漫射照射时的向上反射率\({\overline{R}}_{12}\),但下标的层序号相反。累积的雪层界面光学属性,对于向下传播和向上传播的辐射,是通过循环求解的,从雪层的顶部向下积分,然后从底部向上积分。在从顶部向下计算辐射传输时,如果所在波段总透射率(在顶部归一化为1.0)小于0.001,则终止\(R\left( x_{0} \right)\)、\(T\left( x_{0} \right)\)、\(R\)、\(T\)的计算。对于每个界面,都有以下变量:

``$\tau^{*}$''为雪层顶部到某一雪层交界面经调整的光学厚度。

``$R_{up}\left( \mu_{0} \right)$''为某一雪层交界面以下所有雪层对来自上方的向下直接辐射的反射率。

``$\overline{R}_{up}$''为某一雪层交界面以下所有雪层对来自上方的向下漫射辐射的反射率。

``$\overline{R}_{dn}$''为某一雪层交界面以上的所有雪层对来自下方的向上漫射辐射的反射率。

``$T_{dn}\left( \mu_{0} \right)$''为直射辐射从雪层顶部到某一雪层交界面的总透射率。

``$\overline{T}_{dn}$''为漫射辐射从雪层顶部到某一雪层交界面的总透射率。

基于以上变量,每个雪层交界面上的向下(\(F_{dir}^{\downarrow}\))和向上(\(F_{dir}^{\uparrow}\))辐射通量标准化为雪层顶部的单位直接入射通量,分别表示为:
\begin{equation}
F_{dir}^{\downarrow} = e^{- \tau^{*}/\mu_{0}} + \frac{\left( T_{dn}\left( \mu_{0} \right) - e^{- \tau^{*}/\mu_{0}} \right) + e^{- \tau^{*}/\mu_{0}}R_{up}\left( \mu_{0} \right){\overline{R}}_{dn}}{1 - {\overline{R}}_{dn}{\overline{R}}_{up}}
\end{equation}
\begin{equation}
F_{dir}^{\uparrow} = \frac{e^{- \tau^{*}/\mu_{0}}R_{up}\left( \mu_{0} \right) + \left( T_{dn}\left( \mu_{0} \right) - e^{- \tau^{*}/\mu_{0}} \right) + {\overline{R}}_{up}}{1 - {\overline{R}}_{dn}{\overline{R}}_{up}}
\end{equation}

每个雪层交界面上的向下(\(F_{dif}^{\downarrow}\))和向上(\(F_{dif}^{\uparrow}\))通量标准化为模型顶部的单位漫射入射通量,分别表示为:
\begin{equation}
F_{dif}^{\downarrow} = \frac{{\overline{T}}_{dn}}{1 - {\overline{R}}_{dn}{\overline{R}}_{up}}
\end{equation}
\begin{equation}
F_{dif}^{\uparrow} = \frac{{\overline{T}}_{dn}{\overline{R}}_{up}}{1 - {\overline{R}}_{dn}{\overline{R}}_{up}}
\end{equation}

最后,反照率\(\alpha\)由顶层界面的通量得到:
\begin{equation}
\alpha = \frac{F_{dir}^{\uparrow}(top) + F_{dif}^{\uparrow}(top)}{F_{dir}^{\downarrow}(top) + F_{dif}^{\downarrow}(top)}
\end{equation}

太阳辐射通量根据五个光谱波段来计算,如下表所示。由于雪的反照率在整个太阳光谱中变化较大,因此采用四个波段来更准确地表示雪在近红外(NIR,0.7--5.0 $\mu \rm{m}$)波段的特性,而对于可见光谱只采用一个波段。根据表中列出的权重划分向下NIR辐射通量,这些权重通过中纬度冬季典型大气的离线光谱辐射传输计算得到\citep{briegleb2007delta}。

\begin{table}[htbp]
\centering
\caption{分波段太阳辐射权重系数}
\label{tab:太阳辐射权重系数}
\begin{tabular}{lll}
\toprule
光谱波段 & 直接太阳辐射权重 & 漫射太阳辐射权重 \\ \midrule
波段1:0.3-0.7 $\mu \rm{m}$(可见光) & 1.0 & 1.0 \\
波段2:0.7-1.0 $\mu \rm{m}$(近红外) & 0.494 & 0.586 \\
波段3:1.0-1.2 $\mu \rm{m}$(近红外) & 0.181 & 0.202 \\
波段4:1.2-1.5 $\mu \rm{m}$(近红外) & 0.121 & 0.109 \\
波段5:1.5-5.0 $\mu \rm{m}$(近红外) & 0.204 & 0.103 \\ \bottomrule
\end{tabular}
\end{table}

\section{海洋反照率}\label{海洋反照率}
\begin{mymdframed}{代码}
对应代码为\texttt{MOD\_Albedo.F90}文件中\texttt{albocean()}函数。
\end{mymdframed}

对于有冰雪覆盖的海洋,当\(\mu < 0.5\)时,积雪区域反照率计算为:
%
\begin{equation}
\alpha_{sno,vis,dir} = \min\left( 0.98,\alpha_{sno,vis,dif} + 0.5\frac{1 - \alpha_{sno,vis,dif}}{\frac{3}{1 + 4\mu} - 1} \right)
\end{equation}
%
\begin{equation}
\alpha_{sno,nir,dir} = \min\left( 0.98,\alpha_{sno,nir,dif} + 0.5\frac{1 - \alpha_{sno,nir,dif}}{\frac{3}{1 + 4\mu} - 1} \right)
\end{equation}
%
其中\(\alpha_{sno,vis,dif} = \alpha_{newsno,vis} = 0.95\),\(\alpha_{sno,nir,dif} = \alpha_{newsno,nir} = 0.7\)。当\(\mu \geq 0.5\)时,相应值计算为:
%
\begin{equation}
\alpha_{sno,vis,dir} = \alpha_{newsno,vis}
\end{equation}
%
\begin{equation}
\alpha_{sno,nir,dir} = \alpha_{newsno,nir}
\end{equation}
%
积雪厚度计算为:
%
\begin{equation}
z_{sno} = \frac{20W_{sno}}{1000}
\end{equation}
%
其中\(W_{sno}\)为雪水当量(mm)。海面积雪覆盖度计算为:
\begin{equation}
f_{sno} = \frac{z_{sno}}{0.25 + z_{sno}}
\end{equation}
考虑积雪覆盖的海洋总体反照率计算为:
\begin{equation}
\alpha_{sea,vis,dir} = \alpha_{ice,vis}\left( 1 - f_{sno} \right) + \alpha_{sno,vis,dir}f_{sno}
\end{equation}
%
\begin{equation}
\alpha_{sea,nir,dir} = \alpha_{ice,nir}\left( 1 - f_{sno} \right) + \alpha_{sno,nir,dir}f_{sno}
\end{equation}
%
\begin{equation}
\alpha_{sea,vis,dif} = \alpha_{ice,vis}\left( 1 - f_{sno} \right) + \alpha_{sno,vis,dif}f_{sno}
\end{equation}
%
\begin{equation}
\alpha_{sea,nir,dif} = \alpha_{ice,nir}\left( 1 - f_{sno} \right) + \alpha_{sno,nir,dif}f_{sno}
\end{equation}
其中\(\alpha_{ice,vis} = 0.7\),\(\alpha_{ice,nir} = 0.5\)。

对于无冰雪覆盖海洋,假设反照率与波段和海表面风速等其他物理量无关,采用太阳高度角计算直射入射反照率,漫射反射率则设为常数,公式表达为:
\begin{equation}
\alpha_{sea,vis,dir}=\alpha_{sea,nir,dir} = \frac{0.026}{\mu^{1.7}+0.065}+0.15(\mu-0.1)(\mu-0.5)(\mu-1)
\end{equation}
%
\begin{equation}
\alpha_{sea,vis,dif}=\alpha_{sea,nir,dif} = 0.06
\end{equation}

\section{植被反照率---二流近似模型}\label{植被反照率二流近似模型}
{
\begin{figure}[htbp]
\centering
\includegraphics[width=0.7\columnwidth]{Figures/辐射过程及辐射通量计算/二流近似模型示意图.png}
\caption{二流近似植被辐射传输模型示意图}
\label{fig:二流近似模型示意图}
\end{figure}
}
植被反照率采用二流近似方法(two-stream approximation,\citet{dickinson1983land,sellers1985canopy}) 进行求解。在此基础上,利用双大叶模型 (two-big-leaf model) \citep{dai2004two} 对阴阳叶分别计算辐射吸收量。二流近似方法是把植被冠层内的辐射场看成是上下两个辐射流组成(图~\ref{fig:二流近似模型示意图}),这样可以把描述辐射传输的微分方程简化为两个联立的常微分方程组:
\begin{equation}\label{di_dx1}
-\bar{\mu} \frac{d I^{\uparrow}}{d x}+\left[1-(1-\beta) \omega\right] I^{\uparrow}-\omega \beta I^{\downarrow}=\omega \bar{\mu} K \beta_{0} \exp (-K x)
\end{equation}
\begin{equation}\label{di_dx2}
\bar{\mu} \frac{d I^{\downarrow}}{d x}+\left[1-(1-\beta) \omega\right] I^{\downarrow}-\omega \beta I^{\uparrow}=\omega \bar{\mu} K\left(1-\beta_{0}\right) \exp (-K x)
\end{equation}
其中$I^{\uparrow}$, $I^{\downarrow}$表示向上和向下的漫射通量,为相对于总入射辐射通量(直射或漫射)的归一化值。$\mu$为太阳天顶角余弦,$\bar{\mu}$为漫射辐射在单位叶面积上的光学厚度倒数平均值。$\beta$、$\beta_{0}$为漫射辐射和直射辐射向上散射(upscatter)系数,$\omega$为叶片单次散射反照率,等于叶片反射率$\rho_{l}$和透射率$\tau_{l}$之和。$K$为直射光在单位叶面积上的光学厚度,计算为$G(\mu) / \mu$,其中$G(\mu)$为叶子在入射辐射方向上的投影面积。$x$表示植被从冠层顶部 ($x=0$) 往下 ($x=L_{AI}$) 累积的叶面积指数。

$G(\mu) / \mu$函数采用 \citet{goudriaan1977crop} 方案,它使用一个参数$\chi_{L}$~\citep{ross1975radiative} 去描述植被的叶倾角分布:
\begin{equation}\label{Gmu}
G(\mu)=\phi_{1}+\phi_{2} \mu
\end{equation}
其中$\phi_{1}=0.5-0.633 \chi_{L}-0.33 \chi_{L}^{2}$,$\phi_{2}=0.877\left(1-2 \phi_{1}\right)$,
\begin{equation}
\chi_{L}=\pm \int_{0}^{\frac{\pi}{2}}\left|\sin \theta_{L}-f_{S}\left(\theta_{L}\right)\right| d \theta_{L}
\end{equation}
上式中$\sin \theta_{L}$为球形叶倾角分布函数,$f_{S}\left(\theta_{L}\right)$为植被的叶倾角分布函数。
因此,对于球形分布来说,$\chi_{L}=0$。 一般来讲,$-0.4 \leqslant \chi_{L} \leqslant 0.6$,分别对应竖直型和平面型叶倾角分布。
模型中实际应用时采用通过查找表得到的不同植被类型$\chi_{L}$设定值。根据 $G(\mu)$ 的表达式,$\bar{\mu}$可以计算为:
\begin{equation}
\bar{\mu}=\int_{0}^{1} \frac{\mu^{\prime}}{G\left(\mu^{\prime}\right)}
 d \mu^{\prime}=\frac{1}{\phi_{2}}\left[1-\frac{\phi_{1}}{\phi_{2}}
 \ln \left(\frac{\phi_{1}+\phi_{2}}{\phi_{1}}\right)\right]
\end{equation}
上面计算表达式是假设$\phi_{1} \neq 0$以及$\phi_{2} \neq 0$,但实际上两者均可为0 \citep{dai2004two}。当$\phi_{1}$、$\phi_{2}$为0时,计算为:
\begin{equation}
\bar{\mu}= \begin{cases} 
1 / 0.877, & \text { 当 }\ \phi_{1}=0 \\
1 /\left(2 \phi_{1}\right), & \text { 当 }\ \phi_{2}=0
\end{cases}
\end{equation}
假设叶片上的散射辐射各项同性(理想朗伯体),漫射辐射和直射辐射向上散射系数$\beta$、$\beta_0$可近似计算为:
\begin{equation}
\omega \beta=\frac{1}{2}\left[\rho_{l}+\tau_{l}+\left(\rho_{l}-\tau_{l}\right)\left(\frac{1+\chi_{L}}{2}\right)^{2}\right]
\end{equation}
%
\begin{equation}\label{beta0}
\beta_{0}=\frac{1+\bar{\mu} K}{\omega \bar{\mu} K} a_{s}(\mu)
\end{equation}
其中$\rho_l$和$\tau_l$表示叶片的反射率和透射率,
并依赖于波段,$\omega=\rho_l+\tau_l$。$a_s$为当植被为无穷厚时($L_{AI}\rightarrow\infty$),
整个植被的单次散射反照率,计算为:
\begin{equation}
\begin{aligned} 
a_{s} &=\frac{\omega}{2} \int_{0}^{1} \frac{\mu G\left(\mu^{\prime}\right)}{\mu G\left(\mu^{\prime}\right)+\mu^{\prime} 
G(\mu)} d \mu^{\prime} \\[1ex]
&=\frac{\omega}{2} \frac{G(\mu)}{G(\mu)+\mu \phi_{2}}\left[1-\frac{\mu \phi_{1}}{G(\mu)+\mu \phi_{2}} 
\ln \left(\frac{G(\mu)+\mu \phi_{1}+\mu \phi_{2}}{\mu \phi_{1}}\right)\right] 
\end{aligned}
\end{equation}

对于方程(\ref{di_dx1})和(\ref{di_dx2}),直射入射情况下(即假定通过冠层顶部向下的漫射辐射通量等于0),其边界条件为:
\begin{equation}
\begin{cases}
I^{\downarrow} =0, &\text {当 }\, x=0 \text { 时 } \\
%\end{equation}
%\begin{equation}
I^{\uparrow} =\alpha_{g, dif} I^{\downarrow}+\alpha_{g, dir} \exp \left(-K L_{AI} \right), &\text {当 }\, x=L_{AI} \text { 时 }
\end{cases}
\end{equation}
方程的解为:
\begin{equation}
I^{\uparrow}=h_{1} e^{-K x / \sigma}+h_{2} e^{-h x}+h_{3} e^{h x}
\end{equation}
%
\begin{equation}
I^{\downarrow}=h_{4} e^{-K x / \sigma}+h_{5} e^{-h x}+h_{6} e^{h x}
\end{equation}
因此可以得到在冠层顶部的漫射辐射通量(即反照率)为:
\begin{equation}
\alpha_{veg, dir}=I^{\uparrow}(0)=\frac{h_{1}}{\sigma}+h_{2}+h_{3}
\end{equation}
通过植被向下的辐射通量,即到达地面的辐射通量为:
\begin{equation}
\tau_{veg, dir}=I^{\downarrow}\left(L_{A I}\right)=\frac{h_{4}}{\sigma} e^{-K} L_{A I}+h_{5} e^{-h L_{A I}}+h_{6} e^{h L_{A I}}
\end{equation}

漫射入射辐射时,边界条件为:
\begin{equation}
\begin{cases}
I^{\downarrow}=1, &\text {当 }\, x=0 \text { 时 } \\
I^{\uparrow}=\alpha_{g, dif} I^{\downarrow}, &\text {当 }\, x=L_{A I} \text { 时 }
\end{cases}
\end{equation}
方程的解为:
\begin{equation}
I^{\uparrow}=h_{7} e^{-h x}+h_{8} e^{h x}
\end{equation}
%
\begin{equation}
I^{\downarrow}=h_{9} e^{-h x}+h_{10} e^{h x}
\end{equation}
因此可以得到在冠层顶部向上的漫射辐射通量(即反照率)为:
\begin{equation}
\alpha_{veg, dif}=I^{\uparrow}(0)=h_{7}+h_{8}
\end{equation}
到达地面的漫射辐射通量为:
\begin{equation}
\tau_{veg, dif}=I^{\downarrow}\left(L_{A I}\right)=h_{9} e^{-h L_{AI}}+h_{10} e^{h L_{AI}}
\end{equation}
以上$\sigma,h_1,h_2,\ldots,h_{10}$计算为(参考~\citet{sellers1985canopy}附录; 注: $h_4$在原文中有误,下面为修改后计算式):
\begin{equation}
b=1-\omega+\omega \beta
\end{equation}
\begin{equation}
c=\omega \beta
\end{equation}
\begin{equation}
d=\omega \bar{\mu} K \beta_{0}
\end{equation}
\begin{equation}
f=\omega \bar{\mu} K\left(1-\beta_{0}\right)
\end{equation}
\begin{equation}
h=\frac{\sqrt{b^{2}-c^{2}}}{\bar{\mu}}
\end{equation}
\begin{equation}
\sigma=(\bar{\mu} K)^{2}+c^{2}-b^{2}
\end{equation}
\begin{equation}
u_{1}=b-c / \alpha_{g, dif}
\end{equation}
\begin{equation}
u_{2}=b-c \alpha_{g, dif}
\end{equation}
\begin{equation}
u_{3}=f+c \alpha_{g, dif}
\end{equation}
\begin{equation}
s_{1}=\exp \left[-\min \left(h L_{AI}, 50\right)\right]
\end{equation}
\begin{equation}
s_{2}=\exp \left[-\min \left(K L_{AI}, 50\right)\right]
\end{equation}
\begin{equation}
p_{1}=b+\bar{\mu} h
\end{equation}
\begin{equation}
p_{2}=b-\bar{\mu} h
\end{equation}
\begin{equation}
p_{3}=b+\bar{\mu} K
\end{equation}
\begin{equation}
p_{4}=b-\bar{\mu} K
\end{equation}
\begin{equation}
d_{1}=\frac{p_{1}\left(u_{1}-\bar{\mu} h\right)}{s_{1}}-p_{2}\left(u_{1}+\bar{\mu} h\right) s_{1}
\end{equation}
\begin{equation}
d_{2}=\frac{u_{2}+\bar{\mu} h}{s_{1}}-\left(u_{2}-\bar{\mu} h\right) s_{1}
\end{equation}
\begin{equation}
h_{1}=-d p_{4}-c f
\end{equation}
\begin{equation}
h_{2}=\frac{1}{d_{1}}\left[\left(d-\frac{h_{1}}{\sigma} p_{3}\right)\left(u_{1}-\bar{\mu} h\right) 
\frac{1}{s_{1}}-p_{2}\left(d-c-\frac{h_{1}}{\sigma}\left(u_{1}+\bar{\mu} K\right)\right) s_{2}\right]
\end{equation}
\begin{equation}
h_{3}=-\frac{1}{d_{1}}\left[\left(d-\frac{h_{1}}{\sigma}\right)\left(u_{1}+\bar{\mu} h\right) 
s_{1}-p_{1}\left(d-c-\frac{h_{1}}{\sigma}\left(u_{1}+\bar{\mu} K\right)\right) s_{2}\right]
\end{equation}
\begin{equation}
h_{4}=-f p_{3}-c d
\end{equation}
\begin{equation}
h_{5}=-\frac{1}{d_{2}}\left[\frac{h_{4}}{\sigma}\left(u_{2}+\bar{\mu} h\right) 
\frac{1}{s_{1}}+\left(u_{3}-\frac{h_{4}}{\sigma}\left(u_{2}-\bar{\mu} K\right)\right) s_{2}\right]
\end{equation}
\begin{equation}
h_{6}=\frac{1}{d_{2}}\left[\frac{h_{4}}{\sigma}\left(u_{2}-\bar{\mu} h\right) 
s_{1}+\left(u_{3}-\frac{h_{4}}{\sigma}\left(u_{2}-\bar{\mu} K\right)\right) s_{2}\right]
\end{equation}
\begin{equation}
h_{7}=\frac{c\left(u_{1}-\bar{\mu} h\right)}{d_{1} s_{1}}
\end{equation}
\begin{equation}
h_{8}=\frac{-c\left(u_{1}+\bar{\mu} h\right) s_{1}}{d_{1}}
\end{equation}
\begin{equation}
h_{9}=\frac{u_{2}+\bar{\mu} h}{d_{2} s_{1}}
\end{equation}
\begin{equation}
h_{10}=\frac{-s_{1}\left(u_{2}-\bar{\mu} h\right)}{d_{2}}
\end{equation}
以上解严格上是在$\sigma \neq 0$时成立,当$\sigma = 0$时,\citet{dai2004two} 给出了直射入射时的解为:
\begin{equation}
I^{\uparrow}=h_{2}^{\prime} e^{-K x}+h_{3}^{\prime} e^{K x}-\frac{h 1}{\bar{\mu}^{2}}\left(x+\frac{1}{2 K}\right) e^{-K x}
\end{equation}
\begin{equation}
I^{\downarrow}=h_{5}^{\prime} e^{-K x}+h_{6}^{\prime} e^{K x}+\frac{1}{c}\left[-\frac{1}{2 K}
 \frac{h_{1}}{\overline{\mu^{2}}}\left(p_{3} x+p_{4} \frac{1}{2 K}\right)-d\right] e^{-K x}
\end{equation}
其中:
\begin{equation}
h_{2}^{\prime}=\left(m_{3} p_{2}-m_{2} n_{3}\right) /\left(m_{1} p_{2}-m_{2} p_{1}\right)
\end{equation}
\begin{equation}
h_{3}^{\prime}=\left(m_{3} p_{1}-m_{1} n_{3}\right) /\left(m_{2} p_{1}-m_{1} p_{2}\right)
\end{equation}
\begin{equation}
h_{5}^{\prime}=h_{2}^{\prime} p_{1} / c
\end{equation}
\begin{equation}
h_{6}^{\prime}=h_{3}^{\prime} p_{2} / c
\end{equation}
\begin{equation}
m_{1}=\left(1-\alpha_{g, dif} p_{1} / c\right) s_{1}
\end{equation}
\begin{equation}
m_{2}=\left(1-\alpha_{g, dif } p_{2} / c\right) s_{1}
\end{equation}
\begin{equation}
\begin{aligned} 
m_{3} &=\frac{h_{1}}{\bar{\mu}^{2}}\left(L_{AI}+\frac{1}{2 K}\right) s_{2} \\ 
           &+\alpha_{g, dif}\frac{1}{c}\left[-\frac{1}{2 K} \frac{h_{1}}{\bar{\mu}^{2}}\left(p_{3} L_{AI}+p_{4} \frac{1}{2 K}\right)-d\right] s_{2} \\ 
           &+\alpha_{g, dir} s_{2} 
 \end{aligned}
\end{equation}
\begin{equation}
n_{3}=\frac{1}{4 K^{2}} \frac{h_{1}}{\bar{\mu}^{2}} p_{4}+d
\end{equation}
植被吸收的辐射通量为:
\begin{equation}
s_{v, dir}=1-\alpha_{veg, dir}-\left(1-\alpha_{g, dif}\right) \tau_{veg, dir}-\left(1-\alpha_{g, dir}\right) s_{2}
\end{equation}
\begin{equation}
s_{v, dif}=1-\alpha_{veg, dif}-\left(1-\alpha_{g, dif}\right) \tau_{veg, dif}
\end{equation}
\citet{dai2004two} 将植被分为阴叶和阳叶来计算其各自辐射通量吸收,对于直射入射光,阳叶、阴叶吸收辐射通量分别如下:
\begin{equation}
s_{sun, dir}=(1-\omega)\left[1-s_{2}+\frac{1}{\bar{\mu}}\left(a_{1}+a_{2}\right)\right]
\end{equation}
\begin{equation}
s_{sha, dir}=s_{v, dir}-s_{sun, dir}
\end{equation}
其中:
\begin{equation}
a_{1}=\frac{h_{1}}{\sigma}\left[\frac{1-s_{2}^{2}}{2 K}\right]+h_{2}\left[\frac{1-s_{2} s_{1}}{K+h}\right]+h_{3}\left[\frac{1-s_{2} / s_{1}}{K-h}\right]
\end{equation}
%
\begin{equation}
a_{2}=\frac{h_{4}}{\sigma}\left[\frac{1-s_{2}^{2}}{2 K}\right]+h_{5}\left[\frac{1-s_{2} s_{1}}{K+h}\right]+h_{6}\left[\frac{1-s_{2} / s_{1}}{K-h}\right]
\end{equation}
对于漫射光源,阳叶、阴叶吸收辐射通量分别如下:
\begin{equation}
s_{sun,dif}=\left[\frac{1-\omega}{\bar{\mu}}\right]\left(a_{1}+a_{2}\right)
\end{equation}
\begin{equation}
s_{sha, dir}=s_{v, dir}-s_{sun, dir}
\end{equation}
其中:
\begin{equation}
a_{1}=h_{7}\left[\frac{1-s_{2} s_{1}}{K+h}\right]+h_{8}\left[\frac{1-s_{2} / s_{1}}{K-h}\right]
\end{equation}
%
\begin{equation}
a_{2}=h_{9}\left[\frac{1-s_{2} s_{1}}{K+h}\right]+h_{10}\left[\frac{1-s_{2} / s_{1}}{K-h}\right]
\end{equation}
注意,以上公式对于可见光波段和近红外波段均成立,为了简化公式表达形式,波段标识信息均已省略。

%当植被部分面积比例($wt$,参见章节 \ref{无植被覆盖地表湍流通量的计算方案})被雪掩埋时,则植被的反照率修改为
%\begin{equation}
%\alpha_{veg}=(1-w t) \alpha_{veg}+w t \cdot \alpha_{sno}
%\end{equation}
%不同波段及直射/漫射情景均按以上加权方式计算。考虑植被覆盖度($f_{veg}$)时的地表反照率为
%%
%\begin{equation}\label{alpha1}
%\alpha=\left(1-f_{veg}\right) \alpha_{g}+f_{veg} \alpha_{veg}
%\end{equation}
%同样,以上公式适用于可见光和近红外波段,以及直射和漫射入射辐射情况。

\section{改进的二流近似植被辐射传输模型}\label{sec:改进的二流近似植被辐射传输模型}
根据 \citet{yuan2017reexamination} 对目前二流近似植被辐射传输参数化方案的对比分析,对章节~\ref{植被反照率二流近似模型} 方案进行改进,主要包括两个方面:
\begin{enumerate}
    \item 入射漫射辐射计算;
    \item 向上散射系数$\beta_0$ [原公式(\ref{beta0})] 计算。
\end{enumerate}

对于球形叶倾角分布($\chi_L=0$)情景,植被在漫射光入射时的透射率表示为:
\begin{equation}
T_{d}^{*}=2 \int_{0}^{1} \exp \left(-\frac{G(\mu) \rm LAI}{\mu}\right) \mu d \mu, \qquad G(\mu)=0.5
\end{equation}
由于以上公式没有解析解,通过不完全伽玛函数展开式,以上公式可近似表达为:
\begin{equation}
T_{d}^{*} \approx \frac{\exp (-0.5 a \rm LAI)}{1+0.5 b \rm LAI}
\end{equation}
通过与数值积分结果拟合得到$a=0.87$, $b=0.92$ (对于$0<\rm LAI<8$,RMSE=0.002)。因此,等效的漫射辐射入射角度(类似于直射辐射入射角度)可表示为:
\begin{equation}
\exp \left(-\frac{0.5 \rm LAI}{\mu^{*}}\right)=\frac{\exp (-0.5 a \rm LAI)}{1+0.5 b \rm LAI}
\end{equation}
即:
\begin{equation}
\mu^{*}=-0.5 \rm LAI \cdot \ln ^{-1} \frac{\exp (-0.5 a \rm LAI)}{1+0.5 b \rm LAI}
\end{equation}

对于非球形叶倾角分布情况,$\mu^\ast$可修订为:
\begin{equation}
\mu^{*}=\cos \left(\operatorname{acos}\left(\mu^{*}\right)+5 \chi_{L}\right)
\end{equation}
上式右边$\operatorname{acos}\left(\mu^{*}\right)$为球形叶倾角分布时的结果(单位表示为角度$^{\circ}$)。


对于向上散射系数$\beta_0$,CoLM2014是沿用 \citet{dickinson1983land} 和 \citet{sellers1985canopy} 方案,其中有一个较不合理的假设是认为叶子的体散射为均一散射,这对入射直射辐射引入一定的误差。在新版CoLM中,采用SAIL模型计算方案:
\begin{equation}
\omega \beta_{0}=\frac{1}{2}\left[\omega+\frac{\mu}{G(\mu)} \delta \int_{0}^{\pi / 2} 
f\left(\theta_{l}\right) \cos ^{2}\left(\theta_{l}\right) \sin \left(\theta_{l}\right) d \theta_{l}\right]
\end{equation}
其中$f\left(\theta_l\right)$为叶倾角分布函数。因为在CLM中叶倾角分布是用$\chi_l$参数来描述,上式中的积分表达式无法算出,这里沿用CLM的计算方法,使用等效(平均)叶倾角$\bar{\theta_l}$来计算,即:
\begin{equation}
\omega \beta_{0}=\frac{1}{2}\left[\omega+\frac{\mu}{G(\mu)} \delta \cos ^{2}\left(\overline{\theta_{l}}\right)\right]
\end{equation}
上式中$\delta=\rho_l-\tau_l$。平均叶倾角近似计算为:
\begin{equation}
\cos \left(\overline{\theta_{l}}\right)=\frac{1+\chi_{L}}{2}
\end{equation}
新改进的二流近似模型在实际应用中先假设土壤反照率为0 (black background),从而计算在入射直射(漫射)辐射时的直射透射率$T_d$ ($T_d^\ast$),漫射透射率$T_i$ ($T_i^\ast$),反照率$\alpha$ ($\alpha^\ast$) 以及植被吸收率$A$ ($A^\ast$)。
根据以上结果,考虑土壤反照率不为0时的植被与土壤之间的多次散射/吸收过程,此时从土壤反射的辐射考虑为漫射辐射,采用以上描述的改进的二流近似方案进行处理。在植被-土壤多次反射达到平衡时,总的植被透射率计算为:
\begin{equation}
[T]=\frac{T}{1-q}
\end{equation}
上式右边$T$为土壤反射率为0时总的透射率(直射入射情景 $T=T_d+T_i$,漫射入射情景 $T=T_d^\ast+T_i^\ast$),$q=r_g\alpha^\ast$。
对于直射入射情景,总的植被吸收为$\left[A\right]=A+r_g\left[T\right]A^\ast$,
总的土壤吸收为$\left[G\right]=\left(1-r_g\right)\left[T\right]$,
总的反照率$\left[\alpha\right]=1-\left[A\right]-\left[G\right]$。漫射入射时计算方式类似。
新方案同样会改变阴阳面辐射吸收比例,其中直射入射辐射阳面吸收为:
\begin{equation}
s_{sun,dir}=(1-\omega)\left[1-s_{2}+\frac{1}{\bar{\mu}}\left(a_{1}+a_{2}\right)\right]
\end{equation}
\begin{equation}
s_{sha,dir}=s_{v,dir}-s_{sun,dir}
\end{equation}
对于漫射入射辐射:
\begin{equation}
s_{sun,dif}=(1-\omega)\left[K\left(1-s_{2} s_{2}^{\prime}\right)+\frac{1}{\bar{\mu}}\left(a_{1}+a_{2}\right)\right]
\end{equation}
\begin{equation}
s_{sha,dif}=s_{v,dif}-s_{sun,dif}
\end{equation}
其中:
\begin{equation}
a_{1}=\frac{h_{1}}{\sigma}\left[\frac{1-s_{2} s_{2}^{\prime}}{K+K^{\prime}}\right]+h_{2}\left[\frac{1-s_{2}^{\prime} 
s_{1}}{K^{\prime}+h}\right]+h_{3}\left[\frac{1-s_{2}^{\prime} / s_{1}}{K^{\prime}-h}\right]
\end{equation}
\begin{equation}
a_{2}=\frac{h_{4}}{\sigma}\left[\frac{1-s_{2} s_{2}^{\prime}}{K+K^{\prime}}\right]+h_{5}\left[\frac{1-s_{2}^{\prime} s_{1}}
{K^{\prime}+h}\right]+h_{6}\left[\frac{1-s_{2} / s_{1}}{K^{\prime}-h}\right]
\end{equation}
当为入射直射辐射时,$s_2^\prime=s_2$、$K^\prime=K$;当为入射漫射辐射时$s_2^\prime$、$K^\prime$为漫射辐射对应的等效直射角度时的$s$和$K$值。


对于土壤反射部分的漫射光源在阳叶面的吸收不同于天空漫射光源在阳叶面的吸收,计算为:
\begin{equation}
s_{s u n, dif}=s_{2}^{\prime}(1-\omega)\left[K\left(1-\frac{s_{2}}{s_{2}^{\prime}}\right)
\left(K-K^{\prime}\right)+\frac{1}{\bar{\mu}}\left(a_{1}+a_{2}\right)\right]
\end{equation}
其中:
\begin{equation}
a_{1}=\frac{h_{1}}{\sigma}\left[\frac{1-s_{2} / s_{2}^{\prime}}{K-K^{\prime}}\right]+h_{2}\left[\frac{1-s_{1} /
 s_{2}^{\prime}}{h-K^{\prime}}\right]+h_{3}\left[\frac{1 / s_{1} / s_{2}^{\prime}-1}{h+K^{\prime}}\right]
\end{equation}
\begin{equation}
a_{2}=\frac{h_{4}}{\sigma}\left[\frac{1-s_{2} / s_{2}^{\prime}}{K-K^{\prime}}\right]+h_{5}\left[\frac{1-s_{1} / 
s_{2}^{\prime}}{h-K^{\prime}}\right]+h_{6}\left[\frac{1 / s_{1} / s_{2}^{\prime}-1}{h+K^{\prime}}\right]
\end{equation}
另外,相比于CoLM2014,植被的光学属性不仅考虑叶片的影响,同时也加入了茎的影响,``叶片'' 等效的光学属性为叶片
和茎光学属性按照其各自面积($\rm LAI$, $\rm SAI$)的加权平均(在上述文档中为了简化,保留原来$\rho_l$和$\tau_l$符号,
但其物理含义已是叶片和茎光学属性的加权值,即等效叶片光学属性)。


\section{三维植被辐射传输模型}\label{三维植被辐射传输模型}
\begin{mymdframed}{代码}
本章节对应的代码为\texttt{MOD\_3DCanopyRadiation.F90}。
\end{mymdframed}

三维植被辐射传输模型\citep{yuan20143d}是在单棵树冠模型\citep{dickinson2008determination,dickinson2008three}的基础上建立起来的(图~\ref{fig:三维植被辐射传输模型的基本框架})。
利用单棵树冠模型来建立单层植被模型,考虑了植被阴影的影响、植被树冠之间的散射吸收及低入射角度(太阳高度角)对反照率的影响。通过考虑层与层之间阴影相互重叠;
并利用单层植被模型的结果对三层植被辐射传输进行计算,来构建三层植被模型。
{
\begin{figure}[htbp]
\centering
\includegraphics[width=0.95\columnwidth]{Figures/辐射过程及辐射通量计算/三维植被辐射传输模型基本框架.png}
\caption{三维植被辐射传输模型计算基本框架示意图}
\label{fig:三维植被辐射传输模型的基本框架}
\end{figure}
}


\subsection{单棵树冠模型}
单棵树冠模型基本假设是把植被树冠看成一个个的圆形球体,
叶片在其中均一分布,且叶倾角为球形分布(即$G$函数为1/2,$G$函数为单位体积叶面在某方向上的平均投影面积)。
首先考虑太阳辐射垂直入射,计算单棵球形植被的直射透射率$T_{d,s}\left(\tau\right)$:
{
\begin{figure}[htbp]
\centering
\includegraphics[width=0.6\columnwidth]{Figures/辐射过程及辐射通量计算/单棵树冠示意图.png}
\caption{单棵树冠示意图}
\label{fig:单棵树冠示意图}
\end{figure}
}
%
\begin{equation}\label{T_ds_tau}
T_{d, s}(\tau)=0.5 \tau^{-2}\left[1-(1+2 \tau) e^{-2 \tau}\right]
\end{equation}
$\tau$为沿球形植被半径长度的光学厚度参数(对于单棵树冠记为$\tau_0$,图~\ref{fig:单棵树冠示意图})。对单次散射的前向、后向散射相函数($\Phi_{1f}\left(\tau\right)$,$\Phi_{1b}\left(\tau\right)$)计算为:
\begin{align}
\Phi_{1 b}(\tau)=0.5\left[1-T_{d, s}(2 \tau)\right]
\Phi_{1 f}(\tau)=\tau^{-2}\left[1-\left(1+2 \tau+2 \tau^{2}\right) e^{-2 \tau}\right]
\end{align}
上式$\Phi_{1b}$和$\Phi_{1f}$为归一化结果(除以$0.25\omega^2/\pi$)。
二次散射前向、后向散射相函数($\Phi_{2f}\left(\tau\right), \Phi_{2b}\left(\tau\right)$)计算为:
\begin{align}
\Phi_{2 b}(\tau) & =a\left[\frac{1}{b+1}-\frac{1}{b-1} T_{d, s}(2 \tau)+\frac{2}{(b+1)(b-1)} T_{d, s}((b+1) \tau)\right] \\[2ex]
%
\Phi_{2 f}(\tau) & =a\left[\frac{2 b}{b^{2}-1} \Phi_{1 f}(\tau)-\left(\frac{1}{(b+1)^{2}}+\frac{1}{(b-1)^{2}}\right) T_{d, s}(\tau)+\frac{1}{(b-1)^{2}} T_{d, s}(b \tau)\right. \notag \\[1ex] 
&\qquad \left.\quad+\frac{1}{(b+1)^{2}} T_{d, s}((b+2) \tau)\right]
\end{align}
其中$a=0.70$,$b=1.74$,以上两式也为均一化的结果(除以$0.25\omega/\pi$)。假设单次和二次散射相函数从前向到后向强度线性变化,则总的单次散射和二次散射辐射计算为:
\begin{equation}
\Phi_{1 a}(\tau)=0.5\left[\Phi_{1 b}(\tau)+\Phi_{1 f}(\tau)\right]
\end{equation}
\begin{equation}
\Phi_{1 a}(\tau)=0.5\left[\Phi_{1 b}(\tau)+\Phi_{1 f}(\tau)\right]
\end{equation}
对于三次及以上散射相函数($\Phi_{3+}\left(\tau\right)$),假设其各向均一,利用二次散射后再次(与植被)碰撞的概率($p_2\left(\tau\right)$)进行计算:
\begin{equation}
\Phi_{3+}(\tau)=\frac{\omega p_{2}(\tau) \Phi_{2 a}(\tau)}{1-\omega p_{2}(\tau)}
\end{equation}
其中
\begin{equation}
p_{2}(\tau)=1-\frac{\Phi_{2 a}(\tau)}{1-T_{d, s}(\tau)-\Phi_{1 a}(\tau)}
\end{equation}
$\Phi_{3+}$为归一化结果(除以$0.25\omega^2/\pi$)。通过累加得到总的前向散射相函数($\Phi_f\left(\tau\right)$)及后向散射相函数($\Phi_b\left(\tau\right)$):
\begin{equation}
\Phi_{b}(\tau)=\frac{\omega}{4 \pi} \Phi_{1 b}(\tau)+\frac{\omega^{2}}{4 \pi}\left[\Phi_{2 b}(\tau)+\Phi_{3+}(\tau)\right]
\end{equation}
\begin{equation}
\Phi_{f}(\tau)=\frac{\omega}{4 \pi} \Phi_{1 f}(\tau)+\frac{\omega^{2}}{4 \pi}\left[\Phi_{2 f}(\tau)+\Phi_{3+}(\tau)\right]
\end{equation}
假设散射相函数的分布由后向到前向,沿天顶角呈线性变化(方位角水平均一),对任意散射出射角度$\theta_{out}$ (天顶角方向,$\mu_{out}=\cos{\theta_{out}}$),其总的散射相函数计算为:
\begin{equation}
\Phi_{\mu_{\mathrm{out}}}(\tau)=\Phi_{a}(\tau)+\mu_{\mathrm{out}} \Phi_{d}(\tau)
\end{equation}
其中$\Phi_a\left(\tau\right)=0.5\left[\Phi_b\left(\tau\right)+\Phi_f\left(\tau\right)\right]$,$\Phi_d\left(\tau\right)=0.5\left[\Phi_b\left(\tau\right)-\Phi_f\left(\tau\right)\right]$。
当太阳入射天顶角为$\theta$ ($\mu=\cos{\theta}$)时,对$\Phi_{\mu_{out}}\left(\tau\right)$进行积分计算,得到单株植物的反照率($\alpha_s\left(\mu,\tau\right)$)及漫射透射率($T_{i,s}\left(\mu,\tau\right)$)为:
\begin{equation}
\alpha_{s}(\mu, \tau)=2 \pi\left[\Phi_{a}(\tau)+0.5 \mu \Phi_{d}(\tau)\right]
\end{equation}
\begin{equation}
T_{i, s}(\mu, \tau)=2 \pi\left[\Phi_{a}(\tau)-0.5 \mu \Phi_{d}(\tau)\right]
\end{equation}
对于漫射辐射入射,模型考虑等效为直射入射为 60\textdegree 时的情形。


\subsection{单层植被模型}
单层植被模型假设球形树冠在水平方向上随机分布,但垂直投影不重叠。在植被覆盖率为$f_c$时,不考虑重叠的植被阴影面积为:
\begin{equation}
S_{0}=f_{c} / \mu
\end{equation}
若植被树冠的阴影可以随机重叠,根据概率理论模型,可以计算阴影面积为$1-e^{-f_c/\mu}$。但考虑在实际中,植被树冠在垂直投影上一般不重叠,修正的阴影面积为:
\begin{equation}\label{S_area}
S=\frac{1-e^{-f_{c} / \mu}}{1-f_{c} e^{-1 / \mu}}
\end{equation}
树冠阴影的重叠不仅直接影响阴影面积的大小,还增加了辐射穿过植被冠层的光学路径,单层植被的光学厚度把原有的单株植物光学厚度$\tau_0$等效修改为:
\begin{equation}\label{tau}
\tau=\tau_{0} S_{0} / S
\end{equation}
随着$f_c$的增加,植被树冠之间的散射吸收增加,为了计算这部分辐射,考虑最为简单的正六边形植被树冠分布(图~\ref{fig:散射吸收计算示意图}),容易计算得到红色树冠相对于绿色树冠的可视因子为:
\begin{equation}
V_{1}=\frac{1}{2}\left(1-\cos \theta_{v}\right)=\frac{1}{2}\left[1-\sqrt{1-\left(\frac{R}{L_{1}}\right)^{2}}\right]
\end{equation}
其中$\left(\frac{R}{L_{1}}\right)^{2}=\frac{\sqrt{3} f_{c}}{2 \pi}$。
考虑中心球周围6个树冠及稍远6个树冠对其散射辐射的吸收,可以得到总的可视因子为:
\begin{equation}
V=6 V_{1}+6 V_{2}=3\left[1-\sqrt{1-\frac{\sqrt{3} f_{c}}{2 \pi}}\right]+3\left[1-\sqrt{1-\frac{\sqrt{3} f_{c}}{6 \pi}}\right]
\end{equation}
假设可视立体角(在各个方位角方向)均一分布在中心球形树冠平行于地面的大圆上下$\pm\mu_v$的角度范围内,于是$\mu_v=V$。对球形树冠的散射相函数进行积分得到初次拦截的散射辐射为:
\begin{equation}
\frac{1}{2 \pi} \int_{0}^{2 \pi} \int_{-\mu_{v}}^{\mu_{v}} 2 \pi\left(\Phi_{a}+\mu \Phi_{d}\right) d \mu d \phi=4 \pi V \Phi_{a}
\end{equation}
利用再次碰撞概率概念估算总的树冠之间散射吸收为:
\begin{equation}
A_{c}=4 \pi V \Phi_{a}\left(\tau_{0}\right)\left[1-T_{d, s}\left(\tau_{0}\right)\right] \frac{1-\omega}{1-\omega p_{2}\left(\tau_{0}\right)}
\end{equation}
其中$\omega$为叶片单次散射反照率。

{
\begin{figure}[htbp]
\centering
\includegraphics[width=0.95\columnwidth]{Figures/辐射过程及辐射通量计算/散射吸收计算示意图.png}
\caption{植被树冠之间散射吸收计算示意图}
\label{fig:散射吸收计算示意图}
\end{figure}
}

当太阳在较低的角度(高度角)入射时,植被树冠阴影重叠概率增大,树冠的光照面主要集中在顶部,此时的植被冠层类似于一维的情形,相对于球形植被计算结果,$albedo$会有所增加。
我们利用 \citet{dickinson1983land} 对一维植被单次散射的分析结果,计算对$albedo$的增量为:
\begin{equation}
\alpha_{L}=(N-1) \alpha_{s}\left(\mu=1, \tau_{0}\right) f_{c}\left(\frac{1}{s}-\frac{1}{s_{0}}\right)
\end{equation}
其中$N$表示相对于垂直入射植被冠层时$albedo$的倍数,是太阳入射角度的函数:
\begin{equation}
N=\frac{1+2 \beta_{2}}{1+2 \beta_{2} \mu}
\end{equation}
其中$\beta_2=\left(1-\omega\right)^\frac{1}{2}(1-\omega+2\omega\beta)$。
植被树冠之间的吸收增量会使$albedo$和漫射透射减少,低入射角度对$albedo$的增量会降低植被的吸收和漫射透射,把以上增加的部分均一分配到减少项,
得到单层植被$albedo$ ($\alpha$),植被直射透射($T_d$)漫射透射($T_i$)及吸收($A$)为:
\begin{equation}
\alpha=\alpha_{s}(\mu, \tau)+\alpha_{L}-0.5 A_{c}
\end{equation}
\begin{equation}
T_{i}=T_{i, s}(\mu, \tau)-0.5 \alpha_{L}-0.5 A_{c}
\end{equation}
\begin{equation}
A=1-\alpha-T_{i}-T_{d}
\end{equation}
对于漫射辐射入射,同样考虑等效为直射入射为 60\textdegree 的情形。

以上的推导均基于正球体和球形叶倾角分布函数情况下,下面将对椭球体和非球形叶倾角分布进行订正。

\textit{1) 椭球树冠}

对于椭球体树冠(图~\ref{fig:椭球体树冠}a),其对辐射计算的影响主要有3个方面(均可等效看成对光学厚度$\tau$的影响):
\begin{enumerate}
    \item 椭球相对于正球的拉升/压缩影响单位长度的光学厚度(因叶密度改变);
    \item 几何形状的改变造成光学路径的长度的改变;
    \item 阴影面积重合部分改变对光学厚度的影响。
\end{enumerate}

 {
    \begin{figure}[htbp]
    \centering
    \includegraphics[width=0.8\columnwidth]{Figures/辐射过程及辐射通量计算/椭球体树冠.png}
    \caption{(a) 树冠形状从正球体扩展到椭球体;(b) 考虑叶倾角分布}
    \label{fig:椭球体树冠}
    \end{figure}
    }

椭球体可能看成是正球体在垂直方向$y$轴的拉伸,在椭球坐标系下的$y$,
在正球体坐标系下就变成$y^\prime$。让$c$为椭球体半轴长(沿$y$轴),$R$为球体水平方向的半径,可以得到:
\begin{equation}
y^{\prime}=\frac{R}{c} y, \text { and } x^{\prime}=x
\end{equation}
于是有$\tan{\theta_s^\prime}=\frac{x^\prime}{y^\prime}=\frac{c}{R}{\tan{\theta}}_s$,
椭球体在地面的投影面积即为$\pi R^2/\cos{\theta_s^\prime}$。将$\theta_s^\prime$进行替换,得到阴影面积计算公式为:
\begin{equation}\label{SSS}
\begin{aligned} S &=\frac{\pi R^{2}}{\cos \theta_{s}^{\prime}} \\ &=\frac{\pi R^{2}}{\cos \theta_{s} 
\sqrt{\frac{1}{c r^{2} \sin ^{2} \theta_{s}+\cos ^{2} \theta_{s}}}} \end{aligned}
\end{equation}
式中$cr=c/R$。可以看出,当$c=R$时,与正球体计算一致。$\cos{\theta_s}$与$\cos{\theta_s^\prime}$的关系从公式 (\ref{SSS}) 很容易看出。%caution


椭球体对$\tau$影响的三个方面,我们用$c_1$, $c_2$, $c_3$三个系数表示。对于$c_1$,很容易得到$c_1=1/cr$。$c_2$的计算稍微复杂一点,可表达为:
\begin{equation}
\begin{aligned} c_2 &=\frac{y}{y^{\prime}} \cdot \frac{\cos \theta_{s}^{\prime}}{\cos \theta_{s}}=
    c r \sqrt{\frac{1}{c r^{2} \sin ^{2} \theta_{s}+\cos ^{2} \theta_{s}}} \\ &=
    \sqrt{\frac{c^{2}}{c^{2} \sin ^{2} \theta_{s}+R^{2} \cos ^{2} \theta_{s}}} \end{aligned}
\end{equation}
对于$c_3$,实际上是计算在等效的天顶角$\theta_s^\prime$下,考虑阴影重叠后的等效$\tau$,其计算表达式在公式(\ref{tau})已给出。

\textit{2) 非球形叶倾角分布}

对于球形分布函数,前面已提到$G=0.5$;对于参数为$\chi_L$的叶倾角分布,
可以利用公式(\ref{Gmu})计算得到特定太阳入射角的$G\left(\theta_s\right)$。
对于光学厚度相对原始$\tau$的比例系数为$c_4=G\left(\theta_s\right)/0.5$。

综合以上几个方面,即把光学厚度参数$\tau$修改为$c_1c_2c_3c_4\tau$,
然后代入直射透射计算公式,便可得到单层植被的直射透射率$T_d^\prime$,
假设在正球体情况下的直射透射率为$T_d$,在不考虑椭球形状和叶倾角分布情况下,能量守恒等式表达为:
\begin{equation}
\alpha+A+T_{d}+T_{i}=1
\end{equation}
定义散射/吸收辐射的订正因子为$\frac{1-T_d^\prime}{1-T_d}$,把由直射透射引起的截获辐射量变化考虑为$S$和$A$的改变,即分别修正为$\frac{1-T_d^\prime}{1-T_d}S$和$\frac{1-T_d^\prime}{1-T_d}A$,新修正的各个变量同样满足能量守恒方程:
\begin{equation}
\frac{1-T_{d}^{\prime}}{1-T_{d}} \alpha+\frac{1-T_{d}^{\prime}}{1-T_{d}} A+T_{d}^{\prime}+\frac{1-T_{d}^{\prime}}{1-T_{d}} T_{i}=1
\end{equation}
再考虑对单层植被辐射散射/吸收特性的改变之外,树冠的形状改变还影响地面的阴影计算,对于单层植被,可以利用等效的$\cos{\theta_s^\prime}$值(即$\cos{\theta_s}\sqrt{\frac{1}{{cr}^2\sin^2{\theta_s}+\cos^2{\theta_s}}}$)。 
若考虑植被与地面之间的多次散射,初次透射到地面的辐射$T$ ($T=1-S+S\left(T_d+T_i\right)$) 经过地面的反射(地表反射率$r_g$),再次与植被冠层反射到达地面,如此反复。到达地面的辐射为一等比数列,其比例因子为$q=r_gS^\ast\alpha^\ast$,上标“$\ast$”表示漫射辐射入射时相应变量值(下同)。
由于地表与植被之间多次散射的植被吸收$A^m=\frac{Tr_gS^\ast A^\ast}{1-q}$,总的植被吸收($\left[A\right]$),透射率($\left[T\right]$),地表吸收($\left[G\right]$)和地表$albedo$ ($\left[\alpha\right]$)计算如下:
\begin{equation}
[A]=S A+A^{m}
\end{equation}
\begin{equation}
[T]=\frac{T}{1-q}
\end{equation}
\begin{equation}
[G]=\left(1-r_{g}\right)[T]
\end{equation}
\begin{equation}
[\alpha]=1-[A]-[G]
\end{equation}


\subsection{三层植被模型}
三层植被模型的假设在单株植物模型和单层植被模型的基础上,认为层与层之间的树冠垂直投影不重叠,
每层植被树冠半径$(R$)和树冠中心高度($h$)相同(图~\ref{fig:三层植被结构示意图})。
若某层植被含有多种PFT,则该层的$f_c$为所含PFT各自$f_c$的累加,
植被参数的层属性(包括$LAI$、$R$、$h$及叶片光学属性)为所含PFT的$f_c$加权平均。

层与层之间的阴影重叠对直射辐射的吸收影响很大,特别是对于较高吸收率的可见光波段。
对于阴影的重叠,3D模型假设高层树冠的“自身阴影”(在低层植被的阴影与其自身垂直投影的重叠区域)
不与低层树冠的阴影重叠(如图~\ref{fig:三层植被结构示意图}所示的$S_{12}$、$S_{13}$和$S_{23}$)。除此之外,层与层之间的阴影可以随机重叠。%
{
\begin{figure}[htbp]
\centering
\includegraphics[width=0.8\columnwidth]{Figures/辐射过程及辐射通量计算/三层植被结构示意图.png}
\caption[三层植被直射辐射传输示意图]{三层植被直射辐射传输示意图。[注:图中树冠大小和形状仅作为示意,每层植被树冠大小不必一致,形状不必为正球体]}
\label{fig:三层植被结构示意图}
\end{figure}
}
用$I_{\left[from\right]\rightarrow\left[to\right]}$表示层与层之间的直射辐射直射透射量,
$S[n]$表示第$n$层植被的阴影面积,$Td$,$[n]$为第$n$层植被的直射辐射直射透射率,可以计算从天空($sky$),
不穿过植被,到达每层植被及地面的初始直射辐射为(为了与三维植被辐射传输模型引文~\citep{yuan20143d} 吻合
,这里保留了植被分层计数从上到下1-3的方式,但之后三维植被长波辐射传输
、三维植被湍流交换过程等采用从上至下3-1的编号,与代码一致):
\begin{equation}
\begin{aligned} I_{0 \rightarrow 1} &=S_{1} \\ I_{0 \rightarrow 2} &=\left(1-S_{1}+S_{12}\right) S_{2} \\ 
    I_{0 \rightarrow 3} &=\left[1-\left(S_{1}-S_{13}\right)-\left(S_{2}-S_{23}\right)+\left(S_{1}-S_{12}\right)\left(S_{2}-S_{23}\right)\right] S_{3} \\ 
    I_{0 \rightarrow g} &=1-S_{1}-S_{2}-S_{3} \\ &+\left(S_{1}-S_{12}\right) S_{2}+\left(S_{1}-S_{13}\right) S_{3}+\left(S_{2}-S_{23}\right) S_{3} \\
     &-\left(S_{1}-S_{12}\right)\left(S_{2}-S_{23}\right) S_{3} \end{aligned}
\end{equation}
到达L1的直射辐射可以继续直射透射到达下层,计算为:
\begin{equation}
\begin{aligned}
I_{1 \rightarrow 2} &=T_{d, 1}\left(S_{1}-S_{12}\right) S_{2} \\[1ex]
I_{1 \rightarrow 3}&=T_{d, 1}\left[S_{1}-S_{13}-\left(S_{1}-S_{12}\right)\left(S_{2}-S_{23}\right)\right] S_{3} \\[1ex]
I_{1 \rightarrow g} &=T_{d, 1}\left[S_{1}-\left(S_{1}-S_{12}\right) S_{2}-\left(S_{1}-S_{13}\right) S_{3}+\left(S_{1}-S_{12}\right)\left(S_{2}-S_{23}\right) S_{3}\right]
\end{aligned}
\end{equation}
同理可以得到:
\begin{equation}
\begin{aligned}
I_{2 \rightarrow 3} & =T_{d, 2}\left[I_{0 \rightarrow 2}+I_{1 \rightarrow 2}\right] \frac{\left(S_{2}-S_{23}\right) S_{3}}{S_{2}} \\[1ex]
I_{2 \rightarrow g} & =T_{d, 2}\left[I_{0 \rightarrow 2}+I_{1 \rightarrow 2}\right] \frac{S_{2}-\left(S_{2}-S_{23}\right) S_{3}}{S_{2}} \\[1ex] 
I_{3 \rightarrow g} &=T_{d, 3}\left[I_{0 \rightarrow 3}+I_{1 \rightarrow 3}+I_{2 \rightarrow 3}\right]
\end{aligned}
\end{equation}

累加以上结果,可以计算到达每层的初始直射辐射为:
\begin{equation}
\begin{array}{l}I_{1}=I_{0 \rightarrow 1} \\ I_{2}=I_{0 \rightarrow 2}+I_{1 \rightarrow 2} \\ 
    I_{3}=I_{0 \rightarrow 3}+I_{1 \rightarrow 3}+I_{2 \rightarrow 3} \\ 
    I_{g}=I_{0 \rightarrow g}+I_{1 \rightarrow g}+I_{2 \rightarrow g}+I_{3 \rightarrow g}\end{array}
\end{equation}
第$n$层植被的直射辐射直射吸收量为:
\begin{equation}
I_{n}\left(1-T_{d, n}\right)(1-\omega)
\end{equation}
第$n$层植被的太阳直射照射面积($p_{sun}$)计算为:
\begin{equation}
p_{sun}=I_{n} / S_{n}
\end{equation}
上式表明每层植被并不是100\%的被太阳直射照射,该参数将会用于修正植被光照叶面积指数(图~\ref{fig:三层植被辐射传输计算示意图})。
{
\begin{figure}[htbp]
\centering
\includegraphics[width=0.7\columnwidth]{Figures/辐射过程及辐射通量计算/三层植被辐射传输计算示意图.png}
\caption{三层植被辐射传输计算示意图}
\label{fig:三层植被辐射传输计算示意图}
\end{figure}
}
三层植被模型的计算采用线性方程组进行计算。六个变量,$I_{\uparrow,1}^\ast$, $I_{\downarrow,1}^\ast$, $I_{\uparrow,2}^\ast$, $I_{\downarrow,2}^\ast$, $I_{\uparrow,3}^\ast$ 和 $I_{\downarrow,3}^\ast$,
被选择作为未知变量。他们分别表示每层植被的向上/向下漫射辐射通量,可以表示为:
\begin{equation}
\begin{aligned}
I_{\uparrow, 1}^{*} & =I_{1} \alpha_{1}+I_{\uparrow, 2}^{*} S_{1}^{*} T_{i, 1}^{*}+I_{\uparrow, 2}^{*}\left(1-S_{1}^{*}\right)+I_{0}^{*} S_{1}^{*} \alpha_{1}^{*} \\[1ex] 
I_{\downarrow, 1}^{*} &=I_{1} T_{i, 1}+I_{0}^{*} S_{1}^{*} T_{i, 1}^{*}+I_{0}^{*}\left(1-S_{1}^{*}\right)+I_{\uparrow, 2}^{*} S_{1}^{*} \alpha_{1}^{*} \\[1ex] 
I_{\uparrow, 2}^{*} &=I_{2} \alpha_{2}+I_{\uparrow, 3}^{*} S_{2}^{*} T_{i, 2}^{*}+I_{\uparrow, 3}^{*}\left(1-S_{2}^{*}\right)+I_{\downarrow, 1}^{*} S_{2}^{*} \alpha_{2}^{*} \\[1ex]
I_{\downarrow, 2}^{*} &=I_{2} T_{i, 2}+I_{\downarrow, 1}^{*} S_{2}^{*} T_{i, 2}^{*}+I_{\downarrow, 1}^{*}\left(1-S_{2}^{*}\right)+I_{\uparrow, 3}^{*} S_{2}^{*} \alpha_{2}^{*} \\[1ex]
I_{\uparrow, 3}^{*} &=I_{3} \alpha_{3}+\left(I_{\downarrow, 3}^{*}+I_{g}\right) r_{g} S_{3}^{*} T_{i, 3}^{*}+\left(I_{\downarrow, 3}^{*}+I_{g}\right) r_{g}\left(1-S_{3}^{*}\right)+I_{\downarrow, 2}^{*} S_{3}^{*} \alpha_{3}^{*} \\[1ex]
I_{\downarrow, 3}^{*} &=I_{3} T_{i, 3}+I_{\downarrow, 2}^{*} S_{3}^{*} T_{i, 3}^{*}+I_{\downarrow, 2}^{*}\left(1-S_{3}^{*}\right)+\left(I_{\downarrow, 3}^{*}+I_{g}\right) r_{g} S_{3}^{*} \alpha_{3}^{*}
\end{aligned}
\end{equation}
经过简单的变换,可以得到在直射入射辐射情况下:
\begin{equation}
\left(\begin{matrix}1&&-{\widetilde{T}}_1&&&\\&1&-{\widetilde{\alpha}}_1&&&\\&-{\widetilde{\alpha}}_2&1&&-{\widetilde{T}}_2&\\&-{\widetilde{T}}_2&&1&-{\widetilde{\alpha}}_2&\\&&&-{\widetilde{\alpha}}_3&1&-{\widetilde{T}}_3r_g\\&&&-{\widetilde{T}}_3&&1-r_g{\widetilde{\alpha}}_3\\\end{matrix}\right)\left(\begin{matrix}I_{\uparrow,1}^\ast\\I_{\downarrow,1}^\ast\\I_{\uparrow,2}^\ast\\I_{\downarrow,2}^\ast\\I_{\uparrow,3}^\ast\\I_{\downarrow,3}^\ast\\\end{matrix}\right)=\left(\begin{matrix}I_1\alpha_1\\I_1T_{i,1}\\I_2\alpha_2\\I_2T_{i,2}\\I_3\alpha_3+I_gr_g{\widetilde{T}}_3\\I_3T_{i,3}+I_gr_g{\widetilde{\alpha}}_3\\\end{matrix}\right)
\end{equation}
及在漫射辐射情况下:
\begin{equation}
\left(\begin{matrix}1&&-{\widetilde{T}}_1&&&\\&1&-{\widetilde{\alpha}}_1&&&\\&-{\widetilde{\alpha}}_2&1&&-{\widetilde{T}}_2&\\&-{\widetilde{T}}_2&&1&-{\widetilde{\alpha}}_2&\\&&&-{\widetilde{\alpha}}_3&1&-{\widetilde{T}}_3r_g\\&&&-{\widetilde{T}}_3&&1-r_g{\widetilde{\alpha}}_3\\\end{matrix}\right)\left(\begin{matrix}I_{\uparrow,1}^\ast\\I_{\downarrow,1}^\ast\\I_{\uparrow,2}^\ast\\I_{\downarrow,2}^\ast\\I_{\uparrow,3}^\ast\\I_{\downarrow,3}^\ast\\\end{matrix}\right)=\left(\begin{matrix}I_0^\ast{\widetilde{\alpha}}_1\\I_0^\ast{\widetilde{T}}_1\\0\\0\\0\\0\\\end{matrix}\right)
\end{equation}
其中${\widetilde{\alpha}}_n=S_n^\ast\alpha_n^\ast$及${\widetilde{T}}_n=S_n^\ast T_n^\ast-S_n^\ast+1$。以上两个方程组可以合并写为:
\begin{equation}
\mathbf{A x}=\mathbf{B}
\end{equation}
有很多方法去解以上方程组,根据方程组的特点,采用经过修改后的高斯消去法。对于直射入射辐射,其解为:
\begin{equation}
\begin{aligned}
\left [\alpha \right ] &=I_{\uparrow, 1}^{*} \\[1ex] \left[A_{1}\right] &=I_{1} A_{1}+I_{\uparrow, 2}^{*} S_{1}^{*} A_{1}^{*} \\[1ex] 
\left[A_{2}\right] &=I_{2} A_{2}+\left(I_{\downarrow, 1}^{*}+I_{\uparrow, 3}^{*}\right) S_{2}^{*} A_{2}^{*} \\[1ex]
\left[A_{3}\right] &=I_{3} A_{3}+\left[I_{\downarrow, 2}^{*}+\left(I_{g}+I_{\downarrow, 3}^{*}\right) r_{g}\right] S_{3}^{*} A_{3}^{*} \\[1ex]
[G] &=\left(I_{g}+I_{\downarrow, 3}^{*}\right)\left(1-r_{g}\right)
\end{aligned}
\end{equation}
对于漫射入射,其解为:
\begin{equation}
\begin{aligned}
\left[\alpha^{*}\right] &=I_{\uparrow, 1}^{*} \\[1ex]
\left[A_{1}^{*}\right]& =S_{1}^{*} A_{1}^{*}+I_{\uparrow, 2}^{*} S_{1}^{*} A_{1}^{*} \\[1ex]
\left[A_{2}^{*}\right] &=\left(I_{\downarrow, 1}^{*}+I_{\uparrow, 3}^{*}\right) S_{2}^{*} A_{2}^{*} \\[1ex]
\left[A_{3}^{*}\right] &=\left(I_{\downarrow, 2}^{*}+I_{\downarrow, 3}^{*} r_{g}\right) S_{3}^{*} A_{3}^{*} \\[1ex] 
\left[G^{*}\right] &=I_{\downarrow, 3}^{*}\left(1-r_{g}\right)
\end{aligned}
\end{equation}
所有的PFT (包括裸土)共享同样的反照率,同样的地面吸收率。对每一PFT,考虑其单独存在时,利用单层植被模型计算其吸收,
并将其作为权重,把每层的植被辐射吸收分配到其所包含的PFT中去(包括上面计算的直射辐射直射吸收)。

\subsection{三维植被长波辐射传输}\label{三维植被长波辐射传输}
\begin{mymdframed}{代码}
本小节对应的代码包含于\texttt{MOD\_LeafTemperaturePC.F90}。
\end{mymdframed}

为了说明三维植被情况下长波辐射传输与一维情况下最大的不同,我们举一个极端的例子加以说明,如图~\ref{fig:单棵树冠长波辐射传输示意图} 所示,
{
\begin{figure}[htb]
\centering
\includegraphics[width=0.5\textwidth]{Figures/辐射过程及辐射通量计算/单棵树冠长波辐射传输示意图.png}
\caption{三维长波辐射计算示意图(对于单棵树冠)}
\label{fig:单棵树冠长波辐射传输示意图}
\end{figure}
}
%
若采用一维长波辐射传输方式进行计算,则到达地面的大气长波辐射量即为$f_c\left(1-\varepsilon_v\right)L^\downarrow+\left(1-f_c\right)L^\downarrow$。
但实际上,由于长波辐射是朝各个方向发射的,一般认为各向均一(类似模式对短波漫射入射辐射假设),能到达树冠表面(或称为阴影)的长波辐射并不完全等同于其覆盖率(垂直投影面积)$f_c$,
简单的计算公式可以表示为:
\begin{equation}
S=\int_{2 \pi} \frac{\cos \theta}{\pi} \cdot \frac{f_{c}}{\cos \theta} \cdot d \Omega=2 f_{c}
\end{equation}
上式中$\cos{\theta}/\pi$为入射角度为$\theta$时的辐射强度,$f_c/\cos{\theta}$为此时的树冠阴影面积(地面投影),
总的阴影面积计算为$2f_c$。到达地面的大气长波辐射量即为:
\begin{equation}
S\left(1-\varepsilon_{v}\right) L ^\downarrow+(1-S) L ^\downarrow
\end{equation}
与一维不同之处是$f_c$替换成了$S$。当植被比较稀疏时,
$f_c\rightarrow0$,$S\rightarrow2$,同上述单棵树冠情况一样;
当植被比较致密时,$f_c\rightarrow1$,$S\rightarrow1$,同一维情形。
对从地面发射的长波辐射,其情况同向下大气长波辐射一样,在稀疏情况下,植被能截获更多的长波辐射。
下面我们对一层植被(植被覆盖度为$f_c$)进行整个长波辐射传输过程推导。植被冠层向下的长波辐射计算为:
\begin{equation}
\begin{aligned} 
L_{v} ^\downarrow &=S\left(1-\varepsilon_{v}\right) L ^\downarrow+(1-S) L ^\downarrow+f_{c} \varepsilon_{v} \sigma T_{v}^{4} \\ 
  &=\left(S-S \varepsilon_{v}+1-S\right) L ^\downarrow+f_{c} \varepsilon_{v} \sigma T_{v}^{4} \\ 
  &=\left(1-S \varepsilon_{v}\right) L ^\downarrow+f_{c} \varepsilon_{v} \sigma T_{v}^{4} 
\end{aligned}
\end{equation}
地面向上的长波辐射(包括自身发射和反射部分)计算为:
\begin{equation}
L_{g} ^\uparrow=\left(1-\varepsilon_{g}\right) L_{v} ^\downarrow+\varepsilon_{g} \sigma T_{g}^{4}
\end{equation}
地面的净长波辐射为:
\begin{equation}
{L_{g}}=\varepsilon_{g} L_{v} ^\downarrow -\varepsilon_{g} \sigma T_{g}^{4}
\end{equation}
植被净辐射通量为:
\begin{equation}
\begin{aligned} {L_{v}} &=-S \varepsilon_{v} L ^\downarrow-S \varepsilon_{v} L_{g} 
    ^\uparrow+2 f_{c} \varepsilon_{v} \sigma T_{v}^{4} \\ 
    &=-S \varepsilon_{v} L ^\downarrow-S 
    \varepsilon_{v}\left(1-\varepsilon_{g}\right)\left(1-S \varepsilon_{v}\right) L^\downarrow \\ 
    &\mathrel{\phantom{=}} -S \varepsilon_{v}\left(1-\varepsilon_{g}\right) f_{c} \varepsilon_{v} \sigma T_{v}^{4}+S
     \varepsilon_{v} \varepsilon_{g} \sigma T_{g}^{4}+2 f_{c} \varepsilon_{v} \sigma T_{v}^{4} \\ 
     &=\left[2-S \varepsilon_{v}\left(1-\varepsilon_{g}\right)\right] f_{c} \varepsilon_{v} 
     \sigma T_{v}^{4}-S \varepsilon_{v}\left[1+\left(1-\varepsilon_{g}\right)\left(1-S \varepsilon_{v}\right)\right]
      L ^\downarrow+S \varepsilon_{v} \varepsilon_{g} \sigma T_{g}^{4} \end{aligned}
\end{equation}
植被层顶向上的长波辐射计算为:
\begin{equation}
\begin{aligned} L_v ^\uparrow=&(1-S) L_{g} ^\uparrow+S\left(1-\varepsilon_{v}\right) L_{g} 
    ^\uparrow+f_{c} \varepsilon_{v} \sigma T_{v}^{4} \\=&\left(1-S \varepsilon_{v}\right) L_{g}
     ^\uparrow+f_{c} \varepsilon_{v} \sigma T_{v}^{4} \\=&\left(1-S \varepsilon_{v}\right)\left(1-\varepsilon_{g}\right)\left(1-S \varepsilon_{v}\right) L^\downarrow \\
      & \quad +\left(1-S \varepsilon_{v}\right)\left(1-\varepsilon_{g}\right) f_{c} \varepsilon_{v} \sigma T_{v}^{4}+\left(1-S \varepsilon_{v}\right) 
      \varepsilon_{g} \sigma T_{g}^{4}+f_{c} \varepsilon_{v} \sigma T_{v}^{4} \end{aligned}
\end{equation}
以上计算还需要考虑植被温度变化所带来的长波辐射量的改变,具体到$L_v^\downarrow$增加$4f_c\varepsilon_v\sigma T_v^3\Delta T_v$,${L_{g}}$增加$\varepsilon_{g} \Sigma 4 f_{c} \varepsilon_{v} \sigma T_{v}^{3} \Delta T_{v}$,$L_v ^\uparrow$增加$4 f_{c} \varepsilon_{v} \sigma T_{v}^{3} \Delta T_{v}+(1-\varepsilon_{g}) 4 f_{c} 
\varepsilon_{v}(1-\varepsilon_{v}) \sigma T_{v}^{3} \Delta T_{v}$。

将以上逻辑应用到之前描述的三层植被中,如图~\ref{fig:三层植被长波辐射传输示意图} 所示。
从计算示意图来看,稍显复杂,因为需要计算每两层之间的辐射传输量,其与短波辐射类似,具有一定规律。
%
{
\begin{figure}[htbp]
\centering
\includegraphics[width=0.8\columnwidth]{Figures/辐射过程及辐射通量计算/三层植被长波辐射传输示意图.png}
\caption[三层植被长波辐射传输示意图]{三层植被长波辐射传输示意图。4:天空,0:地面,1-3为植被层次,$L_{4 \rightarrow 3}$表示大气到达第三层植被的长波辐射,其他同}
\label{fig:三层植被长波辐射传输示意图}
\end{figure}
}
%

从天空下行的长波辐射到达每一层的辐射量为:
\begin{equation}
\begin{aligned} L_{4 \rightarrow 3} &=S_{3} L^\downarrow \\ L_{4 \rightarrow 2} &=\left(1-S_{3}\right) S_{2} L^\downarrow 
    \\ L_{4 \rightarrow 1} &=\left(1-S_{2}-S_{3}+S_{2} S_{3}\right) S_{1} L^\downarrow \\ 
    L_{4 \rightarrow 0} &=\left(1-S_{1}-S_{2}-S_{3}+S_{1} S_{2}+S_{1} S_{3}+S_{2} S_{3}-S_{1} S_{2} S_{3}\right) L^\downarrow \end{aligned}
\end{equation}
上式中,$S_n$表示第$n$层的阴影(等于太阳辐射传输中入射漫射辐射植被阴影)。从第3层植被向下到达各层的辐射量为:
\begin{equation}
\begin{aligned}
L_{3 \rightarrow 2} &=S_{2}\left(L_{3 \downarrow}+L_{3}\right) \\[1ex]
L_{3 \rightarrow 1} &=\left(1-S_{2}\right) S_{1}\left(L_{3 \downarrow}+L_{3}\right) \\[1ex]
L_{3 \rightarrow 0} &=\left(1-S_{1}-S_{2}+S_{1} S_{2}\right)\left(L_{3 \downarrow}+L_{3}\right)
\end{aligned}
\end{equation}
上式中$L_3 ^\downarrow$为从第3层植被向下穿透的长波辐射,$L_3$为第3层植被自身发射的长波辐射。
同理,从第2层植被及第1层植被向下到达各层的辐射量分别为:
\begin{equation}
\begin{aligned}
L_{2 \rightarrow 1} &=S_{1}\left(L_{2 \downarrow}+L_{2}\right) \\[1ex] 
L_{2 \rightarrow 0} &=\left(1-S_{1}\right)\left(L_{2 \downarrow}+L_{2}\right) \\[1ex]
L_{1 \rightarrow 0} &=L_{1 \downarrow}+L_{1}
\end{aligned}
\end{equation}
将以上的表达式经过简单整理,向下达到每一层的辐射量可以写成矩阵相乘的形式:
\begin{equation}
\boldsymbol{L}_{\rm dn}=\left(\begin{matrix}L_{1\downarrow}+L_1&L_{2\downarrow}+L_2&L_{3\downarrow}+L_3&L^\downarrow\\
\end{matrix}\right)\left(\begin{matrix}a_{10}&&&\\a_{20}&a_{21}&&
    \\a_{30}&a_{31}&a_{32}&\\a_{40}&a_{41}&a_{42}&a_{43}\\\end{matrix}\right)
\end{equation}
上式中右边的系数矩阵为:
\begin{equation}
\left(\begin{matrix}1&&&\\1-S_1&S_1&&\\1-S_1-S_2+S_1S_2&{\left(1-S_2\right)S}_1&S_2&\\
    1-S_1-S_2-S_3+S_1S_2+S_1S_3+S_2S_3-S_1S_2S_3&\left(1-S_2-S_3+S_2S_3\right)S_1&
    \left(1-S_3\right)S_2&S_3\\\end{matrix}\right)    
\end{equation}  
上次空白处表示0。若一层有多种PFT,其$L_n$计算为:
\begin{equation}
L_{n}=\sum f_{ci} \varepsilon_{i} \sigma T_{vi}^{4}
\end{equation}
$L_{n\downarrow}$计算为:
\begin{equation}
\begin{aligned}
L_{3 \downarrow} &=\sum \frac{S_{3i}\left(1-\varepsilon_{3i}\right)}{S_{3}} L_{4 \rightarrow 3} \\[1ex]
L_{2 \downarrow} &=\sum \frac{S_{2 i}\left(1-\varepsilon_{2 i}\right)}{S_{2}}\left(L_{4 \rightarrow 2}+L_{3 \rightarrow 2}\right) \\[1ex]
L_{1 \downarrow} &=\sum \frac{S_{1 i}\left(1-\varepsilon_{1 i}\right)}{S_{1}}\left(L_{4 \rightarrow 1}+L_{3 \rightarrow 1}+L_{2 \rightarrow 1}\right)
\end{aligned}
\end{equation}
上式中$S_{ni}$表示第$n$层植被PFT $i$的阴影面积,$\frac{\sum{S_{ni}\left(1-\varepsilon_{ni}\right)}}{S_n}$为每一层的属性,$\varepsilon_{ni}$为单独PFT属性。
先算$L_{3\downarrow}$,然后计算$L_{2\downarrow}$(需要$L_3$的计算结果),接着计算 $L_{1\downarrow}$,容易计算得到:
\begin{equation}
L_{v} ^\downarrow=L_{4 \rightarrow 0}+L_{3 \rightarrow 0}+L_{2 \rightarrow 0}+L_{1 \rightarrow 0} \text {, 即 } \boldsymbol{L}_{\rm dn}[1]
\end{equation}
\begin{equation}
L_{g} ^\uparrow =\left(1-\varepsilon_{g}\right) L_{v} ^\downarrow+\varepsilon_{g} \sigma T_{g}^{4}
\end{equation}
从土壤向上发射的辐射$L_g ^\uparrow $到达每一层植被的计算类似大气长波的向下传输:
\begin{equation}
\begin{aligned}
L_{0 \rightarrow 1} &=S_{1} L_{g} ^\uparrow\\[1ex]
L_{0 \rightarrow 2} &=\left(1-S_{1}\right) S_{2} L_{g} ^\uparrow \\[1ex]
L_{0 \rightarrow 3} &=\left(1-S_{1}-S_{2}+S_{1} S_{2}\right) S_{3} L_{g} ^\uparrow \\[1ex]
L_{0 \rightarrow 4} &=\left(1-S_{1}-S_{2}-S_{3}+S_{1} S_{2}+S_{1} S_{3}+S_{2} S_{3}-S_{1} S_{2} S_{3}\right) L_{g}^\uparrow
\end{aligned}
\end{equation}
从第1层植被向上到达各层的辐射量为:
\begin{equation}
\begin{aligned}
L_{1 \rightarrow 2} &=S_{2}\left(L_{1 \uparrow}+L_{1}\right) \\[1ex]
L_{1 \rightarrow 3} &=\left(1-S_{2}\right) S_{3}\left(L_{1 \uparrow}+L_{1}\right) \\[1ex]
L_{1 \rightarrow 4} &=\left(1-S_{2}-S_{3}+S_{2} S_{3}\right)\left(L_{1 \uparrow}+L_{1}\right)
\end{aligned}
\end{equation}
上式中$L_1\uparrow$为从第1层植被向上穿透的长波辐射,$L_1$为第1层植被自身发射的长波辐射。同理,从第2层植被及第3层植被向上到达各层的辐射量分别为:
\begin{equation}
\begin{aligned}
L_{2 \rightarrow 3} &=S_{3}\left(L_{2 \uparrow}+L_{2}\right) \\[1ex]
L_{2 \rightarrow 4} &=\left(1-S_{3}\right)\left(L_{2 \uparrow}+L_{2}\right) \\[1ex] 
L_{3 \rightarrow 4} &=L_{3 \uparrow}+L_{3} 
\end{aligned}
\end{equation}
将以上的表达式经过简单整理,向上到达每层的辐射量同样可以写成矩阵相乘的形式:
\begin{equation}
\boldsymbol {L}_{\rm up}=\left(\begin{matrix}L_g ^\uparrow &L_{1\uparrow}+L_1&L_{2\uparrow}+L_2&L_{3\uparrow}+L_3\\\end{matrix}
\right)\left(\begin{matrix}a_{01}&a_{02}&a_{03}&a_{04}\\&a_{12}&a_{13}&a_{14}\\
&&a_{23}&a_{24}\\&&&a_{34}
\\\end{matrix}\right)
\end{equation}
上式中右边的系数矩阵为:
\begin{equation}
\left(\begin{matrix}S_1&\left(1-S_1\right)S_2&\left(1-S_1-S_2+S_1S_2\right)S_3&
    1-S_1-S_2-S_3+S_1S_2+S_1S_3+S_2S_3-S_1S_2S_3\\&S_2&{\left(1-S_2\right)S}_3&1-S_2-S_3+S_2S_3\\
    &&S_3&1-S_3\\&&&1\\\end{matrix}\right)   
\end{equation}
同理,
\begin{equation}
L_v ^\uparrow=L_{0 \rightarrow 4}+L_{1 \rightarrow 4}+L_{2 \rightarrow 4}+L_{3 \rightarrow 4}, \text { 即 } \boldsymbol{L}_{\rm up}[4]
\end{equation}
到达每一层植被(或地面、天空)的辐射即为:
\begin{equation}
\boldsymbol L=\boldsymbol L_{\rm dn}+\boldsymbol L_{\rm up}
\end{equation}
单个PFT所吸收的辐射量为:
\begin{equation}
{L_{v i}}=\frac{S_{ni}}{S_{n}} \varepsilon_{ni} L_{* \rightarrow n}-2 f_{ci} \varepsilon_{i} \sigma T_{vi}^{4}
\end{equation}
$L_{\ast\rightarrow n}$即为到达每一层的辐射,即为$L[n]$。注意上式是整个归一化面积上的计算结果,
或者说是在整个土壤patch上的计算,单位面积上的吸收还需要除以各自的覆盖度$f_{ci}$。

在湍流迭代求解植被温度时,我们还需要用到$L_{vi}$对自身温度的偏导数。PFT吸收的长波辐射来自自身发射的部分可计算如下: \\
%
对于第1层植被:
\begin{equation}
f_{ci} \varepsilon_{vi} \sigma T_{vi}^{4}\left(1-\varepsilon_{g}\right) S_{i} \varepsilon_{vi}
\end{equation}
对于第2层植被:
\begin{equation}
f_{ci} \varepsilon_{vi} \sigma T_{vi}^{4}\left[\left(1-S_{1}\right)+\sum S_{1 i}\left(1-\varepsilon_{1 i}\right)\right]
\left(1-\varepsilon_{g}\right)\left[\left(1-S_{1}\right)+\sum S_{1 i}\left(1-\varepsilon_{1 i}\right)\right] S_{i} \varepsilon_{vi}
\end{equation}
对于第3层植被:
\begin{equation}
\begin{aligned} 
    f_{c i} \varepsilon_{v i} \sigma T_{v i}^{4}\left[\left(1-S_{1}-S_{2}+S_{1} S_{2}\right)+
    \sum S_{2 i}\left(1-\varepsilon_{2 i}\right)\left[\left(1-S_{1}\right)+\sum S_{1 i}\left(1-\varepsilon_{1 i}\right)\right]\right.\\[1ex] 
    \left.+\left(1-S_{2}\right) \sum S_{1 i}\left(1-\varepsilon_{1 i}\right)\right]\left(1-\varepsilon_{g}\right)\left[\left(1-S_{1}-S_{2}+S_{1} S_{2}\right)\right.\\[1ex] 
    +\sum S_{1 i}\left(1-\varepsilon_{1 i}\right)\left[\left(1-S_{2}\right)+\sum S_{2 i}\left(1-\varepsilon_{2 i}\right)\right] \\[1ex] 
    \left.+\left(1-S_{1}\right) 
    \sum S_{2 i}\left(1-\varepsilon_{2 i}\right)\right] S_{i} \varepsilon_{v i}
\end{aligned}
\end{equation}
%\begin{equation}
%\begin{aligned} 
%    f_{ci} \varepsilon_{vi} \sigma T_{vi}^{4}
%    \left[\left(1-S_{1}-S_{2}+S_{1}S_{2}\right)+\sum S_{2i}\left(1-\varepsilon_{2i}\right)
%    \left[\left(1-S_{1}\right)+\sum S_{1i}\left(1-\varepsilon_{1i}\right)\right]\\
%    +\left(1-S_{2}\right) \sum S_{1i}\left(1-\varepsilon_{1i}\right)\right]
%     \left(1-\varepsilon_{g}\right)
%     \left[\left(1-S_{1}-S_{2}+S_{1} S_{2}\right)\\
%      &+\sum S_{1i}\left(1-\varepsilon_{1i}\right)\left[\left(1-S_{2}\right)+\sum S_{2i}\left(1-\varepsilon_{2i}\right)\right] \\ 
%      +\left(1-S_{1}\right) \sum S_{2i}\left(1-\varepsilon_{2i}\right)\right] S_{i} \varepsilon_{vi}
 %\end{aligned}
%\end{equation}
同时需要考虑自身发射的长波辐射$-2f_{ci}\varepsilon_{vi}\sigma T_{vi}^4$。
$L_{vi}$对于叶温的变化率为以上两者之和对$T_{vi}$求导,同时需要除以$f_{ci}$ (转换成在单位面积上的值)。 
植被温度最后一次迭代的改变所包含的能量分配到$L_v ^\downarrow$增加
$\Sigma4f_{ci}\varepsilon_{vi}\sigma T_{vi}^3\Delta T_{vi}$,$L_{g}$增加$\varepsilon_g\Sigma4f_{ci}\varepsilon_{vi}\sigma T_{vi}^3\Delta T_{vi}$,$L_v ^\uparrow$增加$\Sigma 4 f_{ci} \varepsilon_{vi} \sigma T_{vi}^{3} \Delta T_{vi}+\left(1-\varepsilon_{g}\right) \Sigma 4 f_{ci} \varepsilon_{vi}\left(1-\varepsilon_{vi}\right) \sigma T_{vi}^{3} \Delta T_{vi}$。


\section{短波吸收辐射通量}\label{短波吸收辐射通量}
\begin{mymdframed}{代码}
本章节对应代码为\texttt{MOD\_NetSolar.F90}。
\end{mymdframed}

在计算得到地表反照率后,地表总的太阳短波辐射通量(包括地面和植被)吸收为:
\begin{equation}
S_{v g}=S_{vis,dir}\left(1-\alpha_{vis,dir}\right)+S_{vis,dif}\left(1-\alpha_{vis,dif}\right)+
S_{nir,dir}\left(1-\alpha_{nir,dir}\right)+S_{nir,dif}\left(1-\alpha_{nir,dif}\right)
\end{equation}
其中等式右边$S$表示太阳短波在不同波段(可见光$vis$、近红外$nir$)和直射/漫射时($dir$/$dif$)的辐射通量。阳叶、阴叶辐射通量吸收分别为:
\begin{equation}
S_{s u n}=S_{vi s, dir} s_{s u n, vi s, dir}+S_{vis, dif} s_{s u n, vis, dif}+S_{ni r, dir} s_{s u n, ni r, dir}+S_{nir, dif} s_{s u n, nir, dif}
\end{equation}
\begin{equation}
S_{s h a}=S_{vis, dir} s_{sun,vis,dir}+S_{vis, dif} s_{sun,vis,dif}+S_{ni r, dir} s_{sun,nir,dir}+S_{nir, dif} s_{sun,nir,dif}
\end{equation}
植被总的吸收辐射通量为:
\begin{equation}
S_{v}=S_{s u n}+S_{s h a}
\end{equation}
地面的吸收辐射通量为:
\begin{equation}\label{eq:sg}
S_{g}=S_{v g}-S_{s u n}-S_{s h a}
\end{equation}
单位面积阳叶、阴叶有效光合辐射吸收通量为:
\begin{equation}
\text{PAR}_{sun} =S_{vis, dir} s_{sun, vis, dir}+S_{vis, dif} s_{sun, vis, dif}
\end{equation}
\begin{equation}
\text{PAR}_{sha} =S_{vis, dir} s_{sha, vis, dir}+S_{vis, dif} s_{sha, vis, dif}
\end{equation}


\section{长波净辐射通量}\label{长波净辐射通量}
\begin{mymdframed}{代码}
本章节对应代码包含于\texttt{MOD\_GroundTemperature.F90, MOD\_LeafTemperature.F90}。
\end{mymdframed}

地表发射的长波辐射是基于斯蒂芬-玻尔兹曼定律来计算。
当为土壤时,地面发射率$\varepsilon_g$设置为0.96;对于冰川,设置为0.97。

对于裸土覆盖情况,地面的长波净辐射通量计算为:
\begin{equation}
L_{g}=\varepsilon_{g} L ^\downarrow
\end{equation}
其中$L ^\downarrow$表示大气向下长波辐射通量。

对于一维植被覆盖情况,植被长波透射率计算为:
\begin{equation}
\tau_{v}=\exp \left(-\frac{\text{LAI+SAI}}{\bar{\mu}}\right)
\end{equation}
植被发射率(即吸收率)计算为:
\begin{equation}
\varepsilon_{v}=1-\exp \left(-\frac{\text{LAI+SAI}}{\bar{\mu}}\right)=1-\tau _v
\end{equation}
其中$\rm SAI$表示植被茎面积指数。植被的长波净辐射通量计算为:
\begin{equation}
L_{v}=\varepsilon_{v}\left(L ^\downarrow-2 \sigma T_{v}^{4}+\varepsilon_{g} \sigma T_{g}^{4}\right)
\end{equation}
其中$T_v$表示叶片温度,$T_g$表示土壤表层温度,$\sigma$表示斯蒂芬-玻尔兹曼常数。地面净长波辐射吸收为:
\begin{equation}\label{eq:lg1}
L_g= \tau_{v} L ^\downarrow \varepsilon_g - \varepsilon_{g} \sigma T_{g}^{4}
\end{equation}

注意,以上的计算并未考虑植被向下长波辐射到达地面后的反射部分,即假设全部被地面吸收。在新版本中,考虑一次土壤反射,然后被植被吸收,其计算表达式为:
\begin{equation}
L_{v}=\varepsilon_{v}\left(L ^\downarrow-2 \sigma T_{v}^{4}+\varepsilon_{g} \sigma T_{g}^{4}\right)+\left(1-\varepsilon_{g}\right)\left(1-\varepsilon_{v}\right) \varepsilon_{v} L ^\downarrow+\left(1-\varepsilon_{g}\right) \varepsilon_{v}^{2} \sigma T_{v}^{4}
\end{equation}
地面净长波辐射计算为:
\begin{equation}\label{eq:lg2}
L_{g}=\left(\tau_{v} L ^\downarrow  + \varepsilon_{v} \sigma T_{v}^{4} \right) \varepsilon_{g} - \varepsilon_{g} \sigma T_{g}^{4}
\end{equation}
以上计算是针对一维植被情景,对于三维植被净长波辐射计算见章节~\ref{三维植被长波辐射传输}。

\section{太阳天顶角}\label{太阳天顶角}
\begin{mymdframed}{代码}
本章节对应代码为\texttt{MOD\_OrbCoszen.F90}。
\end{mymdframed}

太阳天顶角的余弦$\mu$的表达式为
\begin{equation}
\mu= \sin(\phi)\sin(\delta)−\cos(\phi)\cos(\delta)\cos(h)
\end{equation}
%
其中$h$为太阳时角,$\delta$是黄赤交角,$\phi$是纬度(北半球为正;单位均为弧度),太阳时角$h$的表达式为:
\begin{equation}
h=2\pi d+\theta
\end{equation}
$d$是儒略日 (Julian Day),$\theta$是经度(格林威治子午线以东为正)。计算$\delta$需要的轨道参数有地球的倾斜角$\varepsilon$、地球的离心率$e$、近日点相对于春分点的经度$\widetilde{\omega}$。黄赤交角$\delta$计算式为:
\begin{equation}
\delta= \sin^{−1}[\sin(\epsilon) \sin(\lambda)]
\end{equation}
%
$\varepsilon$为地球倾斜角,设置为\num{0.409214646},单位弧度。$\lambda$为地球实际的经度(单位弧度),其计算从春分开始逆时针计算($\lambda$在春分点为0),其表达式为:
\begin{equation}
\lambda = \lambda_{m0} + \left( 2e - \frac{1}{4}e^{3} \right)\sin\left( \lambda_{m} - \widetilde{\omega} \right) + \frac{5}{4}e^{2}\sin(2\left( \lambda_{m} - \widetilde{\omega} \right) ) + \frac{13}{12}e^{3}\sin(3\left( \lambda_{m} - \widetilde{\omega} \right))
\end{equation}
%
$\lambda_{m0}$是春分时的平均经度,设置为\num{-3.2625366e-2}。$e$设置为\num{1.672393084e-2}。$\widetilde{\omega}$设置为\num{4.92251015}。平均经度$\lambda_{m}$计算为:
%
\begin{equation}
\lambda_{m} = \lambda_{m0} + \frac{2\pi\left( d - d_{ve} \right)}{365}
\end{equation}
%
其中\(d_{ve} = 80.5\)是春分时儒略日(3月21日中午)。
%\end{辐射过程及辐射通量计算}

\chapter{地表湍流通量方案}\label{ch:地表湍流通量}
%\addcontentsline{toc}{chapter}{地表湍流交换过程}

%\begin{地表湍流交换过程}
\section{基本理论}\label{基本理论}
\begin{mymdframed}{代码}
  本节对应的代码文件为\texttt{MOD\_FrictionVelocity.F90}。
\end{mymdframed}

在近地层(距离地面几米到几十米高度)中,地面与大气在垂直方向存在显著的湍流输送,且湍流输送通量与风向几乎不随高度变化(固亦称常通量近地层)。
引入尺度因子,则近地层的动量通量$\left|\tau\right|$ (\unit{kg.m^{-1}.s^{-2}})、热量通量$H$ (\unit{W.m^{-2}})和水汽通量$E$ (\unit{kg.m^{-2}.s^{-1}})可分别表示为:
\begin{equation}
  |\tau|=\rho_{\mathrm{a}} \sqrt{ \left(\overline{u^{\prime} w^{\prime}}\right)^{2} + \left(\overline{v^{\prime} w^{\prime}}\right)^{2} }=\rho_{\mathrm{a}} u_{*}^{2}
\end{equation}
(分量形式:$\tau_{\mathrm{x}}=\rho_{\mathrm{a}} \overline{u^{\prime} w^{\prime}}, \quad \tau_{\mathrm{y}}=\rho_{\mathrm{a}} \overline{v^{\prime} w^{\prime}}$)
\begin{equation}
  H=C_{\mathrm{a}} \rho_{\mathrm{a}} \overline{\theta^{\prime} w^{\prime}}=-C_{\mathrm{a}} \rho_{\mathrm{a}} \theta_{\mathrm{*}} u_{*}
\end{equation}
\begin{equation}
  E=\rho_{\mathrm{a}} \overline{q^{\prime} w^{\prime}}=-\rho_{\mathrm{a}} q_{*} u_{*}
\end{equation}
其中$u^\prime$、$v^\prime$、$w^\prime$、$\theta^\prime$、$q^\prime$分别为纬向风速度、经向风速、垂直风速、位温和比湿的湍流脉动,
$C_{\mathrm{a}}$表示干空气的比热容(\unit{J.kg^{-1}.K^{-1}}),$\rho_{\mathrm{a}}$表示大气密度(\unit{kg.m^{-3}});$u_\ast$为湍流速度尺度(即摩擦速度)(\unit{m.s^{-1}}),$\theta_\ast$为湍流位温尺度(K),
$q_\ast$为湍流湿度尺度(\unit{kg.kg^{-1}})。$\rho_{\mathrm{a}}$可由公式$\rho_{\mathrm{a}}=\frac{P_{\mathrm{a}}-0.378e_{\mathrm{a}}}{R_{\mathrm{d}}T_{\mathrm{a}}}$计算得到,
其中$P_{\mathrm{a}}$表示大气压(Pa),$\ T_{\mathrm{a}}$表示大气温度(K),$R_{\mathrm{d}}$表示干空气气体常数(\unit{J.kg^{-1}.K^{-1}}),
$e_{\mathrm{a}}=\frac{q_{\mathrm{a}}P_{\mathrm{a}}}{0.622+0.378q_{\mathrm{a}}}$表示大气水汽分压(Pa),$q_{\mathrm{a}}$表示大气比湿(\unit{kg.kg^{-1}})。
由常通量性可知$u_\ast$、$\theta_\ast$、$q_\ast$均不随高度变化。


近地层湍流通量(动量、感热和水汽等)的计算(即$u_\ast$,$\theta_\ast$,$q_\ast$的表达)主要以Monin--Obukhov相似理论为基础。
根据Monin--Obukhov相似理论,无量纲平均风速($\left|u\right|/u_\ast$)、位温($\theta/\theta_\ast$)、比湿($q/q_\ast$)的
垂直梯度可分别表达为仅依赖于参数$\zeta=\frac{z-d}{L}$的函数:
\begin{equation}\label{kz_u}
  \frac{\kappa (z-d)}{u_{*}} \frac{\partial|u|}{\partial z}=\phi_{\mathrm{m}}(\zeta)
\end{equation}
\begin{equation}\label{kz_theta}
  \frac{\kappa (z-d)}{\theta_{*}} \frac{\partial \theta}{\partial z}=\phi_{\mathrm{h}}(\zeta)
\end{equation}
\begin{equation}\label{kz_q}
  \frac{\kappa (z-d)}{q_{*}} \frac{\partial q}{\partial z}=\phi_{\mathrm{w}}(\zeta)
\end{equation}
其中 $\kappa$ 为 von K\'arman 常数,$z$为离地面高度(m),$d$为零平面位移(m),$L$为Monin-Obukhov长度(m):
\begin{equation}\label{ObukL}
  L=-\frac{u_{*}^{3}}{\kappa \left(\frac{g}{\overline{\theta_{\mathrm{a,v}}}}\right) \overline{\theta_{\mathrm{v}}^{\prime} w^{\prime}}}=\frac{u_{*}^{2} \overline{\theta_{\mathrm{a,v}}}}{\kappa g \theta_{\mathrm{v *}}}
\end{equation}
其中$\overline{\theta_{\mathrm{a,v}}}=\overline{\theta_{\mathrm{a}}}(1+0.61\overline{q_{\mathrm{a}}})$表示参考大气虚位温(K),
大气位温$\theta_{\mathrm{a}}$由其定义的一阶展开近似给出$\theta_{\mathrm{a}}=T_{\mathrm{a}}+0.0098z_{\mathrm{a}}$,$\theta_{\mathrm{v}}^{\prime}$为虚位温湍流脉动。$L$可表征近地层的大气层结稳定度:
在稳定条件下,湍流热通量$\overline{\theta_{\mathrm v}^\prime w^\prime}<0$,$L>0$;在不稳定条件下,
$\overline{\theta_{\mathrm v}^\prime w^\prime}>0$,$L<0$;在中性条件下,$\overline{\theta_{\mathrm v}^\prime w^\prime}=0$,$L\rightarrow\infty$。
$\theta_{\mathrm{v\ast}}$为虚温尺度(K),计算为:
\begin{equation}\label{thvstar}
  \theta_{\mathrm{v\ast}}=\theta_\ast+0.61(\overline{\theta_{\mathrm{a}}}q_\ast + \theta_\ast\overline{q_{\mathrm{a}}})
\end{equation}

相似性函数$\phi_{\mathrm x}$ ($\mathrm{x=m}$、$\rm h$和$\rm w$分别对应动量、感热和水汽)适用于任何下垫面条件,故亦称通用相似性函数,或简称通用函数,
它将近地层的湍流通量与$\left|u\right|$、$\theta$和$q$等的垂直梯度联系起来,构成了近地层湍流通量参数化的基础。在中性条件下,$\phi_{\mathrm m}=\phi_{\mathrm h}=\phi_{\mathrm w}=1$。



为计算得到$u_\ast$、$\theta_\ast$、$q_\ast$,通常将上述关于$\phi_{\mathrm m}\left(\zeta\right)$、$\phi_{\mathrm h}\left(\zeta\right)$、
$\phi_{\mathrm w}(\zeta)$的微分方程(公式~\eqref{kz_u}--\eqref{kz_q})在近地层的两个任意高度$z_1$和$z_2$ ($z_2>z_1$)进行积分,积分结果如下:
\begin{equation}
  |u|_{2}-|u|_{1}=\frac{u_{*}}{\kappa}\left[\ln \left(\frac{z_{2}-d}{z_{1}-d}\right)-\psi_{\mathrm{m}}\left(\frac{z_{2}-d}{L}\right)+\psi_{\mathrm{m}}\left(\frac{z_{1}-d}{L}\right)\right]
\end{equation}
%
\begin{equation}
  \theta_{2}-\theta_{1}=\frac{\theta_{*}}{\kappa}\left[\ln \left(\frac{z_{2}-d}{z_{1}-d}\right)-\psi_{\mathrm{h}}\left(\frac{z_{2}-d}{L}\right)+\psi_{\mathrm{h}}\left(\frac{z_{1}-d}{L}\right)\right]
\end{equation}
%
\begin{equation}
  q_{2}-q_{1}=\frac{q_{*}}{\kappa}\left[\ln \left(\frac{z_{2}-d}{z_{1}-d}\right)-\psi_{\mathrm{w}}\left(\frac{z_{2}-d}{L}\right)+\psi_{\mathrm{w}}\left(\frac{z_{1}-d}{L}\right)\right]
\end{equation}
其中函数$\psi_x\left(y\right)\ (x=\rm m,\ h,\ w)$定义为
\begin{equation}
  \psi_{x}(y)=\int_{\mathrm{0}}^{y} \frac{1-\phi_{x}(s)}{s} {\rm d} s
\end{equation}
$z_{0x}\ (x={\rm m,\ h,\ w})$分别表示动量、感热和水汽的粗糙长度(m)。取$z_1$为地表高度,$z_2$为大气强迫参考(观测)高度,则有如下积分边界条件:
\begin{equation}\label{VaIni}
  \begin{array}{ll}z_{1}=z_{\mathrm{0 m}}+d:|u|_{1}=0, \quad & z_{2}=z_{\mathrm{a, m}}: |u|_{2}=V_{\mathrm{a}}=\sqrt{u_{\mathrm{a}}^{2}+v_{\mathrm{a}}^{2}+U_{\mathrm{c}}^{2}} \geqslant 0.1 \\
    z_{1}=z_{\mathrm{0 h}}+d: \theta_{1}=\theta_{\mathrm{s}}, & z_{2}=z_{\mathrm{a, h}}: \theta_{2}=\theta_{\mathrm{a}} \\
  z_{1}=z_{\mathrm{0 w}}+d: q_{1}=q_{\mathrm{s}}, & z_{2}=z_{\mathrm{a, w}}: q_{2}=q_{\mathrm{a}}\end{array}
\end{equation}
则方程\eqref{kz_u}--\eqref{kz_q}由$z_1$到$z_2$的积分结果为:
\begin{equation}\label{Va}
  V_{\mathrm{a}}=\frac{u_{*}}{\kappa}\left[\ln \left(\frac{z_{\mathrm{a, m}}-d}{z_{\mathrm{0 m}}}\right)-\psi_{\mathrm{m}}\left(\frac{z_{\mathrm{a, m}}-d}{L}\right)+\psi_{\mathrm{m}}\left(\frac{z_{\mathrm{0 m}}}{L}\right)\right]
\end{equation}
%
\begin{equation}\label{theta_atm-theta_s}
  \theta_{\mathrm{a}}-\theta_{\mathrm{s}}=\frac{\theta_{\mathrm{*}}}{\kappa}\left[\ln \left(\frac{z_{\mathrm{a, h}}-d}{z_{\mathrm{0 h}}}\right)-\psi_{\mathrm{h}}\left(\frac{z_{\mathrm{a, h}}-d}{L}\right)+\psi_{\mathrm{h}}\left(\frac{z_{\mathrm{0 h}}}{L}\right)\right]
\end{equation}
%
\begin{equation}\label{q_atm-qs}
  q_{\mathrm{a}}-q_{\mathrm{s}}=\frac{q_{*}}{\kappa}\left[\ln \left(\frac{z_{\mathrm{a, w}}-d}{z_{\mathrm{0 w}}}\right)-\psi_{\mathrm{w}}\left(\frac{z_{\mathrm{a, w}}-d}{L}\right)+\psi_{\mathrm{w}}\left(\frac{z_{\mathrm{0 w}}}{L}\right)\right]
\end{equation}
这里限制$V_{\mathrm {a}}\geqslant0.1$是为了避免太小的风速使得感热与潜热过小导致数值溢出。对流速度$U_{\mathrm {c}}$表示对流边界层中的大涡对近地层湍流通量的贡献,计算方案为:
\begin{equation}
  U_{\mathrm{c}}= \begin{cases}
    0, & \text { 当 }\ \zeta_{\mathrm{a}}=\frac{z_{\mathrm{a, m}}-d}{L} \geqslant 0 \text { 时(即稳定条件下) } \\
    \beta w_{*}, & \text { 当 }\ \zeta_{\mathrm{a}}<0 \text { 时 (即不稳定条件下) }
  \end{cases}
\end{equation}
其中$w_\ast={(\frac{-gu_\ast\theta_{\mathrm{v\ast}}z_{\mathrm {i}}}{\overline{\theta_{\mathrm{a,v}}}})}^{1/3}$为垂直速度尺度 (\unit{m.s^{-1}}),$z_{\mathrm {i}}=1000$ m 代表对流边界层高度(这里设为常数),
参数$\beta=1$。

因$L$是依赖于$u_\ast$、$\theta_\ast$、$q_\ast$的函数,
故方程\eqref{Va}--\eqref{q_atm-qs}可视为如下形式的方程组:
\begin{equation}\label{FGH}
  \left\{\begin{array}{l}F\big(u_{*}, L\left(u_{*}, \theta_{*}, q_{*}\right)\big)=0 \\
      G\big(\theta_{*}, L\left(u_{*}, \theta_{*}, q_{*}\right)\big)=0 \\
  H\big(q_{*}, L\left(u_{*}, \theta_{*}, q_{*}\right)\big)=0\end{array}\right.
\end{equation}
若给$U_{\mathrm {c}}$与$L$一个初始猜测,结合大气强迫场提供的$u_{\mathrm{a}}$、$v_{\mathrm{a}}$、$\theta_{\mathrm{a}}$、$q_{\mathrm{a}}$及对应的参考高度$z_{\mathrm{a},x}\, (x=\mathrm{m,h,w})$,
地表参数$d$和$z_{0x}$的估计,以及地表$z_{0x}+d$高度$\theta_{\mathrm {s}}$、$q_{\mathrm {s}}$,则$u_\ast$、$\theta_\ast$、$q_\ast$可通过方程组
\eqref{FGH}
进行迭代求解,进而求出动量、感热和水汽通量。

根据~\citet{zeng1998intercomparison},通用函数 $\phi_x$ 有如下表达式:
\begin{equation}\label{phim_zeng}
  \phi_{\mathrm{m}}(\zeta)=\begin{cases}
    0.7 \kappa^{\frac{2}{3}}(-\zeta)^{\frac{1}{3}}, & \zeta<-1.574 \text { 时 (即非常不稳定条件下) } \\
    (1-16 \zeta)^{-\frac{1}{4}}, & -1.574 \leqslant \zeta<0 \text { 时 (即不稳定条件下) } \\
    1+5 \zeta, & 0 \leqslant \zeta \leqslant 1 \text { 时 (即稳定条件下) } \\
    5+\zeta, & \zeta>1 \text { 时 (即非常稳定条件下) }
  \end{cases}
\end{equation}
\begin{equation}
  \phi_{\mathrm{h}}(\zeta)=\phi_{\mathrm{w}}(\zeta)=\begin{cases}
    0.9 \kappa^{\frac{4}{3}}(-\zeta)^{-\frac{1}{3}}, & \zeta<-0.465 \text { 时 (即非常不稳定条件下) } \\
    (1-16 \zeta)^{-\frac{1}{2}}, & -0.465 \leqslant \zeta<0 \text { 时 (即不稳定条件下) } \\
    1+5 \zeta, & 0 \leqslant \zeta \leqslant 1 \text { 时 (即稳定条件下) } \\
    5+\zeta, & \zeta>1 \text { 时 (即非常稳定条件下) }
  \end{cases}
\end{equation}

将$\phi_{\mathrm m}$代入风速廓线方程~\eqref{Va},即可得到风速廓线在不同条件下的具体形式:

\noindent 非常不稳定条件下($\zeta_{\mathrm{a}}=\frac{z_{\mathrm{a,m}}-d}{L}<-1.574$)
\begin{equation}\label{Va_VU}
  V_{\mathrm{a}}=\frac{u_{*}}{\kappa}\left\{\ln \frac{-1.574 L}{z_{\mathrm{0 m}}}-\psi_{\mathrm{mu}}(-1.574)+
  1.14\left[\left(-\zeta_{\mathrm{a}}\right)^{\frac{1}{3}}-(1.574)^{\frac{1}{3}}\right]+\psi_{\mathrm{mu}}\left(\frac{z_{\mathrm{0 m}}}{L}\right)\right\}
\end{equation}
不稳定条件下($-1.574\leqslant\zeta_{\mathrm{a}}<0$)
\begin{equation}\label{Va_U}
  V_{\mathrm{a}}=\frac{u_{*}}{\kappa}\left\{\ln \frac{z_{\mathrm{a, m}}-d}{z_{\mathrm{0 m}}}-\psi_{\mathrm{mu}}\left(\zeta_{\mathrm{a}}\right)+\psi_{\mathrm{mu}}\left(\frac{z_{\mathrm{0 m}}}{L}\right)\right\}
\end{equation}
稳定条件下($0\leqslant\zeta_{\mathrm{a}}\leqslant1$)
\begin{equation}\label{Va_S}
  V_{\mathrm{a}}=\frac{u_{*}}{\kappa}\left\{\ln \frac{z_{\mathrm{a, m}}-d}{z_{\mathrm{0 m}}}+5 \zeta_{\mathrm{a}}-5 \frac{z_{\mathrm{0 m}}}{L}\right\}
\end{equation}
非常稳定条件下($\zeta_{\mathrm{a}}>1$)
\begin{equation}\label{Va_VS}
  V_{\mathrm{a}}=\frac{u_{*}}{\kappa}\left\{\left[\ln \frac{L}{z_{\mathrm{0 m}}}+5\right]+\left[5 \ln \zeta_{\mathrm{a}}+\zeta_{\mathrm{a}}-1\right]-5 \frac{z_{\mathrm{0 m}}}{L}\right\}
\end{equation}

\noindent 其中
\begin{equation}\label{Psim}
  \begin{array}{c}\psi_{\mathrm{mu}}\left(\zeta\right)=2\ln{(\frac{1+x}{2})}+\ln{\left(\frac{1+x^2}{2}\right)-2}\tan^{-1}{x}+\frac{\pi}{2} \\
  x={(1-16\zeta)}^{1/4}\end{array}
\end{equation}

将$\phi_{\mathrm h}$和$\phi_{\mathrm w}$代入温度廓线方程(\ref{theta_atm-theta_s})和水汽廓线方程(\ref{q_atm-qs}),即可得:

\noindent 非常不稳定条件下($\zeta_{\mathrm{a}}=\frac{z_{\mathrm{a,h}}-d}{L}$ 或$ \ \frac{z_{\mathrm{a,w}}-d}{L}\ <-0.465$)
\begin{equation}\label{theta_VU}
  \begin{aligned}
    &\theta_{\mathrm{a}}-\theta_{\mathrm{s}}= \\
    &\frac{\theta_{*}}{\kappa}\left\{\ln \frac{-0.465 L}{z_{\mathrm{0 h}}}-\psi_{\mathrm{hu}}(-0.465)+0.8\left[(0.465)^{-\frac{1}{3}}-\left(-\zeta_{\mathrm{a}}\right)^{-\frac{1}{3}}\right]
    +\psi_{\mathrm{hu}}\left(\frac{z_{\mathrm{0 h}}}{L}\right)\right\}
  \end{aligned}
\end{equation}
\begin{equation}\label{q_VU}
  \begin{aligned}
    &q_{\mathrm{a}}-q_{\mathrm{s}}= \\
    &\frac{q_{*}}{\kappa}\left\{\ln \frac{-0.465 L}{z_{\mathrm{0 w}}}-\psi_{\mathrm{wu}}(-0.465)+0.8\left[(0.465)^{-\frac{1}{3}}-
    \left(-\zeta_{\mathrm{a}}\right)^{-\frac{1}{3}}\right]+\psi_{\mathrm{wu}}\left(\frac{z_{\mathrm{0 w}}}{L}\right)\right\}
  \end{aligned}
\end{equation}
不稳定条件下($-0.465\leqslant\zeta_{\mathrm{a}}<0$)
\begin{equation}
  \theta_{\mathrm{a}}-\theta_{\mathrm{s}}=\frac{\theta_{*}}{\kappa}\left\{\ln \frac{z_{\mathrm{a, h}}-d}{z_{\mathrm{0 h}}}-\psi_{\mathrm{hu}}
  \left(\zeta_{\mathrm{a}}\right)+\psi_{\mathrm{hu}}\left(\frac{z_{\mathrm{0 h}}}{L}\right)\right\}
\end{equation}
\begin{equation}
  q_{\mathrm{a}}-q_{\mathrm{s}}=\frac{q_{*}}{\kappa}\left\{\ln \frac{z_{\mathrm{a, w}}-d}{z_{\mathrm{0 w}}}-
  \psi_{\mathrm{wu}}\left(\zeta_{\mathrm{a}}\right)+\psi_{\mathrm{wu}}\left(\frac{z_{\mathrm{0 w}}}{L}\right)\right\}
\end{equation}
稳定条件下($0\leqslant\zeta_{\mathrm{a}}\leqslant1$)
\begin{equation}
  \theta_{\mathrm{a}}-\theta_{\mathrm{s}}=\frac{\theta_{*}}{\kappa}\left\{\left[\ln \frac{z_{\mathrm{a, h}}-d}{z_{\mathrm{0 h}}}+5 \zeta_{\mathrm{a}}\right]-5 \frac{z_{\mathrm{0 h}}}{L}\right\}
\end{equation}
\begin{equation}
  q_{\mathrm{a}}-q_{\mathrm{s}}=\frac{q_{*}}{\kappa}\left\{\left[\ln \frac{z_{\mathrm{a, w}}-d}{z_{\mathrm{0 w}}}+5 \zeta_{\mathrm{a}}\right]-5 \frac{z_{\mathrm{0 w}}}{L}\right\}
\end{equation}
非常稳定条件下($\zeta_{\mathrm{a}}>1$)
\begin{equation}\label{theta_VS}
  \theta_{\mathrm{a}}-\theta_{\mathrm{s}}=\frac{\theta_{*}}{\kappa}\left\{\left[\ln \frac{L}{z_{\mathrm{0 h}}}+5\right]
  +\left[5 \ln \zeta_{\mathrm{a}}+\zeta_{\mathrm{a}}-1\right]-5 \frac{z_{\mathrm{0 h}}}{L}\right\}
\end{equation}
\begin{equation}\label{q_VS}
  q_{\mathrm{a}}-q_{\mathrm{s}}=\frac{q_{*}}{\kappa}\left\{\left[\ln \frac{L}{z_{\mathrm{0 w}}}+5\right]
  +\left[5 \ln \zeta_{\mathrm{a}}+\zeta_{\mathrm{a}}-1\right]-5 \frac{z_{\mathrm{0 w}}}{L}\right\}
\end{equation}
其中$\psi_{\mathrm{hu}}\left(\zeta\right)=\psi_{\mathrm{wu}}\left(\zeta\right)=2\ln{\left(\frac{1+x}{2}\right)}$,$x={(1-16\zeta)}^{1/2}$。

事实上,$L$的初始猜测可由总体理查德森数$R_{\mathrm{ib}}$与$\zeta$的关系得到\citep{arya2001introduction}。$R_{\mathrm{ib}}$表达为
\begin{equation}\label{Rib}
  R_{\mathrm{i b}}=\frac{\theta_{\mathrm{a,v}}-\theta_{\mathrm{s,v}}}{\overline{\theta_{\mathrm{a,v}}}} \frac{g\left(z_{\mathrm{a, m}}-d\right)}{V_{\mathrm{a}}^{2}}
\end{equation}
$R_{\mathrm{ib}}$与$\zeta$的关系为
\begin{equation}
  R_{\mathrm{ib}}=\zeta\left[\ln{\left(\frac{z_{\mathrm{a,h}}-d}{z_{\mathrm{0h}}}\right)-\psi_{\mathrm h}(\zeta)}\right] \left[\ln{\left(\frac{z_{\mathrm{a,m}}-d}{z_{\mathrm{0m}}}\right)-\psi_{\mathrm m}(\zeta)}\right]^{-2}
\end{equation}
其中$\psi_{\mathrm m}(\zeta)$与$\psi_{\mathrm h}(\zeta)$由不稳定条件下$\phi_{\mathrm h}\left(\zeta\right)=\phi_{\mathrm m}^2\left(\zeta\right)=\left(1-16\zeta\right)^{-\frac{1}{2}}$
与稳定条件下$\phi_{\mathrm h}\left(\zeta\right)=\phi_{\mathrm m}\left(\zeta\right)=1+5\zeta$确定,从而可以通过以下反算关系式得到$L$($L=\frac{z_{\mathrm{a,m}}-d}{\zeta}$)的初始猜测值
\begin{equation}\label{ZetaRib}
  \zeta=\begin{cases}
    \frac{R_{\mathrm{i b}} \ln \left(\frac{z_{\mathrm{a, m}}-d}{z_{\mathrm{0 m}}}\right)}{1-5 \min \left(R_{\mathrm{i b}}, 0.19\right)}, \qquad 10^{-6} \leqslant \zeta \leqslant 2, & \text{ 当 }\ R_{\mathrm{i b}} \geqslant 0\ \text{(即中性或稳定条件下)} \\
    R_{\mathrm{ib}} \ln \left(\frac{z_{\mathrm{a, m}}-d}{z_{\mathrm{0 m}}}\right),  \quad -100 \leqslant \zeta \leqslant-10^{-6}, & \text{ 当 }\ R_{\mathrm{i b}}<0\ \text{(即不稳定条件下)}
  \end{cases}
\end{equation}

以上给出了$u_\ast$、$\theta_\ast$、$q_\ast$的求解过程,于是地表与大气参考高度之间的动量、感热和水汽通量即可通过其定义求得。
事实上,根据$u_\ast$、$\theta_\ast$、$q_\ast$的表达形式,动量通量$\tau$、感热通量$H$和水汽通量$E$可写为如下阻抗形式:
\begin{equation}
  \tau_{\mathrm{x}}=-\frac{u_{\mathrm{a}}}{V_{\mathrm{a}}} \rho_{\mathrm{a}} u_{*}^{2}=-\rho_{\mathrm{a}} \frac{u_{\mathrm{a}}}{r_{\mathrm{a m}}}
\end{equation}
\begin{equation}
  \tau_{\mathrm{y}}=-\frac{v_{\mathrm{a}}}{V_{\mathrm{a}}} \rho_{\mathrm{a}} u_{*}^{2}=-\rho_{\mathrm{a}} \frac{v_{\mathrm{a}}}{r_{\mathrm{a m}}}
\end{equation}
\begin{equation}\label{SH}
  H=-C_{\mathrm{a}} \rho_{\mathrm{a}} \theta_{*} u_{\mathrm{*}}=-\rho_{\mathrm{a}} C_{\mathrm{a}} \frac{\left(\theta_{\mathrm{a}}-\theta_{\mathrm{s}}\right)}{r_{\mathrm{a h}}}
\end{equation}
\begin{equation}\label{LH}
  E=-\rho_{\mathrm{a}} q_{*} u_{*}=-\rho_{\mathrm{a}} \frac{\left(q_{\mathrm{a}}-q_{\mathrm{s}}\right)}{r_{\mathrm{a w}}}
\end{equation}
其中,空气动力学阻抗系数$r_{\mathrm{am}}$、$r_{\mathrm{ah}}$、$r_{\mathrm{aw}}$ (\unit{s.m^{-1}}) (结合方程~\eqref{Va}--\eqref{q_atm-qs})为
\begin{equation}\label{ram}
  r_{\mathrm{a m}}=\frac{1}{\kappa^{2} V_{\mathrm{a}}}\left[\ln \left(\frac{z_{\mathrm{a, m}}-d}{z_{\mathrm{0 m}}}\right)-\psi_{\mathrm{m}}\left(\frac{z_{\mathrm{a, m}}-d}{L}\right)+\psi_{\mathrm{m}}\left(\frac{z_{\mathrm{0 m}}}{L}\right)\right]^{2}
\end{equation}
\begin{equation}\label{rah}
  \begin{array}{c}r_{\mathrm{a h}}=\frac{1}{\kappa^{2} V_{\mathrm{a}}}\left[\ln \left(\frac{z_{\mathrm{a, m}}-d}{z_{\mathrm{0 m}}}\right)-\psi_{\mathrm{m}}\left(\frac{z_{\mathrm{a, m}}-d}{L}\right)+\psi_{\mathrm{m}}\left(\frac{z_{\mathrm{0 m}}}{L}\right)\right] \\ {\left[\ln \left(\frac{z_{\mathrm{a, h}}-d}{z_{\mathrm{0 h}}}\right)-\psi_{\mathrm{h}}\left(\frac{z_{\mathrm{a, h}}-d}{L}\right)+\psi_{\mathrm{h}}\left(\frac{z_{\mathrm{0 h}}}{L}\right)\right]}\end{array}
\end{equation}
\begin{equation}\label{raw}
  \begin{array}{c}r_{\mathrm{a w}}=\frac{1}{\kappa^{2} V_{\mathrm{a}}}\left[\ln \left(\frac{z_{\mathrm{a, m}}-d}{z_{\mathrm{0 m}}}\right)-\psi_{\mathrm{m}}\left(\frac{z_{\mathrm{a, m}}-d}{L}\right)+\psi_{\mathrm{m}}\left(\frac{z_{\mathrm{0 m}}}{L}\right)\right] \\ {\left[\ln \left(\frac{z_{\mathrm{a, w}}-d}{z_{\mathrm{0 w}}}\right)-\psi_{\mathrm{w}}\left(\frac{z_{\mathrm{a, w}}-d}{L}\right)+\psi_{\mathrm{w}}\left(\frac{z_{\mathrm{0 w}}}{L}\right)\right]}\end{array}
\end{equation}

为方便与地面观测资料进行比较,可定义2 m温度和湿度、以及10 m风速,它们实际上为$z_{0x}+d$以上2 m高度的温度、比湿,和10 m高度的风速,计算公式(即对原始微分方程从$z_{0x}+d$到$z_{0x}+d+2$ (或$z_{0x}+d+10$进行积分)如下:
  \begin{equation}\label{T2m}
    T_{\mathrm{2 m}}=\theta_{\mathrm{s}}+\frac{\theta_{*}}{\kappa}\left[\ln \left(\frac{2+z_{\mathrm{0 h}}}{z_{\mathrm{0 h}}}\right)-\psi_{\mathrm{h}}\left(\frac{2+z_{\mathrm{0 h}}}{L}\right)+\psi_{\mathrm{h}}\left(\frac{z_{\mathrm{0 h}}}{L}\right)\right]
  \end{equation}
  \begin{equation}\label{q2m}
    q_{\mathrm{2 m}}=q_{\mathrm{s}}+\frac{q_{*}}{\kappa}\left[\ln \left(\frac{2+z_{\mathrm{0 w}}}{z_{\mathrm{0 w}}}\right)-\psi_{\mathrm{w}}\left(\frac{2+z_{\mathrm{0 w}}}{L}\right)+\psi_{\mathrm{w}}\left(\frac{z_{\mathrm{0 w}}}{L}\right)\right]
  \end{equation}
  \begin{equation}\label{u10m}
    u_{\mathrm{10 m}}=\frac{u_{*}}{\kappa}\left[\ln \left(\frac{10+z_{\mathrm{0 m}}}{z_{\mathrm{0 m}}}\right)-\psi_{\mathrm{m}}\left(\frac{10+z_{\mathrm{0 m}}}{L}\right)+\psi_{\mathrm{m}}\left(\frac{z_{\mathrm{0 m}}}{L}\right)\right]
  \end{equation}


\section{考虑大涡影响的新理论}\label{考虑大涡影响的新理论}
\begin{mymdframed}{代码}
本节对应的代码文件为\texttt{MOD\_TurbulenceLEddy.F90}。
\end{mymdframed}

不稳定边界层的大涡对地表湍流交换具有重要影响,随着研究和认识的不断深入,本团队近年来在考虑大涡影响的基本理论方面有了新进展。\citet{liu2019further,liu2022surface}
通过引入边界层高度$z_{\mathrm {i}}$,发展了考虑大涡影响的地表湍流通量方案(图~\ref{fig:LZD2022方案概念图}),该方案需通过设置\texttt{DEF\_USE\_CBL\_HEIGHT} = .true.激活使用。包含大涡影响的动量通量-梯度关系通用函数形式为:
\begin{equation}
  \begin{aligned}
    \phi_{\mathrm{m}}(\zeta) &= B_{\mathrm{m}} (-\zeta)^{-1/2} \\[1ex]
    B_{\mathrm{m}} &= 0.0047(-\frac{z_{i}}{L})+0.1854
  \end{aligned}
\end{equation}
该方案(称为LZD2022方案)适用于不稳定条件($\zeta \leqslant -0.13$)。在弱不稳定条件下($\zeta > -0.13$),则仍然采用方程~\eqref{phim_zeng} (即Zeng1998方案)所列形式,确保
通用函数在由不稳定过渡到中性条件的连续性。
{
  \begin{figure}[htbp]
    \centering
    \includegraphics[scale=0.7]{Figures/地表湍流交换过程/LZD2022方案概念图.jpg}
    \caption{考虑大涡影响的地表通量方案示意图}
    \label{fig:LZD2022方案概念图}
  \end{figure}
}

根据~\citet{liu2023referenceheight},LZD2022方案曲线与Zeng1998方案的不稳定条件曲线相交(图~\ref{fig:LZD2022方案与Zeng1998方案曲线比较图}(a)),相交点的位置$\zeta_{\mathrm{m}}$与${z_{i}}/{L}$
有关(图~\ref{fig:LZD2022方案与Zeng1998方案曲线比较图}(b)),即$\zeta_{\mathrm{m}}$依赖于边界层稳定度${z_{i}}/{L}$,形式如下:
\begin{equation}
  \zeta_{\mathrm{m}}=\frac{-16-\sqrt{256+4 \left(B_{\mathrm{m}}\right)^{-4}}}{2 \left(B_{\mathrm{m}}\right)^{-4}}
\end{equation}
当相交点$\zeta_{\mathrm{m}}=-0.13$时,$B_{\mathrm{m}}=0.2722$。

结合LZD2022和Zeng1998方案,不稳定条件下($\zeta<0$)通用函数$\phi_{\mathrm{m}}$有如下表达式:
\begin{equation}
  \phi_{\mathrm{m}}(\zeta)= \begin{cases}
    B_{\mathrm{m2}}(-\zeta)^{-1/2}, & \zeta<\zeta_{\mathrm{m2}} \text { 时} \\
    (1-16 \zeta)^{-1/4}, & \zeta_{\mathrm{m2}} \leqslant \zeta<0 \text { 时} \\
  \end{cases}
\end{equation}
其中$B_{\mathrm{m2}}=\max(B_{\mathrm{m}},0.2722)$,$\zeta_{\mathrm{m2}}=\min(\zeta_{\mathrm{m}},-0.13)$,稳定条件下的$\phi_{\mathrm{m}}$则直接采用方程~\eqref{phim_zeng} 中形式。基于风速廓线方程~\eqref{Va},注意当
$\zeta_{\mathrm{a}}=\frac{z_{\mathrm{a,m}}-d}{L}<\zeta_{\mathrm{m2}}$时,方程~\eqref{Va} 中的$\psi_{\mathrm{m}}\left(\zeta_{\mathrm{a}}\right)$为:
\begin{align}
  \psi_{\mathrm{m}}\left(\zeta_{\mathrm{a}}\right) &= \int_{0}^{\zeta_{\mathrm{a}}} \frac{1-\phi_{\mathrm{m}}(s)}{s}{\mathrm d} s  \nonumber \\[1ex]
  & = \int_{0}^{\zeta_{\mathrm{m2}}} \frac{1-\phi_{\mathrm{m}}(s)}{s}{\mathrm d} s + \int_{\zeta_{\rm m2}}^{\zeta_{\mathrm{a}}} \frac{1-\phi_{\mathrm{m}}(s)}{s}{\mathrm d} s  \nonumber \\[1.5ex]
  &= \psi_{\mathrm{mu}}(\zeta_{\mathrm{m2}}) + \int_{\zeta_{\mathrm m2}}^{\zeta_{\mathrm{a}}} \frac{1-B_{\mathrm{m2}}(-s)^{-1/2}}{s}{\mathrm d} s  \nonumber \\[1.5ex]
  & = \psi_{\mathrm{mu}}(\zeta_{\mathrm{m2}}) + \ln \frac{\zeta_{\mathrm{a}}}{\zeta_{\mathrm{m2}}} + 2B_{\mathrm{m2}}\left[(-\zeta_{\mathrm{a}})^{-1/2}-(-\zeta_{\mathrm{m2}})^{-1/2}\right]
\end{align}
可得到风速廓线在不稳定条件下的具体形式:

\noindent 当$\zeta_{\mathrm{a}}<\zeta_{\mathrm{m2}}$时
\begin{equation}\label{Va_U_LZD1}
  V_{\mathrm{a}}=\frac{u_{*}}{\kappa}\left\{\ln \frac{L\zeta_{\mathrm{m2}}}{z_{\mathrm{0 m}}}-\psi_{\mathrm{mu}}\left(\zeta_{\mathrm{m2}}\right)-2B_{\mathrm{m2}}\left[(-\zeta_{\mathrm{a}})^{-1/2}-(-\zeta_{\mathrm{m2}})^{-1/2}\right]+\psi_{\mathrm{mu}}\left(\frac{z_{\mathrm{0 m}}}{L}\right)\right\}
\end{equation}
\noindent 当$ \zeta_{\mathrm{m2}} \leqslant \zeta_{\mathrm{a}}<0$时
\begin{equation}\label{Va_U_LZD2}
  V_{\mathrm{a}}=\frac{u_{*}}{\kappa}\left\{\ln \frac{z_{\mathrm{a, m}}-d}{z_{\mathrm{0 m}}}-\psi_{\mathrm{mu}}\left(\zeta_{\mathrm{a}}\right)+\psi_{\mathrm{mu}}\left(\frac{z_{\mathrm{0 m}}}{L}\right)\right\}
\end{equation}
其中$\psi_{\mathrm{mu}}$的定义同方程(\ref{Psim}),这里同样限制$V_{\mathrm {a}}\geqslant0.1$以避免风速过小问题,参考高度风速亦引入对流速度$U_{\mathrm {c}}$ (公式(\ref{VaIni})),其代表大涡在水平方向的影响,即
$V_{\mathrm{a}}=\sqrt{u_{\mathrm{a}}^{2}+v_{\mathrm{a}}^{2}+U_{\mathrm{c}}^{2}} \geqslant 0.1$。稳定条件下的风速积分形式则同方程~\eqref{Va_S} 和~\eqref{Va_VS}。

后面的迭代求解过程参见~\ref{基本理论} 节内容,所得空气动力学阻抗系数$r_{\mathrm{am}}$、$r_{\mathrm{ah}}$和$r_{\mathrm{aw}}$,以及诊断量2 m温度和湿度、10 m风速等的形式都和~\ref{基本理论} 节完全一致。
{
  \begin{figure}[htbp]
    \centering
    \includegraphics[scale=0.7]{Figures/地表湍流交换过程/LZD2022方案与Zeng1998方案曲线比较图.png}
    \caption{LZD2022方案与Zeng1998方案曲线比较图}
    \label{fig:LZD2022方案与Zeng1998方案曲线比较图}
  \end{figure}
}


\section{无植被覆盖地表湍流通量的计算方案}\label{无植被覆盖地表湍流通量的计算方案}
\begin{mymdframed}{代码}
  本节对应的代码文件为\texttt{MOD\_GroundFluxes.F90},部分代码包含于\texttt{MOD\_Thermal.F90}和\texttt{MOD\_SnowFraction.F90}。
\end{mymdframed}

当陆地表面没有植被覆盖(裸土、城市和湿地覆盖时)或植被已被积雪掩埋时,湍流通量按照无植被覆盖情形的方案计算。
对于任一地表类型斑块,已知植被粗糙度$z_{\mathrm{0mv}}$时,则被积雪掩埋的植被占总植被的比例为:
\begin{equation}
  {\mathrm {wt}}=\frac{0.1 z_{\mathrm{sno}}}{z_{\mathrm{0mv}}+0.1 z_{\mathrm{sno}}}
\end{equation}
其中$z_{\mathrm{sno}}$表示积雪厚度(m)。CoLM2014及以前版本计算斑块中的有效植被比例$f_{\mathrm{sig}}=\left(1-{\mathrm {wt}}\right)f_{\mathrm{c}}$,无植被覆盖比例为$\left(1-f_{\mathrm{sig}}\right)$。CoLM2024版本为了考虑与PFT次网格类型的兼容性,同时认为积雪是通过覆盖或掩埋植被叶面积、茎面积进行影响,将$\rm wt$用于修正被积雪掩盖后的SAI,即${\rm SAI=TSAI}\left(1-{\mathrm {wt}}\right)$,TSAI为植被“真实”茎面积指数。当采用卫星遥感LAI时,由于其数值已经是积雪覆盖下的绿色叶面部分,故在此不对其进行积雪覆盖调整。


陆地表面可分为被积雪覆盖与未被积雪覆盖两部分。根据 \citet{swenson2012new}提供的方法,
被积雪覆盖的地表面积比例$f_{\mathrm{sno}}$可分为两步计算:在积分开始时若有降雪发生,则新一步的$f_{\mathrm{sno}}$更新为:
\begin{equation}
  f_{\mathrm{{sno }}}^{(n+1)}=1-\left[1-\tanh\left(0.1 p_{\mathrm{snow}} \Delta t\right)\right]\left(1-f_{\mathrm{{sno }}}^{(n)}\right) \leqslant 1.0
\end{equation}
其中$p_{\mathrm {snow}}$为降雪率(\unit{kg.m^{-2}.s^{-1}}),$\Delta t$为积分步长(s);在水热过程模拟结束后,若有积雪融化发生,则$f_{\mathrm{sno}}$更新为:
\begin{equation}
  f_{\mathrm{sno}}^{(n+1)}=\tanh{\left(\frac{100 z_{\mathrm{s n o}}^{2}}{2.5 z_{\mathrm{lnd}} W_{\mathrm{sno}}}\right)}
\end{equation}
其中$W_{\mathrm{sno}}$为雪水当量(mm),$z_{\mathrm{lnd}}$ 为未被积雪覆盖时的裸土地表粗糙度,取值为0.01 m。

对于无植被覆盖部分的陆地表面,湍流通量只存在于地表与大气之间,此部分动量通量$\tau$、感热通量$H$和水汽通量$E$表达为:
\begin{equation}\label{tau_x}
  \tau_{\mathrm{x}}=-\rho_{\mathrm{a}} \frac{u_{\mathrm{a}}}{r_{\mathrm{a m}}}
\end{equation}
\begin{equation}\label{tau_y}
  \tau_{\mathrm{y}}=-\rho_{\mathrm{a}} \frac{v_{\mathrm{a}}}{r_{\mathrm{a m}}}
\end{equation}
\begin{equation}\label{Hg}
  H_{\mathrm{g}}=-\rho_{\mathrm{a}} C_{\mathrm{a}} \frac{\theta_{\mathrm{a}}-T_{\mathrm{g}}}{r_{\mathrm{a h}}}
\end{equation}
\begin{equation}\label{Eg}
  E_{\mathrm{g}}=-\rho_{\mathrm{a}} \frac{q_{\mathrm{a}}-q_{\mathrm{g}}}{r_{\mathrm{a w}}}
\end{equation}
其中$T_{\mathrm {g}}$表示地表温度(K)。当不区分对待表层土壤和积雪时(namelist \allowbreak \texttt{DEF\_SPLIT\_\allowbreak SOILSNOW \allowbreak = \allowbreak false}),有积雪覆盖(即$f_{\mathrm{sno}}>0$)时,$T_{\mathrm {g}}$为最上层积雪的温度$T_{snl+1}$;无积雪覆盖时,$T_{\mathrm {g}}$为第一层土壤的温度$T_1$。$q_{\mathrm {g}}$表示地表空气比湿(\unit{kg.kg^{-1}}),其计算公式为:
\begin{equation}\label{qg}
  q_{\mathrm{g}}=\left(1-f_{\mathrm{{sno }}}\right) q_{\mathrm{{soil }}}+f_{\mathrm{{sno }}} q_{\mathrm{{sno }}}
\end{equation}
其中积雪表面的比湿$q_{\rm sno}$为温度在$T_{snl+1}$时的饱和比湿$q_{\mathrm{sno}}=q_{\mathrm{sat}}^{T_{snl+1}}$
(计算方案见附录~\ref{饱和水汽压(比湿)及其随温度的变化}),土壤表面的比湿$q_{\mathrm{soil}}$视为正比于温度在$T_1$时的饱和比湿:$q_{\mathrm{soil}}=\alpha_{\mathrm{soil}}q_{\mathrm{sat}}^{T_1}$,
比例系数$\alpha_{\mathrm{soil}}$为表层土壤水势$\psi_1$(mm)的函数\citep{philip1957theory}:$\alpha_{\mathrm{soil}}=\exp \left(\frac{\psi_1g}{{10}^3R_{\mathrm{v}}T_1}\right)$,
其中$R_{\mathrm{v}}$表示水汽气体常数(\unit{J.kg^{-1}.K^{-1}})。表层土壤水势$\psi_1$的计算公式为$\psi_1=\psi_{\mathrm{sat,1}}s_1^{-B_1}\geqslant-1\times{10}^8$,
$\psi_{\mathrm{sat,1}}$表示表层饱和土壤水势(mm),$B_1$表示表层土壤\citep{clapp1978empirical}参数(均由地表参数数据集提供)。$s_1$表示表层土壤对于饱和状态时的相对湿度:
\begin{equation}
  s_{1}= \begin{cases}
    0.001, & \text {如果 }\ \theta_{\mathrm{sat, 1}}<1 \times 10^{-6} \\
    \frac{1}{\Delta z_{1} \theta_{\mathrm{sat, 1}}}\left[\frac{w_{\mathrm{liq, 1}}}{\rho_{\mathrm{liq}}}+\frac{w_{\mathrm{ice, 1}}}{\rho_{\mathrm{ice}}}\right]  \text{ 且  }\  0.001 \leqslant s_{1} \leqslant 1.0, & \text {如果 }\ \theta_{\mathrm{sat, 1}} \geqslant 1 \times 10^{-6}
  \end{cases}
\end{equation}
其中$\Delta z_{1}$为表层土壤厚度(m),$\rho_{\mathrm{liq}}$和$\rho_{\mathrm{ice}}$分别为液态水和固态水密度(\unit{kg.m^{-3}}),
$w_{\mathrm{liq,1}}$和$w_{\mathrm{ice,1}}$分别为表层土壤液态水和固态水含量(\unit{kg.m^{-2}}),
$\theta_{\mathrm{sat,1}}$表示表层土壤体积含水量(即孔隙度,\unit{mm^3.mm^{-3}})。
注:为避免土壤水含量极低时轻微的土壤水变化会导致$q_{\mathrm{soil}}$变化过大,
当$q_{\mathrm{sat}}^{T_1}>q_{\mathrm{a}}$且$q_{\mathrm{a}}>q_{\mathrm{soil}}$时,取$q_{\mathrm{soil}} = q_{\mathrm{a}}$且$q_{\mathrm{soil}}$随土壤表层温度变化导数为0,即$\frac{{\rm d}q_{\mathrm{soil}}}{{\rm d}T_1} = 0$。


由于地表无植被覆盖,在计算阻抗系数$r_{\mathrm{am}}$、$r_{\mathrm{ah}}$、$r_{\mathrm{aw}}$时,零平面位移取为$d=0$。动量粗糙度在无积雪覆盖时取为$z_{\mathrm{0m}}=0.01$ m,有积雪覆盖(即$f_{\mathrm{sno}}>0$)时为积雪、裸土覆盖面积加权平均,即$z_{\mathrm{0m}}=0.01\left( 1-f_{\mathrm{sno}} \right )+ 0.0024 f_{\mathrm{sno}}$。
由于动量输送会受到粗糙物后面湍流波中气压波动的影响,而热量和水汽的输送过程不涉及这种动力学机制,热量和水汽必须通过界面的分子扩散进行传输,
故感热和水汽的粗糙度($z_{\mathrm{0h}}$,$z_{\mathrm{0w}}$)不同于动量粗糙度,根据~\citet{zeng1998effect}:
\begin{equation}\label{z0hw}
  z_{\mathrm{0 h}} = z_{\mathrm{0 w}} = z_{\mathrm{0 m}} \exp \left[-0.13\left(Re_{*}\right)^{0.45}\right]
\end{equation}
其中$Re_{*} = u_{*} \cdot z_{\mathrm{0 m}} / \upsilon$为粗糙雷诺数(可理解为最小湍涡的雷诺数),$\upsilon=$ \qty{1.5e-5}{(m^2.s^{-1}})为空气的动力粘性系数。


综上,无植被覆盖部分的地表湍流通量的数值计算过程可总结如下:
\begin{enumerate}
  \item 根据公式(\ref{VaIni})给出风速$V_{\mathrm{a}}$的初始猜测,其中对流速度$U_{\mathrm {c}}$的初始猜测值为:
    \begin{equation}
      U_{\mathrm{c}}= \begin{cases}
        0, & \theta_{\mathrm{a,v}}-\theta_{\mathrm{s,v}} \geqslant 0\ \text { (即稳定条件下) } \\
        0.5, & \theta_{\mathrm{a,v}}-\theta_{\mathrm{s,v}}<0\ \text { (即不稳定条件下) }
      \end{cases}
    \end{equation}
  \item 基于$R_{\mathrm{ib}}$,根据公式(\ref{Rib})和(\ref{ZetaRib}),给出Monin--Obukhov长度$L$的初始猜测;
  \item 以下过程迭代6次:\\
    a. 计算$u_\ast$ (Zeng1998方案:公式(\ref{Va_VU})-(\ref{Va_VS});LZD2022方案:公式(\ref{Va_U_LZD1})、(\ref{Va_U_LZD2})、(\ref{Va_S})--(\ref{Va_VS})) \\
    b. 计算$\theta_\ast$ (公式(\ref{theta_VU})--(\ref{theta_VS})) \\
    c. 计算$q_\ast$ (公式(\ref{q_VU})--(\ref{q_VS})) \\
    d. 更新感热和水汽粗糙度$z_{\mathrm{0h}}$、$z_{\mathrm{0w}}$ (公式~\eqref{z0hw}) \\
    e. 计算虚位温特征尺度$\theta_{\mathrm{v\ast}}$ (公式(\ref{thvstar})) \\
    f. 更新大气风速$V_{\mathrm {a}}$ (公式~\eqref{VaIni}) \\
    g. 计算新一步$L$ (公式~\eqref{ObukL}) \\
    注:每次迭代完成后,判断$L$是否改变符号,若改变符号超过4次,则视为中性条件,跳出迭代过程,以避免在稳定与不稳定条件之间来回变化;
  \item 计算空气动力学阻抗系数$r_{\mathrm{am}}$、$r_{\mathrm{ah}}$、$r_{\mathrm{aw}}$ (公式(\ref{ram})--(\ref{raw}))
  \item 计算动量通量$\tau_{\mathrm{x}}$和$\tau_{\mathrm{y}}$、感热通量$H$、水汽通量$E$ (公式(\ref{tau_x})--(\ref{Eg}))
  \item 计算地表2 m气温$T_{\mathrm{2m}}$和比湿$q_{\mathrm{2m}}$ (公式(\ref{T2m})和(\ref{q2m}))
\end{enumerate}

其后在计算地表温度并基于该温度变化对感热和水汽通量进行更新时,需提供感热和水汽通量相对地面温度的变化率,由公式(\ref{SH})和(\ref{LH})可得:
\begin{equation}
  \frac{\partial H_{\mathrm{g}}}{\partial T_{\mathrm{g}}}=\frac{\rho_{\mathrm{a}} C_{\mathrm{a}}}{r_{\mathrm{a h}}}
\end{equation}
\begin{equation}\label{Eg/Tg_1}
  \frac{\partial E_{\mathrm{g}}}{\partial T_{\mathrm{g}}}= \frac{\rho_{\mathrm{a}}}{r_{\mathrm{a w}}} \frac{{\rm d} q_{\mathrm{g}}}{{\rm d} T_{\mathrm{g}}}
\end{equation}
其中$\frac{{\rm d}q_{\mathrm {g}}}{{\rm d}T_{\mathrm {g}}}=\left[\left(1-f_{\mathrm{sno}}\right)\alpha_{\mathrm{soil}}+f_{\mathrm{sno}}\right]\frac{{\rm d}q_{\mathrm{sat}}^{T_{\mathrm {g}}}}{{\rm d}T_{\mathrm {g}}}$ (根据公式(\ref{qg})得到),$\frac{{\rm d}q_{\mathrm{sat}}^{T_{\mathrm {g}}}}{{\rm d}T_{\mathrm {g}}}$的计算见附录~\ref{饱和水汽压(比湿)及其随温度的变化}。
由于阻抗系数$r_{\mathrm{ah}}$、$r_{\mathrm{aw}}$对于温度的变化率无法解析给出,故暂时忽略。


\section{一维植被湍流交换模型}\label{一维植被湍流交换模型}
\begin{mymdframed}{代码}
  本节对应的代码文件为\texttt{MOD\_LeafTemperature.F90、MOD\_Thermal.F90}。
\end{mymdframed}

对于有植被覆盖部分的陆地表面,陆地与大气总湍流输运通量为植被冠层周围空气与大气之间的湍流输送,其动量通量$\tau$、感热通量$H$和水汽通量$E$表达为:
\begin{equation}
  \tau_{\mathrm{x}}=-\rho_{\mathrm{a}} \frac{u_{\mathrm{a}}}{r_{\mathrm{a m}}}
\end{equation}
\begin{equation}
  \tau_{\mathrm{y}}=-\rho_{\mathrm{a}} \frac{v_{\mathrm{a}}}{r_{\mathrm{a m}}}
\end{equation}
\begin{equation}
  H=-\rho_{\mathrm{a}} C_{\mathrm{a}} \frac{\theta_{\mathrm{a}}-T_{\mathrm{s}}}{r_{\mathrm{a h}}}
\end{equation}
\begin{equation}
  E=-\rho_{\mathrm{a}} \frac{q_{\mathrm{a}}-q_{\mathrm{s}}}{r_{\mathrm{a w}}}
\end{equation}
由于此时湍流通量与叶片温度耦合紧密,故阻抗系数$r_{\mathrm{am}}$、$r_{\mathrm{ah}}$、$r_{\mathrm{aw}}$与植被冠层周围($d+z_{\mathrm{0x}}$)空气温度$T_{\mathrm {s}}$和比湿$q_{\mathrm {s}}$将随叶片温度一起采用牛顿迭代法进行求解(见章节~\ref{植被叶片温度计算})。
本节就感热与水汽通量先给出其求解的理论表达。


假设植被冠层空气的热存储能力很小并可以忽略,则它与大气之间的热量和水汽交换来源于两部分:
一是地面与植被冠层周围空气之间的湍流通量,二是植被表面(茎、叶等)与植被冠层周围空气之间的湍流通量。
对于植被的描述,植被生理过程采用 \citet{dai2004two}提出的双大叶模型,即将叶片分为阴叶(无阳光直射)与阳叶(有阳光直射)两部分。
阳叶的比例计算为:
\begin{equation}
  f_{\mathrm{sun}}=\frac{1-\exp (-K \cdot \text {LAI})}{K \cdot \text {LAI}}
\end{equation}
其中LAI为叶面积指数,$K$为直射太阳辐射消光系数(见章节~\ref{短波吸收辐射通量}),${10}^{-6}\leqslant K \cdot \text {LAI}\le40$。
当白天植被吸收的太阳辐射小于1 \unit{W.m^{-2}} 或白天结束时,$f_{\mathrm{sun}}=0.5$。叶面积指数也对应表达为两部分:
\begin{equation}
  \begin{array}{c} \text {LAI}_{\mathrm{sun}}= \text {LAI} \cdot f_{\mathrm{sun}} \\ \text {LAI}_{\mathrm{sha}}=\text {LAI} \cdot \left(1-f_{\mathrm{sun}}\right)\end{array}
\end{equation}

CoLM2014对阴阳叶分别计算温度和湍流通量。CoLM2024版本为了减少阴阳叶温度同时进行迭代的不收敛问题,并考虑与植物群落PC次网格下多种植被温度同时迭代的兼容性,植被的湍流通量与叶片温度简化为单叶模型计算,但其他过程,如辐射传输、气孔阻抗计算、植被水力等过程均为双大叶模型。湍流通量计算主要过程描述如下:

当地表有植被覆盖时,假设植被树冠水平空间分布均匀且为单层结构(即一维植被结构),则植被冠层周围空气与大气之间的感热通量$H$须等于其两项来源之和:
\begin{equation}\label{HV_balance}
  H=H_{\mathrm{g}}+H_{\mathrm{v}}
\end{equation}
其中$H_{\mathrm{g}}$表示地面与植被冠层空气顶之间的感热通量,$H_{\mathrm{v}}$表示植被冠层与其周围空气之间的感热通量,计算表达式如下:
\begin{equation}
  H_{\mathrm{g}}=-\rho_{\mathrm{a}} C_{\mathrm{a}} \frac{T_{\mathrm{s}}-T_{\mathrm{g}}}{r_{\mathrm{a h}}^{\prime}}
\end{equation}
\begin{equation}
  H_{\mathrm{v}}=-\rho_{\mathrm{a}} C_{\mathrm{a}}(\text {LAI}+\text {SAI}) \frac{T_{\mathrm{s}}-T_{\mathrm{v}}}{r_{\mathrm{b}}}
\end{equation}
其中$T_{\mathrm {v}}$表示叶片温度(K),$T_{\mathrm {s}}$表示定义在高度$z_{\mathrm{0h}}+d$上的植被冠层顶空气温度(K),SAI表示茎面积指数,$r_{\mathrm {b}}$表示叶面边界层阻抗(\unit{s.m^{-1}}),$r_{\mathrm{ah}}^\prime$(即$r_{\mathrm {d}}$)表示地表与植被冠层周围空气之间的感热阻抗(\unit{s.m^{-1}}),各部分过程如图~\ref{fig:有植被覆盖部分的陆地表面感热通量示意图} 所示。
{
  \begin{figure}[htbp]
    \centering
    \includegraphics[width=0.6\linewidth]{Figures/地表湍流交换过程/有植被感热交换阻抗示意图.png}
    \caption{有植被覆盖时的陆地表面感热通量计算示意图}
    \label{fig:有植被覆盖部分的陆地表面感热通量示意图}
  \end{figure}
}

将$H$、$H_{\mathrm{g}}$、$H_{\mathrm{v}}$的表达式代入平衡方程(\ref{HV_balance}),即可解得$T_{\mathrm {s}}$形式如下:
\begin{equation}
  T_{\mathrm{s}}=\frac{c_{\mathrm{ah}} \theta_{\mathrm{a}}+c_{\mathrm{gh}} T_{\mathrm{g}}+c_{\mathrm{vh}} T_{\mathrm{v}}}{c_{\mathrm{ah}}+c_{\mathrm{gh}}+c_{\mathrm{vh}}}
\end{equation}
其中$c_{\mathrm{ah}}=\frac{1}{r_{\mathrm{ah}}}$,$c_{\mathrm{gh}}=\frac{1}{r_{\mathrm{ah}}^\prime}$,$c_{\mathrm{vh}}=\frac{\text {LAI}+\text {SAI}}{r_{\mathrm {b}}}$,
将$T_{\mathrm {s}}$的表达式代回,即可将$H$、$H_{\mathrm{g}}$、$H_{\mathrm{v}}$全部写为$\theta_{\mathrm{a}}$、$T_{\mathrm {g}}$、$T_{\mathrm v}$的函数,例如:
\begin{equation}
  H_{\mathrm{v}}=-\rho_{\mathrm{a}} C_{\mathrm{a}}\left[c_{\mathrm{ah}} \theta_{\mathrm{a}}+c_{\mathrm{gh}}
    T_{\mathrm{g}}+c_{\mathrm{vh}} T_{\mathrm{v}}-\left(c_{\mathrm{ah}}+c_{\mathrm{gh}}+c_{\mathrm{vh}}\right)
  T_{\mathrm{v}}\right] \frac{c_{\mathrm{vh}}}{c_{\mathrm{ah}}+c_{\mathrm{gh}}+c_{\mathrm{vh}}}
\end{equation}


同样,植被冠层周围空气与大气之间的水汽通量$E$可与另外两项平衡:
\begin{equation}\label{EV_balance}
  E=E_{\mathrm{g}}+E_{\mathrm{v}}
\end{equation}
其中$E_{\mathrm{g}}$表示地表与植被冠层周围空气之间的水汽通量,$E_{\mathrm{v}}$表示植被冠层与其周围空气之间的水汽通量:
\begin{equation}\label{eq:Eg}
  E_{\mathrm{g}}=-\rho_{\mathrm{a}} \frac{q_{\mathrm{s}}-q_{\mathrm{g}}}{r_{\mathrm{a w}}^{\prime}}
\end{equation}
\begin{equation}
  E_{\mathrm{v}}=-\rho_{\mathrm{a}} \frac{q_{\mathrm{s}}-q_{\mathrm{s a t}}^{T v}}{r_{\mathrm{vw}}}
\end{equation}
其中$q_{\mathrm{sat}}^{T_{\mathrm v}}$表示叶片温度在$T_{\mathrm v}$下的饱和比湿(\unit{kg.kg^{-1}}),$q_{\mathrm {s}}$为在高度$z_{\mathrm{0w}}+d$上的植被冠层周围空气比湿(\unit{kg.kg^{-1}}),
$q_{\mathrm {g}}$表示地面空气比湿(\unit{kg.kg^{-1}},其表达见章节~\ref{无植被覆盖地表湍流通量的计算方案}),$r_{\mathrm{aw}}^\prime$(即$r_{\mathrm {d}}$)表示地面与植被冠层顶空气之间的水汽交换阻抗 (\unit{s.m^{-1}})。
$r_{\mathrm{v}}$表示植被表面与植被冠层顶空气之间水汽交换总阻抗 (\unit{s.m^{-1}}),
它来自叶面边界层阻抗$r_{\mathrm {b}}$和气孔阻抗$r_{\mathrm{s,sun}}\left(r_{\mathrm{s,sha}}\right)$的双重贡献。
植被冠层与其周围空气之间的水汽交换包含两部分:湿的叶面及茎面的蒸发水汽交换(通过阻抗$r_{\mathrm {b}}$)和干叶表面的蒸腾水汽通量(通过阻抗$r_{\mathrm {b}}$,$r_{\mathrm{s,sun}}$和$r_{\mathrm{s,sha}}$)。
湿的叶面茎面积占总叶面茎面积的比例 \citep{dickinson1993biosphere} 为:
\begin{equation}\label{eq:fwet}
  f_{\mathrm{wet}}=\left(\frac{L_{\mathrm{dew}}}{L_{\mathrm{dew,max }}}\right)^{2 / 3} \leqslant 1.0
\end{equation}
其中$L_{\mathrm{dew}}$表示植被表面储水量(\unit{kg.m^{-2}},其计算见章节~\ref{植被冠层截留}),$L_{\mathrm{dew,max}}$表示植被表面最大储水量
(\unit{kg.m^{-2}}),$L_{\mathrm{dew,max}}=0.1\left(\text {LAI}+ \text {SAI}\right)$。
有植被覆盖部分的陆地表面水汽通量交换过程如图~\ref{fig:有植被覆盖部分的陆地表面水汽通量示意图} 所示。
{
  \begin{figure}[htbp]
    \centering
    \includegraphics[width=0.6\linewidth]{Figures/地表湍流交换过程/有植被潜热交换阻抗示意图.png}
    \caption{有植被覆盖时的陆地表面水汽通量计算示意图}
    \label{fig:有植被覆盖部分的陆地表面水汽通量示意图}
  \end{figure}
}

将$E$、$E_{\mathrm{g}}$、$E_{\mathrm{v}}$的表达式代入平衡方程(\ref{EV_balance}),即可解得$q_{\mathrm {s}}$表达式如下:
\begin{equation}\label{Eg_2}
  q_{\mathrm{s}}=\frac{c_{\mathrm{aw}} q_{\mathrm{a}}+c_{\mathrm{gw}} q_{\mathrm{g}}+c_{\mathrm{vw}} q_{\mathrm{s a t}}^{T_{\mathrm{v}}}}{c_{\mathrm{aw}}+c_{\mathrm{gw}}+c_{\mathrm{vw}}}
\end{equation}
其中$c_{\mathrm{aw}}=\frac{1}{r_{\mathrm{aw}}}$,$c_{\mathrm{gw}}=\frac{1}{r_{\mathrm{aw}}^\prime}$,$c_{\mathrm{vw}}=\frac{1}{r_{\mathrm{vw}}}$,
将$q_{\mathrm {s}}$的表达式代回,即可将$E$、$E_{\mathrm{g}}$、$E_{\mathrm{v}}$全部写为$q_{\mathrm{a}}$、$q_{\mathrm {g}}$、$q_{\mathrm{sat}}^{T_{\mathrm v}}$的函数,例如:
\begin{equation}\label{Ev}
  E_{\mathrm{v}}=-\rho_{\mathrm{a}}\left[c_{\mathrm{aw}} q_{\mathrm{a}}+c_{\mathrm{gw}} q_{\mathrm{g}}+c_{\mathrm{vw}} q_{\mathrm{s a t}}^{T_{\mathrm{v}}}-
  \left(c_{\mathrm{aw}}+c_{\mathrm{gw}}+c_{\mathrm{vw}}\right) q_{\mathrm{s a t}}^{T_{\mathrm{v}}}\right] \frac{c_{\mathrm{vw}}}{c_{\mathrm{aw}}+c_{\mathrm{gw}}+c_{\mathrm{vw}}}
\end{equation}
根据植被冠层与其周围空气之间的水汽交换为湿的叶面与茎面的蒸发水汽和干叶表面的蒸腾水汽之和,$c_{\mathrm{vw}}$可由下式导出。
当$q_{\mathrm{sat}}^{T_{\mathrm v}}>q_{\mathrm {s}}$即叶片蒸散发可以发生时,%总水汽通量为叶片蒸发通量和叶片蒸腾通量之和:
\begin{equation}
  \underbrace{-\rho_{\mathrm{a}} \frac{q_{\mathrm{s}}-q_{\mathrm{s a t}}^{T_{\mathrm{v}}}}{r_{\mathrm{{v }}}}}_{\text{总水汽通量}}
  =\underbrace{-\rho_{\mathrm{a}}
  \frac{q_{\mathrm{s}}-q_{\mathrm{s a t}}^{T_{\mathrm{v}}}}{\frac{r_{\mathrm{b}}}{(\text {LAI}+\text {SAI}) \cdot f_{\mathrm{{wet }}}}}}_{\text{叶片蒸发通量}}
  \underbrace{-\rho_{\mathrm{a}} \frac{q_{\mathrm{s}}-q_{\mathrm{s a t}}^{T_{\mathrm{v}}}}{\frac{1}{\text {LAI} \cdot \left(1-f_{\mathrm{wet}}\right)\left(\frac{f_{\mathrm{sun}}}{r_{\mathrm{b}}+r_{\mathrm{s,sun}}} + \frac{f_{\mathrm{sha}}}{r_{\mathrm{b}}+r_{\mathrm{s,sha}}}\right)}}}_{\text{叶片蒸腾通量}}
\end{equation}
于是,
\begin{equation}
  c_{\mathrm{vw}}=\frac{1}{r_{\mathrm{vw}}}=\frac{(\text {LAI}+\text {SAI}) \cdot f_{\mathrm{{wet }}}}{r_{\mathrm{b}}}+\text {LAI} \cdot \left(1-f_{\mathrm{wet}}\right)\left(\frac{f_{\mathrm{sun}}}{r_{\mathrm{b}}+r_{\mathrm{s,sun}}} + \frac{f_{\mathrm{sha}}}{r_{\mathrm{b}}+r_{\mathrm{s,sha}}}\right)
\end{equation}
当$q_{\mathrm{sat}}^{T_{\mathrm v}}\leqslant q_{\mathrm {s}}$即植被冠层空气中的水蒸气可以发生液化时,
$-\rho_{\mathrm{a}}\frac{q_{\mathrm {s}}-q_{\mathrm{sat}}^{T_{\mathrm v}}}{r_{\mathrm{v}}}\ =-\rho_{\mathrm{a}} \frac{q_{\mathrm {s}}-q_{\mathrm{sat}}^{T_{\mathrm v}}}{r_{\mathrm {b}}/\left(\text {LAI}+\text {SAI}\right)}$,于是
\begin{equation}
  c_{\mathrm{vw}}=\frac{1}{r_{\mathrm{{vw }}}}=\frac{\text {LAI}+\text {SAI}}{r_{\mathrm{b}}}
\end{equation}

下面给出各个阻抗系数的计算方案。首先,植被冠层周围空气与大气的湍流阻抗系数$r_{\mathrm{am}}$、$r_{\mathrm{ah}}$、$r_{\mathrm{aw}}$的计算方案同章节~\ref{无植被覆盖地表湍流通量的计算方案},
动量、热量与水汽通量的零平面位移$d$与粗糙度$z_{\mathrm{0m}}$、$z_{\mathrm{0h}}$、$z_{\mathrm{0w}}$可以通过土地覆盖类型查找表得到(见
表~\ref{tab:USGS地表覆盖粗糙度及零平面位移与植被高度比值}),或者通过计算得到(见章节~\ref{百分百植被覆盖湍流} 和~\ref{考虑稳定度粗糙度方案})。

其次,对于地表与植被冠层空气的湍流阻抗系数$r_{\mathrm{ah}}^\prime$、$r_{\mathrm{aw}}^\prime$,方案之一为CoLM2014原有方案,计算公式为:
\begin{equation}
  r_{\mathrm{a h}}^{\prime}=r_{\mathrm{a w}}^{\prime}=\frac{1}{C_{\mathrm{s}} u_{*}}:=r_{\mathrm{d}}
\end{equation}
其中$C_{\mathrm {s}}$表示地表与植被冠层空气之间的湍流交换系数,它由裸土表面的数值与浓密植被覆盖时的数值进行线性组合得到 \citep{zeng2005vegetation}:
$C_{\mathrm {s}}=C_{\mathrm{s,bare}}W+C_{\mathrm{s,dense}}\left(1-W\right)$,其中权重$W=\exp {\left[-(\text{LAI}+\text{SAI})\right]}$,
浓密植被覆盖时的湍流交换系数$C_{\mathrm{s,dense}}=0.004$,裸土表面的湍流交换系数$C_{\mathrm{s,bare}}=\frac{\kappa}{0.13}\left(\frac{z_{\mathrm{0m,g}}u_\ast}{\upsilon}\right)^{-0.45}$,
其中空气动力学粘性系数$\upsilon=$ \qty{1.5e-5}{m^2.s^{-1}},地表粗糙度$z_{\mathrm{0m,g}}=0.01$ {m}。另一种方案是采植被冠层内部湍流交换系数垂直剖面积分得到(见章节~\ref{百分百植被覆盖湍流})。

对于叶面边界层阻抗$r_{\mathrm {b}}$,计算公式为:
\begin{equation}
  r_{\mathrm{b}}=\frac{1}{c_{\mathrm{l}}}\left(u_{*} / d_{\mathrm{{leaf }}}\right)^{-0.5}
\end{equation}
其中$c_{\mathrm {l}}=$ \qty{0.01}{m.s^{-0.5}} ,表示植被表面与植被冠层空气之间的湍流交换系数,$d_{\mathrm{leaf}}=0.04$ m 表示叶片在风向的特征值。类似于$r_{\mathrm {d}}$的计算,$r_{\mathrm {b}}$同样可以通过植被冠层内部风速廓线积分求解得到(见章节~\ref{百分百植被覆盖湍流})。
气孔阻抗系数$r_{\mathrm{s,sun}}$和$r_{\mathrm{s,sha}}$的计算将在光合作用部分介绍(章节~\ref{植物的光合作用})。

同样,其后计算地表温度并基于该温度变化对地表感热和水汽通量进行更新时,需提供地表感热和水汽通量对地表温度的变化率,计算为:
%\begin{equation}
\begin{align}
  \frac{\partial H_{\mathrm{g}}}{\partial T_{\mathrm{g}}} & = \frac{\rho_{\mathrm{a}} C_{\mathrm{a}}}{r_{\mathrm{a h}}^{\prime}}
  \frac{c_{\mathrm{ah}}+c_{\mathrm{vh}}}{c_{\mathrm{ah}}+c_{\mathrm{gh}}+c_{\mathrm{vh}}} \\[2ex]
%
  \frac{\partial E_{\mathrm{g}}}{\partial T_{\mathrm{g}}} & =
  \frac{\rho_{\mathrm{a}}}{r_{\mathrm{a w}}^{\prime}} \frac{c_{\mathrm{a}}^{w}+c_{\mathrm{vw}}}{c_{\mathrm{aw}}+c_{\mathrm{gw}}+c_{\mathrm{vw}}} \frac{{\rm d} q_{\mathrm{g}}}{{\rm d} T_{\mathrm{g}}}\label{Eg/Tg_2}
\end{align}
%\end{equation}
其中$\frac{{\rm d}q_{\mathrm {g}}}{{\rm d}T_{\mathrm {g}}}=\left[\left(1-f_{\mathrm{sno}}\right)\alpha_{\mathrm{soil}}+f_{\mathrm{sno}}\right]\frac{{\rm d}q_{\mathrm{sat}}^{T_{\mathrm {g}}}}{{\rm d}T_{\mathrm {g}}}$,
$\frac{{\rm d}q_{\mathrm{sat}}^{T_{\mathrm {g}}}}{{\rm d}T_{\mathrm {g}}}$的计算见附录章节~\ref{饱和水汽压(比湿)及其随温度的变化},阻抗系数$r_{\mathrm{ah}}^\prime$、$r_{\mathrm{aw}}^\prime$对于温度的变化率无法解析给出,故暂时忽略。


\section{三维植被湍流交换模型}\label{三维植被湍流}
\begin{mymdframed}{代码}
  本节对应的代码文件为\texttt{MOD\_LeafTemperaturePC.F90}。
\end{mymdframed}

三维植被湍流交换模型是考虑植被树冠水平空间分布以及不同植被类型垂直分层(同三维植被辐射传输模型植被结构假设),以一维植被湍流交换过程为基础,同样采用Monin-Obukhov相似性理论进行计算。为了与一维植被湍流计算相统一,与CoLM2014相比,本版本模型在零平面位移($d$)、粗糙度($z_0$)、叶片空气动力学阻抗($r_{\mathrm {b}}$)以及对地的空气动力学阻抗($r_{\mathrm {d}}$)等的计算上有了新发展。[注: 这里“三维”主要是对植被结构进行描述,同三维植被辐射传输过程对植被结构描述,并非指植被湍流计算过程为三维求解]


\subsection{100\%覆盖植被情景}\label{百分百植被覆盖湍流}
CoLM2014假设$d$和$z_0$与植被高度成固定比例,即$d=\frac{2}{3}h$,$z_0=\frac{1}{10}h$,其中$h$表示树高。
这对于稀疏植被可能并不成立~\citep{zeng2007consistent}。在本版本,$d$的计算采用~\citet{choudhury1988}的方案,
是通过拟合~\citet{shaw1982aerodynamic}二阶闭合理论结果得到:
\begin{equation}\label{dOh}
  \frac{d}{h}=1.1 \ln \left (1+\left[c_{\mathrm{d}} \cdot (\text {LAI} + \text {SAI})\right]^{0.25} \right)
\end{equation}
其中$c_{\mathrm {d}}=0.2$,表示单位面积叶片平均拖曳系数。粗糙度计算采用~\citet{raupach1992drag,raupach1994simplified}方案:
\begin{equation}\label{zOh}
  \frac{z_{0}}{h}=\left(1-\frac{d}{h}\right) \exp \left(-\kappa \frac{u_{\mathrm{h}}}{u_{*}}+\Psi_{\mathrm{h}}\right)
\end{equation}
其中%
%$\kappa=0.4$为 von K\'arman 常数。
$\Psi_{\mathrm h}$为植被冠层对风速廓线影响函数,设置为0.193。$u_\ast$是摩擦速度,$u_{\mathrm h}$是冠层顶的风速,二者比值计算如下:
\begin{equation}\label{ustrarOuh}
  \frac{u_{*}}{u_{\mathrm{h}}}=\min \left[\left(C_{\mathrm{S}}+C_{\mathrm{R}} \lambda\right)^{0.5}, 0.3\right]
\end{equation}
其中$C_{\mathrm {S}}=0.003$,为无植被情况时$h$高度的拖曳系数;$C_{\mathrm {R}}$为植被覆盖区域的拖曳系数。$\lambda$表示植被迎风面积指数 (FAI):$\lambda=1-\exp{\left[-0.5 (\text {LAI}+\text {SAI})\right]}$。


CoLM2014 假设$r_{\mathrm {b}}$和$r_{\mathrm {d}}$分别与$u_\ast^{0.5}$和$u_\ast$成正比例关系,
但这不一定适合稀疏植被,本版本中$r_{\mathrm {b}}$和$r_{\mathrm {d}}$是通过对冠层内风速($u$)和湍流交换系数($K$)垂直廓线积分计算得到 \citep{dai2019different}。
$u$和$K$的廓线是基于指数衰减假设~\citep{inoue1963turbulent,cowan1968mass}。
另外对临近地面廓线进行了一定的订正,保证$u$和$K$满足到达地面时趋于0 \citep{dai2019different}。$u$和$K$的廓线函数分别表示为:
\begin{equation}\label{uz}
  u(z)=\min \left[u_{\mathrm{exp}}(z), u_{\mathrm{comb}}(z)\right]
\end{equation}
\begin{equation}\label{Kz}
  K(z)=\min \left[K_{\mathrm{exp}}(z), K_{\mathrm{comb}}(z)\right]
\end{equation}
其中$u_{\mathrm{exp}}$ 和 $K_{\mathrm{exp}}$为指数衰减函数:
\begin{equation}
  u_{\mathrm{exp}}(z)=u(h) \exp \left[-a\left(1-\frac{z}{h}\right)\right]
\end{equation}
\begin{equation}
  K_{\mathrm{exp}}(z)=K(h) \exp \left[-a\left(1-\frac{z}{h}\right)\right]
\end{equation}
上式中的$a$表示衰减系数,计算为~\citep{inoue1963turbulent,cowan1968mass,kondo1971relationship}:
\begin{equation}
  a=\frac{1}{h-d} \frac{h}{\ln \left[(h-d) / z_{0}\right]}
\end{equation}
$u_{\mathrm{comb}}$计算为对数廓线和裸土情况下相似性理论计算的风速廓线组合:
\begin{equation}\label{ucomb}
  u_{\mathrm{comb}}(z)=\zeta u_{\mathrm{\log }}(z)+(1-\zeta) u_{\mathrm{{bare }}}(z)
\end{equation}
其中$\zeta$是一个 logistic 函数$\zeta=\frac{1}{1+\exp{\left(-\frac{d/h-0.4}{0.08}\right)}}$。
$K_{\mathrm{comb}}$计算为线性衰减函数和裸土情况下相似性理论计算得到的交换系数廓线的线性组合:
\begin{equation}\label{kcomb}
  K_{\mathrm{comb}}(z)=\zeta K_{\mathrm{\mathrm{lin}}}(z)+(1-\zeta) K_{\mathrm{bare}}(z)
\end{equation}
$\zeta$的计算同上。方程 (\ref{uz}) 和 (\ref{Kz}) 中的 min 函数表示取其小。
这样无论是植被浓密或者稀疏时,既能考虑到植被中上层部分的风速和湍流交换系数的指数衰减,
又能考虑到近地面裸土对风速和湍流减缓系数的影响,避免了到达土壤表面时,$u$和$K$非0的问题。


$r_{\mathrm {b}}$是对植被冠层内风速廓线的积分:
\begin{equation}
  r_{\mathrm{b}}^{-1}=c_{\mathrm{l}} \frac{\int_{0}^{h}[u(z)]^{0.5} {\mathrm d} z}{d_{\mathrm{leaf}}^{0.5} h}
\end{equation}
上式表示单位叶面积的阻抗,其中$c_{\mathrm {l}}=0.01$ \unit{m.s^{-0.5}},$d_{\mathrm{leaf}}$为叶片的平均尺寸,设置为0.04米。$r_{\mathrm {d}}$计算为:
\begin{equation}\label{r_d1}
  r_{\mathrm{d}}=\int_{z_{0 \mathrm{hg}}}^{d+z_{0}} \frac{1}{K(z)} {\mathrm d} z
\end{equation}
其中$z_{\mathrm{0hg}}$表示用于潜热/感热交换计算的粗糙度\citep{zeng1998effect}。

\subsection{考虑稳定度影响的植被粗糙度方案}\label{考虑稳定度粗糙度方案}
\begin{mymdframed}{代码}
  本节对应的代码文件为\texttt{MOD\_RoughnessLength.F90}。
\end{mymdframed}

\citet{raupach1992drag,raupach1994simplified}的植被粗糙度计算方案假设中性条件,也就是忽略了稳定度的影响,
而诸多研究表明稳定度对粗糙度具有不可忽略的影响。以~\citet{raupach1992drag,raupach1994simplified}方案为基础,
我们发展了一个考虑稳定度影响的植被粗糙度计算方案,详述如下。

考虑具有以下一般形式的湍流扩散系数:
\begin{equation}\label{eddydiffusivity}
  K=\frac{\kappa u_{*} (z-d)} {\varphi \phi }
\end{equation}
其中$\kappa=0.4$为von K\'arman常数,$u_{*}$为摩擦速度。$\varphi$为冠层粗糙子层(roughness sublayer)影响函数,定义为:
\begin{equation}
  \varphi = \begin{cases}
    \frac{z-d} {z_{\mathrm{w}}-d}, & h<z<z_{\mathrm{w}} \text { 时} \\
    1, & z \geqslant z_{\mathrm{w}} \text { 时} \\
  \end{cases}
\end{equation}
其中$h$为树高,$z_{\mathrm{w}}$为粗糙子层顶(见图~\ref{fig:修正Raupach粗糙度方案示意图})。$\phi$为稳定度函数,采用~\citet{dyer1974review}关系式:
\begin{equation}
  \phi(\zeta) = \begin{cases}
    (1-16\zeta)^{-1/4}, & \zeta<0 \text { 时} \\
    1+5\zeta, & \zeta \geqslant 0 \text { 时} \\
  \end{cases}
\end{equation}
其中$\zeta = (z-d)/L$为无量纲稳定度参数,$L$为Monin-Obukhov长度(定义见章节~\ref{基本理论})。

根据一阶湍流闭合(即$K$理论),粗糙子层以上($z>h$)及以下($z_0 + d<z \leqslant h$)的风速梯度可表达为:
\begin{equation}\label{kz_u_rsl}
  \frac{\kappa (z-d) u_{*}}{\varphi \phi} \frac{\partial U}{\partial z}=u_{*}^2
\end{equation}
方程(\ref{kz_u_rsl})从$z=z_0+d$积分到$z=h$,有:
\begin{equation}\label{u_rsl_htop}
  \frac{\kappa U_{\mathrm{h}}}{u_{*}} = \ln \left(\frac{h-d}{z_{0}}\right) + \Psi_{\mathrm{h}}
\end{equation}
易得:
\begin{equation}\label{z0_rsl}
  z_{0} = (h-d)\exp (-\kappa \gamma + \Psi_{\mathrm{h}})
\end{equation}
其中$\Psi_{\mathrm{h}}=\int_{z_{0}+d}^{h} \frac{\varphi \phi - 1}{z-d} {\mathrm d} z$,$\gamma \equiv U_{\mathrm{h}}/u_{*}$(由公式(\ref{ustrarOuh})计算)。

方程(\ref{kz_u_rsl})从$z=z_0+d$积分到粗糙子层以上的惯性层(即$z\geqslant z_{\mathrm{w}}$),有:
\begin{align}\label{u_rsl_iner}
  \frac{\kappa U_{\mathrm{h}}}{u_{*}} =& \ln \left(\frac{z-d}{z_{0}}\right) + \Psi_{\mathrm{h}} + \int_{\mathrm{h}}^{z_{\mathrm{w}}} \frac{\varphi \phi - 1}{z-d} \,{\mathrm d}z
  + \int_{\mathrm{z_{w}}}^{z} \frac{\varphi \phi - 1}{z-d} \,{\mathrm d}z \nonumber \\[1ex]
  =& \ln \left(\frac{z-d}{z_{0}}\right) + \Psi_{\mathrm{h}} + \left[\int_{\mathrm{h}}^{z_{\mathrm{w}}} \frac{\phi}{z_{\mathrm{w}}-d} \,{\mathrm d}z - \ln \left(\frac{z_{\mathrm{w}}-d}{h-d}\right) \right]
  + \int_{\mathrm{z_{w}}}^{z} \frac{\phi - 1}{z-d} \,{\mathrm d}z
\end{align}
根据\citet{Garratt1992TheAB}:
\begin{equation}
  \int_{\mathrm{z_{w}}}^{z} \frac{\phi - 1}{z-d} \,{\mathrm d}z = -\psi_{\mathrm {u}} \left(\frac{z-d}{L} \right) + \psi_{\mathrm {u}} \left(\frac{z_{\mathrm{w}}-d}{L} \right)
\end{equation}
其中$\psi_{\mathrm {u}}(\chi )$在稳定和不稳定条件下具有不同的形式:
\begin{equation}
  \psi_{\mathrm {u}}(\chi ) = \begin{cases}
    2\ln \left(\frac{1+x}{2} \right) + \ln \left(\frac{1+x^2}{2} \right) - 2\tan^{-1}x + \frac {\pi}{2}, & \chi<0 \text { 时} \\[1ex]
    -5\chi, & \chi \geqslant 0 \text { 时} \\
  \end{cases}
\end{equation}
其中$x=(1-16\chi)^{\frac{1}{4}}$。

为确保惯性子层风速满足稳定度调整下的对数律,即:
\begin{equation}\label{u_iner}
  \frac{\kappa U_{\mathrm{h}}}{u_{*}} = \ln \left(\frac{z-d}{z_{0}}\right) + \int_{z_{0}+d}^{z} \frac{\phi - 1}{z-d} \,{\mathrm d}z
\end{equation}
方程(\ref{u_rsl_iner})右边最后三项之和(当$z=z_{\mathrm{w}}$)必须满足:
\begin{equation}
  \Psi_{\mathrm{h}} + \left[\int_{\mathrm{h}}^{z_{\mathrm{w}}} \frac{\phi}{z_{\mathrm{w}}-d} \,{\mathrm d}z - \ln \left(\frac{z_{\mathrm{w}}-d}{h-d}\right) \right] = \int_{\mathrm{z_{0}}+d}^{z_{\mathrm{w}}} \frac{\phi - 1}{z-d} \,{\mathrm d}z
\end{equation}
即:
\begin{align}\label{Psih_general}
  \Psi_{\mathrm{h}} =& \ln \left(\frac{z_{\mathrm{w}}-d}{h-d}\right) - \int_{\mathrm{h}}^{z_{\mathrm{w}}} \frac{\phi}{z_{\mathrm{w}}-d} \,{\mathrm d}z
  + \int_{\mathrm{z_{0}}+d}^{z_{\mathrm{w}}} \frac{\phi - 1}{z-d} \,{\mathrm d}z \nonumber \\[1.5ex]
  =& \ln \left(c_{\mathrm{w}}\right) -\frac{1}{z_{\mathrm{w}}-d} \int_{\mathrm{h}}^{z_{\mathrm{w}}} \phi \,{\mathrm d}z - \psi_{\mathrm {u}} \left(\frac{z_{\mathrm{w}}-d}{L} \right)
  + \psi_{\mathrm {u}} \left(\frac{z_{0}}{L} \right)
\end{align}
其中$c_{\mathrm{w}}=(z_{\mathrm{w}}-d)/(h-d)$为大于1的常数。

由方程(\ref{Psih_general})易知$\Psi_{\mathrm{h}}$在不稳定条件下($z_{\mathrm{w}}/L < 0$)的具体表达式:
\begin{equation}\label{Psih_unstable}
  \Psi_{\mathrm{h}} = \ln \left(c_{\mathrm{w}}\right) + \frac{L}{12\left(z_{\mathrm{w}}-d\right)} \left[\left(1-16\frac{z_{\mathrm{w}}}{L}\right)^{3/4}
  - \left(1-16\frac{h}{L}\right)^{3/4} \right] - \psi_{\mathrm {u}} \left(\frac{z_{\mathrm{w}}-d}{L} \right) + \psi_{\mathrm {u}} \left(\frac{z_{\mathrm{0}}}{L} \right)
\end{equation}
在稳定条件下($z_{\mathrm{w}}/L \geqslant 0$):
\begin{equation}\label{Psih_stable}
  \Psi_{\mathrm{h}} = \ln \left(c_{\mathrm{w}}\right) + \left(1-\frac{1}{c_{\mathrm{w}}}\right) \left[1+\frac{5\left(z_{\mathrm{w}}+h\right)}{2L}\right]
  + \frac{5\left(z_{\mathrm{w}}-d\right)}{L} - \frac{5z_{0}}{L}
\end{equation}

结合方程(\ref{z0_rsl})和(\ref{Psih_unstable})、(\ref{Psih_stable})即可用于计算考虑稳定度影响的植被粗糙度,
该计算方案示意图如图~\ref{fig:修正Raupach粗糙度方案示意图} 所示。
{
  \begin{figure}[]
    \centering
    \includegraphics[scale=0.7]{Figures/地表湍流交换过程/修正Raupach粗糙度方案示意图.jpg}
    \caption{考虑稳定度影响的植被粗糙度计算方案意图}
    \label{fig:修正Raupach粗糙度方案示意图}
  \end{figure}
}


\subsection{单层植被湍流交换}\label{单层植被湍流交换}
单层植被不同于100\%植被覆盖假设,而是考虑植被树冠之间可能存在空隙,
即植被树冠存在一定的水平分布,树冠覆盖度可介于0$\sim$100\%之间。这种植被结果假设与三维植被辐射模型假设一致。


对于植被树冠稀疏覆盖时,\citet{raupach1992drag,raupach1994simplified}通过场外和风洞实验,发展了一套用于计算$z_0$和$d$的解析解,其中:
\begin{equation}\label{dooh0}
  \frac{d}{h}=1-\frac{1-\exp \left(-\sqrt{c_{\mathrm{d1}} 2 \lambda}\right)}{\sqrt{c_{\mathrm{d1}} 2 \lambda}}
\end{equation}
其中$c_{\mathrm{d1}}=7.5$,$\lambda$%表示植被迎风面积指数 ($FAI$),
计算为$f_{\mathrm {c}}\left[1-\exp{\left(-0.5\text{(LAI+SAI)}\right)}\right]$。
当植被覆盖$f_{\mathrm {c}}$等于100\%时,$\lambda$计算同100\%植被覆盖情景。为了同时适用于稀疏和浓密覆盖植被,
本版本采用~\citet{dai2019different}方案计算$d$:
\begin{equation}\label{dooh}
  \frac{d}{h}=f_{\mathrm{c}} \cdot 1.1 \ln \left(1+\left(c_{\mathrm{d}} f_{\mathrm{c}} \text{LAI}\right)^{0.25}\right)+\left(1-f_{\mathrm{c}}\right) \cdot\left(1-\frac{1-\exp \left(-\sqrt{c_{\mathrm{d1}} 2 \lambda}\right)}{\sqrt{c_{\mathrm{d1}} 2 \lambda}}\right)
\end{equation}
$z_0$的计算同公式 (\ref{zOh})。从上式可以看出,$d$值是100\%植被覆盖情景和稀疏植被按照植被覆盖度$f_{\mathrm {c}}$的加权平均,
当$f_{\mathrm {c}}\rightarrow100\%$时,公式 (\ref{dooh}) 趋同于公式 (\ref{dOh});当$f_{\mathrm {c}}\rightarrow0\%$时,趋同于公式 (\ref{dooh0})。


对于植被冠层内的风速廓线$u$和湍流交换系数$K$,同样采用$f_{\mathrm {c}}$加权方式。$u$和$K$分别计算为:
\begin{equation}
  u(z)=f_{\mathrm{c}} \cdot \min \big(u_{\mathrm{\exp }}(z), u_{\mathrm{comb}}(z)\big)+\left(1-f_{\mathrm{c}}\right) \cdot u_{\mathrm{comb}}
\end{equation}
\begin{equation}
  \frac{1}{K(z)}=f_{\mathrm{c}} \cdot \frac{1}{\min \big(K_{\mathrm{\exp}}(z), K_{\mathrm{comb}}(z)\big)}+\left(1-f_{\mathrm{c}}\right) \cdot \frac{1}{K_{\mathrm{comb}}}
\end{equation}
$u_{\mathrm{comb}}$和$K_{\mathrm{comb}}$同公式 (\ref{ucomb}) 和 (\ref{kcomb})。


当一层植被有多个PFT类型覆盖时,对于感热通量有平衡方程:
\begin{equation}
  H = H_{\mathrm g} + \sum_{i \in \mbox{\tiny {第1层}}}^{}H_{\mathrm{v},i}
\end{equation}
其中:
\begin{equation}
  H = \frac{\rho_{\mathrm{a}}C_{\mathrm{a}}\left( T_{1} - T_{\mathrm{a}} \right)}{r_{\mathrm{ah}}}
\end{equation}
%
\begin{equation}
  H_{\mathrm{g}} = \frac{\rho_{\mathrm{a}}C_{\mathrm{a}}\left( T_{\mathrm{g}} - T_{1} \right)}{r_{\mathrm{d1}}}
\end{equation}
%
\begin{equation}\label{eq:Hv_1L}
  H_{\mathrm{v},i} = \frac{\rho_{\mathrm{a}}C_{\mathrm{a}}f_{\mathrm{c},i}\left( T_{\mathrm{v},i} - T_{1} \right)}{r_{\mathrm{vh},i}}
\end{equation}
即:
\begin{equation}\label{eq:H_1L}
  \frac{\rho_{\mathrm{a}}C_{\mathrm{a}}\left( T_{1} - T_{\mathrm{a}} \right)}{r_{\mathrm{ah}}} = \frac{\rho_{\mathrm{a}}C_{\mathrm{a}}\left( T_{\mathrm{g}} - T_{1} \right)}{r_{\mathrm{d1}}} + \sum_{i \in \mbox{\tiny {第1层}}}^{}\frac{\rho_{\mathrm{a}}C_{\mathrm{a}}f_{\mathrm{c},i}\left( T_{\mathrm{v},i} - T_{1} \right)}{r_{\mathrm{vh},i}}
\end{equation}
%
上式中\(r_{\mathrm{ah}},r_{\mathrm{d1}},r_{\mathrm{vh},i}\)第一层等效交换高度到大气参考高度、地面和植被之间的感热交换阻抗,$r_{\mathrm{vh},i}$下标$i$表示不同PFT。令:
\begin{equation}
  c_{\mathrm{ah1}} = \frac{1}{r_{\mathrm{ah}}},\ c_{\mathrm{gh1}} = \frac{1}{r_{\mathrm{d1}}},\ c_{\mathrm{vh},i} = \frac{1}{r_{\mathrm{vh},i}}
\end{equation}
%
\begin{equation}
  w_{\mathrm{h1}} = c_{\mathrm{ah1}} + c_{\mathrm{gh1}} + \sum_{i \in \mbox{\tiny {第1层}}}^{}{f_{\mathrm{c},i}c_{\mathrm{vh},i}}
\end{equation}
%
\begin{equation}
  w_{\mathrm{ah1}} = \frac{c_{\mathrm{ah1}}}{w_{\mathrm{h1}}},\ w_{\mathrm{gh1}} = \frac{c_{\mathrm{gh1}}}{w_{\mathrm{h1}}},\ w_{\mathrm{vh},i} = \frac{f_{\mathrm{c},i}c_{\mathrm{vh},i}}{w_{\mathrm{h1}}}
\end{equation}
%
方程~\eqref{eq:H_1L} 进行变量替换,化简后可得:
\begin{equation}
  T_{1} = w_{\mathrm{ah1}}T_{\mathrm{a}} + w_{\mathrm{gh1}}T_{\mathrm{g}} + \sum_{i \in \mbox{\tiny {第1层}}}^{}{w_{\mathrm{vh},i}T_{\mathrm{v},i}}
\end{equation}
%
将上式\(T_{1}\)表达式带入到公式~\eqref{eq:Hv_1L} 中,即可计算叶片感热通量。叶片感热相对叶温变化的导数计算为:
\begin{equation}
  \frac{\partial H_{\mathrm{v},i}}{\partial T_{\mathrm{v},i}} = \rho_{\mathrm{a}}C_{\mathrm{a}}c_{\mathrm{vh},i}\left( 1 - w_{\mathrm{vh},i} \right)
\end{equation}
%
地面感热相对温度变化的导数为:
\begin{equation}
  \frac{\partial H_{\mathrm{g}}}{\partial T_{\mathrm{g}}} = \rho_{\mathrm{a}}C_{\mathrm{a}}c_{\mathrm{gh1}}\left( 1 - w_{\mathrm{gh1}} \right)
\end{equation}

对于潜热通量:
\begin{equation}
  \lambda E = {\lambda E}_{\mathrm{g}} + \sum_{i \in \mbox{\tiny {第1层}}}^{}{\lambda E}_{\mathrm{v},i}
\end{equation}
%
其中:
\begin{equation}
  E = \frac{\rho_{\mathrm{a}}\left( q_{1} - q_{\mathrm{a}} \right)}{r_{\mathrm{aw}}}
\end{equation}
%
\begin{equation}
  E_{\mathrm{g}} = \frac{\rho_{\mathrm{a}}\left( q_{\mathrm{g}} - q_{1} \right)}{r_{\mathrm{d1}}}
\end{equation}
%
\begin{equation}
  E_{\mathrm{v},i} = \frac{\rho_{\mathrm{a}}f_{\mathrm{c},i}\left( q_{\mathrm{v},i} - q_{1} \right)}{r_{\mathrm{vw},i}}
\end{equation}
%
即:
\begin{equation}\label{eq:LE_1L}
  \frac{\rho_{\mathrm{a}}\left( q_{1} - q_{\mathrm{a}} \right)}{r_{\mathrm{aw}}} = \frac{\rho_{\mathrm{a}}\left( q_{\mathrm{g}} - q_{1} \right)}{r_{\mathrm{d1}}} + \sum_{i \in \mbox{\tiny {第1层}}}^{}\frac{\rho_{\mathrm{a}}f_{\mathrm{c},i}\left( q_{\mathrm{v},i} - q_{1} \right)}{r_{\mathrm{vw},i}}
\end{equation}
%
上式中\(r_{\mathrm{aw}},r_{\mathrm{d1}},r_{\mathrm{vw},i}\)表示等效交换高度到大气参考高度、地面和植被之间的潜热交换阻抗。$q_{\mathrm{v},i}$表示叶片表面水汽比湿,为叶片温度$T_{\mathrm{v},i}$时的饱和比湿$q_{\mathrm{sat}}^{T_{\mathrm{v},i}}$。令:
\begin{equation}
  c_{\mathrm{aw1}} = \frac{1}{r_{\mathrm{aw}}},\ c_{\mathrm{gw1}} = \frac{1}{r_{\mathrm{d1}}},\ c_{\mathrm{vh},i} = \frac{1}{r_{\mathrm{vw},i}}
\end{equation}
%
\begin{equation}
  w_{\mathrm{q1}} = c_{\mathrm{aw1}} + c_{\mathrm{gw1}} + \sum_{i \in \mbox{\tiny {第1层}}}^{}{f_{\mathrm{c},i}c_{\mathrm{vw},i}}
\end{equation}
%
\begin{equation}
  w_{\mathrm{aq1}} = \frac{c_{\mathrm{aw1}}}{w_{\mathrm{q1}}},\ w_{\mathrm{gq1}} = \frac{c_{\mathrm{gw1}}}{w_{\mathrm{q1}}},\ w_{\mathrm{vq},i} = \frac{{f_{\mathrm{c},i}c}_{\mathrm{vw},i}}{w_{\mathrm{q}}}
\end{equation}
%
求解方程~\eqref{eq:LE_1L} 可得:
\begin{equation}
  q_{1} = w_{\mathrm{aw1}}q_{\mathrm{a}} + w_{\mathrm{gw1}}q_{\mathrm{g}} + \sum_{i \in \mbox{\tiny {第1层}}}^{}{w_{\mathrm{vq},i}q_{\mathrm{v},i}}
\end{equation}
%
叶片蒸散发相对叶温变化的导数计算为:
\begin{equation}
  \frac{\partial E_{\mathrm{v},i}}{\partial T_{\mathrm{v},i}} = \rho_{\mathrm{a}}c_{\mathrm{vw},i}\left( 1 - w_{\mathrm{vq},i} \right)\frac{{\rm d}q_{\mathrm{sat}}^{T_{\mathrm{v},i}}}{{\rm d}T_{\mathrm{v},i}}
\end{equation}
%
其中叶片蒸腾水汽通量相对叶片温度变化的导数计算为:
\begin{equation}
  \frac{\partial E_{\mathrm{tr},i}}{\partial T_{\mathrm{v},i}} = \rho_{\mathrm{a}}\left( 1 - f_{\mathrm{wet}} \right)\delta\left( \frac{\text{LAI}_{\mathrm{sun}}}{r_{\mathrm{b},i} + r_{\mathrm{s},\mathrm{sun},i}} + \frac{\text{LAI}_{\mathrm{sha}}}{r_{\mathrm{b},i} + r_{\mathrm{s,sha},i}} \right)\left( 1 - w_{\mathrm{vq},i} \right)\frac{{\rm d}q_{\mathrm{sat}}^{T_{\mathrm{v},i}}}{{\rm d}T_{\mathrm{v},i}}
\end{equation}
%
叶片蒸发水汽通量相对叶片温度变化导数计算为:\todo{请补充$\delta$的含义}
\begin{equation}
  \frac{\partial E_{\mathrm{va},i}}{\partial T_{\mathrm{v},i}} = \rho_{\mathrm{a}}\left( 1 - \delta\left( 1 - f_{\mathrm{wet}} \right) \right)\frac{\text{LAI} + \text{SAI}}{r_{\mathrm{b},i}}\left( 1 - w_{\mathrm{vq},i} \right)\frac{{\rm d}q_{\mathrm{sat}}^{T_{\mathrm{v},i}}}{{\rm d}T_{\mathrm{v},i}}
\end{equation}
%
地面水汽通量相对温度变化的导数为:
\begin{equation}
  \frac{\partial E_{\mathrm{g}}}{\partial T_{\mathrm{g}}} = \rho_{\mathrm{a}}c_{\mathrm{gw1}}\left( 1 - w_{\mathrm{gq1}} \right)\frac{{\rm d}q_{\mathrm{g}}}{{\rm d}T_{\mathrm{g}}}
\end{equation}


\subsection{多层植被湍流交换}\label{多层植被湍流交换}
{
  \begin{figure}[htbp]
    \centering
    \includegraphics[width=1\linewidth]{Figures/地表湍流交换过程/三层植被湍流交换示意图_v3.jpg}
    \caption{三层植被湍流交换示意图}
    \label{fig:三层植被湍流交换示意图}
  \end{figure}
}

多层植被湍流计算是以单层植被计算为基础,总体阻抗网络如图~\ref{fig:三层植被湍流交换示意图} 所示。
风速$u$和湍流交换系数$K$廓线从最上层植被往下进行计算。上一层底部的廓线值作为下层植被顶部的值。
$r_{\mathrm {b}}$和$r_{\mathrm {d}}$的计算都是对$u$和$K$廓线的积分。
但积分区间对应到每层植被的等效交换高度(图~\ref{fig:三层植被湍流交换示意图} 中$T_{1}$、$T_{2}$、$T_{3}$位置所示)。
该交换高度计算为该层植被100\%覆盖时的$d+z_0$值。在每一层等效交换高度,建立通量(感热$H$和潜热$\lambda E$)平衡方程,即该点与上层通量交换量等于下层交换量加上该层植被的通量交换量。联立每层在等效高度建立的方程,迭代求解每种植被叶片温度。

通过以上介绍可以看出,三维植被湍流交换与一维植被湍流交换最大的不同在于其计算的对象由原来单一植被扩展到多种植被(PFT),并根据高度进行分层计算。因此阻抗网络发生较大变化,使得多种植被可以在同一环境下共存,并同时求解。在计算时,同一维植被一样,需要用到植被短波辐射吸收量和长波辐射吸收量,此时利用章节~\ref{三维植被辐射传输模型} 和 \ref{三维植被长波辐射传输} 对三维植被辐射传输计算结果。对于光合作用,每种植被均采用双大叶模型模拟计算。整个植被冠层与大气的湍流交换同样采用相似性理论进行求解,其求解过程如图~\ref{fig:三维植被湍流交换模型计算流程图} 所示。
{
  \begin{figure}[htbp]
    \centering
    \includegraphics[width=0.75\linewidth]{Figures/地表湍流交换过程/三维植被湍流交换模型计算流程图_v2.png}
    \caption{三维植被湍流交换模型计算流程图}
    \label{fig:三维植被湍流交换模型计算流程图}
  \end{figure}
}


\subsubsection{两层植被}

当有两层植被覆盖时(较高层记为第2层,低层记为第1层),对于感热通量平衡方程为:
\begin{equation}
  H = H_{1} + \sum_{i \in \mbox{\tiny {第2层}}}^{}H_{\mathrm{v},i}
\end{equation}
%
\begin{equation}
  H_{1} = H_{\mathrm{g}} + \sum_{i \in \mbox{\tiny {第1层}}}^{}H_{\mathrm{v},i}
\end{equation}
%
即:
\begin{equation}\label{eq:H_2L_1}
  \frac{\rho_{\mathrm{a}}C_{\mathrm{a}}\left( T_{2} - T_{\mathrm{a}} \right)}{r_{\mathrm{ah}}} = \frac{\rho_{\mathrm{a}}C_{\mathrm{a}}\left( T_{1} - T_{2} \right)}{r_{\mathrm{d2}}} + \sum_{i \in \mbox{\tiny {第2层}}}^{}\frac{\rho_{\mathrm{a}}C_{\mathrm{a}}f_{\mathrm{c},i}\left( T_{\mathrm{v},i} - T_{2} \right)}{r_{\mathrm{vh},i}}
\end{equation}
%
\begin{equation}\label{eq:H_2L_2}
  \frac{\rho_{\mathrm{a}}C_{\mathrm{a}}\left( T_{1} - T_{2} \right)}{r_{\mathrm{d2}}} = \frac{\rho_{\mathrm{a}}C_{\mathrm{a}}\left( T_{\mathrm{g}} - T_{1} \right)}{r_{\mathrm{d1}}} + \sum_{i \in \mbox{\tiny {第1层}}}^{}\frac{\rho_{\mathrm{a}}C_{\mathrm{a}}f_{\mathrm{c},i}\left( T_{\mathrm{v},i} - T_{1} \right)}{r_{\mathrm{vh},i}}
\end{equation}
%
对于第2层方程,令:
\begin{equation}
  c_{\mathrm{ah2}} = \frac{1}{r_{\mathrm{ah}}},\ c_{\mathrm{gh2}} = \frac{1}{r_{\mathrm{d2}}},\ c_{\mathrm{vh},i} = \frac{1}{r_{\mathrm{vh},i}}
\end{equation}
%
\begin{equation}
  w_{\mathrm{h2}} = c_{\mathrm{ah2}} + c_{\mathrm{gh2}} + \sum_{i \in \mbox{\tiny {第2层}}}^{}{f_{\mathrm{c},i}c}_{\mathrm{vh},i}
\end{equation}
%
\begin{equation}
  w_{\mathrm{ah2}} = \frac{c_{\mathrm{ah2}}}{w_{\mathrm{h2}}},\ w_{\mathrm{gh2}} = \frac{c_{\mathrm{gh2}}}{w_{\mathrm{h2}}},\ w_{\mathrm{vh},i} = \frac{f_{\mathrm{c},i}c_{\mathrm{vh},i}}{w_{\mathrm{h2}}}
\end{equation}
%
对于第1层方程,令:
\begin{equation}
  c_{\mathrm{ah1}} = \frac{1}{r_{\mathrm{d2}}},\ c_{\mathrm{gh1}} = \frac{1}{r_{\mathrm{d1}}},\ c_{\mathrm{vh},i} = \frac{1}{r_{\mathrm{vh},i}}
\end{equation}
%
\begin{equation}
  w_{\mathrm{h1}} = c_{\mathrm{ah1}} + c_{\mathrm{gh1}} + \sum_{i \in \mbox{\tiny {第1层}}}^{}{f_{\mathrm{c},i}c}_{\mathrm{vh},i}
\end{equation}
%
\begin{equation}
  w_{\mathrm{ah1}} = \frac{c_{\mathrm{ah1}}}{w_{\mathrm{h1}}},\ w_{\mathrm{gh1}} = \frac{c_{\mathrm{gh1}}}{w_{\mathrm{h1}}},\ w_{\mathrm{vh},i} = \frac{{f_{\mathrm{c},i}c}_{\mathrm{vh},i}}{w_{\mathrm{h1}}}
\end{equation}
%
将方程~\eqref{eq:H_2L_2}得到的\(T_{1}\)表达式带入到方程~\eqref{eq:H_2L_1},即可求解\(T_{2}\)为:
\begin{equation}
  T_{2} = \frac{w_{\mathrm{ah2}}T_{\mathrm{a}} + w_{\mathrm{gh2}}T_{\mathrm{gv}} + \sum_{i \in \mbox{\tiny {第2层}}}^{}{w_{\mathrm{vh},i}T_{\mathrm{v},i}}}{f_{\mathrm{h}}}
\end{equation}
%
其中:
\begin{equation}
  T_{\mathrm{gv}} = w_{\mathrm{gh1}}T_{\mathrm{g}} + \sum_{i \in \mbox{\tiny {第1层}}}^{}{w_{\mathrm{vh},i}T_{\mathrm{v},i}}
\end{equation}
%
\begin{equation}
  f_{\mathrm{h}} = 1 - w_{\mathrm{gh2}}w_{\mathrm{ah1}}
\end{equation}
%
将计算得到的\(T_{2}\)带入到方程~\eqref{eq:H_2L_2},即可计算出\(T_{1}\)为:
\begin{equation}
  T_{1} = w_{\mathrm{ah1}}T_{2} + w_{\mathrm{gh1}}T_{\mathrm{g}} + \sum_{i \in \mbox{\tiny {第1层}}}^{}{w_{\mathrm{vh},i}T_{\mathrm{v},i}}
\end{equation}
%
对于第2层植被感热相对温度的导数计算为:
\begin{equation}
  \frac{\partial H_{\mathrm{v},i}}{\partial T_{\mathrm{v},i}} = \rho_{\mathrm{a}}C_{\mathrm{a}}c_{\mathrm{vh},i}\left( 1 - \frac{w_{\mathrm{vh},i}}{f_{\mathrm{h}}} \right)
\end{equation}
%
对于第1层植被感热相对温度的导数计算为:
\begin{equation}
  \frac{\partial H_{\mathrm{v},i}}{\partial T_{\mathrm{v},i}} = \rho_{\mathrm{a}}C_{\mathrm{a}}c_{\mathrm{vh},i}\left( 1 - \frac{w_{\mathrm{ah1}}w_{\mathrm{gh2}}w_{\mathrm{vh},i}}{f_{\mathrm{h}}} - w_{\mathrm{vh},i} \right) = \rho_{\mathrm{a}}C_{\mathrm{a}}c_{\mathrm{vh},i}\left( 1 - \frac{w_{\mathrm{vh},i}}{f_{\mathrm{h}}} \right)
\end{equation}
%
地面感热相对温度变化的导数计算为:
\begin{equation}
  \frac{\partial H_{\mathrm{g}}}{\partial T_{\mathrm{g}}} = \rho_{\mathrm{a}}C_{\mathrm{a}}c_{\mathrm{gh1}}\left( 1 - \frac{w_{\mathrm{ah1}}w_{\mathrm{gh2}}w_{\mathrm{gh1}}}{f_{\mathrm h}} - w_{\mathrm{gh1}} \right) = \rho_{\mathrm{a}}C_{\mathrm{a}}c_{\mathrm{gh1}}\left( 1 - \frac{w_{\mathrm{gh1}}}{f_{\mathrm h}} \right)
\end{equation}


对于潜热通量:
\begin{equation}
  \lambda E = {\lambda E}_{1} + \sum_{i \in \mbox{\tiny {第2层}}}^{}{\lambda E_{\mathrm{v},i}}
\end{equation}
%
\begin{equation}
  {\lambda E}_{1} = \lambda E_{\mathrm{g}} + \sum_{i \in \mbox{\tiny {第1层}}}^{}{\lambda E_{\mathrm{v},i}}
\end{equation}
%
即:
\begin{equation}\label{eq:LE_2L_1}
  \frac{\rho_{\mathrm{a}}\left( q_{2} - q_{\mathrm{a}} \right)}{r_{\mathrm{aw}}} = \frac{\rho_{\mathrm{a}}\left( q_{1} - q_{2} \right)}{r_{\mathrm{d2}}} + \sum_{i \in \mbox{\tiny {第2层}}}^{}\frac{\rho_{\mathrm{a}}f_{\mathrm{c},i}\left( q_{\mathrm{v},i} - q_{2} \right)}{r_{\mathrm{vw},i}}
\end{equation}
%
\begin{equation}\label{eq:LE_2L_2}
  \frac{\rho_{\mathrm{a}}\left( q_{1} - q_{2} \right)}{r_{\mathrm{d2}}} = \frac{\rho_{\mathrm{a}}\left( q_{\mathrm{g}} - q_{1} \right)}{r_{\mathrm{d1}}} + \sum_{i \in \mbox{\tiny {第1层}}}^{}\frac{\rho_{\mathrm{a}}f_{\mathrm{c},i}\left( q_{\mathrm{v},i} - q_{1} \right)}{r_{\mathrm{vw},i}}
\end{equation}
%
对于第2层方程,令:
\begin{equation}
  c_{\mathrm{aw2}} = \frac{1}{r_{\mathrm{aw}}},\ c_{\mathrm{gw2}} = \frac{1}{r_{\mathrm{d2}}},\ c_{\mathrm{vw},i} = \frac{1}{r_{\mathrm{vw},i}}
\end{equation}
%
\begin{equation}
  w_{q2} = c_{\mathrm{aw2}} + c_{\mathrm{gw2}} + \sum_{i \in \mbox{\tiny {第2层}}}^{}{f_{\mathrm{c},i}\ c_{\mathrm{vw},i}}
\end{equation}
%
\begin{equation}
  w_{\mathrm{aq2}} = \frac{c_{\mathrm{aw2}}}{w_{\mathrm{q2}}},\ w_{\mathrm{gh2}} = \frac{c_{\mathrm{gw2}}}{w_{\mathrm{q2}}},\ w_{\mathrm{vh},i} = \frac{{f_{\mathrm{c},i}c}_{\mathrm{vw},i}}{w_{\mathrm{q2}}}
\end{equation}
%
对于第1层方程,令:
\begin{equation}
  c_{\mathrm{aw1}} = \frac{1}{r_{\mathrm{d2}}},\ c_{\mathrm{gw1}} = \frac{1}{r_{\mathrm{d1}}},\ c_{\mathrm{vw},i} = \frac{1}{r_{\mathrm{vw},i}}
\end{equation}
%
\begin{equation}
  w_{\mathrm{q1}} = c_{\mathrm{aw1}} + c_{\mathrm{gw1}} + \sum_{i \in \mbox{\tiny {第1层}}}^{}{f_{\mathrm{c},i}c}_{\mathrm{vw},i}
\end{equation}
%
\begin{equation}
  w_{\mathrm{aq1}} = \frac{c_{\mathrm{aw1}}}{w_{\mathrm{q1}}},\ w_{\mathrm{gq1}} = \frac{c_{\mathrm{gw1}}}{w_{\mathrm{q1}}},\ w_{\mathrm{vh},i} = \frac{{f_{\mathrm{c},i}c}_{\mathrm{vw},i}}{w_{\mathrm{q1}}}
\end{equation}
%
将方程~\eqref{eq:LE_2L_2} 得到的\(q_{1}\)表达式带入到方程~\eqref{eq:LE_2L_1},即可求解\(q_{2}\)为:
\begin{equation}
  q_{2} = \frac{w_{\mathrm{aq2}}q_{\mathrm{a}} + w_{\mathrm{gq2}}q_{\mathrm{gv}} + \sum_{i \in \mbox{\tiny {第2层}}}^{}{w_{\mathrm{vq},i}q_{\mathrm{v},i}}}{f_{\mathrm{q}}}
\end{equation}
%
其中:
\begin{equation}
  q_{\mathrm{gv}} = w_{\mathrm{gq1}}q_{\mathrm{g}} + \sum_{i \in \mbox{\tiny {第1层}}}^{}{w_{\mathrm{vq},i}q_{\mathrm{v},i}}
\end{equation}
%
\begin{equation}
  f_{\mathrm{q}} = 1 - w_{\mathrm{gq2}}w_{\mathrm{aq1}}
\end{equation}
%
将计算得到的\(q_{2}\)带入到方程~\eqref{eq:LE_2L_2},即可计算出\(q_{1}\)为:
\begin{equation}
  q_{1} = w_{\mathrm{aq1}}q_{2} + w_{\mathrm{gq1}}q_{\mathrm{g}} + \sum_{i \in \mbox{\tiny {第1层}}}^{}{w_{\mathrm{vq},i}q_{\mathrm{v},i}}
\end{equation}
%
第2层植被叶片蒸散发相对叶温变化的导数计算为:
\begin{equation}
  \frac{\partial E_{\mathrm{v},i}}{\partial T_{\mathrm{v},i}} = \rho_{\mathrm{a}}c_{\mathrm{vw},i}\left( 1 - \frac{w_{\mathrm{vq},i}}{f_{\mathrm{q}}} \right)\frac{{\rm d}q_{\mathrm{sat}}^{T_{\mathrm{v},i}}}{{\rm d}T_{\mathrm{v},i}}
\end{equation}
%
其中叶片蒸腾水汽通量相对叶片温度变化的导数计算为:
\begin{equation}
  \frac{\partial E_{\mathrm{tr},i}}{\partial T_{\mathrm{v},i}} = \rho_{\mathrm{a}}\left( 1 - f_{\mathrm{wet}} \right)\delta\left( \frac{\text{LAI}_{\mathrm{sun}}}{r_{\mathrm{b},i} + r_{\mathrm{s},\mathrm{sun},i}} + \frac{\text{LAI}_{\mathrm{sha}}}{r_{\mathrm{b},i} + r_{\mathrm{s,sha},i}} \right)\left( 1 - \frac{w_{\mathrm{vq},i}}{f_{\mathrm{q}}} \right)\frac{{\rm d}q_{\mathrm{sat}}^{T_{\mathrm{v},i}}}{{\rm d}T_{\mathrm{v},i}}
\end{equation}
%
叶片蒸发水汽通量相对叶片温度变化导数计算为:
\begin{equation}
  \frac{\partial E_{\mathrm{va},i}}{\partial T_{\mathrm{v},i}} = \rho_{\mathrm{a}}\left( 1 - \delta\left( 1 - f_{\mathrm{wet}} \right) \right)\frac{\text{LAI} + \text{SAI}}{r_{\mathrm{b},i}}\left( 1 - \frac{w_{\mathrm{vq},i}}{f_{\mathrm{q}}} \right)\frac{{\rm d}q_{\mathrm{sat}}^{T_{\mathrm{v},i}}}{{\rm d}T_{\mathrm{v},i}}
\end{equation}
%
第1层植被叶片蒸散发相对叶温变化的导数计算为:
\begin{equation}
  \frac{\partial E_{\mathrm{v},i}}{\partial T_{\mathrm{v},i}} = \rho_{\mathrm{a}}c_{\mathrm{vw},i}\left( 1 - \frac{w_{\mathrm{vq},i}}{f_{\mathrm{q}}} \right)\frac{{\rm d}q_{\mathrm{sat}}^{T_{\mathrm{v},i}}}{{\rm d}T_{\mathrm{v},i}}
\end{equation}
%
其中叶片蒸腾水汽通量相对叶片温度变化的导数计算为:
\begin{equation}
  \frac{\partial E_{\mathrm{tr},i}}{\partial T_{\mathrm{v},i}} = \rho_{\mathrm{a}}\left( 1 - f_{\mathrm{wet}} \right)\delta\left( \frac{\text{LAI}_{\mathrm{sun}}}{r_{\mathrm{b},i} + r_{\mathrm{s},\mathrm{sun},i}} + \frac{\text{LAI}_{\mathrm{sha}}}{r_{\mathrm{b},i} + r_{\mathrm{s,sha},i}} \right)\left( 1 - \frac{w_{\mathrm{vq},i}}{f_{\mathrm{q}}} \right)\frac{{\rm d}q_{\mathrm{sat}}^{T_{\mathrm{v},i}}}{{\rm d}T_{\mathrm{v},i}}
\end{equation}
%
叶片蒸发水汽通量相对叶片温度变化导数计算为:
\begin{equation}
  \frac{\partial E_{\mathrm{va},i}}{\partial T_{\mathrm{v},i}} = \rho_{\mathrm{a}}\left( 1 - \delta\left( 1 - f_{\mathrm{wet}} \right) \right)\frac{\text{LAI} + \text{SAI}}{r_{\mathrm{b},i}}\left( 1 - \frac{w_{\mathrm{vq},i}}{f_{\mathrm{q}}} \right)\frac{{\rm d}q_{\mathrm{sat}}^{T_{\mathrm{v},i}}}{{\rm d}T_{\mathrm{v},i}}
\end{equation}
%
地面水汽通量相对温度变化的导数为:
\begin{equation}
  \frac{\partial E_{\mathrm{g}}}{\partial T_{\mathrm{g}}} = \rho_{\mathrm{a}}c_{\mathrm{gw1}}\left( 1 - \frac{w_{\mathrm{gq1}}}{f_{\mathrm {q}}} \right)\frac{{\rm d}q_{\mathrm{g}}}{{\rm d}T_{\mathrm{g}}}
\end{equation}

\subsubsection{三层植被}

当为三层植被覆盖时,对于感热通量:
\begin{equation}
  H = H_{2} + \sum_{i \in \mbox{\tiny {第3层}}}^{}H_{\mathrm{v},i}
\end{equation}
%
\begin{equation}
  H_{2} = H_{1} + \sum_{i \in \mbox{\tiny {第2层}}}^{}H_{\mathrm{v},i}
\end{equation}
%
\begin{equation}
  H_{1} = H_{\mathrm{g}} + \sum_{i \in \mbox{\tiny {第1层}}}^{}H_{\mathrm{v},i}
\end{equation}
%
即:
\begin{equation}\label{eq:H_3L_1}
  \frac{\rho_{\mathrm{a}}C_{\mathrm{a}}\left( T_{3} - T_{\mathrm{a}} \right)}{r_{\mathrm{ah}}} = \frac{\rho_{\mathrm{a}}C_{\mathrm{a}}\left( T_{2} - T_{3} \right)}{r_{\mathrm{d3}}} + \sum_{i \in \mbox{\tiny {第3层}}}^{}\frac{\rho_{\mathrm{a}}C_{\mathrm{a}}f_{\mathrm{c},i}\left( T_{\mathrm{v},i} - T_{3} \right)}{r_{\mathrm{vh},i}}
\end{equation}
%
\begin{equation}\label{eq:H_3L_2}
  \frac{\rho_{\mathrm{a}}C_{\mathrm{a}}\left( T_{2} - T_{3} \right)}{r_{\mathrm{d3}}} = \frac{\rho_{\mathrm{a}}C_{\mathrm{a}}\left( T_{1} - T_{2} \right)}{r_{\mathrm{d2}}} + \sum_{i \in \mbox{\tiny {第2层}}}^{}\frac{\rho_{\mathrm{a}}C_{\mathrm{a}}f_{\mathrm{c},i}\left( T_{\mathrm{v},i} - T_{2} \right)}{r_{\mathrm{vh},i}}
\end{equation}
%
\begin{equation}\label{eq:H_3L_3}
  \frac{\rho_{\mathrm{a}}C_{\mathrm{a}}\left( T_{1} - T_{2} \right)}{r_{\mathrm{d2}}} = \frac{\rho_{\mathrm{a}}C_{\mathrm{a}}\left( T_{\mathrm{g}} - T_{1} \right)}{r_{\mathrm{d1}}} + \sum_{i \in \mbox{\tiny {第1层}}}^{}\frac{\rho_{\mathrm{a}}C_{\mathrm{a}}f_{\mathrm{c},i}\left( T_{\mathrm{v},i} - T_{1} \right)}{r_{\mathrm{vh},i}}
\end{equation}
%
对于第3层方程,令:
\begin{equation}
  c_{\mathrm{ah3}} = \frac{1}{r_{\mathrm{ah}}},\ c_{\mathrm{gh3}} = \frac{1}{r_{\mathrm{d3}}},\ c_{\mathrm{vh},i} = \frac{1}{r_{\mathrm{vh},i}}
\end{equation}
%
\begin{equation}
  w_{\mathrm{h3}} = c_{\mathrm{ah3}} + c_{\mathrm{gh3}} + \sum_{i \in \mbox{\tiny {第3层}}}^{}{f_{\mathrm{c},i}c_{\mathrm{vh},i}}
\end{equation}
%
\begin{equation}
  w_{\mathrm{ah3}} = \frac{c_{\mathrm{ah3}}}{w_{\mathrm{h3}}},\ w_{\mathrm{gh3}} = \frac{c_{\mathrm{gh3}}}{w_{\mathrm{h3}}},\ w_{\mathrm{vh},i} = \frac{f_{\mathrm{c},i}c_{\mathrm{vh},i}}{w_{\mathrm{h3}}}
\end{equation}
%
对于第2层方程,令:
\begin{equation}
  c_{\mathrm{ah2}} = \frac{1}{r_{\mathrm{d3}}},\ c_{\mathrm{gh2}} = \frac{1}{r_{\mathrm{d2}}},\ c_{\mathrm{vh},i} = \frac{1}{r_{\mathrm{vh},i}}
\end{equation}
%
\begin{equation}
  w_{\mathrm{h2}} = c_{\mathrm{ah2}} + c_{\mathrm{gh2}} + \sum_{i \in \mbox{\tiny {第2层}}}^{}{f_{\mathrm{c},i}c_{\mathrm{vh},i}}
\end{equation}
%
\begin{equation}
  w_{\mathrm{ah2}} = \frac{c_{\mathrm{ah2}}}{w_{\mathrm{h2}}},\ w_{\mathrm{gh2}} = \frac{c_{\mathrm{gh2}}}{w_{\mathrm{h2}}},\ w_{\mathrm{vh},i} = \frac{f_{\mathrm{c},i}c_{\mathrm{vh},i}}{w_{\mathrm{h2}}}
\end{equation}
%
对于第1层方程,令:
\begin{equation}
  c_{\mathrm{ah1}} = \frac{1}{r_{\mathrm{d2}}},\ c_{\mathrm{gh1}} = \frac{1}{r_{\mathrm{d1}}},\ c_{\mathrm{vh},i} = \frac{1}{r_{\mathrm{vh},i}}
\end{equation}
%
\begin{equation}
  w_{\mathrm{h1}} = c_{\mathrm{ah1}} + c_{\mathrm{gh1}} + \sum_{i \in \mbox{\tiny {第1层}}}^{}{f_{\mathrm{c},i}c_{\mathrm{vh},i}}
\end{equation}
%
\begin{equation}
  w_{\mathrm{ah1}} = \frac{c_{\mathrm{ah1}}}{w_{\mathrm{h1}}},\ w_{\mathrm{gh1}} = \frac{c_{\mathrm{gh1}}}{w_{\mathrm{h1}}},\ w_{\mathrm{vh},i} = \frac{f_{\mathrm{c},i}c_{\mathrm{vh},i}}{w_{\mathrm{h1}}}
\end{equation}
%
将方程~\eqref{eq:H_3L_1} 关于\(T_{3}\)的表达式和方程~\eqref{eq:H_3L_3}关于\(T_{1}\)的表达式带入方程~\eqref{eq:H_3L_2},可计算\(T_{2}\)为:
\begin{equation}
  T_{2} = \frac{w_{\mathrm{ah2}}T_{\mathrm{av}} + w_{\mathrm{gh2}}T_{\mathrm{gv}} + \sum_{i \in \mbox{\tiny {第二层}}}^{}{w_{\mathrm{vh},i}T_{\mathrm{v},i}}}{f_{\mathrm{h}}}
\end{equation}
%
其中:
\begin{equation}
  T_{\mathrm{av}} = w_{\mathrm{ah3}}T_{\mathrm{a}} + \sum_{i \in \mbox{\tiny {第3层}}}^{}{w_{\mathrm{vh},i}T_{\mathrm{v},i}}
\end{equation}
%
\begin{equation}
  T_{\mathrm{gv}} = w_{\mathrm{gh1}}T_{\mathrm{g}} + \sum_{i \in \mbox{\tiny {第1层}}}^{}{w_{\mathrm{vh},i}T_{\mathrm{v},i}}
\end{equation}
%
\begin{equation}
  f_{\mathrm{h}} = 1 - w_{\mathrm{ah2}}w_{\mathrm{gh3}} - w_{\mathrm{gh2}}w_{\mathrm{ah1}}
\end{equation}
%
将\(T_{2}\)表达式带入到方程中,即可得到\(T_{1}\)和\(T_{3}\)表达式:
\begin{equation}
  T_{1} = w_{\mathrm{ah1}}T_{\mathrm{2}} + w_{\mathrm{gh1}}T_{\mathrm{g}} + \sum_{i \in \mbox{\tiny {第1层}}}^{}{w_{\mathrm{vh},i}T_{\mathrm{v},i}}
\end{equation}
%
\begin{equation}
  T_{3} = w_{\mathrm{ah3}}T_{\mathrm{a}} + w_{\mathrm{gh3}}T_{2} + \sum_{i \in \mbox{\tiny {第3层}}}^{}{w_{\mathrm{vh},i}T_{\mathrm{v},i}}
\end{equation}
%
对于第2层植被感热相对温度的导数计算为:
\begin{equation}
  \frac{\partial H_{\mathrm{v},i}}{\partial T_{\mathrm{v},i}} = \rho_{\mathrm{a}}C_{\mathrm{a}}c_{\mathrm{vh},i}\left( 1 - \frac{w_{\mathrm{vh},i}}{f_{\mathrm{h}}} \right)
\end{equation}
%
第1层植被感热相对温度的导数计算为:
\begin{equation}
  \frac{\partial H_{\mathrm{v},i}}{\partial T_{\mathrm{v},i}} = \rho_{\mathrm{a}}C_{\mathrm{a}}c_{\mathrm{vh},i}\left( 1 - \frac{{w_{\mathrm{ah1}}w_{\mathrm{gh2}}w}_{\mathrm{vh},i}}{f_{\mathrm{h}}} - w_{\mathrm{vh},i} \right)
\end{equation}
%
第3层植被感热相对温度的导数计算为:
\begin{equation}
  \frac{\partial H_{\mathrm{v},i}}{\partial T_{\mathrm{v},i}} = \rho_{\mathrm{a}}C_{\mathrm{a}}c_{\mathrm{vh},i}\left( 1 - \frac{{w_{\mathrm{gh3}}w_{\mathrm{ah2}}w}_{\mathrm{vh},i}}{f_{\mathrm{h}}} - w_{\mathrm{vh},i} \right)
\end{equation}
%
地面感热相对温度变化的导数为:
\begin{equation}
  \frac{\partial H_{\mathrm{g}}}{\partial T_{\mathrm{g}}} = \rho_{\mathrm{a}}C_{\mathrm{a}}c_{\mathrm{gh1}}\left( 1 - \frac{w_{\mathrm{ah1}}w_{\mathrm{gh2}}w_{\mathrm{gh1}}}{f_{\mathrm h}}- w_{\mathrm{gh1}} \right)
\end{equation}

对于潜热通量,其平衡方程为:
\begin{equation}
  \lambda E = {\lambda E}_{2} + \sum_{i \in \mbox{\tiny {第3层}}}^{}{\lambda E_{\mathrm{v},i}}
\end{equation}
%
\begin{equation}
  \lambda E_{2} = {\lambda E}_{1} + \sum_{i \in \mbox{\tiny {第2层}}}^{}{\lambda E_{\mathrm{v},i}}
\end{equation}
%
\begin{equation}
  {\lambda E}_{1} = \lambda E_{\mathrm{g}} + \sum_{i \in \mbox{\tiny {第1层}}}^{}{\lambda E_{\mathrm{v},i}}
\end{equation}
%
即:
\begin{equation}\label{eq:LE_3L_1}
  \frac{\rho_{\mathrm{a}}\left( q_{3} - q_{\mathrm{a}} \right)}{r_{\mathrm{aw}}} = \frac{\rho_{\mathrm{a}}\left( q_{2} - q_{3} \right)}{r_{\mathrm{d3}}} + \sum_{i \in \mbox{\tiny {第3层}}}^{}\frac{\rho_{\mathrm{a}}f_{\mathrm{c},i}\left( q_{\mathrm{v},i} - q_{3} \right)}{r_{\mathrm{vw},i}}
\end{equation}
%
\begin{equation}\label{eq:LE_3L_2}
  \frac{\rho_{\mathrm{a}}\left( q_{2} - q_{3} \right)}{r_{\mathrm{d3}}} = \frac{\rho_{\mathrm{a}}\left( q_{1} - q_{2} \right)}{r_{\mathrm{d2}}} + \sum_{i \in \mbox{\tiny {第2层}}}^{}\frac{\rho_{\mathrm{a}}f_{\mathrm{c},i}\left( q_{\mathrm{v},i} - q_{2} \right)}{r_{\mathrm{vw},i}}
\end{equation}
%
\begin{equation}\label{eq:LE_3L_3}
  \frac{\rho_{\mathrm{a}}\left( q_{1} - q_{2} \right)}{r_{\mathrm{d2}}} = \frac{\rho_{\mathrm{a}}\left( q_{\mathrm{g}} - q_{1} \right)}{r_{\mathrm{d1}}} + \sum_{i \in \mbox{\tiny {第1层}}}^{}\frac{\rho_{\mathrm{a}}f_{\mathrm{c},i}\left( q_{\mathrm{v},i} - q_{1} \right)}{r_{\mathrm{vw},i}}
\end{equation}
%
对于第3层方程,令:
\begin{equation}
  c_{\mathrm{aw3}} = \frac{1}{r_{\mathrm{aw}}},\ c_{\mathrm{gw3}} = \frac{1}{r_{\mathrm{d3}}},\ c_{\mathrm{vw},i} = \frac{1}{r_{\mathrm{vw},i}}
\end{equation}
%
\begin{equation}
  w_{\mathrm{q3}} = c_{\mathrm{aw3}} + c_{\mathrm{gw3}} + \sum_{i \in \mbox{\tiny {第3层}}}^{}{f_{\mathrm{c},i}c_{\mathrm{vw},i}}
\end{equation}
%
\begin{equation}
  w_{\mathrm{aq3}} = \frac{c_{\mathrm{aw3}}}{w_{\mathrm{q3}}},\ w_{\mathrm{gh3}} = \frac{c_{\mathrm{gw3}}}{w_{\mathrm{q3}}},\ w_{\mathrm{vh},i} = \frac{f_{\mathrm{c},i}c_{\mathrm{vw},i}}{w_{\mathrm{q3}}}
\end{equation}
%
对于第2层方程,令:
\begin{equation}
  c_{\mathrm{aw2}} = \frac{1}{r_{\mathrm{d3}}},\ c_{\mathrm{gw2}} = \frac{1}{r_{\mathrm{d2}}},\ c_{\mathrm{vw},i} = \frac{1}{r_{\mathrm{vw},i}}
\end{equation}
%
\begin{equation}
  w_{\mathrm{q2}} = c_{\mathrm{aw2}} + c_{\mathrm{gw2}} + \sum_{i \in \mbox{\tiny {第2层}}}^{}{f_{\mathrm{c},i}c_{\mathrm{vw},i}}
\end{equation}
%
\begin{equation}
  w_{\mathrm{aq2}} = \frac{c_{\mathrm{aw2}}}{w_{\mathrm{q2}}},\ w_{\mathrm{gh2}} = \frac{c_{\mathrm{gw2}}}{w_{\mathrm{q2}}},\ w_{\mathrm{vh},i} = \frac{f_{\mathrm{c},i}c_{\mathrm{vw},i}}{w_{\mathrm{q2}}}
\end{equation}
%
对于第1层方程,令:
\begin{equation}
  c_{\mathrm{aw1}} = \frac{1}{r_{\mathrm{d2}}},\ c_{\mathrm{gw1}} = \frac{1}{r_{\mathrm{d1}}},\ c_{\mathrm{vw},i} = \frac{1}{r_{\mathrm{vw},i}}
\end{equation}
%
\begin{equation}
  w_{\mathrm{q1}} = c_{\mathrm{aw1}} + c_{\mathrm{gw1}} + \sum_{i \in \mbox{\tiny {第1层}}}^{}{f_{\mathrm{c},i}c_{\mathrm{vw},i}}
\end{equation}
%
\begin{equation}
  w_{\mathrm{aq1}} = \frac{c_{\mathrm{aw1}}}{w_{\mathrm{q1}}},\ w_{\mathrm{gq1}} = \frac{c_{\mathrm{gw1}}}{w_{\mathrm{q1}}},\ w_{\mathrm{vq},i} = \frac{f_{\mathrm{c},i}c_{\mathrm{vw},i}}{w_{\mathrm{q1}}}
\end{equation}
%
将方程~\eqref{eq:LE_3L_1} 关于\(q_{3}\)的表达式和方程~\eqref{eq:LE_3L_3}关于\(q_{1}\)的表达式一起带入方程~\eqref{eq:LE_3L_2},可计算\(q_{2}\)为:
\begin{equation}
  q_{2} = \frac{w_{\mathrm{aq2}}q_{\mathrm{av}} + w_{\mathrm{gq2}}q_{\mathrm{gv}} + \sum_{i \in \mbox{\tiny {第2层}}}^{}{w_{\mathrm{vq},i}q_{\mathrm{v},i}}}{f_{\mathrm{q}}}
\end{equation}
%
其中:
\begin{equation}
  q_{\mathrm{av}} = w_{\mathrm{aq3}}q_{\mathrm{a}} + \sum_{i \in \mbox{\tiny {第3层}}}^{}{w_{\mathrm{vq},i}q_{\mathrm{v},i}}
\end{equation}
%
\begin{equation}
  q_{\mathrm{gv}} = w_{\mathrm{gq1}}q_{\mathrm{g}} + \sum_{i \in \mbox{\tiny {第1层}}}^{}{w_{\mathrm{vq},i}q_{\mathrm{v},i}}
\end{equation}
%
\begin{equation}
  f_{\mathrm{q}} = 1 - w_{\mathrm{aq2}}w_{\mathrm{gq3}} - w_{\mathrm{gq2}}w_{\mathrm{aq1}}
\end{equation}
%
将计算得到的\(q_{2}\)带入到方程~\eqref{eq:LE_3L_1} 和~\eqref{eq:LE_3L_3},即可计算出\(q_{1}\)和\(q_{3}\)为:
\begin{equation}
  q_{1} = w_{\mathrm{aq1}}q_{2} + w_{\mathrm{gq1}}q_{\mathrm{g}} + \sum_{i \in \mbox{\tiny {第1层}}}^{}{w_{\mathrm{vq},i}q_{\mathrm{v},i}}
\end{equation}
%
\begin{equation}
  q_{3} = w_{\mathrm{aq3}}q_{\mathrm{a}} + w_{\mathrm{gq3}}q_{\mathrm{2}} + \sum_{i \in \mbox{\tiny {第3层}}}^{}{w_{\mathrm{vq},i}q_{\mathrm{v},i}}
\end{equation}
%
第2层植被叶片蒸散发相对叶温变化的导数计算为:
\begin{equation}
  \frac{\partial E_{\mathrm{v},i}}{\partial T_{\mathrm{v},i}} = \rho_{\mathrm{a}}c_{\mathrm{vw},i}\left( 1 - \frac{w_{\mathrm{vq},i}}{f_{\mathrm{q}}} \right)\frac{{\rm d}q_{\mathrm{sat}}^{T_{\mathrm{v},i}}}{{\rm d}T_{\mathrm{v},i}}
\end{equation}
%
其中叶片蒸腾水汽通量相对叶片温度变化的导数计算为:
\begin{equation}
  \frac{\partial E_{\mathrm{tr},i}}{\partial T_{\mathrm{v},i}} = \rho_{\mathrm{a}}\left( 1 - f_{\mathrm{wet}} \right)\delta\left( \frac{\text{LAI}_{\mathrm{sun}}}{r_{\mathrm{b},i} + r_{\mathrm{s},\mathrm {sun},i}} + \frac{\text{LAI}_{\mathrm{sha}}}{r_{\mathrm{b},i} + r_{\mathrm{s,sha},i}} \right)\left( 1 - \frac{w_{\mathrm{vq},i}}{f_{\mathrm{q}}} \right)\frac{{\rm d}q_{\mathrm{sat}}^{T_{\mathrm{v},i}}}{{\rm d}T_{\mathrm{v},i}}
\end{equation}
%
叶片蒸发水汽通量相对叶片温度变化导数计算为:
\begin{equation}
  \frac{\partial E_{\mathrm{va},i}}{\partial T_{\mathrm{v},i}} = \rho_{\mathrm{a}}\left( 1 - \delta\left( 1 - f_{\mathrm{wet}} \right) \right)\frac{\text{LAI} + \text{SAI}}{r_{\mathrm{b},i}}\left( 1 - \frac{w_{\mathrm{vq},i}}{f_{\mathrm{q}}} \right)\frac{{\rm d}q_{\mathrm{sat}}^{T_{\mathrm{v},i}}}{{\rm d}T_{\mathrm{v},i}}
\end{equation}
%
第1层植被叶片蒸散发相对叶温变化的导数计算为:
\begin{equation}
  \frac{\partial E_{\mathrm{v},i}}{\partial T_{\mathrm{v},i}} = \rho_{\mathrm{a}}c_{\mathrm{vw},i}\left( 1 - \frac{w_{\mathrm{aq1}}w_{\mathrm{gq2}}w_{\mathrm{vq},i}}{f_{\mathrm{q}}} - w_{\mathrm{vq},i} \right)\frac{{\rm d}q_{\mathrm{sat}}^{T_{\mathrm{v},i}}}{{\rm d}T_{\mathrm{v},i}}
\end{equation}
%
其中叶片蒸腾水汽通量相对叶片温度变化的导数计算为:
\begin{equation}
  \frac{\partial E_{\mathrm{tr},i}}{\partial T_{\mathrm{v},i}} = \rho_{\mathrm{a}}\left( 1 - f_{\mathrm{wet}} \right)\delta\left( \frac{\text{LAI}_{\mathrm{sun}}}{r_{\mathrm{b},i} + r_{\mathrm{s},\mathrm {sun},i}} + \frac{\text{LAI}_{\mathrm{sha}}}{r_{\mathrm{b},i} + r_{\mathrm{s,sha},i}} \right)\left( 1 - \frac{w_{\mathrm{aq1}}w_{\mathrm{gq2}}w_{\mathrm{vq},i}}{f_{\mathrm{q}}} - w_{\mathrm{vq},i} \right)\frac{{\rm d}q_{\mathrm{sat}}^{T_{\mathrm{v},i}}}{{\rm d}T_{\mathrm{v},i}}
\end{equation}
%
叶片蒸发水汽通量相对叶片温度变化导数计算为:
\begin{equation}
  \frac{\partial E_{\mathrm{va},i}}{\partial T_{\mathrm{v},i}} = \rho_{\mathrm{a}}\left( 1 - \delta\left( 1 - f_{\mathrm{wet}} \right) \right)\frac{\text{LAI} + \text{SAI}}{r_{\mathrm{b},i}}\left( 1 - \frac{w_{\mathrm{aq1}}w_{\mathrm{gq2}}w_{\mathrm{vq},i}}{f_{\mathrm{q}}} - w_{\mathrm{vq},i} \right)\frac{{\rm d}q_{\mathrm{sat}}^{T_{\mathrm{v},i}}}{{\rm d}T_{\mathrm{v},i}}
\end{equation}
%
第3层植被叶片蒸散发相对叶温变化的导数计算为:
\begin{equation}
  \frac{\partial E_{\mathrm{v},i}}{\partial T_{\mathrm{v},i}} = \rho_{\mathrm{a}}c_{\mathrm{vw},i}\left( 1 - \frac{w_{\mathrm{gq3}}w_{\mathrm{aq2}}w_{\mathrm{vq},i}}{f_{\mathrm{q}}} - w_{\mathrm{vq},i} \right)\frac{{\rm d}q_{\mathrm{sat}}^{T_{\mathrm{v},i}}}{{\rm d}T_{\mathrm{v},i}}
\end{equation}
%
其中叶片蒸腾水汽通量相对叶片温度变化的导数计算为:
\begin{equation}
  \frac{\partial E_{\mathrm{tr},i}}{\partial T_{\mathrm{v},i}} = \rho_{\mathrm{a}}\left( 1 - f_{\mathrm{wet}} \right)\delta\left( \frac{\text{LAI}_{\mathrm{sun}}}{r_{\mathrm{b},i} + r_{\mathrm{s},\mathrm {sun},i}} + \frac{\text{LAI}_{\mathrm{sha}}}{r_{\mathrm{b},i} + r_{\mathrm{s},\mathrm {sha},i}} \right)\left( 1 - \frac{w_{\mathrm{gq3}}w_{\mathrm{aq2}}w_{\mathrm{vq},i}}{f_{\mathrm{q}}} - w_{\mathrm{vq},i} \right)\frac{{\rm d}q_{\mathrm{sat}}^{T_{\mathrm{v},i}}}{{\rm d}T_{\mathrm{v},i}}
\end{equation}
%
叶片蒸发水汽通量相对叶片温度变化导数计算为:
\begin{equation}
  \frac{\partial E_{\mathrm{va},i}}{\partial T_{\mathrm{v},i}} = \rho_{\mathrm{a}}\left( 1 - \delta\left( 1 - f_{\mathrm{wet}} \right) \right)\frac{\text{LAI} + \text{SAI}}{r_{\mathrm{b},i}}\left( 1 - \frac{w_{\mathrm{gq3}}w_{\mathrm{aq2}}w_{\mathrm{vq},i}}{f_{\mathrm{q}}} - w_{\mathrm{vq},i} \right)\frac{{\rm d}q_{\mathrm{sat}}^{T_{\mathrm{v},i}}}{{\rm d}T_{\mathrm{v},i}}
\end{equation}
%
地面水汽通量相对温度变化的导数为:
\begin{equation}
  \frac{\partial E_{\mathrm{g}}}{\partial T_{\mathrm{g}}} = \rho_{\mathrm{a}}c_{\mathrm{gw1}}\left( 1 - \frac{w_{\mathrm{aq1}}w_{\mathrm{gq2}}w_{\mathrm{gq1}}}{f_{\mathrm {q}}} -w_{\mathrm{gq1}} \right)\frac{{\rm d}q_{\mathrm{g}}}{{\rm d}T_{\mathrm{g}}}
\end{equation}

\section{土壤阻抗计算}
\begin{mymdframed}{代码}
  本节对应的代码文件为\texttt{MOD\_SoilSurfaceResistance.F90}。
\end{mymdframed}

土壤蒸发分为两个阶段:第一阶段是以大气需求为主的蒸发速率较高的阶段,第二阶段是以土壤中水汽扩散为主的蒸发速率较低的阶段。当模型计算土壤蒸发时,会先计算地面比湿\(q_{\mathrm{g}}\) (公式~\eqref{Eg})。当土壤蒸发处于第一阶段时,\(q_{\mathrm{g}}\)是指地面处的比湿;当土壤蒸发处于第二阶段时,即土壤处于较为干燥的状态时,所计算的\(q_{\mathrm{g}}\)是指土壤孔隙中自由水面附近的比湿\(q_{\mathrm{soil}}\),这一阶段土壤中的水汽在干燥层中进行扩散才能达到地面,此部分的阻抗即为土壤阻抗(\(r_{\mathrm{ss}}\))。之前CoLM未考虑此部分阻抗,新版本中添加了土壤干燥部分中水汽分子从土壤孔隙向地表扩散的物理过程,即土壤阻抗方案,如图~\ref{fig:土壤阻抗示意图} 所示。

{
  \begin{figure}[htbp]
    \centering
    \includegraphics[width=0.95\textwidth]{Figures/地表湍流交换过程/土壤阻抗示意图_v4.png}
    \caption[土壤阻抗示意图]{土壤阻抗示意图。其中,\(q_{\mathrm{a}}\)、\(q_{\mathrm{g}}\)、\(q_{\mathrm{soil}}\)分别代表一定高度空气比湿(高度取决于有无植被及采用的阻抗交换网络)、地表比湿和土壤孔隙中水附近的比湿,\(r_{\mathrm{aw}}\)、\(r_{\mathrm{ss}}\)分别代表空气动力学阻抗和土壤阻抗}
    \label{fig:土壤阻抗示意图}
  \end{figure}
}

目前CoLM支持5种土壤阻抗方案,每种方案的计算表达式和模型中对应方案的选择如表~\ref{tab:土壤阻抗方案列表} 所示。如果用户需要打开土壤阻抗选项,需设置\texttt{DEF\_RSS\_SCHEME}
= {[}土壤阻抗选项{]},不打开则设置\texttt{DEF\_RSS\_SCHEME} =
0。在CoLM模型默认打开方案1。

图~\ref{fig:土壤阻抗方案差异图} 代表不同土壤含水量下阻抗方案的\(r_{\mathrm{ss}}\)和$\beta$的差异,$\beta$代表了不同方案对潜在蒸发的限制,$\beta$值越大,限制越小。当无植被覆盖时,$\beta$通过\(r_{\mathrm{ss}}\)和\(r_{\mathrm{aw}}\)计算得到:
\begin{equation}
  \beta = \frac{1}{1 + \frac{r_{\mathrm{ss}}}{r_{\mathrm{aw}}}}\
\end{equation}

在土壤特别干燥(如小于萎蔫点)和特别湿润(大于田间持水量)时,不同方案\(r_{\mathrm{ss}}\)的数值有差异,但是土壤阻抗方案的差异较小;当土壤水含量在中间状态时,土壤阻抗的方案差异较大。对不同方案的计算得到的$\beta$进行排序发现:LP92\textgreater TR13\textgreater SZ09\textgreater S92\textgreater SL14,因此SL14方案对土壤蒸发的限制最强,LP92方案限制较弱。

{
  \begin{figure}[htbp]
    \centering
    \includegraphics[width=0.7\textwidth]{Figures/地表湍流交换过程/土壤阻抗方案差异图.png}
    \caption[土壤阻抗方案差异图]{土壤阻抗方案差异图。采用Campbell的土壤参数,其中\(\theta_{\mathrm{s}}\)为0.45 \unit{m^{3}.m^{-3}},B为7.12,\(\varphi_{\mathrm{s}}\)为\num{-123} \unit {mm},土壤表层温度为282.15 K,\(K_{\mathrm{s}}\)为 \num{1.25e-6} \unit{m.s^{-1}}。设定厚度为1.75 cm,SL14方案中\(K_{\mathrm{sl}}\)为0.8,SZ09方案中\(w_{\mathrm{sz}}\)为5,\(r_{\mathrm{aw}}\)=50 \unit {s.m^{-1}}。图中灰色虚线从左到右依次为土壤萎蔫点和田间持水量时的土壤水含量}
    \label{fig:土壤阻抗方案差异图}
  \end{figure}
}

{
  \begin{landscape}
    \begin{table}[htbp]
      \caption{土壤阻抗计算方案列表}
      \label{tab:土壤阻抗方案列表}
      \begin{tabular}{@{}clll@{}}
        \toprule
        模式选项                                                                                & 土壤阻抗表达式                                                                                                                                      & 方法/假设                              & 参考文献                                   \\
        \midrule
        1                                                                                       & \(r_{\mathrm{ss}} = \frac{DSL}{D_{\mathrm{g}}}\)                                                                                                    & 菲克定律                               & \citet{sl2014}, SL14                       \\
        2                                                                                       & \(r_{\mathrm{ss}} = \frac{DSL}{D_{\mathrm{g}}}\)                                                                                                    & 菲克定律                               & \citet{sz2009}, SZ09                       \\
        3                                                                                       & \(\frac{1}{r_{\mathrm{ss}}} = \frac{1}{r_{\mathrm{g\ }}} + \frac{1}{r_{\mathrm{w\ }}}\)                                                             & 菲克定律, 达西定律                     & \citet{tang2013}, TR13                     \\
        4                                                                                       & \(\beta_{\mathrm{soil}} = \left\{ \begin{array}{r} \frac{1}{4}{\lbrack 1 - \cos(\frac{\theta_{\mathrm{1}}}{\theta_{\mathrm{fc,1}}}\pi)\rbrack}^{2} \\ 1 \end{array} \right.\ \begin{matrix}  & \theta_{\mathrm{1}} < \theta_{\mathrm{fc}} \\  & \theta_{\mathrm{1}} \geqslant \theta_{\mathrm{fc}}\text{ or }q_{\mathrm{\text{a }}} > q_{\mathrm{soil}} \end{matrix}\) & 经验拟合 & \citet{lp1992}\textsuperscript{a}, LP92 \\
        5                                                                                       & \(r_{\mathrm{ss}} = \exp\left( 8.206 - 6.0\frac{\theta_{\mathrm{1}}}{\theta_{\mathrm{sat,1}}} \right)\)
                                                                                                & 经验拟合                                                                                                                                            & \citet{s1992}\textsuperscript{b}, S92 \\ \bottomrule
      \end{tabular}
      \footnotesize                                                                            \\
      \textsuperscript{a} $\beta_{\mathrm{soil}}$为限制土壤蒸发的因子,范围为{[}0-1{]},无单位 \\
      \textsuperscript{b} Sellers原方案为$r_{\mathrm{ss}} = \exp\left( 8.206 - 4.255\frac{\theta_{\mathrm{1}}}{\theta_{\mathrm{sat,1}}} \right)\ $,然而~\citet{sz2009}指出该方案在土壤水饱和时,$r_{\mathrm{ss}}$计算值不合理,为了降低土壤较为湿润的$r_{\mathrm{ss}}$,Noah-MP v5对其进行修正为$r_{\mathrm{ss}} = \exp\left( 8.206 - 6.0 \frac{{\theta}_{\mathrm{1}}}{\theta_{\mathrm{sat,1}}} \right)$
    \end{table}
  \end{landscape}
}

以下针对不同的方案进行简单介绍。

\begin{enumerate}
    \def\labelenumi{\arabic{enumi}.}
  \item
    SL14方案
%\end{enumerate}

    在SL14方案中,\(r_{\mathrm{ss}}\)是通过参数化土壤干燥层厚度\(DSL\)和水汽分子在土壤中的扩散系数\(D_{\mathrm{g}}\)得到的。其中,\(DSL\)的计算由下式给出
    \begin{equation}
      DSL = \begin{cases}
        \Delta z_{1} \times \frac{\theta_{\mathrm{init,1\ \ }} - \theta_{1}}{\theta_{\mathrm{init,2\ \ }} - \theta_{\mathrm{a}}} & \theta_{1} < K_{\mathrm{sl}}\theta_{\mathrm{sat,1}} \\
        0   & \theta_{1} \geqslant K_{\mathrm{sl}}\theta_{\mathrm{sat,1}}
      \end{cases}
    \end{equation}
    其中,$\Delta z_{1}$ 代表第一层的土壤厚度 (\unit{m}),\(\theta_{\mathrm{init,1}}\)和\(\theta_{\mathrm{init,2}}\)代表干燥层启动时的土壤体积含水量 (\unit{m^{3}.m^{-3}}),计算为:
    \begin{equation}
      \theta_{\mathrm{init,1}} = K_{\mathrm{sl}}\left(\theta_{\mathrm{sat,1}} - \theta_{\mathrm{ice,1}} \right)
    \end{equation}
    \begin{equation}
      \theta_{\mathrm{init,2}} = K_{\mathrm{sl}}\theta_{\mathrm{sat,1}}\
    \end{equation}
    \(\theta_{1}\)代表第一层土壤体积含水量\((\unit{m^{3}.m^{-3}})\),\(\theta_{\mathrm{sat,1}}\)为第一层土壤的饱和体积含水量\((\unit{m^{3}.m^{-3}})\),\(\theta_{\mathrm{ice,1}}\)为第一层土壤冰的体积含水量\((\unit{m^{3}.m^{-3}})\),\(\theta_{\mathrm{a}}\)代表风干后即\(\psi_{\mathrm{a}} = - 10^{7}\) mm时的土壤体积含水量\((\unit{m^{3}.m^{-3}})\),\(K_{\mathrm{sl}}\)为人为设定的参数,在CoLM中,按照~\citet{sl2014}文章中的测试,$K_{\mathrm{sl}}=0.8$。

    水汽分子在土壤中的扩散系数\(D_{\mathrm{g}}\)则是通过计算水汽分子在空气中的扩散系数\(\ D_{0}\)和水汽穿过土壤基质的路径\(\tau\)得到的,计算如下:
    \begin{equation}
      D_{\mathrm{g}} = D_{0} \times \tau\
    \end{equation}
    \(D_{0\ }\)代表水汽分子在空气中的扩散系数\((\unit{m^{2}.s^{- 1}})\),计算如下:
    \begin{equation}
      D_{0\ } = 2.12 \times 10^{- 5}\left( \frac{T_{1}}{273.15} \right)^{1.75}\
    \end{equation}
    其中,\(T_{1}\)代表土壤表层的温度(K)。

%\begin{enumerate}
%\def\labelenumi{\arabic{enumi}.}
%\setcounter{enumi}{1}
  \item
    SZ09方案
%\end{enumerate}

    SZ09方案与SL14方案同样都是基于土壤干燥层中水汽扩散的菲克定律得到。只是两者对于\(DSL\)的计算方式不同,SZ09方案中\(DSL\)的计算如下所示:
    \begin{equation}
      DSL = \Delta z_{1} \times \frac{{\mathrm e}^{\left( 1 - \frac{\theta_{1}}{\theta_{\mathrm{sat,1}}} \right)^{w_{\mathrm{sz}}}} - 1}{e - 1}\
    \end{equation}
    其中,\(w_{\mathrm{sz}}\)为人为设定的参数,在CoLM中,按照~\citet{sz2009}中的测试,\(w_{\mathrm{sz}}\)设置为5。\(D_{\mathrm{g}}\)、\(\tau\)和\(D_{0\ }\)的计算与SL14方案相同。

%\begin{enumerate}
%\def\labelenumi{\arabic{enumi}.}
%\setcounter{enumi}{2}
  \item
    TR13方案
%\end{enumerate}

    与SL14方案和SZ09方案仅仅只考虑土壤干燥层中的水汽扩散不同,TR13方案同时考虑了土壤层中液态水和气态水的扩散。TR13方案计算的\(r_{\mathrm{ss}}\)进一步被划分为\(r_{\mathrm{g\ }}\)代表土壤孔隙中水汽扩散的阻抗 (\unit{s.m^{-1}})和\(r_{\mathrm{w\ }}\)代表液态水挥发阻抗(\unit{s.m^{-1}})。它们的计算如下:
    \begin{equation}
      \frac{1}{r_{\mathrm{\mathrm{g}}}} = \frac{2D_{\mathrm{g\ }}\varepsilon_{1}}
      {\Delta z_{1}}\ \
    \end{equation}
    \begin{equation}
      \frac{1}{r_{\mathrm{w}}} = \frac{2D_{\mathrm{w}}Bunsen\theta_{1}}
      {\Delta z_{1}}\
    \end{equation}
    其中,\(D_{\mathrm{w\ }}\)代表液态水的扩散系数(\unit{m^{2}.s^{- 1}}),\(\varepsilon_{1}\)代表第一层土壤中空气填充的孔隙空间(\unit{m^{3}.m^{-3}}),\(Bunsen\)为Bunsen溶解系数,计算如下:
    \begin{equation}
      D_{\mathrm{w}} = K_{1}\frac{\partial\varphi_{1}}{\partial\theta_{1}}\
    \end{equation}
    在Campbell土壤水力模型中,\(\frac{\partial\varphi_{1}}{\partial\theta_{1}}\)计算为:
    \begin{equation}
      \frac{\partial\varphi_{1}}{\partial\theta_{1}}\  = - \frac{B_{1}\varphi_{1}}{\theta_{1}}\
    \end{equation}
    在van Genuchten土壤水力模型中,\(\frac{\partial\varphi_{1}}{\partial\theta_{1}}\)计算为:
    \begin{equation}
      S_{1} = {\lbrack 1 + ( - \alpha\varphi_{1})^{n}\rbrack}^{- m}(\alpha > 0)
    \end{equation}
    \begin{equation}
      m = 1-\frac{1}{n}
    \end{equation}
    \begin{equation}
      \frac{\partial\varphi_{1}}{\partial\theta_{1}} = - \frac{m - 1}{\alpha m\left( \theta_{\mathrm{sat,1}} - \theta_{\mathrm{r,1}} \right)}S_{1}^{- \frac{1}{m}}\left( 1 - S_{1}^{\frac{1}{m}} \right)^{- m}\
    \end{equation}
    \begin{equation}
      \varepsilon_{1} = \theta_{\mathrm{sat,1}} - \ \theta_{\mathrm{a}}\
    \end{equation}
    \begin{equation}
      Bunsen = \ \frac{\rho_{\mathrm{liq}}}{\rho_{\mathrm{vap}}} = \frac{\rho_{\mathrm{liq}}}{\rho_{\mathrm{a}} \times q_{\mathrm{soil}}}\
    \end{equation}
    其中,\(\rho_{\mathrm{liq}}\)为液态水密度(\unit{{kg}.m^{-3}}),\(\rho_{\mathrm{a}}\)为空气密度(\unit{{kg}.m^{-3}}),$\alpha$、$m$ 和 $n$为曲线参数。

%\begin{enumerate}
%\def\labelenumi{\arabic{enumi}.}
%\setcounter{enumi}{3}
  \item
    LP92方案
%\end{enumerate}

    与前三个方案不同,LP92方案直接计算\(\beta_{\mathrm{soil}}\),\(\beta_{\mathrm{soil}}\)是将潜在蒸发降到实际蒸发的经验因子,范围为{[}0,1{]},如表~\ref{tab:土壤阻抗方案列表} 所示。\(\theta_{\mathrm{fc,1}}\)为第一层土壤的田间持水量(\unit{m{^3}.m^{-3}}),即土壤水势为\qty{-3399}{mm}时的土壤含水量。

%\begin{enumerate}
%\def\labelenumi{\arabic{enumi}.}
%\setcounter{enumi}{4}
  \item
    S92方案
%\end{enumerate}

    利用土壤水含量经验指数关系计算\(r_{\mathrm{ss}}\),如表~\ref{tab:土壤阻抗方案列表} 所示。
%
\end{enumerate}

对于方案1--3中\(\tau\)的表达式,CoLM提供了6种方案,具体见表~\ref{tab:tau方案列表}。其中,\(\varepsilon_{1}\)代表第一层土壤中空气填充的孔隙空间(\unit{m{^3}.m^{-3}}),\(\varepsilon_{100}\)代表在土壤水势为\num{-1000} mm时的空气填充的孔隙空间(\unit{m{^3}.m^{-3}})。
图~\ref{fig:tau方案差异图} 表明了不同\(\tau\)方案的差异,不同\(\tau\)方案的\(r_{\mathrm{ss}}\)数值有一定差异,但是$\beta$差异不大。

{
  \begin{table}[htbp]
    \centering
    \caption{CoLM可选用$\tau$方案列表}
    \label{tab:tau方案列表}
    \begin{tabular}{lcc}
      \toprule
      模式选项 & 表达式 & 参考文献 \\
      \midrule
      1 &
      \(\varepsilon_{1}^{2} \times {(\frac{\varepsilon_{1}}{\theta_{\mathrm{sat,1}}})}^{3/B_{1}}\)
      & \citet{BBC1999} BBC \\
      2 & \(0.66 \times \varepsilon_{1}{\times (\frac{\varepsilon_{1}}{\theta_{\mathrm{sat,1}}})}\)
      & \citet{moldrup2000} P\_WLR \\
      3 & \(\varepsilon_{1}^{3/2} \times (\frac{\varepsilon_{1}}{\theta_{\mathrm{sat,1}}})\)
      & \citet{moldrup2000} MA\_WLR \\
      4 & \(\varepsilon_{1}^{4/3} \times (\frac{\varepsilon_{1}}{\theta_{\mathrm{sat,1}}})\)
      & \citet{moldrup2000} MI\_WLR \\
      5 & \(\varepsilon_{1}^{4/3} \times {(\frac{\varepsilon_{1}}{\theta_{\mathrm{sat,1}}})}^{2}\)
      & \citet{millington_permeability_1961} MQ \\
      6 & \({\theta_{\mathrm{sat,1}}}^{2} \times {(\frac{\varepsilon_{1}}{\theta_{\mathrm{sat,1}}})}^{2 + \frac{\log\varepsilon_{100}^{\frac{1}{4}}}{\log\frac{\varepsilon_{100}}{\theta_{\mathrm{sat,1}}}}}\) & \citet{POE2005} POE \\ \bottomrule
    \end{tabular}
  \end{table}
}

{
  \begin{figure}[htb]
    \centering
    \includegraphics[width=0.7\textwidth]{Figures/地表湍流交换过程/tau方案差异图.png}
    \caption[\(\ \tau\)方案差异图]{$\tau$方案差异图。参数设置与图~\ref{fig:土壤阻抗方案差异图} 参数设置相同,土壤阻抗方案为SL14方案}
    \label{fig:tau方案差异图}
  \end{figure}
}

土壤阻抗方案1--3需要计算\(\tau\),而方案4和5无需计算\(\tau\)。当开启Campbell土壤参数方案时,所有\(\tau\)方案都可以被选取;当开启van Genuchten方案时,由于van Genuchten参数方案没有Campbell参数方案的B参数,因此\(\tau\)方案的方案1不适用van Genuchten土壤参数方案,其余方案都可以被选取。在CoLM模型中,当开启Campbell土壤参数方案时,\(\tau\)的默认选项为1;开启van Genuchten土壤参数方案时,\(\tau\)的默认选项为6。\(\tau\)的选项可直接在源代码\texttt{MOD\_SoilSurfaceResistance.F90}中进行修改。

土壤阻抗方案仅在地面非凝结或凝华时有效。当使用土壤阻抗方案为1--3、5时,在无植被覆盖下地面蒸发水汽通量\(E_{\mathrm{g}}\)即公式~\eqref{Eg} 修改为:
\begin{equation}
  E_{\mathrm{g}}=-\rho_{\mathrm{a}} \frac{q_{\mathrm{a}}-q_{\mathrm{g}}}{r_{\mathrm{a w}} + r_{\mathrm{s s}}}
\end{equation}
地面蒸发相对地面温度的变化率,即公式~\eqref{Eg/Tg_1} 修改为:
\begin{equation}
  \frac{\partial E_{\mathrm{g}}}{\partial T_{\mathrm{g}}}= \frac{\rho_{\mathrm{a}}}{r_{\mathrm{a w}} + r_{\mathrm{s s}}} \frac{{\rm d} q_{\mathrm{g}}}{{\rm d} T_{\mathrm{g}}}
\end{equation}
在有植被覆盖下,地面与植被冠层周围空气间水汽通量\(E_{\mathrm{g}}\),即公式~\eqref{eq:Eg} 修改为:
\begin{equation}
  E_{\mathrm{g}}=-\rho_{\mathrm{a}} \frac{q_{\mathrm{s}}-q_{\mathrm{g}}}{r_{\mathrm{a w}}^{\prime}+r_{\mathrm{ss}}}
\end{equation}
$c_{\mathrm {g}}^w=\frac{1}{r_{\mathrm{aw}}^\prime}$修改为$c_{\mathrm {g}}^w=\frac{1}{r_{\mathrm{aw}} ^\prime+r_{\mathrm{ss}}}$,
地面蒸发相对地表温度的变化率,即公式~\eqref{Eg/Tg_2} 修改为:
\begin{equation}
  \frac{\partial E_{\mathrm{g}}}{\partial T_{\mathrm{g}}}=
  \frac{\rho_{\mathrm{a}}}{r_{\mathrm{a w}}^{\prime} + r_{\mathrm{s s}}} \frac{c_{\mathrm{a}}^{w}+c_{\mathrm{v}}^{w}}{c_{\mathrm{a}}^{w}+c_{\mathrm{g}}^{w}+c_{\mathrm{v}}^{w}} \frac{{\rm d} q_{\mathrm{g}}}{{\rm d} T_{\mathrm{g}}}
\end{equation}

当使用土壤阻抗方案4时,在无植被覆盖下水汽通量\(E_{\mathrm{g}}\),即公式~\eqref{Eg} 修改为:
\begin{equation}\label{Eg_modify}
  E_{\mathrm{g}}=-\rho_{\mathrm{a}}\beta_{\mathrm{soil}} \frac{q_{\mathrm{a}}-q_{\mathrm{g}}}{r_{\mathrm{a w}}}
\end{equation}
地面蒸发相对地面温度的变化率即公式~\eqref{Eg/Tg_1} 修改为:
\begin{equation}
  \frac{\partial E_{\mathrm{g}}}{\partial T_{\mathrm{g}}}= \frac{\rho_{\mathrm{a}} \beta_{\mathrm{soil}}}{r_{\mathrm{a w}} } \frac{{\rm d} q_{\mathrm{g}}}{{\rm d} T_{\mathrm{g}}}
\end{equation}
在有植被覆盖下地面与植被冠层周围空气间水汽通量\(E_{\mathrm{g}}\),即公式~\eqref{eq:Eg} 修改为:
\begin{equation}
  E_{\mathrm{g}}=-\rho_{\mathrm{a}} \beta_{\mathrm{soil}} \frac{q_{\mathrm{s}}-q_{\mathrm{g}}}{r_{\mathrm{a w}}^{\prime}}
\end{equation}
$c_{\mathrm {g}}^w=\frac{1}{r_{\mathrm{aw}}^\prime}$修改为$c_{\mathrm {g}}^w=\frac{\beta_{\mathrm{soil}}}{r_{\mathrm{aw}}^\prime}$,
地面蒸发相对地面温度的变化率,即公式~\eqref{Eg/Tg_2} 修改为:
\begin{equation}
  \frac{\partial E_{\mathrm{g}}}{\partial T_{\mathrm{g}}}=
  \frac{\rho_{\mathrm{a}} \beta_{\mathrm{soil}}}{r_{\mathrm{a w}}^{\prime}} \frac{c_{\mathrm{a}}^{w}+c_{\mathrm{v}}^{w}}{c_{\mathrm{a}}^{w}+c_{\mathrm{g}}^{w}+c_{\mathrm{v}}^{w}} \frac{{\rm d} q_{\mathrm{g}}}{{\rm d} T_{\mathrm{g}}}
\end{equation}

\chapter{光合作用和气孔导度}
%\addcontentsline{toc}{chapter}{光合作用和气孔导度}

%\begin{光合作用和气孔导度}

\section{植物的光合作用}\label{植物的光合作用}
C3植物的光合作用模拟是基于Farquhar光合作用模型~\citep{farquhar1980biochemical},
C4植物则是基于~\citet{collatz1992} 的光合作用改进模型。卡尔文循环是高等植物光合作用中的重要途径之一,CO$_2$和1,5-二磷酸核酮糖(RuBP)在1,5-二磷酸核酮糖羧化酶(Rubisco)的催化下产生羧化反应,生成3-磷酸甘油酸(PGA),并被还原为3-磷酸甘油醛(PGAL),PGA经过一系列转变再次形成RuBP,这被称为RuBP的更新阶段,完成卡尔文循环。当RuBP充足,卡尔文循环稳定时,多余的PGAL才会合成蔗糖和淀粉作为光合作用的产物存储在植物内。Faquhar模型认为卡尔文循环中的羧化速率($A_{c}$)和RuBP的再生速率是限制($A_{j}$)是光合作用总速率的两个关键限制因子,CoLM光合作用模块在此基础上增加了蔗糖和淀粉产物合成的速率限制($A_{p}$)。另外,叶片净光合同化速率 ($A_{n}$) 等于总光合同化速率减去叶呼吸速率 ($R_d$):
\begin{equation}\label{An1}
A_{n}=\min \left(A_{c}, A_{j}, A_{e}\right)-R_{d}
\end{equation}

RuBP羧化酶限制主要指当羧化酶活性较低时,光合作用羧化速率受限,也称为Rubisco限制。由于Rubisco同时催化RuBP的羧化反应和氧化反应,羧化反应需要CO$_2$而氧化反应需要O$_2$,羧化反应和氧化反应同时进行。因此,C3植物的胞间CO$_2$分压$c_i$和O$_2$分压$o_i$的比例极大影响了Rubisco作为羧化酶的效率。基于这个原因,C3植物的Rubisco限制下的光合同化速率($A_c$)可表达为Michaelis-Menten关于$c_i$和$o_i$的函数。此外,C4植物Rubisco限制下的光合羧化速率不再受胞间CO$_2$分压和O$_2$分压的影响,这是由于C4植物的光合作用途径的最初步骤是PEP的羧化,PEP羧化酶的活性很高,所以转运到维管束鞘细胞中的CO$_2$的浓度很高,大约是空气中的十倍。C4植物的卡尔文循环在在维管束鞘细胞进行,CO$_2$分压和O$_2$分压的变化对羧化速率的影响较小,因此,Rubisco限制下的光合同化速率可表示为:
\begin{equation}\label{A_C1}
A_{c}=\begin{cases}
\frac{V_{c \max }\left(c_{i}-\Gamma\right)}{c_{i}+K_{c}\left(1+\frac{o_{i}}{K_{o}}\right)}
     & \text{for C3 plants} \\ 
V_{c \max } & \text{for C4 plants}
\end{cases}
\end{equation}
其中,$\Gamma$是CO$_2$补偿点,$K_c$和$K_o$分别是对于CO$_2$和O$_2$的Michaelis--Menten常数(Pa),$o_i$是氧气分压(Pa)。\\
C3植物:\\
\begin{equation}\label{V_cmax_a}
V_{c \max }\left(T_{{leaf }}\right)=V_{c \max 25} \cdot \frac{2.1^{\frac{T_{{leaf }}-T_{o p}}{10}}}{1+e^{s_{1}\left(T_{{leaf }}-T_{{high }}\right)}} \cdot \beta
\end{equation}
C4植物:\\
\begin{equation}\label{V_cmax_b}
V_{c \max }\left(T_{{leaf }}\right)=V_{c \max 25} \cdot \frac{2.1^{\frac{T_{{leaf }}-T_{o p}}{10}}}{\left(1+e^{s_{2}\left(T_{{low }}
 - T_{{leaf }}\right)}\right)\left(1+e^{s_{1}\left(T_{{leaf }}-T_{h i g h}\right)}\right)} \cdot \beta
\end{equation}
不同植被的羧化能力存在差异,我们用25 \textcelsius 下的最大羧化速率 $V_{c \max 25}$ 来刻画植被的光合羧化能力,单位: \unit{mol.m^{-2}.s^{-1}};Rubisco羧化酶的活性随温度显著变化,CoLM光合作用模块针对C3和C4植物制定了两套温度响应函数。$T_{op}$是参考温度298 K;$s_1$和$s_2$分别为高温和低温的温度敏感性参数;$T_{low}$和$T_{high}$分别为羧化速率的低温和高温响应参数,
取值根据植被类型而变化,范围分别为: 278$\sim$288 K,303$\sim$313 K;$T_{leaf}$是叶片温度,通过对叶片能量平衡方程进行牛顿迭代方法而求解得到,详见章节~\ref{植被叶片温度计算},$\beta$是植物水分胁迫因子,取值范围0$\sim$1,详见章节~\ref{气孔导度的水分胁迫}。

当光照不足时,RuBP再生速率下降,成为制约光合作用开尔文循环的最主要因素。因此,RuBP再生速率限制下的羧化速率 ($A_j$) 可表达为有效光合辐射 ($PAR$) 的函数:
\begin{equation}\label{A_J1}
A_{J}=\begin{cases}\frac{J_x\left(PAR\right)\left(c_{i}-\Gamma\right)}{4c_{i}+8\Gamma}
     & \text{for C3 plants} \\
\alpha\left(4.6 \times 10^{-6} \cdot PAR\right) & \text{for C4 plants}
\end{cases}
\end{equation}

$J_x$是有效光合辐射(PAR)的函数,并受叶片温度 ($T_{leaf}$) 的调节:
\begin{equation}
J_{x}\left(T_{{leaf }}\right)=\min \left(\alpha\left(4.6 \times 10^{-6} \cdot PAR\right), J_{\max 25}
 \cdot e^{\frac{37000\left(T_{{leaf }}-T_{o p}\right)}{T_{o p} \cdot T_{{leaf }} \cdot R} \cdot \frac{1+e^{\frac{710 \cdot T_{o p}-220000}
 {R \cdot T_{o p}}}}{\frac{710 \cdot T_{{leaf }}-220000}{R \cdot T_{{leaf }}}}}\right) \cdot \beta
\end{equation}
其中$\alpha$是量子效率 (\qty{0.05}{mol.CO_2.mol^{-1}.photon});$PAR$是有效光合辐射,单位: \unit{W.m^{-2}},详细计算见章节~\ref{短波吸收辐射通量};
\num{4.6e-6} 代表单位从 \unit{W.m^{-2}} 转换到 \unit{mol.photon.m^{-2}} 的转换系数;
$J_{\max 25}$是25 \textcelsius 下的最大电子传输速率,单位: \unit{mol.m^{-2}.s^{-1}},$J_{\max 25}=1.97 \cdot V_{c \max 25}$; 
$R$是通用气体常数,$R=$ \qty{8.314467591}{mol.m^{-2}.s^{-1}}。

卡尔文循环中蔗糖和淀粉的合成速率限制($A_p$),用25 \textcelsius 最大羧化速率 ($V_{c \max}$) 进行参数化,并且描述其受叶温和水分胁迫因子的调控:\\
C3植物:\\
\begin{equation}\label{A_e_a}
A_p=\frac{V_{c \max 25}}{2} \cdot \frac{1.8^{\frac{T_{{leaf }}-T_{o p}}{10}}}{1+e^{s_{2}\left(T_{{leaf }}-T_{{low }}\right)}} \cdot \beta
\end{equation}
C4植物:\\
\begin{equation}\label{A_e_b}
A_p=\frac{V_{c \max 25}}{5} \cdot 1.8^{\frac{T_{{leaf }}-298.16}{10}} \cdot \beta
\end{equation}
三方面限制下的光合同化速率共用同一套温度响应常数 ($T_{op}$,$T_{low}$和$T_{high}$)


呼吸速率对温度的响应曲线可表示为$V_{c \max25}$的函数:
\begin{equation}\label{R_d1}
R_{d}=r_{{base }} \cdot V_{cmax 25} \cdot \frac{2.0^{\frac{T_{leaf}-T_{op}}{10}}}{1+e^{s_3 \cdot\left(T_{leaf}-T_{d m}\right)}} \cdot \beta
\end{equation}
其中$T_{dm}$是叶呼吸的高温抑制温度常数,单位 K。

在求解最小值的计算中,我们将三值最小问题的求解拆分为两个二值最小问题的求解:
\begin{equation}\label{min_Ac_Aj_Ae}
\min \left(A_{c}, A_{j}, A_{e}\right)=\min \left(\min \left(A_{c}, A_{j}\right), A_{e}\right)
\end{equation}
引入形状参数$\theta$,构造一元二次方程,将求解最小值问题转换成求一元二次方程较小根的问题,以此避免模拟中不同光合限制之间过渡转换时的光合同化速率突变 \citep{collatz1991,collatz1992}:
\begin{equation}\label{theta_cj}
\theta_{c j} \cdot A_{i1}^{2}-\left(A_{c}+A_{j}\right) A_{i1}+A_{c} A_{j}=0
\end{equation}
\begin{equation}\label{theta_cje}
\theta_{c j e} \cdot A_{i2}^{2}-\left(A_{i1}+A_{e}\right) A_{i2}+A_{i1} A_{e}=0
\end{equation}
其中形状参数$\theta_{cj}=0.877$,$\theta_{cje}=0.95$。$A_{i1}$为方程(\ref{theta_cj})的较小根,代表$A_c$和$A_j$的最小值。$A_{i2}$为方程(\ref{theta_cje})的较小根,代表$A_{i1}$和$A_e$的最小值。


\section{气体扩散方程和气孔导度模型}\label{气体扩散方程和气孔导度模型}
气孔是植被和大气相互作用的最重要通道,陆地生态系统碳水耦合很大程度取决于气孔的行为。CoLM在的光合同化速率和蒸腾速率计算中,考虑植被气孔的行为,结合气体扩散方程,刻画了碳水通量与浓度梯度之间的重要关系~\eqref{A_n2},~\eqref{ea_ei}。CoLM将CO$_2$和水汽的气体扩散问题类比于电路问题进行建模,利用叶绿体细胞间、叶表和冠层大气的$\rm CO_2$浓度和水汽浓度梯度,来刻画环境对植被碳水通量的驱动力,如图~\ref{fig:叶片气孔光合作用导度模型示意图}:

{
\begin{figure}[htbp]
\centering
\includegraphics{Figures/气孔导度和光合作用/叶片气孔光合作用导度模型示意图.png}
\caption{叶片气孔光合作用导度模型示意图。}
\label{fig:叶片气孔光合作用导度模型示意图}
\end{figure}
}



\begin{equation}\label{A_n2}
A_{n}=\left(c_{a}-c_{s}\right) /\left(\frac{1.37}{g_{b}} p_{s}\right)=\left(c_{s}-c_{i}\right) /\left(\frac{1.6}{g_{s}} p_{s}\right)
\end{equation}
\begin{equation}\label{ea_ei}
\left(e_{a}-e_{i}\right) /\left(\frac{1}{g_{b}}+\frac{1}{g_{s}}\right)=\left(e_{s}-e_{i}\right) / \frac{1}{g_{s}}
\end{equation}
其中,$c_i$是胞间CO$_2$分压,$c_s$是叶表CO$_2$分压,$c_a$是冠层大气$\rm CO_2$分压,叶表水汽分压$e_s$,$e_a$是冠层大气水汽分压,$e_i$是胞间水汽分压,单位: Pa,$g_b$是叶片边界层导度,$g_s$是叶片气孔导度,单位: \unit{mol.m^{-2}.s^{-1}};单位: Pa。

气孔导度理论认为植被根据自身光合同化速率、环境大气水分亏缺以及土壤水分胁迫等因素,主动调控气孔导度。CoLM的气孔导度计算沿用Ball-Berry模型,Ball-Berry模型根据叶片净光合同化速率 
($A_n$,单位: \unit{mol.CO_2.m^{-2}.s^{-1}})、叶表水汽压 ($e_s$,单位: Pa)、叶表二氧化碳分压 ($c_s$,单位: Pa) 
基于气孔导度的观测回归经验关系,计算气孔导度 ($g_s$, 单位: \unit{mol.CO_2.m^{-2}.s^{-1}}): 
\begin{equation}\label{rs_a1}
\frac{1}{r_{s}}=g_{s}=m \frac{A_{n}}{c_{s}} \frac{e_{s}}{e_{i}} p_{s}+b\beta
\end{equation}
$r_s$代表叶片气孔阻抗,单位: \unit{s.m^{-1}},是气孔导度$g_s$的倒数;$m$是无量纲经验参数;$b$是最小气孔导度,
单位: \unit{mol.CO_2.m^{-2}.s^{-1}},$m$和$b$是观测拟合的经验系数;$e_i$是饱和水蒸气压,
是气温的函数,单位: Pa;$p_s$是大气压强,单位: Pa,$\beta$是植物水分胁迫因子,取值范围0$\sim$1。


气孔行为的预测需要综合考虑植被自身的光合能力,大气$\rm CO_2$浓度梯度,水汽浓度梯度和土壤水分胁迫,因此,需要联立光合作用模块方程、气体扩散方程,气孔导度模型和土壤水分胁迫方案,求解叶片气孔导度。
由方程~\eqref{A_n2} 可知:
\begin{equation}\label{cs_a1}
c_{s}=c_{a}-\frac{1.37 A_{n}}{g_{b}} p_{s}
\end{equation}
由方程~\eqref{ea_ei} 可知:
\begin{equation}\label{e_s1}
e_{s}=\left(\frac{e_{a}}{g_{s}}+\frac{e_{i}}{g_{b}}\right) /\left(\frac{1}{g_{b}}+\frac{1}{g_{s}}\right)
\end{equation}
将方程~\eqref{e_s1} 代入方程~\eqref{rs_a1}中,得到关于$g_s$的一元二次方程:
\begin{equation}\label{ei_cs}
\frac{e_{i} c_{s}}{m A_{n} p_{s}} g_{s}^{2}+\left(g_{b} \frac{e_{i} c_{s}}{m A_{n} p_{s}}-e_{i}-b \beta \frac{e_{i} c_{s}}{m A_{n} p_{s}}\right) g_{s}
-\left(e_{a} g_{b}+b \beta g_{b} \frac{e_{i} c_{s}}{m A_{n} p_{s}}\right)=0
\end{equation}
气孔导度 ($g_s$) 的解即为一元二次方程的正根,其中叶片表层$\mathrm{CO_2}$分压 ($c_s$) 由方程~\eqref{cs_a1} 得出,$A_n$由光合作用模块公式~\eqref{An1} 得出,
但仍然包含未知变量胞间 $\mathrm{CO_2}$ 分压 ($c_i$),完整求解光合气孔模式还需根据~\eqref{A_n2} 得出:
\begin{equation}\label{ci_1}
c_{i}=c_{s}-\frac{1.6 A_{n} p_{s}}{g_{s}}
\end{equation}
联立~\eqref{An1}, \eqref{cs_a1}, \eqref{ei_cs} 和 \eqref{ci_1} 可以求解 $g_s$,$c_i$,$c_s$ 和 $A_n$。
通过牛顿迭代数值解法 (章节~\ref{数值计算方案}),对胞间$\mathrm{CO_2}$分压 $c_i$ 求解,从而求解所有未知量。

\chapter{植被水力模式}
%\addcontentsline{toc}{chapter}{植被水力模式}

%\begin{植被水力模式}
CoLM植被水力模式根据土壤-植物-大气连通体的概念计算陆气水分交换的蒸腾分量。
CoLM植被水力模式中的植物水分传输得益于土壤、根、茎、叶和大气之间形成的水势梯度。植物各部位的水势也密切影响着各植被水力过程。
首先,根、茎、叶的水势通过植物栓塞过程影响着植物水分传输能力 (详见章节~\ref{植被水力导度的衰减})。
其次,植物根与每层土壤的水势梯度将影响根的水力重分配过程 (详见章节~\ref{地下植被水力过程})。最后,叶片的水势降低将影响叶片气孔导度,
同时影响植物水分传输和光合作用 (详见章节~\ref{气孔导度的水分胁迫})。


\section{植物水势动态}\label{植物水势动态}
植物水势的动态模拟是CoLM植被水力模式的关键。CoLM植被水力模式的水势模拟包括地上和地下植被水力过程两部分。
地上部分包括在阳叶、阴叶、茎和地表根四个节点间的水势 ($\Psi_{sun}$,$\Psi_{sha}$,$\Psi_{stem}$,$\Psi_{root,0}$)
%
 模拟;地下部分包括地表根以及$n$层地下根 $n+1$个节点的水势 ($\Psi_{root,i}$,$i=0,1,2,\ldots,n$) 模拟。地上、地下两部分通过地表根水势
 ($\Psi_{root,0}$)的模拟被密切的耦合(如图~\ref{fig:CoLM植被水力模型示意图})。

{
\begin{figure}[htb]
\centering
\includegraphics{Figures/植被水力模式/CoLM植被水力模型示意图.png}
\caption{CoLM植被水力模型示意图}
\label{fig:CoLM植被水力模型示意图}
\end{figure}
}


CoLM植物水势的动态模拟假设植物水分传输为稳恒流,并将其类比为电路问题~
\citep{van1948water},即水分传输速率正比于水势差和水力导度。
地上部分四个节点的水势梯度和水通量之间的关系由Darcy定律表达,满足以下方程:
\begin{equation}\label{q_sunstem}
q_{ {sun \leftarrow stem }}=k_{{sun} \leftarrow  {stem}}\left(\Psi_{sun}-\Psi_{stem}\right)
\end{equation}
\begin{equation}
q_{ {sha \leftarrow stem }}=k_{ {sha} \leftarrow {stem}}\left(\Psi_{sha}-\Psi_{ {stem }}\right)
\end{equation}
\begin{equation}
q_{ {stem \leftarrow root }}=k_{ {stem } \leftarrow  { root }}\left(\Psi_{ {stem }}-\Psi_{ {root }, 0}\right)
\end{equation}
$k_{sun \leftarrow stem}$,$k_{sha \leftarrow stem }$,$k_{stem \leftarrow root }$ 分别代表茎到阳叶的水力导度、茎到阴叶的水力导度和根到茎的水力导度。
水力导度随水势降低而降低,是关于茎和根水势的函数 (详见章节~\ref{植被水力导度的衰减}) 。地表根水势$\Psi_{root,0}$是关于每层土壤水势 ($\Psi_{soil,i}$) 
及根总吸水速率 ($q_{root,0}$) 的函数,由地下植被水力过程所计算 (详见章节~\ref{地下植被水力过程}):
\begin{equation}\label{Psi_root_0}
\Psi_{root, 0}=R\left(\Psi_{ {soil }, i}, q_{root, 0}\right)
\end{equation}
$q_{sun \leftarrow stem}$,$q_{sha \leftarrow stem }$,$q_{stem \leftarrow root }$分别代表茎到阳叶的水流通量、茎到阴叶的水流通量和根到茎的水流通量。
水流通量在各节点由于稳恒流假设,满足水分守恒方程:
\begin{equation}
E_{sun}=q_{sun \leftarrow  stem}
\end{equation}
\begin{equation}
E_{ {sha }}=q_{ sha \leftarrow stem}
\end{equation}
\begin{equation}
q_{ {sun \leftarrow stem }}+q_{ {sha \leftarrow stem }}=q_{ {stem \leftarrow root }}
\end{equation}
\begin{equation}\label{q_stemroot}
q_{stem \leftarrow root}=q_{root, 0}
\end{equation}
$E_{sun}$代表阳叶蒸腾速率,$E_{sha}$代表阴叶蒸腾速率。它们是由无水分胁迫情况下的叶片蒸腾 ($E_{sun,max}$, $E_{sha,max}$) 
和受叶片水势 ($\Psi_{sun}$,$\Psi_{sha}$) 控制的蒸腾衰减函数所组成(详见章节~\ref{植被水力导度的衰减})。


\section{植被水力导度的衰减}\label{植被水力导度的衰减}
植物水势下降导致的空穴现象会严重降低水分在植物内部的传导能力。我们将实验上观测到水力导度随水势变化的S型脆弱性曲线 
\citep{sperry1988method,gentine2016allometry,neufeld1992genotypic,pammenter1998mathematical,plaut2012hydraulic}
 引入到模型,对植被水力栓塞进行参数化,得到节点间的水力导度$k_{i\gets j}$ (从节点$j$到$i$的传输):
\begin{equation}
k_{i \leftarrow j}=k_{max} \cdot 2^{-\left(\frac{\Psi_{\mathbf{j}}}{p 50}\right)^{c_{k}}}
\end{equation}
其中$k_{max}$表示最大水力导度 (\unit{s^{-1}})。$\Psi_j$代表节点$j$的水势 (\unit{mm.H_2O}),$p50$代表水力导度降低 50\% 时的水势 (\unit{mm.H_2O}),$c_k$代表脆弱性曲线的形状参数。


实验数据发现气孔导度和叶片水势同样存在S型曲线关系 \citep{kennedy2019implementing,klein2014variability},我们用如下方程~\eqref{Esun_psi} 和~\eqref{Esha_psi} 表达叶片水势降低对阳叶和阴叶的实际蒸腾速率($E_{sun}$和$E_{sha}$)的影响:
\begin{equation}\label{Esun_psi}
E_{sun}=E_{sun,max} \cdot 2^{-\left(\frac{\Psi_{\mathbf{sunleaf}}}{p 50}\right)^{c_{k}}}
\end{equation}
\begin{equation}\label{Esha_psi}
E_{sha}=E_{sha,max} \cdot 2^{-\left(\frac{\Psi_{\mathbf{shaleaf}}}{p 50}\right)^{c_{k}}}
\end{equation}
其中,$E_{sun,max}$和$E_{sha,max}$是无水分胁迫条件下的蒸腾速率,由最大气孔导度($g_{s,max}^{sun}$, $g_{s,max}^{sha}$)结合地表湍流参数化方案算出,最大气孔导度由CoLM光合气孔耦合模型($FvCB$)在无水分胁迫条件下计算得出 (最大蒸腾速率和最大气孔导度的计算详见章节~\ref{气孔导度的水分胁迫})。


\section{地下植被水力过程}\label{地下植被水力过程}
地下植被水力过程主要体现在水力重分配过程上,水力重分配描述了水分通过地下根系从湿润土壤层向干燥土壤层传输的过程。
是一个受水势梯度所驱动的生物物理过程。CoLM引入了Amenu水力重分配模型\citep{amenu2008,zhu2017incorporating},
并将地下植被水力过程和地上植被水力过程相耦合~\citep{li2021new},考虑刻画除根、茎和叶等地上节点外的
$n$个不同土壤深度的根系水力节点的水势动态变化 (如图~\ref{fig:CoLM植被水力模型示意图})。模式中的水力传导包括轴向水力传导和径向水力传导。

根据Darcy定律,根轴向水力传导方程为:
\begin{equation}\label{k_axi}
k_{ax,i}\left(\Psi_{r,i}-\Psi_{r,i+1}\right)=q_{ax,i}
\end{equation}
其中$k_{ax,i}$代表第$i+1$层到第i层根节点间的根轴向水力导度,$q_{ax,i}$代表相应的轴向水传输速率,
$\Psi_{r,i}$代表第$i$层根节点的水势。$i$取值1到$n-1$,$n$代表土壤总层数。

根径向水力传导方程为:
\begin{equation}\label{k_radi}
k_{rad,i}\left(\Psi_{soil,i}-\Psi_{r,i}\right)=q_{rad,i}
\end{equation}
其中$k_{rad,i}$代表第$i$层土壤节点到根节点间的径向水力导度,
$q_{rad,i}$代表相应的径向水分传输速率,$\Psi_{soil,i}$代表第$i$层土壤节点的水势,$i$取值1到$n$。



对于第2到$n$层的土壤根节点存在水分平衡方程:
\begin{equation}\label{q_axi}
q_{a x, i}+q_{r a d, i}=q_{a x, i-1}
\end{equation}
其中$i=2, \ldots, n$,方程~\eqref{q_axi} 实际上是$n-1$个方程组。


另外,由表层根节点和地上植物水力网络构成的水分平衡关系,可得:
\begin{equation}\label{q_ax1}
q_{ax,1}+q_{rad, 1}=q_{root,0}
\end{equation}
将方程~\eqref{k_axi} 和~\eqref{k_radi} 代入~\eqref{q_axi} 和~\eqref{q_ax1},则得到关于$ \Psi_{r,i}$的$n$个线性方程组。
这$n$个线性方程组描述了在植物地下水力网络中,根水势的动态变化。CoLM地下植被水力过程的输入包括土壤水势垂直分布 ($\Psi_{soil,i}$)和根总吸水速率 ($q_{root,0}$),输出包括根水势垂直分布($\Psi_{r,i}$)和根吸水速率的垂直分布 ($q_{rad,i}$)。因此,地下植被水力过程可以概括地表达为土壤水势和总吸水速率的函数,如方程~\eqref{Psi_root_0}。


\section{气孔导度的水分胁迫}\label{气孔导度的水分胁迫}
土壤水分亏缺对植物的影响主要表现在对植被生理过程的限制,通常用水分胁迫因子来量化,CoLM中的水分胁迫因子作为取值 $0\sim 1$ 的无量纲变量,限制气孔导度和光合羧化能力,见公式~\eqref{V_cmaxsun_a}--\eqref{R_d1_sha}。

植被水力模式关闭时(土壤水分胁迫方案SMS),水分胁迫因子 ($\beta$) 取决于根的垂直分布比例和土壤水势:
\begin{equation}\label{beta_0}
\beta=\sum_{i=1}^{n} froot_i \beta_{i}
\end{equation}
%
\begin{equation}\label{beta_i}
\beta_{i}=\frac{\Psi_{\max }-\Psi i}{\Psi_{\max }-\Psi_{s a t}}
\end{equation}
$froot_i$是第i层土壤中根的比例,$\beta_i$是第$i$层土壤的水分胁迫因子贡献;它由每层土壤的实际水势 (${\Psi}_i$),最大水势 (${\Psi}_{max}$)和饱和水势 (${\Psi}_{sat}$)计算得出。

当植被水力模式开启时(植被水力方案PHS),植物叶水势($\Psi_{\mathbf{sunleaf}}$和$\Psi_{\mathbf{shaleaf}}$)的降低直接造成植物水分胁迫,从而导致气孔关闭以保持植物茎的导水能力。
气孔这一行为符合植物生理学理论,水分胁迫因子按阴叶和阳叶区分,由气孔导度($\beta_{sun}$,$\beta_{sha}$)的衰减比例来定义:
\begin{equation}\label{beta_sun}
\beta_{sun}=\frac{g_{s}^{sun}}{g_{s,max}^{sun}}
\end{equation}
\begin{equation}\label{beta_sha}
\beta_{sha}=\frac{g_{s}^{sha}}{g_{s,max}^{sha}}
\end{equation}
$g_{s}^{sun}$和$g_{s}^{sha}$分别为阳叶和阴叶实际的气孔导度,$g_{s,max}^{sun}$和$g_{s,max}^{sha}$分别为阳叶和阴叶无水分胁迫条件下的气孔导度。其计算从无水分胁迫条件出发($\beta_{sun}=1$, $\beta_{sha}=1$),应用光和气孔耦合模型($FvCB$),计算得出无水分胁迫条件下的气孔导度。光合气孔耦合模型($FvCB$)由气孔导度经验模型~\eqref{ei_cssun} 和~\eqref{ei_cssha},气体扩散方程~\eqref{ci_1sun} 和~\eqref{ci_1sha},和Farquhar光合作用模型~\eqref{An1sun} 和~\eqref{An1sha} 联立而成,气孔导度是该模型重要的输出变量:
\begin{equation}\label{gs_sunmax}
g_{s,max}^{sun}=FvCB\left(\beta_{sun}=1\right)
\end{equation}
\begin{equation}\label{gs_shamax}
g_{s,max}^{sha}=FvCB\left(\beta_{sha}=1\right)
\end{equation}


相应地,最大蒸腾速率($E_{sun,max}$和$E_{sha,max}$)可由最大气孔导度结合地表冠层湍流参数化方案~\eqref{Ev} 在阳叶和阴叶上分别计算得出 (如图~\ref{fig:光合气孔模式和植被水力模式的耦合示意图}):

\begin{equation}\label{E_sunmax}
E_{sun,max}=\rho_{atm} c_{vsun,max}^{w} \cdot \frac{c_{a}^{w} q_{atm}+c_{g}^{w} q_{g}-
\left(c_{a}^{w}+c_{g}^{w}\right) q_{s a t}^{T_{v}}}{c_{a}^{w}+c_{g}^{w}+c_{vsun,max}^{w}+c_{vsha,max}^{w}}
\end{equation}
%
\begin{equation}\label{Eshamax}
E_{sha,max}=\rho_{atm} c_{vsha,max}^{w} \cdot \frac{c_{a}^{w} q_{atm}+c_{g}^{w} q_{g}-
\left(c_{a}^{w}+c_{g}^{w}\right) q_{s a t}^{T_{v}}}{c_{a}^{w}+c_{g}^{w}+c_{vsun,max}^{w}+c_{vsha,max}^{w}}
\end{equation}


$cf$为导度的单位转换系数(\unit{umol.m^{-3}}),将\unit{umol.m^{-2}.s^{-1}}转换为\unit{m.s^{-1}}。$c_{vsun,max}^{w}$和$c_{vsha,max}^{w}$是阳叶和阴叶的最大冠层导度,即无植被水分胁迫条件下的冠层导度,可表达为最大气孔导度的函数,如公式~\eqref{cvsun_gsmax} 和~\eqref{cvsha_gsmax}:
\begin{equation}\label{cvsun_gsmax}
    c_{vsun,max}^{w}=\left(1-f_{wet}\right)\frac{f_{sun}\text{LAI}}{cf\left(\frac{1}{g_b}+\frac{1}{g_{s,max}^{sun}}\right)}
\end{equation}
%
\begin{equation}\label{cvsha_gsmax}
    c_{vsha,max}^{w}=\left(1-f_{wet}\right)\frac{f_{sha}\text{LAI}}{cf\left(\frac{1}{g_b}+\frac{1}{g_{s,max}^{sha}}\right)}
\end{equation}

结合叶水势对蒸腾的影响~\eqref{Esun_psi} 和~\eqref{Esha_psi},可计算得到的实际蒸腾速率$E_{sun}$和$E_{sha}$,从地表冠层湍流参数化方案反推,可得到干阳叶和干阴叶的冠层导度 ($c_{vsun}^{w}$和$c_{vsha}^{w}$)。如公式~\eqref{cv_Esun} 和~\eqref{cv_Esha}:
\begin{equation}\label{cv_Esun}c_{vsun}^{w}=\frac{E_{sun}\left(c_a^w+c_g^w+c_{v,wet}^{w}\right)}{\left(\left(c_a^w+c_g^w\right)q_{sat}-c_a^w q_m - c_g^w q_g\right)\rho-E_{sun}-E_{sha}}
\end{equation}
%
\begin{equation}\label{cv_Esha}
c_{vsha}^{w}=\frac{E_{sha}\left(c_a^w+c_g^w+c_{v,wet}^{w}\right)}{\left(\left(c_a^w+c_g^w\right)q_{sat}-c_a^w q_m - c_g^w q_g\right)\rho_{atm}-E_{sun}-E_{sha}}
\end{equation}
其中,$c_{v,wet}^{w}$是湿叶蒸发导度,
%如公式~\eqref{cwet}
\begin{equation}\label{cwet}
c_{v,wet}=\frac{f_{wet}\left(\text{LAI}+\text{SAI}\right)g_b}{cf}
\end{equation}
从干阳叶和干阴叶的冠层导度和气体扩散方程可进一步算出叶片气孔导度:\textcolor{red}{待补充}
%\begin{equation}\label{gssun_cvsun}
%
%\end{equation}

耦合植被水力模式、光合气孔模式和地表冠层湍流参数化方案,再利用公式~\eqref{beta_sun} 和~\eqref{beta_sha} 可以完整地求解植被水力模式下的植物水分胁迫 (如图~\ref{fig:光合气孔模式和植被水力模式的耦合示意图})。
{
    \begin{figure}[htbp]
    \centering
    \includegraphics{Figures/植被水力模式/光合气孔模式和植被水力模式的耦合示意图.png}
    \caption{光合气孔模式和植被水力模式的耦合示意图}
    \label{fig:光合气孔模式和植被水力模式的耦合示意图}
    \end{figure}
}


\section{数值计算方案}\label{数值计算方案}
求解植物水势、水分胁迫、气孔导度和叶片蒸腾速率,需要耦合植被水力模式、光合气孔模式和地表冠层参数化方案。
耦合描述植物水势变化的动力方程组(公式~\eqref{q_sunstem}--\eqref{q_stemroot}, \eqref{Esun_psi} 和~\eqref{Esha_psi}),
地表冠层参数化方案 (公式~\eqref{E_sunmax}--\eqref{cvsha_gsmax}),
光合气孔模式的最大气孔导度计算 (公式~\eqref{gs_sunmax} 和~\eqref{gs_shamax}), 植物水分胁迫计算 (公式~\eqref{beta_sun} 和~\eqref{beta_sha}),共18个方程,
求解包括4个植物地上水势节点 ($\Psi_{sunleaf}$, $\Psi_{shaleaf}$, $\Psi_{stem}$和$\Psi_{root,0}$),8个水分传输速率 
($q_{sun-stem}$,$q_{sha-stem}$,$q_{stem-root}$,$q_{root,0}$,$E_{sun}$,$E_{sun,max}$和$E_{sun,max}$) ,4个气孔导度变量
 ($g_{s,sun}$,$g_{s,sun}$,$g_{s,sun,max}$,$g_{s,sun,max}$),2个水分胁迫变量 ($\beta_{sun}$,$\beta_{sha}$) 在内的共18个未知量。

然而,由于以上18个方程中存在隐形方程,我们求解该问题时,引入三重嵌套数值求解办法。其主要步骤如下(图~\ref{fig:植被水力模式的数值模拟流程图}):
\begin{enumerate}
    \item 利用冠层模型,计算叶片温度 ($T_{l,sun}$,$T_{l,sha}$);
    \item 利用光合气孔模式,计算最大气孔导度 ($g_{s,sun,max}$和$g_{s,sun,max}$);
    \item 根据最大气孔导度,计算叶片最大蒸腾速率 ($E_{sun,max}$和$E_{sun,max}$);
    \item 将叶片最大蒸腾速率输入植被水力模式,计算水分胁迫 ($\beta_{sun}$,$\beta_{sha}$);
    \item 更新光合气孔模式的水分胁迫,迭代计算,判断胞间二氧化碳浓度是否收敛,若收敛,进入第 (6) 步;若不收敛,重复此步骤;
    \item 更新气孔导度,判断阴叶、阳叶水分胁迫是否收敛,若收敛,进入第 (7) 步,若不收敛,回到第 (2) 步;
    \item 更新植物水势和叶片蒸腾,判断叶片温度是否收敛,若收敛,植被水力模式求解完成,若不收敛,回到第 (1) 步。
\end{enumerate}

气孔导度详细迭代

{
    \begin{figure}[htbp]
    \centering
    \includegraphics{Figures/植被水力模式/植被水力模式的数值模拟流程图.png}
    \caption{植被水力模式的数值模拟流程图}
    \label{fig:植被水力模式的数值模拟流程图}
    \end{figure}
}


\chapter{降水与地表的能量交换}
%\addcontentsline{toc}{chapter}{降水与地表的能量交换}
降水是陆面与大气之间保证水循环以及维持能量平衡的重要过程,CoLM模式通过多个物理参数化方案考虑降水相态分配、水凝物密度计算、雨水与植被/地表之间的潜热感热交换等物理过程。

\begin{mymdframed}{代码}
本部分的对应代码为\texttt{MOD\_GroundTemperature.F90}、\texttt{MOD\_RainSnowTemp.F90}和\\
\texttt{MOD\_WetBulb.F90}。
\end{mymdframed}

\section{降水相态分配}\label{相态分配}
在CoLM中共有四种基于温、湿参量近似雨水温度分配降水相态的方法,具体如表~\ref{tab:相态区分方案}~所示。

\begin{table}[]
\caption{相态区分方案}
\label{tab:相态区分方案}
\begin{tabular}[h]{p{1cm}p{5cm}p{10cm}}
\toprule
选项 & 注释(*代表的是默认方案) & 参考文献 \\\midrule
I & 湿球温度方案      & \citet{Wang-etal_19WetBulb}   \\
II*  & CLM5方案      & Ben Andre et al., 2020, CLM5 Documentation      \\
III  & PSY方案     & \citet{harder2013estimating}     \\
IV  & 单阈值方案     & \citet{us1956snow}     \\    \bottomrule

\end{tabular}
\end{table}

\subsection{湿球温度方案}
已有研究(如\citet{anderson1998moored})表明,
雨水温度非常接近于大气湿球温度,故该方案中雨水温度由湿球温度近似,根据湿球温度定义有如下关系:
\begin{equation}
C_{p a}\left(T_{w b}-T\right)=\lambda_{v}\left(r-r_{s a t}^{T_{w b}}\right)
\end{equation}
其中 $C_{pa}$ 表示干空气的比热容 (\unit{J.kg^{-1}.K^{-1}}),$T_{wb}$ 表示大气湿球温度 (K),
$T$ 表示大气环境温度 (K),$r$ 表示混合比,$r_{sat}^{T_{wb}}$ 表示 $T_{wb}$ 温度下的饱和混合比,
$\lambda_v$ 表示液态水的蒸发潜热 (\unit{J.kg^{-1}})。于是,取 $T_{wb}$ 的初始值 $T_{wb}^{\left(0\right)}=T$,
$T_{wb}$ 可通过如下过程进行迭代求解:
\begin{enumerate}
    \item 由附录~\ref{饱和水汽压(比湿)及其随温度的变化} 计算$T_{wb}^{\left(n\right)}$温度下的饱和比湿$q_{sat}^{T_{wb}^{\left(n\right)}}$;
    \item 通过混合比和比湿的换算关系($r=\frac{q}{1-q}$),计算混合比$r$与$r_{sat}^{T_{wb}^{\left(n\right)}}$;
    \item 由上述关系更新湿球温度:$T_{wb}^{\left(n\right)\ast}=T+\frac{\lambda_v}{C_{pa}}\left(r-r_{sat}^{T_{wb}^{\left(n\right)}}\right)$;
    \item 取更新前后湿球温度的平均值作为新一步的湿球温度:$T_{wb}^{\left(n+1\right)}=\left(T_{wb}^{\left(n\right)}+T_{wb}^{\left(n\right)\ast}\right)/2.0$。
\end{enumerate}
将上述过程迭代6次($n=0$, $\ldots$, 5),作为最终的湿球温度$T_{wb}$,再用其近似雨水温度界定液态水占比,具体如下:

\begin{equation}
f_{pl}= \begin{cases}
1, & \text{当 }\ 3\le T_{wb} - T_{frz} \text{ 时} \\
1 - \frac{1}{1 + 5e-5\exp{(T_{wb} - T_{frz}+4)}} , & \text{当 }\ -2\le T_{wb} - T_{frz} < 3 \text{ 时} \\
0, & \text{当 }\ T_{wb} - T_{frz} < -2 \text{ 时}
\end{cases}
\end{equation}
%
其中$T_{frz}$ = 273.15 K.


\subsection{CLM5方案}
CLM5方案假定大气环境温度为雨水温度,其区别于其他方法的地方在于方案会判断下垫面类型是否为冰川,并对冰川这个下垫面类型做单独的考虑,实现的逻辑具体如下。

当地表水体为冰面时(land water type = 3, land ice):
\begin{equation}
Allsnow_{tc} = T_{frz} - 2,\quad Allrain_{tc} = T_{frz}
\end{equation}

其他下垫面情况:
\begin{equation}
Allsnow_{tc} = T_{frz},\quad Allrain_{tc} = T_{frz} +2
\end{equation}

\begin{equation*}
f_{pl}= \begin{cases}
1, & \text{当 }\ Allrain_{tc}\le forc_t \text{ 时}\\
(forc_t - Allsnow_{tc})/2, & \text{当 }\ Allsnow_{tc}\le forc_t < Allrain_{tc} \text{ 时} \\
0, & \text{当 }\ forc_t < Allsnow_{tc} \text{ 时}
\end{cases}
\end{equation*}


\subsection{PSY方案}
PSY (Psychrometric Energy Balance Method)方案假设:
\begin{enumerate}
    \item 在无限接近水凝物(液态水或冰晶)的表面有一理想气团,其温度与水凝物一致,且满足Tetents饱和水汽压经验公式;
    \item 理想气团与环境大气之间满足质量守恒以及能量守恒且两者都满足理想气体方程;
    \item 水凝物的凝结(凝华)潜热以及空气导热率与环境温度(\textcelsius)满足一定函数关系。
\end{enumerate}

综上,方案基于假设,计算出水凝物表面空气温度以代表水凝物温度(\textcelsius),具体如下:
\begin{equation}
L\frac{dm}{dt}=4\pi C\lambda_t(T_i - T)
\end{equation}

\begin{equation}
\frac{dm}{dt}=4\pi CD_v(\rho - \rho_i)
\end{equation}

简化后:
\begin{equation}
T_i = T + \frac{D_v}{\lambda_{t}}L(\rho - \rho_i)
\end{equation}
%
其中$L$为凝结或凝华潜热,$C$为水凝物形状因子,$\lambda_t$为空气导热率因子,$T_i$为水凝物表面空气温度,$T$为环境温度,$D_v$为水汽扩散系数,$\rho$为环境空气密度,$\rho_i$为水凝物表面空气密度,以上方程可通过牛顿迭代求解,得出水凝物表面空气温度,相较于(1.1)湿球温度的等焓假设,计算(1.4)水凝物温度时并未做出类似假设,两者的物理基础不同,计算方程不能相互转化。

具体的液态水占比如下:
\begin{equation*}
f_{pl}= \begin{cases}
1, & \text{当 }\ 5\le T_i \text{ 时}\\
\frac{1}{1 + 2.50286\times 0.125006^{T_{i}}}, & \text{当 }\-5\le T_i < 5 \text{ 时} \\
0, & \text{当 }\ T_i < -5 \text{ 时}
\end{cases}
\end{equation*}
%
上式的Sigmoid函数拟合有别于湿球温度方案

PSY方案的拟合数据来源于降水量:
\begin{equation}
f_{pl} = \frac{Rainfall(mm)}{Rainfall(mm) + Snowfall(mm)}
\end{equation}
%
而湿球温度方案的拟合数据源于降水事件数:
\begin{equation}
f_{pl} = \frac{Rain_{event}}{Rain_{event} + Snow_{event}}
\end{equation}


\subsection{单阈值方案}
单阈值方案就是设定一个关键温度以界定液态降水的占比,假设认为液态降水占比随温度升高逐渐线性增加,在2 \textcelsius 时占40\%,当环境温度高于2 \textcelsius 后全转变为液态降水,过程具体如下:
\begin{equation*}
f_{pl}= \begin{cases}
1, & \text{当 }\ T_{frz}+2\le forc_t \text{ 时}\\
\max(0,-54.632 + 0.2\times forc_t), & \text{当 }\ forc_t < T_{frz} + 2 \text{ 时} \\
\end{cases}
\end{equation*}


\section{新雪密度}
雪的密度影响到地表的雪深、雪的堆积等一系列物理过程,是一个很重要的参量,在CoLM中采用CLM5的方案计算新雪密度,具体过程如下:
\begin{equation*}
Bifall= \begin{cases}
50 + 1.7\times 17^{1.5}, & \text{当 }\ 2\le forc_t - T_{frz} \text{ 时}\\
50 + 1.7(forc_t - T_{frz}+15)^{1.5}, & \text{当 }\ -15\le forc_t - T_{frz} < 2 \text{ 时} \\
-3.8328(forc_t - T_{frz}) - 0.0333(forc_t - T_{frz})^2, & \text{当 }\ -57.55\le forc_t - T_{frz} < -15 \text{ 时} \\
110.2898, & \text{当 }\ forc_t - T_{frz} < -57.55 \text{ 时}\\
\end{cases}
\end{equation*}

当风速达到一定阈值时,雪密度在考虑温度后还需要继续考虑风速:
\begin{equation}
forc_{wind} = \sqrt{(u^2 + v^2)}
\end{equation}

当 $forc_{wind}$ > 0.1 时
\begin{equation}
Bifall_{new} = Bifall_{old} + \left\{266.81\left[1 + {\left(\frac{\tanh (\frac{forc_{wind}}{5})}{2}\right)}\right]^{8.8}\right\}
\end{equation}


\section{植被/地表的雨水感热}\label{植被地表的雨水感热}
雨水与地表/植被之间存在温度差异,当降水发生时,它们之间可发生能量交换(雨水感热)。
\citet{wei2014impact} 表明,虽然在气候尺度上此能量交换的量级很小,但它可通过改变大气环流影响不同地区的气候特征,
故在气候模式中不可被忽略。感热计算中默认湿球温度为雨水温度,下面给出雨水感热的计算方案。

当地表有植被覆盖时,植被叶片与雨水的能量交换 (\unit{W.m^{-2}}) 分别计算如下:
\begin{equation}
H_{prcv}=C_{pl} q_{pl}\left(T_{p}-T_{v}\right)+C_{pi} q_{pi}\left(T_{p}-T_{v}\right)
\end{equation}
%
其中$C_{pl}$与$C_{pi}$分别表示液态水与固态水的比热容(\unit{J.kg^{-1}.K^{-1}}),
$q_{pl}$与$q_{pi}$分别表示植被冠层对液态水和固态水的截流率(\unit{mm.H_2O.s^{-1}})(计算见章节~\ref{植被冠层截留})。


地表与雨水的能量交换 (\unit{W.m^{-2}}) 为:
\begin{equation}
H_{p r c g}=C_{p l} p_{l}\left(T_{p}-T_{g}\right)+C_{p i} p_{i}\left(T_{p}-T_{g}\right)
\end{equation}

其中$p_l$与$p_i$分别表示落到地面的液态水与固态水降水率(\unit{mm.H_2O.s^{-1}}),
当有植被覆盖时,$p_l$与$p_i$为直接穿透植被冠层的降水率与沿叶茎流出冠层的降水率之和(计算见章节~\ref{植被冠层截留})。












\part{植被冠层、积雪和土壤温度计算方案}{Canopy, Snow and Soil Thermal Schemes}\label{part:temp}
%\epart{Canopy, Snow and Soil Thermal Schemes}
\chapter{植被叶片温度计算}\label{植被叶片温度计算}
%\addcontentsline{toc}{chapter}{植被叶片温度计算}

%\begin{植被叶片温度计算}
假设植被冠层的比热容为0,则叶片能量平衡方程为:
\begin{equation}\label{FT_V}
F\left(T_{v}\right):=S_{v}+L_{v}\left(T_{v}\right)-H_{p v}\left(T_{v}\right)-\lambda_{v} E_{p v}\left(T_{v}\right)+H_{p r c v}\left(T_{v}\right)=0
\end{equation}
其中$S_v$表示叶片吸收的净太阳辐射(见章节 \ref{短波吸收辐射通量}),
$L_v$表示叶片吸收的净长波辐射。$T_v$可通过对方程 (\ref{FT_V}) 实施牛顿迭代法进行求解,迭代公式为:
\begin{equation}
\Delta T_{v}=-\frac{F\left(T_{v}^{(n)}\right)}{F^{\prime}\left(T_{v}^{(n)}\right)}
\end{equation}
其中$\Delta T_v=T_v^{\left(n+1\right)}-T_v^{\left(n\right)}$,$n$代表迭代次数。
此外,因为植被湍流通量与叶片温度相互耦合,故在温度迭代求解过程中,湍流通量也随之更新。


在能量平衡方程中,$S_v$不依赖于温度的变化,其表达式在第 \ref{短波吸收辐射通量} 节中已经给出。
$L_v$可做如下表达(假设植被和被植被覆盖的地表的长波辐射发射率为1):
\begin{equation}
L_{v}=f_{sig} *\left(1-\mu_{t h}\right)\left(L \downarrow+\varepsilon_{g} \sigma T_{g}^{4}-2 \sigma T_{v}^{4}\right)
\end{equation}
即植被吸收的净长波辐射等于植被吸收的来自大气和地表的长波辐射减去植被向大气和地表方向同时发出的长波辐射,
其中$L\downarrow$表示近地面大气下行长波辐射,$\varepsilon_g=0.96$表示未被植被覆盖的地表的长波辐射发射率,
$\mu_{th}$表示长波辐射直接穿过植被冠层的比例(见章节~\ref{长波净辐射通量})。$L_v$对叶片温度的变化率如下:
\begin{equation}
\begin{array}{c}\frac{\partial L_{v}}{\partial T_{v}}=-8 f_{sig} *\left(1-\mu_{t h}\right)\left(\sigma T_{v}^{3}\right) \\ \frac{\partial L_{v, s u n}}{\partial T_{v, s h a}}=0\end{array}
\end{equation}
感热通量表达式$H_{pv}$已在~\ref{一维植被湍流交换模型} 节给出,其对温度的变化率为:
\begin{equation}
\frac{\partial H_{p v}}{\partial T_{v}}=f_{sig} \rho_{atm} C_{pa} c_{v}^{h}
\end{equation}
对于水汽通量$E_{pv}$,虽然其表达式也已在 \ref{一维植被湍流交换模型} 节给出,但这里为了表达上的便利,
我们引入对于植被表面蒸散发是否发生的判别因子$\delta$如下:
\begin{equation}
\delta=\left\{\begin{array}{cc}1 & \text { 当 } q_{s a t}^{T_{v}}>q_{s} \text { 即阳叶蒸散发可以发生时 } \\ 0 & \text { 否则 }\end{array}\right.
\end{equation}
于是$c_v^w$可重新表达为如下形式:
\begin{equation}
c_{v}^{w}=\frac{1}{r_{total}}=\frac{\left[1-\delta\left(1-f_{wet}\right)\right](LAI+SAI)}{r_{b}}+\frac{\delta LAI\left(1-f_{wet}\right)}{r_{b}+r_{s}}
\end{equation}
由于蒸发与蒸腾率均受到可利用水分的限制,所以$E_{pv}$将分别针对蒸发与蒸腾作用进行调整:\\
对于叶片的蒸腾通量:
\begin{equation}
\begin{array}{c}E_{pvt}=-\frac{f_{sig} \rho_{atm}}{c_{a}^{w}+c_{g}^{w}+c_{v}^{w}} \frac{\delta LAI\left(1-f_{w e t}\right)}{r_{b}+r_{s}}\left[c_{a}^{w} q_{atm}+c_{g}^{w} q_{g}-\left(c_{a}^{w}+c_{g}^{w}+c_{v}^{w}\right) q_{s a t}^{T_{v}}\right] \\ \frac{\partial E_{pvt}}{\partial T_{v}}=\frac{f_{sig} \rho_{atm} \delta LAI\left(1-f_{w e t}\right)}{r_{b}+r_{s}} \frac{c_{a}^{w}+c_{g}^{w}}{c_{a}^{w}+c_{g}^{w}+c_{v}^{w}} \frac{d q_{s a t}^{T_{v}}}{d T_{v}}\end{array}
\end{equation}
当$ E_{pvt} \geq f_{sig}\ast E_{pvt,max} $即叶片最大蒸腾率($\rm kg/m^2/s$)时,
\begin{equation}
\begin{array}{c}E_{pvt}=f_{sig} * E_{pvt, \max } \\ \frac{\partial E_{pvt}}{\partial T_{v}}=0\end{array}
\end{equation}
对于叶片的蒸发通量:
\begin{equation}
\begin{array}{c}E_{pva}=-\frac{f_{\text {sig }} \rho_{atm}}{c_{a}^{w}+c_{g}^{w}+c_{v}^{w}} \frac{\left[1-\delta\left(1-f_{w e t}\right)\right](LAI+SAI)}{r_{b}}\left[c_{a}^{w} q_{atm}+c_{g}^{w} q_{g}\right. \\ \left.-\left(c_{a}^{w}+c_{g}^{w}+c_{v}^{w}\right) q_{s a t}^{T_{v}}\right] \\ \frac{\partial E_{pva}}{\partial T_{v}}=\frac{f_{\text {sig }} \rho_{atm}\left[1-\delta\left(1-f_{w e t}\right)\right](LAI+SAI)}{r_{b}} \frac{c_{a}^{w}+c_{g}^{w}}{c_{a}^{w}+c_{g}^{w}+c_{v}^{w}} \frac{d q_{s a t}^{T_{v}}}{d T_{v}}\end{array}
\end{equation}
当$E_{pva} \geq \frac{W_{can}}{\Delta t} \frac{LAI+SAI}{LAI+SAI}$即叶片最大蒸发率($\rm kg/m^2/s$)时,
\begin{equation}
\begin{array}{c}E_{pva}=\frac{W_{can}}{\Delta t} \\ \frac{\partial E_{pva}}{\partial T_{v}}=0\end{array}
\end{equation}
于是对于叶片的总水汽通量:
\begin{equation}
\begin{array}{l}E_{pv}=E_{pvt}+E_{pva} \\ 
    \frac{\partial E_{pv}}{\partial T_{v}}=\frac{\partial E_{pvt}}{\partial T_{v}}+\frac{\partial E_{pva}}{\partial T_{v}}\end{array}
\end{equation}
雨水感热表达式$H_{prcv}$已在 \ref{植被地表的雨水感热} 节给出,其对温度的变化率为:
\begin{equation}
\frac{\partial H_{prcv}}{\partial T_{v}}=-\left[C_{p l} q_{p l}+C_{p i} q_{p i}\right]
\end{equation}
基于以上表达式以及$T_v$的迭代公式,下面给出$T_v$以及植被湍流通量的求解流程:
\begin{enumerate}
    \item 给出植被冠层空气温度和比湿的初始猜测:$T_s=\frac{T_g+\theta_{atm}}{2}$,$q_s=\frac{q_g+q_{atm}}{2}$;
    \item 给出$U_c$的初始猜测如下:\\
    \begin{equation*}
    U_c= \begin{cases}
      0,  & \theta_{v,atm}-\theta_{v,s}\geq0 \text{ 即稳定条件下;} \\
      0.5, & \theta_{v,atm}-\theta_{v,s}<0 \text{ 即不稳定条件下;}
     \end{cases}
    \end{equation*}
    \item 通过$R_{ib}$给出$L$的初始猜测;
    \item 迭代以下过程以求得$T_v$以及植被湍流通量:\\
    a. 通过风速、温度、比湿的微分方程积分结果求得$u_\ast$、$\theta_\ast$、$q_\ast$ \\
    b. 计算植被冠层空气与大气之间的阻抗系数$r_{am}$、$r_{ah}$、$r_{aw}$ \\
    c. 计算叶面边界层阻抗$r_b$ \\
    d. 计算植被冠层空气与地表之间的阻抗系数$r_{ah}^\prime$、$r_{aw}^\prime$ \\
    e. 计算气孔阻抗$r_s$ \\
    f. 分别计算叶片吸收的长波辐射、感热通量、潜热通量和雨水感热$L_v$、$H_{pv}$、$\lambda_vE_{pv}$、$H_{prcv}$ \\
    g. 若前后两次迭代过程中潜热通量的符号发生变化($\lambda_vE_{pv}^{\left(n\right)}\times\lambda_vE_{pv}^{\left(n+1\right)}<0$),
    则在该次迭代计算温度时,潜热通量的量级限制为原量级的10\%,由此产生的能量差最后将加到感热通量中 \\
    h. 计算温度变化$\Delta T_v$,并由此更新$T_v^{\left(n+1\right)}=\Delta T_v^{\left(n\right)}+T_v^{\left(n\right)}$。在每次迭代过程中,对于温度的变化作出如下两个限制:
    (1)温度的变化不得超过1K,若超过,则强制其变化只有1K;
    (2)若本次迭代温度变化的方向与上一次变化的方向相反,则本次温度的变化将取为两次变化的平均值(若$\Delta T_v^{\left(n-1\right)} \cdot \Delta T_v^{\left(n\right)}<0$,则$\Delta T_v^{\left(n\right)}=\left(\Delta T_v^{\left(n-1\right)}+\Delta T_v^{\left(n\right)}\right)/2$)。\\
    由温度调整所带来的能量平衡误差最后将加到感热通量中;\\
    i. 更新饱和比湿$q_{sat}^{T_v}$及其对$T_v$的变化率 \\
    j. 更新植被冠层空气温度和比湿$T_s$, $q_s$ \\ 
    k. 更新特征位温$\theta_\ast$和特征比湿$q_\ast$ \\
    l. 更新特征虚位温$\theta_{v\ast}$ \\
    m. 更新大气风速$V_a\left(U_c\right)$ \\
    n. 计算新一步$L$,并计算$\zeta$,根据稳定性条件限制$\zeta$的取值范围 \\
    o. 根据限制条件后的$\zeta$重新计算$L=\frac{z_{atm,m}-d}{\zeta}$ \\
    p. 判断$L$与上一步迭代相比是否改变符号,若改变符号累计超过4次,则视为中性条件,
    $L$取固定值$L=\frac{z_{atm,m}-d}{\left(-0.01\right)}$,以避免在稳定与不稳定条件之间来回变化。\\
    q. 判断迭代停止条件:若迭代过程中满足下列全部条件或迭代次数已超过40次,则迭代停止:
    \begin{equation}
        \begin{array}{c}\max\left( \sqrt{\left[F^{(n+1)}+G^{(n+1)}-F^{(n)}-G^{(n)}\right]^{\ast\ast2}}, \sqrt{\left[F^{(n)}+G^{(n)}-F^{(n-1)}-G^{(n-1)}\right]^{\ast\ast2}} \right) \\ \leq 0.1 \\ 
        \max\left( \sqrt{\left(\Delta T_{v}^{(n)}\right)^{2}}, \sqrt{\left(\Delta T_{v}^{(n-1)}\right)^{2}} \right) \leq 0.01\end{array}
    \end{equation}
    其中$\left[\bullet\right]^{\ast\ast2}$表示各个相同能量项合并后的平方和;
    \item 由最终叶片温度更新植被表面与植被冠层空气之间的潜热通量,其中蒸发率不得超过最大蒸发率$W_{can} \Delta t$,蒸腾率不得超过最大蒸腾率$f_{sig}\ast E_{pvt,max}$,
    若超过则蒸发(腾)率强制取为最大蒸发(腾)率,由此产生的能量差最后将加到感热通量中。
    \item 由最终叶片温度更新植被表面与植被冠层空气之间的感热通量以及上述因为潜热与温度的调整导致的能量误差之和。
    \item 由最终叶片温度更新植被冠层的雨水感热。
    \item 计算总动量通量为
    \begin{equation}
    \begin{array}{c}\tau_{x}=\tau_{b,x}-f_{sig} \rho_{atm} \frac{u_{atm}}{r_{a m}} \\ 
        \tau_{y}=\tau_{b,y}-f_{sig} \rho_{atm} \frac{v_{atm}}{r_{am}}\end{array}
    \end{equation}
    \item 计算有植被覆盖下的地表感热通量$H_{pg}$和潜热通量$\lambda_{vE_{pg}}$及其对地表温度的变化率,
    并给出地表总感热通量$H_g$和潜热通量$\lambda_{vE_g}$:
    \begin{equation}
    \begin{aligned} H_{g} &=H_{b}+H_{p g} \\ \lambda_{vE_{g}} &=\lambda_{vE_{b}}+\lambda_{vE_{pg}} \end{aligned}
    \end{equation}
    \item 计算植被覆盖下地表吸收的下行长波辐射$L_{pg}\downarrow$和返回大气的上行长波辐射$L_p\uparrow$:
    \begin{equation}
    \begin{array}{c}L_{pg} \downarrow=f_{sig} \mu_{th} L \downarrow+f_{sig} \cdot \left(1-\mu_{t h}\right) \sigma T_{v}^{4} \\
         L_{p} \uparrow=f_{sig} \mu_{t h} \varepsilon_{g} \sigma T_{g}^{4}+f_{sig} *\left(1-\mu_{t h}\right) \sigma T_{v}^{4}\end{array}
    \end{equation}
    \item 计算2m温度与比湿$T_{2m}$、$q_{2m}$。
\end{enumerate}

\chapter{雪盖土壤热力过程}
%\addcontentsline{toc}{chapter}{雪盖土壤热力过程}

%\begin{雪盖土壤热力过程}

在雪盖和土壤中,热力过程主要根据热传导第二定律进行描述。假设雪盖土壤无水平物质能量交换,则其垂直方向上的一维能量平衡方程如下:
\begin{equation}\label{eq:1d_energy_balance}
  c \frac{\partial T}{\partial t}=-\frac{\partial F}{\partial z}
\end{equation}
其中$c$表示雪盖或土壤的体积热容量(\unit{J.m^{-3}.K^{-1}}),$T$表示雪盖土壤温度 (K),$t$表示时间(s),$z$表示雪盖高度或土壤深度 (m),
F表示垂直方向的热传导通量,方向向上为正(\unit{W.m^{-2}}),表达式为$F=-\lambda\frac{\partial T}{\partial z}$,$\lambda$表示热力传导率(\unit{W.m^{-1}.K^{-1}})。
将方程(\ref{eq:1d_energy_balance})进行离散,并结合上边界由大气输送到雪盖土壤表面的热通量$h_{\mathrm {s}} $以及下边界土壤底层的零热通量,
即可求出雪盖土壤的温度廓线,之后再根据相态变化条件对温度进行进一步调整。


\section{温度求解的数值格式}\label{温度求解的数值格式}

\begin{mymdframed}{代码}
  本节对应的代码文件为\texttt{MOD\_GroundTemperature.F90}。
\end{mymdframed}

在离散上述能量平衡方程时,垂直方向上的离散方式可采用第~\ref{土壤和积雪的垂直分层} 节给出的土壤和积雪的垂直分层方案进行。
需要额外注意的是,在计算雪盖土壤温度之前,若有降雪发生($p_{\mathrm {i}} >0$)且无雪盖分层($snl=0$),则此时需通过下式判断第一层雪能否形成:
\begin{equation}
  \begin{aligned}
    z_{\mathrm{sno}} &= z_{\mathrm{sno}}+\frac{p_{\mathrm{i}} \Delta t}{\rho_{\text {sno,new }}} \\[1ex]
    W_{\mathrm{sno}} &= W_{\mathrm{sno}}+p_{\mathrm{i}} \Delta t
  \end{aligned}
\end{equation}
其中$\rho_{\mathrm{sno,new}}$表示新降的干雪密度(\unit{kg.m^{-3}}),计算方案为~\citep{anderson1976point}:
\begin{equation}
  \rho_{\mathrm{sno, new}}=\begin{cases}
    169 & \text { 当 }\ T_{\mathrm{a}}>T_{\mathrm{frz}}+2.0 \\
    50+1.7\left(T_{\mathrm{a}}-T_{\mathrm{frz}}+15\right)^{1.5}  & \text { 当 }\ T_{\mathrm{frz}}-15.0<T_{\mathrm{a}} \leqslant T_{\mathrm{frz}}+2.0 \\
    50 & \text { 当 }\ T_{\mathrm{a}} \leqslant T_{\mathrm{frz}}-15.0
  \end{cases}
\end{equation}
若此时$z_{\mathrm{sno}} \geqslant 0.01$,则按第~\ref{土壤和积雪的垂直分层} 节方案对积雪进行分层,且每一层雪的温度取为$T_{\mathrm {p}} $,液态水含量取为0,固态水含量按雪层厚度权重分配$W_{\mathrm{sno}}$。若之前已有雪层且此时有降雪发生,则第一层雪的相关物理量作如下更新:
\begin{equation}
  w_{\mathrm{ice},snl+1}=w_{\mathrm{ice},snl+1}+p_{\mathrm{i}} \Delta t
\end{equation}
\begin{equation}
  \Delta z_{snl+1}=\Delta z_{snl+1}+\frac{p_{\mathrm{i}} \Delta t}{\rho_{\mathrm{sno, new}}}
\end{equation}
\begin{equation}
  z_{snl+1}=z_{\mathrm{h},snl+1}-0.5 \Delta z_{snl+1}
\end{equation}
\begin{equation}
  z_{\mathrm{h, snl}}=z_{\mathrm{h},snl+1}-\Delta z_{snl+1}
\end{equation}
雪盖土壤垂直分层及其热传导过程示意图见图~\ref{fig:雪盖土壤垂直分层及其热传导过程示意图}。其中土壤热力学状态变量
(如雪盖土壤温度$T_{i} $,热力传导率$\lambda_i$,体积比热容$c_i$等)定义在每一层的中间深度,从第$i+1$层到第$i$层的热传导通量$F_i$定义在两层的交界处,它可离散为:
\begin{equation}
  F_{i}=\lambda\left[z_{\mathrm{h},i}\right] \frac{T_{i}-T_{i+1}}{z_{i}-z_{i+1}}
\end{equation}
其中$\lambda\left[z_{\mathrm{h},i}\right]$表示第$i+1$层与第 $i$ 层交界处的热力传导率。求解$\lambda\left[z_{\mathrm{h},i}\right]$时,
假设从第$i+1$层到第$i$层的热传导通量等于从第$i+1$层到第$i+1$层与第$i$层交界层的热传导通量,又等于从交接层到第$i$层的热传导通量,即
\begin{equation}
  \lambda\left[z_{\mathrm{h},i}\right] \frac{T_{i}-T_{i+1}}{z_{i}-z_{i+1}}=\lambda_{i+1} \frac{T_{\mathrm{m}}-T_{i+1}}{z_{\mathrm{h},i}-z_{i+1}}=\lambda_{i} \frac{T_{i}-T_{\mathrm{m}}}{z_{i}-z_{\mathrm{h},i}}
\end{equation}
即可通过后两项解出第$i+1$层与第$i$层交界处的温度$T_{\mathrm {m}} $;再将$T_{\mathrm {m}} $代回,通过前两项即可解出$\lambda\left[z_{\mathrm{h},i}\right]$为:
\begin{equation}
  \lambda\left[z_{\mathrm{h},i}\right]=\begin{cases}
    \frac{\lambda_{i} \lambda_{i+1}\left(z_{i}-z_{i+1}\right)}{\lambda_{i}\left(z_{\mathrm{h},i}-z_{i+1}\right)+\lambda_{i+1}\left(z_{i}-z_{\mathrm{h},i}\right)} & i=snl+1, \ldots, 9 \\
    0 & i=10
  \end{cases}
\end{equation}
特殊地,对于土壤与雪盖的交界面,为防止最下层雪盖厚度过大导致$\lambda\left[z_{\mathrm{h},i}\right]$计算不准,
当$i=0$且$\left(z_{i+1}-z_{\mathrm{h},i}\right)<\left(z_{\mathrm{h},i}-z_i\right)$时,$\lambda\left[z_{\mathrm{h},i}\right]$重新计算为:
\begin{equation}
  \lambda\left[z_{\mathrm{h},i}\right]=\frac{2 \lambda_{i} \lambda_{i+1}}{\lambda_{i}+\lambda_{i+1}} \geqslant 0.5 \lambda_{i+1}
\end{equation}
{
  \begin{figure}[htbp]
    \centering
    \includegraphics{Figures/雪盖土壤热力过程/雪盖土壤垂直分层及其热传导过程示意图.png}
    \caption{雪盖土壤垂直分层及其热传导过程示意图}
    \label{fig:雪盖土壤垂直分层及其热传导过程示意图}
  \end{figure}
}


基于以上离散方案,第$i$层雪盖土壤的能量平衡方程可表达为:
\begin{equation}
  \frac{c_{i} \Delta z_{i}}{\Delta t}\left(T_{i}^{n+1}-T_{i}^{n}\right)=F_{i}-F_{i-1}
\end{equation}
%
其中$\Delta t$表示积分时间步长,$n$表示积分步数。此方程采用Crank--Nicholson半隐式格式求解,既包含前一时刻已有的温度与热通量信息,又包含后一时刻的预报信息。于是此方程可写为如下形式:
\begin{equation}
  \frac{c_{i} \Delta z_{i}}{\Delta t}\left(T_{i}^{n+1}-T_{i}^{n}\right)=\alpha\left(F_{i}^{n}-F_{i-1}^{n}\right)+(1-\alpha)\left(F_{i}^{n+1}-F_{i-1}^{n+1}\right)
\end{equation}
其中权重因子$\alpha=0.5$。此方程展开,即有
\begin{equation}
  \begin{aligned}
    \frac{c_{i} \Delta z_{i}}{\Delta t}\left(T_{i}^{n+1}-T_{i}^{n}\right)=& 0.5\left\{\lambda\left[z_{\mathrm{h},i}\right] \frac{T_{i}^{n}-T_{i+1}^{n}}{z_{i}-z_{i+1}}-\lambda\left[z_{\mathrm{h},i-1}\right] \frac{T_{i-1}^{n}-T_{i}^{n}}{z_{i-1}-z_{i}}\right.\\[1ex] &\left.+\lambda\left[z_{\mathrm{h},i}\right] \frac{T_{i}^{n+1}-T_{i+1}^{n+1}}{z_{i}-z_{i+1}}-\lambda\left[z_{\mathrm{h},i-1}\right] \frac{T_{i-1}^{n+1}-T_{i}^{n+1}}{z_{i-1}-z_{i}}\right\}
  \end{aligned}
\end{equation}
将所有层雪盖土壤能量平衡方程联立,可形成关于预报变量$T_{i-1}^{n+1}$,$T_i^{n+1}$和$T_{i+1}^{n+1}$的三对角方程组形式:$r_i=a_iT_{i-1}^{n+1}+b_iT_i^{n+1}+c_iT_{i+1}^{n+1}$,
其中$a_i$, $b_i$, $c_i$分别为三对角矩阵中上三角、对角线和下三角位置中的元素。用追赶法解此方程组即可快速求得每一层雪盖土壤的温度$T_i^{n+1}$。


对于雪盖土壤的中间层($snl+1<i<10$),三对角矩阵中的系数表达如下:
\begin{equation}
  \begin{aligned}
    a_{i} &= -(1-\alpha) \frac{\Delta t}{c_{i} \Delta z_{i}} \frac{\lambda\left[z_{\mathrm{h},i-1}\right]}{z_{i}-z_{i-1}} \\[1ex]
    b_{i} &= 1+(1-\alpha) \frac{\Delta t}{c_{i} \Delta z_{i}}\left[\frac{\lambda\left[z_{\mathrm{h},i-1}\right]}{z_{i}-z_{i-1}}+\frac{\lambda\left[z_{\mathrm{h},i}\right]}{z_{i+1}-z_{i}}\right] \\[1ex]
    c_{i} &= -(1-\alpha) \frac{\Delta t}{c_{i} \Delta z_{i}} \frac{\lambda\left[z_{\mathrm{h},i}\right]}{z_{i+1}-z_{i}} \\[1ex]
    r_{i} &= T_{i}^{n}+\alpha \frac{\Delta t}{c_{i} \Delta z_{i}} \lambda\left[z_{\mathrm{h},i}\right] \frac{T_{i}^{n}-T_{i+1}^{n}}{z_{i}-z_{i+1}}-\lambda\left[z_{\mathrm{h},i-1}\right] \frac{T_{i-1}^{n}-T_{i}^{n}}{z_{i-1}-z_{i}}
  \end{aligned}
\end{equation}

对于雪盖土壤的顶层和底层,需要考虑对应的边界条件:

(1) 对于雪盖土壤顶层($i=snl+1$),来自大气的热通量$h_{\mathrm {s}} $将会进入到地表中,即
\begin{equation}
  h_{\mathrm{s}}^{n+1}=-\alpha F_{i-1}^{n}-(1-\alpha) F_{i-1}^{n+1}
\end{equation}
对$h_{\mathrm {s}} ^{n+1}$采用一阶泰勒展开近似,则顶层能量平衡方程变为:
\begin{equation}
  \begin{split}
    &\mathrel{\phantom{\approx}}\frac{c_{i} \Delta z_{i}}{\Delta t}\left(T_{i}^{n+1}-T_{i}^{n}\right)=h_{\mathrm{s}}^{n+1}+\alpha F_{i}^{n}+(1-\alpha) F_{i}^{n+1} \\[1ex]
    &\approx h_{\mathrm{s}}^{n}+\frac{\partial h_{\mathrm{s}}}{\partial T_{i}}\left(T_{i}^{n+1}-T_{i}^{n}\right)+\alpha \lambda\left[z_{\mathrm{h},i}\right] \frac{T_{i}^{n}-T_{i+1}^{n}}{z_{i}-z_{i+1}}+(1-\alpha) \lambda\left[z_{\mathrm{h},i}\right] \frac{T_{i}^{n+1}-T_{i+1}^{n+1}}{z_{i}-z_{i+1}}
  \end{split}
\end{equation}
于是,基于此方程可得雪盖土壤顶层的三对角矩阵系数为:
\begin{equation}
  \begin{aligned}
    a_{i} &= 0 \\[1ex]
    b_{i} &= 1+\frac{\Delta t}{c_{i} \Delta z_{i}}\left[(1-\alpha) \frac{\lambda\left[z_{\mathrm{h},i}\right]}{z_{i+1}-z_{i}}-\frac{\partial h_{\mathrm{s}}}{\partial T_{i}}\right] \\[1ex]
    c_{i} &= -(1-\alpha) \frac{\Delta t}{c_{i} \Delta z_{i}} \frac{\lambda\left[z_{\mathrm{h},i}\right]}{z_{i+1}-z_{i}} \\[1ex]
    r_{i} &= T_{i}^{n}+\frac{\Delta t}{c_{i} \Delta z_{i}}\left[h_{\mathrm{s}}^{n}-\frac{\partial h_{\mathrm{s}}}{\partial T_{i}} T_{i}^{n}+\alpha \lambda\left[z_{\mathrm{h},i}\right] \frac{T_{i}^{n}-T_{i+1}^{n}}{z_{i}-z_{i+1}}\right]
  \end{aligned}
\end{equation}
这里进入到地表的热通量$h_{\mathrm {s}} $ (\unit{W.m^{-2}})具体表达为:
\begin{equation}
  \begin{aligned}
    h_{\mathrm{s}} &= S_{\mathrm{g}}+L_{\mathrm{g}}\left(T_{\mathrm{g}}\right)-H_{\mathrm{g}}\left(T_{\mathrm{g}}\right)-\lambda E_{\mathrm{g}}\left(T_{\mathrm{g}}\right)+H_{\mathrm{p r c g}}\left(T_{\mathrm{g}}\right) \\[1.5ex]
    \frac{\partial h_{\mathrm{s}}}{\partial T_{\mathrm{g}}} &= \frac{\partial L_{\mathrm{g}}}{\partial T_{\mathrm{g}}}-\frac{\partial H_{\mathrm{g}}}{\partial T_{\mathrm{g}}}-\frac{\partial \lambda E_{\mathrm{g}}}{\partial T_{\mathrm{g}}}+\frac{\partial H_{\mathrm{p r c g}}}{\partial T_{\mathrm{g}}}
  \end{aligned}
\end{equation}
其中$S_{\mathrm {g}} $和$L_{\mathrm {g}} $分别表示地表吸收的净太阳辐(公式~\eqref{eq:sg})和净长波辐射(公式~\eqref{eq:lg1}, \eqref{eq:lg2}) [\unit{W.m^{-2}}]。$h_{\mathrm {s}} $中感热$H_{\mathrm{g}}$和潜热$\lambda E_{\mathrm {g}} $的计算见第~\ref{ch:地表湍流通量} 章,雨水感热$H_{\mathrm{prcg}}$的计算见章节~\ref{植被地表的雨水感热}。

另外,对于潜热通量系数$\lambda$,当雪盖土壤顶层不存在液态水时,$\lambda$取为升华潜热系数$\lambda_{\mathrm {sub}} $,即:
\begin{equation}
  \lambda=\left\{\begin{array}{lr}\lambda_{\mathrm{sub}} & \text { 当 }\ w_{\mathrm{liq},snl+1}=0 \text { 并且 }\ w_{\mathrm{ice},snl+1}>0 \\ \lambda_{\mathrm{vap}} & \text { 其他情况 }\end{array}\right.
\end{equation}
其中$w_{\mathrm{liq},snl+1}$和$w_{\mathrm{ice},snl+1}$分别表示雪盖土壤顶层的液态水和固态水含量 (\unit{kg.m^{-2}},其计算见第~\ref{积雪和土壤中水分的垂直运动}~章)。

在模式中,地表温度$T_{\mathrm {g}} $与$T_{snl+1}$取为同一值,但$T_{snl+1}$表示第一层雪盖或土壤的平均温度,与实际的$T_{\mathrm {g}} $相比具有减弱的日变化幅度。 为改进这一缺陷,在求解第一层能量平衡方程时,其厚度$\Delta z_{i} $作以下调整,以求得更接近实际的$T_{\mathrm {g}}$:
\begin{equation}
\Delta z_{i}=0.5\left[z_{i}-z_{\mathrm{h},i-1}+c_{\mathrm{a}}\left(z_{i+1}-z_{\mathrm{h},i-1}\right)\right]
\end{equation}
其中调整参数$c_{\mathrm {a}} =0.34$。

(2) 对于土壤底层($i=10$),假设热传导通量为0,则能量平衡方程变为:
\begin{equation}
  \frac{c_{i} \Delta z_{i}}{\Delta t}\left(T_{i}^{n+1}-T_{i}^{n}\right)=-\alpha \lambda\left[z_{\mathrm{h},i-1}\right] \frac{T_{i-1}^{n}-T_{i}^{n}}{z_{i-1}-z_{i}}-(1-\alpha) \lambda\left[z_{\mathrm{h},i-1}\right] \frac{T_{i-1}^{n+1}-T_{i}^{n+1}}{z_{i-1}-z_{i}}
\end{equation}
于是,基于此方程可得土壤底层的三对角矩阵系数为:
\begin{equation}
  \begin{aligned}
    a_{i} &= -(1-\alpha) \frac{\Delta t}{c_{i} \Delta z_{i}} \frac{\lambda\left[z_{\mathrm{h},i-1}\right]}{z_{i}-z_{i-1}} \\
    b_{i} &= 1+(1-\alpha) \frac{\Delta t}{c_{i} \Delta z_{i}} \frac{\lambda\left[z_{\mathrm{h},i-1}\right]}{z_{i}-z_{i-1}} \\
    c_{i} &= 0 \\
    r_{i} &= T_{i}^{n}-\alpha \frac{\Delta t \lambda\left[z_{\mathrm{h},i-1}\right]}{c_{i} \Delta z_{i}} \frac{T_{i-1}^{n}-T_{i}^{n}}{z_{i-1}-z_{i}}
  \end{aligned}
\end{equation}

各层雪盖土壤的热力传导率与体积热容量的计算方案将在~\ref{sec_thermalpar} 节给出。


\section{温度的相态变化调整}\label{sec:温度的相态变化调整}

\begin{mymdframed}{代码}
  本节对应的代码文件为\texttt{MOD\_PhaseChange.F90}。
\end{mymdframed}

通过解能量平衡方程组计算出下一时刻的雪盖土壤温度后,需要考虑相态变化过程对温度进行调整。在模式中,相态变化发生的条件为:\\
若$T_i^{n+1}>T_{\mathrm {frz}} $并且$w_{\mathrm{ice},i}>0$ \ \   \ \  \ \   固态水融化\\
若$T_i^{n+1}<T_{\mathrm {frz}} $并且$w_{\mathrm{liq},i}>0$  \ \   \ \  \ \         液态水冻结

一个特殊情况是,当土壤表面积雪存在 ($W_{\mathrm{sno}}>0$) 但并无雪层($snl=0$,$z_{\mathrm{sno}}<0.01$) 时,\\
若$T_1^{n+1}>T_{\mathrm {frz}} $      \ \   \ \  \ \                 积雪融化\\
%
以上三种情况下,$T_i^{n+1}$被调整为$T_{\mathrm {frz}} $。


相态变化的程度是由$T_{i} ^{n+1}$调整为$T_{\mathrm {frz}} $后产生的能量冗余或亏损决定的。温度调整后,能量的冗余或亏损$H_i$ (\unit{W.m^{-2}})计算如下:
\begin{equation}
  H_{i}=\begin{cases}
    h_{\mathrm{s}}^{n}+\frac{\partial h_{\mathrm{s}}}{\partial T_{i}}\left(T_{\mathrm{frz}}-T_{i}^{n}\right)+\alpha F_{i}^{n}+(1-\alpha) F_{i}^{n+1}-\frac{c_{i} \Delta z_{i}}{\Delta t}\left(T_{\mathrm{frz}}-T_{i}^{n}\right) & i=snl+1 \\
    \alpha\left(F_{i}^{n}-F_{i-1}^{n}\right)+(1-\alpha)\left(F_{i}^{n+1}-F_{i-1}^{n+1}\right)-\frac{c_{i} \Delta z_{i}}{\Delta t}\left(T_{\mathrm{frz}}-T_{i}^{n}\right) & snl+1<i<10 \\
    -\alpha F_{i-1}^{n}-(1-\alpha) F_{i-1}^{n+1}-\frac{c_{i} \Delta z_{i}}{\Delta t}\left(T_{f}-T_{i}^{n}\right) & i=10
  \end{cases}
\end{equation}
对应地,相态变化的质量调整量为$H_{\mathrm{m}}=\frac{H_{i} \Delta t}{\lambda_{\mathrm {fus}}}$,其中$\lambda_{\mathrm {fus}} $表示固态水液化潜热(\unit{J.kg^{-1}})。当固态水融化条件被满足且$H_{\mathrm {m}} >0$时,固态水含量被调整为:
\begin{equation}
  w_{\mathrm{ice},i}^{n+1}=w_{\mathrm{ice},i}^{n}-H_{\mathrm{m}} \geqslant 0
\end{equation}
当液态水冻结条件被满足且$H_{\mathrm {m}} <0$时,固态水含量被调整为:
\begin{equation}
  w_{\mathrm{ice},i}^{n+1}=\min{\left(w_{\mathrm{ice},i}^{n}-H_{\mathrm{m}}, w_{\mathrm{liq},i}^{n}+w_{\mathrm{ice},i}^{n}\right)}
\end{equation}
以上情况液态水含量均被调整为:
\begin{equation}
  w_{\mathrm{liq},i}^{n+1}=w_{\mathrm{liq},i}^{n}+w_{\mathrm{ice},i}^{n}-w_{\mathrm{ice},i}^{n+1} \geqslant 0
\end{equation}
若水分调节过程不足以消耗全部的能量冗余或填补全部的能量亏损,
则在相态变化过程之后再次进行能量结余计算:$ H_{i *}=H_{i}+\frac{\lambda_{\mathrm {fus}}\left(w_{\mathrm{ice},i}^{n+1}-w_{\mathrm{ice},i}^{n}\right)}{\Delta t}$。
若$\left|H_{i\ast}\right|>0$,则此部分能量可用来再次暖化或冷却雪盖土壤层,温度调整如下:
\begin{equation}
  T_{i}^{n+1}=\left\{\begin{array}{lr}T_{\mathrm{frz}}+\frac{\Delta t}{c_{i} \Delta z_{i}} H_{i *} /\left(1-\frac{\Delta t}{c_{i} \Delta z_{i}} \frac{\partial T_{\mathrm{s}}}{\partial T_{i}}\right) & i=s n l+1 \\ T_{\mathrm{frz}}+\frac{\Delta t}{c_{i} \Delta z_{i}} H_{i *} & i \neq s n l+1 \\ T_{\mathrm{f}} & w_{\mathrm{liq},i}^{n+1} \cdot w_{\mathrm{ice},i}^{n+1}>0\end{array}\right.
\end{equation}
对于特殊的土壤表面积雪融化的情形,当$H_{\mathrm {m}} >0$时,雪水当量减少为
\begin{equation}
  W_{\mathrm{sno}}^{n+1}=W_{\mathrm{s no}}^{n}-H_{\mathrm{m}} \geqslant 0
\end{equation}
同样,雪的厚度减少为
\begin{equation}
  z_{\mathrm{sno}}^{n+1}=\frac{W_{\mathrm{sno}}^{n+1}}{W_{\mathrm{sno}}^{n}} z_{\mathrm{sno}}^{n}
\end{equation}
若积雪融化不足以消耗全部的能量冗余,则剩余能量$H_{1\ast}$为
\begin{equation}
  H_{1 *}=H_{1}+\frac{\lambda_{\mathrm {fus}}\left(W_{\mathrm{sno}}^{n+1}-W_{\mathrm{sno}}^{n}\right)}{\Delta t}
\end{equation}
此时$H_{1\ast}$将作为新的$H_1$对第一层土壤进行上述相态变化计算。


综上,若发生土壤表面积雪融化的情形,则用于相态变化的能量累计$E$为:
\begin{equation}
  E=\frac{\lambda_{\mathrm {fus}}\left(W_{\mathrm{sno}}^{n}-W_{\mathrm{sno}}^{n+1}\right)}{\Delta t}+\sum_{i=1}^{10} \frac{\lambda_{\mathrm {fus}}\left(w_{\mathrm{ice},i}^{n}-w_{\mathrm{ice},i}^{n+1}\right)}{\Delta t}
\end{equation}
否则,
\begin{equation}
  E=\sum_{i=1}^{s n l+1} \frac{\lambda_{\mathrm {fus}}\left(w_{\mathrm{ice},i}^{n}-w_{\mathrm{ice},i}^{n+1}\right)}{\Delta t}
\end{equation}
以上过程即是雪盖土壤温度计算的全部过程。得到$T_i^{n+1}$后,地表发出的上行长波辐射$L_{\mathrm {g}} \uparrow$,感热通量$H_{\mathrm {g}} $与潜热通量$\lambda E_{\mathrm {g}} $需再做一次更新以作为输出的状态变量,
其中用于蒸发的水汽不能超过雪盖土壤顶层的总含水量$\left(w_{\mathrm{liq},snl+1}^{n+1}+w_{\mathrm{ice},snl+1}^{n+1}\right)/\Delta t$,
否则地表蒸发水汽取为$\left(w_{\mathrm{liq},snl+1}^{n+1}+w_{\mathrm{ice},snl+1}^{n+1}\right)/\Delta t$,产生的能量误差将加到感热通量上。


\section{雪盖土壤热力参数的计算方案}\label{sec_thermalpar}
\begin{mymdframed}{代码}
  本节对应的代码文件为\texttt{MOD\_SoilThermalParameters.F90}。
\end{mymdframed}

雪盖土壤的热力参数是完成雪盖土壤热力过程数值模拟、求解前文所述垂直方向能量平衡方程的必要输入参数,主要包含两类:雪盖土壤的体积热容量和导热率。下面就这两类参数给出在CoLM中的计算方案。

\subsection{雪盖土壤的体积热容量}

体积热容量的计算较为简单。根据 \citet{de1963thermal},土壤体积热容量由固体土壤、液态水和固态水的热容量基于各自的体积百分比加权平均得到,表达为:
\begin{equation}
  c_{i}=c_{\mathrm{s},i}\left(1-\theta_{\mathrm {sat},i}\right)+\frac{w_{\mathrm{ice},i}}{\Delta z_{i}} C_{\mathrm{ice}}+\frac{w_{\mathrm{liq},i}}{\Delta z_{i}} C_{\mathrm{p l}}
\end{equation}
其中$c_{\mathrm{s},i}$表示第$i$层的固体土壤体积热容量,由地表参数数据集提供。若地表有积雪但无雪层,则积雪的热容量也需考虑:
\begin{equation}
  c_{i}=c_{\mathrm{s},i}\left(1-\theta_{\mathrm {sat},i}\right)+\frac{w_{\mathrm{ice},i}}{\Delta z_{i}} C_{\mathrm{ice}}+\frac{w_{\mathrm{liq},i}}{\Delta z_{i}} C_{\mathrm{liq}}+\frac{W_{\mathrm{sno}}}{\Delta z_{i}} C_{\mathrm{ice}}
\end{equation}
其中,固体土壤假设由矿物质土壤、有机质土壤和砾石组成,每种成分的体积热容量基于各自的体积百分比加权平均即得到固体土壤的体积热容量。有机质土壤和砾石的体积热容量分别取值为$2.51\times 10^6$ \unit{J.m^{-3}.K^{-1}}和$2.35\times 10^6$ \unit{J.m^{-3}.K^{-1}},矿物质土壤的体积热容量(\unit{J.m^{-3}.K^{-1}})由如下方案给出:$$c_{\mathrm{minerals}}=\frac{2.128\times\%sand+2.385\times\%clay}{\%sand+\%clay}\times10^6$$
其中$\%sand$和$\%clay$表示沙土和黏土的质量百分比。

雪盖体积热容量由液态水和固态水的热容量基于各自的体积百分比加权平均得到,表达为:
\begin{equation}
  c_{i}=\frac{w_{\mathrm{ice},i}}{\Delta z_{i}} C_{\mathrm{ice}}+\frac{w_{\mathrm{liq},i}}{\Delta z_{i}} C_{\mathrm{liq}}
\end{equation}
其中,液态水和固态水的热容量可参见表~\ref{tab:物理常数}。

\subsection{雪盖土壤的导热率}

在CoLM中,土壤导热率(\unit{W.m^{-1}.K^{-1}})引入了8种计算方案供用户选择,分别为:\citet{Johansen1975}方案及其5个衍生方案(\citet{farouki1981thermal}, \citet{cote2005}, \citet{balland2005},\citet{lu2007improved}和~\citet{Yan2019thermal}),一个经验方案(\citet{tarnawski2012series})和一个理论方案(\citet{de1963thermal})。以上方案均在各自原始方案基础上,假设固体土壤由矿物质土壤、有机质土壤和砾石组成的前提下进行了改进。根据\citep{dai2019evaluation}在一定观测样本下的评估结果,\citet{balland2005}方案的模拟性能相对较优,故设为默认方案。下面将给出以上8种方案的具体计算方法。

\subsubsection{\citet{Johansen1975}方案}
\citet{Johansen1975}将土壤导热率$k$视为干土壤导热率$k_{\mathrm{dry}}$和饱和土壤导热率$k_{\mathrm{sat}}$的线性组合,其中饱和土壤导热率的权重系数为$K_{\mathrm {e}} $(称为Kersten数)。于是,土壤导热率可表达为:
\begin{equation}\label{eq:STC}
  k=(k_{\mathrm{sat}}-k_{\mathrm{dry}})K_{\mathrm {e}} +k_{\mathrm{dry}}
\end{equation}
其中,$K_{\mathrm {e}} $可视为土壤饱和度$S_{\mathrm {r}} $(定义为实际土壤含水量与饱和土壤含水量的比值)、土壤水相态和土壤粒径的函数,计算为:
\begin{equation}
  K_{\mathrm {e}} =\begin{cases}
    0.7\log_{10}S_{\mathrm {r}} +1 & \text {对于粗质颗粒的未冻结土壤且}\ Sr>0.05\ \text {时} \\
    \log_{10}S_{\mathrm {r}} +1 & \text {对于细质颗粒的未冻结土壤且}\ Sr>0.1\ \text {时} \\
    S_{\mathrm {r}}  & \text {对于冻结土壤时}
  \end{cases}
\end{equation}

在方程~\eqref{eq:STC} 中,干土壤导热率$k_{\mathrm{dry}}$由固体土壤各成分导热率基于其在固体土壤中的体积分数进行算术加权平均得到,计算表达式为:
\begin{equation}\label{eq:STC_dry}
  k_{\mathrm{dry}}=f_{\mathrm{minerals}}k_{\mathrm{minerals\_dry}}+f_{\mathrm{om}}k_{\mathrm{om\_dry}}+f_{\mathrm{gravel}}k_{\mathrm{gravel\_dry}}
\end{equation}
其中,$f_{\mathrm{minerals}}$,$f_{\mathrm{om}}$和$f_{\mathrm{gravel}}$分别表示矿物质土壤、有机质土壤和砾石在固体土壤中的体积分数;$k_{\mathrm{om\_dry}}$和$k_{\mathrm{gravel\_dry}}$分别表示有机质土壤和砾石在干燥状态下的整体导热率,取值为$k_{\mathrm{om\_dry}}=0.05$,$k_{\mathrm{gravel\_dry}}=0.039\theta_{\mathrm {sat}} ^{-2.2}$($\theta_{\mathrm {sat}} $表示土壤孔隙度);$k_{\mathrm{minerals\_dry}}$表示矿物质土壤在干燥状态下的导热率,计算为:$$k_{\mathrm{minerals\_dry}}=\frac{0.135\rho_{\mathrm {d}} +0.0647}{2.7-0.947\rho_{\mathrm {d}} }$$
$\rho_{\mathrm {d}} $表示矿物质土壤的干容重(\unit{g.cm^{-3}})。

饱和土壤导热率$k_{\mathrm{sat}}$由固体土壤各成分导热率和水(冰)导热率基于其体积分数进行几何加权平均得到,计算表达式为:
\begin{equation}\label{eq:STC_wet}
  k_{\mathrm{sat}}=k_{\mathrm{minerals\_wet}}^{v_{\mathrm {m}} }k_{\mathrm{om\_wet}}^{v_{\mathrm{om}}}k_{\mathrm{gravel\_wet}}^{v_{\mathrm {g}} }k_{\mathrm {w}} ^{\theta_{\mathrm {sat}} }
\end{equation}
其中,$v_{\mathrm {m}} $,$v_{\mathrm{om}}$和$v_{\mathrm {g}} $分别表示矿物质土壤、有机质土壤和砾石在整体土壤柱内的体积分数;$k_{\mathrm{minerals\_wet}}$,$k_{\mathrm{om\_wet}}$和$k_{\mathrm{gravel\_wet}}$分别表示矿物质土壤、有机质土壤和砾石在饱和土壤状态下的整体导热率,$k_{\mathrm {w}} $表示水(冰)的导热率(参见表~\ref{tab:物理常数})。$k_{\mathrm{om\_wet}}$和$k_{\mathrm{gravel\_wet}}$分别取值为0.25和2.875;$k_{\mathrm{minerals\_wet}}$由矿物质土壤中石英部分导热率$k_{\mathrm {q}} $与非石英部分导热率$k_{\mathrm {o}} $基于其在矿物质土壤中的体积分数进行几何加权平均得到:$$k_{\mathrm{minerals\_wet}}=k_{\mathrm {q}} ^{v_{\mathrm {q}} }k_{\mathrm {o}} ^{1-v_{\mathrm {q}} }$$
其中,$k_{\mathrm {q}} $取值为7.7,$k_{\mathrm {o}} $依据石英的体积分数$v_{\mathrm {q}} $决定:
\begin{equation}
  k_{\mathrm {o}} =\begin{cases}
    2.0 & \text {当}\ v_{\mathrm {q}} >0.2\ \text {时} \\
    3.0 & \text {当}\ v_{\mathrm {q}} \leqslant 0.2\ \text {时}
  \end{cases}
\end{equation}
$v_{\mathrm {q}} $可由~\citet{PL_98}基于12种土壤类型(依据美国农业部USDA的土壤质地分类标准)给出的经验值查表获得。

\subsubsection{\citet{farouki1981thermal}方案}
\citet{farouki1981thermal}方案作为~\citet{Johansen1975}方案的衍生方案,土壤导热率$k$与干土壤导热率$k_{\mathrm{dry}}$的计算方式完全一致(\eqref{eq:STC} 和~\eqref{eq:STC_dry})。饱和土壤导热率$k_{\mathrm{sat}}$由固体土壤导热率和水(冰)导热率基于其体积分数进行几何加权平均得到,而固体土壤导热率取值为各成分导热率基于其在固体土壤中的体积分数的算术加权平均值。$k_{\mathrm{sat}}$计算表达式为:
\begin{equation}\label{eq:STC_wet_Farouki}
  k_{\mathrm{sat}}=(f_{\mathrm{minerals}}k_{\mathrm{minerals\_wet}}+f_{\mathrm{om}}k_{\mathrm{om\_wet}}+f_{\mathrm{gravel}}k_{\mathrm{gravel\_wet}})^{1-\theta_{\mathrm {sat}} }k_{\mathrm {w}} ^{\theta_{\mathrm {sat}} }
\end{equation}
其中,$k_{\mathrm{minerals\_wet}}$也与~\citet{Johansen1975}方案的计算方式不同,表达为:$$k_{\mathrm{minerals\_wet}}=\frac{8.80\times\%sand+2.92\times\%clay}{\%sand+\%clay}$$
其中$\%sand$和$\%clay$表示沙土和黏土的质量百分比。

Kersten数$K_{\mathrm {e}} $的计算方式相较~\citet{Johansen1975}方案更为简化,表达为:
\begin{equation}
  K_{\mathrm {e}} =\begin{cases}
    \log_{10}S_{\mathrm {r}} +1 & \text {对于未冻结土壤时} \\
    S_{\mathrm {r}}  & \text {对于冻结土壤时}
  \end{cases}
\end{equation}

\subsubsection{\citet{cote2005}方案}
在~\citet{cote2005}方案中,土壤导热率$k$与饱和土壤导热率$k_{\mathrm{sat}}$的计算方式与~\citet{Johansen1975}方案完全一致 (\eqref{eq:STC} 和~\eqref{eq:STC_wet})。而干土壤导热率$k_{\mathrm{dry}}$和Kersten数$K_{\mathrm {e}} $的计算方案则通过大量实测数据拟合得到经验关系式。$k_{\mathrm{dry}}$表达为:
\begin{equation}\label{eq:STC_dry_CK}
  k_{\mathrm{dry}}=\sum_if_{i} \chi_{i} (10^{-\eta_i\theta_{\mathrm {sat}} }),\quad i=\text{矿物质土壤、有机质土壤或砾石}
\end{equation}
$K_{\mathrm {e}} $表达为:
\begin{equation}
  K_{\mathrm {e}} =\frac{\kappa S_{\mathrm {r}} }{1+(\kappa -1)S_{\mathrm {r}} }
\end{equation}
其中$\chi_i$、$\eta_i$和$\kappa$为经验参数,按照表~\ref{tab:Cote2005方案中k计算参数取值} 和~\ref{tab:Cote2005方案中k计算参数kappa取值} 取值:
\begin{table}[htbp]
  \centering
  \caption{\citet{cote2005}方案中$k_{\mathrm{dry}}$计算公式中的参数取值}
  \label{tab:Cote2005方案中k计算参数取值}
  \begin{tabular}{@{}lcc@{}}
    \toprule
    土壤成分   & $\chi$       & $\eta$             \\
    \midrule
    砾石       & 1.70         & 1.80               \\
    有机质土壤 & 0.30         & 0.87               \\
    矿物质土壤 & 0.75         & 1.20               \\
    \bottomrule
  \end{tabular}
\end{table}
%
\begin{table}[htbp]
  \centering
  \caption{\citet{cote2005}方案中$K_{\mathrm {e}} $计算公式中$\kappa$的参数取值}
  \label{tab:Cote2005方案中k计算参数kappa取值}
  \begin{tabular}{@{}lcccc@{}}
    \toprule
    土壤状态   & 粗质颗粒土壤 & 中等颗粒或砂质土壤  & 粉粒或黏粒土壤 & 有机质土壤 \\
    \midrule
    未冻结土壤 & 4.60         & 3.55                & 1.90           & 0.60       \\
    冻结土壤   & 1.70         & 0.95                & 0.85           & 0.25       \\
    \bottomrule
  \end{tabular}
\end{table}

\subsubsection{\citet{balland2005}方案}
在~\citet{balland2005}方案中,土壤导热率$k$、干土壤导热率$k_{\mathrm{dry}}$和饱和土壤导热率$k_{\mathrm{sat}}$的计算方式与~\citet{Johansen1975}方案完全一致 (\eqref{eq:STC},\eqref{eq:STC_dry} 和~\eqref{eq:STC_wet})。而Kersten数$K_{\mathrm {e}} $的计算方案则通过大量实测数据得到从干土壤到饱和土壤、细颗粒土壤到粗颗粒土壤过渡更为平滑的计算方式,表达式为:
\begin{equation}
  K_{\mathrm {e}} =\begin{cases}
    S_{\mathrm {r}} ^{0.5(1+v_{\mathrm{om}}-\alpha v_{\mathrm{sand}}-v_{\mathrm {g}} )}\left[\left(\frac{1}{1+{\mathrm e}^{-\beta S_{\mathrm {r}} }}\right)^3-\left(\frac{1-S_{\mathrm {r}} }{2}\right)^3\right]^{1-v_{\mathrm{om}}} & \text {对于未冻结土壤时} \\
    S_{\mathrm {r}} ^{1+v_{\mathrm{om}}} & \text {对于冻结土壤时}
  \end{cases}
\end{equation}
其中,$v_{\mathrm{sand}}$、$v_{\mathrm{om}}$和$v_{\mathrm {g}} $分别表示砂质土壤、有机质土壤和砾石在整体土壤柱内的体积分数。$\alpha$和$\beta$为经验参数,取值见表~\ref{tab:Balland-Arp方案中Ke计算参数取值} (来自~\citet{Barry2015thermal})。

{
  \begin{table}[htbp]
    \centering
    \caption{\citet{balland2005}方案中$K_{\mathrm {e}} $计算公式中的参数取值}
    \label{tab:Balland-Arp方案中Ke计算参数取值}
    \begin{tabular}{@{}lcc@{}}
      \toprule
      土壤成分     & $\alpha$ & $\beta$ \\
      \midrule
      粗质颗粒土壤 & 0.38     & 35.0    \\
      中等颗粒土壤 & 0.24     & 26.0    \\
      细质颗粒土壤 & 0.20     & 10.0    \\
      \bottomrule
    \end{tabular}
  \end{table}
}


\subsubsection{\citet{lu2007improved}方案}
在~\citet{lu2007improved}方案中,土壤导热率$k$与饱和土壤导热率$k_{\mathrm{sat}}$的计算方式与~\citet{Johansen1975}方案完全一致(\eqref{eq:STC} 和~\eqref{eq:STC_wet})。而干土壤导热率$k_{\mathrm{dry}}$也沿用~\citet{Johansen1975}方案中的计算公式~\eqref{eq:STC_dry},但矿物质土壤在干燥状态下的导热率$k_{\mathrm{minerals\_dry}}$则通过观测数据重新拟合了新的公式,表达为:
\begin{equation}\label{eq:STC_dry_Lu}
  k_{\mathrm{minerals\_dry}}=-0.56\theta_{\mathrm {sat}} +0.51
\end{equation}

此外,Kersten数$K_{\mathrm {e}} $的计算方案也通过大量实测数据拟合得到了新的经验关系式,表达为:
$$K_{\mathrm {e}} ={\mathrm e}^{\alpha\left(1-S_{\mathrm {r}} ^{\alpha-\beta}\right)}$$
其中,$\alpha$和$\beta$为经验参数,取值见表~\ref{tab:lu2007方案中Ke参数取值}。

{
  \begin{table}[htbp]
    \centering
    \caption{\citet{lu2007improved}方案中$K_{\mathrm {e}} $计算公式中的参数取值}
    \label{tab:lu2007方案中Ke参数取值}
    \begin{tabular}{@{}lcc@{}}
      \toprule
      土壤成分     & $\alpha$ & $\beta$ \\
      \midrule
      粗质颗粒土壤 & 0.728    & 1.165   \\
      细质颗粒土壤 & 0.37     & 1.29    \\
      \bottomrule
    \end{tabular}
  \end{table}
}


\subsubsection{\citet{Yan2019thermal}方案}
在~\citet{Yan2019thermal}方案中,土壤导热率$k$与饱和土壤导热率$k_{\mathrm{sat}}$的计算方式与~\citet{Johansen1975}方案完全一致 (\eqref{eq:STC} 和~\eqref{eq:STC_wet})。而干土壤导热率$k_{\mathrm{dry}}$则通过观测数据重新拟合了新的公式,表达为:
\begin{equation}\label{eq:STC_dry_Yan}
  k_{\mathrm{dry}}=-0.5815\theta_{\mathrm {sat}}  + 0.4999
\end{equation}

此外,Kersten数$K_{\mathrm {e}} $的计算方案也通过大量实测数据拟合得到了新的经验关系式,表达为:$$K_{\mathrm {e}} =\frac{1+\left(\frac{\theta_{\mathrm {sat}} }{\beta}\right)^{-\beta}}{1+\left(\frac{\theta_{\mathrm{liq}}}{\beta}\right)^{-\beta}}$$
其中,$\theta_{\mathrm{liq}}$表示液态土壤体积含水量,$\beta$为经验参数,表达为:$$\beta = -0.303k_{\mathrm{sat}} - 0.201m_{\mathrm{sand}} + 1.532$$
$m_{\mathrm{sand}}$表示砂质土壤在整体土壤柱内的质量分数。

\subsubsection{\citet{tarnawski2012series}方案}
在~\citet{Johansen1975}方案及其衍生方案中,假设土壤组分的排列方式在热流方向上是并行的或处在并行和串行之间的某种方式,因此总热导率可计算为所有土壤组分热导率的算术或几何平均值。作为更为复杂的方案,\citet{tarnawski2012series}提出了一种基于土壤组分串行-并行混合排列下的土壤导热率计算方案。他们假设热流通过三种介质通道传导,即固体均匀介质通道($\Theta_{\mathrm{sb}}$),由土壤固体与一部分土壤水($n_{\mathrm {w}} $)和一部分土壤空气($n_{\mathrm {a}} $)的并联路径相连构成的串行-并行通道,以及土壤水($\Theta_{\mathrm {w}} $)和空气($\Theta_{\mathrm {a}} $)并行排列的通道。括号中的符号表示土壤中每个组分的体积分数。通过将经典电阻模型应用于每个热流通道,\citet{tarnawski2012series}提出的总热导率计算方案可以通过以下方程计算:
\begin{equation}
  \begin{aligned}
    k=&k_{\mathrm {s}} \Theta_{\mathrm{sb}}+\frac{(1-\theta_{\mathrm {sat}} -\Theta_{\mathrm{sb}}+n_{\mathrm{wm}})^2}{\frac{1-\theta_{\mathrm {sat}} -\Theta_{\mathrm{sb}}}{k_{\mathrm {s}} }+\frac{n_{\mathrm{wm}}}{k_{\mathrm {w}} \frac{n_{\mathrm {w}} }{n_{\mathrm{wm}}}+k_{\mathrm {a}} \left(1-\frac{n_{\mathrm {w}} }{n_{\mathrm{wm}}}\right)}}+k_{\mathrm {w}} \left(\theta_{\mathrm {sat}}S_{\mathrm {r}} -n_{\mathrm{wm}}\frac{n_{\mathrm {w}} }{n_{\mathrm{wm}}}\right) \\
    +&k_{\mathrm {a}} \left[\theta_{\mathrm {sat}} (1-S_{\mathrm {r}} )-n_{\mathrm{wm}}\left(1-\frac{n_{\mathrm {w}} }{n_{\mathrm{wm}}}\right)\right]
  \end{aligned}
\end{equation}
其中,$k_{\mathrm {s}} $,$k_{\mathrm {w}} $和$k_{\mathrm {a}} $分别表示固体土壤、水(冰)和空气的导热率,而固体土壤导热率由矿物质土壤、有机质土壤和砾石在饱和土壤状态下的导热率基于其体积分数进行几何加权平均得到,每种土壤组分在饱和状态下的导热率计算方案与~\citet{Johansen1975}方案完全一致。$\Theta_{\mathrm{sb}}$和$n_{\mathrm{wm}}$可通过砾石和砂质土壤含量拟合得到,计算表达式为:$$\Theta_{\mathrm{sb}}=0.0237-0.0175\left(m_{\mathrm{gravel}}+m_{\mathrm{sand}}\right)^3$$$$n_{\mathrm{wm}}=n_{\mathrm {a}} +n_{\mathrm {w}} =0.088-0.037\left(m_{\mathrm{gravel}}+m_{\mathrm{sand}}\right)^3$$
其中,$m_{\mathrm{gravel}}$和$m_{\mathrm{sand}}$表示砾石和砂质土壤在整体土壤柱内的质量分数。$\frac{n_{\mathrm {w}} }{n_{\mathrm{wm}}}$可由如下关系近似得到:
\begin{equation}
  \frac{n_{\mathrm {w}} }{n_{\mathrm{wm}}}=\begin{cases}
    0  & \text{当}\ Sr=0\ \text{时} \\
    {\mathrm e}^{1-S_{\mathrm {r}} ^{-X}} & \text{当}\ 0<S_{\mathrm {r}} \leqslant1 \text{时}
  \end{cases}
\end{equation}
其中,$X=0.6-0.3\left(m_{\mathrm{gravel}}+m_{\mathrm{sand}}\right)^3$为表征小土壤空隙持水能力的因子。

\subsubsection{\citet{de1963thermal}方案}
\citet{de1963thermal}方案是一种经典的理论方案,其发展基于麦克斯韦方程。在该方案中,假设土壤结构由椭球形颗粒组成,这些颗粒在连续的孔隙流体(空气或水)中可自由浮动。因此,土壤总热导率可被估算为考虑土壤各组分形状因子的基础上各热导率的加权平均值,表达式如下:$$k=\frac{F_{\mathrm {w}} \theta_{\mathrm {sat}}S_{\mathrm {r}}k_{\mathrm {w}} +F_{\mathrm {a}} \theta_{\mathrm {sat}} \left(1-S_{\mathrm {r}} \right)k_{\mathrm {a}} +F_{\mathrm {s}} \left(1-\theta_{\mathrm {sat}} \right)k_{\mathrm {s}} }{F_{\mathrm {w}} \theta_{\mathrm {sat}}S_{\mathrm {r}} +F_{\mathrm {a}} \theta_{\mathrm {sat}} \left(1-S_{\mathrm {r}} \right)+F_{\mathrm {s}} \left(1-\theta_{\mathrm {sat}} \right)}$$
其中,$F_{\mathrm {s}} $、$F_{\mathrm {w}} $和$F_{\mathrm {a}} $分别表示固体土壤、水和空气的形状因子,它们可通过如下经验公式得到:
\begin{equation}
  F_{i} =\frac{1}{3}\left[\frac{2}{1+\left(\frac{k_{i} }{k_{\mathrm {w}} }-1\right)g_{i} }+\frac{1}{1+\left(\frac{k_{i} }{k_{\mathrm {w}} }-1\right)(1-2g_{i} )}\right],\quad i=w,a,s
\end{equation}
$g_{i} $为表征椭球形颗粒的拟合参数,可由下式计算:
\begin{equation}
  g_{i} =\begin{cases}
    0.013+0.944\theta_{\mathrm {sat}}S_{\mathrm {r}}   & \text{对于空气分子且}\ \theta_{\mathrm {sat}}S_{\mathrm {r}} \leqslant 0.09\ \text{时} \\
    0.333-\left(0.333-0.035\right)\left(1-S_{\mathrm {r}} \right) & \text{对于空气分子且}\ \theta_{\mathrm {sat}}S_{\mathrm {r}} >0.09\ \text{时} \\
    0.125 &\text{对于固体土壤颗粒}
  \end{cases}
\end{equation}


\subsubsection{积雪导热率计算方案}
根据~\citet{jordan1991one},雪盖热力传导率$\lambda_{i} $ (\unit{W.m^{-1}.K^{-1}}, $i=snl+1,\ldots,0$)的计算公式如下:
\begin{equation}
  \lambda_{i}=K_{\mathrm {a}}+\left(7.75 \times 10^{-5} \rho_{\mathrm{sno},i}+1.105 \times 10^{-6} \rho_{\mathrm{sno},i}^{2}\right)\left(K_{\mathrm {ice}}-K_{\mathrm {a}}\right)
\end{equation}
其中$k_{\mathrm {a}} $表示空气的热力传导率(\unit{W.m^{-1}.K^{-1}}),$\rho_{\mathrm{sno},i}$表示第 $i$ 层雪的平均密度(\unit{kg.m^{-3}}):
\begin{equation}
  \rho_{\mathrm{sno},i}=\frac{w_{\mathrm{liq},i}+w_{\mathrm{ice},i}}{\Delta z_{i}}
\end{equation}


\part{植被冠层、积雪和土壤水分计算方案}{Canopy, Snow and Soil Water Schemes}\label{part:SPC}
%\epart{Canopy, Snow and Soil Water Schemes}
% \chapter{植被冠层和土壤水分}
%\addcontentsline{toc}{chapter}{陆地表面的水分循环}

\chapter{植被冠层水分}\label{植被冠层截留}

% \section{植被冠层截留}\label{植被冠层截留}
\begin{mymdframed}{代码}
  本章对应的代码文件为\texttt{MOD\_LeafInterception.F90}。
\end{mymdframed}

冠层是降水的首要接触层,它对降水的再分配有着重要的作用。CoLM冠层的水量平衡方程主要采用SiB2方案~\citep{sellers1996revised},基于以下控制方程:
\begin{equation}\label{eq:冠层水量控制方程}
  \frac{\partial L_{\mathrm{dew}}}{\partial t} = p-D_{\mathrm{d}}-D_{\mathrm{c}}-E_{\mathrm{va}}
\end{equation}
式中$p$为总降水量(\unit{mm.s^{-1}}),分为对流降水和大尺度降水类型:
\begin{equation}\label{eq:降水类型}
  p=p_{\mathrm{c}}+p_{\mathrm{l}}=\left(p_{\mathrm{c,rain}}+p_{\mathrm{c,snow}}\right)+\left(p_{\mathrm{l,rain}}+p_{\mathrm{l,snow}}\right)
\end{equation}
其中下标$\rm c$和$\rm l$代表对流和大尺度降水类型,$\text{rain}$和$\text{snow}$分别代表降雨率与降雪率(\unit{mm.s^{-1}})。当驱动强迫场中只有总降水可用时,CoLM假设大尺度降水等于$1/3p$,对流降水占$2/3p$。
$D_{\mathrm {d}} $为降雨穿透速率(\unit{mm.s^{-1}}),即从冠层间隙下落的降水速率,$D_{\mathrm {c}} $为冠层排水速度(\unit{mm.s^{-1}}),$\frac{\partial L_{\mathrm{lew}}}{\partial t}$为冠层蓄水变化率(\unit{mm.s^{-1}}),$E_{\mathrm{va}}$为蒸发速率。

CoLM支持包括CoLM2014、CLM4.5、CLM5.0、VIC、MATSIRO、JULES 和Noah-MP等多种冠层截留计算参数化方案。具体的区别如表~\ref{tab:不同截留方案比较} 所示。
%
%\begin{table}[htbp]
%\caption{不同截留方案的比较}
%\label{tab:不同截留方案比较}
%\begin{center}
%\begin{tabular}{p{2cm}{c}p{1.5cm}{c}p{1.5cm}p{1.5cm}p{1.5cm}p{1.5cm}p{1.5cm}p{1.5cm}p{1.5cm}{c}}
%\toprule
%模式 & 叶倾角 & 独立的雨雪过程 & 雪的卸载 & 叶片水的相变 & 可变最大冠层水深 & 降水的形态和网格分布 & 独立叶温计算 & 重力排水\\\midrule
%CoLM2014 & 是 & 否 & 否 & 否 & 否 & 是 & 否 & 否 \\
%CLM4.5 & 否 & 否 & 否 & 否 & 否 & 否 & 否 & 否 \\
%CLM5 & 否 & 是 & 是 & 否 & 否 & 否 & 否 & 否 \\
%Noah-Mp & 否 & 是 & 否 & 是 & 是 & 是 & 是 & 否 \\
%VIC & 否 & 是 & 否 & 是 & 是 & 是 & 是 & 否 \\
%MATSIRO & 否 & 是 & 否 & 是 & 是 & 是 & 否 & 是 \\
%JULES & 否 & 是 & 否 & 是 & 是 & 是 & 否 & 是 \\
%\bottomrule
%\end{tabular}
%\end{center}
%\end{table}

\begin{table}[htbp]
  \centering \renewcommand{\arraystretch}{1.5}
  \caption{不同截留方案的比较}
  \label{tab:不同截留方案比较}
  \begin{tabular}{p{2cm}ccccccc}
    \toprule
    模式                 & CoLM2014   & CLM4.5   & CLM5       & Noah-MP    & VIC        & MATSIRO    & JULES      \\ \midrule
    叶倾角               & \checkmark & $\times$ & $\times$   & $\times$   & $\times$   & $\times$   & $\times$   \\
    独立的雨雪过程       & \checkmark & $\times$ & \checkmark & \checkmark & \checkmark & \checkmark & \checkmark \\
    雪的卸载             & \checkmark & $\times$ & \checkmark & \checkmark & $\times$   & $\times$   & $\times$   \\
    叶片水的相变         & \checkmark & $\times$ & $\times$   & \checkmark & \checkmark & \checkmark & \checkmark \\
    可变最大冠层储水深度 & \checkmark & $\times$ & $\times$   & \checkmark & \checkmark & \checkmark & \checkmark \\
    降水的形态和网格分布 & \checkmark & $\times$ & $\times$   & \checkmark & \checkmark & \checkmark & \checkmark \\
    截留是否影响叶温计算 & \checkmark & $\times$ & $\times$   & \checkmark & \checkmark & $\times$   & $\times$   \\
    重力排水             & $\times$   & $\times$ & $\times$   & $\times$   & $\times$   & \checkmark & \checkmark \\ \bottomrule
  \end{tabular}
\end{table}
以下针对不同方案分别进行详细介绍。

\section{CoLM2014方案}
在CoLM2014方案中,公式~\eqref{eq:冠层水量控制方程} 中的直接穿透量$D_{\mathrm {d}} $与辐射传输中的光穿透量的计算方法一致,由下式给出:
\begin{equation}
  D_{\mathrm{d}}=\delta_{\mathrm{p}} \cdot p=\delta_{\mathrm{p}} \cdot\left(p_{\mathrm{c,rain}}+p_{\mathrm{c,snow}}+p_{\mathrm{l,rain}}+p_{\mathrm{l,snow}}\right)
\end{equation}
其中$\delta_{\mathrm {p}} $是冠层穿透系数,等于
\begin{equation}
  \delta_{\mathrm{p}}=1.0-\alpha \left[1-\exp \left(-K_{\mathrm{p}}  {\rm LSAI}\right)\right]
\end{equation}
该公式中反映叶片集水能力的经验系数$\alpha$被设定为0.25~\citep{lawrence2011parameterization}。
${\rm LSAI}$是${\rm SAI}$和${\rm LAI}$之和(${\rm LAI}$和${\rm SAI}$分别是叶指数和茎面积指数);$K_{\mathrm {p}} $是消光系数,与辐射垂直入射冠层时计算相同:
\begin{equation}\label{eq:消光系数}
  \begin{aligned}
    K_{\mathrm{p}} &= \phi_{1}+\phi_{2} \\
    \phi_{2} &= 0.877\left(1-2 \phi_{1}\right) \\
    \phi_{1} &= 0.5-0.633 \chi_{\mathrm{L}}-0.33 \chi_{\mathrm{L}}^{2}
  \end{aligned}
\end{equation}
$K_{\mathrm {p}} $通过叶角分布的经验参数$\chi_{\mathrm {L}} $的变化来控制,其中$\chi_{\mathrm {L}} =0$表示球形叶角分布,$\chi_{\mathrm {L}} = 1$表示平面型叶倾角分布,$\chi_{\mathrm {L}} = -1$表示竖直型叶倾角分布。根据植被类型,$\chi_{\mathrm {L}} $的范围在$-0.3\sim0.25$之间。由此计算得出的$\delta_{\mathrm {p}} $的变动范围如图~\ref{fig:穿透系数与叶面积指数} 所示。
{
  \begin{figure}[htbp]
    \centering
    \includegraphics[width=1.0\textwidth]{Figures/陆地表面的水分循环/穿透系数与叶面积指数.jpg}
    \caption{穿透系数$\delta_{\mathrm {p}} $与叶面积指数的关系及其变动范围}
    \label{fig:穿透系数与叶面积指数}
  \end{figure}
}

为防止升华或冷凝水超过最大树冠储存量,CoLM在计算冠层截留之前首先更新树叶上的水深($L_{\mathrm{dew}}$) (mm):
\begin{equation}
  \begin{aligned}
    L_{\mathrm{dew}} &= L_{\mathrm{dew}}-x_{\mathrm{sc}} \\
    x_{\mathrm{sc}} &= \max\left(0., L_{\mathrm{dew}}-L_{\mathrm {dew,max}}\right)
  \end{aligned}
\end{equation}
其中$x_{\mathrm{sc}}$是超过最大树冠储存量的水(mm),$L_{\mathrm {dew,max}} =0.1 \cdot{\rm LSAI}$代表最大蓄水量(mm)。
$x_{\mathrm{sc}}$的相位首先由冠层温度决定。如果冠层温度大于冰点温度,CoLM假定所有过量的水都处于液相,否则处于冰雪相。
{
  \begin{figure}[htbp]
    \centering
    \includegraphics[width=0.9\textwidth]{Figures/植被冠层和土壤水分/CoLM冠层截留示意图-new.pdf}
    \caption[(a) CoLM中使用的降水面积-总量分布图;(b) CoLM中植被冠层截留降水的动力学过程]{(a) CoLM中使用的降水面积-总量分布图(修改自SiB2)。其中变量$x$为网格面积的比例,变量$I_{\left(x\right)}$为相对降水量。值得注意的是,大尺度降水$I_{\mathrm {l}} \left(x\right)$在网格区几乎是不变的,而对流降水$I_{\mathrm {c}} \left(x\right)$是非均匀分布的;(b) CoLM中植被冠层截留降水的动力学过程。降雨截留前已储存在冠层中的水量被假设是均匀分布在网格区域上的。$x_s$是网格区域拦截降雨的比例加上先前存在的树冠蓄水所超过了$L_{\mathrm {dew,max}}$的水量。所有大于$L_{\mathrm {dew,max}}$的水量假设形成树干径流,而低于$L_{\mathrm {dew,max}}$的水量则被保存在冠层之中}
    \label{fig:CoLM冠层截留示意图}
  \end{figure}
}

如图~\ref{fig:CoLM冠层截留示意图}a 所示,CoLM假设网格内的降水的分布是不均匀的。其中对流降水面积的网格占比$I_{\mathrm {c}} $可以由下式给出
\begin{equation}
  I_{\mathrm{c}}(x)=a_{\mathrm{c}} {\mathrm e}^{-bx}+c_{\mathrm{c}}
\end{equation}
其中$a_{\mathrm {c}} =20$,$b=20$和$c_{\mathrm {c}} =0.206\times10^{-8}$是常数。大尺度降水面积的网格占比$I_{\mathrm {l}} $可以用相同的形式表示:
\begin{equation}
  I_{\mathrm{l}}(x)=a_{\mathrm{l}} {\mathrm e}^{-b x}+c_{\mathrm{l}}
\end{equation}
其中$a_{\mathrm {l}} =0.0001$和$c_{\mathrm {l}} =0.9999$。因此,通过组合这两个方程给出了有效降水占比区域的降雨强度为:
\begin{equation}
  p I(x)=\left(a_{\mathrm{l}} p_{\mathrm{l}}+a_{\mathrm{c}} p_{\mathrm{c}}\right) {\mathrm e}^{-b x}+\left(c_{\mathrm{l}} p_{\mathrm{l}}+c_{\mathrm{c}} p_{\mathrm{c}}\right)
\end{equation}
如图~\ref{fig:CoLM冠层截留示意图}b 所示,因此,有效降水区域的树冠排水($D_{\mathrm {t,rain}}$)由下式给出:
\begin{equation}
  D_{\mathrm {t,rain}}=\int_{0}^{x_{\mathrm{s}}} p I(x){\mathrm { d}} x+L_{\mathrm{dew}} x_{\mathrm{s}}-L_{\mathrm {dew,max}} x_{\mathrm{s}}
\end{equation}
其中$x_{\mathrm {s}} $是单位网格中截获降水加上原有冠层蓄水量超过冠层上允许的最大水深($L_{\mathrm {dew,max}}$)的比例\todo{图中仍然使用$S_c$,且为Xs,标题以后后面介绍中为小写x}:
\begin{equation}
  x_{\mathrm{s}}=\frac{-1}{b} \log \left[\frac{L_{\mathrm {dew,max}}-L_{\mathrm{dew}}}{a_{\mathrm{p}}\left(p-D_{\mathrm{d}}\right)}-\frac{c_{\mathrm{p}}}{a_{\mathrm{p}}}\right]
\end{equation}
即树冠截取的雨量等于或者超过其饱和极限网格面积的比例。其中$a_{\mathrm {p}} =\frac{a_{\mathrm {l}}p_{\mathrm {l}} +a_{\mathrm {c}}p_{\mathrm {c}} }{p}$,$c_{\mathrm {p}} =\frac{c_{\mathrm {l}}p_{\mathrm {l}} +c_{\mathrm {c}}p_{\mathrm {c}} }{p}$。这里假设只有液态水能够被从冠层落下,我们有
\begin{equation}
  D_{\mathrm {t,rain}}=\left(p_{\mathrm{c,rain}}+p_{\mathrm{l,rain}}\right)\left(1-\delta_{\mathrm{p}}\right) \frac{a_{\mathrm{p}}}{b}\left(1-{\mathrm e}^{-b x_{\mathrm{s}}}\right)+c_{\mathrm{p}} x_{\mathrm{s}}+L_{\mathrm{dew}} x_{\mathrm{s}}-p_{\mathrm{c,snow}} x_{\mathrm{s}}
\end{equation}
在整体网格尺度上$D_{\mathrm {c}} $被更新为
\begin{equation}
  D_{\mathrm {c}} =D_{\mathrm {t,r a i n}}+x_{\mathrm{s c}}
\end{equation}
因此,保留在树冠上的可用于树冠蒸发(截留蒸发)的蓄水量$L_{\mathrm{dew}}$ (mm)可以由下式进行更新:
\begin{equation}
  L_{\mathrm{dew}}={p}-D_{\mathrm{c}}-D_{\mathrm{d}}
\end{equation}
在冠层蒸发量的计算上首先需要计算湿叶茎面积占总叶茎面积的比例 ($f_{\mathrm{wet}}$)。CoLM模型采用~\citet{dickinson1993biosphere}提出的经验方法:
\begin{equation}
  f_{\mathrm{wet}}=\left(\frac{L_{\mathrm{dew}}}{L_{\mathrm{dew,max}}}\right)^{2 / 3} \leqslant 1.0
\end{equation}
其中$L_{\mathrm{dew}}$是上一步计算得出的可用于树冠蒸发的蓄水量 (m)。因此,树冠蒸发量计算如下:
\begin{equation}
  E_{\mathrm{va}} =\min \left(\frac{q_{\mathrm{s}}-q_{\mathrm{sat}}^{T_{\mathrm{v}}}}{r_{\mathrm{b}}}\cdot{\rm LSAI}\cdot f_{\mathrm{wet}}, \frac{L_{\mathrm{dew}}}{\Delta t}\right)
\end{equation}
这里$r_{\mathrm {b}} $是平均边界层阻力,由冠层顶部的风速和特征叶片尺寸确定;$q_{\mathrm{sat}}^{T_{\mathrm {v}} }$ (\unit{kg.kg^{-1}})是冠层温度下的饱和比湿度;$q_{\mathrm {s}} $为植被周围空气比湿,$\Delta t$为时间步长。


\section{CLM4.5方案}
CLM4.5的冠层截留方案~\citep{oleson2013technical}~和CoLM2014的方案非常类似。主要的区别在于公式~\eqref{eq:消光系数} 中的$K_{\mathrm {p}} $的计算。
$K_{\mathrm {p}} $在CLM4.5冠层截留方案中取固定值0.5,即:
%
\begin{equation}
  \delta_{\mathrm{p}}=1.0-\alpha \left[1-\exp \left(-0.5  \cdot{\rm LSAI}\right)\right]
\end{equation}
则直接穿透量由下式给出:
\begin{equation}
  D_{\mathrm{d}}=\delta_{\mathrm{p}} \cdot p=\delta_{\mathrm{p}} \cdot\left(p_{\mathrm{c,rain}}+p_{\mathrm{c,snow}}+p_{\mathrm{l,rain}}+p_{\mathrm{l,snow}}\right)
\end{equation}

为了保持模型构架的一致性,我们和CoLM2014方案一样在计算冠层截留之前首先更新树叶上的水深 ($L_{\mathrm{dew}}$) (mm):
\begin{equation}
  \begin{aligned}
    L_{\mathrm{dew}} &= L_{\mathrm{dew}}-x_{\mathrm{sc}} \\
    x_{\mathrm{s c}} &= \max \left(0., L_{\mathrm{dew}}-L_{\mathrm{dew,max}}\right)
  \end{aligned}
\end{equation}
请注意在原生CLM4.5方案中没有这一步骤。

由于CLM4.5方案中并不考虑网格内部的降水类型和分布(假设降水均匀分布在整个网格上),则降水区域的冠层排水$D_{\mathrm {t}}$:
\begin{equation}
  D_{\mathrm {t}}=p\left(1-\delta_{\mathrm{p}}\right)+L_{\mathrm{dew}}-p_{\mathrm{c,snow}}
\end{equation}
在整体网格尺度上$D_{\mathrm {c}} $被更新为
\begin{equation}
  D_{\mathrm {c}} =D_{\mathrm {t}}+x_{\mathrm{s c}}
\end{equation}
其他均与CoLM2014方案保持一致。

\section{CLM5.0方案}
CLM5.0的参数化方案与CLM4.5有较大的差别~\citep{lawrence2019community}。其中最重要的差别之一是对雨和雪的截留方案的进一步细化,但不考虑网格内部的降水类型和分布。在CLM5中,液态水的降落过程和CoLM、CLM4.5类似。液态降水在通过冠层时要么被树冠截留,要么直接落到雪/土壤表面(穿透)或从植被上滴下(树冠滴流)。固态降水的处理也类似,但需要额外考虑先前截留的雪的卸载(unloading)。具体计算流程如下:

\begin{enumerate}
  \item 分别计算固态和液态水在叶片上的最大蓄水量(mm):
    \begin{equation}
      S_{\mathrm{c,rain}}=0.1\cdot{\rm LSAI}
    \end{equation}
    \begin{equation}
      S_{\mathrm{c,snow}}=6.0\cdot{\rm LSAI}
    \end{equation}
  \item 分别计算固态和液态降水的直接穿透速率:
    \begin{equation}
      \delta_{\mathrm{p,rain}}=1.0 - \alpha_{\mathrm{rain}} \tanh({\rm LSAI})
    \end{equation}
    \begin{equation}
      \delta_{\mathrm{p,snow}}=1.0 - \alpha_{\mathrm{snow}}\left\{1-\exp\left[-0.5\cdot{\rm LSAI}\right]\right\}
    \end{equation}
    这里$\alpha_{\mathrm{rain}}$和$\alpha_{\mathrm{snow}}$的值与CLM4.5、CoLM的$\alpha$ (0.25)不同,在CLM5中被设定为1.0。穿透速率也被拆分为:
    \begin{equation}
      D_{\mathrm{d,rain}}=\delta_{\mathrm{p,rain}}(p_{\mathrm {c,rain}} +p_{\mathrm{l,rain}})
    \end{equation}
    \begin{equation}
      D_{\mathrm{d,snow}}=\delta_{\mathrm{p,snow}}(p_{\mathrm {c,rain}} +p_{\mathrm{l,snow}})
    \end{equation}
  \item 计算冠层排水速率:
    \begin{equation}
      D_{\mathrm {t,rain}}=(p_{\mathrm {c,rain}} +p_{\mathrm{l,rain}})(1-\delta_{\mathrm{p,rain}})+L_{\mathrm{dew,rain}}-S_{\mathrm{c,rain}}
    \end{equation}
    \begin{equation}
      D_{\mathrm {t,snow}}=(p_{\mathrm {c,rain}} +p_{\mathrm{l,snow}})(1-\delta_{\mathrm{p,rain}})+L_{\mathrm{dew,snow}}-S_{\mathrm{c,snow}}
    \end{equation}
    CLM5额外考虑了冠层积雪的卸载速率:
    \begin{equation}\label{q_unl_wind}
      q_{\mathrm{unload,wind}}=\frac{uL_{\mathrm{dew,snow}}}{1.56\times 10^5}
    \end{equation}
    \begin{equation}
      q_{\mathrm{unload,temp}}=\max\left[0, \frac{L_{\mathrm{dew,snow}}(T-270)}{1.87\times 10^5}\right]
    \end{equation}
    \begin{equation}\label{q_unl_tot}
      q_{\mathrm{unload}}=\min(q_{\mathrm{unload,wind}}+q_{\mathrm{unload,temp}},L_{\mathrm{dew,snow}})
    \end{equation}
    固态和液态的排水速率为:
    \begin{equation}
      D_{\mathrm{c,rain}}=D_{\mathrm {t,rain}}
    \end{equation}
    \begin{equation}
      D_{\mathrm{c,snow}}=D_{\mathrm {t,snow}}+q_{\mathrm{unload}}
    \end{equation}
\end{enumerate}
其他参数计算均参照CoLM2014方案。


\section{Noah-MP方案}
Noah-MP模型冠层截留方案的计算流程与CoLM2014、CLM4.5和CLM5有较大的区别~\citep{niu2011community,he2023modernizing}。具体介绍如下:
\begin{enumerate}
  \item 分别计算理想状态下固态和液态水在叶片上的最大蓄水量(mm)\\
    \begin{equation}
      S_{\mathrm{c,rain}}=0.1\cdot{\rm LSAI}\\
    \end{equation}
    \begin{equation}
      S_{\mathrm{c,snow}}= 6.6\left(0.27+{\frac{46}{\rho_{\mathrm{sno}}}}\right) \cdot{\rm LSAI}
    \end{equation}
    其中$\rho_{\mathrm{sno}}$是降雪堆积密度:
    \begin{equation}
      \rho_{\mathrm{sno}}=67.92+51.25 {\mathrm e}^{\frac{T_{\mathrm{v}}-T_{\mathrm{frz}}}{2.59}}
    \end{equation}
其中$T_{\mathrm{v}}$为植被叶片表面温度(K)。
  \item 计算冠层闭郁度$F_{\mathrm{veg}}$:
    \begin{equation}
      F_{\mathrm{veg}} = \max\left[0.05,1.0-{\mathrm e}^{-0.52\cdot{\rm LSAI}}\right]
    \end{equation}
  \item 分别计算固态和液态降水直接穿透速率:
    \begin{equation}
      D_{\mathrm{d,rain}}=\left(1.0-F_{\mathrm{veg}}\right)  (p_{\mathrm{c,rain}}+p_{\mathrm{l,rain}})
    \end{equation}
    \begin{equation}
      D_{\mathrm{d,snow}}=\left(1.0-F_{\mathrm{veg}}\right)  (p_{\mathrm{c,snow}}+p_{\mathrm{l,snow}})
    \end{equation}
  \item 如有积雪,则考虑其积雪的卸载,该方案与CLM5所用的方案完全一致(公式~\eqref{q_unl_wind}--\eqref{q_unl_tot})。
  \item 计算当前冠层水的相态变化:
    \begin{enumerate}
      \item 冠层固态水转变为液态水,如果表面温度高于0度:
        \begin{equation}
          \begin{aligned}
            q_{\mathrm{v, melt}} &=\min \left(\frac{L_{\mathrm{dew,snow}}}{\Delta t},\ \left(T_{\mathrm{v}}-T_{\mathrm{frz}}\right)  \frac{C_{\mathrm{ice}}  L_{\mathrm{dew,snow}}}{\rho_{\mathrm{i c e}}  \lambda_{\mathrm{fus}}  \Delta t}\right) \\
            L_{\mathrm{dew,snow}} &=\max \left(0.0,\ L_{\mathrm{dew,snow}}-q_{\mathrm{v, melt}}  \Delta t\right) \\
            L_{\mathrm{dew,rain}} &=\max \left(0.0,\ L_{\mathrm{dew,rain}}-L_{\mathrm{dew,snow}}\right)
          \end{aligned}
        \end{equation}
      \item 冠层液态水转变为固态水,如果表面温度低于0度:
        \begin{equation}
          \begin{aligned}
            q_{\mathrm{v, f r z}} &=\min \left(\frac{L_{\mathrm{dew,rain}}}{\Delta t},\ \left(T_{\mathrm{f r z}}-T_{\mathrm{v}}\right)  \frac{C_{\mathrm{liq}}  L_{\mathrm{dew,rain}}}{\rho_{\mathrm{liq}}  \lambda_{\mathrm{fus}}  \Delta t}\right) \\
            L_{\mathrm{dew,rain}} &=\max \left(0.0,\ L_{\mathrm{dew,rain}}-q_{\mathrm{v, frz}}  \Delta t\right) \\
            L_{\mathrm{dew,snow}} &=\max \left(0.0,\ L_{\mathrm{dew,snow}}-L_{\mathrm{dew,rain}}\right)
          \end{aligned}
        \end{equation}
    \end {enumerate}
  其中$q_{\mathrm{v, melt}}$是固态转换为液态的速率(融化速率),$q_{\mathrm{v, frz}}$是液态转换为固态的速率(凝结速率),$T_{\mathrm{v}}$为叶片温度(K),$T_{\mathrm{frz}}$是相态变化温度(273.16 K),$C_{\mathrm{ice}}$是冰的比热容(\unit{J.m^{-3}.K^{-1}}),$C_{\mathrm{liq}}$是液态水的比热容(\unit{J.m^{-3}.K^{-1}}),$\Delta t$是时间步长(s),$\lambda_{\mathrm{fus}}$是融合潜热(\unit{J.kg^{-1}}),$\rho_{\mathrm{ice}}$是冰的密度(917 \unit{kg.m^{-3}}),$\rho_{\mathrm{liq}}$是水的密度(1000 \unit{kg.m^{-3}})。这一步的主要目的是更新$L_{\mathrm{dew,rain}}$和$L_{\mathrm{dew,snow}}$。

\item 如有降水发生,则计算当前降雨条件和闭郁度情况下接触冠层的最大水量(interception capability):
  \begin{equation}
    q_{\mathrm{intr,rain}} =F_{\mathrm{veg}}\left(p_{\mathrm{c,rain}}+p_{\mathrm{l,rain}}\right) {\rm FP}
  \end{equation}
  \begin{equation}
    q_{\mathrm{intr,snow}} =F_{\mathrm{veg}}\left(p_{\mathrm{c,snow}}+p_{\mathrm{l,snow}}\right) {\rm FP}
  \end{equation}
  假设对流降水在网格中是不均匀分布的,则${\mathrm {FP}}$(网格中降水发生的比率)计算如下:
  \begin{equation}
    {\rm FP} = p/ (10p_{\mathrm{c}} + p_{\mathrm{l}})
  \end{equation}
\item 计算当前时刻降水被拦截且被保留的实际网格水量:
  \begin{equation}
    q_{\mathrm{intr,rain}} = \min\left[q_{\mathrm{intr,rain}}, \ \frac{(S_{\mathrm{c,rain}} - L_{\mathrm{dew,rain}})}{\Delta{t}}  (1-{\mathrm e}^{-\left(p_{\mathrm{c,rain}}+p_{\mathrm{l,rain}}\right)\Delta{t}/S_{\mathrm{c,rain}}})\right]
  \end{equation}
  \begin{equation}
    q_{\mathrm{intr,snow}} = \min\left[q_{\mathrm{intr,snow}},\ \frac{(S_{\mathrm{c,snow}} - L_{\mathrm{dew,snow}})}{\Delta{t}} (1-{\mathrm e}^{-\left(p_{\mathrm{c,snow}}+p_{\mathrm{l,snow}}\right)\Delta{t}/S_{\mathrm{c,snow}}})\right]
  \end{equation}
  其中${\Delta{t}}$是计算时间步长。
\item 更新当前冠层固态和液态的排水速率,以及冠层的对应水深
  \begin{equation}
    D_{\mathrm{c,rain}}=F_{\mathrm{veg}}  (p_{\mathrm{c,rain}}+p_{\mathrm{l,rain}})-q_{\mathrm{intr,rain}}
  \end{equation}
  \begin{equation}
    D_{\mathrm{c,snow}}=F_{\mathrm{veg}}  (p_{\mathrm{c,snow}}+p_{\mathrm{l,snow}})-q_{\mathrm{intr,snow}}
  \end{equation}
\end{enumerate}
其他计算均参照CoLM2014方案。


\section{VIC方案}
VIC的冠层截留方案~\citep{liang1994simple,hamman2018variable}主要基于~\citet{storck2002measurement}在俄勒冈一个山区气候站点的积雪积累和消融观测结果。
\begin{enumerate}
  \item 分别计算固态和液态水在叶片上的最大蓄水量(mm)\\
    首先计算雨加上雪的总最大拦截容量为:
    \begin{equation}
      I_{\mathrm{max}} = 4.0 \cdot m \cdot{\rm LSAI}
    \end{equation}
    其中$m=0.5$ mm 是根据最大雪拦截容量观测结果确定。
    除了对水总量的储水能力有限制之外,模式还分别对固态和液态水在叶片上的最大蓄水量进行了限制:
    \begin{equation}
      S_{\mathrm{c,rain}}=0.1\cdot{\rm LSAI}+0.035 L_{\mathrm{dew,snow}}\\
    \end{equation}
    \begin{equation}
      S_{\mathrm{c,snow}}=L_{\mathrm {r}}\cdot m \cdot{\rm LSAI}
    \end{equation}
    其中$L_{\mathrm {r}} $是大气强迫温度的函数:
    \begin{equation}
      L_{\mathrm {r}}  = \begin{cases}
        4.0, & T_{\mathrm {a}}  > -1 \text{ \textcelsius}\\
        1.5, & -3 \text{ \textcelsius} < T_{\mathrm {a}}  \leqslant -1 \text{ \textcelsius}\\
        1.0, & T_{\mathrm {a}}  \leqslant -3 \text{ \textcelsius}
      \end{cases}
    \end{equation}
    以上是基于站点植被截雪观测数据计算得到~\citep{storck2002measurement}。\citet{kobayashi1987snow}~观察到随着气温降低到 $-3$ \textcelsius 以下,较窄表面上的最大雪拦截负荷急剧下降。 \citet{kobayashi1987snow}和~\citet{pfister1999snow}的观测结果表明,当温度低于 $-1$ \textcelsius 时,拦截效率显著下降,当温度低于 $-3$ \textcelsius 时大致恒定,因此导致函数的不连续。

  \item 计算冠层闭郁度$F_{\mathrm{veg}}$
    \begin{equation}
      F_{\mathrm{veg}} = \min\left[1.0,\ {\rm LSAI} \right]
    \end{equation}

  \item 分别计算固态和液态降水直接穿透速率:
    \begin{equation}
      D_{\mathrm{d,rain}}=\left(1.0-F_{\mathrm{veg}}\right)  (p_{\mathrm{c,rain}}+p_{\mathrm{l,rain}})
    \end{equation}
    \begin{equation}
      D_{\mathrm{d,snow}}=\left(1.0-F_{\mathrm{veg}}\right)  (p_{\mathrm{c,snow}}+p_{\mathrm{l,snow}})
    \end{equation}

  \item 计算当前冠层水的相态变化,更新$L_{\mathrm{dew,snow}}$ 和 $L_{\mathrm{dew,rain}}$,这一部分的方案和Noah-MP一致。

  \item 当发生降水,如果降水的形态是固态,且温度小于 $-3$ \textcelsius ,风速大于1 \unit{m.s^{-1}},则当前时刻冠层积雪的卸载由风速和冠层雪水当量共同决定:
    \begin{equation}
      D_{\mathrm{d,snow}} = \min\left[L_{\mathrm{dew,snow}},\ (0.2u-0.2)  L_{\mathrm{dew,snow}}\right]
    \end{equation}

  \item 冠层排水速率的计算
    \begin{equation}
      D_{\mathrm {t,rain}}=\left(p_{\mathrm{c,rain}}+p_{\mathrm{l,rain}}\right)\left(1-\delta_{\mathrm{p,rain}}\right)+L_{\mathrm{dew,rain}}-S_{\mathrm{c,rain}}
    \end{equation}
    \begin{equation}
      D_{\mathrm {t,snow}}=\left(p_{\mathrm{c,snow}}+p_{\mathrm{l,snow}}\right)\left(1-\delta_{\mathrm{p,snow}}\right)+L_{\mathrm{dew,snow}}-S_{\mathrm{c,snow}}+D_{\mathrm{d,snow}}
    \end{equation}

  \item 如更新排水计算后所得的截留的雨水和积雪的总量($L_{\mathrm{dew,rain}}+L_{\mathrm{dew,snow}}$)超过了树木的最大承载能力($I_{\mathrm{max}}$),则需要将最大承载能力按固液比例进行卸载:
  \begin{equation}
    L_{\mathrm{dew,overload}}=(L_{\mathrm{dew,rain}}+L_{\mathrm{dew,snow}}-I_{\mathrm{max}})
  \end{equation}
  \begin{equation}
    L_{\mathrm{dew,rain}}=L_{\mathrm{dew,rain}} - L_{\mathrm{dew,overload}}  \frac{L_{\mathrm{dew,rain}}}{L_{\mathrm{dew,rain}}+L_{\mathrm{dew,snow}}}
  \end{equation}
  \begin{equation}
    L_{\mathrm{dew,snow}}=L_{\mathrm{dew,snow}} - L_{\mathrm{dew,overload}}    \frac{L_{\mathrm{dew,snow}}}{L_{\mathrm{dew,rain}}+L_{\mathrm{dew,snow}}}
  \end{equation}
  \begin{equation}
    D_{\mathrm {t,rain}}=D_{\mathrm {t,rain}} + \frac{L_{\mathrm{dew,rain}}}{L_{\mathrm{dew,rain}}+L_{\mathrm{dew,snow}}}
  \end{equation}
  \begin{equation}
    D_{\mathrm {t,snow}}=D_{\mathrm {t,snow}} + \frac{L_{\mathrm{dew,snow}}}{L_{\mathrm{dew,rain}}+L_{\mathrm{dew,snow}}}
  \end{equation}
\end{enumerate}
其他计算均参照CoLM2014方案。


\section{MATSIRO方案}
MATSIRO的冠层截留方案主要基于~\cite{storck2002measurement}在俄勒冈一个山区气候站点的积雪积累和消融观测结果。
\begin{enumerate}
  \item 分别计算固态和液态水在叶片上的最大蓄水量(mm)
    \begin{equation}
      S_{\mathrm{c,rain}}=0.2\cdot{\rm LSAI}
    \end{equation}
    \begin{equation}
      S_{\mathrm{c,snow}}=0.2\cdot{\rm LSAI}
    \end{equation}

  \item 计算冠层闭郁度$F_{\mathrm{veg}}$
    \begin{equation}
      F_{\mathrm{veg}} = \min\left[1,~ {\rm LSAI}\right]
    \end{equation}

  \item 分别计算固态和液态降水直接穿透速率:
    \begin{equation}
      D_{\mathrm{d,rain}}=\left(1.0-F_{\mathrm{veg}}\right)  (p_{\mathrm{c,rain}}+p_{\mathrm{l,rain}})
    \end{equation}
    \begin{equation}
      D_{\mathrm{d,snow}}=\left(1.0-F_{\mathrm{veg}}\right)  (p_{\mathrm{c,snow}}+p_{\mathrm{l,snow}})
    \end{equation}
  \item 计算当前冠层水的相态变化,方案和Noah-MP一致。

  \item 当降水发生时,将网格拆分成强对流区域(Storm area)和大尺度降水区域(Non-Storm area)。其中对流降水面积的比例假设为均匀状态(0.1),而大尺度降水面积的比例则假设为1。则在强对流区域,固态和液态水在叶片上的蓄水量为:
    \begin{align}
      L_{\mathrm{dew,rain,c}} &= L_{\mathrm{dew,rain}}+\left(F_{\mathrm{veg}}\,p_{\mathrm{l,rain}} + F_{\mathrm{veg}}\,p_{\mathrm{c,rain}}/0.1\right)\Delta t \\[2.2ex]
    %\end{equation}
    %\begin{equation}
      L_{\mathrm{dew,snow,c}} &= L_{\mathrm{dew,snow}}+\left(F_{\mathrm{veg}}\,p_{\mathrm{l,snow}} + F_{\mathrm{veg}}\,p_{\mathrm{c,snow}}/0.1\right)\Delta t
    \end{align}
    冠层排水速率的计算同时考虑了超过冠层容水量导致的滴水和重力导致的自然滴水:
    \begin{align}
      D_{\mathrm{t,rain,c}} &= L_{\mathrm{dew,rain,c}} -S_{\mathrm{c,rain}} + D_{1} \exp\left(D_{2} S_{\mathrm{c,rain}}/1000\right) \\[2.2ex]
      %
      D_{\mathrm{t,snow,c}} &= L_{\mathrm{dew,snow,c}} - S_{\mathrm{c,snow}} + D_{1} \exp\left(D_{2} S_{\mathrm{c,snow}}/1000\right)
    \end{align}
    下标${\mathrm{c}}$代表对流降水,$D_{1}=1.14 \times 10^{-11}$,$D_{2}=3.7 \times 10^{3}$。在计算完冠层排水之后,再次更新强对流区域的冠层水量。

  \item 而在大尺度降水区域,固态和液态水在叶片上的蓄水量则为:
    \begin{equation}
      L_\mathrm{dew,rain,l} = L_\mathrm{dew,rain} + F_\mathrm{veg}\,p_\mathrm{l,rain} \Delta t
    \end{equation}
    \begin{equation}
      L_\mathrm{dew,snow,l} = L_\mathrm{dew,snow} + F_\mathrm{veg}\,p_\mathrm{l,snow} \Delta t
    \end{equation}
    下标${\mathrm {l}}$代表大尺度降水。
    同样冠层排水速率的计算同时考虑了超过冠层容水量导致的滴水和重力导致的自然滴水:
    %\begin{equation}
      \begin{align}
      D_{\mathrm{t,rain,l}} &=L_{\mathrm{dew,rain,l}}-S_{\mathrm{c,rain}}+D_{1} \exp \left(D_{2} S_{\mathrm{c,rain}}/1000\right) \\[2.2ex]
    %
      D_{\mathrm{t,snow,l}} &=L_{\mathrm{dew,snow,l}}-S_{\mathrm{c,snow}}+D_{1} \exp \left(D_{2} S_{\mathrm{c,snow}}/1000\right)
      \end{align}
    %\end{equation}
    在计算完冠层排水之后,再次更新大尺度区域的冠层水量。

  \item 由于在拆分强对流和大尺度降水区域计算过程中,网格区域有所重叠(在计算强对流区域截留时同时考虑了大尺度降水),需要进一步进行加权平均得到网格平均:
%
    \begin{equation}
      %\begin{aligned}
        L_{\mathrm{dew,rain}}  = L_{\mathrm{dew,rain,l}} + 0.1(L_{\mathrm{dew,rain,c}} - L_{\mathrm{dew,rain,l}}) 
    \end{equation}
    \begin{equation}
        L_{\mathrm{dew,snow}}  = L_{\mathrm{dew,snow,l}} + 0.1(L_{\mathrm{dew,snow,c}} - L_{\mathrm{dew,snow,l}}) 
      %\end{aligned}
    \end{equation}
%
    在整体网格尺度上$D_{\mathrm {c}} $亦被更新。

\end{enumerate}

\section{JULES方案}
JULES的参数化方案与CLM5.0方案较为接近。对雨和雪的截留方案的进一步细化,但不考虑网格内部的降水类型和分布。
\begin{enumerate}
  \item 分别计算固态和液态水在叶片上的最大蓄水量(mm)\\
    \begin{equation}
      S_{\mathrm{c,rain}}=0.1\cdot{\rm LSAI}\\
    \end{equation}
    \begin{equation}
      S_{\mathrm{c,snow}}=4.4\cdot{\rm LSAI}
    \end{equation}
  \item 计算冠层闭郁度$F_{\mathrm{veg}}$
    \begin{equation}
      F_{\mathrm{veg}} = \min\left[1.0,~{\rm LSAI}\right]
    \end{equation}

  \item 计算当前冠层水的相态变化,方案和Noah-MP一致。

  \item 当有降水发生时,分别计算固态和液态降水直接穿透速率:
    \begin{equation}
      D_{\mathrm{d,rain}}=\left(1.0-F_{\mathrm{veg}}\right)  (p_{\mathrm{c,rain}}+p_{\mathrm{l,rain}})
    \end{equation}
    \begin{equation}
      D_{\mathrm{d,snow}}=\left(1.0-F_{\mathrm{veg}}\right)  (p_{\mathrm{c,snow}}+p_{\mathrm{l,snow}})
    \end{equation}
  \item 计算当前时刻降水被拦截且被保留的实际网格水量的方法与Noah-MP类似,不同点在于在额外增加了一个拦截效率系数$k$(取值为0.6):
    \begin{equation}
      q_{\mathrm{intr,rain}} = \min\left[q_{\mathrm{intr,rain}},\,  k  \frac{(S_{\mathrm{c,rain}} - L_{\mathrm{dew,rain}})}{\Delta{t}} \left(1-{\mathrm e}^{-\left(p_{\mathrm{c,rain}}+p_{\mathrm{l,rain}}\right)\Delta{t}/S_{\mathrm{c,rain}}}\right)\right]
    \end{equation}
    \begin{equation}
      q_{\mathrm{intr,snow}} = \min\left[q_{\mathrm{intr,snow}},\, k  \frac{(S_{\mathrm{c,snow}} - L_{\mathrm{dew,snow}})}{\Delta{t}}\left(1-{\mathrm e}^{-\left(p_{\mathrm{c,snow}}+p_{\mathrm{l,snow}}\right)\Delta{t}/S_{\mathrm{c,snow}}}\right)\right]
    \end{equation}
    其中${\Delta{t}}$是计算时间步长。
  \item 更新当前冠层固态和液态的排水速率,以及冠层的对应水深
    \begin{equation}
      D_{\mathrm{d,rain}}=F_{\mathrm{veg}}  (p_{\mathrm{c,rain}}+p_{\mathrm{l,rain}})-q_{\mathrm{intr,rain}}
    \end{equation}
    \begin{equation}
      D_{\mathrm{d,snow}}=F_{\mathrm{veg}}  (p_{\mathrm{c,snow}}+p_{\mathrm{l,snow}})-q_{\mathrm{intr,snow}}
    \end{equation}
  \item 冠层排水速率的计算
    \begin{equation}
      D_{\mathrm{c,rain}}=\left(p_{\mathrm{c,rain}}+p_{\mathrm{l,rain}}\right)\left(1-\Delta_{\mathrm{p,rain}}\right)+L_{\mathrm{dew,rain}}-S_{\mathrm{c,rain}}
    \end{equation}
    \begin{equation}
      D_{\mathrm{c,snow}}=\left(p_{\mathrm{c,snow}}+p_{\mathrm{l,snow}}\right)\left(1-\Delta_{\mathrm{p,rain}}\right)+L_{\mathrm{dew,snow}}-S_{\mathrm{c,snow}}
    \end{equation}

\end{enumerate}

\chapter{积雪和土壤中水分的垂直运动}\label{积雪和土壤中水分的垂直运动}

\begin{mymdframed}{代码}
  本章对应的代码文件为\texttt{MOD\_SoilSnowHydrology.F90}.
\end{mymdframed}

CoLM中地表的水分平衡分为无积雪和有积雪两种情形。

当无积雪层时,
\begin{equation}
  \Delta W_{\mathrm{ponding}} = p_{\mathrm{g,rain}}+S_{\mathrm{melt}}+q_{\mathrm{sdew}}-q_{\mathrm{seva}}-r_{\mathrm{sur}}-q_{\mathrm{infl}}
\end{equation}
其中$\Delta W_{\mathrm{ponding}}$为地表积水的变化,$p_{\mathrm{g,rain}}$为经过植被截留后到达地面的降水,$S_{\mathrm{melt}}$为积雪融化后的水分,$q_{\mathrm{sdew}}$为凝结的水分,$q_{\mathrm{seva}}$为土壤表面的蒸发,$r_{\mathrm{sur}}$为地表产流,$q_{\mathrm{infl}}$为入渗到土壤中的水分。

当有积雪层时,
\begin{equation}
  \Delta W_{\mathrm{ponding}} = S_{\mathrm{melt}}-r_{\mathrm{sur}}-q_{\mathrm{infl}}
\end{equation}
其中$\Delta W_{\mathrm{ponding}}$为地表积水的变化,$S_{\mathrm{melt}}$为由积雪空隙到达地表的液态水,$r_{\mathrm{sur}}$和$q_{\mathrm{infl}}$同上。

积雪中的液态水平衡为,
\begin{equation}
  \Delta W_{\mathrm{snow}} = p_{\mathrm{g,rain}}+q_{\mathrm{sdew}}-S_{\mathrm{melt}}-q_{\mathrm{seva}}
\end{equation}
其中$\Delta W_{\mathrm{snow}}$为积雪空隙中液态水的变化,$p_{\mathrm{g,rain}}$为经过植被截留后到达地面的降水,$q_{\mathrm{sdew}}$为凝结的水分,$S_{\mathrm{melt}}$为由积雪空隙到达地表的液态水,$q_{\mathrm{seva}}$为积雪表面的蒸发。

积雪中的液态水受毛细管力和重力的共同作用而流动,但由于毛细管力比重力小两个以上数量级,在计算的时候可以忽略。
水流通量一般表达为$K\times{\rm ss}^3$,其中$K$为导水率,${\rm ss}$为液态水在孔隙中的饱和度。由于没有有效的参数化方案来计算导水率$K$,
模式中采用简化的方案来近似液态水在积雪中的流动:当某一层中的液态水的含量超过了这一层的持水能力的时候,多余的液态水自这一层流入其下的一层。
某一层可保持的液态水的最大体积百分比计算为$ssi\cdot\left(1-f_{\mathrm{ice}}\right)$,其中,$f_{\mathrm{ice}}$为冰的体积百分比,$ssi$为束缚水饱和度,模式中取常数$ssi=0.033$。

土壤水的垂直运动受地表入渗、重力、土壤水吸力或压力以及植被吸水等过程的共同作用,CoLM中用Richards方程进行描述,
\begin{equation}
  \frac{\partial \theta}{\partial t}=-\frac{\partial q}{\partial z}-S
\end{equation}
其中$z$为土壤深度,取土壤表面为0,垂直向下为正方向;$\theta$为土壤体积含水量,$q$为土壤水通量;$S$为源汇项,主要包含由植被水力过程和地下水的侧向运动引起的水分变化。

土壤水通量$q$在模式中用Buckingham-Darcy定律来描述,
\begin{equation}
  q=-K \frac{\partial}{\partial z}(\psi-z)
\end{equation}
其中$\psi$为土壤水势,$K$为土壤导水率。

为了使Richards方程闭合,需使用土壤水含量$\theta$,土壤水势$\psi$和土壤导水率$K$三者之间关系的经验公式,称为土壤水力特征曲线。CoLM中可使用~\citet{campbell1974} 和~\citet{van1980closed} 建立的两种土壤水力特征曲线。

\citet{campbell1974}建立的曲线为,
\begin{equation}\label{eq:SW_CB}
  \psi=\psi_{\mathrm{sat}}\left(\frac{\theta}{\theta_{\mathrm{sat}}}\right)^{-B}
\end{equation}
\begin{equation}\label{eq:Ks_CB}
  K=K_{\mathrm{sat}}\left(\frac{\theta}{\theta_{\mathrm{sat}}}\right)^{2 B+3}
\end{equation}
其中$\psi_{\mathrm {sat}} $为饱和土水势,$\theta_{\mathrm{sat}}$为饱和土壤体积含水量,$K_{\mathrm{sat}}$为饱和导水率,$B$为曲线参数。

\citet{van1980closed} 建立的曲线为
\begin{equation}\label{eq:SW_VG}
  \Theta = \frac{\theta-\theta_{\mathrm {r}} }{\theta_{\mathrm {sat}} -\theta_{\mathrm {r}} } = \left[\frac{1}{1+\left(\alpha h\right)^n}\right]^{1-1/n}
\end{equation}
\begin{equation}\label{eq:Ks_VG}
  K = K_{\mathrm {sat}}  \Theta^L \left[1-\left(1-\Theta^{1/\left(1-1/n\right)}\right)^{1-1/n}\right]^2
\end{equation}
其中$h$为土壤吸力势,$\theta_{\mathrm {r}} $为残余含水量,$\theta_{\mathrm {sat}} $为饱和含水量,$K_{\mathrm{sat}}$为饱和导水率,$\alpha$、$L$ 和 $n$为曲线参数。

CoLM中可选两个数值方案进行Richards方程的求解:1. 2014版CoLM中的方案;2. 可变饱和流算法。

\section{2014版Richards方程求解方案}

{
  \begin{figure}[htb]
    \centering
    \includegraphics[width=0.8\textwidth]{Figures/植被冠层和土壤水分/土壤水离散.pdf}
    \caption{离散Richards方程中各项的定义}
    \label{fig:土壤水离散}
  \end{figure}
}

\subsection{基于土壤湿度的Richards方程求解方案}

首先建立离散的Richards方程。土壤的具体分层方案见章节~\ref{土壤和积雪的垂直分层},离散后变量的定义见示意图~\ref{fig:土壤水离散}。在第$i$层上,对Richards方程进行空间上的积分可得
\begin{equation}
  \Delta z_{i} \frac{\partial}{\partial t} \theta_{i}=-\left(q_{i+\frac{1}{2}}-q_{i-\frac{1}{2}}\right)-S_{i}
\end{equation}
\begin{equation}
  q_{i+\frac{1}{2}}=-K_{i+\frac{1}{2}}\left(\frac{\psi_{i+1}-\psi_{i}}{\Delta z_{i+\frac{1}{2}}}-1\right)
\end{equation}
其中$\Delta {z_i}$为第$i$层的厚度,$\theta_i$为第$i$层的平均土壤含水量,
$q_{i+\frac{1}{2}}$为第$i$层和第$i+1$层之间的土壤水通量,$K_{i+\frac{1}{2}}$为第$i$层和第$i+1$层之间的等效导水率,$\Delta z_{i+\frac{1}{2}}$表示从第$i$层到第$i+1$层中心的距离,$S_i$为第$i$层内的源汇项。在时间上采用隐格式,可得
\begin{equation}
  \Delta z_{i} \frac{\theta_{i}^{n+1}-\theta_{i}^{n}}{\Delta t}=-\left(q_{i+\frac{1}{2}}^{n+1}-q_{i-\frac{1}{2}}^{n+1}\right)-S_{i}
\end{equation}
\begin{equation}
  q_{i+\frac{1}{2}}^{n+1}=-K_{i+\frac{1}{2}}^{n+1}\left(\frac{\psi_{i+1}^{n+1}-\psi_{i}^{n+1}}{\Delta z_{i+\frac{1}{2}}}-1\right)
\end{equation}
为了简化计算,将$q_{i+\frac{1}{2}}^{n+1}$表达式中的各项在$\theta_i^n$附近做一阶近似,可得
\begin{equation}
  \begin{aligned}
    K_{i+\frac{1}{2}}^{n+1} &\approx K_{i+\frac{1}{2}}^{n}+\frac{\partial K_{i+\frac{1}{2}}}
    {\partial \theta_{i}}\left(\theta_{i}^{n+1}-\theta_{i}^{n}\right)+\frac{\partial K_{i+\frac{1}{2}}}
    {\partial \theta_{i+1}}\left(\theta_{i+1}^{n+1}-\theta_{i+1}^{n}\right) \\
    \psi_{i+1}^{n+1} &\approx \psi_{i+1}^{n}+\frac{\partial \psi_{i+1}}{\partial \theta_{i+1}}\left(\theta_{i+1}^{n+1}-\theta_{i+1}^{n}\right) \\
    \psi_{i}^{n+1} &\approx \psi_{i}^{n}+\frac{\partial \psi_{i}}{\partial \theta_{i}}\left(\theta_{i}^{n+1}-\theta_{i}^{n}\right)
  \end{aligned}
\end{equation}

记$\Delta \theta_i=\theta_i^{n+1}-\theta_i^n$,则
\begin{equation}
  \begin{split}
    q_{i+\frac{1}{2}}^{n+1} \approx &-\left(K_{i+\frac{1}{2}}^{n}+\frac{\partial K_{i+\frac{1}{2}}}{\partial \theta_{i}} \Delta \theta_{i} + \frac{\partial K_{i+\frac{1}{2}}}{\partial \theta_{i+1}} \Delta \theta_{i+1}\right)  \\
    & \times \left[\frac{1}{\Delta z_{i+\frac{1}{2}}}\left(\psi_{i+1}^{n}+\frac{\partial \psi_{i+1}}{\partial \theta_{i+1}} \Delta \theta_{i+1}-\psi_{i}^{n}-\frac{\partial \psi_{i}}
    {\partial \theta_{i}} \Delta \theta_{i}\right)-1\right] \\
    \approx & -K_{i+\frac{1}{2}}^{n}\left[\frac{1}{\Delta z_{i+\frac{1}{2}}}\left(\psi_{i+1}^{n}-\psi_{i}^{n}\right)-1\right] \\
    &-\left\{-K_{i+\frac{1}{2}}^{n} \frac{1}{\Delta z_{i+\frac{1}{2}}} \frac{\partial \psi_{i}}{\partial \theta_{i}}+\frac{\partial K_{i+\frac{1}{2}}}{\partial
    \theta_{i}}\left[\frac{1}{\Delta z_{i+\frac{1}{2}}}\left(\psi_{i+1}^{n}-\psi_{i}^{n}\right)-1\right]\right\} \Delta \theta_{i} \\
    &-\left\{K_{i+\frac{1}{2}}^{n} \frac{1}{\Delta z_{i+\frac{1}{2}}} \frac{\partial \psi_{i+1}}{\partial \theta_{i+1}}+\frac{\partial K_{i+\frac{1}{2}}}{\partial
    \theta_{i+1}}\left[\frac{1}{\Delta z_{i+\frac{1}{2}}}\left(\psi_{i+1}^{n}-\psi_{i}^{n}\right)-1\right]\right\} \Delta \theta_{i+1}
  \end{split}
\end{equation}
其中第一个约等号是对各项做一阶近似,第二个约等号是舍弃二阶项。对$q_{i-\frac{1}{2}}^{n+1}$也可做同样的近似。从而,完全离散后的Richards方程可表达为,
\begin{equation}
  a_i \Delta \theta_{i-1}+b_i \Delta \theta_{i}+c_i \Delta \theta_{i+1}=r_i
\end{equation}
其中,
\begin{equation}
  \begin{aligned}
    a_i &= - K_{i-\frac{1}{2}}^{n} \frac{1}{\Delta z_{i-\frac{1}{2}}}
    \frac{\partial \psi_{i-1}}{\partial \theta_{i-1}}+\frac{\partial K_{i-\frac{1}{2}}}
    {\partial \theta_{i-1}}\left[\frac{1}{\Delta z_{i-\frac{1}{2}}}\left(\psi_{i}^{n}-\psi_{i-1}^{n}\right)-1\right] \\
    b_i &= \frac{\Delta z_{i}}{\Delta t}+K_{i+\frac{1}{2}}^{n} \frac{1}{\Delta z_{i+\frac{1}{2}}}
    \frac{\partial \psi_{i}}{\partial \theta_{i}}-\frac{\partial K_{i+\frac{1}{2}}}{\partial \theta_{i}}
    \left[\frac{1}{\Delta z_{i+\frac{1}{2}}}\left(\psi_{i+1}^{n}-\psi_{i}^{n}\right)-1\right] \\
    &\mathrel{\phantom{=}}+K_{i-\frac{1}{2}}^{n} \frac{1}{\Delta z_{i-\frac{1}{2}}} \frac{\partial \psi_{i}}{\partial
    \theta_{i}}+\frac{\partial K_{i-\frac{1}{2}}}{\partial \theta_{i}}\left[\frac{1}{\Delta z_{i-\frac{1}{2}}}\left(\psi_{i}^{n}-\psi_{i-1}^{n}\right)-1\right] \\
    c_i &= - K_{i+\frac{1}{2}}^{n} \frac{1}{\Delta z_{i+\frac{1}{2}}}
    \frac{\partial \psi_{i+1}}{\partial \theta_{i+1}} - \frac{\partial K_{i+\frac{1}{2}}}{\partial \theta_{i+1}}\left[\frac{1}
    {\Delta z_{i+\frac{1}{2}}}\left(\psi_{i+1}^{n}-\psi_{i}^{n}\right)-1\right] \\
    r_i &= K_{i+\frac{1}{2}}^{n}
    \left[\frac{1}{\Delta z_{i+\frac{1}{2}}}\left(\psi_{i+1}^{n}-\psi_{i}^{n}\right)-1\right] - K_{i-\frac{1}{2}}^{n}
    \left[\frac{1}{\Delta z_{i-\frac{1}{2}}}\left(\psi_{i}^{n}-\psi_{i-1}^{n}\right)-1\right]-S_{i}
  \end{aligned}
\end{equation}

对最上面的土壤分层(第1层),使用给定的入渗通量$q_{\mathrm{infl}}$,则
\begin{equation}
  \begin{aligned}
    a_1 &= 0 \\
    b_1 &= \frac{\Delta z_1}{\Delta t}+K_{1+\frac{1}{2}}^{n}
    \frac{1}{\Delta z_{1+\frac{1}{2}}} \frac{\partial \psi_{1}}{\partial \theta_{1}}-\frac{\partial K_{1+\frac{1}{2}}}{\partial \theta_{1}}\left[\frac{1}{\Delta z_{1+\frac{1}{2}}}
    \left(\psi_{2}^{n}-\psi_{1}^{n}\right)-1\right] \\
    c_1 &= -K_{1+\frac{1}{2}}^{n} \frac{1}{\Delta z_{1+\frac{1}{2}}}
    \frac{\partial \psi_{2}}{\partial \theta_{2}}-\frac{\partial K_{1+\frac{1}{2}}}{\partial \theta_{2}}\left[\frac{1}{\Delta z_{1+\frac{1}{2}}}
    \left(\psi_{2}^{n}-\psi_{1}^{n}\right)-1\right] \\
    r_1 &= q_{\mathrm{infl}}+K_{1+\frac{1}{2}}^{n}
    \left[\frac{1}{\Delta z_{1+\frac{1}{2}}}\left(\psi_{2}^{n}-\psi_{1}^{n}\right)-1\right]-S_{1}
  \end{aligned}
\end{equation}

对最下面的土壤分层(第$I$层),假定边界条件为重力排水边界条件,即$q_{I+\frac{1}{2}}^{n+1}=K_{I}^{n+1}$,则
\begin{equation}
  \begin{aligned}
    a_I &= - K_{I-\frac{1}{2}}^{n} \frac{1}{\Delta z_{I-\frac{1}{2}}}
    \frac{\partial \psi_{I-1}}{\partial \theta_{I-1}}+\frac{\partial K_{I-\frac{1}{2}}}
    {\partial \theta_{I-1}}\left[\frac{1}{\Delta z_{I-\frac{1}{2}}}\left(\psi_{I}^{n}-\psi_{I-1}^{n}\right)-1\right] \\
    b_I &= \frac{\Delta z_{I}}{\Delta t}+\frac{\partial K_{I}}
    {\partial \theta_{I}}+K_{I-\frac{1}{2}}^{n} \frac{1}{\Delta z_{I-\frac{1}{2}}} \frac{\partial \psi_{I}}{\partial \theta_{I}}+
    \frac{\partial K_{I-\frac{1}{2}}}{\partial \theta_{I}}\left[\frac{1}{\Delta z_{I-\frac{1}{2}}}\left(\psi_{I}^{n}-\psi_{I-1}^{n}\right)-1\right] \\
    c_I &= 0 \\
    r_I &= -K_{I}^{n}-K_{I-\frac{1}{2}}^{n}\left[\frac{1}{\Delta z_{I-\frac{1}{2}}}\left(\psi_{I}^{n}-\psi_{I-1}^{n}\right)-1\right]-S_{I}
  \end{aligned}
\end{equation}

求解上述方程组,即可得到土壤水含量的变化量。方程组的系数矩阵为三对角阵,矩阵中各项均定义在第$n$步,可显式计算出,所得代数方程组可用追赶法快速求解。

在2014版Richards方程求解方案中,仅使用~\citet{campbell1974}建立的土壤水力特征曲线。$K_{i+\frac{1}{2}}$是两层土壤土壤之间的等效导水率,若采用上游格式,其定义为
\begin{equation}
  K_{i+\frac{1}{2}}=
  \begin{cases}
    K_{\mathrm{sat},i}\left(\frac{\theta_{i}}{\theta_{\mathrm{sat},i}}\right)^{2 B_i+3}
    & q_{i+\frac{1}{2}} \geqslant 0 \quad \mbox{(水流方向向下)}\\
    K_{\mathrm{sat},i +1}\left(\frac{\theta_{i+1}}{\theta_{\mathrm{sat},i +1}}\right)^{2 B_{i+1}+3}
    & q_{i+\frac{1}{2}} < 0 \quad \mbox{(水流方向向上)}
  \end{cases}
\end{equation}
在系数矩阵中,需要对土壤水势和等效导水率求导数。对土壤水势,容易计算得到
\begin{equation}
  \frac{\partial \psi_{i}}{\partial \theta_{i}}=-B_i \frac{\psi_{i}}{\theta_{i}}
\end{equation}
对等效导水率,则导数为
\begin{align}
  \frac{\partial K_{i+\frac{1}{2}}}{\partial \theta_{i}} & =
  \begin{cases}
    \left(2B_i+3\right) K_{\mathrm{sat},i}\left(\frac{\theta_{i}}{\theta_{\mathrm{sat},i}}\right)^{2B_i+2}
    \frac{1}{\theta_{\mathrm{sat},i}}, & q_{i+\frac{1}{2}} \geqslant 0 \\
    0, & q_{i+\frac{1}{2}} < 0
  \end{cases}
  \\
  \frac{\partial K_{i+\frac{1}{2}}}{\partial \theta_{i+1}} & =
  \begin{cases}
    0, & q_{i+\frac{1}{2}} \geqslant 0 \\
    (2B_{i+1}+3) K_{\mathrm{sat},i +1}\left(\frac{\theta_{i+1}}{\theta_{\mathrm{sat},i +1}}\right)^{2B_{i+1}+2} \frac{1}{\theta_{\mathrm{ {sat},i +1}}}, & q_{i+\frac{1}{2}} < 0
  \end{cases}
\end{align}
对计算区域的底部(第$I$层),
\begin{equation}
  K_{I}=K_{\mathrm{sat},I}\left(\frac{\theta_{I}}{\theta_{\mathrm{sat},I}}\right)^{2B_I+3}, \quad \frac{\partial K_{I}}
  {\partial \theta_{I}} = (2B_I+3) K_{\mathrm{sat},I}\left(\frac{\theta_{I}}{\theta_{\mathrm{sat},I}}\right)^{2B_I+2} \frac{1}{\theta_{\mathrm{sat},I}}
\end{equation}

有冰存在的时候,需考虑冰对导水率的影响,使用如下经验公式,
\begin{equation}
  {f}_{\mathrm{imped},i+\frac{1}{2}}=10^{-e_{\mathrm{ice}}\cdot\frac{f_{\mathrm{ice},i}+f_{\mathrm{ice},i +1}}{2}}
\end{equation}
其中$e_{\mathrm{ice}}=6.0$为冰的阻抗因子。对等效导水率及其导数的限制为
\begin{equation}
  \begin{aligned}
    K_{i+\frac{1}{2}} & = {f}_{\mathrm{imped},i+\frac{1}{2}} \cdot K_{i+\frac{1}{2}} \\
    \frac{\partial K_{i+\frac{1}{2}}}{\partial \theta_{i}}
    & = {f}_{\mathrm{imped},i+\frac{1}{2}} \cdot \frac{\partial K_{i+\frac{1}{2}}}{\partial \theta_{i}} \\
    \frac{\partial K_{i+\frac{1}{2}}}{\partial \theta_{i+1}} & =
    {f}_{\mathrm{imped},i+\frac{1}{2}} \cdot \frac{\partial K_{i+\frac{1}{2}}}{\partial \theta_{i+1}}
  \end{aligned}
\end{equation}

\subsection{饱和带和非饱和带的水分交换} \label{sec:exchange_sat_unsat}
模式中使用两个变量$w_{\mathrm {a}} $和$z_{\mathrm{wt}}$来描述饱和含水层的状态,分别代表总水量和地下水位。饱和含水层中的水分可与土壤层中的水分在垂直方向进行交换,也可在水平方向进行流动。其中水平方向的流动被称为地下径流,相关内容见第六部分水文过程。

土壤层与饱和含水层的交换通量定义为土壤层底部的水流通量,
\begin{equation}
  {q}_{\mathrm{recharge}}=q_{I+\frac{1}{2}}^{n+1}=K_{I}^{n+1} = K_I^n + \frac{\partial K_{I}}{\partial \theta_{I}} \Delta \theta_{I}
\end{equation}

含水层总水量变化为
\begin{equation}
  w_{\mathrm{a}}=w_{\mathrm{a}}+q_{\mathrm{recharge}} \cdot \Delta t
\end{equation}

给水度$S_{\mathrm {y}} $~(specific yield)反映了含水层水量变化时地下水位的变化。假设地下水位之上的土壤为均质土壤,且土壤水处于平衡态,则其土壤水势可表达为深度的函数,
\begin{equation}
  \psi \left(z\right) = \psi_{\mathrm {sat}}  - \left(z_{\mathrm{wt}} - z\right)
\end{equation}
地下水位之上的土壤中所含水分的总量为
\begin{equation}
  w_{\mathrm{unsat}} = \int^{z_{\mathrm{wt}}}_0 \theta\left[\psi\left(z\right)\right] \mathrm{d}z = \int^{z_{\mathrm{wt}}}_0 \theta\left[\psi_{\mathrm {sat}}  - \left(z_{\mathrm{wt}} - z\right)\right] \mathrm{d}z
\end{equation}
使用~\citet{campbell1974}建立的土壤水力特征曲线时,
\begin{equation}
  w_{\mathrm{unsat}} = \int^{z_{\mathrm{wt}}}_0 \theta_{\mathrm {sat}} \left[\frac{\psi_{\mathrm {sat}}  - \left(z_{\mathrm{wt}} - z\right)}{\psi_{\mathrm {sat}} }\right]^{-\frac{1}{B}} \mathrm{d}z
\end{equation}
假设饱和含水层最大到$z_{\mathrm{max}}$,则地下水位之下的水分总量为
\begin{equation}
  w_{\mathrm{sat}} = \theta_{\mathrm {sat}} \left(z_{\mathrm{max}}-z_{\mathrm{wt}}\right)
\end{equation}
土壤$0$到$z_{\mathrm{max}}$深度的总水量为
\begin{equation}
  w_{\mathrm{total}} = w_{\mathrm{unsat}} + w_{\mathrm{sat}} = \int^{z_{\mathrm{wt}}}_0 \theta_{\mathrm {sat}} \left[\frac{\psi_{\mathrm {sat}}  - \left(z_{\mathrm{wt}} - z\right)}{\psi_{\mathrm {sat}} }\right]^{-\frac{1}{B}} \mathrm{d}z + \theta_{\mathrm {sat}} \left(z_{\mathrm{max}}-z_{\mathrm{wt}}\right)
\end{equation}
对$z_{\mathrm{wt}}$求导数可得给水度$S_{\mathrm {y}} $,
\begin{equation}
  {sat}_{\mathrm{{y}}} = -\frac{w_{\mathrm{total}}}{z_{\mathrm{wt}}} = \theta_{\mathrm{sat}}\left[1-\left(1-\frac{z_{\mathrm{w t}}}{\psi_{\mathrm{sat}}}\right)^{-\frac{1}{B}}\right] \label{eqn:s_y}
\end{equation}
这里取符号是因为深度的正方向向下,总水量越多,则地下水位越浅。不难发现,$S_{\mathrm {y}} $等于地下水位位于$z_{\mathrm{wt}}$且土壤水处于平衡状态时,土壤表层($z=0$)的空气体积含量百分比,这是因为在平衡态的假设下,水分的增加会使得含水量曲线整体向上移动,等效于最上层干旱部分被地下水位附近的饱和部分取代。

若地下水位位于最下层土壤之下,公式(\ref{eqn:s_y})中的参数$\theta_{\mathrm {sat}} $、$\psi_{\mathrm {sat}} $ 和 $B$取最下层土壤的值,则水位的变化为
\begin{equation}
  z_{\mathrm{w t}}=z_{\mathrm{w t}}-\frac{q_{\mathrm{recharge}} \cdot \Delta t}{S_{\mathrm{y}}}
\end{equation}

若地下水位位于分层土壤之内,则需要逐层计算水位的变化。当$q_{\mathrm{recharge}}>0$时,记地下水位所在层为$j$,
\begin{equation}
  {sat}_{\mathrm{y},j}=\theta_{\mathrm{sat},j}\left[1-\left(1-\frac{z_{\mathrm{wt}}}{\psi_{\mathrm{sat},j}}\right)^{-\frac{1}{B_j}}\right]
\end{equation}
在第$j$层可容纳水量的限制下,实际可补给到第$j$层的水量为($z_{j-\frac{1}{2}}$为第$j$层的上边界)
\begin{equation}
  q_{\mathrm{recharge},j}=\min \left[q_{\mathrm{recharge}}, \frac{S_{\mathrm{y},j} \cdot \left(z_{\mathrm{w t}}-z_{j-\frac{1}{2}}\right)}{\Delta t} \right]
\end{equation}
地下水位在第$j$层内的变化为
\begin{equation}
  z_{\mathrm{w t}}=z_{\mathrm{w t}}-\frac{q_{\mathrm{recharge},j} \cdot \Delta t}{S_{\mathrm{y},j}}
\end{equation}
若$q_{\mathrm{recharge},j}<q_{\mathrm{recharge}}$,则此时地下水位上升至第$j-1$层,补给水未用尽,可继续对第$j-1$层进行补给,直至补给水用尽或者到达土壤表面。

当$q_{\mathrm{recharge}}<0$时,水分从饱和含水层向上层土壤进行补给,$w_{\mathrm {a}} $减少,$z_{\mathrm{wt}}$增加,除方向相反外,计算过程与从上层土壤向饱和含水层的补给类似,这里不再赘述。

\subsection{由地下产流引起的地下水和土壤水变化}

地下产流主要指由地形引起的地下水的侧向流动。当不考虑基于物理过程的侧向流时,模式中通过~\ref{section:rsub_par}节中的参数化方案来计算地下产流的量,这部分水分需从地下水和土壤水中扣除。

假设地下产流的量为$r_{\mathrm{sub}}$。
地下水位位于分层土壤之内时,需要逐层计算水位的变化。记起始时地下水位$z_{\mathrm{wt}}$所在层为$j$,则实际可从第$j$层中扣除的水量为($z_{j+\frac{1}{2}}$为第$j$层的下边界)
\begin{equation}
  r_{\mathrm{sub}},j = \min \left[r_{\mathrm{sub}}, \frac{S_{\mathrm{y},j} \cdot \left(z_{j+\frac{1}{2}} - z_{\mathrm{w t}}\right)}{\Delta t} \right]
\end{equation}
其中$S_{\mathrm{y},j}$为第$j$层内的给水度,由~\eqref{eqn:s_y}计算。地下水位在第$j$层内的变化为
\begin{equation}
  z_{\mathrm{w t}}=z_{\mathrm{w t}}+\frac{r_{\mathrm{sub},j} \cdot \Delta t}{S_{\mathrm{y},j}}
\end{equation}
土壤水的变化为
\begin{equation}
  w_{\mathrm{liq},j} =  w_{\mathrm{liq},j} - r_{\mathrm{sub},j} \cdot \Delta t
\end{equation}
若$r_{\mathrm{sub},j}<r_{\mathrm{sub}}$,则此时地下水位下降到第$j+1$层,$r_{\mathrm{sub}}$更新为$r_{\mathrm{sub}}-r_{\mathrm{sub},j}$,继续从第$j+1$层进行扣除,直至$r_{\mathrm{sub}}=0$或地下水位降到了最下层土壤之下.

若地下水位降到了最下层土壤之下,公式~\eqref{eqn:s_y}中的参数$\theta_{\mathrm {sat}} $、$\psi_{\mathrm {sat}} $ 和 $B$取最下层土壤的值,则含水层总水量和水位的变化为
\begin{equation}
  \begin{aligned}
    w_{\mathrm{a}} & = w_{\mathrm{a}} - r_{\mathrm{sub}} \cdot \Delta t \\
    z_{\mathrm{wt}} & = z_{\mathrm{w t}}+\frac{r_{\mathrm{sub}} \cdot \Delta t}{S_{\mathrm{y}}}
  \end{aligned}
\end{equation}

\section{可变饱和流数值算法}

\begin{mymdframed}{代码}
  本节对应的代码文件为\texttt{HYDRO/MOD\_Hydro\_SoilWater.F90}.
\end{mymdframed}

可变饱和流数值算法(Variably Saturated Flow, VSF)\citep{dai2019vsf}在多个方面对Richards方程的数值算法进行了改进。

{
  \begin{figure}[htbp]
    \centering
    \includegraphics{Figures/陆地表面的水分循环/可变饱和流数值算法预报区域空间结构示意图.png}
    \caption{可变饱和流数值算法空间离散方法示意图}
    \label{fig:可变饱和流数值算法预报区域空间结构示意图}
  \end{figure}
}

首先,引入了两个新的预报变量(地下水位,湿润锋面的位置)来追踪饱和区的边界(图 \ref{fig:可变饱和流数值算法预报区域空间结构示意图})。借助于这两个新的预报变量,在非饱和区,Richards方程具有类似抛物型方程的性质,将含水量作为主要预报变量;而在饱和区,Richards方程具有类似椭圆型方程的性质,仅需确定区域边界上的水势压力便可求解区域内部均一的水流通量。这一方法属于“土壤水势-含水量”混合形式Richards方程的范畴,但相比传统的混合形式Richards方程,能更清晰地刻画土壤水运动在不同区域内的物理和数学性质。

其次,可变饱和流算法使用了一个新的土壤层间等效导水率计算公式。与常用的简单导水率计算公式相比,新公式具有更高的精度。新计算公式与隐式时间积分格式结合,可导出理论上无数值振荡的离散格式,提高了数值解的稳定性。

再者,可变饱和流算法使用了混合隐式-显式时间积分方案和自适应时间步长。隐式时间积分的求解需要在单步内进行迭代,当迭代算法失效或者时间步长较小时,使用一个满足质量守恒的显式积分格式进行替代。自适应时间步长可在精度和效率之间做出合理的平衡。

最后,可变饱和流数值算法将地表入渗、土壤水运动和土壤水与地下水的相互交换这三个垂直方向的水流运动进行统一求解。侧向的地表水运动(坡面流)将地表水深作为预报变量,侧向的地下水运动(基流)将地下水位作为预报变量,在模拟土壤水运动时,也使用了这两个变量,实现了垂直方向和水平方向的变量统一,使陆面模型在物理上更协调。


\subsection{土壤水预报方程}
在每个土壤层内,可使用三个预报变量来描述土壤中液态水的分布情况。
土壤体积含水量为第一个预报变量($\theta_i$)。为了追踪土壤中饱和水层的变化(包括地下水、滞水层和入渗情况下由地表向下发展的饱和水层),
每层土壤中引入了另外两个潜在的预报变量($w_{\mathrm{f},i}$和$w_{\mathrm{t},i}$),
分别代表土壤层内上部饱和区域的厚度,和土壤层内下部饱和区域的厚度(图~\ref{fig:可变饱和流数值算法预报区域空间结构示意图}),
当某层土壤部分饱和时,这两个变量可被激活。
{
  \begin{figure}[htbp]
    \centering
    \includegraphics{Figures/陆地表面的水分循环/饱和-非饱和过渡状态的土壤层.png}
    \caption[饱和-非饱和过渡状态的土壤层]{饱和-非饱和过渡状态的土壤层。如果土壤层内上部有饱和区(左图),
    层内饱和区的厚度作为预报变量被激活;如果土壤层内下部有饱和区(右图),层内饱和区的厚度作为预报变量被激活}
    \label{fig:饱和-非饱和过渡状态的土壤层}
  \end{figure}
}

对预报变量$\theta_i$,离散后的预报方程为
\begin{equation}\label{si_in1}
  \left(\Delta z_{i}-w_{\mathrm{f},i}^{n+1}-w_{\mathrm{t},i}^{n+1}\right) \cdot\left(\theta_{i}^{n+1}-\theta_{i}^{n}\right)=\Delta t \cdot\left(q_{\mathrm{ {uin,i }}}^{n+1}-q_{\mathrm{uout},i}^{n+1}\right)
\end{equation}
其中,$\Delta z_i$ 为第$ i $层土壤的厚度,$\Delta t$ 为时间步长,$q_{\mathrm{uin},i}^{n+1}$为非饱和区域上边界处的水流通量,$q_{\mathrm{uout},i}^{n+1}$为非饱和区域下边界处的水流通量。


预报变量$w_{\mathrm{f},i}$表示土壤层上部饱和区域的厚度,其离散后的预报方程为
\begin{equation}\label{si_in2}
  \left(\theta_{\mathrm{sat},i}-\theta_{i}^{n}\right) \cdot\left(w_{\mathrm{f},i}^{n+1}-w_{\mathrm{f},i}^{n}\right)=\Delta t \cdot\left(q_{i-1/2}^{n+1}-q_{\mathrm{w f},i}^{n+1}\right)
\end{equation}
其中,$\theta_{\mathrm{sat},i}$ 为第$ i$ 层土壤的饱和体积含水量,$ q_{{i-1/2}}^{n+1}$为第$i$层土壤上边界处的水流通量,$q_{\mathrm{wf},i}^{n+1}$为饱和区域下边界处(湿润锋面)的水流通量。


预报变量$w_{\mathrm{t},i}$表示土壤层下部饱和区域的厚度,其离散后的预报方程为
\begin{equation}\label{si_in3}
  \left(\theta_{\mathrm{sat},i}-\theta_{i}^{n}\right) \cdot\left(w_{\mathrm{t},i}^{n+1}-w_{\mathrm{t},i}^{n}\right)=\Delta t \cdot\left(q_{\mathrm{w t},i}^{n+1}-q_{i+1 / 2}^{n+1}\right)
\end{equation}
其中,$q_{\mathrm{wt},i}^{n+1}$  为饱和区域上边界处的水流通量,$q_{i+1/2}^{n+1}$为第$i$层土壤下边界处的水流通量。

当三个变量同时被激活时,满足$q_{\mathrm{ {uin,i }}}^{n+1}=q_{\mathrm{w f},i}^{n+1},~ q_{\mathrm{ {uout,i }}}^{n+1}=q_{\mathrm{wt},i}^{n+1}$,将~\eqref{si_in1}\eqref{si_in2}\eqref{si_in3}相加可得,
\begin{equation}\label{si_in4}
  \begin{aligned}
    m_{\mathrm{total}}^n & \triangleq \theta_{\mathrm{sat},i}\left(w_{\mathrm{f},i}^n+w_{\mathrm{t},i}^{n}\right)+\theta_{i}^{n}\left(\Delta z - w_{\mathrm{f},i}^{n} - w_{\mathrm{t},i}^{n}\right) \\
    m_{\mathrm{total}}^{n+1} - m_{\mathrm{total}}^n & =\Delta t \cdot\left(q_{i-1/2}^{n+1}-q_{i+1 / 2}^{n+1}\right)
  \end{aligned}
\end{equation}
式~\eqref{si_in4}表示了第$i$层内的水量守恒。

\subsection{地表积水的变化}
为了预报由入渗所产生的地表水深的变化,地表积水深度($h_{\mathrm{pond}}$)也是算法中的预报变量,其预报方程为
\begin{equation}\label{hpond}
  h_{\mathrm{ {pond }}}^{n+1}-h_{\mathrm{ {pond }}}^{n}=\Delta t \cdot\left(q_{\mathrm{ {sur }}}^{n+1}-q_{\mathrm{ {infl }}}^{n+1}\right)
\end{equation}
其中,$q_{\mathrm{sur}}^{n+1} $ 为从地表水上部进入的水流通量,
其可以是降水、蒸发、地表水与积雪的液态水交换或者它们的组合;$q_{\mathrm{infl}}^{n+1}$为从地表进入土壤的水流通量。


当地表有积水,或者由于进入地表的水流通量大于入渗到土壤中的通量而形成积水时,积水深度的预报方程(\ref{hpond})也加入到土壤水的方程组中,进行统一求解。


\subsection{地下水的变化}
当地下水位在土壤水计算区域内部时,其位置由$w_{\mathrm{t},i}$进行预报。

当地下水位处于土壤水计算区域之下时,使用预报变量$w_{\mathrm {a}} $来表示计算区域之下蓄水层的蓄水状态。$w_{\mathrm {a}} $定义为
\begin{equation}
  w_{\mathrm{a}}=-\int_{\mathrm{z_{\mathrm{bot}}}}^{z_{\mathrm{wt}}}\left(\theta_{\mathrm{sat}}-\theta\right){\mathrm { d}} z
\end{equation}
$w_{\mathrm {a}} $的绝对值表示土壤水计算区域之下的空气的总量(单位为:单位面积上的体积),负号表示计算区域之下液态水相对于饱和状态是亏缺的,$w_{\mathrm {a}} $的最大值为0(表示土壤水饱和)。

在垂直方向上,描述$w_{\mathrm {a}} $变化的预报方程为
\begin{equation}\label{wa_n1_n}
  w_{\mathrm{a}}^{n+1}-w_{\mathrm{a}}^{n}=\Delta t \cdot q_{\mathrm{bot}}^{n+1}
\end{equation}
当地下水位处于土壤水计算区域之下时,$w_{\mathrm {a}} $的预报方程(\ref{wa_n1_n})也加入到土壤水的方程组中,进行统一求解。

\subsection{水流通量的计算}
预报方程~\eqref{si_in1}\eqref{si_in2}\eqref{si_in3}\eqref{hpond}\eqref{wa_n1_n}中,右端项中均包含了水流通量,其计算分为以下几种情形,
\begin{enumerate}
  \item 饱和区域内部的水流通量,使用公式
    \begin{equation}\label{q_sat1}
      q_{\mathrm{sat}}=-\frac{\sum_{i=i_{1}}^{i_{2}} \Delta z_{i}}{\sum_{i=i_{1}}^{i_{2}} \frac{\Delta z_{i}}{K_{\mathrm{sat},i}}}
      \cdot \frac{h_{\mathrm{l}}-h_{\mathrm{u}}-\sum_{i=i_{1}}^{i_{2}} \Delta z_{i}}{\sum_{i=i_{1}}^{i_{2}} \Delta z_{i}}
    \end{equation}
    其中,$i_1,i_1+1,…,i_2$为从上至下连续的饱和土壤层的层号,$\Delta z_i$为第i层的厚度,
    $K_{\mathrm{sat},i}$为第$i$层的饱和土壤导水率,$h_{\mathrm {l}} $为饱和区域下边界处的土壤水势。
    当饱和区域上边界为地表时,$h_{\mathrm {u}} $取为地表积水深度$h_{\mathrm{pond}}$;
    当饱和区域上边界在土壤内部时,$h_{\mathrm {u}} $为饱和区域上边界处的土壤水势。
    公式(\ref{q_sat1})右端第一个分式计算了饱和区域的等效导水率(带权重$\Delta z_i$的调和平均),第二个分式计算了总水势的差商。公式~\eqref{si_in2}中的$q_{i-1/2}$,公式~\eqref{si_in3}中的$q_{i+1/2}$使用公式~\eqref{q_sat1}进行计算。

  \item 相邻异质不饱和土壤层之间的水流通量,使用公式
    \begin{equation}\label{qht1}
      q_{\mathrm{h t}}=q_{\mathrm{h m}}\left(z_{i}-z_{\mathrm{u}}, h_{\mathrm{u}}, h_{i}\right)=q_{\mathrm{h m}}\left(z_{\mathrm{l}}-z_{i}, h_{i}, h_{\mathrm{l}}\right)
    \end{equation}
    其中,$q_{\mathrm{ht}}$为两层土壤间的水流通量,$z_{\mathrm {u}} $为上层土壤的中心点的位置,$z_{\mathrm {l}} $为下层土壤的中心点的位置,
    $z_i$为异质土壤的交界面的位置,$h_{\mathrm {u}} $为$z_{\mathrm {u}} $处的土壤水势,$h_{\mathrm {l}} $为$z_{\mathrm {l}} $处的土壤水势,$h_i$为$z_i$处的土壤水势,$q_{\mathrm{hm}}$为计算均质土壤内水流通量的函数,它依赖于土壤层的厚度和上下边界处的土壤水势。

    公式~\eqref{qht1}的含义为土壤层界面处的水流通量等于界面上方均质土壤内的水流通量,也等于界面下方均质土壤内的水流通量,为相邻土壤层之间的衔接条件。在(\ref{qht1})第二个等号两边,仅土壤交界面处的土壤水势$h_i$为未知量,因此,可据其求解出$h_i$的值。将求解出的$h_i$的值代入到(\ref{qht1})中任一函数$q_{\mathrm{hm}}$中,可计算得到$q_{\mathrm{ht}}$的值。

    函数$q_{\mathrm{hm}}$为
    \begin{equation}
      q_{\mathrm{h m}}\left(\Delta z, h_{\mathrm{u}}, h_{\mathrm{l}}\right)=-k_{\mathrm{h m}} \cdot\left(\frac{h_{\mathrm{l}}-h_{\mathrm{u}}}{\Delta z}-1\right)
    \end{equation}
    其中,$\Delta z$为土壤层的厚度,$h_{\mathrm {u}} $,$h_{\mathrm {l}} $分别为土壤层上下两个边界处的土壤水势;
    $k_{\mathrm{hm}}$为等效水力导度,可变饱和流算法中采用带权重的几何平均,分为三种情况进行计算,对入渗情形($h_{\mathrm {u}} >h_{\mathrm {l}} $),
    \begin{equation}
      k_{\mathrm{h m}}=\frac{1}{1-\frac{h_{\mathrm{l}}-h_{\mathrm{u}}}{\Delta z}} \cdot\left[k_{\mathrm{u}}+\frac{h_{\mathrm{u}}-h_{\mathrm{l}}}{\Delta z}\left(k_{\mathrm{u}}\right)^{1-r_{0}} \cdot\left(k_{\mathrm{l}}\right)^{r_{0}}\right]
    \end{equation}
    对排水情形($h_{\mathrm {l}} -\Delta z<h_{\mathrm {u}} <h_{\mathrm {l}} $),
    \begin{equation}
      k_{\mathrm{h m}}=\left(k_{\mathrm{u}}\right)^{r} \cdot\left(k_{\mathrm{l}}\right)^{1-r}, r=\max \left(1+r_{0} \cdot \frac{h_{\mathrm{l}}}{\Delta z}, 1-r_{0}\right)
    \end{equation}
    对毛细上升情形($h_{\mathrm {u}} <h_{\mathrm {l}} -\Delta z$),
    \begin{equation}
      k_{\mathrm{h m}}=\left(k_{\mathrm{u}}\right)^{r_{0}} \cdot\left[k\left(h_{\mathrm{l}}-\Delta z\right)\right]^{1-r_{0}}
    \end{equation}
    其中,土壤水力导度$k$为土壤水势$h$的函数,上述三个公式中,$k_{\mathrm {u}} =k(h_{\mathrm {u}}  )$,$k_{\mathrm {l}} =k(h_{\mathrm {l}}  )$;
    $r_0$是依赖于土壤水力模型和土壤水力参数的参数,对Campbell模型~\eqref{eq:Ks_CB},
    \begin{equation}
      r_{0}=\frac{1}{3 \lambda+2}
    \end{equation}
    对van Genuchten--Mualem模型~\eqref{eq:Ks_VG},
    \begin{equation}
      r_{0}=\frac{1}{L(n-1)+2 n}
    \end{equation}

    当采用隐式时间积分格式时,对均质土壤,上述水流通量的计算公式可保证Richards方程的解为本质无振荡的。

    在公式~\eqref{si_in1}中,若非饱和区相邻的上层也为非饱和区,则$q_{\mathrm{uin},i}$的计算采用公式~\eqref{qht1},若非饱和区相邻的下层也为非饱和区,则$q_{\mathrm{uout},i}$的计算采用公式~\eqref{qht1}.

  \item 饱和区位于非饱和区上方时两者之间的水流通量,即湿润锋面处的水流通量,公式为
    \begin{equation}
      q_{\mathrm{wf}}=q_{\mathrm{h m}}\left(z_{\mathrm{l}}-z_{\mathrm{i}}, h_{\mathrm{sat}}, h_{\mathrm{l}}\right)
    \end{equation}
    其中,$z_{\mathrm {l}} $为下方非饱和土壤层中心点的位置,$z_i$为饱和区和非饱和区之间界面的位置,
    $h_{\mathrm {sat}} $为饱和土壤水势,$h_{\mathrm {l}} $为非饱和土壤层中心点处的土壤水势。

  \item 饱和区位于非饱和区上方时两者之间的水流通量,即饱和水位处的水流通量,公式为
    \begin{equation}
      q_{\mathrm{wt}}=q_{\mathrm{h m}}\left(z_{\mathrm{i}}-z_{\mathrm{u}}, h_{\mathrm{u}}, h_{\mathrm{sat}}\right)
    \end{equation}
    其中,$z_{\mathrm {u}} $为上方非饱和土壤层中心点的位置,$z_i$为饱和区和非饱和区之间界面的位置,
    $h_{\mathrm {sat}} $为饱和土壤水势,$h_{\mathrm {u}} $为非饱和土壤层中心点处的土壤水势。

  \item 地表入渗通量$q_{\mathrm{infl}}$的计算分两种情况。第一种情况,地表有积水或者因到达地表的水流速度较大而有形成积水的趋势时,自地表向下形成湿润锋面,则$q_{\mathrm{infl}}$等于最上层饱和区的水流速度;第二种情况,地表无积水且入渗速度大于到达地表的水流速度时,$q_{\mathrm{infl}}$等于到达地表的水流速度(可为负值,表示蒸发等情况)。

\end{enumerate}


\subsection{预报方程的求解}
方程(\ref{si_in4})代表了单层土壤内的水量守恒,\eqref{hpond}\eqref{wa_n1_n}分别代表了地表和地下含水层的水量守恒。令包含地表和地下含水层在内,总的计算层数为$A$. 在其中第$\alpha$层内可将~\eqref{si_in4}\eqref{hpond}\eqref{wa_n1_n}简写为
\begin{equation}\label{m_alpha_x}
  \delta m_{\alpha}(\vec{x})=\Delta t \cdot \delta q_{\alpha}(\vec{x})
\end{equation}
其中$\vec{x}$代表所有活动变量组成的向量。

由(\ref{m_alpha_x})得带约束的非线性最小二乘问题
\begin{equation}\label{richards_nls}
  \begin{aligned}
    \min _{\vec{x}} f_{2}(\vec{x})=& \min _{\vec{x}} \sum_{\alpha=1}^{A}\left[\delta m_{\alpha}(\vec{x})-\Delta t \cdot \delta q_{\alpha}(\vec{x})\right]^{2} \\
    & \theta_{\mathrm{r},i}<\theta_{\mathrm{i}} \leqslant \theta_{\mathrm{sat},i}, & \forall \theta_{i} \in \vec{x} \\
    & 0 \leqslant w_{\mathrm{f},i} \leqslant \Delta z_{i},               & \forall w_{\mathrm{f},i} \in \vec{x} \\
    & 0 \leqslant w_{\mathrm{t},i} \leqslant \Delta z_{i},               & \forall w_{\mathrm{t},i} \in \vec{x} \\
    & h_{\mathrm{ {pond }}} \geqslant 0,                               & \text{ if } h_{\mathrm{ {pond }}} \in \vec{x}
  \end{aligned}
\end{equation}

模式中采用Gauss--Newton迭代算法对问题(\ref{richards_nls})进行求解。为减少自由度,在每一个迭代步中,对土壤层使用了变量转换的方法,将问题变为在每层土壤内有且只有一个活动变量。活动变量的选择方法为,
\begin{itemize}
  \item 当土壤层的底部有饱和区时,若$\delta m_{\alpha}>\Delta t \cdot \delta q_{\alpha}$且底部饱和区的入流小于出流,或者$\delta m_{\alpha}<\Delta t \cdot \delta q_{\alpha}$且底部饱和区的入流大于出流,则$w_{\mathrm {t}} $为活动变量,否则进行下一条选择;
  \item 若上一条中未选择$w_{\mathrm {t}} $为活动变量且土壤层顶部有饱和区时,若$\delta m_{\alpha}>\Delta t \cdot \delta q_{\alpha}$且顶部饱和区的入流小于出流,或者$\delta m_{\alpha}<\Delta t \cdot \delta q_{\alpha}$且顶部饱和区的入流大于出流,则$w_{\mathrm {f}} $为活动变量,否则进行下一条选择;
  \item 若上一条中未选择$w_{\mathrm {f}} $为活动变量,则$\theta$为活动变量。
\end{itemize}

变量转换方法使得问题(\ref{richards_nls})在每一个迭代步都形成了具有$A$个未知量的由$A$个方程组成的线性方程组,且系数矩阵为稀疏矩阵,模式中基于Givens变换对线性方程组进行求解。

\subsection{由地下产流引起的地下水和土壤水变化} \label{sec:change_of_zwt_vsf}

地下产流主要指由地形引起的地下水的侧向流动。当不考虑基于物理过程的侧向流时,模式中通过~\ref{section:rsub_par}节中的参数化方案来计算地下产流的量,这部分水分需从地下水和土壤水中扣除。

假设地下产流的量为$r_{\mathrm{sub}}$。 模式中采用“预估-调整”的方式逐层减少土壤水,降低地下水位。图~\ref{fig:地下水变化}显示了在第$i$层内对地下水位和土壤水的计算方法。

{
  \begin{figure}[htbp]
    \centering
    \includegraphics[width=\textwidth]{Figures/植被冠层和土壤水分/地下水变化.pdf}
    \caption{地下水减少时地下水位及土壤水变化的计算方案示意图}
    \label{fig:地下水变化}
  \end{figure}
}

在“预估”步,根据目前的地下水位以及需要减少的水量,预估降低后的地下水位,其满足
\begin{equation} \label{formula:prediction_zwt}
  \begin{aligned}
    \theta_{\mathrm{pre}} & = \Theta_i\left[\psi_{\mathrm{sat},i} - 0.5\left(z_{\mathrm{wt}}^{\mathrm{new}} - z_{\mathrm{wt}}^{\mathrm{old}}\right)\right] \\
    - \Delta W & = \left(z_{\mathrm{wt}}^{\mathrm{new}} - z_{\mathrm{wt}}^{\mathrm{old}}\right) \times  \left(\theta_{\mathrm{sat},i} - \theta_{\mathrm{pre}}\right)
  \end{aligned}
\end{equation}
其中,$z_{\mathrm{wt}}^{\mathrm{old}}$为目前的地下水位,$z_{\mathrm{wt}}^{\mathrm{new}}$为预估的地下水位,$\theta_{\mathrm{sat},i}$为饱和含水量,$\psi_{\mathrm{sat},i}$为饱和时的土壤水势,$\Theta_i$表示第$i$层土壤内由土壤水势到土壤含水量的映射。式~\eqref{formula:prediction_zwt}为$z_{\mathrm{wt}}^{\mathrm{new}}$的隐函数,可用迭代方法求解。

若$z_{\mathrm{wt}}^{\mathrm{new}}$低于第$i$层土壤的下边界$z_{i+1/2}$,则先计算$z_{\mathrm{wt}}^{\mathrm{old}}$至$z_{i+1/2}$区域内的土壤水势和含水量为
\begin{equation} \label{formula:adjust_zwt1}
  \begin{aligned}
    \psi_i^{\prime} & = \psi_{\mathrm{sat},i} - \left(z_{\mathrm{wt}}^{\mathrm{new}} - z_{i+\frac{1}{2}}\right) -0.5\left( z_{i+\frac{1}{2}} - z_{\mathrm{wt}}^{\mathrm{old}}\right) \\
    \theta_i^{\prime} & = \Theta_i\left(\psi_i^{\prime}\right)
  \end{aligned}
\end{equation}
然后将地下水位$z_{\mathrm{wt}}^{\mathrm{new}}$“调整”至$z_{i+1/2}$. 此种情况下,土壤水的减少量小于$- \Delta W$,需继续减少第$i+1$层的土壤水量。

若$z_{\mathrm{wt}}^{\mathrm{new}}$不低于第$i$层土壤的下边界$z_{i+1/2}$,则只需更新$z_{\mathrm{wt}}^{\mathrm{old}}$至$z_{\mathrm{wt}}^{\mathrm{new}}$区域内的的土壤含水量为$\theta_i^{\prime} = \theta_{\mathrm{pre}}$.


若次网格单元上的地下水量是增加的($\Delta W > 0$),则从地下水位所在土壤层开始,向上逐层补充土壤水,抬升地下水位。方法如下:1)剩余补充量的初始值为$\Delta W$;2)对第$i$层进行补充时,若剩余的补充量超过了第$i$层的空气体积$A_i$,则将土壤水补充至饱和,剩余的补充量减少$A_i$,继续对第$i-1$层土壤水进行补充,地下水位抬升至第$i-1$层土壤的下边界;对第$i$层进行补充时,若剩余的补充量小于第$i$层的空气体积$A_i$,则将剩余的补充量全部补充到第$i$层的非饱和区,增加土壤的体积含水量,地下水位不变;3)若整个土壤层全部达到饱和后补充量还有剩余,则将剩余的水分补充至地表积水。



\section{蒸腾引起的土壤水含量变化}

这里对土壤水方程中的源汇项$S$的主要来源,即根系吸水作用(蒸腾)过程做一简单介绍。植物吸水作用由每层土壤水吸收占比$f_{\mathrm{etr},i}$和叶面蒸腾$E_{\mathrm{tr}}$的乘积来表示\citep{dai2003common}。有效根比例为:
\begin{equation}
  {f}_{\mathrm{ {etr }}, {i}}=\frac{{f}_{\mathrm{{root}}, {i}} {f}_{\mathrm{w},i}}{f_{\mathrm{w}}}
\end{equation}
其中$f_{\mathrm{w}} = \sum_{i=1}^{n}{f_{\mathrm{root},i}}f_{\mathrm{w},i}$;$f_{\mathrm{root},i}$表示模式中第$i$层土壤中的根系比例;$f_{\mathrm{w},i}$表示第$i$层土壤中的水分胁迫状况:
\begin{equation}
  {f}_{\mathrm{w},i}=\frac{\psi_{\max }-\psi_{i}}{\psi_{\max }-\psi_{\mathrm{sat}}}
\end{equation}
其中,$\psi_{\mathrm{max}}$为植物达到萎蔫点时的土壤水势。

植被最大蒸腾速率$E_{\mathrm{tr,max}}$设定为:
\begin{equation}
  {E}_{\mathrm{ {tr,max }}}=2 \times 10^{-4} \cdot f_{\mathrm{w}}
\end{equation}
即:
\begin{equation}
  {E}_{\mathrm{{tr}}} \leqslant {E}_{\mathrm{{tr,max}}}
\end{equation}
为满足植被蒸腾,每层土壤中根系吸水量计算为:
\begin{equation}
  {E}_{\mathrm{tr},i} = {f}_{\mathrm{ {etr }}, {i}}{E}_{\mathrm{tr}}
\end{equation}

\section{土壤水力参数的计算}\label{sec_hydropar}
土壤水力参数主要涉及CoLM中模拟土壤水分垂直运动使用的~\citet{campbell1974}和~\citet{van1980closed}两种土壤水力特征曲线关系(方程\eqref{eq:SW_CB}、\eqref{eq:Ks_CB}、\eqref{eq:SW_VG}和\eqref{eq:Ks_VG})中包含的参数,主要有$\theta_{\mathrm {sat}} $(饱和体积含水量)、$\psi_{\mathrm {sat}} $(饱和基质势)、$K_{\mathrm {sat}} $(饱和导水率)、$B=\frac{1}{\lambda}$(孔隙大小分布指数)、$\theta_{\mathrm {r}} $(残余土壤含水量)、$\alpha$(形状参数)、$n$(形状参数)、$L$(孔隙导度参数)。针对上述参数,CoLM采用基于土壤基础数据集GSDE和SoilGrids开发的全球水平方向1 km分辨率、垂直方向分为8层的土壤水热特征参数数据以模拟土壤水热传输过程。数据垂直方向8层的深度分别对应CoLM模式土壤垂直分层中的第2-9层,第1层数据同样也应用于第1层模式,第8层数据同样也应用于第10层模式。两套数据集在全球模拟验证中具有相近的表现~\citep{李文耀2020土壤}。土壤水热特征参数数据的制作方法简述如下。

针对饱和土壤含水量$\theta_{\mathrm {sat}} $,假设其与土壤孔隙度相同,则可采用如下公式计算:
\begin{equation}
  \begin{aligned}
    \theta_{\mathrm {sat}}  =& 1-\frac{\rho_{\mathrm {b}} }{\rho_{\mathrm {d}} }\\
    =& 1-\rho_{\mathrm {b}} \left(\frac{m_{\mathrm{minerals}}}{\rho_{\mathrm{minerals}}}+\frac{m_{\mathrm{om}}}{\rho_{\mathrm{om}}}+\frac{m_{\mathrm{gravels}}}{\rho_{\mathrm{gravels}}}\right)
  \end{aligned}
\end{equation}
其中$\rho_{\mathrm {b}} $代表土壤(干)容重(\unit{g.cm^{-3}}),$\rho_{\mathrm {d}} $代表土粒密度(\unit{g.cm^{-3}})。$m_{\mathrm{minerals}}$、$m_{\mathrm{om}}$和$m_{\mathrm{gravels}}$分别表示矿物质土壤、有机质土壤和砾石在固体土壤中的质量分数,$\rho_{\mathrm{minerals}}$、$\rho_{\mathrm{om}}$和$\rho_{\mathrm{gravels}}$分别表示矿物质土壤、有机质土壤和砾石各自的土粒密度,取值为2.71、1.3和2.80。土壤容重$\rho_{\mathrm {b}} $可通过下式计算;$$\rho_{\mathrm {b}} =\left(1-\frac{v_{\mathrm{gravels}}}{1-n_{\mathrm{gravels}}}\right)\rho_{\mathrm{fineearth}}+v_{\mathrm{gravels}}\rho_{\mathrm{gravels}}$$
其中,$v_{\mathrm{gravels}}$表示固体砾石部分在土壤柱中的体积分数,$n_{\mathrm{gravels}}$表示砾石空隙度(假设为0.24),$\rho_{\mathrm{fineearth}}$表示细质土壤的容重,由土壤基础数据直接获取。

对于其他参数,CoLM模式研发团队针对~\citet{campbell1974}和~\citet{van1980closed}建立的土壤水力特征曲线关系中包含的参数,收集了超过30种较为常用或新近开发的土壤转换函数模型(PTF)(见表~\ref{tab:PTFs}),采用拟合所有PTF对应的土壤水力特征曲线关系的方式获取最优土壤水力特征曲线关系,从而得到最优关系下的土壤水力参数。以~\citet{campbell1974}土壤水力特征曲线关系中参数的计算为例,两个待定参数$\psi_{\mathrm {sat}} $和$B$通过求解下列极值问题得到:$$\chi\left(\psi_{\mathrm {sat}} ,B\right)=\min\sum_{i=1}^N\left[\theta\left(\psi_{\mathrm {sat}} ,B\right)-\theta_i\left(\psi_{\mathrm{si}},B_{i}\right)\right]^2$$
其中,$\psi_{\mathrm{si}}$和$B_{i}$为每一组PTF对应的参数。通过此方法得到的土壤水力特征曲线关系最为接近PTF集合内每一组PFT得到的土壤水力特征曲线关系,因此其对应的参数$\psi_{\mathrm{sat}}$和$B$可视为PFT集合内的最优参数。\citet{van1980closed}土壤水力特征曲线关系中的参数同理可得。\citet{dai2019parameters}通过与NCSS提供的土壤水分特征曲线的观测数据进行比对,发现基于最优拟合参数的~\citet{campbell1974}和~\citet{van1980closed}两种土壤水力特征曲线关系对土壤含水量的模拟效果精度较高,且最优拟合参数相较于传统的PTF中位值参数针对土壤水力特征曲线关系对土壤含水量的模拟结果具有一定程度上的改进。

饱和土壤导水率的估算仍采用传统的PTF集合中位值法。

% Please add the following required packages to your document preamble:
% \usepackage{booktabs}
\begin{landscape}
  \begin{ThreePartTable}
    \begin{TableNotes}
      \footnotesize
%\item 注:
    \item[1] 土壤转换函数的索引次数基于 \url{http://scholar.google.com} 查询,截止于2019年3月20日。

    \item[2] “提供参数种类”一项中,Campbell代表该土壤转换函数提供~\citet{campbell1974}土壤水力特征曲线关系中包含的参数,VG代表该土壤转换函数提供~\citet{van1980closed}土壤水力特征曲线关系中包含的参数,$K_{\mathrm {sat}} $代表饱和土壤导水率
    \end{TableNotes}



    \begin{center}
      \begin{longtable}{p{2cm}<{\centering}p{4cm}<{\centering}p{2cm}<{\centering}p{2.8cm}<{\centering}p{2cm}<{\centering}p{2cm}<{\centering}p{2cm}<{\centering}p{2cm}<{\centering}}
        \caption{用于估算土壤水力参数的土壤转换函数模型列表}
        \label{tab:PTFs}
        \\
        \toprule
        \textbf{土壤转换函数名称} & \textbf{来源} & \textbf{索引次数} & \textbf{提供参数种类} & \textbf{是否输入土壤类型} & \textbf{是否输入土壤质地比例} & \textbf{是否输入容重} & \textbf{是否输入土壤有机碳含量} \\
        \midrule
        \endfirsthead

        \multicolumn{8}{c}%
        {{\bfseries \tablename\ \thetable{} -- \kaishu 续表}} \\
        \toprule
        \textbf{土壤转换函数名称} & \textbf{来源} & \textbf{索引次数} & \textbf{提供参数种类} & \textbf{是否输入土壤类型} & \textbf{是否输入土壤质地比例} & \textbf{是否输入容重} & \textbf{是否输入土壤有机碳含量} \\
%\midrule
        \endhead

%\bottomrule
        \multicolumn{8}{r}{{\kaishu 接下一页表格}} \\
        \hline
        \endfoot

        \bottomrule
        \insertTableNotes
        \endlastfoot

        Cosby0       & \citet{cosby1984statistical}    & 1343 & Campbell, $K_{\mathrm {sat}} $     & + &   &   &   \\ \hline
        Cosby1       & \citet{cosby1984statistical}    & 1343 & Campbell, $K_{\mathrm {sat}} $     &   & + &   &   \\ \hline
        Cosby2       & \citet{cosby1984statistical}    & 1343 & Campbell, $K_{\mathrm {sat}} $     &   & + &   &   \\ \hline
        Campbell1    & \citet{campbell_Shiozawa_1992}  & 237  & Campbell                           &   & + & + &   \\ \hline
        Rawls1       & \citet{Rawls_Brakensiek_1989}   & 572  & Campbell, VG, $K_{\mathrm {sat}} $ &   & + & + & + \\ \hline
        Mayr         & \citet{Mayr_Jarvis_1999}        & 114  & Campbell                           &   & + & + & + \\ \hline
        Williams     & \citet{Williams_Ross_1992}      & 256  & Campbell                           &   & + & + &   \\ \hline
        Clapp        & \citet{clapp1978empirical}      & 2378 & Campbell, $K_{\mathrm {sat}} $     & + &   &   &   \\ \hline
        Carsel       & \citet{Carsel_Parrish_1988}     & 1801 & VG                                 & + &   &   &   \\ \hline
        Wosten1      & \citet{Wösten_Lilly_1999}       & 911  & VG, $K_{\mathrm {sat}} $           &   & + & + & + \\ \hline
        Wosten2      & \citet{Wösten_Lilly_1999}       & 911  & VG, $K_{\mathrm {sat}} $           & + & + &   &   \\ \hline
        Weynants     & \citet{Weynants_Vereecken_2009} & 96   & VG, $K_{\mathrm {sat}} $           &   & + &   & + \\ \hline
        Rosetta3-H1w & \citet{Zhang_Schaap_2017}       & 1638 & VG                                 & + &   &   &   \\ \hline
        Rosetta3-H3w & \citet{Zhang_Schaap_2017}       & 1638 & VG, $K_{\mathrm {sat}} $           &   & + & + &   \\ \hline
        Gupta        & \citet{Gupta_Larson_1979}       & 860  & VG                                 &   & + & + & + \\ \hline
        Rawls2       & \citet{Rawls_Brakensiek_1982}   & 1740 & VG                                 &   & + & + & + \\ \hline
        Rawls3       & \citet{Rawls_Brakensiek_1983}   & 184  & VG                                 &   & + & + & + \\ \hline
        Tomasella    & \citet{Tomasella_Hodnett_1998}  & 243  & VG                                 &   & + & + & + \\ \hline
        Ahuja        & \citet{Ahuja_1989}              & 231  & $K_{\mathrm {sat}} $               &   & + & + & + \\ \hline
        Suleiman     & \citet{Suleiman_2001}           & 61   & $K_{\mathrm {sat}} $               &   & + & + & + \\ \hline
        Spychalski   & \citet{Spychalski_2007}         & 7    & $K_{\mathrm {sat}} $               &   & + & + & + \\ \hline
        Dane         & \citet{Dane_Puckett_1994}       & 256  & $K_{\mathrm {sat}} $               &   & + &   &   \\ \hline
        Jabro        & \citet{Jabro_1992}              & 183  & $K_{\mathrm {sat}} $               &   & + & + &   \\ \hline
        Brakensiek   & \citet{Brakensiek_1984}         & -    & $K_{\mathrm {sat}} $               &   & + & + & + \\ \hline
        Julia        & \citet{Julia_2004}              & 69   & $K_{\mathrm {sat}} $               &   & + &   &   \\ \hline
        Campbell2    & \citet{Campbell_1985}           & -    & $K_{\mathrm {sat}} $               &   & + & + &   \\ \hline
        Vereecken    & \citet{Vereecken_Maes_1990}     & 771  & $K_{\mathrm {sat}} $               &   & + & + & + \\ \hline
        Merdun1      & \citet{Merdun_2010}             & 12   & $K_{\mathrm {sat}} $               &   & + & + & + \\ \hline
        Merdun2      & \citet{Merdun_2010}             & 12   & $K_{\mathrm {sat}} $               &   & + & + & + \\ \hline
        Aimrun       & \citet{Aimrun_2009}             & 23   & $K_{\mathrm {sat}} $               &   & + & + & + \\ \hline
        Rahmati1     & \citet{Rahmati_2018}            & 5    & $K_{\mathrm {sat}} $               & + &   &   &   \\ \hline
        Rahmati2     & \citet{Rahmati_2018}            & 5    & $K_{\mathrm {sat}} $               & + &   &   &   \\
%\hline
      \end{longtable}
    \end{center}
  \end{ThreePartTable}
\end{landscape}

\chapter{积雪的水文过程}
\echapter{Snow Hydrological Processes}
在模式中,覆盖在地表的积雪根据雪盖高度$z_{\mathrm{sno}}$被分为最多五层,这种情况下,雪层从上到下分别用$i = −4, −3, −2, −1, 0$编号。编号$i = 0$表示底层,与土壤相邻,$i = snl + 1$表示顶层,其中变量$snl$是雪盖总层数的相反数($-5\leqslant snl\leqslant 0$)。雪层的厚度表示为$\Delta z_i$(m),雪层的深度$z_i$(m)取为其上边界深度$z_{\mathrm{h},i-1}$和下边界深度$z_{\mathrm{h},i}$的中点(图~\ref{fig:模式中积雪雪层示意图})。注意,由于土壤表面被定义为深度0值,且向下为正,因此雪盖部分的“深度”实际为高度的负值。
{
  \begin{figure}[htbp]
    \centering
    \includegraphics[width=0.7\textwidth]{Figures/雪盖土壤热力过程/模式中积雪雪层示意图.png}
    \caption{模式中雪层示意图(以三层为例)}
    \label{fig:模式中积雪雪层示意图}
  \end{figure}
}
\section{积雪覆盖比例}\label{积雪覆盖比例}
\esection{Snow Fractional Cover}
\begin{mymdframed}{代码}
  本节对应的代码文件为\texttt{MOD\_SnowFraction.F90}。
\end{mymdframed}

陆地表面可分为被积雪覆盖与未被积雪覆盖两部分。根据 \citet{swenson2012new}提供的方法,
被积雪覆盖的地表面积比例$f_{\mathrm{sno}}$可分为两步计算:在积分开始时若有固态降水发生,则新一步的$f_{\mathrm{sno}}$更新为
\begin{equation}
  f_{\mathrm{{sno }}}^{(n+1)}=1-\left[1-\tanh\left(0.1 p_{\mathrm{snow}} \Delta t\right)\right]\left(1-f_{\mathrm{{sno }}}^{(n)}\right) \leqslant 1.0
\end{equation}
其中$p_{\mathrm {snow}} $为固态降水速率(\unit{kg.m^{-2}.s^{-1}}),$\Delta t$为积分时间步长(s);在水热过程模拟结束后,若有积雪融化发生,则$f_{\mathrm{sno}}$的更新采用\citet{niu2007ObservationbasedFormulationSnow}方案,计算为:
\begin{equation}
  f_{\mathrm{sno}}^{(n+1)}=\tanh{\left(\frac{z_{\mathrm{sno}}}{2.5 z_{\rm lnd} \left ( \rho_{\mathrm{sno}}/\rho_{\mathrm {sno,new}} \right )^m}\right)}
\end{equation}
其中$z_{\mathrm{sno}}$表示积雪厚度(m)。$\rho_{\mathrm{sno}}$为雪的密度($\unit{kg.m^{-3}}$),$\rho_{\mathrm {sno,new}}$为新雪密度,取值为100 $\unit{kg.m^{-3}}$。$z_{\rm lnd}$为裸土覆盖时的地表粗糙度,取值为0.01 m。$m$为融雪因子(参数),模型中取值为1。

当植被被积雪掩埋时,已知植被粗糙度$z_{\rm 0mv}$,则被积雪掩埋的植被占总植被的比例计算为
\begin{equation}
  w_{\rm t}=\frac{0.1 z_{\mathrm{sno}}}{z_{\rm 0mv}+0.1 z_{\mathrm{sno}}}
\end{equation}
CoLM2014及以前版本计算斑块中的有效植被比例$f_{\mathrm{sig}}=\left(1-w_{\rm t}\right)f_{\mathrm{veg}}$,无植被覆盖比例为$\left(1-f_{\mathrm{sig}}\right)$。CoLM2024版本为了考虑与PFT次网格类型的兼容性,同时认为积雪是通过覆盖或掩埋植被叶面积、茎面积进行影响,将$w_{\rm t}$用于修正被积雪掩盖后的SAI,即${\rm SAI=TSAI}\left(1-w_{\rm t}\right)$,TSAI为植被“真实”茎面积指数。当采用卫星遥感LAI时,由于其数值已经是积雪覆盖下的绿色叶面部分,故在此不对其进行积雪覆盖调整。


\section{雪层的建立}\label{sec:雪层的建立}
\esection{Initialization of Snow Layer}
\begin{mymdframed}{代码}
  本节对应的代码文件为\texttt{MOD\_NewSnow.F90}。
\end{mymdframed}


当固态降水$p_{\mathrm{snow}}$的发生导致雪盖高度$z_{\mathrm{sno}}$大于0.01 m且此时尚无雪盖分层,则将在模拟开始时创建一个新的雪层,相关物理量设置如下:
\begin{equation}
  \begin{aligned}
    & \Delta z_{0} &&= &{z}_{\mathrm{sno}}& \\
    & z_0 &&= &-0.5\Delta z_0& \\
    & z_{\mathrm{h,-1}} &&= &-\Delta z_0& \\
    & T_0 &&= &\min \left(T_{\mathrm {frz}} ,T_{\mathrm{a}}\right)& \\
    & w_{\mathrm{ice,0}} &&= &W_{\mathrm{sno}}& \\
    & w_{\mathrm{liq,0}} &&= &0&
  \end{aligned}
\end{equation}
其中$T_{\mathrm {frz}} $为液态水凝结温度(~\ref{tab:物理常数}),$T_{\mathrm{a}}$为大气温度,$w_{\mathrm{ice}}$和$w_{\mathrm{liq}}$分别表示雪层中固态水的含量和液态水的含量(\unit{kg.m^{-2}})。

\section{雪盖中水的垂直运动}\label{雪盖的水量平衡}
\esection{Vertical Water Movement in Snow}
\begin{mymdframed}{代码}
  本节对应的代码文件为\texttt{MOD\_SoilSnowHydrology.F90}。
\end{mymdframed}

在垂直方向上,雪盖的液态水质量守恒方程为
\begin{equation}
  \frac{\partial w_{\mathrm{liq},i}}{\partial t}=\left(q_{\mathrm{liq},i-1}-q_{\mathrm{liq},i}\right)+\frac{{\left(\Delta w_{\mathrm{liq},i}\right)}_{\mathrm {p}} }{\Delta t}
\end{equation}
其中$q_{\mathrm{liq},i-1}$表示第$i-1$层流入至第$i$层的液态水,$q_{\mathrm{liq},i}$表示第$i$层流出至第$i+1$层的液态水,${{\left(\Delta w_{\mathrm{liq},i}\right)}_{\mathrm {p}} }/{\Delta t}$表示由于相态变化导致的水量改变率(见章节~\ref{sec:温度的相态变化调整})。对于雪盖顶层,需先考虑固态水的变化导致的液态水改变。由于升华和凝华作用,雪盖顶层固态水的变化为
\begin{equation}
  w_{\mathrm{ice},snl+1}^{n+1}=w_{\mathrm{ice},snl+1}^n+\left(q_{\mathrm{frost}}-q_{\mathrm{subl}}\right)\Delta t
\end{equation}
其中$q_{\mathrm{subl}}$和$q_{\mathrm{frost}}$分别表示水的升华和凝华速率(\unit{kg.m^{-2}.s^{-1}}或 \unit{mm.H_2O.s^{-1}})。如果$w_{\mathrm{ice},snl+1}^{n+1}<0$,将固态水含量重置为0,并从液态水中减去固态水增加到0所需的水含量。雪盖顶层的$q_{\mathrm{liq},i-1}$则由以下公式计算
\begin{equation}
  q_{\mathrm{liq},i-1}=q_{\mathrm{g,rain}}+\left(q_{\mathrm{sdew}}-q_{\mathrm{seva}}\right)
\end{equation}
其中$q_{\mathrm{g,rain}}$为到达雪盖的液态降水速率,$q_{\mathrm{seva}}$和$q_{\mathrm{sdew}}$分别表示水的蒸发和凝结速率。基于此,顶层液态水含量更新为
\begin{equation}w_{\mathrm{liq},snl+1}^{n+1}=w_{\mathrm{liq},snl+1}^n+\left(q_{\mathrm{g,rain}}+q_{\mathrm{sdew}}-q_{\mathrm{seva}}\right)\Delta t
\end{equation}

当液态水含量超过了雪层的最大持水量时,多余的液态水将下渗到相邻雪层,渗透能力取决于雪层的有效孔隙度($1-\theta_{\mathrm{ice}}$)。如果两层中任一层有效孔隙度($1-\theta_{\mathrm{ice},i}$或$1-\theta_{\mathrm{ice},i+1}$)小于不透水体积含水量$\theta_{\mathrm{imp}}=0.05$,则假定两层间液态水通量为0。否则,对于雪层$i=snl+1,\ ...,\ 0$,液态水通量$q_{\mathrm{liq},i}$被计算为
\begin{equation}
  q_{\mathrm{liq},i}=\frac{\rho_{\mathrm{liq}}\left[\theta_{\mathrm{liq},i}-S_{\mathrm {r}} \left(1-\theta_{\mathrm{ice},i}\right)\right]\Delta z_{i} }{\Delta t}\geqslant 0
\end{equation}
其中固态水和液态水的体积含水量分别为
\begin{align}
  \theta_{\mathrm{ice},i}&=\frac{w_{\mathrm{ice},i}}{\Delta z_i \rho_{\mathrm{ice}}} \leqslant 1 \\
  \theta_{\mathrm{liq},i}&=\frac{w_{\mathrm{liq},i}}{\Delta z_i \rho_{\mathrm{ice}}} \leqslant 1-\theta_{\mathrm{ice},i}
\end{align}
$S_{\mathrm {r}} =0.033$称为束缚水饱和度(由于毛细管的滞留作用,渗透结束后雪层仍会保留一定含量的液态水)。$q_{\mathrm{liq},i}$同时也受到相邻下层最大持水量的限制,除非其相邻下层为土壤层,即
\begin{equation}
  q_{\mathrm{liq},i} \leqslant \frac{\rho_{\mathrm{liq}}\left(1-\theta_{\mathrm{ice},i+1}-\theta_{\mathrm{liq},i+1}\right)\Delta z_{i+1}}{\Delta t} \qquad i=snl+1,\ ...,\ -1
\end{equation}
于是,液态水含量更新为
\begin{equation}\label{eq:SnowWater}
  w_{\mathrm{liq},i}^{n+1}=w_{\mathrm{liq},i}^n+\left(q_{i-1}-q_i\right)\Delta t
\end{equation}
对于每个时间步长,依次从雪盖顶层到底层计算式~\eqref{eq:SnowWater}。到达土壤层的液态水即为$q_{\mathrm{liq,0}}$,将被用于地表径流和土壤层的下渗计算当中。

\section{雪盖的黑碳、有机碳和矿物粉尘}
\esection{Black and organic carbon and mineral dust within snow}
\begin{mymdframed}{代码}
  本节对应的代码文件为\texttt{MOD\_SoilSnowHydrology.F90}。
\end{mymdframed}

由于大气的气溶胶沉降,积雪中会存在一定的颗粒物,影响雪盖的辐射传输。雪中颗粒物含量仅受到质量守恒的约束,这部分的计算均被定义在SNICAR模型中。对于每个雪层,该模型包括了以下八种颗粒物的含量,亲水性黑碳、疏水性黑碳、亲水性有机碳、疏水性有机碳和四种矿物粉尘(颗粒半径分别为0.1-1.0,1.0-2.5,2.5-5.0,5.0-10.0 \unit{\mu m})。每种颗粒物都具有不同的光学属性和融水清除率。

黑碳和有机碳的沉降率被分为以下四类
\begin{align}
  D_{\mathrm{bc,hphob}}&=D_{\mathrm{bc,dryphob}} \\
  D_{\mathrm{oc,hphil}}&=D_{\mathrm{oc,dryhphil}}+D_{\mathrm{oc,wethphil}} \\
  D_{\mathrm{bc,hphil}}&=D_{\mathrm{bc,dryhphil}}+D_{\mathrm{bc,wethphil}} \\
  D_{\mathrm{oc,hphob}}&=D_{\mathrm{oc,dryphob}}
\end{align}

假定沉降的颗粒物立即在表面雪层中均匀混合,并在计算雪层层间液态水通量之后添加,以使沉降后的一些气溶胶颗粒位于顶层,并且在进行反照率相关的计算之前不会被冲走。对于每个时间步长,在进行雪层间液态水的垂直运动以及雪层的合并和再分层计算中,都将根据质量守恒更新每层颗粒物含量。每种颗粒物的质量变化$\Delta m_{\mathrm{sp},i}$为
\begin{equation}
  \Delta m_{\mathrm{sp},i}=\left[k_{\mathrm{sp}}\left(q_{\mathrm{liq},i-1} c_{\mathrm{sp},i-1}-q_{\mathrm{liq},i} c_{\mathrm{sp},i}\right)+D_{\mathrm{sp}}\right] \Delta t
\end{equation}
其中$k_{\mathrm{sp}}$表示每种颗粒物的融水清除率,具体参见表~\ref{lab:融水清除率}。$c_{\mathrm{sp},i-1}$和$c_{\mathrm{sp},i}$分别表示$i-1$层和$i$层的颗粒物质量混合比(\unit{kg.kg^{-1}}),$D_{\mathrm{sp}}$表示大气气溶胶沉降率(仅在$snl+1$层考虑)。颗粒物质量混合比为
\begin{equation}
  c_{i}=\frac{m_{\mathrm{sp},i}}{w_{\mathrm{liq},i}+w_{\mathrm{ice},i}}
\end{equation}

$k_{\mathrm{sp}}$的具体值根据~\citet{Conway2012}等人的实验得出。雪盖底层被融水带走的颗粒物被认为从雪盖中永久消失,模式中不再考虑。

\begin{table}[htbp]
  \centering
  \caption{雪盖中不同颗粒物的融水清除率}
  \begin{tabular}{lccc}
    \toprule
    颗粒种类                           & $k_{\mathrm{sp}}$ \\ \midrule
    亲水性黑碳                         & 0.20              \\
    疏水性黑碳                         & 0.03              \\
    亲水性有机碳                       & 0.20              \\
    疏水性有机碳                       & 0.03              \\
    1类矿物粉尘(0.1-1.0 \unit{\mu m})  & 0.02              \\
    2类矿物粉尘(1.0-2.5 \unit{\mu m})  & 0.02              \\
    3类矿物粉尘(2.5-5.0 \unit{\mu m})  & 0.01              \\
    4类矿物粉尘(5.0-10.0 \unit{\mu m}) & 0.01              \\ \bottomrule
  \end{tabular}
  \label{lab:融水清除率}
\end{table}



\section{雪的压实}\label{雪的压实}
\esection{Snow Compaction}
\begin{mymdframed}{代码}
  本节对应的代码文件为\texttt{MOD\_SnowLayersCombineDivide.F90}。
\end{mymdframed}

积雪的压实主要包括以下四个过程:
\begin{enumerate}
  \item 破坏变质作用(新雪的冰晶粒子在风和热力作用下树状结构的破裂),
  \item 上覆积雪自重引起的压实,
  \item 融化变质作用(积雪经历多次冻融循环后融雪水出流导致雪层结构的改变),
  \item 风吹雪引起的压实。
\end{enumerate}

前两个过程的处理方法分别来自 SNTHERM.99 \citep{jordan1999heat}和 SNTHERM.89 \citep{jordan1991one},融化变质的贡献取决于雪层融化过程中前后时刻固态水的变化率,风吹雪压实则考虑了下降风对雪的影响。积雪的总压实率可写为上述四个过程的和:
%
\begin{equation}
  C_{\mathrm{R},i}=\frac{1}{\Delta {z_i}} \frac{\partial \Delta {z_i}}{\partial {t}}=C_{\mathrm{R1},i}+C_{\mathrm{R2},i}+C_{\mathrm{R3},i}+C_{\mathrm{R4},i}
\end{equation}
当雪层达到饱和
\begin{equation}
  1-\left(\frac{w_{\mathrm{ice},i}}{ \Delta {z_i} \rho_{\mathrm{ice}}}+\frac{w_{\mathrm{liq},i}}{ \Delta {z_i} \rho_{\mathrm{liq}}}\right) \leqslant 0.001
\end{equation}
或固态水含量$w_{\mathrm{ice},i}\leqslant0.1$时,不再考虑雪的压实。

经过压实后雪层厚度更新为:
\begin{equation}
  \Delta z_i^{n+1}=\Delta z_i^n\left(1+C_{\mathrm{R},i} \Delta t\right)
\end{equation}


\subsection{破坏变质引起的压实}
\esubsection{Destructive Metamorphism Compaction}
雪在到达地面后,随即快速变化。在热力作用的影响下,单个雪花原有的树状结构发生破裂,向球状结构演变。这些雪花又会和其他雪花融合生长,使雪粒之间结合得更加紧密,最终发生沉降堆积。对于密度小于 \qty{100}{kg.m^{-3}} 的新雪来说,破坏变质引起的沉降非常重要。雪粒的树状结构会使它们之间产生一种类似于“齿轮咬合”的作用,从而具有一定的强度,这种强度在破坏变质的过程中会逐渐减弱。\citet{anderson1976point}对这一阶段的压实过程提出了以下经验函数:
\begin{equation}\label{eq:DestruciveCompact}
  C_{\mathrm{R1},i}=\left[\frac{1}{\Delta {z_i}} \frac{\partial \Delta {z_i}}{\partial {t}}\right]_{\text {destructive}}=-2.777 \times 10^{-6} {c}_{3} {c}_{4} {e}^{-0.04\left(T_{\mathrm {frz}} -T\right)}
\end{equation}
其中
\begin{equation}
  c_3=\begin{cases}
    1 &\text{当}\ \frac{w_{\mathrm{ice},i}}{\Delta z_i} \leqslant 100 \;\unit{kg.m^{-3}}\text{ 时} \\
    {\mathrm e}^{-0.046\left(\frac{w_{\mathrm{ice},i}}{\Delta z_i}-100\right)} &\text{当}\ \frac{w_{\mathrm{ice},i}}{\Delta z_i}>100 \;\unit{kg.m^{-3}}\text{ 时}
  \end{cases}
\end{equation}
\begin{equation}
  c_4=\begin{cases}
    1 &\qquad \quad \qquad \quad \;\text{当}\ \frac{w_{\mathrm{liq},i}}{\Delta z_i} \leqslant 0.01 \;\unit{kg.m^{-3}}\text{ 时} \\
    2 &\qquad \quad \qquad \quad \;\text{当}\ \frac{w_{\mathrm{liq},i}}{\Delta z_i}>0.01 \;\unit{kg.m^{-3}}\text{ 时}
  \end{cases}
\end{equation}
这两个系数表明,当雪层中固态水的体积密度超过 \qty{100}{kg.m^{-3}} 时,破坏变质的速率会有所降低;当雪层中存在一定液态水时,破坏变质的速率将成倍增加。

\subsection{雪层负重引起的压实}
\esubsection{Overburden Pressure Compaction}
随着积雪的累积,上覆积雪的自重会进一步地压实雪层。上覆雪产生的压力(负重)使雪粒粘结生长速度加快,形成更加有效的堆积形状。在经过前一过程的压实后,这一阶段的压实速率会减慢,主要和雪层的负重压力有关。在低压力范围的季节性积雪内,这一阶段的压实率是负重的线性函数\citep{anderson1976point},即
\begin{equation}\label{eq:OverburdenCompact}
  C_{\mathrm{R2},i}=\left[\frac{1}{\Delta {z_i}} \frac{\partial \Delta {z_i}}{\partial {t}}\right]_{\text {overburden}}=-\frac{P_{\mathrm{s},i}}{\eta}
%=-\frac{{P}_{\mathrm{{s}}}}{9 \times 10^{5}} {e}^{-0.08\left(T_{\mathrm {frz}} -{T}\right)-0.023 \rho_{\mathrm{{i}}} \theta_{\mathrm{{i}}}}
\end{equation}
其中$P_{\mathrm{s},i}$是雪层负重的质量(\unit{kg.m^{-2}}),等于其上覆雪层中固态水和液态水的质量总和加上该层自身固态水和液态水质量的一半,即
\begin{equation}
  P_{\mathrm{s},i}=\frac{w_{\mathrm{ice},i}+w_{\mathrm{liq},i}}{2}+\sum_{{j}={snl}+1}^{{j}={i}-1}\left({w}_{\mathrm{ice},j}+{w}_{\mathrm{liq},j}\right)
\end{equation}
~\eqref{eq:OverburdenCompact} 式中变量$\eta$为粘滞系数(\unit{kg.s.m^{-2}}),与雪层的密度和温度有关:
\begin{equation}
  \eta=f_1 f_2 \eta_0 \frac{\rho_i}{c_\eta} {\mathrm e}^{a_\eta \left(T_{\mathrm {frz}} -T_i\right)+b_\eta \rho_i}
\end{equation}
其中$\rho_i=\frac{w_{\mathrm{ice},i}}{\Delta z_i}$为雪层中固态水的体积密度,常系数$\eta_0=7.62237 \times 10^6$ \unit{kg.s^{-1}.m^{-2}},$a_\eta=0.1$ \unit{K^{-1}},$b_\eta=0.023$ \unit{m^{-3}.kg^{-1}},$c_\eta=450$ \unit{kg.m^{-3}} \citep{Kampenhout2017}。系数$f_1$和雪层中液态水的含量有关~\citep{Vionnet2012}:
\begin{equation}
  f_1=\frac{1}{1+60\frac{w_{\mathrm{liq},i}}{\rho_{\mathrm{liq}}\Delta z_i}}
\end{equation}
系数$f_2$则与雪层中棱状雪粒(angular grains)的含量有关,目前的计算中固定$f_2=4$。

\subsection{雪层融化引起的压实}
\esubsection{Compaction by Melt}
在融雪过程的后期(积雪经过多次冻融循环后),融雪水的出流导致雪堆更加致密,雪层产生压实。这一阶段的压实率取决于雪层融化过程中前后两个时间步数固态水的变化率
\begin{equation}
  C_{\mathrm{R3},i}=\left[\frac{1}{\Delta {z_i}} \frac{\partial \Delta {z_i}}{\partial {t}}\right]_{\text{melt}}=-\frac{1}{\Delta {t}}\max\left(0,\frac{{f}_{\mathrm{{ice},i}}^{n}-{f}_{\mathrm{{ice},i}}^{n+1}}{{f}_{\mathrm{ice},i}^{n}}\right)
\end{equation}
其中$f_{\mathrm{ice},i}=w_{\mathrm{ice},i}/\left({w_{\mathrm{ice},i}+w_{\mathrm{liq},i}}\right)$为雪层中固态水占全部水含量的比例。

\subsection{风吹雪引起的压实}
\esubsection{Compaction by Drifting Snow}
在高纬的冰原地区,低温使破坏变质的过程发生缓慢,此时高速的下降风将占据主导,气流使雪粒的树状结构破碎,引发雪粒堆积,产生压实。在这种情况下,引入风吹雪压实的参数化方案~\citep{Vionnet2012}:
\begin{equation}
  C_{\mathrm{R4},i}=\left[\frac{1}{\Delta z_i}\frac{\partial \Delta z_i}{\partial t}\right]_{\text{drift}}=-\frac{\max \left(0,\rho_{\text{max}}-\rho_i\right)}{\tau_i}
\end{equation}
其中,$\rho_{\text{max}}=350$ \unit{kg.m^{-3}}是这一过程的有效密度上限,$\tau_i$是一个与雪层深度有关的时间尺度变量:
\begin{equation}
  \tau_i=\frac{\tau}{\Gamma_{\text{drift}}^i}, \quad \Gamma_{\text{drift}}^i=\max \left(0,S_{\mathrm {I}} ^i {\mathrm e}^{-z_i/0.1}\right)
\end{equation}
常数$\tau$是风吹雪压实过程的一个特征时间尺度,根据经验被设置为48 \unit{h},$z_{i} =\sum_j \Delta z_j \cdot \left(3.25-S_{\mathrm {I}} ^j\right)$称为伪深度,与当前雪层$i$上方的$j$个雪层的硬化程度有关。$S_{\mathrm {I}} $称为吹雪飘移指数(driftability index):
\begin{equation}
  S_{\mathrm {I}} =-2.868 {\mathrm e}^{-0.085 U} + 1 + M_{\mathrm {O}}
\end{equation}
其由10 \unit{m}风速$U$和雪的流动指数(mobility index)$M_{\mathrm {O}}$共同作用,反映了雪粒在风的作用下受到的飘移影响。雪的流动指数
\begin{equation}
  M_{\mathrm {O}}=0.34\left(-0.583g_{\mathrm {s}} -0.833s+0.833\right)+0.66F\left(\rho\right)
\end{equation}
与雪的微观结构有关,描述了雪层受侵蚀的可能性,其中
$$F\left(\rho\right)=1.25-0.0042 \cdot \\\left[\max \left(\rho_{\mathrm{min}},\rho\right)-\rho_{\mathrm{min}}\right],$$$\rho_{\mathrm{min}}=50$ \unit{kg.m^{-3}}。$g_{\mathrm {s}} $和$s$是两个和雪粒形态有关的变量,$s$称为雪粒的球度(sphericity),从0 -- 1不等,$g_{\mathrm {s}} $为雪粒大小,一般为0.3至0.4 \unit{mm}。


\section{雪层的合并}\label{雪层的合并}
\esection{Snow Layer Combination}
\begin{mymdframed}{代码}
  本节对应的代码文件为\texttt{MOD\_SnowLayersCombineDivide.F90}。
\end{mymdframed}

雪层的合并包括以下两种情况:

(1) 考虑固态水的含量。当任何一层几近融化(即$w_{\mathrm{ice},i} \leqslant 0.1$ \unit{kg.m^{-2}})时,移除该雪层,且将其液态水和固态水含量分配到与之相邻的下层中:
\begin{equation}
  w_{\mathrm{liq},i+1} = w_{\mathrm{liq},i+1} + w_{\mathrm{liq},i}
\end{equation}
\begin{equation}
  w_{\mathrm{ice},i+1} = w_{\mathrm{ice},i+1} + w_{\mathrm{ice},i}
\end{equation}
这也包括了与土壤相邻的底层雪层,该层的液态水和固态水含量将直接被分配到土壤层顶层中。每移除一层雪层,该层以上的雪层编号将增加1,该层以下的雪层编号保持不变,以对应新的雪盖分层。

此时,如果已无雪层存在($snl=0$),雪水当量$W_{\mathrm{sno}}$和雪盖高度$z_{\mathrm{sno}}$将被设为0。如果仍有雪层存在,则$W_{\mathrm{sno}}$和$z_{\mathrm{sno}}$根据下式更新为
\begin{equation}
  W_{\mathrm{sno}} = \sum_{\mathrm{i=snl+1}}^{i=0}\left(w_{\mathrm{ice},i}+w_{\mathrm{liq},i}\right)
\end{equation}
\begin{equation}
  z_{\mathrm{sno}} = \sum_{i=snl+1}^{i=0} \Delta z_i
\end{equation}
若雪盖高度$z_{\mathrm{sno}} < 0.01$ \unit{m},则雪盖层数$snl$仍被设为0,此时雪水当量
$$W_{\mathrm{sno}}=\sum_{i=snl+1}^{i=0} w_{\mathrm{ice},i}$$
将仅计算雪盖固态水含量,雪盖液态水含量$\sum_{i=snl+1}^{i=0} w_{\mathrm{liq},i}$将分配给土壤层顶层。

(2) 考虑雪层的厚度。当某一雪层的厚度$\Delta z_i$小于规定的最小值$\Delta z_{\mathrm{min}}$时,则将该雪层与相邻雪层合并。如果该雪层:
\begin{enumerate}
  \item 为顶层,则与相邻的下层雪层合并;
  \item 为底层(与土壤相邻),则与相邻的上层雪层合并;
  \item 为中间层,则与相邻的厚度较薄的雪层合并。
\end{enumerate}

五个雪层(从顶层到底层)规定的厚度最小值$\Delta z_{\mathrm{min}}$分别为 0.010、0.015、0.025、0.055 和 0.115 \unit{m}。

当两个雪层(这里用编号1和2表示)合并时,合并层的厚度为
\begin{equation}\label{eq:SnowCombThick}
  \Delta {z}_{\mathrm {c}} =\Delta {z}_{1}+\Delta {z}_{2}
\end{equation}
根据质量守恒,合并层的质量计算为
\begin{equation}
  w_{\mathrm{liq,c}}=w_{\mathrm{liq,1}}+w_{\mathrm{liq,2}}
\end{equation}
\begin{equation}
  w_{\mathrm{ice,c}}=w_{\mathrm{ice,1}}+w_{\mathrm{ice,2}}
\end{equation}
根据焓变守恒,合并层的温度计算为
\begin{equation}\label{eq:SnowCombTemp}
  T_{\mathrm {c}} =\begin{cases}
    T_{\mathrm {frz}} +{h_{\mathrm {c}} }/\left(C_{\mathrm{ice}}w_{\mathrm{ice,c}}+C_{\mathrm{liq}}w_{\mathrm{liq,c}}\right) &\text{当}\ h_{\mathrm {c}} <0 \text{ 时} \\
    T_{\mathrm {frz}}  &\text{当}\ 0 \leqslant h_{\mathrm {c}}  \leqslant \lambda_{\mathrm {fus}}  w_{\mathrm{liq,c}} \text{ 时}\\
    T_{\mathrm {frz}} +\left(h_{\mathrm {c}} -\lambda_{\mathrm {fus}}  w_{\mathrm{liq,c}}\right)/\left(C_{\mathrm{ice}} w_{\mathrm{ice,c}}+C_{\mathrm{liq}} w_{\mathrm{liq,c}}\right) &\text{当}\ h_{\mathrm {c}}  > \lambda_{\mathrm {fus}}  w_{\mathrm{liq,c}} \text{ 时}
  \end{cases}
\end{equation}
其中$h_{\mathrm {c}} =h_1+h_2$为两个雪层合并后的焓,第$i$层的焓$h_i$可通过以下公式得出
\begin{equation}
  h_i=\left(C_{\mathrm{ice}}w_{\mathrm{ice},i}+C_{\mathrm{liq}}w_{\mathrm{liq},i}\right)\left(T_i-T_{\mathrm {frz}} \right)+\lambda_{\mathrm {fus}}w_{\mathrm{liq},i}
\end{equation}
$\lambda_{\mathrm {fus}} $为固态水融化潜热,$C_{\mathrm{liq}}$和$C_{\mathrm{ice}}$分别为液态水和固态水的比热容(表~\ref{tab:物理常数})。

最后,根据下式更新雪层深度和雪层交界面深度
\begin{equation}
  z_i=z_{\mathrm{h},i}-0.5\Delta z_i \;\;\;\;\;i=0,\;...,\;snl+1
\end{equation}
\begin{equation}
  z_{\mathrm{h},i-1}=z_{\mathrm{h},i}-\Delta z_i \;\;\;\;\;i=0,\;...,\;snl+1
\end{equation}



\section{雪层的再分层}\label{雪层的再分层}
\esection{Snow Layer Subdivision}
\begin{mymdframed}{代码}
  本节对应的代码文件为\texttt{MOD\_SnowLayersCombineDivide.F90}。
\end{mymdframed}

当某一雪层的厚度$\Delta z_i$大于规定的最大值$\Delta z_{\mathrm{max}}$时,该雪层将被再分。最大厚度$\Delta z_{\mathrm{max}}$和雪盖的层数有关。例如,如果雪盖只被分为一层,那么顶层(也就是该层)的最大厚度将为0.03 \unit{m};如果雪盖被分为多层,那么顶层的最大厚度则为0.02 \unit{m}。不同情况雪层的最大厚度如图~\ref{fig:不同情况下雪层的最大厚度} 所示。

{
  \begin{figure}[htbp]
    \centering
    \includegraphics[width=0.6\columnwidth]{Figures/雪盖土壤热力过程/不同情况下雪层的最大厚度.png}
    \caption{不同情况下雪层的最大厚度示意图}
    \label{fig:不同情况下雪层的最大厚度}
  \end{figure}
}

如果被再分的雪层为底层,则将该雪层等分为厚度相等的两层,并均分液态水含量和固态水含量,温度和原雪层保持一致,此时雪盖总层数将增加1(即$snl$将减少1)。如果被再分的雪层不为底层,先将该层分为上下两层,其中上层雪层的厚度限为该层的规定最大厚度$\Delta z_{\mathrm{max}}$,下层雪层的厚度则为$\Delta z_i - \Delta z_{\mathrm{max}}$,原雪层的液态水含量和固态水含量按厚度比例分配给上下两个雪层,温度在再分前后保持不变,再将下层雪层作为多余部分根据式~\eqref{eq:SnowCombThick} -~\eqref{eq:SnowCombTemp} 和原雪层的相邻下层雪层合并,从而保证雪盖总层数不变。

如图~\ref{fig:以三层为例雪层再分层的方法},假设某一时间步,模式中包含三层雪层,其厚度从上至下分别为0.02、0.08和0.17 m,参考图~\ref{fig:不同情况下雪层的最大厚度} 中三层情况下的最大厚度标准考虑,则第$i=-1$层厚度$\Delta z_{-1}=0.08 > \Delta z_{\mathrm{max,-1}}=0.05 $,此时将其厚度截取0.03 m分配至相邻下层雪盖,使其厚度减少为$\Delta z_{-1}=0.05$,刚好满足最大阈值。这时,底层雪盖厚度将增加为$\Delta z_{0}=0.20$,同样超过了$\Delta z_{\mathrm{max,0}}=0.18$,则将该层再次划分为厚度相等的两层,均为0.10 m。于是,总雪盖层数便增加至四层,每层厚度如图中所标注,之后的计算则以四层为标准。

{
  \begin{figure}[htbp]
    \centering
    \includegraphics[width=0.8\columnwidth]{Figures/雪盖土壤热力过程/以三层为例雪层再分层的方法.png}
    \caption{以三层为例,雪层再分层的方法示意图}
    \label{fig:以三层为例雪层再分层的方法}
  \end{figure}
}


\part{水文过程}{Hydrologic Processes}\label{part:hydro}
%\epart{Hydrologic Processes}
\chapter{耦合径流模型的产流和汇流过程}

\section{产流参数化方案}
%\addcontentsline{toc}{chapter}{陆地表面的水分循环}

\subsection{地表产流}

地表径流的参数化方案\citep{niu2005simple}考虑了地形、地下水位、降水和入渗速度等因素。

模式网格内饱和区域的面积$f_{\mathrm{sat}}$通过以下方案估算,
\begin{equation}
f_{\mathrm{sat}}=f_{\mathrm{wt}} \times \mathrm{e}^{-0.5 \times f_{\mathrm{decay}} \times z_{\mathrm{wt}}}
\end{equation}
其中,$f_{\mathrm{wt}}$为网格内地下水位较高的区域的面积百分比(模式中取固定值0.38),$f_{\mathrm{decay}}$为径流的衰减因子(模式中取固定值0.5),$z_{\mathrm{wt}}$为地下水位的位置。

最大入渗能力的计算考虑了最上面三层土壤的物理状态和属性,
\begin{equation}
q_{\mathrm{in}, \max }= \min _{i=1,2,3} 10^{-6.0 \times f_{\mathrm{ice}, i} \times K_{\mathrm{sat}, i}}
\end{equation}
其中,$f_{\mathrm{ice},i}$表示第$i$层中冰占土壤孔隙的体积百分比,$K_{\mathrm{sat},i}$表示第$i$层的饱和导水率。

假设到达地表的净水流通量为$G_{\mathrm{water}}$. 一个模式计算单元内,饱和区域的地表水全部转化为径流流走,非饱和区域的地表水,部分入渗到土壤中,剩余部分转化为径流,总的地表径流为,
\begin{equation}
r_{\mathrm{surface}}=f_{\mathrm{sat}} \times G_{\mathrm{water}}+\left(1-f_{\mathrm{sat}}\right) \times \left(G_{\mathrm{water}}-q_{\mathrm{in},\max}\right)
\end{equation}
入渗到土壤中的部分等于输入的净水流通量减去地表径流,即
\begin{equation}
q_{\mathrm{infl}}={G}_{\mathrm{water}}-r_{\mathrm{surface}}
\end{equation}

\subsection{地下产流} \label{section:rsub_par}
地下产流的大小与地形和地下水位有关\citep{niu2005simple},
\begin{equation}
r_{\mathrm{subsurface}} = r_{\mathrm{sub,max}} \exp \left(-f_{\mathrm{drai}} \times z_{\mathrm{wt}}\right)
\end{equation}
其中,$r_{\mathrm{sub,max}}$为产流的最大值,取决于地形坡面的大小,
模式中取全球统一的数值$5.5\times 10^{-3}~\unit{mm~s^{-1}}$;$f_{\mathrm{drai}}=2.5$ \unit{m^{-1}} 为衰减因子。

当土壤中含有冰时,需考虑冰对地下径流的阻力作用,
\begin{equation}
\begin{aligned}
f_{\mathrm{impd,ice}} & = 1 - \frac{\exp \left[-3 \times\left(1-f_{\mathrm{ice,sum}}\right)\right]
    -\exp (-3)}{1-\exp (-3)} \\
 r_{\mathrm{subsurface}} & = f_{\mathrm{impd,ice}} \times r_{\mathrm{sub,max}} 
    \exp \left(-f_{\mathrm{drai}} \times z_{\mathrm{wt}}\right)
\end{aligned}
\end{equation}
其中,$f_{\mathrm{ice,sum}}$为地下水位所在及以下土壤层内总的冰的体积含量(定义为冰的总体积除以土壤层的总体积)。

\section{径流模型CaMa-Flood}
%\addcontentsline{toc}{chapter}{陆地表面的水分循环}

汇流计算是通过耦合 Dai Yamazaki 等人于2011年提出的大尺度分布式汇流模型 CaMa-Flood (Catchment-based Macro-scale Floodplain) 实现的\citep{yamazaki2011physically}。
CaMa-Flood将全球河流网络分割为称为流域单元 (unit catchment) 的水文单元,在各单元集水区内利用河道 (river channel) 和漫滩 (floodplain) 的次网格 (subgrid) 
地形参数以及陆面模式生成的产流量 (total runoff),对总蓄水量 (total storage) 以及总流量 (total discharge) 进行预测,
并进一步实现对河道及漫滩流量 (river discharge and flood discharge),漫滩面积 (flood area) 以及平均漫滩水深 (flood depth) 
等日常所需的诊断量的高速计算。总蓄水量和总流量的时间演变通过求解局部惯性方程 \citep{bates2010} 得出。
CaMa-Flood同时考虑了上游单元的入流、下游单元的流出和每个流域单元的径流强迫输入,是目前高速求解河道水动力方程-圣维南方程最为高速有效的方法之一。
关于CaMa-Flood模型的详细描述可以从相关文献获取~\citep{yamazaki2011physically,yamazaki2013improving,yamazaki2014development,yamazaki2014regional}。


CaMa-Flood 模型的主要优势之一在于除了能够精确描述河道径流以外,还能够模拟包括漫滩水位和漫滩面积变化等洪泛过程。
因此,对模拟结果的验证不仅仅局限于传统测量河流流量,还可以直接将模拟结果与卫星高度计对水面高程的观测以及微波成像仪对洪泛面积的估算进行比较;
能够有效加强全球河流模型各个输出的校准/验证~\citep{yamazaki2012analysis,yamazaki2012adjustment}。
CaMa-Flood 模型的另一个优势在于它具有极高的模拟计算效率。通过引入次网格地形参数,复杂的漫滩淹没物理过程被合理地简化。
与此同时,通过实现局部惯性方程和自适应时间步长等物理方案~\citep{bates2010},
河流流量和蓄水量的预测计算成本得到压缩,有利于进行集成/长期实验等计算要求高的实验或者与陆面模式之间的动态耦合。
以下对 CaMa-Flood 内部各个模块进行详细的介绍。目前耦合版本陆面过程模式分系统已经包含 CaMa-Flood v4.12版本。



\subsection{诊断洪泛状态}\label{诊断洪泛状态}
在 CaMa-Flood 中指定网格的漫滩 (洪泛) 状态是通过计算该网格的蓄水总量得出的。如图~\ref{fig:CaMa-Flood流域单位示意图}
所示,河流河道蓄水量$S_r$,漫滩蓄水量$S_f$,河道水深度$D_r$, 漫滩淹没深度$D_f$,漫滩面积$A_f$等均通过求解基于总蓄水量的水量方程得出。
首先,模式中引发当前流域单元洪水蓄水量$S_{ini}$由如下公式进行确定:
\begin{equation}
S_{ ini }=B WL
\end{equation}
其中$B$是河道深度,$W$是河道宽度,及$L$是河道长度。如果总蓄水量$S$小于等于引发洪水蓄水量$S_{ini}$,
CaMa-Flood 假设不存在漫滩 (洪泛) 事件,上述参数则通过如下方程计算得出:
\begin{equation}
    \begin{array}{l}S_r=S \\ D_r=\frac{S_r}{WL} \\ S_f=0 \\ D_f=0 \\ A_f=0 \\ S_f=0\end{array}
\end{equation}
当总蓄水量$S$大于引发洪水蓄水量$S_{ini}$时,触发漫滩 (洪泛) 事件,则上述参数通过联立如下方程计算得出:
\begin{equation}
\begin{aligned}
S_r &=S-S_f \\ 
D_r &=\frac{S_r}{W L} \\
S_f &=\int_{0}^{A_f}(D_f-D(A)) d_A \\
D_f &=D_r-B \\ 
A_f &=D^{-1}(D_f)
\end{aligned}
\end{equation}
上式中的$D_f = D_r - B$表示河道与漫滩的水面高度相同。该方程是基于河道与漫滩之间的水量瞬间完成交换的假设。
函数$D^{-1}(D_f)$是漫滩高程剖面$D(A_f)$的反函数,它将泛滥地区$A_f$描述为漫滩水深$D_f$的函数 (见图~\ref{fig:CaMa-Flood流域单位示意图})。


{
\begin{figure}[htbp]
\centering
\includegraphics[width=\textwidth]{Figures/陆地表面的水分循环/CaMa-Flood流域单位示意图.png}
\caption{CaMa-Flood流域单位示意图}
\label{fig:CaMa-Flood流域单位示意图}
\end{figure}
}

\subsection{河道径流流量计算}
CaMa-Flood 分别计算了各流域单元向其下游单元的河流径流量和漫滩流量。
二者的计算均通过忽略如下 St. Venant 动量方程式第二项,得到径流计算所使用的局部惯性方程~\citep{bates2010}:
\begin{equation}
\frac{\partial Q}{\partial t}+\frac{\partial}{\partial x}\left[\frac{Q^{2}}{A}\right]+\frac{g A \partial(h+z)}{\partial x}+\frac{g n^{2} Q^{2}}{R^{4 / 3} A}=0
\end{equation}
式中$Q$为河流流量 (\unit{m^3.s^{-1}}),$A$为水流横截面面积 (\unit{m^2}),$h$为水流深度 (m),$z$为河床高程 (m),
$R$为水力半径 (m),$g$为重力加速度 (\unit{m.s^{-2}}),$n$为曼宁摩擦系数(\unit{m^{-1/3}.s^{-1}})。
$x$和$t$分别为流动距离和时间。第一项、第二项、第三项和第四项分别表示局部加速度、平流、水面坡度和摩擦坡度。CaMa-Flood模型采用局部惯性方程的显式形式: 
%
\begin{equation}
Q^{t+\Delta t}=\frac{Q^{t}+\Delta t g A i S}{1+\frac{\Delta t g n^{2}\left|Q^{t \mid}\right|}{R^{4 / 3} A}}
\end{equation}
其中$S$是水面坡度,$Q^t$为当前时刻的流量, $Q^{t+\Delta t}$是单位时间间隔 $\Delta t$ 之后的流量。水力半径 $R$ 近似为水流深度。曼宁系数默认设置为$n=0.03$。
在局部惯性方程计算中可能出现的负向河流量,代表了下游流域单元向当前流域单元的反向水流(回水)。同时为防止当前网格的总的流出量超过蓄水量,
CaMa-Flood 引入限流器的概念:当总出水量大于网格的总库存量时,CaMa-Flood 使用修正系数对径流流量进行修正。


\subsection{洪水漫滩流量计算}
漫滩流量计算与河道径流流量计算方法相同。
其区别在于漫滩流量包括所使用的水流面积$A$的计算方法是漫滩蓄水量除以河道长度;
水流深度$h$为漫滩深度;漫滩流量的曼宁系数被设置为$n=0.10$。


\subsection{蓄水量变化计算}
{
\begin{figure}[htbp]
\centering
\includegraphics[width=0.8\textwidth]{Figures/陆地表面的水分循环/蓄水量变化计算流程图.png}
\caption{CaMa-Flood~蓄水量变化计算流程图 }
\label{fig:蓄水量变化计算流程图}
\end{figure}
}
蓄水量随时间的变化的计算流程如图~\ref{fig:蓄水量变化计算流程图} 所示,指定流域单元的蓄水量变化的计算基于质量平衡方程:
\begin{equation}
S_{i}^{t+\Delta t}=S_{i}^{t}+\sum_{k}^{Upstream} Q_{k}^{t} \Delta t-Q_{i}^{t} \Delta t+A c_{i} R_{i}^{t} \Delta t
\end{equation}
式中$S_{i}^{t}$和$S_{i}^{t+\Delta t}$分别代表单元$i$在时间$t$到时间$t+\Delta t$蓄水量的变化,$Q_i^t$代表在时间$t$该单元河流径流出流量 (河道内+漫滩),
$Q_k^t$代表在时间 $t$ 该单元从上游网格接收的河流径流流入流量 (河道内+漫滩),$Ac_i$ 是单元$i$的面积,$R_i^t$ 代表流域单元 $i$ 的产流量。


\subsection{自适应时间步长的估算}
为避免固定时间步长计算所产生的数值振荡,提高数值方案稳定性,
CaMa-Flood 采用了~\citet{bates2010}提出的基于局部惯性方程并满足 Courant-Friedrichs-Lewy (CFL) 
条件的自适应时间步长 ($DT_{adp}$) 的估算方法:
\begin{equation}
{DT}_{\max }={\alpha} \frac{\Delta x}{\sqrt{g h_{t}}}
\end{equation}
上式中$DT_{max}$是最大可接受的时间步长,$\delta{x}$是该流域单元连接下游流域单元的河道长度 (river length) (\unit{m}),
$\alpha$是稳定性系数设为0.97,$h_t$是该流域单元在时刻$t$的水流深度 (water depth) (\unit{m}),$g$是重力加速度设为9.81 (\unit{m.s^{-2}})。
图~\ref{fig:自适应时间步长的估算} 展示了基于上述公式计算的某一时刻的$DT_{max}$ (minute)。
在计算过程中,如果用户指定的默认时间步长$DT$大于$DT_{max}$,则$DT$将被划分为满足$CFL$条件的更小的时间等分的时间步长$DT_{adp}$;
如果用户指定的默认时间步长$DT$小于$DT_{max}$,则实际计算步长按照用户指定的默认时间步长。

{
\begin{figure}[htbp]
\centering
\includegraphics[width=0.8\textwidth]{Figures/陆地表面的水分循环/自适应时间步长的估算.png}
\caption{CaMa-Flood自适应时间步长的估算案例}
\label{fig:自适应时间步长的估算}
\end{figure}
}
\subsection{双向耦合过程}
{
\begin{figure}[htbp]
\centering
\includegraphics[width=0.8\textwidth]{Figures/陆地表面的水分循环/双向耦合.png}
\caption{CaMa-Flood和CoLM双向耦合示意图}
\label{fig:双向耦合}
\end{figure}
}

CoLM 和CaMa-Flood已经实现了双向耦合,能够进行对水文过程的完整描述。如图~\ref{fig:双向耦合}所示,其耦合模拟包含一下过程:
\begin{enumerate}
\item CoLM~模拟每个计算网格的地表能量平衡和地表水量平衡(融雪过程等)
\item CoLM~模拟地表径流、各次表层的土壤含水量和地下水动态变化,并将土壤/地下水动态变化,并将运移结果分别反馈给~CoLM~和~CaMa-Flood~
\item 通过接收CoLM的产流量,CaMa-Flood 计算河道径流、泛滥区域以及泛区水深,将是否发生洪水、洪水发生的区域、网格占比和洪水深度等相关结果反馈给CoLM
\item 如果发生洪水,则在下一个时间步长、在非洪泛的网格占比,按原有的下垫面类型进行正常的~CoLM~模拟。而在洪水发生的网格占比,则将洪水深度作类似降水水量处理,并假设洪泛网格占比的下垫面为水面进行陆面状态模拟。最后再通过面积加权平均求出各类通量和状态变量
\item 将加权平均后的下渗量和蒸发量传回给~CaMa-Flood,从总水量中扣除,更新水文水动力各项变量。

\end{enumerate}
%\chapter{径流模型CaMa-Flood}
%\addcontentsline{toc}{chapter}{陆地表面的水分循环}

汇流计算是通过耦合 Dai Yamazaki 等人于2011年提出的大尺度分布式汇流模型 CaMa-Flood (Catchment-based Macro-scale Floodplain) 实现的\citep{yamazaki2011physically}。
CaMa-Flood将全球河流网络分割为称为流域单元 (unit catchment) 的水文单元,在各单元集水区内利用河道 (river channel) 和漫滩 (floodplain) 的次网格 (subgrid) 
地形参数以及陆面模式生成的产流量 (total runoff),对总蓄水量 (total storage) 以及总流量 (total discharge) 进行预测,
并进一步实现对河道及漫滩流量 (river discharge and flood discharge),漫滩面积 (flood area) 以及平均漫滩水深 (flood depth) 
等日常所需的诊断量的高速计算。总蓄水量和总流量的时间演变通过求解局部惯性方程 \citep{bates2010} 得出。
CaMa-Flood同时考虑了上游单元的入流、下游单元的流出和每个流域单元的径流强迫输入,是目前高速求解河道水动力方程-圣维南方程最为高速有效的方法之一。
关于CaMa-Flood模型的详细描述可以从相关文献获取\citep{yamazaki2011physically,yamazaki2013improving,yamazaki2014regional,yamazaki2014development}。


CaMa-Flood 模型的主要优势之一在于除了能够精确描述河道径流以外,还能够模拟包括漫滩水位和漫滩面积变化等洪泛过程。
因此,对模拟结果的验证不仅仅局限于传统测量河流流量,还可以直接将模拟结果与卫星高度计对水面高程的观测以及微波成像仪对洪泛面积的估算进行比较;
能够有效加强全球河流模型各个输出的校准/验证\citep{yamazaki2012adjustment,yamazaki2012analysis}。
CaMa-Flood 模型的另一个优势在于它具有极高的模拟计算效率。通过引入次网格地形参数,复杂的漫滩淹没物理过程被合理地简化。
与此同时,通过实现局部惯性方程和自适应时间步长等物理方案\citep{bates2010},
河流流量和蓄水量的预测计算成本得到压缩,有利于进行集成/长期实验等计算要求高的实验或者与陆面模式之间的动态耦合。
以下对 CaMa-Flood 内部各个模块进行详细的介绍。目前耦合版本陆面过程模式分系统已经包含 CaMa-Flood v4.07版本。



\section{诊断洪泛状态}\label{诊断洪泛状态}
在 CaMa-Flood 中指定网格的漫滩 (洪泛) 状态是通过计算该网格的蓄水总量得出的。如图~\ref{fig:CaMa-Flood流域单位示意图}
所示,河流河道蓄水量$S_r$,漫滩蓄水量$S_f$,河道水深度$D_r$, 漫滩淹没深度$D_f$,漫滩面积$A_f$等均通过求解基于总蓄水量的水量方程得出。
首先,模式中引发当前流域单元洪水蓄水量$S_{ini}$由如下公式进行确定:
\begin{equation}
S_{ ini }=B WL
\end{equation}
其中$B$是河道深度,$W$是河道宽度,及$L$是河道长度。如果总蓄水量$S$小于等于引发洪水蓄水量$S_{ini}$,
CaMa-Flood 假设不存在漫滩 (洪泛) 事件,上述参数则通过如下方程计算得出:
\begin{equation}
    \begin{array}{l}S_r=S \\ D_r=\frac{S_r}{WL} \\ S_f=0 \\ D_f=0 \\ A_f=0 \\ S_f=0\end{array}
\end{equation}
当总蓄水量$S$大于引发洪水蓄水量$S_{ini}$时,触发漫滩 (洪泛) 事件,则上述参数通过联立如下方程计算得出:
\begin{equation}
\begin{array}{l}S_r=S-S_f \\ D_r=\frac{S_r}{W L} \\ S_f=\int_{0}^{A_f}(D_f-D(A)) d_A \\ D_f=D_r-B \\ A_f=D^{-1}(D_f)\end{array}
\end{equation}
上式中的$D_f = D_r - B$表示河道与漫滩的水面高度相同。该方程是基于河道与漫滩之间的水量瞬间完成交换的假设。
函数$D^{-1}(D_f)$是漫滩高程剖面$D(A_f)$的反函数,它将泛滥地区$A_f$描述为漫滩水深$D_f$的函数 (见图~\ref{fig:CaMa-Flood流域单位示意图})。


{
\begin{figure}[htbp]
\centering
\includegraphics{Figures/陆地表面的水分循环/CaMa-Flood流域单位示意图.png}
\caption{CaMa-Flood流域单位示意图,摘自\citet{yamazaki2011physically}。 }
\label{fig:CaMa-Flood流域单位示意图}
\end{figure}
}

\section{河道径流流量计算}
CaMa-Flood 分别计算了各流域单元向其下游单元的河流径流量和漫滩流量。
二者的计算均通过忽略如下 St. Venant 动量方程式第二项,得到径流计算所使用的局部惯性方程 \cite{bates2010}:
\begin{equation}
\frac{\partial Q}{\partial t}+\frac{\partial}{\partial x}\left[\frac{Q^{2}}{A}\right]+\frac{g A \partial(h+z)}{\partial x}+\frac{g n^{2} Q^{2}}{R^{4 / 3} A}=0
\end{equation}
式中$Q$为河流流量 (\unit{m^3.s^{-1}}),$A$为水流横截面面积 (\unit{m^2}),$h$为水流深度 (m),$z$为河床高程 (m),
$R$为水力半径 (m),$g$为重力加速度 (\unit{m.s^{-2}}),$n$为曼宁摩擦系数(\unit{m^{-1/3}.s^{-1}})。
$x$和$t$分别为流动距离和时间。第一项、第二项、第三项和第四项分别表示局部加速度、平流、水面坡度和摩擦坡度。CaMa-Flood模型采用局部惯性方程的显式形式: 

\begin{equation}
Q^{t+\Delta t}=\frac{Q^{t}+\Delta t g A i S}{1+\frac{\Delta t g n^{2}\left|Q^{t \mid}\right|}{R^{4 / 3} A}}
\end{equation}
其中$S$是水面坡度,$Q^t$为当前时刻的流量, $Q^{t+\Delta t}$是单位时间间隔 $\Delta t$ 之后的流量。水力半径 $R$ 近似为水流深度。曼宁系数默认设置为$n=0.03$。
在局部惯性方程计算中可能出现的负向河流量,代表了下游流域单元向当前流域单元的反向水流(回水)。同时为防止当前网格的总的流出量超过蓄水量,
CaMa-Flood 引入限流器的概念:当总出水量大于网格的总库存量时,CaMa-Flood 使用修正系数对径流流量进行修正。


\section{洪水漫滩流量计算}
漫滩流量计算与河道径流流量计算方法相同。
其区别在于漫滩流量包括所使用的水流面积$A$的计算方法是漫滩蓄水量除以河道长度;
水流深度$h$为漫滩深度;漫滩流量的曼宁系数被设置为$n=0.10$。

\section{蓄水量变化计算}
{
\begin{figure}[htbp]
\centering
\includegraphics{Figures/陆地表面的水分循环/蓄水量变化计算流程图.png}
\caption{蓄水量变化计算流程图。 }
\label{fig:蓄水量变化计算流程图}
\end{figure}
}
蓄水量随时间的变化的计算流程如图~\ref{fig:蓄水量变化计算流程图} 所示,指定流域单元的蓄水量变化的计算基于质量平衡方程:
\begin{equation}
S_{i}^{t+\Delta t}=S_{i}^{t}+\sum_{k}^{Upstream} Q_{k}^{t} \Delta t-Q_{i}^{t} \Delta t+A c_{i} R_{i}^{t} \Delta t
\end{equation}
式中$S_{i}^{t}$和$S_{i}^{t+\Delta t}$分别代表单元$i$在时间$t$到时间$t+\Delta t$蓄水量的变化,$Q_i^t$代表在时间$t$该单元河流径流出流量 (河道内+漫滩),
$Q_k^t$代表在时间 $t$ 该单元从上游网格接收的河流径流流入流量 (河道内+漫滩),$Ac_i$ 是单元$i$的面积,$R_i^t$ 代表流域单元 $i$ 的产流量。


\section{自适应时间步长的估算}
为避免固定时间步长计算所产生的数值振荡,提高数值方案稳定性,
CaMa-Flood 采用了\citet{bates2010}提出的基于局部惯性方程并满足 Courant-Friedrichs-Lewy (CFL) 
条件的自适应时间步长 ($DT_{adp}$) 的估算方法:
\begin{equation}
{DT}_{\max }=\boldsymbol{\alpha} \frac{\Delta x}{\sqrt{g h_{t}}}
\end{equation}
上式中$DT_{max}$是最大可接受的时间步长,$∆x$是该流域单元连接下游流域单元的河道长度 (river length) (m),
$\alpha$是稳定性系数设为0.9,$h_t$是该流域单元在时刻$t$的水流深度 (water depth) (m),$g$是重力加速度设为9.81 (\unit{m.s^{-2}})。
图~\ref{fig:自适应时间步长的估算} 展示了基于上述公式计算的某一时刻的$DT_{max}$ (minute)。
在计算过程中,如果用户指定的默认时间步长$DT$大于$DT_{max}$,则$DT$将被划分为满足$CFL$条件的更小的时间等分的时间步长$DT_{adp}$;
如果用户指定的默认时间步长$DT$小于$DT_{max}$,则实际计算步长按照用户指定的默认时间步长。

{
\begin{figure}[htbp]
\centering
\includegraphics{Figures/陆地表面的水分循环/自适应时间步长的估算.png}
\caption{自适应时间步长的估算,摘自\citet{yamazaki2013improving}。}
\label{fig:自适应时间步长的估算}
\end{figure}
}

\chapter{基于物理过程的侧向流模拟}\label{ch:侧向流模拟}

CoLM中基于流域单元网格,发展了对坡面流、河道径流和地下水侧向流的模拟方案(图~\ref{fig:主要水文过程})。

{
\begin{figure}[htbp]
\centering
\includegraphics[width=\textwidth]{Figures/侧向流/主要水文过程.jpg}
\caption{基于物理过程的侧向流模拟中的主要水文过程}
\label{fig:主要水文过程}
\end{figure}
}

河道径流发生在流域单元之间。流域单元网格基于集水和汇流关系对模拟区域进行剖分,一个流域单元与一段河道相关联。模式中假设一个流域单元内的水分在地形的主要作用下汇集到河道后,在河道中沿地势而向下流动。

坡面流发生在高度带单元之间,在流域单元内部进行。流域单元内部的高度带都是连通的,且每个高度带具有唯一的下游单元,因此,水流在高度带单元之间的流动路径是清晰的,可基于物理方程进行模拟。通过将河道作为最低处的高度带,CoLM实现了对河道径流和坡面流的相互作用的描述。

地下水的侧向流动包含三个级别:流域单元之间、高度带单元之间以及次网格单元之间。在模拟高度带单元之间的地下水侧向流时,通过将河道作为最低处的单元,实现了对河道径流和地下水侧向流的相互作用的描述。

\section{坡面流、河道径流}

\begin{mymdframed}{代码}
本节对应的代码文件为\texttt{HYDRO/MOD\_Hydro\_HillslopeFlow.F90} \\
和\texttt{HYDRO/MOD\_Hydro\_RiverLakeFlow.F90}.
\end{mymdframed}

{
\begin{figure}[htbp]
\centering
\includegraphics[width=\textwidth]{Figures/侧向流/坡面流和河道径流.pdf}
\caption{CoLM中的坡面流和河道径流}
\label{fig:坡面流和河道径流}
\end{figure}
}

对坡面流和河道径流的模拟基于完整的浅水波方程.
 \subsection{浅水波方程}
 质量方程
  \begin{equation}
 \frac{\partial h}{\partial t} + \frac{\partial \left(uh\right)}{\partial s} = 0
 \end{equation}
 动量方程
 \begin{equation}
 \frac{\partial \left(uh\right)}{\partial t} + \frac{\partial}{\partial s}\left(u^2h+\frac{1}{2}gh^2\right) = -gh\frac{\partial z_b}{\partial s}-gh\frac{n^2\left|q\right|}{h^{10/3}}q
  \end{equation}
  其中,$h$表示水深(单位m),$u$表示水流速度(单位 \unit{m.s^{-1}}),$t$表示时间(单位s),$s$表示沿水流路径的长度(单位m),$q=uh$表示单位宽度的流量(单位 \unit{m^2.s^{-1}}),$z_b$表示坡面的高度(单位m),$n$表示曼宁系数(单位 \unit{m^{-1/3}.s})。


\subsection{数值离散格式}
坡面流是指在一个流域单元内部地表水的流动,它主要描述地表水从集水区域到河道的过程,即产流的过程。CoLM中将一个流域单元划分为多个面积相近的高度带单元,作为计算坡面流的基本离散单元。一般情况下,水流由地势较高的单元流向地势较低的单元,发生洪水时,也可由河道向上淹没地势较高的区域。浅水波方程的预报变量水深$h$和水流速度$u$定义在高度带单元上(图~\ref{fig:坡面流和河道径流})。

河道径流是指流域单元之间地表水的流动,主要沿河道进行。CoLM基于水文学数据划分流域单元,与流域单元关联的分段河道之间有明确的上下游关系,为计算河道径流的基本离散单元。一般情况下,水流由地势较高的河道流向地势较低的河道,可经过湖泊或者水库,最后汇流入海,或者终止于内陆洼地。浅水波方程的预报变量水深$h$和水流速度$u$定义在分段河道上(图~\ref{fig:坡面流和河道径流})。

{
\begin{figure}[htbp]
\centering
\includegraphics[width=0.8\textwidth]{Figures/侧向流/浅水波方程求解.pdf}
\caption{浅水波方程求解方案中的变量}
\label{fig:浅水波方程求解}
\end{figure}
}

 质量方程离散为
  \begin{equation} \label{formula:mass_swe}
 \frac{ h^{n+1}_i - h^n_i}{\Delta t} A_i+\sum_{j\in K_i} Q^n_{ij} = 0
 \end{equation}
 动量方程离散为
 \begin{equation}
 \frac{ \left(uh\right)^{n+1}_i - \left(uh\right)^n_i}{\Delta t} A_i + \sum_{j\in K_i} F^n_{ij} = \sum_{j\in K_i} G^n_{ij}  + T^{n+1}_i  A_i \label{swe-d-2}
  \end{equation}
其中(图~\ref{fig:浅水波方程求解}),
\begin{itemize}
\item $n$和$n+1$表示时间步数,时间步长为$\Delta t$;
\item $C_i$表示标号为$i$的空间单元,$A_i$表示$C_i$的面积;
\item $K_i$表示与$C_i$有水分交换的其余空间单元的标号的集合;
\item $Q_{ij}$表示从$C_i$到$C_j$的质量通量,$F_{ij}$表示从$C_i$到$C_j$的动量通量;
\item $G_{ij}$表示$C_i$与$C_j$之间由坡面梯度形成的动量的源汇项;
\item $T_i$表示$C_i$上坡面的摩擦力。
\end{itemize}

假设两个相邻的水文单元为$C_i$和$C_j$,其中$C_i$为上游单元,$C_j$为下游单元。离散格式中各项的计算方法如下:

{
\begin{figure}[htbp]
\centering
\includegraphics[width=0.6\textwidth]{Figures/侧向流/水深重构.pdf}
\caption{浅水波方程求解方案中对水深的重构(公式见~\ref{formula:water_depth_reconstruction})}
\label{fig:水深重构}
\end{figure}
}

(1)首先对边界上的$h$进行重构~\citep{audusse2004scientificcomputing}(见图~\ref{fig:水深重构}),
\begin{equation} \label{formula:water_depth_reconstruction}
\begin{aligned}
z_{b,ij} & = \max\left(z_{b,i}, z_{b,j}\right) \\
h_{ij-} & =  \max\left(0, h_i + z_{b,i} - z_{b,ij} \right) \\
h_{ij+} & =  \max\left(0, h_j + z_{b,j} - z_{b,ij} \right)
\end{aligned}
\end{equation}
其中$h_{ij-}$表示界面处$C_i$一侧的重构值,$h_{ij+}$表示界面处$C_j$一侧的重构值。

(2)局部黎曼问题中,中间区域的速度$u_{ij*}$计算为
	\begin{equation}
		u_{ij*} = \frac{1}{2}\left(u_i + u_j\right) + \sqrt{g h_{ij-}} - \sqrt{g h_{ij+}}
	\end{equation}
其中,$u_i$和$u_j$分别为$C_i$和$C_j$中的水流的速度。 

局部黎曼问题中,中间区域的水深$h_{ij*}$计算为
	\begin{equation}
		h_{ij*} = \frac{1}{g}\left[\frac{1}{2}\left(\sqrt{g h_{ij-}} + \sqrt{g h_{ij+}}\right) + \frac{1}{4}\left(u_i - u_j\right)\right]^2
	\end{equation}

(3)上游波速$S_{ij-}$计算为
	\begin{equation}
		S_{ij-} = \left\{
		\begin{aligned}
			&u_j - 2\sqrt{gh_{ij+}}, && \mbox{if} \quad h_{ij-} = 0 \\
			& \min \left(u_i - \sqrt{gh_{ij-}}, u_{ij*} - \sqrt{gh_{ij*}}\right), && \mbox{if} \quad h_{ij-} > 0
		\end{aligned}\right.
	\end{equation}
下游波速$S_{ij+}$计算为
	\begin{equation}
		S_{ij+} = \left\{
		\begin{aligned}
			&u_i + 2\sqrt{gh_{ij-}}, && \mbox{if} \quad h_{ij+} = 0 \\
			& \max \left(u_j + \sqrt{gh_{ij+}}, u_{ij*} + \sqrt{gh_{ij*}}\right), && \mbox{if} \quad h_{ij+} > 0
		\end{aligned}\right.
	\end{equation}

(4)边界上的通量计算为
\begin{equation}
	Q_{ij} = L_{ij} \cdot \left\{
	\begin{aligned}
		& Q_{ij-}, && \mbox{if} \quad 0\leqslant S_{ij-} \\
		& \frac{S_{ij+} Q_{ij-} - S_{ij-} Q_{ij+} + S_{ij-} S_{ij+} \left(h_{ij+} - h_{ij-}\right)}{S_{ij+}-S_{ij-}} , && \mbox{if} \quad S_{ij-} \leqslant 0\leqslant S_j\\
		& Q_{ij+}, && \mbox{if} \quad 0\geqslant S_{ij+}
	\end{aligned}\right.
\end{equation}

\begin{equation}
F_{ij} = L_{ij} \cdot 
\begin{cases}
	 F_{ij-}, & \mbox{if} \quad 0\leqslant S_{ij-} \\
	 \frac{S_{ij+} F_{ij-} - S_{ij-} F_{ij+} + S_{ij-} S_{ij+} \left(q_{ij+} - q_{ij-}\right)}{S_{ij+}-S_{ij-}} , & \mbox{if} \quad S_{ij-} \leqslant 0\leqslant S_{ij+}\\
	F_{ij+}, & \mbox{if} \quad 0\geqslant S_{ij+}
\end{cases}
\end{equation}
其中,$L_{ij}$为单元$C_i$与$C_j$之间交界线的长度,边界两侧的重构量
\begin{align*}
&Q_{ij-} = q_{ij-} = u_i h_{ij-},  && Q_{ij+} = q_{ij+} = u_j h_{ij+}, \\
&F_{ij-} = u_i^2h_{ij-}+\frac{1}{2}gh_{ij-}^2, && F_{ij+} = u_j^2h_{ij+}+\frac{1}{2}gh_{ij+}^2
\end{align*}

(5)由坡面梯度形成的动量的源汇项计算为~\citep{audusse2004scientificcomputing}
\begin{equation}
G_{ij} = \frac{1}{2}gh_{ij-}^2 \cdot L_{ij}, \quad
G_{ji}=-\frac{1}{2}g h_{ij+}^2 \cdot L_{ij}
\end{equation}
注意,$G_{ji}$不出现在$C_i$单元的离散动量方程(\ref{swe-d-2})中,而是在$C_j$单元的离散动量方程中使用. 因为不是通量项,所以$G_{ij}$与$G_{ji}$不一定互为相反数。

通量项满足$Q_{ji} = -Q_{ji}$,$F_{ji}=-F_{ji}$.

(6)摩擦力项采用半隐离散格式
\begin{equation}
T^{n+1}_i = -g \left(\frac{n^2}{h^{7/3}} \left|q\right|\right)^n_i q^{n+1}_i
\end{equation}
代入离散后的动量方程可得,
 \begin{equation}
 \frac{ q^{n+1}_i - q^n_i}{\Delta t} A_i + \sum_{j\in K_i} F^n_{ij} = \sum_{j\in K_i} G^n_{ij}  -g \left(\frac{n^2}{h^{7/3}} \left|q\right|\right)^n_i q^{n+1}_i  A_i 
  \end{equation}

综合起来,单宽流量的更新采用下列格式
\begin{equation}
q^{n+1}_i = \frac{q^n_i - \frac{\Delta t}{A_i}\sum_{j\in K_i} \left(F^n_{ij} - G^n_{ij}\right)}{1 + g \left(\frac{n^2}{h^{7/3}} \left|q\right|\right)^n_i \Delta t}
\end{equation}

\subsection{自适应时间步长方法}
对时间步长的约束包含三个,$\Delta t = \min \Delta t_i \quad (i=1, \ldots, N)$,
\begin{enumerate}
\item CFL条件
\begin{equation}
\qquad \Delta t_i \leqslant \mathrm{C}\frac{ D_i }{\left| u_{i}\right| + \sqrt{gh_{i}}}
\end{equation}
其中,$D_i$为第$i$个单元的平均水流路径长度,C为Courant数,模式里取值为0.8。
\item 质量限制:
当$\sum_{j\in K_i} Q_{ij}>0$时,
  \begin{equation}
 \Delta t_i \leqslant \frac{h^n_i\cdot A_i}{\sum_{j\in K_i} Q_{ij}}
 \end{equation}
\item 动量限制:
当$q^n_i \cdot \left[ \frac{1}{A_i}\sum_{j\in K_i} \left(F^n_{ij} - G^n_{ij} \right)\right] > 0$ 且 $\mathrm{abs}\left(q^n_i\right) > q_{min}$时,
  \begin{equation}
 \Delta t_i \leqslant \frac{q^n_i}{\frac{1}{A_i}\sum_{j\in K_i} \left(F^n_{ij} - G^n_{ij} \right)}
 \end{equation}
其含义为在一个时间步长内,速度不改变方向。
 \end{enumerate}

\subsection{数值算法在模式中的实现}

\subsubsection{参数的计算}
离散算法中需要的参数有平均水流路径的长度$D_i$,两个单元的边界线的长度$L_{ij}$,单元的面积$A_i$和单元表面的高度$z_{b,i}$,以及曼宁系数$n$.

对高度带单元:1)单元面积$A_i$指土壤、城市和湿地面积的总和(排除冰川和水体);2)单元表面的高度$z_{b,i}$定义为排水高度(一个像素点和它流入的河道点的高度差Height Above Nearest Drainage, HAND);3)长度$L_{ij}$为两个单元的边界线的长度;4)单元内一个像素点的水流路径定义为:自这个像素点开始,沿水流方向直至离开这个单元前的最后一个像素点之间的距离;平均水流路径定义为所有像素点的水流路径的算术平均值;5)曼宁系数$n$取常数$0.3$。

对分段河道:1)单元面积$A_i$指河道的面积;2)单元表面的高度$z_{b,i}$定义为河床的高度;3)长度$L_{ij}$为相邻两段河道的宽度的算术平均值,河道的宽度使用河道的面积除以河道的长度计算得到;4)平均水流路径$D_i$为河道的长度;5)曼宁系数$n$取常数$0.03$。

河道深度根据集水区域上的产流量进行估算,采用CaMa-Flood~(v4)~\citep{yamazaki2011physically}中的经验公式,
\begin{equation}
B = \max \left[0.1\times R_{\mathrm{up}}^{0.5}, 1.00\right]
\end{equation}
其中,$R_{\mathrm{up}}$为河道上游集水区域的平均产流速度(\unit{m^3.s^{-1}})。

\subsubsection{洪泛过程} \label{sec:lateral_flood}

{
\begin{figure}[htbp]
\centering
\includegraphics[width=0.8\textwidth]{Figures/侧向流/发生洪泛时的水深.pdf}
\caption{发生洪泛的水深示意图}
\label{fig:lateral_flood}
\end{figure}
}
在河道径流的模拟中,当河道中的水深超过河道深度时,即发生洪泛事件。模式中不考虑发生洪泛事件时河道与高度带单元之间的水动力过程,因此,由预报方程~\eqref{formula:mass_swe}得到总水量$V^{n+1}_i$后,超出河道深度的水量按照从低到高的顺序分配到高度带单元上~(图~\ref{fig:lateral_flood})。假设由低到高的高度带单元(河道为最低的单元)的面积和高度分别为$A_{0},A_{1},\ldots,A_{J}$和$d_{0},d_{1},\ldots,d_{J}$,则从总水量$V$到水深$h$的分段函数定义为
\begin{equation}
    h(V) = \begin{cases}
			\frac{V}{A_0}, & V \leqslant A_{0} \left(d_1 - d_0\right)\\
            \frac{V-A_{0} \left(d_1 - d_0\right)}{A_0+A_1}, & A_{0} \left(d_1 - d_0\right) < V \leqslant A_{0} \left(d_2 - d_0\right) + A_{1} \left(d_2 - d_1\right) \\
            \cdots, & \cdots \\
            \frac{V-\sum^{j-1}_{k=0}A_k(d_{j}-d_k)}{\sum^j_{k=0}A_k}, & \sum^{j-1}_{k=0}A_k(d_{j}-d_k) < V \leqslant \sum^{j}_{k=0}A_k(d_{j+1}-d_k)\\
            \cdots, & \cdots  \\
            \frac{V-\sum^{J-1}_{k=0}A_k(d_{J}-d_k)}{\sum^J_{k=0}A_k}, & \sum^{J-1}_{k=0}A_k(d_{J}-d_k) < V 
		 \end{cases}
\end{equation}

总水量$V$和水深$h$之间的映射为一一映射,由水深也可计算得到总水量。模式中计算河道径流时,水深定义为流域单元内最低的水面位置到河道底部的距离。

\subsubsection{湖泊单元}
湖泊的水量变化与与河道径流一起进行计算。在模式中,一个完整的湖泊(边界来自于HydroLAKES数据)设置为一个湖泊单元,因此,其面积不受流域单元面积阈值的限制。对面积较大的湖泊单元,将其进一步划分为多个小单元(划分方法见图~\ref{fig:湖泊划分})。使用高精度的湖泊深度数据,可聚合得到这些小单元上的平均深度,根据小单元的面积和深度可建立总水量和水深的函数映射(方法同\ref{sec:lateral_flood})。对湖泊单元,始终使用总水量和水深的函数映射,湖泊中的水深和总水量的更新步骤为:1)使用水深计算湖泊单元与邻居单元之间的水流通量;2)使用水流通量计算变化后的总水量;3)根据更新后的总水量,使用函数映射计算新的水深。对流域单元,只在发生洪泛事件时使用总水量和水深的函数映射。

目前模式中未考虑湖泊水体的动量的变化。

\section{地下水侧向流}

\begin{mymdframed}{代码}
本节对应的代码文件为\texttt{HYDRO/MOD\_Hydro\_SubsurfaceFlow.F90}.
\end{mymdframed}

CoLM中计算三个等级上的单元之间的地下水侧向流:流域单元之间、高度带单元之间以及次网格单元之间(图~\ref{fig:地下水侧向流})。模型基于坡面蓄水型Boussinesq方程。

{
\begin{figure}[htbp]
\centering
\includegraphics[width=\textwidth]{Figures/侧向流/地下水侧向流.pdf}
\caption{CoLM中的地下水侧向流}
\label{fig:地下水侧向流}
\end{figure}
}

\subsection{方程及数值算法}
\subsubsection{坡面蓄水型Boussinesq方程}
坡面蓄水型Boussinesq方程为
\begin{equation}
\frac{\partial \left(fh\right)}{\partial t} = \frac{\partial}{\partial x} \left[\cos i \cdot \left(kh\frac{\partial h}{\partial x}\right)+\sin i\cdot k\frac{\partial h}{\partial x}\right]
\end{equation}
其中,$h$表示饱和水层在垂直于不透水面方向上的厚度(单位m);$t$表示时间(单位s);$x$表示沿不透水面向上的方向(单位m);$i$表示不透水面的倾斜角;$k$表示水力导度(单位 \unit{m.s^{-1}});$f$表示可排水的孔隙度(单位 \unit{m.m^{-1}}).

\subsubsection{方程的简化}
假设垂直于不透水面的方向为$y$方向(从地表向下为正方向),土壤厚度为$H$,当$k$在$y$方向有变化时,$kh$替换为$y$方向的积分
\begin{equation}
\int^H_{H-h} k \  \mathrm{d}y
\end{equation}
记地下水位为$z_{wt}=H-h$,代入可得
\begin{equation}
\frac{\partial \left(fz_{wt}\right)}{\partial t} = \frac{\partial}{\partial x} \left[ \cos i \cdot \left(\int^{H}_{z_{wt}} k \mathrm{d}y\right)\frac{\partial z_{wt}}{\partial x} +\sin i\cdot k\frac{\partial z_{wt}}{\partial x}\right]
\end{equation}
当不考虑不透水面(基岩)时,可令$H\to \infty$,$i=0$,则方程变为
\begin{equation}
\frac{\partial \left(fz_{wt}\right)}{\partial t} = \frac{\partial}{\partial x} \left[ \left(\int^\infty_{z_{wt}} k\ \mathrm{d}z\right)\frac{\partial z_{wt}}{\partial x} \right]
\end{equation}

\subsubsection{通量的计算:相邻单元之间的半隐格式}
假设相邻水文单元之间交界面的宽度为$w$,则它们之间的水流通量为
$$q=-w\cdot \left(\int^\infty_{z_{wt}} k\ \mathrm{d}z\right)\frac{\partial z_{wt}}{\partial x} $$
水分只在两个单元$C_i$和$C_j$($x$由$C_i$指向$C_j$)之间交换时,方程可离散为
\begin{eqnarray}
z_{wt,i}^{n+1} &=& z_{wt,i}^{n} - \frac{\Delta t}{g_i}\left[ -  w k^n_{ij} \frac{z_{wt,j}^{n+1} - z_{wt,i}^{n+1}}{d_i+d_j} \right], \\
z_{wt,j}^{n+1} &=& z_{wt,j}^{n} + \frac{\Delta t}{g_j}\left[ -  w k^n_{ij} \frac{z_{wt,j}^{n+1} - z_{wt,i}^{n+1}}{d_i+d_j} \right]
\end{eqnarray}
其中,$k^n_{ij}$为界面处的导水率,$g_i=f_iA_i,g_j=f_jA_j$为孔隙度和面积之积。

由此可解出
\begin{eqnarray}
z_{wt,i}^{n+1} &=& \frac{g_i\left(g_j+\sigma\right)z_{wt,i}^{n} +g_j\sigma z_{wt,j}^{n}}{g_ig_j+\left(g_i+g_j\right)\sigma},\\
z_{wt,j}^{n+1} &=& \frac{g_i\sigma z_{wt,i}^{n} +g_j\left(g_i+\sigma\right)z_{wt,j}^{n}}{g_ig_j+\left(g_i+g_j\right)\sigma}\quad \\
\sigma &=& \frac{\Delta t wk_{ij}^n}{d_i+d_j}
\end{eqnarray}
从而水流通量为
\begin{equation}\label{formula:subsurface_lateral_flow}
q = \frac{\sigma}{\Delta t} \frac{g_ig_j}{g_ig_j+\left(g_i+g_j\right)\sigma} \left( z_{wt,i}^{n} - z_{wt,j}^{n} \right)
\end{equation}

公式~\eqref{formula:subsurface_lateral_flow}应用于流域单元之间和高度带单元之间的地下水侧向流计算中。对$C_i$与所有相邻单元之间的水流通量进行叠加,即可得到$C_i$单元上的地下水总量的变化。

\subsubsection{参数的计算}
\begin{itemize}
\item 导水率的计算(来自TOPMODEL)
\begin{equation}
k(z) = K_0\exp{(- z/b)},\quad \int^\infty_{z_{wt}} k\ \mathrm{d}z = b K_0\exp{(-z_{wt}/b)}
\end{equation}
\end{itemize}

\subsubsection{次网格单元之间的地下水交换}
第$i$个高度带单元内部第$p$个patch的地下水侧向流通量为
\begin{equation}
q^{\mathrm{I}}_{i,p} = b K_0 \exp{(-z_{wt,i}/b)}\cdot\frac{z_{wt,i,p}-z_{wt,i}}{\sqrt{A_i/\pi}}
\end{equation}
其中
$$z_{wt,i} = \sum^P_{p=1} \alpha_p z_{wt,i,p}$$
$\alpha_p$为第$p$个patch在整个单元占据的面积比。

不难验证
$$\sum^P_{p=1} \alpha_p q^{\mathrm{I}}_{i,p} =0$$
即$q^{\mathrm{I}}_{i,p}$表达了高度带单元内部的地下水交换。

地下水位和土壤水的变化最终在次网格单元上计算。模式中假设,由流域单元之间的地下水侧向流引起的地下水量的变化在流域单元内部是均匀的,由高度带单元之间的地下水侧向流引起的地下水量的变化在高度带单元内部也是均匀的。因此,在次网格单元上,地下水的总变化量为三个等级的单元上的变化量之和(图~\ref{fig:地下水侧向流})。

由地下侧向流引起的土壤水和地下水的变化计算方案同~\ref{sec:change_of_zwt_vsf}节。

% 若次网格单元上地下水量是减少的($\Delta W < 0$),模式中采用“预估-调整”的方式逐层减少土壤水,降低地下水位。图~\ref{fig:地下水变化}显示了在第$i$层内对地下水位和土壤水的计算方法。

% {
% \begin{figure}[htbp]
% \centering
% \includegraphics[width=\textwidth]{Figures/侧向流/地下水变化.pdf}
% \caption{地下水减少时地下水位及土壤水变化的计算方案示意图}
% \label{fig:地下水变化}
% \end{figure}
% }

% 在“预估”步,根据目前的地下水位及需减少的水量,预估降低后的地下水位,其满足
% \begin{equation} \label{formula:prediction_zwt}
% \begin{aligned}
% \theta_{\mathrm{pre}} & = \Theta_i\left(\psi_{\mathrm{s},i} - 0.5\left(z_{\mathrm{wt}}^{\mathrm{new}} - z_{\mathrm{wt}}^{\mathrm{old}}\right)\right) \\
%     - \Delta W & = \left(z_{\mathrm{wt}}^{\mathrm{new}} - z_{\mathrm{wt}}^{\mathrm{old}}\right) \times  \left(\theta_{\mathrm{s},i} - \theta_{\mathrm{pre}}\right)
% \end{aligned}
% \end{equation}
% 其中,$z_{\mathrm{wt}}^{\mathrm{old}}$为目前的地下水位,$z_{\mathrm{wt}}^{\mathrm{new}}$为预估的地下水位,$\theta_{\mathrm{s},i}$为饱和含水量,$\psi_{\mathrm{s},i}$为饱和时的土壤水势,$\Theta_i$表示第$i$层土壤内由土壤水势到土壤含水量的映射。式~\eqref{formula:prediction_zwt}为$z_{\mathrm{wt}}^{\mathrm{new}}$的隐函数,可用迭代方法求解。

% 若$z_{\mathrm{wt}}^{\mathrm{new}}$低于第$i$层土壤的下边界$z_{i+1/2}$,则先计算$z_{\mathrm{wt}}^{\mathrm{old}}$至$z_{i+1/2}$区域内的土壤水势和含水量为
% \begin{equation} \label{formula:adjust_zwt1}
% \begin{aligned}
% \psi_i^{\prime} & = \psi_{\mathrm{s},i} - \left(z_{\mathrm{wt}}^{\mathrm{new}} - z_{i+\frac{1}{2}}\right) -0.5\left( z_{i+\frac{1}{2}} - z_{\mathrm{wt}}^{\mathrm{old}}\right) \\
% \theta_i^{\prime} & = \Theta_i\left(\psi_i^{\prime}\right) 
% \end{aligned}
% \end{equation}
% 然后将地下水位$z_{\mathrm{wt}}^{\mathrm{new}}$“调整”至$z_{i+1/2}$. 此种情况下,土壤水的减少量小于$- \Delta W$,需继续减少第$i+1$层的土壤水量。

% 若$z_{\mathrm{wt}}^{\mathrm{new}}$不低于第$i$层土壤的下边界$z_{i+1/2}$,则只需更新$z_{\mathrm{wt}}^{\mathrm{old}}$至$z_{\mathrm{wt}}^{\mathrm{new}}$区域内的的土壤含水量为$\theta_i^{\prime} = \theta_{\mathrm{pre}}$.


% 若次网格单元上的地下水量是增加的($\Delta W > 0$),则从地下水位所在土壤层开始,向上逐层补充土壤水,抬升地下水位。方法如下:1)剩余补充量的初始值为$\Delta W$;2)对第$i$层进行补充时,若剩余的补充量超过了第$i$层的空气体积$A_i$,则将土壤水补充至饱和,剩余的补充量减少$A_i$,继续对第$i-1$层土壤水进行补充,地下水位抬升至第$i-1$层土壤的下边界;对第$i$层进行补充时,若剩余的补充量小于第$i$层的空气体积$A_i$,则将剩余的补充量全部补充到第$i$层的非饱和区,增加土壤的体积含水量,地下水位不变;3)若整个土壤层全部达到饱和后补充量还有剩余,则将剩余的水分补充至地表积水。

\chapter{湖泊模式}\label{ch:湖泊模式}
\echapter{Lake Model}
%\addcontentsline{toc}{chapter}{湖泊模式}
\begin{mymdframed}{代码}
  本章对应的代码文件为\texttt{MOD\_Lake.F90}。
\end{mymdframed}

湖泊作为陆地表面一种特定形式的水体,对区域尺度的水热平衡过程及气候变化具有显著影响,其中大面积的湖泊对于局地的天气和气候可起到决定性作用。
为模拟湖泊的热力和动力过程,湖泊模式在上世纪80年代开始发展~\citep{henderson1985new},
并逐渐形成针对不同湖泊的简易湖泊模式~\citep{hostetler1993interactive,hostetler1994lake,hostetler1990simulation}。
此后,湖泊模式作为陆面过程模式的重要组成模块,其应用逐渐发展到针对全球湖泊的普适性模拟(e.g. \citet{bonan1995sensitivity,bonan1996land})。
\citet{zeng2002coupling}对当时已有的湖泊模式进行了改进,
使得湖泊模式的模拟效果得到显著提高,成为通用陆面模式CoLM中湖泊模型的早期版本(CoLM-Lake)。
近几年来,随着湖泊模型的不断发展,CoLM-Lake逐渐纳入了新的针对湖泊方案的发展与改进,
并在2014年发布的新版CoLM中对湖泊模型进行了更新。本章将对新版CoLM-Lake的物理方案做一详细描述(图~\ref{fig:湖泊系统垂直分层示意图})。

{
  \begin{figure}[htbp]
    \centering
    \includegraphics{Figures/湖泊模式/湖泊系统垂直分层示意图.png}
    \caption[湖泊系统垂直分层示意图]{湖泊系统垂直分层示意图 \citep{subin2012improved}}
    \label{fig:湖泊系统垂直分层示意图}
  \end{figure}
}


\section{湖泊模式结构}
\esection{Lake Model Structure}
湖泊模式由三部分组成:雪层、湖泊层和淤泥层。湖泊层位于雪层和淤泥层之间,
共分为10层。湖泊深度可随降水量、蒸发量、与湖泊连接的河流的流入流出量等发生变化(见~\ref{湖泊水文}~节)。在初始时刻,默认湖深$d$为50 m,每层湖泊的厚度$\Delta z_{\mathrm{lake},i}$从上到下分别为0.1, 1, 2, 3, 4, 5, 7, 7, 10.45, 10.45 m,
中心深度$z_{\mathrm{lake},i}$为每层湖泊的中间位置所在深度,分别为0.05, 0.6, 2.1, 4.6, 8.1, 12.6, 18.6, 25.6, 34.325, 44.775 m。
湖泊深度$d$亦可由地表参数数据集提供。当$d\neq50$ m且$d\geqslant 1$ m时,第一层湖泊的厚度仍保持0.1 m,
其余层厚度按照上述分层厚度的比例进行分配。
当$d<1$ m时,10层湖泊进行等分。在模式中,每层湖泊的质量固定不变,由液态水密度与每层湖泊的厚度确定,
湖泊状态变量由混合层深度、湖冰厚度和每一层湖泊的温度$T_{\mathrm{lake}}$与冻结部分所占的质量比$I$来表征。雪层与淤泥层的分层方式与~\ref{温度求解的数值格式} 节介绍的计算雪盖淤泥温度时的分层方案相同
,物理方案遵从无植被覆盖下的土壤雪盖状态变量的计算方案来设定。


\section{降水与湖泊表面的相互作用}
\esection{Precipitation-Lake Interaction}
降水到达湖泊表面可发生能量转移与相态变化等一系列过程。此过程可改变湖泊表面条件,尤其当降雪存在时,对于湖泊之上雪层的形成起到决定性作用。


首先,在进行湖泊过程计算之前,对湖泊表面的积雪厚度$z_{\mathrm{sno}}$ (m)与雪水当量$W_{\mathrm{sno}}$ (mm或 \unit{kg.m^{-2}})进行更新:
\begin{equation}
  Z_{\mathrm{sno}}=Z_{\mathrm{sno}}+\frac{p_{\mathrm {snow}} \Delta t}{\rho_{\mathrm{sno,new}}}
\end{equation}
\begin{equation}
  W_{\mathrm{sno}}=W_{\mathrm{sno}}+p_{\mathrm {snow}} \Delta t
\end{equation}
其中$p_{\mathrm {snow}} $表示固态降水率(\unit{kg.m^{-2}.s^{-1}}),$\rho_{\mathrm{sno,new}}$表示新降的干雪密度(\unit{kg.m^{-3}})(见章节~\ref{温度求解的数值格式})。

接下来,计算湖泊表面与降水的能量转移过程,并且当降水与湖泊表面的相态不同时,相态变化过程同时考虑。
此过程将根据更新后的积雪厚度,分为湖泊表层
\begin{enumerate}
  \item 未达到第一层雪的产生条件;
  \item 已达到第一层雪产生条件但之前无雪层;
  \item 之前已存在雪层,三种情况进行计算。
\end{enumerate}

\noindent\textbf {(1)未达到第一层雪产生条件(即$z_{\mathrm{sno}}<0.01$且$snl=0$)}

此情形下,降水视为与第一层湖泊表面充分接触并立即发生能量转移,并且当降水与湖泊表层的相态不同时,相态变化过程同时发生。
此过程的演变由以下八个能量组份(\unit{J.m^{-2}})决定:
\begin{align*}
  a &= C_{\mathrm{liq}} p_{\mathrm {rain}} \Delta t\left(T_{\mathrm{p}}-T_{\mathrm {frz}}\right)   & \text{液态降水冷却到$T_{\mathrm {frz}} $释放的能量} \\
  b &= \lambda_{\mathrm {fus}}  p_{\mathrm {rain}}  \Delta t                                           & \text{液态降水凝结为固态释放的能量} \\
  c &= C_{\mathrm{ice}} \rho_{\mathrm{liq}} \Delta z_{\mathrm{lake, 1}} I_{1}\left(T_{\mathrm {frz}} -T_{\mathrm{lake, 1}}\right)  & \text{第一层湖泊固态下加热到$T_{\mathrm {frz}} $吸收的能量} \\
  d &= \lambda_{\mathrm {fus}}  \rho_{\mathrm{liq}} \Delta z_{\mathrm{lake, 1}} I_{1}         & \text{第一层湖泊固态下完全融化吸收的能量} \\
  e &= C_{\mathrm{ice}} p_{\mathrm {snow}}  \Delta t\left(T_{\mathrm {frz}}  -T_{\mathrm {p}} \right)        & \text{固态降水加热到$T_{\mathrm {frz}} $吸收的能量} \\
  f &= \lambda_{\mathrm {fus}}  p_{\mathrm {snow}}  \Delta t                                           & \text{固态降水融化为液态吸收的能量} \\
  g &= C_{\mathrm{liq}} \rho_{\mathrm{liq}} \Delta z_{\mathrm{lake, 1}}\left(1-I_{1}\right)\left(T_{\mathrm{lake, 1}}-T_{\mathrm {frz}} \right)  & \text{第一层湖泊液态下冷却到$T_{\mathrm {frz}} $释放的能量} \\
  h &= \lambda_{\mathrm {fus}}  \rho_{\mathrm{liq}} \Delta z_{\mathrm{lake, 1}}\left(1-I_{1}\right)  & \text{第一层湖泊液态下完全冻结释放的能量}
\end{align*}
其中,$p_{\mathrm {rain}} $、$p_{\mathrm {snow}} $分别为液态与固态降水率(\unit{kg.m^{-2}.s^{-1}}),$C_{\mathrm{liq}}$、$C_{\mathrm{ice}}$分别为液态水与固态水的比热容(\unit{J.kg^{-1}.K^{-1}}),
$T_{\mathrm {p}} $、$T_{\mathrm {frz}} $分别为雨水温度与液态水凝结温度(K),$\Delta t$表示时间积分步长(s),$\lambda_{\mathrm {fus}} $表示液态水凝结潜热(\unit{J.kg^{-1}}),
$\rho_{\mathrm{liq}}$表示液态水密度(\unit{kg.m^{-3}}),$T_{\mathrm{lake,1}}$,$\Delta z_{\mathrm{lake,1}}$,$I_1$分别为第一层湖泊的温度,
厚度与冻结部分所占的质量比。为表述清晰,此过程将按照第一层湖泊的不同相态分别考虑,计算如下:

A. 第一层湖泊完全冻结($I_1>0.999$)

此时,应有$T_{\mathrm{lake,1}}<T_{\mathrm {frz}} $。若固态降水发生($T_{\mathrm {p}} <T_{\mathrm {frz}} $),它将直接与湖泊表层达到一个平衡温度;
若液态降水发生($T_{\mathrm {p}} >T_{\mathrm {frz}} $),相态变化过程将会同时触发,具体如下:

若表层湖泊不需发生相态变化足以使得液态降水到达表面后凝结为固态,即$a\leqslant c-b$,则湖泊吸收的总能量为:
\begin{equation}
  C_{\mathrm{ice}} \rho_{\mathrm{liq}} \Delta z_{\mathrm{lake,1}} I_{1}\left(T_{\mathrm{lake, 1}}^{(n+1)}-T_{\text {lake,1 }}^{(n)}\right)=a+b+C_{\mathrm{ice}} p_{\mathrm {rain}} \Delta t\left(T_{\mathrm {frz}}-T_{\text {lake, }}^{(n+1)}\right)
\end{equation}
解出$T_{\mathrm{lake,1}}^{\left(n+1\right)}$即为此时更新的第一层湖泊温度。
同时后续计算中,液态降水可视为固态降水,雪水当量与积雪厚度随之更新:
\begin{equation}
  \begin{aligned}
    W_{\mathrm{sno}} &= W_{\mathrm{sno}}+p_{\mathrm {rain}}  \Delta t \\
    z_{\mathrm{sno}} &= z_{\mathrm{sno}}+\frac{p_{\mathrm {rain}}  \Delta t}{\rho_{\mathrm{sno,new}}} \\
    p_{\mathrm {snow}}  &= p_{\mathrm {snow}}  + p_{\mathrm {rain}}  \\
    p_{\mathrm {rain}}  &= 0.0
  \end{aligned}
\end{equation}


若表层湖泊将液态降水冷却到$T_{\mathrm {frz}} $后不足以冻结所有降水,即$c-b<a\leqslant c$,
则第一层湖泊温度更新为$T_{\mathrm{lake,1}}=T_{\mathrm {frz}} $。于是湖泊吸收的来自降水凝结的有效能量为$c-a$,
液态降水转为固态降水的量以及雪水当量和积雪厚度可更新为:
\begin{equation}
  \begin{aligned}
    W_{\mathrm{sno}} &= W_{\mathrm{sno}} + (c-a) / \lambda_{\mathrm {fus}}  \\
    z_{\mathrm{sno}} &= z_{\mathrm{sno}} + \frac{c-a}{\lambda_{\mathrm {fus}}  \rho_{\mathrm{sno,new}}} \\
    p_{\mathrm {snow}}  &= p_{\mathrm {snow}}  + \frac{c-a}{\lambda_{\mathrm {fus}}  \Delta t} \\
    p_{\mathrm {rain}}  &= p_{\mathrm {rain}}  - \frac{c-a}{\lambda_{\mathrm {fus}}  \Delta t}
  \end{aligned}
\end{equation}
若表层湖泊需要部分融化才可将液态降水冷却到$T_{\mathrm {frz}} $而降水无需发生相态变化,即$c<a\leqslant c+d$,
则第一层湖泊温度仍更新为$T_{\mathrm{lake,1}}=T_{\mathrm {frz}} $。于是降水冷却用于融化湖泊的有效能量为$a-c$,
此时湖泊冻结部分所占的质量比更新为:
\begin{equation}
  I_{1}=\frac{\rho_{\mathrm{liq}} \Delta z_{\mathrm{lake, 1}}-(a-c) / \lambda_{\mathrm {fus}}}{\rho_{\mathrm{liq}} \Delta z_{\mathrm{lake, 1}}}
\end{equation}
若表层湖泊可被液态降水完全融化而降水无需发生相态变化,即$a>c+d$,则降水释放的总能量为:
\begin{equation}
  C_{\mathrm{liq}} p_{\mathrm {rain}} \Delta t\left(T_{\mathrm{p}}-T_{\mathrm{lake, 1}}^{(n+1)}\right)=c+d+C_{\mathrm{liq}} \rho_{\mathrm{liq}} \Delta z_{\mathrm{lake, 1}}\left(T_{\mathrm{lake, 1}}^{(n+1)}-T_{\mathrm {frz}}\right)
\end{equation}
解出$T_{\mathrm{lake,1}}^{\left(n+1\right)}$即为此时更新的第一层湖泊温度,同时湖泊冻结部分质量比更新为$I_1=0.0$。


注意,经过以上情况讨论后,若发生液态降水冻结为固态降水,则雪水当量与积雪厚度将会增加。
此时若$z_{\mathrm{sno}}\geqslant 0.01$m,则第一层雪立即生成,且相应的状态变量设置为:
\begin{equation}
  \begin{aligned}
    snl &= -1 \\
    \Delta z_{0} &= z_{\mathrm{sno}} \\
    z_{0} &= -0.5 z_{\mathrm{sno}} \\
    z_{\mathrm{h,-1}} &= -\Delta z_{0} \\
    T_{0} &= T_{\mathrm{lake, 1}} \\
    w_{\mathrm{ice, 0}} &= W_{\mathrm{sno}} \\
    w_{\mathrm{liq, 0}} &= 0
  \end{aligned}
\end{equation}

B. 第一层湖泊为冰水混合态($0.001\leqslant I_1\leqslant 0.999$)

此时,应有$T_{\mathrm{lake,1}}=T_{\mathrm {frz}} $。若液态降水发生($T_{\mathrm {p}} >T_{\mathrm {frz}} $),
当液态降水可以将第一层湖泊中的冰全部融化时($a\geqslant d$),第一层湖泊的状态变量更新为:
\begin{equation}
  T_{\mathrm{lake, 1}}=\frac{C_{\mathrm{liq}} p_{\mathrm {rain}} \Delta t T_{\mathrm{p}}+C_{\mathrm{liq}} \rho_{\mathrm{liq}} \Delta z_{\mathrm{lake, 1}} T_{\mathrm {frz}}-d}{C_{\mathrm{liq}} \rho_{\mathrm{liq}} \Delta z_{\mathrm{lake, 1}}+C_{\mathrm{liq}} p_{\mathrm {rain}} \Delta t}
\end{equation}
\begin{equation}
  I_{1}=0.0
\end{equation}
否则,保持$T_{\mathrm{lake,1}}=T_{\mathrm {frz}} $,用于湖泊中冰融化的能量为$a$,则湖泊冻结部分的质量比更新为:
\begin{equation}
  I_{1}^{(n+1)}=\frac{\rho_{\mathrm{liq}} \Delta z_{\mathrm{lake, 1}} I_{1}^{(n)}-a / \lambda_{\mathrm {fus}}}{\rho_{\mathrm{liq}} \Delta z_{\mathrm{lake, 1}}}
\end{equation}
同样地,若固态降水发生($T_{\mathrm {p}} <T_{\mathrm {frz}} $),当固态降水可以将第一层湖泊中的水全部冻结时($e\geqslant h$),第一层湖泊的状态变量更新为:
\begin{equation}
  T_{\mathrm{lake, 1}}=\frac{C_{\mathrm{ice}} p_{\mathrm {snow}} \Delta t T_{\mathrm{p}}+C_{\mathrm{ice}} \rho_{\mathrm{liq}} \Delta z_{\mathrm{lake, 1}} T_{\mathrm {frz}}+h}{C_{\mathrm{ice}} \rho_{\mathrm{liq}} \Delta z_{\mathrm{lake, 1}}+C_{\mathrm{ice}} p_{\mathrm {snow}} \Delta t}
\end{equation}
\begin{equation}
  I_{1}=1.0
\end{equation}
否则,保持$T_{\mathrm{lake,1}}=T_{\mathrm {frz}} $,用于湖泊中水冻结的能量为$e$,则湖泊冻结部分的质量比更新为:
\begin{equation}
  I_{1}^{(n+1)}=\frac{\rho_{\mathrm{liq}} \Delta z_{\mathrm{lake, 1}} I_{1}^{(n)}+e / \lambda_{\mathrm {fus}}}{\rho_{\mathrm{liq}} \Delta z_{\mathrm{lake, 1}}}
\end{equation}

C. 	第一层湖泊完全融化($I_1<0.001$)

此时,应有$T_{\mathrm{lake,1}}>T_{\mathrm {frz}} $。若液态降水发生($T_{\mathrm {p}} >T_{\mathrm {frz}} $),它将直接与湖泊表层达到一个平衡温度;
若固态降水发生($T_{\mathrm {p}} <T_{\mathrm {frz}} $),相态变化过程将会同时触发,具体如下:

若表层湖泊不需发生相态变化足以使得固态降水到达表面后融化为液态,即$e\leqslant g-f$,则湖泊释放的总能量为:
\begin{equation}
  C_{\mathrm{liq}} \rho_{\mathrm{liq}} \Delta z_{\mathrm{lake, 1}}\left(1-I_{1}\right)\left(T_{\mathrm{lake, 1}}^{(n)}-T_{\mathrm{lake, 1}}^{(n+1)}\right)=
  e+f+C_{\mathrm{liq}} p_{\mathrm {snow}} \Delta t\left(T_{\mathrm{lake, 1}}^{(n+1)}-T_{\mathrm {frz}}\right)
\end{equation}
解出$T_{\mathrm{lake,1}}^{\left(n+1\right)}$即为此时更新的第一层湖泊温度。
同时后续计算中,固态降水可视为液态降水,雪水当量与积雪厚度随之更新:
\begin{equation}
  \begin{aligned}
    W_{\mathrm{sno}} &= W_{\mathrm{sno}}-p_{\mathrm {snow}} \Delta t \\
    z_{\mathrm{sno}} &= z_{\mathrm{sno}}-\frac{p_{\mathrm {snow}}  \Delta t}{\rho_{\mathrm{sno,new}}} \\
    p_{\mathrm {rain}}  &= p_{\mathrm {snow}}  + p_{\mathrm {rain}}  \\
    p_{\mathrm {snow}}  &= 0.0
  \end{aligned}
\end{equation}
若表层湖泊将固态降水加热到$T_{\mathrm {frz}} $后不足以融化所有降水,即$g-f<e\leqslant g$,
则第一层湖泊温度更新为$T_{\mathrm{lake,1}}=T_{\mathrm {frz}} $。于是湖泊释放的用于降水融化的有效能量为$g-e$,
固态降水转为液态降水的量以及雪水当量和积雪厚度可更新为:
\begin{equation}
  \begin{aligned}
    W_{\mathrm{sno}} &= W_{\mathrm{sno}}-(g-e) / \lambda_{\mathrm {fus}} \\
    z_{\mathrm{sno}} &= z_{\mathrm{sno}}-\frac{g-e}{\lambda_{\mathrm {fus}}  \rho_{\mathrm{sno,new}}} \\
    p_{\mathrm {rain}} &= p_{\mathrm {rain}}+\frac{g-e}{\lambda_{\mathrm {fus}}  \Delta t} \\
    p_{\mathrm {snow}} &= p_{\mathrm {snow}}-\frac{g-e}{\lambda_{\mathrm {fus}}  \Delta t}
  \end{aligned}
\end{equation}
若表层湖泊需要部分冻结才可将固态降水加热到$T_{\mathrm {frz}} $而降水无需发生相态变化,即$g<e\leqslant g+h$,
则第一层湖泊温度仍更新为$T_{\mathrm{lake,1}}=T_{\mathrm {frz}} $。于是湖泊冻结用于加热降水的有效能量为$e-g$,
此时湖泊冻结部分所占的质量比更新为:
\begin{equation}
  I_{1}=\frac{(e-g) / \lambda_{\mathrm {fus}}}{\rho_{\mathrm{liq}} \Delta z_{\mathrm{lake, 1}}}
\end{equation}

若表层湖泊可被固态降水完全冻结而降水无需发生相态变化,即$e>g+h$,则降水吸收的总能量为:
\begin{equation}
  C_{\mathrm{ice}} p_{\mathrm {snow}} \Delta t\left(T_{\mathrm{lake, 1}}^{(n+1)}-T_{\mathrm{p}}\right)=g+h+C_{\mathrm{ice}} \rho_{\mathrm{liq}} \Delta z_{\mathrm{lake, 1}}
  \left(T_{\mathrm {frz}}-T_{\mathrm{lake,1}}^{(n+1)}\right)
\end{equation}
解出$T_{\mathrm{lake,1}}^{\left(n+1\right)}$即为此时更新的第一层湖泊温度,同时湖泊冻结部分质量比更新为$I_1=1.0$。


\noindent\textbf {(2) 已达到第一层雪产生条件但之前无雪层(即$z_{\mathrm{sno}}\geqslant 0.01$且$snl=0$)}

此情形下,第一层雪立即生成,且相应的状态变量设置为:
\begin{equation}
  \begin{aligned}
    snl &= -1 \\
    \Delta z_{0} &= z_{\mathrm{sno}} \\
    z_{0} &= -0.5 z_{\mathrm{sno}} \\
    z_{\mathrm{h,-1}} &= -\Delta z_{0} \\
    T_{0} &= \min \left(T_{\mathrm{p}}, T_{\mathrm {frz}}\right) \\
    w_{\mathrm{ice, 0}} &= W_{\mathrm{sno}} \\
    w_{\mathrm{liq, 0}} &= 0
  \end{aligned}
\end{equation}


\noindent\textbf {(3) 之前已存在雪层(即$snl<0$)}

此情形下,降水与第一层雪将达到一个平衡温度,能量平衡关系为:
\begin{equation}
  \left(C_{\mathrm{liq}} p_{\mathrm {rain}}+C_{\mathrm{ice}} p_{\mathrm {snow}}\right) \Delta t\left(T_{snl+1}^{(n+1)}-T_{\mathrm{p}}\right)=
  \left(C_{\mathrm{liq}} w_{\mathrm{liq},snl+1}+C_{\mathrm{ice}} w_{\mathrm{ice},snl+1}\right)\left(T_{snl+1}^{(n)}-T_{snl+1}^{(n+1)}\right)
\end{equation}
解出$T_{snl+1}^{\left(n+1\right)}\leqslant T_{\mathrm {frz}} $即为此时更新的第一层雪的温度。
同时,类似雪盖土壤含水量的计算方案,固态降水先加到第一层雪的固态含水量中,
液态降水将在蒸发液化的能量过程计算之后加入到第一层雪的液态含水量:
\begin{equation}
  \begin{aligned}
    w_{\mathrm{ice},snl+1} &=  w_{\mathrm{ice},snl+1}+p_{\mathrm {snow}} \Delta t \\
    \Delta z_{snl+1} &= \Delta z_{snl+1}+\frac{p_{\mathrm {snow}} \Delta t}{\rho_{\mathrm{sno,new}}} \\
    z_{\mathrm{h},snl+1} &= z_{h,snl+1}-0.5 \Delta z_{snl+1} \\
    z_{\mathrm{h}, snl} &= z_{\mathrm{h},snl+1}-\Delta z_{snl+1}
  \end{aligned}
\end{equation}


\section{湖泊温度计算方案}
\esection{Lake Temperature Scheme}
湖泊温度的计算方案整体上与无植被覆盖下的雪盖土壤温度的计算方案类似。
首先,求解湖泊表面的能量通量,同时湖泊表面温度($T_{\mathrm {g}} $)作为湖泊顶层与大气底层的交界面温度被同时求出。
然后,雪层、湖泊层和淤泥层温度作为一个系统被同时求解,其上边界条件来自湖泊表面的地表热通量。具体方案描述如下。

\subsection{湖泊表面能量通量与温度的计算}\label{湖泊表面能量通量与温度的计算}
\esubsection{Surface Energy and Temperature}
湖泊表面的能量平衡关系可表达为:
\begin{equation}
  \beta S_{\mathrm{g}}+L_{\mathrm{g}}\left(T_{\mathrm{g}}\right)-H_{\mathrm{g}}\left(T_{\mathrm{g}}\right)-\lambda E_{\mathrm{g}}\left(T_{\mathrm{g}}\right)-G\left(T_{\mathrm{g}}\right)+H_{\mathrm{in}}-H_{\mathrm{out}}=0
\end{equation}
其中$S_{\mathrm {g}} $表示可被湖泊吸收的净太阳辐射(\unit{W.m^{-2}})(见章节~\ref{短波吸收辐射通量}),$\beta$表示被表面吸收的比例,
$L_{\mathrm {g}} $表示湖泊表面吸收的净长波辐射(\unit{W.m^{-2}}),$H_{\mathrm {g}} $,$E_{\mathrm {g}} $分别表示陆面向大气输送的感热通量(\unit{W.m^{-2}})
和水汽通量(\unit{kg.m^{-2}}),$G$表示输入到湖泊表面以下的地表热通量(\unit{W.m^{-2}}),$H_{\mathrm{in}}=C_wQ_{\mathrm{in}}(T_{\mathrm{river,up}}-T_{\mathrm {g}} )$和$H_{\mathrm{out}}=C_wQ_{\mathrm{out}}(T_{\mathrm {g}} -T_{\mathrm{river,down}})$分别表示湖泊上游河水入流和下游河水出流导致的能量变化(\unit{W.m^{-2}}),$Q_{\mathrm{in}}$和$Q_{\mathrm{out}}$分别表示湖泊如流流量和出流流量(\unit{m^{3}.s^{-1}})。
所有能量通量均表达为湖泊表面温度$T_{\mathrm {g}} $的函数。$\lambda$用于将水汽通量转换为潜热通量,取值为(\unit{J.kg^{-1}},见表~\ref{tab:物理常数})
\begin{equation}
  \lambda=\left\{\begin{array}{ll}\lambda_{\mathrm{sub}} & T_{\mathrm{g}} \leqslant T_{\mathrm {frz}} \\ \lambda_{\mathrm{vap}} & T_{\mathrm{g}}>T_{\mathrm {frz}}\end{array}\right.
\end{equation}
$\beta$的取值依赖于湖泊表面的状态。若湖泊表面存在雪层,则忽略短波辐射在雪层中的传递,
取$\beta=1$;若不存在雪层,则$\beta$取为短波辐射中近红外辐射所占的比例(见章节~\ref{短波吸收辐射通量}),
其余辐射的部分($1-\beta$)被湖泊水体及下面的淤泥吸收。


湖泊表面湍流通量的计算与无植被覆盖下雪盖土壤表层的湍流通量计算非常类似。感热通量表达为:
\begin{equation}
  H_{\mathrm{g}}=-\rho_{\mathrm{a}} C_{\mathrm{ a}} \frac{\left(\theta_{\mathrm{a}}-T_{\mathrm{g}}\right)}{r_{\mathrm{a h}}}
\end{equation}
其中$\rho_{\mathrm{a}}$表示空气密度(\unit{kg.m^{-3}})(计算见章节~\ref{基本理论}),$C_{\mathrm{a}}$表示空气的比热容(\unit{J.kg^{-1}.K^{-1}},见表~\ref{tab:物理常数}),
$\theta_{\mathrm{a}}$表示大气位温(K)(计算见章节~\ref{基本理论}),$r_{\mathrm{ah}}$表示感热通量的空气动力学阻抗(\unit{s.m^{-1}})(计算见章节~\ref{基本理论})。
水汽通量表达为:
\begin{equation}
  E_{\mathrm{g}}=-\rho_{\mathrm{a}} \frac{\left(q_{\mathrm{a}}-q_{\mathrm{g}}\right)}{r_{\mathrm{a w}}}
\end{equation}
其中$q_{\mathrm{a}}$表示空气比湿(\unit{kg.kg^{-1}}),$q_{\mathrm {g}} =q_{\mathrm{sat}}^{T_{\mathrm {g}} }$为温度在$T_{\mathrm {g}} $时的饱和比湿(\unit{kg.kg^{-1}})(计算见章节~\ref{饱和水汽压(比湿)及其随温度的变化}),
$r_{\mathrm{aw}}$表示水汽通量的空气动力学阻抗(\unit{s.m^{-1}})(计算见章节~\ref{基本理论})。
计算空气动力学阻抗系数需估算湖泊表面动力、热力和水汽粗糙度,他们依赖于湖泊表面的状态。
若湖泊冻结($T_{\mathrm {g}} \leqslant T_{\mathrm {frz}} $)且湖泊之上存在雪层,则动力粗糙度$z_{\mathrm{0m}}=0.0024$ m,
热力与水汽粗糙度$z_{\mathrm{0h}}=z_{\mathrm{0w}}=z_{\mathrm{0m}}\exp{\left(-0.13R_0^{0.45}\right)}$~\citep{zilitinkevich1972dynamics},
其中$R_0$表示近粗糙雷诺数$R_0=\frac{z_{\mathrm{0m}}u_\ast}{\upsilon}$,
$\upsilon=1.5\times{10}^{-5}$ \unit{m^2.s^{-1}} 表示空气动力学粘性系数。
若湖泊冻结但湖泊之上不存在雪层,则动力粗糙度$z_{\mathrm{0m}}=0.001$ m~\citep{subin2012improved},
热力与水汽粗糙度同上。若湖泊为非冻结状态($T_{\mathrm {g}} >T_{\mathrm {frz}} $),则动力粗糙度为~\citep{subin2012improved}:
\begin{equation}
  z_{0 m}=\max \left(\frac{0.1 v}{u_{*}},\ C \frac{u_{*}^{2}}{g}\right) \geqslant 10^{-5}
\end{equation}
其中$g$表示重力加速度,$\upsilon$表示如下方式给出的空气动力学粘性系数(\unit{m^2.s^{-1}}):
\begin{equation}
  v=v_{0}\left(\frac{T_{\mathrm{g}}}{T_{0}}\right)^{1.5} \frac{P_{0}}{P_{\mathrm{r e f}}}
\end{equation}
其中$T_0=293.15$ K,$P_0=1.013\times{10}^5$ Pa,
$\upsilon_0$表示此温度压强下的空气动力粘性系数$\upsilon_0=1.51\times{10}^{-5}$ \unit{m^2.s^{-1}},
$P_{\mathrm{ref}}$表示大气参考高度气压。$z_{\mathrm{0m}}$的计算中$C$表示有效Charnock系数,按如下方式表达:
\begin{equation}
  C=C_{\mathrm{min}}+\left(C_{\mathrm{max}}-C_{\mathrm{min}}\right) \exp\left[\max (A, B)\right]
\end{equation}
其中$C_{\mathrm{max}}=0.11$,$C_{\mathrm{min}}=0.01$分别表示最大与最小Charnock系数,
$A$、$B$分别表示来自湖泊风浪区长度$F$与湖泊深度$d$的限制:
$A=-\left(\frac{Fg}{u^2}\right)^{1/3}/f_{\mathrm {c}} $,$B=-\frac{\sqrt{dg}}{u}$,
其中$u$表示大气参考高度风速(\unit{m.s^{-1}}),$f_{\mathrm {c}} =22$,$F(m)$依赖于湖泊深度,
假定浅湖具有较小的湖泊风浪区,
计算公式为:$$F=\left\{\begin{array}{ll}100.0 & d<4.0 \\ 25.0 * d & d \geqslant 4.0\end{array}\right.$$


根据~\citet{Zilitinkevich2001},非冻结湖泊的热力和水汽粗糙度表达为:
\begin{equation}
  z_{0 h}=z_{0 m} \exp \left[-\frac{\kappa}{P_{\mathrm{r}}}\left(4 \sqrt{R_{0}}-3.2\right)\right] \geqslant 10^{-5}
\end{equation}
\begin{equation}
  z_{0 w}=z_{0 m} \exp \left[-\frac{\kappa}{S_{\mathrm{c}}}\left(4 \sqrt{R_{0}}-4.2\right)\right] \geqslant 10^{-5}
\end{equation}
其中$\kappa$表示von K\'arman常数,$P_{\mathrm {r}} =0.713$表示空气分子的 Prandtl 数,
$S_{\mathrm {c}} =0.66$表示空气中水分子的 Schmidt 数,
$R_0=\frac{z_{\mathrm{0m}}u_\ast}{\upsilon}\geqslant0.1$表示粗糙雷诺数,$\upsilon$的计算方式同上。


输入到湖泊表面以下的地表热通量$G$ (\unit{W.m^{-2}})表达为
\begin{equation}
  G=\frac{2 \lambda_{snl+1}}{\Delta z_{snl+1}}\left(T_{\mathrm{g}}-T_{snl+1}\right)
\end{equation}
其中$T_{snl+1}$,$\Delta z_{snl+1}$分别表示湖泊顶层(雪层或湖泊层)的温度(K)与厚度(m),
$\lambda_{snl+1}$表示湖泊顶层的热力传导率(\unit{W.m^{-1}.K^{-1}})。
若顶层为雪层,$\lambda_{snl+1}$需考虑雪层的固态和液态含水量,计算见章节~\ref{温度求解的数值格式};
若顶层为湖泊层且湖泊层已冻结($T_{\mathrm {g}} \leqslant T_{\mathrm {frz}} $),则$\lambda_{snl+1}=k_{\mathrm {ice}}$(见表~\ref{tab:物理常数});
若顶层为湖泊层且湖泊层未冻结($T_{\mathrm {g}} >T_{\mathrm {frz}} $),则$\lambda_{snl+1}$需同时考虑分子与大涡扩散率,计算见章节~\ref{雪层湖泊层淤泥层温度的计算}。


湖泊表面吸收的净长波辐射$L_{\mathrm {g}} $ (\unit{W.m^{-2}})可达表为:
\begin{equation}
  L_{\mathrm{g}}=L ^\downarrow-L_{\mathrm{g}} ^\uparrow
\end{equation}
其中$L^\downarrow$表示近地面大气下行长波辐射,$L_{\mathrm {g}} ^\uparrow$表示湖泊表层发射的上行长波辐射:
\begin{equation}
  L_{\mathrm{g}} ^\uparrow=\left(1-\varepsilon_{\mathrm{g}}\right) L ^\downarrow+\varepsilon_{\mathrm{g}}
  \sigma\left(T_{\mathrm{g}}^{(n)}\right)^{4}+4 \varepsilon_{\mathrm{g}}
  \sigma\left(T_{\mathrm{g}}^{(n)}\right)^{3}\left(T_{\mathrm{g}}^{(n+1)}-T_{\mathrm{g}}^{(n)}\right)
\end{equation}
其中$\varepsilon_{\mathrm {g}} =0.97$表示湖泊表面的长波辐射发射率,$\sigma$表示 Stefan-Boltzmann 常数(见表~\ref{tab:物理常数}),
$T_{\mathrm {g}} ^{\left(n+1\right)}-T_{\mathrm {g}} ^{\left(n\right)}$表示$T_{\mathrm {g}} $在使用牛顿迭代法求解时相邻两次迭代结果的差别,
过程见下面叙述。


基于如上描述的各个能量组分的表达,湖泊表面的温度$T_{\mathrm {g}} $与能量通量可通过牛顿迭代法求解能量平衡方程得到:
\begin{equation}
  \Delta T_{\mathrm{g}}=\frac{\beta S_{\mathrm{g}}+L_{\mathrm{g}}\left(T_{\mathrm{g}}\right)-H_{\mathrm{g}}\left(T_{\mathrm{g}}\right)
  -\lambda E_{\mathrm{g}}\left(T_{\mathrm{g}}\right)-G\left(T_{\mathrm{g}}\right)+H_{\mathrm{in}}-H_{\mathrm{out}}}{-\frac{\partial L_{\mathrm{g}}}{\partial T_{\mathrm{g}}}
  +\frac{\partial H g}{\partial T_{\mathrm{g}}}+\frac{\partial \lambda E_{\mathrm{g}}}{\partial T_{\mathrm{g}}}+\frac{\partial G}{\partial T_{\mathrm{g}}}+\frac{\partial H_{\mathrm{in}}}{\partial T_{\mathrm{g}}}-\frac{\partial H_{\mathrm{out}}}{\partial T_{\mathrm{g}}}}
\end{equation}
其中$\Delta T_{\mathrm {g}} =T_{\mathrm {g}} ^{\left(n+1\right)}-T_{\mathrm {g}} ^{\left(n\right)}$。在迭代求解$T_{\mathrm {g}} $的过程中,除短波辐射外的其他能量通量均可同时求解。
此外,根据之前描述,湖泊表面粗糙度依赖于摩擦速度$u_\ast$,所以粗糙度的计算也同时结合$T_{\mathrm {g}} $的求解过程进行迭代计算。
能量平衡方程的求解过程大致如下(参考章节~\ref{植被地表的雨水感热}):

\begin{enumerate}
  \item 给出计算风速$V_{\mathrm {a}} $时$U_{\mathrm {c}} $的初始猜测:
    \begin{equation}
      \begin{array}{ll}
        U_{\mathrm{c}}=0 & \theta_{\mathrm{v, atm}}-\theta_{\mathrm{v, s}} \geqslant 0 \text{ (即稳定条件下)} \\
        U_{\mathrm{c}}=0.5 & \theta_{\mathrm{v, atm}}-\theta_{\mathrm{v, s}}<0 \text{ (即不稳定条件下)}
      \end{array}
    \end{equation}
  \item 给出湖泊表面粗糙度$z_{\mathrm{0m}}$,$z_{\mathrm{0h}}$,$z_{\mathrm{0w}}$的初始猜测:\\
    具体方法为:令$u_\ast=0.06$,通过迭代下式5次计算得到的$u_\ast$并结合前述方法来计算$z_{\mathrm{0m}}$,$z_{\mathrm{0h}}$,$z_{\mathrm{0w}}$:
    \begin{equation}
      Z_{0 m}=\frac{0.013 u_{*}^{2}}{g}+\frac{0.11 v}{u_{*}}
    \end{equation}
    \begin{equation}
      u_{*}=\kappa V_{\mathrm{a}} / \ln \frac{z_{\mathrm{a, m}}-d}{z_{0 m}}
    \end{equation}
    其中$\kappa$表示 von K\'arman常数,$z_{\mathrm{a,m}}$表示风速观测高度(m),$d=0$表示湖泊表面零平面位移,
    $\upsilon$表示通过下式计算的空气动力学粘性系数(\unit{m^2.s^{-1}})~\citep{andreas1989thermal}:
    \begin{equation}
      \begin{array}{cl}
        \upsilon=&1.326\times{10}^{-5}\left[1+6.542\times{10}^{-3}\left(T_{\mathrm{a}}-T_{\mathrm {frz}} \right)\right.\\
        & \left. +8.301\times{10}^{-6}\left(T_{\mathrm{a}}-T_{\mathrm {frz}} \right)^2-4.84\times{10}^{-9}\left(T_{\mathrm{a}}-T_{\mathrm {frz}} \right)^3\right]
      \end{array}
    \end{equation}
  \item 通过$R_{\mathrm{ib}}$给出$L$的初始猜测。
  \item 迭代以下过程以求得$T_{\mathrm {g}} $以及湖泊表面的能量通量:\\
    a. 根据湖泊表面条件(积雪、冻结或未冻结)判断湖泊表面升华/蒸发潜热系数$\lambda$,并计算湖泊顶层的热力传导率$\lambda_{snl+1}$ \\
    b. 通过风速、温度、比湿的微分方程积分结果求得$u_\ast$,$\theta_\ast$,$q_\ast$ \\
    c. 计算湖泊表面与大气之间的阻抗系数$r_{\mathrm{am}}$,$r_{\mathrm{ah}}$,$r_{\mathrm{aw}}$ \\
    d. 通过能量平衡方程计算温度变化$\Delta T_{\mathrm {g}} $,并由此更新$T_{\mathrm {g}} ^{\left(n+1\right)}=\Delta T_{\mathrm {g}} ^{\left(n\right)}+T_{\mathrm {g}} ^{\left(n\right)}$ \\
    e. 根据$T_{\mathrm {g}} ^{\left(n+1\right)}$更新感热通量$H_{\mathrm {g}} $与水汽通量$E_{\mathrm {g}} $ \\
    f. 更新饱和比湿$q_{\mathrm{sat}}^{T_{\mathrm {g}} ^{\left(n+1\right)}}$及其对$T_{\mathrm {g}} $的变化率 \\
    g. 更新特征位温$\theta_\ast$和特征比湿$q_\ast$ \\
    h. 更新特征虚位温$\theta_{\mathrm{v\ast}}$ \\
    i. 更新大气风速$V_{\mathrm {a}} \left(U_{\mathrm {c}} \right)$ \\
    j. 计算新一步$L$,并计算$\zeta$,根据稳定性条件限制$\zeta$的取值范围 \\
    k. 根据限制条件后的$\zeta$重新计算$L=\frac{z_{\mathrm{a,m}}-d}{\zeta}$ \\
    l. 由前述方法更新湖泊表面粗糙度$z_{\mathrm{0m}}$,$z_{\mathrm{0h}}$,$z_{\mathrm{0w}}$\\
    m. 判断迭代停止条件:若迭代过程中,$\Delta T_{\mathrm {g}} \leqslant 0.01$ K已出现4次,或迭代次数已超过40次,则迭代停止。
  \item 迭代求解$T_{\mathrm {g}} $结束后,为保证湖泊温度的一致性,$T_{\mathrm {g}} $需根据湖泊顶层条件进行如下调整:\\
    当$T_{\mathrm {g}} >T_{\mathrm {frz}} $时,若存在雪层($snl<0$)或$T_{\mathrm{lake,1}}\leqslant T_{\mathrm {frz}} $,则$T_{\mathrm {g}} =T_{\mathrm {frz}} $ \\
    当$T_{\mathrm {m}} <T_{\mathrm {g}} <T_{\mathrm{lake,1}}$时,$T_{\mathrm {g}} =T_{\mathrm{lake,1}}$ \\
    当$T_{\mathrm {frz}} <T_{\mathrm{lake,1}}<T_{\mathrm {g}} <T_{\mathrm {m}} $时,$T_{\mathrm {g}} =T_{\mathrm{lake,1}}$ \\
    第一个条件表示当湖泊表面存在雪或冰时,$T_{\mathrm {g}} $不得高于凝结点$T_{\mathrm {frz}} $;
    后两个条件表示湖泊应遵循垂直稳定性,即湖泊上层密度应不大于下层密度,$T_{\mathrm {m}} =T_{\mathrm {frz}} +4$表示液态水密度最大时的温度。
  \item 根据最终求得的$T_{\mathrm {g}} $更新湖泊表面升华/蒸发的潜热系数$\lambda$,上行长波辐射$L_{\mathrm {g}} ^\uparrow$,
    感热通量$H_{\mathrm {g}} $以及潜热通量$\lambda E_{\mathrm {g}} $。
  \item 地表热通量$G$最终更新为能量平衡方程的能量残余,作为雪层、湖泊层、淤泥层温度计算的上边界条件:
    \begin{equation}
      G=\beta S_{\mathrm{g}}+L_{\mathrm{g}}\left(T_{\mathrm{g}}\right)-H_{\mathrm{g}}\left(T_{\mathrm{g}}\right)-\lambda E_{\mathrm{g}}\left(T_{\mathrm{g}}\right)
    \end{equation}
  \item 计算动量通量为
    \begin{equation}
      \begin{array}{c}\tau_{\mathrm{x}}=-\rho_{\mathrm{a}} \frac{u_{\mathrm{a}}}{r_{\mathrm{am}}} \\ \tau_{\mathrm{y}}=-\rho_{\mathrm{a}} \frac{v_{\mathrm{a}}}{r_{\mathrm{am}}}\end{array}
    \end{equation}
  \item 计算2 m温度与比湿$T_{2m}$,$q_{2m}$
  \item 根据湖泊表面条件更新湖泊与大气水汽通量的形式如下:\\
    当$E_{\mathrm {g}} \geqslant 0.0$时,若湖泊表面存在雪层,则蒸发通量$E_{\mathrm{g,eva}}=\min \left(\frac{w_{\mathrm{liq},snl+1}}{\Delta t}, E_{\mathrm{g}}\right)$,
    升华通量$E_{\mathrm{g,sub}}=E_{\mathrm{g}}-E_{\mathrm{g,eva}}$;\\
    否则蒸发通量$E_{\mathrm{g,eva}}=\min \left(\frac{\left(1-I_{1}\right) \rho_{\mathrm{liq}} \Delta z_{\mathrm{lake, 1}}}{\Delta t}, E_{\mathrm{g}}\right)$,
    升华通量$E_{\mathrm{g,sub}}=E_{\mathrm{g}}-E_{\mathrm{g,eva}}$。\\
    当$E_{\mathrm {g}} <0.0$时,若湖泊表面存在雪层或湖泊首层冻结,则水汽通量全部视为结霜通量;否则水汽通量全部视为露水通量。

\end{enumerate}

\section{雪层、湖泊层、淤泥层温度的计算}\label{雪层湖泊层淤泥层温度的计算}
\esection{Snow, Lake, and Sediment Temperature}
在湖泊系统中,雪层、湖泊层、淤泥层被作为一个整体同时求解温度,
其控制方程与求解雪盖土壤温度的控制方程非常类似,表达如下:
\begin{equation}
  C \frac{\partial T}{\partial t}=\frac{\partial}{\partial z}\left(\tau \frac{\partial T}{\partial z}\right)-\frac{{\mathrm d} \phi}{{\mathrm {d}} z}
\end{equation}
其中$c$表示雪盖、湖泊或淤泥的体积热容量(\unit{J.m^{-3}.K^{-1}}),$t$表示时间(s),$\tau$表示热力传导率(\unit{W.m^{-1}.K^{-1}}),
$\phi$表示穿透深度$z$所吸收的太阳辐射(\unit{W.m^{-2}})。该系统共分为$N=n_{\mathrm{sno}}+N_{\mathrm{lake}}+N_{\mathrm{soi}}$层,
其中$n_{\mathrm{sno}}$表示此时刻存在的雪层数,$N_{\mathrm{lake}}$、$N_{\mathrm{soi}}$分别表示湖泊与淤泥的层数。
当湖泊发生相态变化等使得湖泊的固液态水比例发生变化的过程时,体积热容量$c$与热力传导率$\tau$会进行相应的调整。
$\tau$,$c$,$\phi$的计算方案见下。


首先,对于热力传导率$\tau$,雪层与淤泥层的计算与章节~\ref{温度求解的数值格式} 计算雪盖土壤的热力传导率完全相同,
只是当大部分淤泥水冻结使得淤泥的体积含水量超过孔隙度$\theta_{\mathrm{sat},i}$,
即淤泥相对于其饱和状态的潮湿程度$S_{\mathrm{r},i}$大于1时:
\begin{equation}
  S_{\mathrm{r},i}=\left(\frac{w_{\mathrm{liq},i}}{\rho_{\mathrm{liq}} \Delta z_{i}}+\frac{w_{\mathrm{ice},i}}{\rho_{\mathrm{ice}} \Delta z_{i}}\right)
  \frac{1}{\theta_{\mathrm{sat},i}}>1
\end{equation}
热力传导率作如下调整:
\begin{equation}
  \tau_{i}=\frac{\tau_{i}+X k_{\mathrm {ice}}}{(1+X)^{2}}
\end{equation}
其中$k_{\mathrm {ice}}$表示固态水热力传导率(见表~\ref{tab:物理常数}),$X=\frac{S_{\mathrm{r},i}-1}{\theta_{\mathrm{sat},i}}$。
此外,雪层(若存在)与第一层湖泊交界面的热力传导率取为最下层雪层的热力传导率。

下面考虑湖泊层本身热力传导率的计算。令第$i$ $\left(1\leqslant i<N_{\mathrm{lake}}\right)$层湖泊中非冻结部分的热力传导率表达为:
\begin{equation}
  \tau_{\mathrm{lake,liq},i}=k_{\mathrm{w}} C_{\mathrm{liq}} \rho_{\mathrm{liq}}
\end{equation}
其中$C_{\mathrm{liq}}$,$\rho_{\mathrm{liq}}$分别表示液态水的比热容(\unit{J.kg^{-1}.K^{-1}})与密度(\unit{kg.m^{-3}})(见表~\ref{tab:物理常数}),
$k_{\mathrm {w}} $表示热力扩散率(\unit{m^2.s^{-1}})。$k_{\mathrm {w}} $可由三部分组成~\citep{subin2012improved}:
\begin{equation}
  k_{\mathrm{w}}=m_{\mathrm{d}}\left(k_{\mathrm{e}}+k_{\mathrm{ed}}+k_{\mathrm{m}}\right)
\end{equation}
其中$k_{\mathrm {e}} $表示由风驱动的大涡扩散率~\citep{hostetler1990simulation},
$k_{\mathrm{ed}}$表示一些无法表达的混合过程所产生的增强扩散率~\citep{fang1996long},
$k_{\mathrm {m}} =\frac{k_{\mathrm {liq}}}{C_{\mathrm{liq}}\rho_{\mathrm{liq}}}$表示液态水分子扩散率。
对于大湖,三维混合过程可使得热力扩散率再次增强,$m_{\mathrm {d}} $即表示依赖于湖泊深度$d$的增强因子:
\begin{equation}
  m_{\mathrm{d}}=\left\{\begin{array}{ll}1 & d<25 \text{ m} \\ 5 & d \geqslant 25 \text{ m}\end{array}\right.
\end{equation}
由风驱动的大涡扩散率$k_{\mathrm {e}} $只存在于完全非冻结的湖泊层中,计算方式为
\begin{equation}
  k_{\mathrm{e},i}=\begin{cases}
    \frac{\kappa w^\ast z_{\mathrm{lake},i}}{P_{0}\left(1+37 R_{i}^{2}\right)}
    \exp \left(-\kappa^{*} z_{\mathrm{lake},i}\right) & T_{\mathrm{g}}>T_{\mathrm {frz}} \\
    0 & T_{\mathrm{g}} \leqslant T_{\mathrm {frz}}
  \end{cases}
\end{equation}
其中$\kappa$表示 von K\'arman 常数(见表~\ref{tab:物理常数}),$P_0=1$表示中性条件下的湍流${\mathrm {Prandtl}}$数,
$z_{\mathrm{lake},i}$表示第$i$层湖泊的中心深度,$w^\ast$表示表层摩擦速度,$w^\ast=0.0012u_2$,
$\kappa^\ast$随着纬度$\phi$的变化而变化,$\kappa^\ast=6.6u_2^{-1.84}\sqrt{\left|sin\phi\right|}$。
根据 \citet{hostetler1990simulation},计算$w^\ast$与$\kappa^\ast$时,
使用2 m风速$u_2$比使用10 m风速模拟的效果要好,故这里使用2 m风速$u_2$进行计算$u_2=\frac{u_\ast}{\kappa}\ln{\left(\frac{2}{z_{\mathrm{0m}}}\right)}\geqslant 0.1$。
$R_{i} $表示理查德森数,计算公式为
\begin{equation}
  R_{i}=\frac{-1+\sqrt{\left.1+\frac{40 N^{2} \kappa^{2} z_{\mathrm{lake},i}^{2}}{w^{\ast 2} \exp \left(-2 \kappa^{\ast} z_{\mathrm{lake},i}\right)}\right.}}{20}
\end{equation}
其中$N^2=\frac{g}{\rho_i}\frac{\rho_{i+1}-\rho_i}{z_{i+1}-z_i}\geqslant 7.5\times{10}^{-5}$,
$g$表示重力加速度,$\rho_i$表示第$i$层湖泊的密度~\citep{hostetler1990simulation}:
\begin{equation}\label{rho_i}
  \rho_{i}=\left(1-I_{i}\right) \rho_{\mathrm{liq}}\left(1-1.9549 \times 10^{-5}\left|T_{\mathrm{lake},i}-277\right|^{1.68}\right)+I_{i} \rho_{\mathrm{ice}}
\end{equation}
其中$I_i$表示湖泊冻结部分的质量比。增强扩散率$k_{\mathrm{ed}}$的计算公式为~\citep{fang1996long}:
\begin{equation}
  k_{\mathrm{e d}}=1.039 \times 10^{-8}\left(N^{2}\right)^{-0.43}
\end{equation}
综上,非冻结部分的热力传导率$\tau_{\mathrm{lake,liq},i}$即由上述三部分组成
$\tau_{\mathrm{lake,liq},i}=K_wC_{\mathrm{liq}}\rho_{\mathrm{liq}}=m_{\mathrm {d}} \left(k_{\mathrm {e}} +k_{\mathrm{ed}}+k_{\mathrm {m}} \right)C_{\mathrm{liq}}\rho_{\mathrm{liq}}$。
对于湖泊层的冻结部分,热力传导率可表达为:
\begin{equation}
  \tau_{\mathrm{lake,ice},i}=k_{\mathrm {ice}} \frac{\rho_{\mathrm{ice}}}{\rho_{\mathrm{liq}}}
\end{equation}
其中$k_{\mathrm {ice}}$表示固态水的热力传导率(见表 \ref{tab:物理常数})。
这里对固态水的热力传导率按照固液态水的密度比例进行调整,是因为湖泊厚度与质量假定不变,
每一层湖泊按照完全非冻结状态时的厚度进行设定。于是,第$i$ $\left(1\leqslant i<N_{\mathrm{lake}}\right)$层湖泊的热力传导率$\tau_{\mathrm{lake},i}$可按照冻结部分质量比
$I_i$表达为$\tau_{\mathrm{lake,liq},i}$与$\tau_{\mathrm{lake,ice},i}$的调和平均数:
\begin{equation}
  \tau_{\mathrm{lake},i}=\frac{\tau_{\mathrm{lake,liq},i} \tau_{\mathrm{lake,ice},i}}{\tau_{\mathrm{lake,liq},i} I_{i}+\tau_{\mathrm{lake,liq},i}\left(1-I_{i}\right)}
\end{equation}
最底层湖泊的热力传导率取为上一层湖泊的热力传导率$\tau_{\mathrm{lake},N_{\mathrm{lake}}}=\tau_{\mathrm{lake},N_{\mathrm{lake}}-1}$。


对于体积热容量的计算,雪层淤泥层的体积热容量与计算雪盖土壤温度时体积热容量的计算完全相同(见 \ref{温度求解的数值格式} 节)。
而湖泊层的体积热容量$c_{\mathrm{lake},i}$ (\unit{J.m^{-3}.K^{-1}})即为液态水与固态水体积热容量基于$I_i$的加权平均:
\begin{equation}
  c_{\mathrm{lake},i}=\rho_{\mathrm{liq}}\left[C_{\mathrm{liq}}\left(1-I_{i}\right)+C_{\mathrm{ice}} I_{i}\right]
\end{equation}
湖泊系统对于太阳辐射的吸收可发生于湖泊表面、雪层、湖泊层以及淤泥顶层。湖泊表面吸收太阳辐射的量为$\beta S_{\mathrm {g}} $ (见章节~\ref{湖泊表面能量通量与温度的计算}),
剩余部分$\left(1-\beta\right)S_{\mathrm {g}}$将被湖泊表面之下的每一层吸收,吸收量$\phi_i$依赖于湖泊表面条件。
若湖泊表面存在雪层,则忽略太阳辐射在雪层中的传递,假设辐射量完全被雪层表面吸收,即$\beta=1$,$\phi_i=0$;
若湖泊表面不存在雪层但湖泊处于冻结状态,则假设冰对于太阳辐射不透明,表面剩余太阳辐射完全被第一层湖泊吸收,
即$\phi_{\mathrm{lake,1}}=\left(1-\beta\right)S_{\mathrm {g}} $;若湖泊处于非冻结状态,
假设湖泊表面吸收的太阳辐射的作用深度为$z_{\mathrm {a}} $ (m),取值根据湖泊深度$d$为
\begin{equation}
  z_{\mathrm{a}}=\left\{\begin{array}{ll}0.5 & d<4 \text{m} \\ 0.6 & d \geqslant 4 \text{m}\end{array}\right.
\end{equation}
则$z_{\mathrm {a}} $之下深度为z的太阳辐射剩余量为
\begin{equation}
  \phi=(1-\beta) S_{\mathrm{g}} \exp \left[-\eta\left(z-z_{\mathrm{a}}\right)\right]
\end{equation}
其中$\eta(m-1)$表示消光系数$\eta=1.1925d^{-0.424}$ \citep{subin2012improved},
那么对于$z_{\mathrm {a}} $之下的每一层湖泊,太阳辐射吸收量为:
\begin{equation}
  \begin{split}
    \phi_{\mathrm{lake},i}= & (1-\beta) S_{\mathrm{g}} \exp \left[-\eta\left(z_{\mathrm{lake},i}-\frac{\Delta z_{\mathrm{lake},i}}{2}-z_{\mathrm{a}}\right)\right] \\
    & -\exp \left[-\eta\left(z_{\mathrm{lake},i}+\frac{\Delta z_{\mathrm{lake},i}}{2}-z_{\mathrm{a}}\right)\right]
  \end{split}
\end{equation}
湖泊底层的出射太阳辐射即为淤泥顶层吸收的太阳辐射:
\begin{equation}
  \phi_{1}=(1-\beta) S_{\mathrm{g}} \exp \left[-\eta\left(d-z_{\mathrm{a}}\right)\right]
\end{equation}
湖泊系统求解能量平衡方程的方法与求解雪盖土壤温度用的方法完全相同(见 \ref{温度求解的数值格式} 节),
即Crank--Nicholson半隐式格式求解,只是层数扩充到$N=n_{\mathrm{sno}}+N_{\mathrm{lake}}+N_{\mathrm{soi}}$层。
热传导通量$F_i$的计算采用相邻两层介质交界面处的热力传导率$\tau\left[z_{\mathrm{h},i}\right]$,
而此量的计算与计算雪盖土壤温度时交界面热力传导率的计算完全相同(见~\ref{温度求解的数值格式} 节):
\begin{equation}
  \tau\left[z_{\mathrm{h},i}\right]=\frac{\tau_i\tau_{i+1}\left(z_i-z_{i+1}\right)}{\tau_i\left(z_{\mathrm{h},i}-z_{i+1}\right)
  +\tau_{i+1}\left(z_i-z_{\mathrm{h},i}\right)}
\end{equation}
雪层与湖泊层交界面的热力传导率和湖泊层与淤泥层交界面的热力传导率亦是如此。第$i$层介质离散化的能量平衡方程为:
\begin{equation}
  \frac{c_{i} \Delta z_{i}}{\Delta t}\left(T_{i}^{n+1}-T_{i}^{n}\right)=F_{i-1}-F_{i}+\phi_{i}
\end{equation}
将此方程在整个湖泊系统中进行联立,仍得到三对角形式方程组如下:
\begin{equation}
  r_{i}=a_{i} T_{i-1}^{n+1}+b_{i} T_{i}^{n+1}+c_{i} T_{i+1}^{n+1}
\end{equation}
\begin{equation}
  \begin{aligned}
    a_{i} &=-0.5 \frac{\Delta t}{c_{i} \Delta z_{i}}\frac{\partial F_{i-1}}{\partial T_{i-1}^{n}} \\
    b_{i} &=1+0.5 \frac{\Delta t}{c_{i} \Delta z_{i}}\left[\frac{\partial F_{i-1}}{\partial T_{i-1}^{n}}+\frac{\partial F_{i}}{\partial T_{i}^{n}}\right] \\
    c_{i} &=-0.5 \frac{\Delta t}{c_{i} \Delta z_{i}} \frac{\partial F_{i}}{\partial T_{i}^{n}} \\
    r_{i} &=T_{i}^{n}+0.5 \frac{\Delta t}{c_{i} \Delta z_{i}}\left(F_{i-1}-F_{i}\right)+\frac{\Delta t}{c_{i}\Delta z_{i}} \phi_{i}
  \end{aligned}
\end{equation}
$F_i$的定义如下:对于顶层,$F_{i-1}=G$,$a_i=0$,$G$表示湖泊顶层的地表热通量;对于底层,$F_i=0$;
对于其他层,$F_i$的定义为:
\begin{equation}
  F_{i}=\tau\left[z_{\mathrm{h},i}\right] \frac{T_{i}^{n}-T_{i+1}^{n}}{z_{i+1}-z_{i}}
\end{equation}
用追赶法解此三对角形式方程组即可快速同时求得湖泊系统中每一层雪盖、湖泊、淤泥的温度$T_i^{n+1}$。
\section{湖水相态变化}\label{湖水相态变化}
\esection{Lake Phase Change}
与雪盖土壤温度的计算类似,湖泊系统温度计算之后,需根据每一层介质的温度与固液态水含量进行相态变化调整。
当某一层介质的温度高于凝结点$T_{\mathrm {frz}} $而固态水存在,则融化过程发生;当温度低于凝结点$T_{\mathrm {frz}} $而液态水存在,则冻结过程发生。

若融化条件满足,则该层介质可用于融化过程的能量表达为(\unit{J.m^{-2}}):
\begin{equation}
  Q_{\mathrm{avail}}=\left(T_{i}^{n+1}-T_{\mathrm {frz}}\right) c_{i} \Delta z_{i}
\end{equation}
融化的质量表达为(\unit{kg.m^{-2}}):
\begin{equation}
  M=\min \left(M_{\mathrm{ice}}, \frac{Q_{\mathrm{a v a i l}}}{\lambda_{\mathrm {fus}}}\right)
\end{equation}
其中$\lambda_{\mathrm {fus}} $表示融化潜热(\unit{J.kg^{-1}}),$M_{\mathrm{ice}}$表示该层介质的固态水含量:
\begin{equation}
  M_{\mathrm{ice}}=\left\{\begin{array}{lr}I_{i} \rho_{\mathrm{liq}} \Delta z_{\mathrm{lake},i} &
  \text { 湖泊层 } \\ w_{\mathrm{ice},i} & \text { 雪盖淤泥层 }\end{array}\right.
\end{equation}
融化过程发生后的能量剩余为:
\begin{equation}
  Q_{\mathrm{rem}}=Q_{\mathrm{avail}}-M \lambda_{\mathrm {fus}}
\end{equation}
最后,固液态水含量调整为:
\begin{equation}
  \begin{aligned}
    I_{i}^{n+1} &=I_{i}^{n}-\frac{M}{\rho_{\mathrm{liq}} \Delta z_{\mathrm{lake},i}} & \text { 湖泊层 } \\
    w_{\mathrm{ice},i}^{n+1} &=w_{\mathrm{ice},i}^{n}-M & \text { 雪盖淤泥层 } \\
    w_{\mathrm{liq},i}^{n+1} &=w_{\mathrm{liq},i}^{n}+M & \text { 雪盖淤泥层 }
  \end{aligned}
\end{equation}
体积热容量与温度调整为:
\begin{equation}
  c_{i}^{n+1}=c_{i}^{n}+\frac{M}{\Delta z_{i}}\left(C_{\mathrm{liq}}-C_{\mathrm{ice}}\right)
\end{equation}
\begin{equation}
  T_{i}^{n+1}=T_{\mathrm {frz}}+\frac{Q_{\mathrm{rem}}}{c_{i}^{n+1} \Delta z_{i}}
\end{equation}

若冻结条件满足,则$Q_{\mathrm{avail}}$表达与上述相同只是符号相反,冻结的质量也为负值,表达为:
\begin{equation}
  M=\max \left(-M_{\mathrm{liq}}, \frac{Q_{\mathrm{avail}}}{\lambda_{\mathrm {fus}}}\right)
\end{equation}
其中$M_{\mathrm{liq}}$表示该层介质的液态水含量:
\begin{equation}
  M_{\mathrm{liq}}=\left\{\begin{array}{cc}\left(1-I_{i}\right) \rho_{\mathrm{liq}} \Delta z_{\mathrm{lake},i} & \text { 湖泊层 } \\
  w_{\mathrm{liq},i} & \text { 雪盖淤泥层 }\end{array}\right.
\end{equation}
同样地,冻结过程发生后的能量亏损$Q_{\mathrm{rem}}$亦为负值。最后,介质的固液态水含量、体积比热容以及温度均进行如上类似调整。


特殊地,若湖泊处于非冻结状态($T_{\mathrm{lake,1}}>T_{\mathrm {frz}} $),其表面不存在雪层,但雪水当量$W_{\mathrm{sno}}>0$,
则湖泊表层可用于融化的能量$Q_{\text {avail }}=c_{\mathrm{lake, 1}} \Delta z_{\mathrm{lake, 1}}\left(T_{\mathrm{lake, 1}}-T_{\mathrm {frz}}\right)$将首先用于融化积雪,
融化质量$M=\min{\left(W_{\mathrm{sno}},\frac{Q_{\mathrm{avail}}}{\lambda_{\mathrm {fus}} }\right)}$。若可用能量完全消耗,
则湖泊表层温度调整为$T_{\mathrm {frz}} $;若积雪被完全融化而有能量结余,则剩余能量为$Q_{\mathrm{rem}}=Q_{\mathrm{avail}}-M\lambda_{\mathrm {fus}} $,
它将返回湖泊表层对湖泊重新进行加热,其温度调整为$T_{\mathrm{lake, 1}}=T_{\mathrm {frz}}+\frac{Q_{\mathrm{rem}}}{c_{\mathrm{lake, 1}} \Delta z_{\mathrm{lake, 1}}}$。
最后,积雪厚度$z_{\mathrm{sno}}$与雪水当量$W_{\mathrm{sno}}$调整为
\begin{equation}
  z_{\mathrm{sno}}=z_{\mathrm{sno}}\left(1-\frac{M}{W_{\mathrm{sno}}}\right)
\end{equation}
\begin{equation}
  W_{\mathrm{sno}}=W_{\mathrm{sno}}-M
\end{equation}

\section{基于湖泊层稳定性的垂直对流混合}
\esection{Stability-Based Vertical Convection}
基于稳定性的湖泊对流混合方案来自于~\citet{hostetler1993interactive,hostetler1994lake} 在其提出的湖泊大气耦合模型中采用的方案:
在热传导过程与相态变化过程模拟结束后,湖泊温度应再次调整,使得湖泊具有稳定的层结结构:
即湖泊上层密度应不大于下层密度。具体操作为,对于完全非冻结状态的湖泊,
若相邻两层的密度出现$\rho_i>\rho_{i+1}$,则湖泊发生对流混合过程,
湖泊温度从第1层到第$i+1$层均更新为前$i+1$层湖泊根据其厚度加权平均的温度,
每一层湖泊的密度根据前文公式同时进行更新。此过程的实施从第1层一直迭代进行到第$N_{\mathrm{lake}}-1$层。

对于有冰存在的湖泊,在保证湖泊总能量与冰含量守恒的前提下,冰应连续存在于湖泊上层。于是,
除上述湖泊密度条件外,湖泊冰含量的分布亦可触发湖泊垂直混合过程发生,判断条件为:
若在任何一层未完全冻结的湖泊之下有冰存在,即$I_i<1$且$I_{i+1}>0$,则前$i+1$层湖泊将发生垂直混合。
前$i+1$层湖泊的总能量表达为:
\begin{equation}
  Q=\sum_{j=1}^{i+1} \Delta z_{\mathrm{lake},j} \rho_{\mathrm{liq}}\left(T_{\mathrm{lake},j}-T_{\mathrm {frz}}\right)\left[\left(1-I_{j}\right) C_{\mathrm{liq}}+I_{j} C_{\mathrm{ice}}\right]
\end{equation}
加权平均的冻结部分质量比表达为
\begin{equation}
  I_{\mathrm{a v}}=\frac{\sum_{j=1}^{i+1} I_{j} \Delta z_{\mathrm{lake},j}}{Z_{i+1}}
\end{equation}
\begin{equation}
  Z_{i+1}=\sum_{j=1}^{i+1} \Delta z_{\mathrm{lake},j}
\end{equation}
对于混合后的前$i+1$层湖泊,冻结部分($T_{\mathrm{froz}}$)与非冻结部分($T_{\mathrm{unfr}}$)的温度将分别计算。
若$Q>0$,则说明前$i+1$层湖泊中部分层的温度在凝结点之上,那么此能量盈余将分配给完全非冻结状态的湖泊层:
\begin{equation}
  T_{\mathrm{unfr}}=\frac{Q}{\rho_{\mathrm{liq}} Z_{i+1}\left[\left(1-I_{\mathrm{av}}\right) C_{\mathrm{liq}}\right]}+T_{\mathrm {frz}}
\end{equation}
完全冻结状态的湖泊层保持在凝结点温度$T_{\mathrm{froz}}=T_{\mathrm {frz}} $。
相反,若$Q<0$,则此能量亏损将分配给完全冻结状态的湖泊层:
\begin{equation}
  T_{\mathrm{froz}}=\frac{Q}{\rho_{\mathrm{liq}} Z_{i+1} I_{\mathrm{a v}} C_{\mathrm{ice}}}+T_{\mathrm {frz}}
\end{equation}
而完全非冻结状态的湖泊层保持在凝结点温度$T_{\mathrm{unfr}}=T_{\mathrm {frz}} $。
对于只有部分冻结的湖泊层,温度将按照冻结与非冻结部分的质量比进行加权平均得到。
因为混合后的前$i+1$层湖泊中,冰应连续集中于上层,故对于前$i+1$层湖泊的任何一层$j$,
$I_j$与$T_{\mathrm{lake},j}$的计算将按照如下方式进行,记${Z_j}=\sum_{m=1}^{j} \Delta z_{\mathrm{lake},m} $:\\
\begin{enumerate}
  \item 若$Z_j\leqslant Z_{i+1}I_{\mathrm{av}}$,则$I_j=1,\ T_{\mathrm{lake},j}=T_{\mathrm{froz}}$
  \item 否则,若$Z_{j-1}<Z_{i+1} I_{\mathrm{a v}}$,则此第j层湖泊既包含水也包含冰,
    冻结部分质量比表达为$I_{j}=\frac{Z_{i+1} I_{\mathrm{a v}}-Z_{j-1}}{\Delta z_{\mathrm{lake},j}}$ ,
    温度按照冻结与非冻结部分的质量比进行加权平均,表达为
    \begin{equation}
      T_{\mathrm{lake},j}=\frac{T_{\mathrm{f r o z}} I_{j} C_{\mathrm{ice}}+T_{\mathrm{u n f r}}\left(1-I_{j}\right) C_{\mathrm{liq}}}{I_{j} C_{\mathrm{ice}}+\left(1-I_{j}\right) C_{\mathrm{liq}}}
    \end{equation}
  \item 否则,$I_j=0$,\ \ $T_{\mathrm{lake},j}=T_{\mathrm{unfr}}$
    此过程的实施亦从第1层迭代进行到第$N_{\mathrm{lake}}-1$层,并且湖泊温度更新后,密度根据公式~\eqref{rho_i} 一并更新。
\end{enumerate}


\section{湖泊模式的水文过程}\label{湖泊水文}
\esection{Lake Hydrological Processes}
在湖泊系统中,由于湖泊层的厚度与质量固定不变,故湖泊层之间可视为无水流通量交换,
湖泊系统的水文过程主要集中于湖泊之上的雪层和湖泊之下的淤泥层。
湖泊系统的质量守恒关系可表达如下:
\begin{equation}
  W_{\mathrm{sno}}^{n+1}-W_{\mathrm{sno}}^{n}+\sum_{i=1}^{N_{\mathrm{s o i}}}\left(w_{\mathrm{liq},i}^{n+1}+w_{\mathrm{ice},i}^{n+1}-w_{\mathrm{liq},i}^{n}-w_{\mathrm{ice},i}^{n}\right)=\left(p_{\mathrm {rain}}+p_{\mathrm {snow}}-E_{\mathrm{g}}-q_{\mathrm{r u n}}\right) \Delta t
\end{equation}
其中$W_{\mathrm{sno}}$表示雪水当量(\unit{kg.m^{-2}}),$w_{\mathrm{liq},i}$,$w_{\mathrm{ice},i}$分别表示第i层淤泥的液态水与固态水含量(\unit{kg.m^{-2}}),
$n$表示时间步数指标,$p_{\mathrm {rain}} $,$p_{\mathrm {snow}} $分别表示液态与固态降水量(\unit{kg.m^{-2}.s^{-1}}),$\Delta t$表示积分时间步长(s),
$E_{\mathrm {g}} $表示湖泊表面与大气的水汽通量(\unit{kg.m^{-2}.s^{-1}}),$q_{\mathrm{run}}$表示地表径流,用来平衡由湖泊质量固定所带来的水分转移与补充。


对于淤泥层,由于它位于湖泊之下,淤泥层被视为始终处于饱和状态,其体积含水量$\theta_i$表达为:
\begin{equation}
  \theta_{i}=\frac{1}{\Delta z_{i}}\left(\frac{w_{\mathrm{ice},i}}{\rho_{\mathrm{ice}}}+\frac{w_{\mathrm{liq},i}}{\rho_{\mathrm{liq}}}\right)
\end{equation}
相态变化发生后,由于冰可能发生融化,使得融化后的淤泥体积含水量小于饱和体积含水量$\theta_i<\theta_{\mathrm{sat},i}$,
此时将用液态水进行补充,淤泥液态含水量调整为:
\begin{equation}
  w_{\mathrm{liq},i}=\left(\theta_{\mathrm{sat},i} \Delta z_{i}-\frac{w_{\mathrm{ice},i}}{\rho_{\mathrm{ice}}}\right) \rho_{\mathrm{liq}} \geqslant 0.0
\end{equation}
\begin{equation}
  w_{\mathrm{ice},i}=\left(\theta_{\mathrm{sat},i} \Delta z_{i}-\frac{w_{\mathrm{liq},i}}{\rho_{\mathrm{liq}}}\right) \rho_{\mathrm{ice}} \geqslant 0.0
\end{equation}
同样,若水发生冻结使得$\theta_i>\theta_{\mathrm{sat},i}$,此时将从液态水中进行削减,淤泥液态含水量调整为:
\begin{equation}
%\begin{align}
  w_{\mathrm{liq},i}=w_{\mathrm{liq},i}-\left(\theta_{i}-\theta_{\mathrm{sat},i}\right) \Delta z_{i} \rho_{\mathrm{liq}} \geqslant 0.0
\end{equation}
\begin{equation}
  w_{\mathrm{ice},i}=\left(\theta_{\mathrm{sat},i} \Delta z_{i}-\frac{w_{\mathrm{liq},i}}{\rho_{\mathrm{liq}}}\right) \rho_{\mathrm{ice}} \geqslant 0.0
%\end{align}
\end{equation}
特殊地,若过量的冰融化使得$w_{\mathrm{liq},i}>\theta_{\mathrm{sat},i}\rho_{\mathrm{liq}}\Delta z_i$,则淤泥液态含水量被重置为:
\begin{equation}
  w_{\mathrm{liq},i}=\theta_{\mathrm{sat},i} \rho_{\mathrm{liq}} \Delta z_{i}
\end{equation}
\begin{equation}
  w_{\mathrm{ice},i}=0.0
\end{equation}

对于雪层,包含雪的压实、合并、分层等水文过程与雪盖土壤的水文过程完全一致。
一个特殊的情形是,当湖泊温度计算后,雪层可能存在于完全非冻结状态的湖泊之上($T_{\mathrm{lake,1}}>T_{\mathrm {frz}} $),
若此时湖泊可提供足够的能量将雪层融化,则雪层将会消失;否则,若湖泊完全冻结都无法将雪层融化,
则雪层将会以原有状态保留。具体计算如下,考虑四个状态变量(\unit{J.m^{-2}}):
雪层在凝结点之下升温所需的总能量:
\begin{equation}
  a=\sum_{i=s n l+1}^{0}\left(w_{\mathrm{ice},i} C_{\mathrm{ice}}+w_{\mathrm{liq},i} C_{\mathrm{liq}}\right)\left(T_{\mathrm {frz}}-T_{i}\right)
\end{equation}
雪层固态水融化所需的总能量:
\begin{equation}
  b=\sum_{i=s n l+1}^{0} \lambda_{\mathrm {fus}} w_{\mathrm{ice},i}
\end{equation}
湖泊冷却所释放的总能量:
\begin{equation}
  c=C_{\mathrm{liq}} \rho_{\mathrm{liq}} \Delta z_{\mathrm{lake, 1}}\left(T_{\mathrm{lake, 1}}-T_{\mathrm {frz}}\right)
\end{equation}
湖泊冻结所释放的总能量
\begin{equation}
  d=\lambda_{\mathrm {fus}} \rho_{\mathrm{liq}} \Delta z_{\mathrm{lake, 1}}
\end{equation}
若$c\geqslant a+b$,即湖泊不需冻结即可将雪层全部融化,则此时第一层湖泊温度调整为:
\begin{equation}
  T_{\mathrm{lake, 1}}=\frac{C_{\mathrm{liq}} \rho_{\mathrm{liq}} \Delta z_{\mathrm{lake, 1}} T_{\mathrm{lake, 1}}+\sum_{i=s n l+1}^{0} C_{\mathrm{liq}}
  T_{\mathrm {frz}}\left(w_{\mathrm{ice},i}+w_{\mathrm{liq},i}\right)-a-b}{C_{\mathrm{liq}}\left(\rho_{\mathrm{liq}} \Delta z_{\mathrm{lake, 1}}+\sum_{i=s n l+1}^{0}
  \left(w_{\mathrm{ice},i}+w_{\mathrm{liq},i}\right)\right)}
\end{equation}
否则若$c+d\geqslant a+b$,即湖泊通过部分冻结也可将雪层全部融化,则此时第一层湖泊温度和冻结部分质量比调整为:
\begin{equation}
  T_{\mathrm{lake, 1}}=T_{\mathrm {frz}}
\end{equation}
\begin{equation}
  I_{1}=\frac{a+b-c}{d}
\end{equation}
以上两种情况由于雪层已全部融化,雪水当量($W_{\mathrm{sno}}$)、积雪厚度($z_{\mathrm{sno}}$)与雪层数($snl$)更新为0。
%
\section{其他湖泊模型}\label{其他湖泊模型}
\esection{Other Lake Model}
\begin{mymdframed}{代码}
  本节对应的代码文件位于\texttt{main/LAKE}。
\end{mymdframed}

目前,一维湖泊模型大致可以分为三类:
\begin{enumerate}
  \item 多层涡流扩散模型,如CoLM-Lake模型~\citep{daiLakeSchemeCommon2018},使用半经验参数化方法来估计涡流扩散系数;
  \item 一维湍流闭合模型,如Simstrat模型~\citep{goudsmit2002application}和XOML模型~\citep{ling2015multilevel},直接量化湍流动能在整个水柱中的生成、传输和耗散;
  \item 两层参数化模型,如FLake模型~\citep{mironovCOSMOTechnicalReport2008a},基于自相似理论定义上层混合层和下层温跃层。
\end{enumerate}
这些模型由于采用不同的参数化方案,其适用性各有差异。

为使研究者能够根据具体的研究需求和湖泊特征灵活选择最适合的模型,我们将Simstrat模型、XOML模型和FLake模型等三个一维湖泊模型与CoLM进行了耦合,作为处理湖泊过程的可选方案。这些模型与CoLM-Lake模型共享相同的湖泊模式结构和表面通量计算方法,主要区别在于温度的求解方式。此外,需要注意的是,Simstrat模型和XOML模型都不考虑湖底淤泥层的影响。
%
\subsection{Simstrat模型}\label{Simstrat模型}
\esubsection{Simstrat Model}\label{Simstrat Model}

Simstrat模型由\citet{goudsmit2002application}开发,是一个基于浮力扩展的双方程$k-\epsilon$湍流闭合模型。其独特之处在于,通过分离风能来参数化由盆地尺度内波引发的湍流。此外,Simstrat模型在湖泊能量方程的计算中考虑了湖泊水平横截面积的深度变化,而动量方程则包含了科里奥利力和垂直粘度的影响。其对湖泊内部湍流动能的生成与耗散的详细描述,为深湖动力学模拟提供了一个可靠且精确的模型选择,尤其适用于对湖泊内部复杂物理过程的高精度模拟。总体而言,该模型的湖泊层温度的控制方程与CoLM-Lake的控制方程相似,具体表达如下:
\begin{equation}
    \frac{\partial T}{\partial t}=\frac{1}{A_{\mathrm{z}}} \frac{\partial}{\partial z}\left(A_{\mathrm{z}} \tau \frac{\partial T}{\partial z}\right)+\frac{1}{\rho_{\mathrm{liq}} C_{\mathrm{liq}}} \frac{\partial \phi}{\partial z}
\end{equation}
其中,$A_{\mathrm{z}}$表示湖泊在深度$z$处的横截面积(\unit{m^2}),其他变量的含义与CoLM-Lake中的一致。热力传导率$\tau$表达式为:
\begin{equation}
    \tau = v^{\mathrm{'}}_{\mathrm{t}} + k_{\mathrm{m}}
\end{equation}
其中,$k_{\mathrm{m}}$表示液态水分子热扩散率(\unit{m^2.s^{-1}}),取值与CoLM-Lake相同。$v^{\mathrm{'}}_{\mathrm{t}}$表示湍流扩散率(\unit{m^2.s^{-1}}),其表达式为:
\begin{equation}
    v^{\mathrm{'}}_{\mathrm{t}} = c^{\mathrm{'}}_{\mathrm{\mu}}\frac{k^\mathrm{2}}{\epsilon}
\end{equation}
其中,$c^{\mathrm{'}}_{\mathrm{\mu}}=0.072$是无量纲常数,$k$表示单位质量的湍流动能(\unit{J.kg^{-1}}),$\epsilon$表示湍流动能耗散率(\unit{W.kg^{-1}})。$k$和$\epsilon$的控制方程分别为:
\begin{equation}
    \frac{\partial k}{\partial t} = \frac{1}{A_{\mathrm{z}}}\frac{\partial}{\partial z}\left(A_{\mathrm{z}} v_{\mathrm{k}} \frac{\partial k}{\partial z}\right) + P + P_{\mathrm{seiche}} + B - \epsilon
\end{equation}
\begin{equation}
    \frac{\partial \epsilon}{\partial t} = \frac{1}{A_{\mathrm{z}}}\frac{\partial}{\partial z}\left(A_{\mathrm{z}} v_{\mathrm{\epsilon}} \frac{\partial \epsilon}{\partial z}\right) + \frac{\epsilon}{k}\left[c_{\mathrm{\epsilon 1}} \left(P + P_{\mathrm{seiche}}\right) + c_{\mathrm{\epsilon 3}} B - c_{\mathrm{\epsilon 2}} \epsilon\right]
\end{equation}
其中,$P$表示由于剪切应力而产生的湍流动能(\unit{W.kg^{-1}}),$P_{\mathrm{seiche}}$表示由于湖泊内波而产生的湍流动能(\unit{W.kg^{-1}})。$B$=-$v^{\mathrm{'}}_{\mathrm{t}} N^2$表示浮力通量(\unit{W.kg^{-1}})。$v_{\mathrm{k}}$和$v_{\mathrm{\epsilon}}$分别表示$k$和$\epsilon$的湍流扩散率(\unit{m^2.s^{-1}}),是基于前一时间步长的$k$和$\epsilon$计算得出的,计算方法如下:
\begin{equation}
    v_{\mathrm{k}} = \frac{c_{\mathrm{\mu}}}{\sigma_{\mathrm{k}}}\frac{k^2}{\epsilon}
\end{equation}
\begin{equation}
    v_{\mathrm{\epsilon}} = \frac{c_{\mathrm{\mu}}}{\sigma_{\mathrm{\epsilon}}}\frac{k^2}{\epsilon}
\end{equation}
上述等式中,$c_{\mathrm{\mu}}$、$\sigma_{\mathrm{k}}$、$\sigma_{\mathrm{\epsilon}}$、$c_{\mathrm{\epsilon 1}}$、$c_{\mathrm{\epsilon 2}}$和$c_{\mathrm{\epsilon 3}}$均为无量纲常数,取值分别为$c_{\mathrm{\mu}}=0.09$、$\sigma_{\mathrm{k}}=1.00$、$\sigma_{\mathrm{\epsilon}}=1.30$、 $c_{\mathrm{\epsilon 1}}=1.44$、$c_{\mathrm{\epsilon 2}}=1.92$,$c_{\mathrm{\epsilon 3}}$的值取决于浮力通量方向:
\begin{equation}
  c_{\mathrm{\epsilon 3}}=\left\{\begin{array}{ll}0.4 & B<0  \\ 1 & B \geqslant 0 \end{array}\right.
\end{equation}

由剪切应力和湖泊内波引发的湍流动能生成分别被表示为:
\begin{equation}
    P = v_{\mathrm{t}} \left[\left(\frac{\partial u_{\mathrm{liq}}}{\partial z}\right)^2 + \left(\frac{\partial v_{\mathrm{liq}}}{\partial z}\right)^2 \right]
\end{equation}
\begin{equation}
    P_{\mathrm{seiche}} = \alpha A_{\mathrm{surf}} \rho_{\mathrm{a}} C_{\mathrm{10}} u^{\mathrm{3}}_{\mathrm{10}}
\end{equation}
其中,$v_{\mathrm{t}}$表示湍流粘度(\unit{m^2.s^{-1}}),其计算方式与湍流扩散率$v^{\mathrm{'}}_{\mathrm{t}}$相近,表达式为:$v_{\mathrm{t}}=c_{\mathrm{\mu}} \frac{k^2}{\epsilon}$。$u_{\mathrm{10}}$表示10\unit{m}处的平均风速(\unit{m.s^{-1}}),$A_{\mathrm{surf}}$表示湖泊表面积(\unit{m^2}),$\rho_{\mathrm{a}}$表示大气密度(\unit{kg.m^{-3}})。$C_{\mathrm{10}}$为风阻系数,取值为1。$\alpha$表示风能转化为湖泊内波的效率,取值为0.002。$u_{\mathrm{liq}}$和$A_{\mathrm{surf}}$分别为$x$方向和$y$方向的水流速度(\unit{m.s^{-1}}),其控制方程分别为:
\begin{equation}
    \frac{\partial u_{\mathrm{liq}}}{\partial t} = \frac{1}{A_{\mathrm{z}}} \frac{\partial}{\partial z} \left[A_{\mathrm{z}} \left(v_{\mathrm{t}} + k_{\mathrm{v}} \right) \frac{\partial u_{\mathrm{liq}}}{\partial z}\right] + f v_{\mathrm{liq}}
\end{equation}
\begin{equation}
    \frac{\partial v_{\mathrm{liq}}}{\partial t} = \frac{1}{A_{\mathrm{z}}} \frac{\partial}{\partial z} \left[A_{\mathrm{z}} \left(v_{\mathrm{t}} + k_{\mathrm{v}} \right] \frac{\partial v_{\mathrm{liq}}}{\partial z}\right) - f u_{\mathrm{liq}}
\end{equation}
其中,$k_{\mathrm{v}}=1.5×10^{-6}$表示分子动能扩散率(\unit{m^2.s^{-1}}),$f=2 \Omega \mathrm{sin}⁡\theta$为科里奥利参数。在此,$\Omega=7.29×10^{-5}$为地球自转角速度(\unit{s^{-1}}),$\theta$为地理纬度。

关于Simstrat模型更详细的描述,可参见\citet{goudsmit2002application}和\citet{gaudardOptimizingParameterizationDeep2017}。

\subsection{XOML模型}\label{XOML模型}
\esubsection{XOML Model}\label{XOML Model}

XOML模型由\citet{ling2015multilevel}开发,采用单方程$k$湍流闭合方案,借鉴了\citet{noh2002simulation,noh2011prediction}提出的二阶湍流海洋模型。与Simstrat相比,XOML使用了更简化的参数化方式,并且特别注重湖泊的日循环变化,对模拟湖泊昼夜分层现象具有更强的模拟能力。XOML的湖泊层控制方程与Simstrat相似,但其未考虑湖泊水平横截面积随深度的变化及分子扩散率。具体方程如下:
\begin{equation}
    \frac{\partial T}{\partial t} = \frac{\partial}{\partial z} \left( \tau \frac{\partial T}{\partial z} \right) + \frac{1}{\rho_{\mathrm{liq}} C_{\mathrm{liq}}} \frac{\partial \phi}{\partial z}
\end{equation}
\begin{equation}
    \frac{\partial k}{\partial t} = \frac{\partial}{\partial z} \left( v_{\mathrm{k}} \frac{\partial k}{\partial z} \right) + P + B - \epsilon
\end{equation}
\begin{equation}
    \frac{\partial u_{\mathrm{liq}}}{\partial t} = \frac{\partial}{\partial z} \left( v_{\mathrm{t}} \frac{\partial u_{\mathrm{liq}}}{\partial z} \right) + f v_{\mathrm{liq}}
\end{equation}
\begin{equation}
    \frac{\partial v_{\mathrm{liq}}}{\partial t} = \frac{\partial}{\partial z} \left( v_{\mathrm{t}} \frac{\partial v_{\mathrm{liq}}}{\partial z} \right) - f u_{\mathrm{liq}}
\end{equation}

与Simstrat采用湍流动能和动能耗散率计算扩散率的方式不同,XOML通过湍流速度尺度和长度尺度来参数化扩散率,具体公式为,$\tau = \frac{c_{\mathrm{q}}}{0.8} q l$,$v_{\mathrm{k}} = \frac{c_{\mathrm{q}}}{1.95} q l$,$v_{\mathrm{t}} = c_{\mathrm{q}} q l$,这里,$c_{\mathrm{q}}$的计算通过考虑湍流理查德数($R_{\mathrm{t}}$)来反映湖泊的分层效应,计算公式如下:
\begin{equation}
    c_{\mathrm{q}} = 0.39(1 + \varrho R_{\mathrm{t}})^{-1/2}
\end{equation}
其中,$\varrho$为经验参数,取值为120。$R_{\mathrm{t}}$的计算方法为:
\begin{equation}
    R_{\mathrm{t}} = \left( \frac{N l}{q} \right)^2
\end{equation}
其中,$q$和$l$分别表示湍流速度尺度和长度尺度,它们的计算方式如下:
\begin{equation}
    q = \sqrt{2k}
\end{equation}
\begin{equation}
    l = \frac{1}{\left[ \frac{1}{\kappa(z + z_{\mathrm{0m}})} + \frac{1}{h} \right]}
\end{equation}
在此,$\kappa$为von K\'arman常数(见表~\ref{tab:物理常数}),$z$为湖泊深度(\unit{m}),$z_{\mathrm{0m}}$表示动量粗糙度(\unit{m})。$h$表示湖泊混合层深度(\unit{m}),可以通过温度梯度和密度梯度的阈值来确定。

关于XOML模型更详细的描述,可参见\citet{ling2015multilevel}和\citet{noh2002simulation, noh2011prediction}。

\subsection{FLake模型}\label{FLake模型}
\esubsection{FLake Model}\label{FLake Model}

FLake是基于温跃层结构自相似性概念而开发的一维湖泊模型~\citep{mironovCOSMOTechnicalReport2008a}。该模型将湖泊分为上层和下层。上层称为混合层,保持均匀的温度,并假定混合良好。下层位于混合层和湖底之间,代表温跃层。根据自相似理论,下层厚度的变化不会改变其温度曲线的基本特征,即:
\begin{equation}\label{flaketmp}
    T_{\mathrm{(z)}}=\left\{\begin{array}{ll} T_{\mathrm{s}} & 0 \leqslant z \leqslant h  \\ T_{\mathrm{s}} - \left(T_{\mathrm{s}}-T_{\mathrm{b}} \right) \Phi_{\mathrm{T}} \left(\zeta \right) & h < z \leqslant d \end{array}\right.
\end{equation}
其中,$T_{\mathrm{(z)}}$表示湖泊在深度$z$处的湖泊温度,$T_{\mathrm{s}}$和$T_{\mathrm{b}}$分别表示湖泊上层和下层水温。$h$表示混合层深度,$d$代表湖泊深度。$\Phi_{\mathrm{T}} \left(\zeta \right)$表示温跃层中的无量纲温度分布曲线,可以使用深度的四阶多项式函数进行参数化:
\begin{equation}
    \Phi_{\mathrm{T}} \left(\zeta \right) = \left( \frac{40}{3} C_{\mathrm{T}} - \frac{20}{3} \right) \zeta + (18 - 30 C_{\mathrm{T}}) \zeta^2 + (20 C_{\mathrm{T}} - 12) \zeta^3 + \left( \frac{5}{3} - \frac{10}{3} C_{\mathrm{T}} \right) \zeta^4
\end{equation}
$C_{\mathrm{T}}$是形状因子,计算方式为:
\begin{equation}
    \frac{\mathrm{d} C_{\mathrm{T}}}{\mathrm{d} t} = \mathrm{sign} \left( \frac{\mathrm{d} h}{\mathrm{d} t} \right) \frac{ \left( C_{\mathrm{T}}^{\mathrm{max}} - C_{\mathrm{T}}^{\mathrm{min}} \right) }{t_{\mathrm{rc}}}, \quad C_{\mathrm{T}}^{\min} \leq C_{\mathrm{T}} \leq C_{\mathrm{T}}^{\mathrm{max}}
\end{equation}
其中,$C_{\mathrm{T}}^{\text{max}}$ 和 $C_{\mathrm{T}}^{\text{min}}$取值为0.8和0.5,$t_{\mathrm{rc}} = \frac{\left( d - h \right) ^ {\mathrm{2}} \overline{N}}{C_{\mathrm{rc}} u_{\mathrm{T}} ^ {\mathrm{2}}}$表示松弛时间。在此,$\overline{N}$表示温跃层中的均方浮力频率。$u_{\mathrm{T}}=\mathrm{max}\left(w_*, u_* \right)$,其中$w_*$和$u_*$分别表示对流速度尺度(计算方式见公式\eqref{flakews})和表面摩擦速度(计算方式见章节\ref{湖泊表面能量通量与温度的计算})。$C_{\mathrm{rc}} = 0.003$是无量纲常数。无量纲深度$\zeta$的计算方式为:
\begin{equation}
    \zeta \equiv \frac{\left( z - h \right)}{\Delta h}
\end{equation}

根据公式\eqref{flaketmp},$h$、$d$、$T_{\mathrm{s}}$、$T_{\mathrm{b}}$与整个水柱的平均温度$\overline{T}$之间的关系可以表示为:
\begin{equation}\label{flakemeanT}
    \overline{T} = T_{\mathrm{s}} - C_{\mathrm{T}} \left( 1 - \frac{h}{d} \right) \left( T_{\mathrm{s}} - T_{\mathrm{b}} \right)
\end{equation}

湖泊总热量收支方程为:
\begin{equation}\label{flakedmeanT}
    d \frac{\mathrm{d} \overline{T}}{\mathrm{d}t} = \frac{1}{\rho_{\mathrm{liq}} C_{\mathrm{liq}}} \left[ Q_{\mathrm{s}} - Q_{\mathrm{b}} + \phi_{\mathrm{s}} - \phi(d) \right]
\end{equation}
式中,$Q_{\mathrm{b}}$表示通过湖底的热通量(\unit{W.m^{2}}),$Q_{\mathrm{s}}$表示空气-水界面的感热通量、潜热通量以及净长波辐射通量之和(\unit{W.m^{2}})。$\phi_{\mathrm{s}} - \phi(d)$表示湖泊吸收的总太阳短波辐射。
湖泊上层(混合层)热量收支方程为:
\begin{equation}\label{flakedTs}
    h \frac{\mathrm{d} T_{\mathrm{s}}}{\mathrm{d}t} = \frac{1}{\rho_{\mathrm{liq}} C_{\mathrm{liq}}} \left[ Q_{\mathrm{s}} - Q_{\mathrm{h}} + \phi_{\mathrm{s}} - \phi \left( h \right) \right]
\end{equation}
式中,$Q_{\mathrm{h}}$为混合层底部的热通量(\unit{W.m^{2}})。

在使用CoLM-Lake的表面通量方案给定$Q_{\mathrm{s}}$和$\phi_{\mathrm{s}}$(见章节\ref{湖泊表面能量通量与温度的计算}),以及短波辐射通量的衰减规律(见章节\ref{雪层湖泊层淤泥层温度的计算})后,公式\eqref{flakemeanT}、\eqref{flakedmeanT}和\eqref{flakedTs}中包含6个未知数,即 $h$、$\overline{T}$、$T_{\mathrm{s}}$、$T_{\mathrm{b}}$、$Q_{\mathrm{h}}$、$Q_{\mathrm{b}}$。为计算$Q_{\mathrm{h}}$和$Q_{\mathrm{b}}$,FLake假设在上层(混合层)加深的情况下,即$\frac{\mathrm{d} h}{\mathrm{d} t} > 0$,温跃层垂直湍流热通量剖面可以用自相似的形式表示:
\begin{equation}\label{flakeQ}
    Q = Q_{\mathrm{h}} - (Q_{\mathrm{h}} - Q_{\mathrm{b}}) \Phi_{\mathrm{Q}} (\zeta), \quad h < z \leq d
\end{equation}
其中,形状曲线$\Phi_{\mathrm{Q}}$满足边界条件$\Phi_{\mathrm{Q}}(0) = 0$和$\Phi_{\mathrm{Q}}(1) = 1$,并与$\Phi_{\mathrm{T}}(\zeta)$的关系如下:
\begin{equation}
    \Phi_{\mathrm{Q}}(\zeta) = C_{\mathrm{T}} ^ {\mathrm{-1}} \left[ \left( 1 - \zeta \right) \Phi_{\mathrm{T}}(\zeta) + \int_{\mathrm{0}} ^ {\zeta} \Phi_{\mathrm{T}}(\zeta') \mathrm{d} \zeta' \right]
\end{equation}

在公式\eqref{flaketmp}中温度分布应该满足传热方程:
\begin{equation}\label{flakedrhocT}
    \frac{\partial}{\partial t} \left( \rho_{\mathrm{liq}} C_{\mathrm{liq}} T \right) = -\frac{\partial}{\partial z} \left( Q + \phi \right)
\end{equation}
在上式中考虑公式\eqref{flaketmp}和\eqref{flakeQ},并对$z'$从$h$到$z > h$积分,然后对积分结果的$z$从$h$到$d$进行积分,可以得到:
\begin{equation}\label{flakennnn}
\begin{split}
    &\frac{1}{2} (d - h)^{\mathrm{2}} \frac{\mathrm{d} T_{\mathrm{s}}}{\mathrm{d} t} - \frac{\mathrm{d} \left[ C_{\mathrm{TT}} (d - h)^{\mathrm{2}} (T_{\mathrm{s}} - T_{\mathrm{b}}) \right]} {\mathrm{d} t}  \\
    &= \frac{1}{\rho_{\mathrm{liq}} C_{\mathrm{liq}}} \left[ C_{\mathrm{Q}} (d - h)(Q_{\mathrm{h}} - Q_{\mathrm{b}}) + (d - h) \phi(h) - \int_{\mathrm{h}} ^ {\mathrm{d}} \phi(z) {\mathrm{d}} z \right]
\end{split}
\end{equation}
式中$C_{\mathrm{Q}}$是热流量的形状因子,表达式为:
\begin{equation}
    C_{\mathrm{Q}} = \frac{2 C_{\mathrm{TT}}}{C_{\mathrm{T}}}
\end{equation}
$C_{\mathrm{TT}}$是与$C_{\mathrm{T}}$相关的无量纲参数:
\begin{equation}
    C_{\mathrm{TT}} = \frac{11}{18} C_{\mathrm{T}} - \frac{7}{45}
\end{equation}

在混合层稳定或减退情况下,即$\frac{\mathrm{d} h}{\mathrm{d} t} \leqslant 0$,公式\eqref{flakeQ}不成立。此时假定底部温度为处于恒定状态,即$\frac{\mathrm{d} T_{\mathrm{b}}}{\mathrm{d} t} = 0$。而当$h = d$时,$T_{\mathrm{s}} = T_{\mathrm{b}} = \overline{T}$并使用公式\eqref{flakedmeanT}计算湖泊平均温度。

FLake模型中,混合层深度$h$的计算需要考虑对流和稳定分层的情况。当浮力通量$B_* = \frac{\beta}{\rho_{\mathrm{liq}} C_{\mathrm{liq}}}$为负时,表明湍流动能由对流不稳定性产生,此时混合层的加深由夹卷方程计算。其中$\beta = g \cdot 1.6509 \times 10^{-5}$为浮力参数。夹卷比$A$定义为混合层底部的热通量$Q_{\mathrm{h}}$与一个适当热通量尺度$Q_*$的负比。为考虑辐射加热的垂直分布,使用广义对流热通量尺度:
\begin{equation}
    Q_* = Q_{\mathrm{s}} + \phi_{\mathrm{s}} + \phi(h) - 2h^{\mathrm{-1}} \int_{\mathrm{0}} ^ \mathrm{h} \phi(z) \mathrm{d} z
\end{equation}
对流速度尺度和夹带比定义为:
\begin{equation}\label{flakews}
    w_* = \left[-\frac{h \beta(T_{\mathrm{s}}) Q_*}{\rho_{\mathrm{liq}} C_{\mathrm{liq}}}\right]^{1/3}, \quad A = -\frac{Q_{\mathrm{h}}}{Q_*}
\end{equation}
采用夹带方程的形式指定$A$:
\begin{equation}\label{flakeA}
    A + \frac{C_{\mathrm{c2}}}{w_*} \frac{\mathrm{d} h}{\mathrm{d} t} = C_{\mathrm{c1}}
\end{equation}
其中,$C_{\mathrm{c1}} = 0.17$和 $C_{\mathrm{c2}} = 1.0$是无量纲常数。公式\eqref{flakeA}的第二项是自旋修正项,当混合层较浅时,该项可防止$h$过快增长。如果自旋项很小,方程简化为$A = C_{\mathrm{c1}}$。结合上述公式即可计算对流情况下的混合层深度。

在稳定分层时,利用松弛型速率方程来计算稳定或中性分层风混合层的深度$h$:
\begin{equation}
    \frac{\mathrm{d} h}{\mathrm{d} t} = \frac{(h_{\mathrm{e}} - h)}{t_{\mathrm{rh}}}
\end{equation}
其中,$h_{\mathrm{e}}$为平衡混合层深度,$t_{\mathrm{rh}}$为松弛时间标度:
\begin{equation}
    t_{\mathrm{rh}} = \frac{h_{\mathrm{e}}}{C_{\mathrm{rh}} u_*}
\end{equation}
式中$C_{\mathrm{rh}} = 0.03$为无量纲常数。平衡混合层深度$h_\mathrm{e}$由以下方程给出:
\begin{equation}
    \left( \frac{f h_{\mathrm{e}}}{C_{\mathrm{n}} u_*} \right)^2 + \frac{h_{\mathrm{e}}}{C_{\mathrm{s}} L} + \frac{N h_{\mathrm{e}}}{C_{\mathrm{i}} u_*} = 1
\end{equation}
其中,$L$为奥布霍夫长度,$C_{\mathrm{n}}$、$C_{\mathrm{s}}$和$C_{\mathrm{i}}$为无量纲常数,分别取为:0.5、10、20。$f$和$N$分别为科里奥利参数和混合层以下浮力频率。

最后,上述自相似性概念还用于参数化FLake模型湖底淤泥层的温度,以确保下边界的热通量平衡。需要注意的是,由于模型采用双层热结构,未包含深湖中温跃层与湖底之间常见的湖泊底层,因此对湖泊的适用深度有所限制。在CoLM中,我们遵循\citet{perroud2009simulation}的设置,对于深度超过60\unit{m}的湖泊,设置了一个虚拟深度。并且为与CoLM-Lake的湖泊分层结构保持一致,FLake在计算完成后,将根据每个湖泊层中点的深度计算该层的湖泊温度(计算方法见公式\eqref{flaketmp})。

关于FLake模型更详细的描述,可参见\citet{mironovCOSMOTechnicalReport2008a}。
\chapter{冰川模式}
\begin{mymdframed}{代码}
  本节对应的代码文件为\texttt{MOD\_Glacier.F90}。
\end{mymdframed}
\section{冰川模式结构}
冰川模式的结构类似于雪盖土壤层,区别在于冰川模式将土壤层替换为了冰层。即在CoLM中,冰川分为上部的雪盖层和下部的冰川冰层(图~\ref{fig:模式中雪盖土壤和雪盖冰川的分层对比})。雪盖根据雪盖高度$z_{\mathrm{sno}}$被分为最多五层,从上到下分别用$i=-4,-3,-2,-1,0$编号。$i=0$表示底层,与冰面相邻,$i=snl+1$表示顶层,$snl$表示雪层总数的相反数($-5\leqslant snl\leqslant 0$)。冰川冰的分层规则和土壤一致,默认分为10层,具体分层规则见章节~\ref{土壤和积雪的垂直分层}。雪盖冰川层的厚度表示为$\Delta z_i$(m),每一层的深度$z_i$(m)取为其上边界深度 $z_{\mathrm{h},i-1}$和下边界深度$z_{\mathrm{h},i}$的中点\todo{好像表述有问题}。冰川模式的物理方案均遵从无植被覆盖下的雪盖土壤层的计算方案来设定\todo{意思不太明确}。

{
  \begin{figure}[htbp]
    \centering
    \includegraphics[width=0.8\textwidth]{Figures/冰川模式/模式中雪盖土壤和雪盖冰川的分层对比.jpg}
    \caption{模式中雪盖土壤和雪盖冰川的分层对比}
    \label{fig:模式中雪盖土壤和雪盖冰川的分层对比}
  \end{figure}
}

\section{冰川表面湍流通量计算方案}
冰川表面湍流通量的计算方案类似无植被覆盖下的雪盖土壤地表的湍流通量计算方案(见章节~\ref{无植被覆盖地表湍流通量的计算方案}),这里再对计算流程做一个简单说明。

冰川表面可以分为被积雪覆盖和未被积雪覆盖两部分,因此在计算湍流通量前,需先计算地表的积雪覆盖比例$f_{\mathrm{sno}}$,其计算同陆表积雪覆盖比例(章节~\ref{积雪覆盖比例})。

由于冰川属于无植被覆盖的陆地表面,因此湍流通量只存在于地面和大气之间,动量通量$\tau$、感热通量$H$和水汽通量$E$分别表达为:
\begin{align}
  \tau_{\mathrm {x}}  &= -\rho_{\mathrm{a}} \frac{u_{\mathrm{a}}}{r_{\mathrm{am}}} \\
  \tau_{\mathrm {y}}  &= -\rho_{\mathrm{a}} \frac{v_{\mathrm{a}}}{r_{\mathrm{am}}} \\
  H_{\mathrm {g}}  &= -\rho_{\mathrm{a}} C_{\mathrm{a}} \frac{\theta_{\mathrm{a}}-T_{\mathrm {g}} }{r_{\mathrm{ah}}} \\
  E_{\mathrm {g}}  &= -\rho_{\mathrm{a}} \frac{q_{\mathrm{a}}-q_{\mathrm {g}} }{r_{\mathrm{aw}}}
\end{align}
其中$\rho_{\mathrm{a}}$表示空气密度(\unit{kg.m^{-3}}),$u_{\mathrm{a}}$和$v_{\mathrm{a}}$分别表示纬向风速和经向风速,$C_{\mathrm{a}}$表示空气的比热容(\unit{J.kg^{-1}.K^{-1}}),$\theta_{\mathrm{a}}$表示大气位温(K),$r_{\mathrm{am}}$、$r_{\mathrm{ah}}$和$r_{\mathrm{aw}}$分别表示空气动力学、感热通量和水汽通量的湍流阻抗系数(\unit{s.m^{-1}}),$T_{\mathrm {g}} $表示地表温度(K),当有积雪覆盖时,$T_{\mathrm {g}} $为最上层雪层的温度,否则,$T_{\mathrm {g}} $取为第一层冰层的温度。$q_{\mathrm {g}} $表示地表空气的比湿,取为温度在$T_{\mathrm {g}} $时的饱和比湿,即$q_{\mathrm {g}} =q^{T_{\mathrm {g}} }_{\mathrm{sat}}$。

由于地表无植被覆盖,在计算阻抗系数$r_{\mathrm{am}}$、$r_{\mathrm{ah}}$、$r_{\mathrm{aw}}$时,零平面位移取为$d=0$ m。动量粗糙度在无积雪覆盖时取为$z_{\rm 0m}=0.001$ m,有积雪覆盖时取为$z_{\rm 0m}=0.002$ m~\citep{brock_willis_sharp_2006}。感热和水汽粗糙度取为:
\begin{equation}z_{\rm 0h}=z_{\rm 0w}=z_{\rm 0m}\exp{\left[-0.13\left(Re_*\right)^{0.45}\right]}
\end{equation}
其中$Re_*=u_*\cdot z_{\rm 0m}/\upsilon$表示粗糙雷诺数,$\upsilon= 1.5 \times 10^{-5}$ \unit{m^2.s^{-1}}为空气动力学粘性系数。

基于此,冰川表面湍流通量的具体计算流程如下:
\begin{enumerate}
  \item 给出计算风速$V_{\mathrm {a}} $时$U_{\mathrm {c}} $的初始猜测:
    \begin{equation}
      U_{\mathrm {c}}  = \begin{cases}
        0, &\text{当}\ \theta_{\mathrm{v,atm}}-\theta_{\mathrm{v,s}} \geqslant 0 \text{ 时(稳定条件)} \\
        0.5, &\text{当}\ \theta_{\mathrm{v,atm}}-\theta_{\mathrm{v,s}} < 0 \text{ 时(不稳定条件)}
      \end{cases}
    \end{equation}
  \item 通过$R_{\mathrm{ib}}$给出Monin-Obukhov长度$L$的初始猜测;
  \item 迭代以下过程以获得冰川表面的能量通量:\\
    a. 通过风速、温度和比湿的微风方程积分结果求解$u_*$、$\theta_*$和$q_*$,\\
    b. 更新感热和水汽粗糙度$z_{\rm 0h}$和$z_{\rm 0w}$,\\
    c. 计算虚位温尺度$\theta_{\mathrm{v*}}$,\\
    d. 更新大气风速$V_{\mathrm {a}} $,\\
    e. 计算新一步$L$,\\
    每完成上面的一次迭代过程判断$L$是否改变符号,若符号改变达到四次或以上,视为中性条件,跳出迭代,否则持续迭代直至6次;
  \item 计算湍流阻抗系数$r_{\mathrm{am}}$、$r_{\mathrm{ah}}$和$R_{\mathrm{aw}}$;
  \item 计算动量通量$\tau_{\mathrm {x}} $、$\tau_{\mathrm {y}} $,感热通量$H_{\mathrm {g}} $和水汽通量$E_{\mathrm {g}} $;
  \item 计算 2 m气温$T_{\rm 2m}$和比湿$q_{\rm 2m}$。
\end{enumerate}

\section{冰川温度计算方案}
冰川垂直层温度的计算方案同样类似于雪盖土壤层的垂直层温度计算方案。假设冰川无水平物质能量交换,则垂直方向上的一维能量平衡方程如下:
\begin{equation}\label{eq:GlacierThermalCons}
  c \frac{\partial T}{\partial t}=-\frac{\partial F}{\partial z},  \quad F=-\lambda \frac{\partial T}{\partial z}
\end{equation}
其中$c$表示雪盖或冰川冰的体积热容(\unit{J.m^{-3}.K^{-1}}),$T$表示雪盖或冰川冰温度(K),$t$表示时间(s),$z$表示雪盖冰川层的深度,$F$表示垂直方向的热通量(向上为正方向,\unit{W.m^{-2}}),$\lambda$表示热导率(\unit{W.m^{-1}.K^{-1}})。由于冰川仅考虑由液态水和固态水组成,其体积热容和热导率计算与其它雪盖土壤层稍有不同,采用简化的计算方案,下面详细说明。

对于体积热容,其由每层液态水和固态水的体积热容根据各自体积百分比加权得到,即
\begin{equation}
  c_i = \frac{w_{\mathrm{ice},i}}{\Delta z_i}C_{\mathrm{ice}} + \frac{w_{\mathrm{liq},i}}{\Delta z_i}C_{\mathrm{liq}}
\end{equation}
其中,$w_{\mathrm{ice},i}$和$w_{\mathrm{liq},i}$分别表示第$i$层固态水含量和液态水含量(\unit{kg.m^{-2}}),$C_{\mathrm{liq}}$和$C_{\mathrm{ice}}$分别表示固态水和液态水的体积热容量(见表~\ref{tab:物理常数})。特别地,若此时无雪盖分层但雪水当量$W_{\mathrm{sno}}>0$,则将这部分热容量考虑为固态水的热容量加到冰层顶层中,冰层顶层(即编号$i=1$)热容量重新计算为:
\begin{equation}
  c_1 = \frac{w_{\mathrm{ice,1}}+W_{\mathrm{sno}}}{\Delta z_1}C_{\mathrm{ice}} + \frac{w_{\mathrm{liq,1}}}{\Delta z_1}C_{\mathrm{liq}}
\end{equation}

对于热导率,雪层部分(即$snl+1\leqslant i\leqslant 0$)采用~\citet{jordan1991one}提出的方案:
\begin{equation}
  \lambda_i = k_{\mathrm {a}}  + \left(7.75 \times 10^{-5} \rho_{\mathrm{sno},i} + 1.105\times 10^{-6} \rho^2_{\mathrm{sno},i}\right)\left(k_{\mathrm {ice}}-k_{\mathrm {a}} \right)
\end{equation}
其中$k_{\mathrm {a}} $表示空气的热导率(\unit{W.m^{-1}.K^{-1}}),$\rho_{\mathrm{sno},i}=\left(w_{\mathrm{liq},i}+w_{\mathrm{ice},i}\right)/\Delta z_i$表示第$i$层雪层的平均密度(\unit{kg.m^{-3}}),$k_{\mathrm {ice}}$为固态水的热导率(表A.1)。冰层部分(即$1\leqslant i \leqslant 10$)采用~\citet{yen1981review}提出的方案:
\begin{equation}
  \lambda_i =\begin{cases}
    k_{\mathrm {liq}} &\text{当}\ T_i > T_{\mathrm f} \text{ 时} \\
    9.828 {\mathrm e}^{-0.0057 T_i} &\text{当}\ T_i \leqslant T_{\mathrm {frz}}  \text{ 时}
  \end{cases}
\end{equation}
其中$k_{\mathrm {liq}}$为液态水的热导率(表~\ref{tab:物理常数})

对方程~\eqref{eq:GlacierThermalCons} 进行离散(见章节~\ref{温度求解的数值格式}),则第$i$层雪盖冰川层的能量平衡方程可表达为:
\begin{equation}\label{eq:GlacierThermal}
  \frac{c_i \Delta z_i}{\Delta t} \left(T^{n+1}_i - T^n_i\right) = F_i - F_{i-1}
\end{equation}
其中$\Delta t$表示积分时间步长,$n$表示时间步数,$F_i$表示第$i+1$层传导到第$i$层的热通量,其离散形式为:
\begin{equation}
  F_i = \lambda \left[z_{\mathrm{h},i}\right] \frac{T_i-T_{i+1}}{z_i-z_{i+1}}
\end{equation}
$\lambda\left[z_{\mathrm{h},i}\right]$表示第$i+1$层和第$i$层交界面处的热导率:
\begin{equation}
  \lambda \left[z_{\mathrm{h},i}\right] = \begin{cases}
    \frac{\lambda_i\lambda_{i+1}\left(z_i-z_{i-1}\right)}{\lambda_i\left(z_{\mathrm{h},i}-z_{i+1}\right)+\lambda_{\mathrm{i+1}}\left(z_i-z_{\mathrm{h},i}\right)}  &\text{对于}\ i=snl+1,\ \ldots,\ 9 \\
    0 &\text{对于}\ i=10
  \end{cases}
\end{equation}
特别的,对于雪盖与冰川冰的交界面,为防止最下层雪层厚度过大导致$\lambda\left[z_{\mathrm{h},i}\right]$计算不准,当$i=0$且$z_{i+1}-z_{\mathrm{h},i}<z_{\mathrm{h},i}-z_i$时,该处的$\lambda\left[z_{\mathrm{h,0}}\right]$重新计算为:
\begin{equation}
  \lambda\left[z_{\mathrm{h,0}}\right]=\frac{2\lambda_0\lambda_1}{\lambda_0+\lambda_1} \geqslant 0.5\lambda_1
\end{equation}

对方程~\eqref{eq:GlacierThermal}采用Crank-Nicholson半隐式格式求解,得到以下形式:
\begin{equation}
  \frac{c_i\Delta z_i}{\Delta t}\left(T^{n+1}_i - T^n_i\right)=\alpha \left(F^n_i - F^n_{i-1}\right) + \left(1-\alpha \right) \left(F^{n+1}_i - F^{n+1}_{i-1}\right)
\end{equation}
其中$\alpha = 0.5$为权重因子。将所有雪盖冰川层的能量平衡方程联立,得到三对角矩阵形式的方程组:
\begin{equation}
  r_i = a_i T^{n+1}_{i-1} + b_i T^{n+1}_i + c_i T^{n+1}_{i+1}
\end{equation}
其中$a_i$,$b_i$和$c_i$分别为三对角矩阵中上三角、对角线和下三角位置中的元素。下面分别阐述不同情况下三对角矩阵中系数的具体表达。

(1)对于冰川的中间层(即$snl+1<i<10$),三对角矩阵中的系数表达如下
\begin{equation}
  \begin{aligned}
    a_i &= -\left(1-\alpha \right) \frac{\Delta t}{c_i \Delta z_i} \frac{\lambda \left[z_{\mathrm{h},i-1}\right]}{z_i-z_{i-1}} \\
    b_i &= 1+\left(1-\alpha \right) \frac{\Delta t}{c_i \Delta z_i} \left[\frac{\lambda \left[z_{\mathrm h},i-1\right]}{z_i-z_{i-1}} + \frac{\lambda \left[z_{\mathrm{h},i}\right]}{z_{i+1}-z_i}\right] \\
    c_i &= -\left(1-\alpha \right)\frac{\Delta t}{c_i\Delta z_i}\frac{\lambda \left[z_{\mathrm{h},i}\right]}{z_{i+1}-z_i} \\
    r_i &= T_{i}^{n}+\alpha \frac{\Delta t}{c_{i} \Delta z_{i}} \lambda\left[z_{\mathrm{h},i}\right] \frac{T_{i}^{n}-T_{i+1}^{n}}{z_{i}-z_{i+1}}-\lambda\left[z_{\mathrm{h},i-1}\right] \frac{T_{i-1}^{n}-T_{i}^{n}}{z_{i-1}-z_{i}}
  \end{aligned}
\end{equation}

(2)对于冰川顶层(即$i=snl+1$),其向上的热通量即为大气进入到地表的热通量$h_{\mathrm {s}} $
\begin{equation}
  h^{n+1}_{\mathrm {s}} =-\alpha F^n_{i-1}-\left(1-\alpha\right)F^{n+1}_{i-1}
\end{equation}
此时顶层的能量平衡方程为:
\begin{equation}
  \frac{c_i\Delta z_i}{\Delta t}\left(T^{n+1}_i-T^n_i\right) = h^{n+1}_s+\alpha F^n_i+\left(1-\alpha \right)F^{n+1}_{i-1}
\end{equation}
其中$h^{n+1}_{\mathrm {s}} $取一阶泰勒近似:
\begin{equation}
  h^{n+1}_{\mathrm {s}}  \approx h^n_{\mathrm {s}}  + \frac{\partial h_{\mathrm {s}} }{\partial T_i}\left(T^{n+1}_i-T^n_i\right)
\end{equation}
于是,冰川顶层的三对角矩阵系数即为:
\begin{equation}
  \begin{aligned}
    a_{i} &= 0 \\
    b_{i} &= 1+\frac{\Delta t}{c_{i} \Delta z_{i}}\left[(1-\alpha) \frac{\lambda\left[z_{\mathrm{h},i}\right]}{z_{i+1}-z_{i}}-\frac{\partial h_{\mathrm{s}}}{\partial T_{i}}\right] \\
    c_{i} &= -(1-\alpha) \frac{\Delta t}{c_{i} \Delta z_{i}} \frac{\lambda\left[z_{\mathrm{h},i}\right]}{z_{i+1}-z_{i}} \\
    r_{i} &= T_{i}^{n}+\frac{\Delta t}{c_{i} \Delta z_{i}}\left[h_{\mathrm{s}}^{n}-\frac{\partial h_{\mathrm{s}}}{\partial T_{i}} T_{i}^{n}+\alpha \lambda\left[z_{\mathrm{h},i}\right] \frac{T_{i}^{n}-T_{i+1}^{n}}{z_{i}-z_{i+1}}\right]
  \end{aligned}
\end{equation}

大气进入地表的热通量$h_{\mathrm {s}} $和其偏导可计算为:
\begin{equation}\label{eq:GlacierSrfEnergyBalance}
  h_{\mathrm {s}}  = S_{\mathrm {g}}  + L_{\mathrm {g}}  - H_{\mathrm {g}}  - \lambda E_{\mathrm {g}}  + H_{\mathrm{prcg}}
\end{equation}
\begin{equation}
  \frac{\partial h_{\mathrm {s}} }{\partial T} = \frac{\partial L_{\mathrm {g}} }{\partial T} -\frac{\partial H_{\mathrm {g}} }{\partial T} -\frac{\partial \lambda E_{\mathrm {g}} }{\partial T} +\frac{\partial H_{\mathrm{prcg}}}{\partial T}
\end{equation}
其中$S_{\mathrm {g}} $和$L_{\mathrm {g}} $分别表示地表吸收的净太阳辐射和净长波辐射(\unit{W.m^{-2}}),$H_{\mathrm {g}} $和$E_{\mathrm {g}} $分别表示地表向大气输送的感热通量(\unit{W.m^{-2}})和水汽通量(\unit{kg.m^{-2}}),$H_{\mathrm{prcg}}$表示降水与地表的能量交换
\begin{equation}
  H_{\mathrm{prcg}} = C_{\mathrm{liq}}p_{\mathrm {rain}} \left(T_{\mathrm {p}} -T_{\mathrm {g}} \right) + C_{\mathrm{ice}}p_{\mathrm {snow}} \left(T_{\mathrm {p}} -T_{\mathrm {g}} \right)
\end{equation}
其中$p_{\mathrm {rain}} $和$p_{\mathrm {snow}} $分别表示到达地面的液态降水和固态降水(\unit{mm.H_2O.s^{-1}}),$T_{\mathrm {p}} $表示降水温度(K)。
式~\eqref{eq:GlacierSrfEnergyBalance} 中$\lambda$表示潜热通量系数,用于将水汽通量转换为潜热通量
\begin{equation}
  \lambda = \begin{cases}
    \lambda_{\mathrm {sub}}  &\text{当}\ w_{\mathrm{liq},snl+1}=0\text{ 且}\ w_{\mathrm{ice},snl+1}>0\text{ 时}\\
    \lambda_{\mathrm {vap}}  &\text{当}\ w_{\mathrm{liq},snl+1}>0\text{ 时}
  \end{cases}
\end{equation}
$\lambda_{\mathrm {sub}} $和$\lambda_{\mathrm {vap}} $分别为固态水升华潜热系数和液态水蒸发潜热系数(表~\ref{tab:物理常数})。

对于式~\eqref{eq:GlacierSrfEnergyBalance} 中的净长波辐射$L_{\mathrm {g}} $,其可计算为
\begin{equation}
  L_{\mathrm {g}}  = \varepsilon_{\mathrm {g}}  L ^\downarrow - L_{\mathrm {g}} ^\uparrow
\end{equation}
其中$L^ \downarrow$表示近地面大气下行长波辐射,$L_{\mathrm {g}} ^\uparrow=\varepsilon_{\mathrm {g}} \sigma T^4_{\mathrm {g}} $表示冰川表面发出的上行长波辐射,$\varepsilon_{\mathrm {g}} =0.97$表示冰川表面的长波辐射发射率,$\sigma$表示Stefan-Boltzmann常数(表~\ref{tab:物理常数})。

另外,为改进由于冰川表面温度取为冰川顶层的平均温度带来的缺陷,在求解第一层能量平衡方程时,其厚度$\Delta z_i$调整为:
\begin{equation}
  \Delta z_i = 0.5\left[z_i-z_{\mathrm{h},i-1}+c_{\mathrm {a}} \left(z_{i+1}-z_{\mathrm{h},i-1}\right)\right]
\end{equation}
其中调整参数取为$c_{\mathrm {a}} =0.34$。

(3)对于冰川底层(即$i=10$),假定向下的热通量为0,则能量平衡方程变为:
\begin{equation}
  \frac{c_{i} \Delta z_{i}}{\Delta t}\left(T_{i}^{n+1}-T_{i}^{n}\right)=-\alpha \lambda\left[z_{\mathrm{h},i-1}\right] \frac{T_{i-1}^{n}-T_{i}^{n}}{z_{i-1}-z_{i}}-(1-\alpha) \lambda\left[z_{\mathrm{h},i-1}\right] \frac{T_{i-1}^{n+1}-T_{i}^{n+1}}{z_{i-1}-z_{i}}
\end{equation}
此时的三对角矩阵系数为:
\begin{equation}
  \begin{aligned}
    a_{i} &= -(1-\alpha) \frac{\Delta t}{c_{i} \Delta z_{i}} \frac{\lambda\left[z_{\mathrm{h},i-1}\right]}{z_{i}-z_{i-1}} \\
    b_{i} &= 1+(1-\alpha) \frac{\Delta t}{c_{i} \Delta z_{i}} \frac{\lambda\left[z_{\mathrm{h},i-1}\right]}{z_{i}-z_{i-1}} \\
    c_{i} &= 0 \\
    r_{i} &= T_{i}^{n}-\alpha \frac{\Delta t \lambda\left[z_{\mathrm{h},i-1}\right]}{c_{i} \Delta z_{i}} \frac{T_{i-1}^{n}-T_{i}^{n}}{z_{i-1}-z_{i}}
  \end{aligned}
\end{equation}

于是,通过求解上述的能量平衡方程组,即可计算出下一时刻的雪盖冰川层温度。之后,需再根据水的相态变化对温度进一步调整。冰川温度的相态变化调整与雪盖土壤部分完全一致,读者可参见章节\ref{sec:温度的相态变化调整}。至此,已完成冰川温度计算的全部过程,在得到$T^{n+1}_i$后,冰川表面发出的上行长波辐射$L_{\mathrm {g}} ^\uparrow$、感热通量$H_{\mathrm {g}} $与潜热通量$\lambda E_{\mathrm {g}} $需做一次更新以作为输出的状态变量,其中用于蒸发的水汽不能超过冰川顶层的总水含量$\left(w^{n+1}_{\mathrm{liq},snl+1}+w^{n+1}_{\mathrm{ice},snl+1}\right)/\Delta t$,否则将顶层的总水含量用于蒸发,并将多余的能量误差加到感热通量上。


\section{冰川的水文过程计算方案}
在CoLM的冰川系统中,冰层的厚度与质量始终固定不变,故水文过程主要发生在冰层之上的雪盖层。采用这种处理方法的原因是,冰川通常位于水的冻结温度以下,CoLM假定冰层的融水会滞留在原地等待重新冻结,即使在较为温暖的一些区域,冰层融化的液态水也被认为永久滞留在冰层之中,不与外界进行水通量的交换。

对于仅需考虑的雪盖水文过程,按以下流程进行计算,其中每个部分在前文均有完整介绍,读者可自行翻阅。
\begin{enumerate}
  \item 计算雪盖垂直方向的液态水通量,并更新下一积分步数每一层的液态水含量,由雪盖底层流出的液态水通量则用于地表径流的计算当中(章节~\ref{雪盖的水量平衡});
  \item 考虑积雪的压实过程,更新每一层雪盖的厚度(章节~\ref{雪的压实});
  \item 当雪层发生消融至固态水含量过低或不足规定的最小厚度时,对雪层进行合并(章节~\ref{雪层的合并});当雪层积累至规定的最大厚度时,对雪层进行再分层(章节~\ref{雪层的再分层})。
\end{enumerate}

特别的,若冰川表面无雪盖存在,则冰层顶层(即$i=1$)的液态水和固态水含量根据下式更新:
\begin{equation}
  \begin{aligned}
    w^{n+1}_{\mathrm{liq,1}} &= w^{n}_{\mathrm{liq,1}} + q_{\mathrm{sdew}} \Delta t \\
    w^{n+1}_{\mathrm{ice},i} &= w^{n}_{\mathrm{ice,1}} + \left(q_{\mathrm{frost}}-q_{\mathrm{subl}}\right) \Delta t
  \end{aligned}
\end{equation}
其中$q_{\mathrm{sdew}}$、$q_{\mathrm{frost}}$和$q_{\mathrm{subl}}$分别表示水的凝结、凝华和升华速率(\unit{kg.m^{-2}.s^{-1}}或\unit{mm.H_2O.s^{-1}})。

\chapter{湿地模式}

\section{湿地模式结构}
湿地模式的结构与雪盖土壤层相同,即在CoLM中,湿地由上部的雪层和下部的土壤层组成,具体分层方式见章节~\ref{土壤和积雪的垂直分层},湿地的物理方案遵从雪盖土壤层的计算方案来设定。

\section{湿地表面湍流通量计算方案}
湿地表面湍流通量的计算方案除地表空气比湿计算过程不同,其余过程与雪盖土壤地表的湍流通量计算方案(见章节~\ref{ch:地表湍流通量})一致。

湿地表面可分为被积雪覆盖与未被积雪覆盖两部分。因此在计算湍流通量前,需先计算地表的积雪覆盖比例($f_{\mathrm{sno}}$)及被积雪掩埋的植被占总植被的比例($wt$)。

首先根据~\citet{swenson2012new}提供的方法,计算被积雪覆盖的地表面积比例($f_{\mathrm{sno}}$),其计算包含以下两个过程:

1.在积分开始时,若发生固态降水,则下一时间步数的$f_{\mathrm{sno}}$更新为
\begin{equation}
  f^{n+1}_{\mathrm{sno}}=1-\left[1-\tanh{\left(0.1 p_{\mathrm {i}}  \Delta t\right)}\right]\left(1-f^n_{\mathrm{sno}}\right) \leqslant 1.0
\end{equation}
其中$p_{\mathrm {i}} $表示到达湿地表面的固态降水率(\unit{kg.m^{-2}.s^{-1}}),$\Delta t$为积分的时间步长(s);

2.在水热过程模拟结束后,考虑积雪融化,$f_{\mathrm{sno}}$更新为
\begin{equation}
  f^{n+1}_{\mathrm{sno}}=\tanh \left(\frac{100 z^2_{\mathrm{sno}}}{2.5z_{\mathrm{0m,ice}} W_{\mathrm{sno}}}\right)
\end{equation}
其中$z_{\mathrm{sno}}$为雪盖高度(mm),$W_{\mathrm{sno}}$为雪水当量(mm),$z_{\mathrm{0m,ice}}$表示未被积雪覆盖时湿地的地表粗糙度。\\

当积雪掩埋植被时,被积雪掩埋的植被占总植被的比例($wt$)可通过植被粗糙度($z_{\mathrm{0mv}}$)计算
\begin{equation}
  w_{\mathrm {t}} =\frac{0.1 z_{\mathrm{sno}}}{z_{\mathrm{0mv}}+0.1 z_{\mathrm{sno}}}
\end{equation}
其中$z_{\mathrm{sno}}$表示积雪厚度(m)。CoLM2014及以前版本计算斑块中的有效植被比例$f_{\mathrm{sig}} =
(1 −wt)f_{\mathrm{veg}}$,无植被覆盖比例为 $(1−f_{\mathrm{sig}})$。CoLM2024版本为了考虑与PFT次网格类型的兼容性,同时认为积雪是通过覆盖或掩埋植被叶面积、茎面积进行影响,将$wt$用于修正被积雪掩盖后的SAI,即SAI=TSAI(1 − $wt$),TSAI为植被“真实”茎面积指数。当采用卫星遥感LAI时,由于其数值已经是非积雪覆盖下的绿色叶面部分,故在此不对其进行积雪覆盖调整。\\

\textbf {(1)当湿地表面无植被覆盖或植被积雪掩埋时}\\

当湿地表面无植被覆盖或植被积雪掩埋时,采用无植被覆盖地表湍流通量的计算方案,因此湍流通量只存在于地表和大气之间,动量通量$\tau$、感热通量$H$和水汽通量$E$分别表达为:
\begin{align}
  \tau_{\mathrm {x}}  &= -\rho_{\mathrm{a}} \frac{u_{\mathrm{a}}}{r_{\mathrm{am}}} \\
  \tau_{\mathrm {y}}  &= -\rho_{\mathrm{a}} \frac{v_{\mathrm{a}}}{r_{\mathrm{am}}} \\
  H_{\mathrm {g}}  &= -\rho_{\mathrm{a}} C_{\mathrm{a}} \frac{\theta_{\mathrm{a}}-T_{\mathrm {g}} }{r_{\mathrm{ah}}} \\
  E_{\mathrm {g}}  &= -\rho_{\mathrm{a}} \frac{q_{\mathrm{a}}-q_{\mathrm {g}} }{r_{\mathrm{aw}}}
\end{align}
其中$\rho_{\mathrm{a}}$表示空气密度(\unit{kg.m^{-3}}),$u_{\mathrm{a}}$和$v_{\mathrm{a}}$分别表示纬向风速和经向风速,$C_{\mathrm{a}}$表示空气的比热容(\unit{J.kg^{-1}.K^{-1}}),$\theta_{\mathrm{a}}$表示大气位温(K),$r_{\mathrm{am}}$、$r_{\mathrm{ah}}$和$r_{\mathrm{aw}}$分别表示空气动力学、感热通量和水汽通量的湍流阻抗系数(\unit{s.m^{-1}}),$T_{\mathrm {g}} $表示地表温度(K),当有积雪覆盖时,$T_{\mathrm {g}} $为最上层雪层的温度,否则,$T_{\mathrm {g}} $取为第一层土壤的温度。$q_{\mathrm {g}} $表示地表空气的比湿,取为温度在$T_{\mathrm {g}} $时的饱和比湿,即$q_{\mathrm {g}} =q^{T_{\mathrm {g}} }_{\mathrm{sat}}$。

由于地表无植被覆盖,在计算阻抗系数$r_{\mathrm{am}}$、$r_{\mathrm{ah}}$、$r_{\mathrm{aw}}$时,零平面位移取为$d=0$。动量粗糙度在无积雪覆盖时取为$z_{\mathrm{0m}}=0.01$,有积雪覆盖时取为积雪、裸土覆盖面积加权平均,即$z_{\mathrm{0m}}=0.01 \left(1-f_{\mathrm{sno}}\right)+0.0024 f_{\mathrm{sno}}$。感热和水汽粗糙度取为:
\begin{equation}
  z_{\mathrm{0h}}=z_{\mathrm{0w}}=z_{\mathrm{0m}}\exp{\left[-0.13\left(Re_*\right)^{0.45}\right]}
\end{equation}
其中$Re_*=u_*\cdot z_{\mathrm{0m}}/v$表示粗糙雷诺数,$v= 1.5 \times 10^{-5}$ \unit{m^2.s^{-1}}为空气动力学粘性系数。

基于此,湿地表面湍流通量的具体计算流程如下:
\begin{enumerate}
  \item 给出计算风速$V_{\mathrm {a}} $时$U_{\mathrm {c}} $的初始猜测:
    \begin{equation}
      U_{\mathrm {c}}  = \begin{cases}
        0, &\text{当}\ \theta_{\mathrm{v,atm}}-\theta_{\mathrm{v,s}} \geqslant 0 \text{ 时(稳定条件)} \\
        0.5, &\text{当}\ \theta_{\mathrm{v,atm}}-\theta_{\mathrm{v,s}} < 0 \text{ 时(不稳定条件)}
      \end{cases}
    \end{equation}
  \item 通过$R_{\mathrm{ib}}$给出Monin-Obukhov长度$L$的初始猜测;
  \item 迭代以下过程以获得湿地表面的能量通量:\\
    a. 通过风速、温度和比湿的微风方程积分结果求解$u_*$、$\theta_*$和$q_*$,\\
    b. 更新感热和水汽粗糙度$z_{\mathrm{0h}}$和$z_{\mathrm{0w}}$,\\
    c. 计算虚位温尺度$\theta_{\mathrm{v*}}$,\\
    d. 更新大气风速$V_{\mathrm {a}} $,\\
    e. 计算新一步$L$,\\
    每完成上面的一次迭代过程判断$L$是否改变符号,若符号改变达到四次或以上,视为中性条件,跳出迭代,否则持续迭代直至6次;
  \item 计算湍流阻抗系数$r_{\mathrm{am}}$、$r_{\mathrm{ah}}$和$R_{\mathrm{aw}}$;
  \item 计算动量通量$\tau_{\mathrm {x}} $、$\tau_{\mathrm {y}} $,感热通量$H_{\mathrm {g}} $和水汽通量$E_{\mathrm {g}} $;
  \item 计算\qty{2}{m}气温$T_{\rm 2m}$和比湿$q_{\rm 2m}$。
\end{enumerate}

\textbf {(2)当湿地表面被植被覆盖时}\\

当湿地表面被植被覆盖,且未被积雪掩埋时,采用一维植被湍流交换模型,陆地与大气总湍流输运通量为植被冠层周围空气与大气之间的湍流输送,其动量通量$\tau$、感热通量$H$和水汽通量$E$表达为:
\begin{equation}
  \tau_{\mathrm{x}}=-\rho_{\mathrm{a}} \frac{u_{\mathrm{a}}}{r_{\mathrm{a m}}}
\end{equation}
\begin{equation}
  \tau_{\mathrm{y}}=-\rho_{\mathrm{a}} \frac{v_{\mathrm{a}}}{r_{\mathrm{a m}}}
\end{equation}
\begin{equation}
  H=-\rho_{\mathrm{a}} C_{\mathrm{p a}} \frac{\theta_{\mathrm{a}}-T_{\mathrm{s}}}{r_{\mathrm{a h}}}
\end{equation}
\begin{equation}
  E=-\rho_{\mathrm{a}} \frac{q_{\mathrm{a}}-q_{\mathrm{s}}}{r_{\mathrm{a w}}}
\end{equation}
由于此时湍流通量与叶片温度耦合紧密,故阻抗系数$r_{\mathrm{am}}$、$r_{\mathrm{ah}}$、$r_{\mathrm{aw}}$与植被冠层周围($d+z_{\mathrm{0mx}}$)空气温度$T_{\mathrm {s}} $和比湿$q_{\mathrm {s}} $将随叶片温度一起采用牛顿迭代法进行求解(见章节~\ref{植被叶片温度计算})。其感热与水汽通量求解与一维植被湍流交换模型计算过程完全相同,具体计算过程请读者参见章节~\ref{一维植被湍流交换模型}。

\section{湿地温度计算方案}
湿地垂直层温度的计算方案同样类似于有植被覆盖的雪盖土壤层的垂直层温度计算方案。

对于有植被覆盖部分的湿地,需先计算植被叶片温度(见章节~\ref{植被叶片温度计算}),其后计算雪盖土壤层。无植被覆盖部分则直接计算雪盖土壤层的垂直层温度即可。\\

\textbf {(1)植被叶片温度计算}\\

假设植被冠层的比热容很小,可忽略不计,则叶片能量平衡方程为:
\begin{equation}\label{FT_V2}
  F\left(T_{\mathrm{v}}\right):=S_{\mathrm{v}}+L_{\mathrm{v}}\left(T_{\mathrm{v}}\right)-H_{\mathrm{v}}\left(T_{\mathrm{v}}\right)-\lambda E_{\mathrm{v}}\left(T_{\mathrm{v}}\right)+H_{\mathrm{p r c v}}\left(T_{\mathrm{v}}\right)=0
\end{equation}
其中$S_{\mathrm {v}} $表示叶片吸收的净太阳辐射(见章节~\ref{短波吸收辐射通量}),
$L_{\mathrm {v}} $表示叶片吸收的净长波辐射。$T_{\mathrm {v}} $可通过对方程 (\ref{FT_V2}) 实施牛顿迭代法进行求解,迭代公式为:
\begin{equation}
  \Delta T_{\mathrm{v}}=-\frac{F\left(T_{\mathrm{v}}^{(n)}\right)}{F^{\prime}\left(T_{\mathrm{v}}^{(n)}\right)}
\end{equation}
其中$\Delta T_{\mathrm {v}} =T_{\mathrm {v}} ^{\left(n+1\right)}-T_{\mathrm {v}} ^{\left(n\right)}$,$n$代表迭代次数。
此外,因为植被湍流通量与叶片温度相互耦合,故在温度迭代求解过程中,湍流通量也随之更新。

各项公式求解过程可见章节\ref{植被叶片温度计算},下面给出$T_{\mathrm {v}} $以及植被湍流通量的求解流程:
\begin{enumerate}
  \item 给出植被冠层空气温度和比湿的初始猜测:$T_{\mathrm {s}} =\frac{T_{\mathrm {g}} +\theta_{\mathrm{a}}}{2}$,$q_{\mathrm {s}} =\frac{q_{\mathrm {g}} +q_{\mathrm{a}}}{2}$;
  \item 给出$U_{\mathrm {c}} $的初始猜测如下:\\
    \begin{equation*}
      U_{\mathrm {c}} = \begin{cases}
        0,  & \theta_{\mathrm{v,atm}}-\theta_{\mathrm{v,s}}\geqslant0 \text{ 即稳定条件下;} \\
        0.5, & \theta_{\mathrm{v,atm}}-\theta_{\mathrm{v,s}}<0 \text{ 即不稳定条件下;}
      \end{cases}
    \end{equation*}
  \item 通过$R_{\mathrm{ib}}$给出$L$的初始猜测;
  \item 迭代以下过程以求得$T_{\mathrm {v}} $以及植被湍流通量:\\
    a. 通过风速、温度、比湿的微分方程(M-O相似性理论)积分结果求得$u_\ast$、$\theta_\ast$、$q_\ast$ \\
    b. 计算植被冠层空气与大气之间的阻抗系数$r_{\mathrm{am}}$、$r_{\mathrm{ah}}$、$r_{\mathrm{aw}}$ \\
    c. 计算叶片边界层阻抗$r_{\mathrm {b}} $ \\
    d. 计算植被冠层空气与地表之间的阻抗系数$r_{\mathrm{ah}}^\prime$、$r_{\mathrm{aw}}^\prime$ \\
    e. 计算阳叶、阴叶气孔阻抗$r_{\mathrm{s,sun}}$和$r_{\mathrm{s,sha}}$ \\
    f. 分别计算叶片吸收的长波辐射、感热通量、潜热通量和雨水感热$L_{\mathrm {v}} $、$H_{\mathrm{v}}$、$\lambda E_{\mathrm{v}}$、$H_{\mathrm{prcv}}$ \\
    g. 若前后两次迭代过程中潜热通量的符号发生变化($\lambda E_{\mathrm{v}}^{\left(n\right)}\times\lambda E_{\mathrm{v}}^{\left(n+1\right)}<0$),
    则在该次迭代计算温度时,潜热通量的量级限制为原量级的10\%,由此产生的能量差最后将加到感热通量中 \\
    h. 计算温度变化$\Delta T_{\mathrm {v}} $,并由此更新$T_{\mathrm {v}} ^{\left(n+1\right)}=\Delta T_{\mathrm {v}} ^{\left(n\right)}+T_{\mathrm {v}} ^{\left(n\right)}$。在每次迭代过程中,对于温度的变化作出如下两个限制:
    (1)温度的变化不得超过1 K,若超过,则强制其变化只有1K;
    (2)若本次迭代温度变化的方向与上一次变化的方向相反,则本次温度的变化将取为两次变化的平均值(若$\Delta T_{\mathrm {v}} ^{\left(n-1\right)} \cdot \Delta T_{\mathrm {v}} ^{\left(n\right)}<0$,则$\Delta T_{\mathrm {v}} ^{\left(n\right)}=\left(\Delta T_{\mathrm {v}} ^{\left(n-1\right)}+\Delta T_{\mathrm {v}} ^{\left(n\right)}\right)/2$)\\
    由温度调整所带来的能量平衡误差最后将加到感热通量中\\
    i. 更新饱和比湿$q_{\mathrm{sat}}^{T_{\mathrm {v}} }$及其对$T_{\mathrm {v}} $的变化率 \\
    j. 更新植被冠层空气温度和比湿$T_{\mathrm {s}} $, $q_{\mathrm {s}} $ \\
    k. 更新特征位温$\theta_\ast$和特征比湿$q_\ast$ \\
    l. 更新特征虚位温$\theta_{\mathrm{v\ast}}$ \\
    m. 更新大气风速$V_{\mathrm {a}} \left(U_{\mathrm {c}} \right)$ \\
    n. 计算新一步$L$,并计算$\zeta$,根据稳定性条件限制$\zeta$的取值范围 \\
    o. 根据限制条件后的$\zeta$重新计算$L=\frac{z_{\mathrm{a,m}}-d}{\zeta}$ \\
    p. 判断$L$与上一步迭代相比是否改变符号,若改变符号累计超过4次,则视为中性条件,
    $L$取固定值$L=\frac{z_{\mathrm{a,m}}-d}{-0.01}$,以避免在稳定与不稳定条件之间来回变化。\\
    q. 判断迭代停止条件:若迭代过程中满足下列全部条件或迭代次数已超过40次,则迭代停止
    \begin{equation}
      \begin{array}{l}\max\left( \sqrt{\left[F^{(n+1)}-F^{(n)}\right]^{\ast\ast2}}, \sqrt{\left[F^{(n)}-F^{(n-1)}\right]^{\ast\ast2}} \right) \leqslant 0.1 \\[3.0 ex]
      \max\left( \sqrt{\left(\Delta T_{\mathrm{v}}^{(n)}\right)^{2}}, \sqrt{\left(\Delta T_{\mathrm{v}}^{(n-1)}\right)^{2}} \right) \leqslant 0.01\end{array}
    \end{equation}
    其中$\left[\bullet\right]^{\ast\ast2}$表示各个相同能量项相邻时间步变化量(相减后)的平方和
  \item 由最终叶片温度更新植被表面与植被冠层空气之间的潜热通量,其中蒸发量不得超过植被截水量$W_{\mathrm{can}}$,蒸腾率不得超过最大蒸腾率$ E_{\mathrm{vt,max}}$,若超过则蒸发(腾)率强制取为最大蒸发(腾)率,由此产生的能量差最后将加到感热通量中
  \item 由最终叶片温度更新植被表面与植被冠层空气之间的感热通量以及上述因为潜热与温度的调整导致的能量误差之和
  \item 由最终叶片温度更新植被冠层的雨水感热
  \item 计算总动量通量为
    \begin{equation}
      \begin{aligned}
        \tau_{\mathrm{x}} &=- \rho_{\mathrm{a}} \frac{u_{\mathrm{a}}}{r_{\mathrm{a m}}} \\[1ex]
        \tau_{\mathrm{y}} & =- \rho_{\mathrm{a}} \frac{v_{\mathrm{a}}}{r_{\mathrm{am}}}
      \end{aligned}
    \end{equation}
  \item 计算有植被覆盖下的地面感热通量$H_{\mathrm{g}}$和潜热通量$\lambda E_{\mathrm{g}}$及其对地面温度的变化率,
    并给出地表总感热通量$H_{\mathrm {g}} $和潜热通量$\lambda E_{\mathrm {g}} $随地面温度变化率
  \item 计算植被覆盖下地表吸收的下行长波辐射$L_{\mathrm{v}}^\downarrow$和返回大气的上行长波辐射$L_{\mathrm {v}}  ^\uparrow$:
    \begin{equation}
      L_{\mathrm{v}} ^\downarrow =  \tau_{\mathrm{v}} L ^\downarrow+\varepsilon_{\mathrm{v}}\sigma \left (T_{\mathrm{v}}^{(n - 1)}\right)^3\left( T_{\mathrm{v}}^{(n - 1)} + 4\Delta T_{\mathrm{v}}^{(n - 1)} \right)\\
    \end{equation}
    \begin{equation}
      \begin{aligned}
        L_{\mathrm{v}}^ \uparrow &=  \tau_{\mathrm{v}} \varepsilon_{\mathrm{g}} \sigma T_{\mathrm{g}}^{4}+ \varepsilon_{\mathrm{v}}\sigma \left ( T_{\mathrm{v}}^{(n - 1)}\right )^3\left( T_{\mathrm{v}}^{(n - 1)} + 4\Delta T_{\mathrm{v}}^{(n - 1)} \right) \\[1ex]
        &\mathrel{\phantom{=}} + \left ( 1- \varepsilon_{\mathrm{g}} \right)\mu_{\mathrm{v}}^2 L ^\downarrow + \left ( 1- \varepsilon_{\mathrm{g}} \right) \mu_{\mathrm{v}} \varepsilon_{\mathrm{v}} \sigma \left (T_{\mathrm{v}}^{(n - 1)}\right) ^4 \\[1ex]
        &\mathrel{\phantom{=}} + 4 \left ( 1- \varepsilon_{\mathrm{g}} \right) \mu_{\mathrm{v}} \varepsilon_{\mathrm{v}} \sigma \left (T_{\mathrm{v}}^{(n - 1)}\right)^3 \Delta T_{\mathrm{v}}^{(n - 1)}
      \end{aligned}
    \end{equation}
  \item 计算地表2 m温度与比湿$T_{\rm 2m}$、$q_{\rm 2m}$。
\end{enumerate}

\textbf {(2)雪盖土壤热力过程}\\

假设湿地无水平物质能量交换,则垂直方向上的一维能量平衡方程如下:
\begin{equation}\label{eq:WetlandThermalCons1}
  c \frac{\partial T}{\partial t}=-\frac{\partial F}{\partial z},  \quad F=-\lambda \frac{\partial T}{\partial z}
\end{equation}
其中$c$表示雪盖或湿地土壤的体积热容(\unit{J.m^{-3}.K^{-1}}),$T$表示雪盖土壤温度(K),$t$表示时间(s),$z$表示雪盖高度或土壤深度,$F$表示垂直方向的热传导通量(向上为正方向,\unit{W.m^{-2}}),$\lambda$表示热传导率(\unit{W.m^{-1}.K^{-1}})。

对于体积热容,雪盖体积热容量由液态水和固态水的热容量基于体积百分比加权平均求得
\begin{equation}
  c_{i}=\frac{w_{\mathrm{ice},i}}{\Delta z_{i}} C_{\mathrm{ice}}+\frac{w_{\mathrm{liq},i}}{\Delta z_{i}} C_{\mathrm{liq}}
\end{equation}
其中,$w_{\mathrm{ice},i}$和$w_{\mathrm{liq},i}$分别表示第$i$层固态水含量和液态水含量(\unit{kg.m^{-2}}),$C_{\mathrm{ice}}$和$C_{\mathrm{liq}}$分别表示固态水和液态水的体积热容量(见~\ref{tab:物理常数})。

土壤相比雪盖多了固体土壤部分,体积热容量由每层固体土壤、液态水和固态水的体积热容根据各自体积百分比加权得到,即
\begin{equation}
  c_{i}=c_{\mathrm{s},i}\left(1-\theta_{\mathrm{s},i}\right)+\frac{w_{\mathrm{ice},i}}{\Delta z_{i}} C_{\mathrm{ice}}+\frac{w_{\mathrm{liq},i}}{\Delta z_{i}} C_{\mathrm{p l}}
\end{equation}
其中,$c_{\mathrm{s},i}$表示第 i 层的固体土壤体积热容量,由地表参数数据集提供。特别地,若此时无雪盖分层但雪水当量$W_{\mathrm{sno}}>0$,则将这部分热容量考虑为固态水的热容量加到土壤顶层中,土壤顶层(即编号$i=1$)热容量重新计算为:
\begin{equation}
  c_{1}=c_{\mathrm{s,1}}\left(1-\theta_{\mathrm{s, 1}}\right)+\frac{w_{\mathrm{ice, 1}}}{\Delta z_{1}} C_{\mathrm{ice}}+\frac{w_{\mathrm{liq,1}}}{\Delta z_{1}} C_{\mathrm{liq}}+\frac{W_{\mathrm{sno}}}{\Delta z_{1}} C_{\mathrm{ice}}
\end{equation}
固体土壤热容量则由土壤含量的各项体积百分比加权平均得到。其中有机质土壤和砾石的体积热容量分别取值为$2.51 \times 10^{6}$ \unit{J.m^{−3}.K^{−1}} 和 $2.35 \times 10^{6}$ \unit{J.m^{−3}.K^{−1}},矿物质土壤的体积热容量(\unit{J.m^{−3}.K^{−1}})由如下方案给出:
\begin{equation}
  c_{\mathrm{minerals}}=\frac{2.128\times\%sand+2.385\times\%clay}{\%sand+\%clay}\times10^6
\end{equation}
其中$\%sand$和$\%clay$表示沙土和黏土的质量百分比。

对方程~\eqref{eq:WetlandThermalCons1}进行离散(见章节~\ref{温度求解的数值格式}),则第$i$层雪盖土壤层的能量平衡方程可表达为:
\begin{equation}\label{eq:WetlandThermal1}
  \frac{c_i \Delta z_i}{\Delta t} \left(T^{n+1}_i - T^n_i\right) = F_i - F_{i-1}
\end{equation}
其中$\Delta t$表示积分时间步长,$n$表示时间步数,$F_i$表示第$i+1$层传导到第$i$层的热通量,其离散形式为:
\begin{equation}
  F_i = \lambda \left[z_{\mathrm{h},i}\right] \frac{T_i-T_{i+1}}{z_i-z_{i+1}}
\end{equation}
$\lambda\left[z_{\mathrm{h},i}\right]$表示第$i+1$层和第$i$层交界面处的热导率:
\begin{equation}
  \lambda \left[z_{\mathrm{h},i}\right] = \begin{cases}
    \frac{\lambda_i\lambda_{i+1}\left(z_i-z_{i-1}\right)}{\lambda_i\left(z_{\mathrm{h},i}-z_{i+1}\right)+\lambda_{i+1}\left(z_i-z_{\mathrm{h},i}\right)}  &\text{对于}\ i=snl+1,\ \ldots,\ 9 \\
    0 &\text{对于}\ i=10
  \end{cases}
\end{equation}
特别的,对于雪盖与土壤的交界面,为防止最下层雪层厚度过大导致$\lambda\left[z_{\mathrm{h},i}\right]$计算不准,当$i=0$且$z_{i+1}-z_{\mathrm{h},i}<z_{\mathrm{h},i}-z_i$时,该处的$\lambda\left[z_{\mathrm{h,0}}\right]$重新计算为:
\begin{equation}
  \lambda\left[z_{\mathrm{h,0}}\right]=\frac{2\lambda_0\lambda_1}{\lambda_0+\lambda_1} \geqslant 0.5\lambda_1
\end{equation}

对方程~\eqref{eq:WetlandThermal1}采用Crank-Nicholson半隐式格式求解,得到以下形式:
\begin{equation}
  \frac{c_i\Delta z_i}{\Delta t}\left(T^{n+1}_i - T^n_i\right)=\alpha \left(F^n_i - F^n_{i-1}\right) + \left(1-\alpha \right) \left(F^{n+1}_i - F^{n+1}_{i-1}\right)
\end{equation}
其中$\alpha = 0.5$为权重因子。将所有雪盖土壤层的能量平衡方程联立,得到三对角矩阵形式的方程组:
\begin{equation}
  r_i = a_i T^{n+1}_{i-1} + b_i T^{n+1}_i + c_i T^{n+1}_{i+1}
\end{equation}
其中$a_i$,$b_i$和$c_i$分别为三对角矩阵中上三角、对角线和下三角位置中的元素。下面分别阐述不同情况下三对角矩阵中系数的具体表达。

(1)对于雪盖土壤的中间层(即$snl+1<i<10$),三对角矩阵中的系数表达如下
\begin{equation}
  \begin{aligned}
    a_i &= -\left(1-\alpha \right) \frac{\Delta t}{c_i \Delta z_i} \frac{\lambda \left[z_{\mathrm{h},i-1}\right]}{z_i-z_{i-1}} \\
    b_i &= 1+\left(1-\alpha \right) \frac{\Delta t}{c_i \Delta z_i} \left[\frac{\lambda \left[z_{\mathrm{h}},i-1\right]}{z_i-z_{i-1}} + \frac{\lambda \left[z_{\mathrm{h},i}\right]}{z_{i+1}-z_i}\right] \\
    c_i &= -\left(1-\alpha \right)\frac{\Delta t}{c_i\Delta z_i}\frac{\lambda \left[z_{\mathrm{h},i}\right]}{z_{i+1}-z_i} \\
    r_i &= T_{i}^{n}+\alpha \frac{\Delta t}{c_{i} \Delta z_{i}} \lambda\left[z_{\mathrm{h},i}\right] \frac{T_{i}^{n}-T_{i+1}^{n}}{z_{i}-z_{i+1}}-\lambda\left[z_{\mathrm{h},i-1}\right] \frac{T_{i-1}^{n}-T_{i}^{n}}{z_{i-1}-z_{i}}
  \end{aligned}
\end{equation}

(2)对于雪盖土壤顶层(即$i=snl+1$),其向上的热通量即为大气进入到地表的热通量$h_{\mathrm {s}} $
\begin{equation}
  h^{n+1}_{\mathrm {s}} =-\alpha F^n_{i-1}-\left(1-\alpha\right)F^{n+1}_{i-1}
\end{equation}
此时顶层的能量平衡方程为:
\begin{equation}
  \frac{c_i\Delta z_i}{\Delta t}\left(T^{n+1}_i-T^n_i\right) = h^{n+1}_{\mathrm {s}} +\alpha F^n_i+\left(1-\alpha \right)F^{n+1}_{i-1}
\end{equation}
其中$h^{n+1}_{\mathrm {s}} $取一阶泰勒近似:
\begin{equation}
  h^{n+1}_{\mathrm {s}}  \approx h^n_{\mathrm {s}}  + \frac{\partial h_{\mathrm {s}} }{\partial T_i}\left(T^{n+1}_i-T^n_i\right)
\end{equation}
于是,雪盖土壤顶层的三对角矩阵系数即为:
\begin{equation}
  \begin{aligned}
    a_{i} &= 0 \\
    b_{i} &= 1+\frac{\Delta t}{c_{i} \Delta z_{i}}\left[(1-\alpha) \frac{\lambda\left[z_{\mathrm{h},i}\right]}{z_{i+1}-z_{i}}-\frac{\partial h_{\mathrm{s}}}{\partial T_{i}}\right] \\
    c_{i} &= -(1-\alpha) \frac{\Delta t}{c_{i} \Delta z_{i}} \frac{\lambda\left[z_{\mathrm{h},i}\right]}{z_{i+1}-z_{i}} \\
    r_{i} &= T_{i}^{n}+\frac{\Delta t}{c_{i} \Delta z_{i}}\left[h_{\mathrm{s}}^{n}-\frac{\partial h_{\mathrm{s}}}{\partial T_{i}} T_{i}^{n}+\alpha \lambda\left[z_{\mathrm{h},i}\right] \frac{T_{i}^{n}-T_{i+1}^{n}}{z_{i}-z_{i+1}}\right]
  \end{aligned}
\end{equation}

大气进入地表的热通量$h_{\mathrm {s}} $和其偏导可计算为:
\begin{equation}\label{eq:WetlandSrfEnergyBalance1}
  h_{\mathrm {s}}  = S_{\mathrm {g}}  + L_{\mathrm {g}}  - H_{\mathrm {g}}  - \lambda E_{\mathrm {g}}  + H_{\mathrm{prcg}}
\end{equation}
\begin{equation}
  \frac{\partial h_{\mathrm {s}} }{\partial T} = \frac{\partial L_{\mathrm {g}} }{\partial T} -\frac{\partial H_{\mathrm {g}} }{\partial T} -\frac{\partial \lambda E_{\mathrm {g}} }{\partial T} +\frac{\partial H_{\mathrm{prcg}}}{\partial T}
\end{equation}
其中$S_{\mathrm {g}} $和$L_{\mathrm {g}} $分别表示地表吸收的净太阳辐射和净长波辐射(\unit{W.m^{-2}}),$H_{\mathrm {g}} $和$E_{\mathrm {g}} $分别表示地表向大气输送的感热通量(\unit{W.m^{-2}})和水汽通量(\unit{kg.m^{-2}}),$H_{\mathrm{prcg}}$表示降水与地表的能量交换
\begin{equation}
  H_{\mathrm{prcg}} = C_{\mathrm{liq}}p_{\mathrm {l}} \left(T_{\mathrm {p}} -T_{\mathrm {g}} \right) + C_{\mathrm{ice}}p_{\mathrm {i}} \left(T_{\mathrm {p}} -T_{\mathrm {g}} \right)
\end{equation}
其中$p_{\mathrm {l}} $和$p_{\mathrm {i}} $分别表示到达地面的液态降水和固态降水(\unit{mm.H_2O.s^{-1}}),$T_{\mathrm {p}} $表示降水温度(K)。
式~\eqref{eq:WetlandSrfEnergyBalance1}中$\lambda$表示潜热通量系数,用于将水汽通量转换为潜热通量
\begin{equation}
  \lambda = \begin{cases}
    \lambda_{\mathrm {sub}}  &\text{当}\ w_{\mathrm{liq},snl+1}=0\text{ 且}\ w_{\mathrm{ice},snl+1}>0\text{ 时}\\
    \lambda_{\mathrm {vap}}  &\text{当}\ w_{\mathrm{liq},snl+1}>0\text{ 时}
  \end{cases}
\end{equation}
$\lambda_{\mathrm {sub}} $和$\lambda_{\mathrm {vap}} $分别为固态水升华潜热系数和液态水蒸发潜热系数(表A.1)。

对于式~\eqref{eq:WetlandSrfEnergyBalance1}中的净长波辐射$L_{\mathrm {g}} $,其可计算为
\begin{equation}
  L_{\mathrm {g}}  = \varepsilon_{\mathrm {g}}  L_{\mathrm{bg}}\downarrow - L_{\mathrm {g}} \uparrow
\end{equation}
其中$L_{\mathrm{bg}}\downarrow$表示近地面大气下行长波辐射,$L_{\mathrm {g}} \uparrow=\varepsilon_{\mathrm {g}} \sigma T^4_{\mathrm {g}} $表示湿地表面发出的上行长波辐射,$\varepsilon_{\mathrm {g}} =0.96$表示湿地表面的长波辐射发射率,$\sigma$表示Stefan-Boltzmann常数(表A.1)。

另外,为改进由于湿地表面温度取为第一层雪盖或土壤的平均温度带来的缺陷,在求解第一层能量平衡方程时,其厚度$\Delta z_i$调整为:
\begin{equation}
  \Delta z_i = 0.5\left[z_i-z_{\mathrm{h},i-1}+c_{\mathrm {a}} \left(z_{i+1}-z_{\mathrm{h},i-1}\right)\right]
\end{equation}
其中调整参数取为$c_{\mathrm {a}} =0.34$。

(3)对于雪盖土壤底层(即$i=10$),假定向下的热通量为0,则能量平衡方程变为:
\begin{equation}
  \frac{c_{i} \Delta z_{i}}{\Delta t}\left(T_{i}^{n+1}-T_{i}^{n}\right)=-\alpha \lambda\left[z_{\mathrm{h},i-1}\right] \frac{T_{i-1}^{n}-T_{i}^{n}}{z_{i-1}-z_{i}}-(1-\alpha) \lambda\left[z_{\mathrm{h},i-1}\right] \frac{T_{i-1}^{n+1}-T_{i}^{n+1}}{z_{i-1}-z_{i}}
\end{equation}
此时的三对角矩阵系数为:
\begin{equation}
  \begin{aligned}
    a_{i} &= -(1-\alpha) \frac{\Delta t}{c_{i} \Delta z_{i}} \frac{\lambda\left[z_{\mathrm{h},i-1}\right]}{z_{i}-z_{i-1}} \\
    b_{i} &= 1+(1-\alpha) \frac{\Delta t}{c_{i} \Delta z_{i}} \frac{\lambda\left[z_{\mathrm{h},i-1}\right]}{z_{i}-z_{i-1}} \\
    c_{i} &= 0 \\
    r_{i} &= T_{i}^{n}-\alpha \frac{\Delta t \lambda\left[z_{\mathrm{h},i-1}\right]}{c_{i} \Delta z_{i}} \frac{T_{i-1}^{n}-T_{i}^{n}}{z_{i-1}-z_{i}}
  \end{aligned}
\end{equation}

于是,通过求解上述的能量平衡方程组,即可计算出下一时刻的雪盖土壤层温度。其后温度相态变化调整与雪盖土壤完全一致,详见章节~\ref{sec:温度的相态变化调整}。


\section{湿地的水文过程计算方案}
湿地的水文过程计算方案同样类似于雪盖土壤层,区别在于湿地的土壤水默认为饱和状态。故水文过程主要发生在土壤层之上的雪盖层和土壤层顶层。

对于雪盖的水文过程,其计算方案除雪层的建立外与章节~\ref{积雪和土壤中水分的垂直运动}基本一致。而在土壤层的水文过程中,同样认为土壤中含有饱和土壤水,采用CoLM2014中的方案和用可变饱和流数值算法求解Richards方程时有少许不同。

特别地,若湿地地表产生降雪,且地表温度大于冻结温度,此时模式不计算降雪过程,如果开启可变饱和流方案,则此时降雪累计到湿地水量中
\begin{equation}
  W_{\mathrm{wet}}^{n+1}=W_{\mathrm{wet}}^{n}+W_{\mathrm{sno}}
\end{equation}
其中$W_{\mathrm{sno}}$为雪水当量,$W_{\mathrm{wet}}$表示湿地中的水量。\\

\textbf {(1)雪盖水文过程}\\

\begin{enumerate}
  \item 当固态降水$p_{\mathrm {i}} $发生且湿地地表温度小于冻结温度时,雪盖高度开始累计。若雪盖高度$z_{\mathrm{sno}}$ 大于0.01m且此时尚无雪盖分层,则将在模拟开始时创建一个新的雪层(章节~\ref{sec:雪层的建立});
  \item 雪盖建立后,根据雪盖的液态水质量守恒方程
    \begin{equation}
      \frac{\partial w_{\mathrm{liq},i}}{\partial t}=\left(q_{\mathrm{liq},i-1}-q_{\mathrm{liq},i}\right)+\frac{{\left(\Delta w_{\mathrm{liq},i}\right)}_{\mathrm {p}} }{\Delta t}
    \end{equation}
    计算每一层的液态水通量,并更新下一积分步长的液态水含量,由雪盖底层流出的液态水通量则用于地表径流的计算当中(章节~\ref{雪盖的水量平衡});
  \item 考虑积雪的压实过程,更新每一层雪盖的厚度(章节~\ref{雪的压实});
  \item 当某个雪层发生消融至不足规定的最小厚度,或积累至规定的最大厚度时,对雪层进行合并(章节~\ref{雪层的合并})或再分层(章节~\ref{雪层的再分层})。
\end{enumerate}

\textbf {(2)土壤水文过程-2014版Richards方程求解方案}\\

采用2014版CoLM中的方案对Richards方程求解时,模式网格均视为地表水饱和区域,其土壤层均视为饱和含水层($\theta_{\mathrm{i}}=\theta_{\mathrm{s}}$),无水分垂直运动,无液态水入渗($Q_{\mathrm{infl}}=0$),无地下水补给($q_{\mathrm{charge}}=0$),地表水全部化作径流流走($r_{\mathrm{surface}}=G_{\mathrm{water}}$),地下水总水量设为阈值($w_{\mathrm {a}} =4800$),地下水水位与地表齐平($z_{\mathrm{wt}}=0$)
\begin{equation}
  \begin{aligned}
    &\theta_{i} &&= &\theta_{\mathrm{s}}& \\
    &q_{\mathrm{charge}} &&=&0 &\\
    &w_{\mathrm {a}}  &&=&4800& \\
    &z_{\mathrm{wt}} &&=&0& \\
    &Q_{\mathrm{infl}} &&=&0& \\
    &r_{\mathrm{surface}} &&=&G_{\mathrm{water}}& \\
  \end{aligned}
\end{equation}
其中$\theta_{i}$为土壤体积含水量,$\theta_{\mathrm{s}}$为饱和土壤体积含水量,$q_{\mathrm{charge}}$土壤层底部水流通量,$w_{\mathrm{a}}$为含水层总水量,$z_{\mathrm{wt}}$为地下水位深度,$Q_{\mathrm{infl}}$为入渗到土壤中的水分,$r_{\mathrm{surface}}$为地表径流,$G_{\mathrm{water}}$为表面水分输入。\\

\textbf {(3)土壤水文过程-可变饱和流数值算法}\\

采用可变饱和流数值算法对Richards方程求解时,为表示湿地整体的水资源变化,湿地水量$W_{\mathrm{wet}}$也是算法中的预报变量,用于表示湿地中的表面水量,在湿地过程计算之前,对湿地水量进行更新:

存在雪层时为,凝结通量、冻结通量和升华通量在雪层顶层计算
\begin{equation}
  W_{\mathrm{wet}}^{n+1}-W_{\mathrm{wet}}^{n}=W_{\mathrm{srf}}+w_{\mathrm{a}}+\left(G_{\mathrm{water}}-E_{\mathrm{tr}}\right)*{\Delta t}
\end{equation}

不存在雪层时为
\begin{equation}
  W_{\mathrm{wet}}^{n+1}-W_{\mathrm{wet}}^{n}=W_{\mathrm{srf}}+w_{\mathrm{a}}+\left(G_{\mathrm{water}}-E_{\mathrm{tr}}+q_{\mathrm{sdew}}+q_{\mathrm{fros}}-q_{\mathrm{subl}}\right)*{\Delta t}
\end{equation}
其中$W_{\mathrm{wet}}$表示湿地中的水量,$W_{\mathrm{srf}}$为表面水深度,$w_{\mathrm{a}}$为含水层总水量,$G_{\mathrm{water}}$为表面水分输入,$E_{\mathrm{tr}}$为实际蒸腾通量,$q_{\mathrm{sdew}}$为凝结通量,$q_{\mathrm{fros}}$为冻结通量,$q_{\mathrm{subl}}$为升华通量,${\Delta t}$为积分时间步长。

可变饱和流方案中$w_{\mathrm {a}} $与2014版方案中$w_{\mathrm {a}} $同样作为表示地下水总量的状态变量,但含义不同。此处$w_{\mathrm {a}} $更多作为土壤水亏缺值存在。只有当地下水位处于土壤水计算区域之下时,才会使用预报变量$w_{\mathrm {a}} $来表示计算区域之下蓄水层的蓄水状态。而湿地中默认地下水位处于土壤水计算区域之下。$w_{\mathrm {a}} $定义为
\begin{equation}
  w_{\mathrm{a}}=-\int_{z_{\mathrm{b t m}}}^{z_{\mathrm{w t}}}\left(\theta_{\mathrm{s}}-\theta_{i}\right){\mathrm d} z
\end{equation}
$w_{\mathrm {a}} $的绝对值为土壤水计算区域之下空气的体积百分比,负号的含义是计算区域之下液态水相对于饱和状态是亏缺的,$w_{\mathrm {a}} $的最大值为0(表示土壤水饱和)。

随后对每层土壤层含水量进行计算,
\begin{equation}
  \theta_{i}^{n+1}=\begin{cases}
    \theta_{i}^{n} &\qquad \quad \qquad \quad \;\text{当}\ \theta_{i}^{n} \leqslant \theta_{\mathrm{s}} \ \text{时} \\
    \theta_{\mathrm{s}}     &\qquad \quad \qquad \quad \;\text{当}\ \theta_{i}^{n} > \theta_{\mathrm{s}} \ \text{时} \\
  \end{cases}
\end{equation}
其中$\theta_{i}$为第$i$层土壤体积含水量,$\theta_{\mathrm{s}}$为饱和土壤体积含水量。

湿地蓄水状态按湿地水量($W_{\mathrm{wet}}$)和湿地水量阈值($W_{\mathrm{wetmax}}$)的比较分三种情况给定状态变量:

\begin{enumerate}
  \item 若此时湿地地表水饱和($W_{\mathrm{wetmax}} \leq W_{\mathrm{wet}}$),则使用表面水量($W_{\mathrm{srf}}$)储存湿地含水量超出阈值部分($W_{\mathrm{wet}}-W_{\mathrm{wetmax}}$),湿地水量($W_{\mathrm{wet}}$)则等于饱和值,此时土壤水饱和($w_{\mathrm{a}}=0$)
% \begin{equation}
% \begin{aligned}
% W_{srf}&=W_{wet}-W_{wetmax} \\
% W_{wet}&=W_{wetmax} \\
% w_{a}&=0 \\
% \end{aligned}
% \end{equation}

  \item 若湿地地表水未饱和($0 \leq W_{\mathrm{wet}} < W_{\mathrm{wetmax}}$),则表面水量($W_{\mathrm{srf}}$)为0,土壤水饱和($w_{\mathrm{a}}=0$)
% \begin{equation}
% \begin{aligned}
% W_{srf}&=0 \\
% W_{wet}&=W_{wet} \\
% w_{a}&=0 \\
% \end{aligned}
% \end{equation}

  \item 若计算中湿地地表缺水($W_{\mathrm{wet}} < 0$),则湿地土壤水未饱和($w_{\mathrm{a}}=W_{\mathrm{wet}}$)
% \begin{equation}
% \begin{aligned}
% W_{srf}&=0 \\
% W_{wet}&=0 \\
% w_{a}&=W_{wet} \\
% \end{aligned}
% \end{equation}
\end{enumerate}

径流计算则通过设置侧向流模块计算。若开启侧向流模块(\texttt{\#define LATERAL\_FLOW}),则径流计算方案见章节~\ref{ch:侧向流模拟}。

若未开启侧向流模块(\texttt{\#undef LATERAL\_FLOW}),则简单计算径流在湿地为地表水饱和区($W_{\mathrm{wetmax}} \leq W_{\mathrm{wet}}$)时
\begin{equation}
  \begin{aligned}
    &r_{\mathrm{srf}}=\frac{\left(W_{\mathrm{srf}}-h_{\mathrm{pond}}\right)}{\Delta t} \\
    &W_{\mathrm{srf}}=h_{\mathrm{pond}}
  \end{aligned}
\end{equation}
其中$r_{\mathrm{srf}}$为总的地表径流,$h_{\mathrm{pond}}$为地表积水深度。


\part{生物地球化学循环过程}{Biogeochemical Cycles}\label{part:BGC}
%\epart{Biogeochemical Cycles}
\chapter{碳氮库结构}\label{碳氮库结构}
%\addcontentsline{toc}{chapter}{碳氮库结构}

%\begin{碳氮库结构}
CoLM生物地球化学循环模块模拟陆地生态系统碳氮元素储量的动态变化,随着大气$\rm CO_2$浓度升高,
陆地生态系统碳储量的动态模拟反映出陆地生态系统的累积固碳状况。
由于植被和土壤均具有较为稳定的碳氮比属性,氮元素储量的动态模拟,量化了生态系统固碳的氮限制。
碳氮储量在模型中被分为了21个植被碳库、22个植被氮库、7个土壤凋落物碳库和8个土壤凋落物氮库。
运用箱式模型和碳氮平衡方程,碳氮元素在生态系统的传输网络得以刻画,每个碳氮库的动态变化得以模拟。


根据植被次网格结构,植被碳氮库对土壤碳氮库的共享形式存在差异。
LCT次网格结构中,不存在植物类型间土壤无机氮的竞争,
单个植被类型的一套植被碳氮库(21个植被碳库+ 22个植被氮库)对应一套土壤凋落物碳氮库(7个土壤凋落物碳库+ 8个土壤凋落物氮库)。
PFT次网格结构中,存在植被类型间的土壤无机氮竞争,多个植被功能型的多套植被氮碳库(n个植被功能型×(21个植被碳库+ 22个植被氮库))
共享一套土壤凋落物碳氮库(7个土壤凋落物碳库+ 8个土壤凋落物氮库)。


\section{植被碳氮库结构}\label{植被碳氮库结构}
植被碳库包括叶、细根、活茎、死茎、活粗根和死粗根6个植被营养器官,在接下来章节的公式中,我们用$leaf$, $foot$, $livestem$, $deadstem$, $livecroot$和$deadcroot$代表这6个营养器官。每个植被营养器官包含组织结构库、存储库和传输库3个子库,公式中我们用$disp$, $stor$和$xfer$表示。
因此,每套植被类型有总共18个植被碳库。其中,组织库代表植被营养器官的主要结构组织部分的碳储量,
是植被碳库的主要组成部分;存储碳库和传输碳库分别是长期和短期的非结构碳库,为落叶植被萌发初期提供初始碳,
初始碳供用可以保证植被生长出足够的叶,维持生长季的光合作用。植被氮库除了包含与碳库相对应的18个营养器官氮库之外,
还包含1个再利用氮库,公式中我们用$retran$代表。以刻画植被机体在凋落前回收利用氮的生理机能,因此每套植被类型有总共19个植被氮库。
当作物模式打开时,额外增加谷粒器官的组织库、存储库和传输库。因此,总共21个植被碳库和22个植被氮库。
通过对光合作用碳输入的限制,每个植被的碳库和氮库之间具有相对稳定的碳氮比,详见章节~\ref{植被土壤的氮竞争} 植被土壤的氮竞争。

植被碳库之间或植被氮库之间存在复杂的元素循环网络,见图 \ref{fig:CoLM植被碳氮循环网络示意图}。
植被从光合作用中得到碳,并分配到不同器官。存储库、传输库和组织库的元素循环、
叶和活根的胁迫凋落和季节凋落将在章节 \ref{植被物候过程的耦合预报方案} 植被的物候过程中介绍。
活茎和活粗根每年将有固定比例的碳分别转至死茎和死粗根的碳库中。
死茎和死粗根的周转仅存在于植被的自然死亡或火灾中。
{
\begin{figure}[htbp]
\centering
\includegraphics{Figures/碳氮库结构/CoLM植被碳氮循环网络示意图.png}
\caption{CoLM植被碳氮循环网络示意图 \citep{lu2020full}}
\label{fig:CoLM植被碳氮循环网络示意图}
\end{figure}
}

\section{土壤凋落物碳氮库结构}\label{土壤凋落物碳氮库结构}
土壤凋落物的碳储存同样动态模拟,被按照成分和周转时间的长短进行分类,土壤凋落物碳氮库均具有垂直分层结构的刻画。
\subsection{碳氮库分类}\label{碳氮库分类}
土壤凋落物的碳储分为7个有机碳库:代谢凋落物、纤维素凋落物、木质素凋落物、粗木质残体、快速土壤周转库、
慢速土壤周转库和惰性土壤周转库,在公式中我们分别用$met$, $cel$, $lig$, $cwd$, $fast$, $slow$, $pass$来代表。凋落物和快速土壤周转库的周转时间都在几个月到几年之间,粗木质残体、
慢速土壤周转库和惰性土壤周转库的周转时间范围分布可以从几年到上千年,表现出极大的土壤异质性。
氮存储除了以上7个有机库外,还包括1个无机氮库,在公式中用$nmin$表示。由于植被根系吸收只能利用无机氮,无机氮库是连接土壤和植被的枢纽,
土壤和植被对无机氮的竞争将在章节 \ref{植被土壤的氮竞争} 中介绍。


通过从植被库中获取凋落物,地下元素循环从凋落物到土壤存在较为复杂的元素循环网络,
见图 \ref{fig:CoLM土壤凋落物碳氮循环网络示意图}。土壤碳库和土壤氮库之间在循环过程中保持相对稳定的碳氮比,凋落物的碳氮比大于土壤库的碳氮比,
因此缺氮会造成凋落物的分解速率下降,详见章节 \ref{土壤分解的氮限制} 氮限制因子。
{
\begin{figure}[htbp]
\centering
\includegraphics{Figures/碳氮库结构/CoLM土壤凋落物碳氮循环网络示意图.png}
\caption{CoLM土壤凋落物碳氮循环网络示意图 \citep{huang2018matrix}  }
\label{fig:CoLM土壤凋落物碳氮循环网络示意图}
\end{figure}
}
\subsection{垂直分层结构}\label{垂直分层结构}
由于土壤水热条件的垂直差异,土壤碳氮存储垂直分布模拟也极其重要,且与土壤水热条件的垂直分布特点联系紧密。
CoLM模拟土壤和凋落物的碳氮库垂直结构,与土壤水热模拟保持一致,土壤和凋落物的碳氮库垂直方向上也分为10层。
每层土壤和凋落物分别具有不同的输入(叶片和根系的凋落输入)和输出(分解呼吸)。
同时,除了粗木质残体以外的6个有机库均存在碳氮有机物的垂直扩散混合过程,体现了微生物的作用,详见章节 \ref{土壤凋落物的垂直传输} 土壤凋落物的垂直传输过程。

\chapter{植被生物地球化学循环过程}\label{植被生物地球化学循环过程}
%\addcontentsline{toc}{chapter}{植被生物地球化学循环过程}

%\begin{植被生物地球化学循环过程}
CoLM植被生物地球化学循环模块存在复杂的碳氮循环网络,植物生理和物候等过程是量化不同碳氮库相互转换和传输的关键。
光合作用是植被生物地球化学循环的初始输入。光合作用的碳输入扣除植被的自养呼吸得到净初级生产力将被分配到植被不同营养器官中,
植被的自养呼吸和碳氮分配是其中的关键过程。同时,植被生长存在季节变化特征,特别是落叶植被功能型,
物候过程影响植被生物地球化学循环的季节性模拟。物候过程同样模拟叶和细根的凋落过程,
植被碳氮库的周转由物候过程和植被自然死亡过程共同控制。
经过物候过程和植被自然死亡过程,植被凋落物进入土壤进一步进行地下生物地球化学循环。
\section{植被自养呼吸}\label{植被自养呼吸}
CoLM的自养呼吸$CF_{ar,total}$计算包括维持呼吸$CF_{mr,total}$和生长呼吸$CF_{gr,total}$,
模型对他们分别进行模拟 \citep{lavigne1997growth,sprugel1995respiration}。
自养呼吸指植被活体组织维持正常的代谢活动所消耗的碳,生长呼吸指植被生长所消耗的碳。
\begin{equation}
CF_{ar,total}=CF_{mr, total}+CF_{gr,total}
\end{equation}
\subsection{维持呼吸}
叶片的维持呼吸($R_d$)有光合作用模块中公式(\ref{R_d1})计算。
除此之外,活茎、活粗根和细根同样存在维持呼吸。
维持呼吸和植被器官单位面积的氮含量成正比,同时受到温度的调控:
\begin{equation}
CF_{{livestem}}=N_{{livestem }} \cdot R_{{base }} \cdot R_{q10}^{\left(T_{2m}-20\right) / 10}
\end{equation}
\begin{equation}
CF_{ {livecroot }}=N_{ {livecroot }} \cdot R_{ {base }} \cdot R_{q10}^{\left(T_{2m}-20\right) / 10}
\end{equation}
\begin{equation}
CF_{ {froot }}=N_{{froot}} \cdot R_{{base}} \cdot R_{q10}^{\left(T_{2m}-20\right) / 10}
\end{equation}
其中$CF_{livestem}$,$CF_{livecroot}$和$CF_{froot}$分别是活茎、活粗根和细根的维持呼吸速率。
$R_{q10}$是维持呼吸的温度敏感性参数,$T_{2m}$是2m高度气温, $N_{livestem}$,$N_{livecroot}$和$N_{froot}$
分别代表单位面积活茎氮含量、活粗根氮含量和活细根氮含量。$R_{base}$是基础维持呼吸速率。
木本植物功能型存在死茎和死粗根库,但维持呼吸的计算不包括死茎和死粗根库。因此,假设基础维持呼吸速率为常数。
总维持呼吸$CF_{mr,total}$的计算包括叶、活茎、活粗根和细根的维持呼吸的总和:
\begin{equation}
CF_{mr,total}=R_{d}+CF_{livestem}+CF_{livecroot}+CF_{froot}
\end{equation}
\subsection{生长呼吸}\label{生长呼吸}
生长呼吸$CF_{gr,total}$由植被单位面积的净生长速率乘以系数0.11得到:
\begin{equation}
CF_{gr,total}=NPP \cdot 0.11
\end{equation}
植被单位面积单位时间的净生长速率由净初级生产力($NPP$)来代表,$NPP$是光合作用速率和自养呼吸速率的差值,
并且同时考虑了土壤的氮限制,其详细计算见章节 \ref{植被土壤的氮竞争} 植被土壤的氮竞争。生长呼吸的计算方案是基于\citet{atkins2018quantifying}
的通量站点的木质和非木质组织的构造消耗研究得出的。在模型中,假设生长呼吸发生时间与碳分配的发生同步。
\section{植被碳氮分配}\label{植被碳氮分配}
\subsection{维持呼吸的碳消耗}
CoLM碳分配首先满足维持呼吸($CF_{mr,total}$)的需求,其次碳分配需要填补由于夜间或冬天光合作用少于维持呼吸所造成的碳存储亏缺,
最后剩余碳分配才能用于植被各营养器官库的生长。所以由光合作用支持的维持呼吸碳供给($CF_{GPP,mr}$)可以表达为:
\begin{equation}\label{F_GPP_mr}
CF_{GPP,mr}=\left\{\begin{array}{c}CF_{mr, total}\left(CF_{mr, total} \leq CF_{GPP}\right) \\ CF_{GPP}\left(CF_{mr,total}>CF_{GPP}\right)\end{array}\right.
\end{equation}
其中,$CF_{GPP}$代表实际光合作用碳收入,正常情况,$CF_{GPP}\geq CF_{mr,\ total}$,
维持呼吸的消耗($CF_{mr,total}$)可以完全由光合作用($CF_{GPP}$)提供。
但夜间和冬天光合作用的碳收入通常会低于呼吸作用的碳消耗,维持呼吸仅部分由光合作用支持,另外一部分($CF_{xs,mr}$)由植被呼吸碳储存库($CS_{xs}$)支持:
\begin{equation}\label{CF_xs_mr}
CF_{xs, mr}=\left\{\begin{array}{c}0\left(CF_{mr, total} \leq CF_{GPP}\right) \\ CF_{mr, total}-CF_{GPP}\left(CF_{mr, total}>CF_{GPP}\right)\end{array}\right.
\end{equation}
联合公式(\ref{F_GPP_mr})和(\ref{CF_xs_mr}),可以保证维持呼吸的碳平衡关系:
\begin{equation}
CF_{GPP, mr}+CF_{xs, mr}=CF_{mr, total}
\end{equation}
\subsection{用于植被呼吸的碳储存库}
植被呼吸的碳储存库($CS_{xs}$)会根据光合作用对维持呼吸的亏缺和补充进行更新,其每个时间步长的变化量($\Delta CS_{xs}$)可表示为:
\begin{equation}
\Delta CS_{xs}=\left(CF_{GPP, xs}-CF_{xs, mr}\right) \cdot \Delta t
\end{equation}
$\Delta t$代表模型时间步长,$CF_{GPP,xs}$是光合作用对植被的呼吸碳储存库($CS_{xs}$)的补充。



由于光合作用对维持呼吸存在亏缺的可能性,植被的呼吸碳储存库($CS_{xs}$)也存在负值。
光合作用对植被呼吸碳存储库($CS_{xs}$)的补充($CF_{GPP,xs}$)仅当该库为负值时存在,并且须首先满足维持呼吸的需要。
\begin{equation}
CF_{GPP, xs}=\left\{\begin{array}{ll}0 & CS_{xs} \geq 0 \\ \min \left(-\frac{CS_{xs}}{86400 \cdot \tau_{xs}}, \max \left(CF_{GPP}-CF_{GPP, mr}, 0\right)\right) & CS_{xs}<0\end{array}\right.
\end{equation}
其中$\tau_{xs}=30$天,$-\frac{CS_{xs}}{86400 \tau_{xs}}$代表如果光合作用碳收入充足,最快需要30天将亏缺的植被呼吸碳存储库($CS_{xs}$)填补上。
当然,如果扣除维持呼吸后的碳收入不足以维持$-\frac{CS_{xs}}{86400 \tau_{xs}}$,其补充的碳通量为光合作用扣除维持呼吸后的剩余碳收入,
即$max{\left(CF_{GPP}-CF_{GPP,mr},0\right)}$。
\subsection{植被的碳氮生长分配比例}
用于植被生长的碳收入,即可分配碳($CF_{avail_alloc}$),是光合作用碳通量($CF_{GPP}$)扣除维持呼吸($CF_{GPP,mr}$)和对植被呼吸碳存储的补充($CF_{GPP,xs}$)后的剩余碳收入:
\begin{equation}
CF_{ {availalloc }}=CF_{GPP}-CF_{GPP, mr}-CF_{GPP, xs}
\end{equation}
植物各器官所分配到的碳的比例由分配系数参数决定:
\begin{enumerate}
  \item 新生长细根和新生长叶的比例$a_1$; 
  \item 新生长活粗根和新生长活茎的比例 $a_2$;
  \item 新生长活茎和新生长叶的比例$a_3$;
  \item 新生长活茎和新生长叶的比例$a_3$;
  \item 新生长活茎在新生长总茎(活茎+死茎)碳含量的比例$a_4$;
  \item 生长呼吸在总生长碳的比例 $g_1$,这些分配系数参数取决于其所属植被功能型。其中,新生长活茎和新生长叶的比例$a_3$由前一年的年$NPP_{ann}$ ($g C m^{-2} year^{-1}$)决定:
    \begin{equation}
      a_{3}=\frac{2.7}{1+e^{-0.004 \cdot\left(NPP_{ann}^{-300}\right)}}-0.4
    \end{equation}
\end{enumerate}

因此,随着$NPP$增加,植被倾向于将更多的碳分配给茎 \citep{allen2005,vanninen2005carbon}。


因此叶片的单位碳生长需要植被的总碳输入是:
\begin{equation}\label{C_allom}
C_{ {allom }}=\left\{\begin{array}{lr}\left(1+g_{l}\right)\left(1+a_{1}+a_{3}\left(1+a_{2}\right)\right) &  \text{ woody } \text{PFT} \\ 
  \left(1+g_{l}\right)\left(1+a_{1}\right) &  \text{ non }- \text{ woody } \text{PFT}\end{array}\right.
\end{equation}
植被氮分配与碳分配共享同样一套分配系数参数,结合各器官的碳氮比参数,叶片的单位碳生长需要植被的总氮输入是:
\begin{equation}\label{N_allom}
N_{ {allom }}=\left\{\begin{array}{lr}\frac{1}{CN_{ {leaf }}}+\frac{a_{1}}{CN_{ {froot }}}+\frac{a_{3} a_{4}\left(1+a_{2}\right)}
  {CN_{l w}}+\frac{a_{3}\left(1-a_{4}\right)\left(1+a_{2}\right)}{CN_{d w}} \ \ \  \text { woody PFT } \\ 
  \frac{1}{CN_{ {leaf }}}+\frac{a_{1}}{CN_{ {froot }}} & \text { non }- \text{ woody PFT }\end{array}\right.
\end{equation}
$CN_{leaf}$,$CN_{froot}$,$CN_{lw}$,和$CN_{dw}$分别代表叶、细根、活木和死木的碳氮比。
根据植被可分配碳($CF_{avail_{alloc}}$),可以算出植被的氮需求($NF_{plant_{demand}}$):
\begin{equation}
N F_{ {plant_{demand }}}=CF_{ {avail_{alloc }}} \cdot \frac{N_{ {allom }}}{C_{ {allom }}}
\end{equation}
然而,实际碳生长($CF_{actual_{alloc}}$)需要根据实际的氮供给($NF_{actual_{alloc}}$)来计算。
实际氮供给($NF_{actual_{alloc}}$)来源于植被氮重利用($NF_{retran_{alloc}}$)和土壤无机氮摄取($NF_{sminn_{alloc}}$)两部分:
\begin{equation}\label{NF_actual_alloc}
NF_{actual_{alloc}}=NF_{retran_{alloc}}+NF_{sminn_{alloc}}
\end{equation}



\subsection{植被氮重利用}\label{植被氮重利用}
植被营养器官在凋落前通常会回收部分氮以用于新组织的生长,
被称为植被氮的重利用 \citep{magill1997biogeochemical,oikawa2005dynamics,son1991aboveground}。
植被氮重利用的计算依赖于植被氮重利用库($NS_{retrans}$)和植被氮需求($NF_{plant_{demand}}$)。
其中,可用于生长的重利用氮($NF_{avail_{retran}}$)与植被氮需求($NF_{plant_{demand}}$)成正比:
\begin{equation}
NF_{avail_{retrans}}=\min{\left(\frac{NF_{retrans_{ann}}\ \cdot \frac{NF_{plant_{demand}}}{NF_{plant_{demand_{ann}}}}}{{\Delta t}},\frac{NS_{retrans}}{{\Delta t}}\right)}
\end{equation}
其中,$NF_{retrans_{ann}}$是去年植被重利用氮总量 ($\rm g N m^{-2} year^{-1}$),$NF_{plant_{demand_{ann}}}$是去年植被氮需求总量 ($\rm g N m^{-2} year^{-1}$),${\Delta t}$是模型的时间步长。
实际来源于重利用氮库的氮通量($NF_{retran_{alloc}}$)是可利用氮($NF_{avail_{retrans}}$)和氮需求($NF_{plant_{demand}}$)的最小值:
\begin{equation}\label{NF_retran_alloc}
  NF_{retran_{alloc}}=min{\left(NF_{avail_{retrans}},NF_{plant_{demand}}\right)}
\end{equation}



\subsection{土壤无机氮摄取}\label{土壤无机氮摄取}
由于植被氮的重利用,植被对土壤的无机氮需求($NF_{plant_{demand_{soil}}}$)降低为:
\begin{equation}\label{NF_plant_demand_soil}
  NF_{plant_{demand_{soil}}}=NF_{plant_{demand}}-NF_{retran_{alloc}}	       
\end{equation}
由于不同植被功能型对于有限的土壤无机氮的竞争,以及土壤微生物和植被之间的竞争,植被从土壤中的无机氮摄取将进一步缩减:
\begin{equation}\label{NF_sminn_alloc}
NF_{sminn_{alloc}}=NF_{plant_{demand_{soil}}}\cdot f_{plant_{demand}}
\end{equation}
其中,$f_{plant_{demand}}$是植被氮摄取的限制因子,其范围在0到1之间。其具体计算将在章节 \ref{植被土壤的氮竞争} 植被土壤的氮竞争中介绍。



\subsection{植被的碳氮生长分配}\label{植被的碳氮生长分配}
通过公式(\ref{NF_sminn_alloc}), (\ref{NF_retran_alloc})和(\ref{NF_actual_alloc}),实际氮供给($NF_{actual_{alloc}}$)可以被模型计算,
再联立公式(\ref{C_allom}和\ref{N_allom}),实际植被碳生长($CF_{actual_{alloc}}$)可以被计算:
\begin{equation}
  CF_{actual_{alloc}}=NF_{actual_{alloc}}\frac{C_{allom}}{N_{allom}}
\end{equation}
同时,容易计算叶碳的实际生长:
\begin{equation}
  CF_{alloc_{leaf_{total}}}=CF_{actual_{alloc}}/C_{allom}
\end{equation}
假设碳分配到组织碳库和非结构碳库的比例固定$\left(f_{cur}:\left(1-f_{cur}\right)\right)$,
$f_{cur}$为新生长的组织库所占的比例。根据分配系数参数,各个器官的碳生长可以被计算:
\begin{equation}\label{CF_alloc_{leaf}}
  CF_{alloc,leaf}=CF_{alloc_{leaf_{total}}}\cdot  f_{cur}
\end{equation}
\begin{equation}
  CF_{alloc,leaf_{storage}}=CF_{alloc_{leaf_{total}}}\cdot \left(1-f_{cur}\right)
\end{equation}
\begin{equation}
  CF_{alloc,froot}=CF_{alloc_{leaf_{total}}}\cdot a_1\cdot f_{cur}
\end{equation}
\begin{equation}
  CF_{alloc,{froot_{storage}}}=CF_{alloc_{leaf_{total}}}\cdot a_1\cdot \left(1-f_{cur}\right)
\end{equation}
如果是木本植物类型,碳分配器官还包括茎和粗根:
\begin{equation}
  CF_{alloc,livestem}=CF_{alloc_{leaf_{total}}}\cdot a_3a_4\cdot f_{cur}
\end{equation}
\begin{equation}
  CF_{alloc,livestem_{storage}}=CF_{alloc_{leaf_{total}}}\cdot a_3a_4\cdot \left(1-f_{cur}\right)
\end{equation}
\begin{equation}
  CF_{alloc,deadstem}=CF_{alloc_{leaf_{total}}}\cdot a_3\left(1-a_4\right)\cdot f_{cur}
\end{equation}
\begin{equation}
  CF_{alloc,deadstem_{storage}}=CF_{alloc_{leaf_{total}}}\cdot a_3\left(1-a_4\right)\cdot \left(1-f_{cur}\right)
\end{equation}
\begin{equation}
  CF_{alloc,livecroot}=CF_{alloc_{leaf_{total}}}\cdot a_2a_3a_4\cdot f_{cur}
\end{equation}
\begin{equation}
  CF_{alloc,livecroot_{storage}}=CF_{alloc_{leaf_{total}}}\cdot a_2a_3a_4\cdot \left(1-f_{cur}\right)
\end{equation}
\begin{equation}
  CF_{alloc,deadcroot}=CF_{alloc_{leaf_{total}}}\cdot a_2a_3\cdot \left(1-a_4\right)\cdot f_{cur}
\end{equation}
\begin{equation}\label{CF_alloc_deadcroot_{storage}}
  CF_{alloc,deadcroot_{storage}}=CF_{alloc_{leaf_{total}}}\cdot a_2a_3\left(1-a_4\right)\cdot \left(1-f_{cur}\right)
\end{equation}
将公式(\ref{CF_alloc_{leaf}})-(\ref{CF_alloc_deadcroot_{storage}})求和可得,因此碳平衡得以保证:
\begin{equation}
  \sum_{i}{CF_{alloc,i}}=CF_{actuall\ {alloc}}
\end{equation}
其中$CF_{alloc,i}$代表每个植被库的新生长碳含量,
$i$等于$leaf$、$leaf_{storage}$、$froot$、$froot_{storage}$、$livestem$、$livestem_{storage}$、$deadstem$、
 $deadsteam_{storage}$、$livecroot$、 $livecroot_{storage}$、$deadcroot$
 和$deadcroot_{storage}$分别代表叶库、叶存储库、细根库、细根存储库、活茎库、
 活茎存储库、死茎库、死茎存储库、活粗根库、活粗根存储库、死粗根库和死粗根存储库的新生长碳含量。



 对应各个器官的氮生长也可以计算:
\begin{equation}
  NF_{alloc,leaf}=\frac{CF_{alloc_{leaf_{total}}}}{CN_{leaf}}\cdot f_{cur}
\end{equation}
\begin{equation}
  NF_{alloc,leaf_{storage}}=\frac{CF_{alloc_{leaf_{total}}}}{CN_{leaf}}\cdot \left(1-f_{cur}\right)
\end{equation}
\begin{equation}
  NF_{alloc,froot}=CF_{alloc_{leaf_{total}}}\cdot \frac{a_1}{CN_{froot}}\cdot f_{cur}
\end{equation}
\begin{equation}
  NF_{alloc,froot_{storage}}=CF_{alloc_{leaf_{total}}}\cdot \frac{a_1}{CN_{froot}}\cdot \left(1-f_{cur}\right)
\end{equation}

如果是木本植物类型,碳分配器官还包括茎和粗根:
\begin{equation}
  NF_{alloc,livestem}=CF_{alloc_{leaf_{total}}}\cdot \frac{a_3a_4}{CN_{lw}}\cdot f_{cur}
\end{equation}
\begin{equation}
  NF_{alloc,livestem_{storage}}=CF_{alloc_{leaf_{total}}}\cdot \frac{a_3a_4}{CN_{lw}}\cdot \left(1-f_{cur}\right)
\end{equation}
\begin{equation}
  NF_{alloc,deadstem}=CF_{alloc_{leaf_{total}}}\cdot \frac{a_3\left(1-a_4\right)}{CN_{dw}}\cdot f_{cur}
\end{equation}
\begin{equation}
  NF_{alloc,deadstem_{storage}}=CF_{alloc_{leaf_{total}}}\cdot \frac{a_3\left(1-a_4\right)}{CN_{dw}}\cdot \left(1-f_{cur}\right)
\end{equation}
\begin{equation}
  NF_{alloc,livecroot}=CF_{alloc_{leaf_{total}}}\cdot \frac{a_2a_3a_4}{CN_{lw}}\cdot f_{cur}
\end{equation}
\begin{equation}
  NF_{alloc,livecroot_{storage}}=CF_{alloc_{leaf_{total}}}\cdot \frac{a_2a_3a_4}{CN_{lw}}\cdot \left(1-f_{cur}\right)
\end{equation}
\begin{equation}
  NF_{alloc,deadcroot}=CF_{alloc_{leaf_{total}}}\cdot \frac{a_2a_3\left(1-a_4\right)}{CN_{dw}}\cdot f_{cur}
\end{equation}
\begin{equation}
  NF_{alloc,deadcroot_{storage}}=CF_{alloc_{leaf_{total}}}\cdot \frac{a_2a_3\left(1-a_4\right)}{CN_{dw}}\cdot \left(1-f_{cur}\right)
\end{equation}
将公式(13.24)-(13.35)求和可得,因此氮平衡得以保证:
\begin{equation}
  \sum_{i}{NF_{alloc,i}}=CF_{actuall,alloc}\cdot \frac{N_{allom}}{C_{allom}}=NF_{actuall_alloc}
\end{equation}
其中$NF_{alloc,i}$代表每个植被库的新生长氮含量, 
$i$等于$leaf$,$leaf_{storage}$,$froot$,$froot_{storage}$,$livestem$,$livestem_{storage}$,
$deadstem$,$deadsteam_{storage}$,$livecroot$,$livecroot_{storage}$,$deadcroot$,
 和$deadcroot_{storage}$分别代表叶库、叶存储库、细根库、细根存储库、活茎库、
 活茎存储库、死茎库、死茎存储库、活粗根库、活粗根存储库、死粗根库和死粗根存储库的新生长氮含量。

\section{植被物候过程的耦合预报方案}\label{植被物候过程的耦合预报方案}
CoLM植被物候过程的耦合预报方案通过对叶片碳的收支控制,模拟叶面积指数的季节变化特征。
\subsection{模型中的基本物候变量和概念}\label{模型中的基本物候变量和概念}
a.	发芽展叶期\\
CoLM物候过程模块假设落叶植被功能型存在发芽展叶期,即叶面积指数伴随着叶碳库在10多天的发芽展叶期内逐渐增加。发芽初期,非结构存储库将其一半的碳存储转给碳传输库,
在随后10多天时间里,碳传输库逐渐将碳转移给组织库,以模拟叶碳含量在此期间逐渐升高的发芽现象。模拟上通过控制从传输库$CS_{i_{xfer}}$
到组织库$CS_i$的碳转移通量$CF_{i_{xfer}\rightarrow i}$来实现
($i$等于$leaf$,$froot$,$livestem$,$deadstem$,$livecroot$和
$deadcroot$分别代表叶、细根、活茎、死茎、活粗根和死粗根):
\begin{equation}
  CF_{leaf_{xfer}\rightarrow leaf}=r_{{xfer}_{on}}\cdot CS_{leaf_{xfer}}\ 
\end{equation}
\begin{equation}
  CF_{froot_{xfer}\rightarrow froot}=r_{{xfer}_{on}}\cdot CS_{froot_{xfer}}\ 
\end{equation}
\begin{equation}
  CF_{livestem_{xfer}\rightarrow leaf}=r_{{xfer}_{on}}\cdot CS_{livestem_{xfer}}\ 
\end{equation}
\begin{equation}
  CF_{deadstem_{xfer}\rightarrow froot}=r_{{xfer}_{on}}\cdot CS_{deadstem_{xfer}}\ 
\end{equation}
\begin{equation}
  CF_{livecroot_{xfer}\rightarrow leaf}=r_{{xfer}_{on}}\cdot CS_{livecroot_{xfer}}\ 
\end{equation}
\begin{equation}
  CF_{deadcroot_{xfer}\rightarrow froot}=r_{{xfer}_{on}}\cdot CS_{deadcroot_{xfer}}\ 
\end{equation}
在碳转移的同时,也伴随着氮转移通量($NF_{i_{xfer}\rightarrow i}$),控制从氮传输库$NS_{i_{xfer}}$到氮组织库$NS_{i}$的转移:
\begin{equation}
  NF_{leaf_{xfer}\rightarrow leaf}=r_{{xfer}_{on}}\cdot NS_{leaf_{xfer}}\ 
\end{equation}
\begin{equation}
  NF_{froot_{xfer}\rightarrow froot}=r_{{xfer}_{on}}\cdot NS_{froot_{xfer}}\ 
\end{equation}
\begin{equation}
  NF_{livestem_{xfer}\rightarrow leaf}=r_{{xfer}_{on}}\cdot NS_{livestem_{xfer}}\ 
\end{equation}
\begin{equation}
  NF_{deadstem_{xfer}\rightarrow froot}=r_{{xfer}_{on}}\cdot NS_{deadstem_{xfer}}\ 
\end{equation}
\begin{equation}
  NF_{livecroot_{xfer}\rightarrow leaf}=r_{{xfer}_{on}}\cdot NS_{livecroot_{xfer}}\ 
\end{equation}
\begin{equation}
  NF_{deadcroot_{xfer}\rightarrow froot}=r_{{xfer}_{on}}\cdot NS_{deadcroot_{xfer}}\ 
\end{equation}
其中,$r_{{xfer}_{on}}$是控制传输库碳转移到组织库速率的变量($s-1$),是随时间变化的变量。
\begin{equation}
r_{xfer_{o} n}=\left\{\begin{array}{ll}\frac{2}{t_{ {onset}}} &  { 当 } t_{ {onset}} \neq \Delta t \\ \frac{1}{\Delta t} &  { 当 } t_{onset}=\Delta t\end{array}\right.
\end{equation}
$t_{onset}$以倒计时的形式记录发芽展叶期还剩多少秒,$\Delta t$是模型时间步长$t_{onset}\neq\Delta t$时,
$\frac{2}{t_{onset}}$随着时间的推移转移速率,
即叶片生长速率逐渐加快;$t_{onset}=\Delta t$时为发芽展叶期最后一个时间步长,所有传输库都将转移给组织库。\\
b.落叶期 \\
CoLM物候模块同样假设落叶植被功能型存在落叶期,在落叶期内,叶碳库和细根碳库在为期10多天时间里逐渐降为0:
\begin{equation}
CF_{ {leaf,litter }}^{n}=\left\{\begin{array}{ll}CF_{ {leaf,litter }}^{n-1}+r_{xfer_{off}}\left(CS_{ {leaf }}-CF_{ {leaf }}^{n-1} t_{ {offset }}\right) &  { 当 } t_{ {offset }} \neq \Delta t \\ \frac{CS_{ {leaf }}}{\Delta t}+CF_{ {alloc,leaf }} &  { 当 } t_{offset}=\Delta t\end{array}\right.
\end{equation}
\begin{equation}
CF_{ {froot },  { litter }}^{n}=\left\{\begin{array}{ll}CF_{ {froot }, l i t t e r}^{n-1}+r_{xfer_{o} f f}\left(CS_{ {froot }}-CF_{ {froot }}^{n-1} t_{offset}\right) &  { 当 } t_{offset} \neq \Delta t \\ \frac{CS_{ {froot }}}{\Delta t}+CF_{ {alloc,froot }} &  { 当 } t_{offset}=\Delta t\end{array}\right.
\end{equation}
\begin{equation}
r_{xfer_{o} f f}=\frac{2 \Delta t}{t_{offset}^{2}}
\end{equation}
其中$CF_{leaf,litter}^{n-1}$和$CF_{leaf}$,$_{littern}$分别代表上一个模拟时间步长和这一个时间步长的叶碳凋落通量。
$CF_{froot,litter}^{n-1}$和$CF_{froot,litter}^n$分别代表上一个模拟时间步长和这一个时间步长的细根碳凋落通量。
$t_{offset}$以倒计时的形式记录落叶期还剩多少秒。$CS_{leaf}$和$CS_{froot}$代表组织库的叶碳和细根碳含量,将随着落叶期的推进逐渐下降。
另外,落叶期的凋落速率参数($r_{{xfer}_{off}}$)将随时间逐渐增加。$t_{offset}=\Delta t$时为落叶期最后一个时间步长,
所有叶和细根组织库内的所有碳氮都将凋落。


叶氮和细根氮库在落叶期的凋落通量($NF_{leaf,litter}$,$NF_{froot,litter}$)
还与叶和根的碳氮比($CN_{leaf}$和$CN_{froot}$)息息相关,同时,
离开叶库的氮将有一部分被存进氮重利用库($NF_{leaf,retrans}$),
以备下一次生长再使用,其中$CN_{leaf,litter}$是叶片凋落物的碳氮比,作为预设参数被读入模型:
\begin{equation}
N F_{leaf,litter}=\frac{CF_{leaf,litter}}{CN_{leaf,litter}}
\end{equation}
\begin{equation}
N F_{froot,litter}=\frac{CF_{froot,litter}}{CN_{froot}}
\end{equation}
\begin{equation}
N F_{leaf, retrans}=\frac{CF_{leaf,litter}}{CN_{leaf}}-NF_{leaf,litter}
\end{equation}


落叶植被功能型的发芽展叶期和落叶期一般成对出现,其触发条件和温度、
土壤湿度以及落叶植被类型有关,详细描述在章节 \ref{季节落叶植被的物候} 和 \ref{物候凋落物} 中介绍。物候示意图如图\ref{fig:CoLM物候示意图}。\\

{
\begin{figure}[]
\centering
\includegraphics{Figures/植被生物地球化学循环过程/CoLM物候示意图.png}
\caption{CoLM物候示意图 \citep{lawrence2018}。 }
\label{fig:CoLM物候示意图}
\end{figure}
}
c. 活茎周转\\
CoLM物候模块中,在活茎的周转过程中,活茎细胞或活粗根最终死亡后变成死茎库或死粗根库的组成部分,
因此,活茎或活粗根碳每年固定比例($r_{lwt}$)进入死茎库或死粗根库:
\begin{equation}
CF_{ {livestem,deadstem }}=CS_{ {livestem }} r_{l w t}
\end{equation}
\begin{equation}
CF_{ {livecroot,deadcroot }}=CS_{ {livecroot }} r_{ {lwt }}
\end{equation}
其中,$CF_{livestem,deadstem}$是活茎死亡变成死茎的碳通量,
$CF_{livecroot,deadcroot}$是活粗根死亡变成死粗根的碳通量。活茎和活粗根的周转时间是0.7年:
\begin{equation}
  \tau_{l w t}=\frac{0.7}{365 \cdot 86400}
  \end{equation}
  对应的氮通量为:
  \begin{equation}
N F_{livestem,deadstem}=CS_{livestem} r_{l w t} / CN_{d w}
\end{equation}
\begin{equation}
N F_{livecroot,deadcroot}=CS_{livecroot} r_{lwt} / CN_{d w}
\end{equation}
其中,$NF_{livestem,deadstem}$是活茎死亡变成死茎的氮,
$NF_{livecroot,deadcroot}$是活粗根死亡变成死粗根的氮,$CN_{dw}$是死茎和死粗根的碳氮比。


由于活茎和活粗根的碳氮比低于死茎和死粗根的碳氮比,所以从活茎或活粗根到死茎或死粗根的氮将会有一部分结余,存入植被再利用氮库:
\begin{equation}
N F_{ {livestem,retrans }}=\left(\frac{CF_{ {livestem,deadstem }}}{CN_{lw}}\right)-N F_{ {livestem,deadstem }}
\end{equation}
\begin{equation}
N F_{ {livecroot,retrans }}=\left(\frac{CF_{ {livecroot,deadcroot }}}{CN_{lw}}\right)-N F_{ {livecroot,deadcroot }}
\end{equation}
其中,$NF_{livestem,retrans}$是活茎死亡时回收再利用的氮,$NF_{livecroot,retrans}$是活粗根死亡时回收再利用的氮。
\subsection{常绿植被的物候}\label{常绿植被的物候}
常绿植被功能型假设光合作用的净碳收入以及从土壤中的氮摄取直接分配给叶、细根、活茎和活粗茎的组织库。
代表非结构碳库的存储库和传输库将不会有任何碳存储。常绿植被功能型不存在特别的发芽展叶期和落叶期,
叶碳储量存在不随时间变化的固定碳周转速率:
\begin{equation}
\tau_{bglf}=\frac{1}{\tau_{leaf} \cdot 365 \cdot 86400}
\end{equation}
因此,叶面积指数的季节变化将主要由光合作用净碳收入的季节变化引起。
叶和细根的碳凋落通量($CF_{leaf,litter}$,$CF_{froot,litter}$)为:
\begin{equation}
CF_{leaf,litter}=r_{bglf} CS_{leaf}
\end{equation}
\begin{equation}
CF_{froot,litter}=r_{bglf} CS_{froot}
\end{equation}
相应的叶氮凋落通量($NF_{leaf,litter}$)、细根氮凋落通量($NF_{froot,litter}$)和再利用氮通量($NF_{leaf,retrans}$)为:
\begin{equation}
N F_{leaf,litter}=CS_{leaf,litter} / CN_{leaf_{litter}}
\end{equation}
\begin{equation}
N F_{froot,litter}=CS_{froot,litter} / CN_{froot}
\end{equation}
\begin{equation}
N F_{leaf,retrans}=\left(\frac{CF_{leaf,litter }}{CN_{leaf}}\right)-N F_{leaf,litter}
\end{equation}
\subsection{季节落叶植被的物候}\label{季节落叶植被的物候}
季节落叶植被功能型根据积温计算植被物候的发芽展叶期和落叶期,CoLM定义季节性落叶植被功能型仅存在于纬度大于19.5度非赤道区域。
季节性落叶植被功能型假设植被每年仅存在一次发芽展叶期和落叶期。由于南北半球冬夏反季,
所以,通过白昼时长的变化判断冬季转变节点,积温($GDD$)的计算在冬季转变节点开始累积\citep{white1997continental}:
\begin{equation}
GDD _{sum}^{n}=\left\{\begin{array}{ll}GDD _{sum}^{n-1}+T_{s, 3} \cdot \Delta t \cdot 86400 & T_{s, 3} \geq 0^{\circ}{C} \\ GDD _{sum}^{n-1} & T_{s, 3}<0^{\circ}{C}\end{array}\right.
\end{equation}
其中$GDD_{sum}^n$是第$n$天后的积温,积温通过对高于0 \textcelsius 的第三层土壤温度($T_{s,3}$) (\textcelsius)进行累加。
当$GDD_{sum}^n>{GDD}_{sum_{crit}}$时,季节性落叶树开始发芽展叶期。${GDD}_{sum_{crit}}$为发芽物候关键参量,
与前一年的温度($T_{2m,ann_{avg}}$) (\textcelsius)有关:
\begin{equation}
GDD _{sum_{c r i t}}=\exp \left(4.8+0.13 \cdot T_{2 m, ann_{avg}}\right)
\end{equation}
当发芽展叶期开始时,积温${GDD}_{sum}$重设为0,发芽展叶期倒计时重设:
\begin{equation}
t_{o n s e t}=86400 \cdot n_{ {dayso }_{o} n}
\end{equation}
其中$n_{days_{on}}=30$,代表发芽展叶期倒计时30天。
同时,发芽展叶期开始的第一个时间步长$\Delta t$内,50\%的存储库中的碳进入传输库:
\begin{equation}
  CF_{leaf_{stor},leaf_{xfer}}=0.5 CS_{leaf_{stor}}/\Delta t
\end{equation}
\begin{equation}
  CF_{froot_{stor},froot_{xfer}}=0.5  CS_{froot_{stor}}/\Delta t
\end{equation}
\begin{equation}
  CF_{livestem_{stor},livestem_{xfer}}=0.5  CS_{livestem_{stor}}/\Delta t
\end{equation}
\begin{equation}
  CF_{deadstem_{stor},deadstem_{xfer}}=0.5  CS_{deadstem_{stor}}/\Delta t
\end{equation}
\begin{equation}
  CF_{livecroot_{stor},livecroot_{xfer}}=0.5  CS_{livecroot_{stor}}/\Delta t
\end{equation}
\begin{equation}
  CF_{deadcroot_{stor},deadcroot_{xfer}}=0.5 CS_{deadcroot_{stor}}/\Delta t
\end{equation}
\begin{equation}
  CF_{gresp_{stor},gresp_{xfer}}=0.5  CS_{gresp_{stor}}/\Delta t
\end{equation}
同时相应的氮传输:
\begin{equation}
NF_{leaf_{stor},leaf_{xfer}}=0.5  NS_{leaf_{stor}}/\Delta t
\end{equation}

\begin{equation}
  NF_{froot_{stor},froot_{xfer}}=0.5  NS_{froot_{stor}}/\Delta t
\end{equation}

\begin{equation}
  NF_{livestem_{stor},livestem_{xfer}}=0.5  NS_{livestem_{stor}}/\Delta t
\end{equation}

\begin{equation}
  NF_{deadstem_{stor},deadstem_{xfer}}=0.5 NS_{deadstem_{stor}}/\Delta t
\end{equation}

\begin{equation}
  NF_{livecroot_{stor},livecroot_{xfer}}=0.5  NS_{livecroot_{stor}}/\Delta t
\end{equation}

\begin{equation}
  NF_{deadcroot_{stor},deadcroot_{xfer}}=0.5 NS_{deadcroot_{stor}}/\Delta t
\end{equation}

\begin{equation}
  NF_{gresp_{stor},gresp_{xfer}}=0.5 NS_{gresp_{stor}}/\Delta t
\end{equation}


如果到夏天,日照长度缩短时还未开始发芽展叶期,
$GDD_{sum}^n$重设为0直至冬天。当发芽展叶期开始后,30天倒计时开始,直至$t_{onset}=0$,发芽展叶期结束:

\begin{equation}
t_{onset}^n=t_{onset}^{n-1}-\Delta t
\end{equation}
当日照长度低于39300秒时,植被进入落叶期,落叶期倒计时为15天:
\begin{equation}
  t_{offset}^n=t_{offset}^{n-1}-\Delta t
\end{equation}

胁迫落叶植被的物候\\

胁迫落叶植被功能型包括草地和热带干旱落叶树等可以既响应干旱又响应温度胁迫。
当胁迫不存在时,该植被类型的物候还可以转变为常绿树的物候。当胁迫触发时,传输库的碳很快转变为组织库的碳。


当温度暖和,干旱成为触发胁迫的主要条件。当上一次落叶期结束后,土壤水分因子就开始从0累加:
\begin{equation}
SWI_{sum}^{n}=\left\{\begin{array}{ll}SWI_{sum}^{n-1}+f_{d a y} & \text{ 当 } \Psi_{s, 3} \geq-0.6 M P a \\ SWI_{sum}^{n} &  \text{ 当 } \Psi_{s, 3}<-0.6 M P a\end{array}\right.
\end{equation}
其中$\Psi_{s,3}$是第三层土壤的水势,当土壤水势高于-0.6MPa时,土壤足够湿润,土壤水分因子就开始累加。
当土壤水分因子高于15,同时过去十天有至少20mm的降水,并且冷温胁迫没有触发,日照时长超过6小时,植被进入发芽展叶期。


同时,因为土壤温度过低($FD_{sum}^n>15$),需要应用冷气候标准
\begin{equation}
F D_{sum}^{n}=\left\{\begin{array}{ll}F D_{sum}^{n-1}+f_{d a y} &  \text{ 当 } T_{s, 3} > \text{0 \textcelsius} \\ 
F D_{sum}^{n-1} &  \text{ 当 } T_{s, 3} \leq \text{0 \textcelsius}\end{array}\right.
\end{equation}
即当气温低,发芽展叶期的触发需要土壤温度和土壤湿度同时满足条件:
$SWI_{sum}>15,\ GDD_{sum}>GDD_{sum_crit}$,和日照时长大于6小时。
当发芽展叶期开始后,30天倒计时开始,直至$t_{onset}=0$,发芽展叶期结束:
\begin{equation}
t_{o n s e t}^{n}=t_{o n s e t}^{n-1}-\Delta t
\end{equation}
当持续土壤干旱或持续低温或日照长度低于6小时,胁迫落叶期就将触发。
土壤干旱用累积落叶土壤水分因子($OSWI_{sum}^n$)来量化:
\begin{equation}
OSWI_{sum}^{n}=\left\{\begin{array}{ll}OSWI_{sum}^{n-1}+f_{d a y}, &  \text{ 当 } \Psi_{s, 3}<-2 M P a \\ 
\max \left(OSWI_{sum}^{n-1}-f_{d a y}, 0\right) &  \text{ 当 } \Psi_{s, 3}>-2 M P a\end{array}\right.
\end{equation}
当前面一个发芽展叶期已经完成时,且$OSWI_{sum}^n\geq15$,即触发落叶期。
冷温胁迫用累积落叶冷冻天数来量化:
\begin{equation}
OFD_{sum}^{n}=\left\{\begin{array}{ll}OFD_{sum}^{n-1}+f_{d a y} &  \text{ 当 }{T}_{s, 3} \leq \text{0 \textcelsius} \\ 
\max \left(OFD_{sum}^{n-1}-f_{d a y}, 0\right) & \text{ 当 }{T}_{s, 3}> \text{0 \textcelsius}\end{array}\right.
\end{equation}
当前面一个发芽展叶期已经完成时,且$OFD_{sum}^n\geq15$,即触发落叶期。


总而言之,当以上三个条件($OSWI_{sum}^n\geq15$,$OFD_{sum}^n\geq15$,日照时长小于6小时)满足其一,植被进入落叶期,落叶期倒计时为15天:
\begin{equation}
t_{offset}^{n}=t_{offset}^{n-1}-\Delta t
\end{equation}
当落叶条件始终不满足(1年以上),胁迫落叶植被功能型就表现为常绿物候。
长生长季控制变量($LGS$)被用来刻画该常绿物候的周转速率:
\begin{equation}
\tau_{b g l f}=\frac{L G S}{\tau_{leaf} \cdot 365 \cdot 86400}
\end{equation}

\begin{equation}
L G S=\left\{\begin{array}{cc}0 &  { 当 } n_{ {days }_{active}}<365 \\ \left.\frac{n_{ {days_{active}}}}{365}-1\right) &  { 当 } 365 \leq n_{ {days }_{active} }<730 \\ 1 &  { 当 } n_{ {days}_{active}} \geq 730\end{array}\right.
\end{equation}
$n_{days_{active}}$是植被只从上次发芽开始的天数。


每个时间步长,都有储存库的碳进入传输库:
\begin{equation}
CF_{leaf_{stor,leaf}} f_{xfer}=CS_{leaf_{stor} } \tau_{bgtr}
\end{equation}
\begin{equation}
CF_{ {froot }_{stor},{ froot }_{xfer }}=CS_{{froot }_{stor}} \tau_{bgtr}
\end{equation}
\begin{equation}
  CF_{livestem_{stor},livestem_{xfer}}=CS_{livestem_{stor}}\tau_{bgtr}
\end{equation}
\begin{equation}
  CF_{deadstem_{stor},deadstem_{xfer}}=CS_{deadstem_{stor}}\tau_{bgtr}
\end{equation}
\begin{equation}
  CF_{livecroot_{stor},livecroot_{xfer}}=CS_{livecroot_{stor}}\tau_{bgtr}
\end{equation}
\begin{equation}
  CF_{deadcroot_{stor},deadcroot_{xfer}}=CS_{deadcroot_{stor}}\tau_{bgtr}
\end{equation}
传输库碳全部进入组织库:
\begin{equation}
  CF_{leaf_{xfer},leaf}=CS_{leaf_{xfer}}/\Delta t
\end{equation}
\begin{equation}
  CF_{froot_{xfer},froot}=CS_{froot_{xfer}}/\Delta t
\end{equation}
\begin{equation}
  CF_{livestem_{xfer},livestem}=CS_{livestem_{xfer}}/\Delta t
\end{equation}
\begin{equation}
  CF_{deadstem_{xfer},deadstem}=CS_{deadstem_{xfer}}/\Delta t
\end{equation}
\begin{equation}
  CF_{livecroot_{xfer},livecroot}=CS_{livecroot_{xfer}}/\Delta t
\end{equation}
\begin{equation}
  CF_{deadcroot_{xfer},deadcroot}=CS_{deadcroot_{xfer}}/\Delta t
\end{equation}
对应的氮凋落物:
\begin{equation}
  NF_{leaf_{stor},leaf_{xfer}}=NS_{leaf_{stor}}\tau_{bgtr}
\end{equation}
\begin{equation}
  NF_{froot_{stor},froot_{xfer}}=NS_{froot_{stor}}\tau_{bgtr}
\end{equation}
\begin{equation}
  NF_{livestem_{stor},livestem_{xfer}}=NS_{livestem_{stor}}\tau_{bgtr}
\end{equation}
\begin{equation}
  NF_{deadstem_{stor},deadstem_{xfer}}=NS_{deadstem_{stor}}\tau_{bgtr}
\end{equation}
\begin{equation}
  NF_{livecroot_{stor},livecroot_{xfer}}=NS_{livecroot_{stor}}\tau_{bgtr}
\end{equation}
\begin{equation}
  NF_{deadcroot_{stor},deadcroot_{xfer}}=NS_{deadcroot_{stor}}\tau_{bgtr}
\end{equation}
传输库氮全部进入组织库:
\begin{equation}
  NF_{leaf_{xfer},leaf}=NS_{leaf_{xfer}}/\Delta t
\end{equation}
\begin{equation}
  NF_{froot_{xfer},froot}=NS_{froot_{xfer}}/\Delta t
\end{equation}
\begin{equation}
  NF_{livestem_{xfer},livestem}=NS_{livestem_{xfer}}/\Delta t
\end{equation}
\begin{equation}
  NF_{deadstem_{xfer},deadstem}=NS_{deadstem_{xfer}}/\Delta t
\end{equation}
\begin{equation}
  NF_{livecroot_{xfer},livecroot}=NS_{livecroot_{xfer}}/\Delta t
\end{equation}
\begin{equation}
  NF_{deadcroot_{xfer},deadcroot}=NS_{deadcroot_{xfer}}/\Delta t
\end{equation}


\subsection{物候凋落物}\label{物候凋落物}
物候的凋落物包括叶和细根,其中叶凋落物将进入代谢凋落物、
纤维素凋落物和木质部凋落物,细根凋落物同样进入代谢凋落物、纤维素凋落物和木质部凋落物:
\begin{equation}
  CF_{leaf,lit1}=\sum_{p}^{npft}{CF_{leaf,litter}f_{lab_{leaf},p}{wcol_p}}
\end{equation}
\begin{equation}
  CF_{leaf,lit2}=\sum_{p}^{npft}{CF_{leaf,litter}f_{cel_{leaf},p}{wcol_p}}
\end{equation}
\begin{equation}
  CF_{leaf,lit3}=\sum_{p}^{npft}{CF_{leaf,litter}f_{lig_{leaf},p}{wcol_p}}
\end{equation}
\begin{equation}
  CF_{froot,lit1}=\sum_{p}^{npft}{CF_{froot,litter}f_{lab_{froot},p}{wcol_p}}
\end{equation}
\begin{equation}
  CF_{froot,lit2}=\sum_{p}^{npft}{CF_{froot,litter}f_{cel_{froot},p}{wcol_p}}
\end{equation}
\begin{equation}
  CF_{froot,lit3}=\sum_{p}^{npft}{CF_{froot,litter}f_{lig_{froot},p}{wcol_p}}
\end{equation}
其中$f_{lab_{leaf},p}$,$f_{cel_{leaf},p}$和$f_{lig_{leaf},p}$ 是叶片凋落物中,
活性凋落物,纤维素和木质素凋落物占得比例,$f_{lab_{leaf},p}$,$f_{cel_{leaf},p}$
, 和$f_{lig_{leaf},p}$ 是细根凋落物中,活性凋落物,
纤维素和木质素凋落物占的比例,${wcol_p}$代表每个patch的面积比。

对应的氮通量可表达为:
\begin{equation}
  NF_{leaf,lit1}=\sum_{p}^{npft}{NF_{leaf,litter}f_{lab_{leaf},p}{wcol_p}}
\end{equation}
\begin{equation}
  NF_{leaf,lit2}=\sum_{p}^{npft}{NF_{leaf,litter}f_{cel_{leaf},p}{wcol_p}}
\end{equation}
\begin{equation}
  NF_{leaf,lit3}=\sum_{p}^{npft}{NF_{leaf,litter}f_{lig_{leaf},p}{wcol_p}}
\end{equation}
\begin{equation}
  NF_{froot,lit1}=\sum_{p}^{npft}{NF_{froot,litter}f_{lab_{froot},p}{wcol_p}}
\end{equation}
\begin{equation}
  NF_{froot,lit2}=\sum_{p}^{npft}{NF_{froot,litter}f_{cel_{froot},p}{wcol_p}}
\end{equation}
\begin{equation}
  NF_{froot,lit3}=\sum_{p}^{npft}{NF_{froot,litter}f_{lig_{froot},p}{wcol_p}}
\end{equation}
\subsection{植被的自然死亡}\label{植被的自然死亡}
植被的整体自然死亡率被假设为2\%每年。
所有的叶、细根、活茎、死茎、或粗根和死粗根的组织库、存储库和传输库受死亡率的控制,
其凋落物碳氮通量为: 


\begin{equation}
  CF_{leaf_{mort}}=CS_{leaf}\cdot \frac{0.2\%}{365\cdot 86400}
\end{equation}
\begin{equation}
  CF_{froot_mort}=CS_{froot}\cdot \frac{0.2\%}{365\cdot 86400}
\end{equation}
\begin{equation}
  CF_{livestem_mort}=CS_{livestem}\cdot \frac{0.2\%}{365\cdot 86400}
\end{equation}
\begin{equation}
  CF_{deadstem_mort}=CS_{deadstem}\cdot \frac{0.2\%}{365\cdot 86400}
\end{equation}
\begin{equation}
  CF_{livecroot_mort}=CS_{livecroot}\cdot \frac{0.2\%}{365\cdot 86400}
\end{equation}
\begin{equation}
  CF_{deadcroot_mort}=CS_{deadcroot}\cdot \frac{0.2\%}{365\cdot 86400}
\end{equation}
\begin{equation}
  NF_{leaf_{mort}}=NS_{leaf}\cdot \frac{0.2\%}{365\cdot 86400}
\end{equation}
\begin{equation}
  NF_{froot_mort}=NS_{froot}\cdot \frac{0.2\%}{365\cdot 86400}
\end{equation}
\begin{equation}
  NF_{livestem_mort}=NS_{livestem}\cdot \frac{0.2\%}{365\cdot 86400}
\end{equation}
\begin{equation}
  NF_{deadstem_mort}=NS_{deadstem}\cdot \frac{0.2\%}{365\cdot 86400}
\end{equation}
\begin{equation}
  NF_{livecroot_mort}=NS_{livecroot}\cdot \frac{0.2\%}{365\cdot 86400}
\end{equation}
\begin{equation}
  NF_{deadcroot_mort}=NS_{deadcroot}\cdot \frac{0.2\%}{365\cdot 86400}
\end{equation}
存储库的凋落:
\begin{equation}
  CF_{leaf_{{stor}_{mort}}}=CS_{leaf_{stor}}\cdot \frac{0.2\%}{365\cdot 86400}
\end{equation}
\begin{equation}
  CF_{froot_{{stor}_{mort}}}=CS_{froot_{stor}}\cdot \frac{0.2\%}{365\cdot 86400}
\end{equation}
\begin{equation}
CF_{livestem_{{stor}_{mort}}}=CS_{livestem_{stor}}\cdot \frac{0.2\%}{365\cdot 86400}
\end{equation}
\begin{equation}
  CF_{deadstem_{{stor}_{mort}}}=CS_{deadstem_{stor}}\cdot \frac{0.2\%}{365\cdot 86400}
\end{equation}
\begin{equation}
  CF_{livecroot_{{stor}_{mort}}}=CS_{livecroot_{stor}}\cdot \frac{0.2\%}{365\cdot 86400}
\end{equation}
\begin{equation}
  CF_{deadcroot_{{stor}_{mort}}}=CS_{deadcroot_{stor}}\cdot \frac{0.2\%}{365\cdot 86400}
\end{equation}
\begin{equation}
  NF_{leaf_{{stor}_{mort}}}=NS_{leaf_{stor}}\cdot \frac{0.2\%}{365\cdot 86400}
\end{equation}
\begin{equation}
  NF_{froot_{{stor}_{mort}}}=NS_{froot_{stor}}\cdot \frac{0.2\%}{365\cdot 86400}
\end{equation}
\begin{equation}
  NF_{livestem_{{stor}_{mort}}}=NS_{livestem_{stor}}\cdot \frac{0.2\%}{365\cdot 86400}
\end{equation}
\begin{equation}
  NF_{deadstem_{{stor}_{mort}}}=NS_{deadstem_{stor}}\cdot \frac{0.2\%}{365\cdot 86400}
\end{equation}
\begin{equation}
  NF_{livecroot_{{stor}_{mort}}}=NS_{livecroot_{stor}}\cdot \frac{0.2\%}{365\cdot 86400}
\end{equation}
\begin{equation}
  NF_{deadcroot_{{stor}_{mort}}}=NS_{deadcroot_{stor}}\cdot \frac{0.2\%}{365\cdot 86400}
\end{equation}
\begin{equation}
  CF_{leaf_{{xfer}_{mort}}}=CS_{leaf_{xfer}}\cdot \frac{0.2\%}{365\cdot 86400}
\end{equation}
\begin{equation}
  CF_{froot_{{xfer}_{mort}}}=CS_{froot_{xfer}}\cdot \frac{0.2\%}{365\cdot 86400}
\end{equation}
\begin{equation}
  CF_{livestem_{{xfer}_{mort}}}=CS_{livestem_{xfer}}\cdot \frac{0.2\%}{365\cdot 86400}
\end{equation}
\begin{equation}
  CF_{deadstem_{{xfer}_{mort}}}=CS_{deadstem_{xfer}}\cdot \frac{0.2\%}{365\cdot 86400}
\end{equation}
\begin{equation}
  CF_{livecroot_{{xfer}_{mort}}}=CS_{livecroot_{xfer}}\cdot \frac{0.2\%}{365\cdot 86400}
\end{equation}
\begin{equation}
  CF_{deadcroot_{{xfer}_{mort}}}=CS_{deadcroot_{xfer}}\cdot \frac{0.2\%}{365\cdot 86400}
\end{equation}
\begin{equation}
  NF_{leaf_{{xfer}_{mort}}}=NS_{leaf_{xfer}}\cdot \frac{0.2\%}{365\cdot 86400}
\end{equation}
\begin{equation}
  NF_{froot_{{xfer}_{mort}}}=NS_{froot_{xfer}}\cdot \frac{0.2\%}{365\cdot 86400}
\end{equation}
\begin{equation}
  NF_{livestem_{{xfer}_{mort}}}=NS_{livestem_{xfer}}\cdot \frac{0.2\%}{365\cdot 86400}
\end{equation}
\begin{equation}
  NF_{deadstem_{{xfer}_{mort}}}=NS_{deadstem_{xfer}}\cdot \frac{0.2\%}{365\cdot 86400}
\end{equation}
\begin{equation}
  NF_{livecroot_{{stor}_{mort}}}=NS_{livecroot_{stor}}\cdot \frac{0.2\%}{365\cdot 86400}
\end{equation}
\begin{equation}
  NF_{deadcroot_{{stor}_{mort}}}=NS_{deadcroot_{stor}}\cdot \frac{0.2\%}{365\cdot 86400}
\end{equation}
死亡的植被碳氮通量将进入凋落物库:
\begin{equation}
CF_{leaf_{mort},lit1}=\sum_{p=0}^{npft}{CF_{leaf_{mort}}f_{lab_{leaf},p}{wcol_p}}
\end{equation}
\begin{equation}
  CF_{leaf_{mort},lit2}=\sum_{p=0}^{npft}{CF_{leaf_{mort}}f_{cel_{leaf},p}{wcol_p}}
\end{equation}
\begin{equation}
  CF_{leaf_{mort},lit3}=\sum_{p=0}^{npft}{CF_{leaf_{mort}}f_{lig_{leaf},p}{wcol_p}}
\end{equation}
\begin{equation}
  CF_{froot_mort,lit1}=\sum_{p=0}^{npft}{CF_{froot_mort}f_{lab_{leaf},p}{wcol_p}}
\end{equation}
\begin{equation}
  CF_{froot_mort,lit2}=\sum_{p=0}^{npft}{CF_{froot_mort}f_{cel_{leaf},p}{wcol_p}}
\end{equation}
\begin{equation}
  CF_{froot_mort,lit3}=\sum_{p=0}^{npft}{CF_{froot_mort}f_{lig_{leaf},p}{wcol_p}}
\end{equation}
\begin{equation}
  NF_{leaf_{mort},lit1}=\sum_{p=0}^{npft}{NF_{leaf_{mort}}f_{lab_{leaf},p}{wcol_p}}
\end{equation}
\begin{equation}
  NF_{leaf_{mort},lit2}=\sum_{p=0}^{npft}{NF_{leaf_{mort}}f_{cel_{leaf},p}{wcol_p}}
\end{equation}
\begin{equation}
  NF_{leaf_{mort},lit3}=\sum_{p=0}^{npft}{NF_{leaf_{mort}}f_{lig_{leaf},p}{wcol_p}}
\end{equation}
\begin{equation}
  NF_{froot_mort,lit1}=\sum_{p=0}^{npft}{NF_{froot_mort}f_{lab_{leaf},p}{wcol_p}}
\end{equation}
\begin{equation}
  NF_{froot_mort,lit2}=\sum_{p=0}^{npft}{NF_{froot_mort}f_{cel_{leaf},p}{wcol_p}}
\end{equation}
\begin{equation}
  NF_{froot_mort,lit3}=\sum_{p=0}^{npft}{NF_{froot_mort}f_{lig_{leaf},p}{wcol_p}}
\end{equation}
其中$f_{lab_{leaf},p}$,$f_{cel_{leaf},p}$,和$f_{lig_{leaf}}$,$p $是叶片凋落物中,
活性凋落物,纤维素和木质素凋落物占得比例,$f_{lab_{leaf},p}$,$f_{cel_{leaf},p}$,和$fligleaf$,$p $
是细根凋落物中,活性凋落物,纤维素和木质素凋落物占的比例,${wcol_p}$代表每个patch的面积比。


活茎、死茎、活粗根和死粗根在树木死亡后进入木质部残体库:
\begin{equation}
  CF_{livestem_{mort},cwd}=\sum_{p=0}^{npft}{CF_{livestem_{mort}}{wcol_p}}
\end{equation}
\begin{equation}
  CF_{deadstem_{mort},cwd}=\sum_{p=0}^{npft}{CF_{deadstem_{mort}}{wcol_p}}
\end{equation}
\begin{equation}
  CF_{livecroot_{mort},cwd}=\sum_{p=0}^{npft}{CF_{livecroot_{mort}}{wcol_p}}
\end{equation}
\begin{equation}
  CF_{deadcroot_{mort},cwd}=\sum_{p=0}^{npft}{CF_{deadcroot_{mort}}{wcol_p}}
\end{equation}
\begin{equation}
  NF_{livestem_{mort},cwd}=\sum_{p=0}^{npft}{NF_{livestem_{mort}}{wcol_p}}
\end{equation}
\begin{equation}
  NF_{deadstem_{mort},cwd}=\sum_{p=0}^{npft}{NF_{deadstem_{mort}}{wcol_p}}
\end{equation}
\begin{equation}
  NF_{livecroot_{mort},cwd}=\sum_{p=0}^{npft}{NF_{livecroot_{mort}}{wcol_p}}
\end{equation}
\begin{equation}
  NF_{deadcroot_{mort},cwd}=\sum_{p=0}^{npft}{NF_{deadcroot_{mort}}{wcol_p}}
\end{equation}
此外,所有的含有非结构碳氮的储存库和传输库在植被死亡后都进入代谢凋落物库中:
\begin{equation}
  CF_{leaf_{{stor}_{mort}},lit1}=\sum_{p=0}^{npft}{CF_{leaf_{{stor}_{mort}}}{wcol_p}}
\end{equation}
\begin{equation}
  CF_{froot_{{stor}_{mort}},lit1}=\sum_{p=0}^{npft}{CF_{froot_{{stor}_{mort}}}{wcol_p}}
\end{equation}
\begin{equation}
  CF_{livestem_{{stor}_{mort}},lit1}=\sum_{p=0}^{npft}{CF_{livestem_{{stor}_{mort}}}{wcol_p}}
\end{equation}
\begin{equation}
  CF_{deadstem_{{stor}_{mort}},lit1}=\sum_{p=0}^{npft}{CF_{deadstem_{{stor}_{mort}}}{wcol_p}}
\end{equation}
\begin{equation}
  CF_{livecroot_{{stor}_{mort}},lit1}=\sum_{p=0}^{npft}{CF_{livecroot_{{stor}_{mort}}}{wcol_p}}
\end{equation}
\begin{equation}
  CF_{deadcroot_{{stor}_{mort}},lit1}=\sum_{p=0}^{npft}{CF_{deadcroot_{{stor}_{mort}}}{wcol_p}}
\end{equation}
\begin{equation}
  CF_{gresp_{{stor}_{mort}},lit1}=\sum_{p=0}^{npft}{CF_{gresp_{{stor_{mort}}}}{wcol_p}}
\end{equation}
\begin{equation}
  CF_{leaf_{xfer_{mort,lit1}}}=\sum_{p=0}^{npft}{CF_{leaf_{{xfer_{mort}}}}{wcol_p}}
\end{equation}
\begin{equation}
  CF_{froot_{xfer_{mort,lit1}}}=\sum_{p=0}^{npft}{CF_{froot_{{xfer_{mort}}}}{wcol_p}}
\end{equation}
\begin{equation}
  CF_{livestem_{xfer_{mort,lit1}}}=\sum_{p=0}^{npft}{CF_{livestem_{{xfer_{mort}}}}{wcol_p}}
\end{equation}
\begin{equation}
  CF_{deadstem_{xfer_{mort,lit1}}}=\sum_{p=0}^{npft}{CF_{deadstem_{xfer_{mort}}}{wcol_p}}
\end{equation}
\begin{equation}
  CF_{livecroot_{xfer_{mort,lit1}}}=\sum_{p=0}^{npft}{CF_{livecroot_{{xfer_{mort}}}}{wcol_p}}
\end{equation}
\begin{equation}
  CF_{deadcroot_{xfer_{mort,lit1}}}=\sum_{p=0}^{npft}{CF_{deadcroot_{{xfer_{mort}}}}{wcol_p}}
\end{equation}
\begin{equation}
  CF_{gresp_{xfer_{mort,lit1}}}=\sum_{p=0}^{npft}{CF_{gresp_{{xfer}_{mort}}{wcol_p}}}
\end{equation}
对应的氮通量中,植被死亡后,再利用氮库进入代谢凋落物库:
\begin{equation}
  NF_{leaf_{{stor}_{mort}},lit1}=\sum_{p=0}^{npft}{NF_{leaf_{{stor}_{mort}}}{wcol_p}}
\end{equation}
\begin{equation}
  NF_{froot_{{stor}_{mort}},lit1}=\sum_{p=0}^{npft}{NF_{froot_{{stor}_{mort}}}{wcol_p}}
\end{equation}
\begin{equation}
  NF_{livestem_{{stor}_{mort}},lit1}=\sum_{p=0}^{npft}{NF_{livestem_{{stor}_{mort}}}{wcol_p}}
\end{equation}
\begin{equation}
  NF_{deadstem_{{stor}_{mort}},lit1}=\sum_{p=0}^{npft}{NF_{deadstem_{{stor}_{mort}}{wcol_p}}}
\end{equation}
\begin{equation}
  NF_{retrans_{mort,lit1}}=\sum_{p=0}^{npft}{NF_{retrans_{mort}}{wcol_p}}
\end{equation}
\begin{equation}
  NF_{leaf_{{xfer}_{mort,lit1}}}=\sum_{p=0}^{npft}{NF_{{leaf}_{xfer_{mort}}}{wcol_p}}
\end{equation}
\begin{equation}
  NF_{froot_{xfer_{mort,lit1}}}=\sum_{p=0}^{npft}{NF_{froot_{xfer_{mort}}}}{wcol_p}
\end{equation}
\begin{equation}
  NF_{livestem_{xfer_{mort,lit1}}}=\sum_{p=0}^{npft}{NF_{livestem_{xfer_{mort}}}{wcol_p}}
\end{equation}
\begin{equation}
NF_{deadstem_{xfer_{mort,lit1}}}=\sum_{p=0}^{npft}{NF_{deadstem_{xfer_{mort}}}{wcol_p}}
\end{equation}
\begin{equation}
  NF_{livecroot_{xfer_{mort,lit1}}}=\sum_{p=0}^{npft}{NF_{livecroot_{xfer_{mort}}}{wcol_p}}
\end{equation}
\begin{equation}
  NF_{deadcroot_{xfer_{mort,lit1}}}=\sum_{p=0}^{npft}{NF_{deadcroot_{xfer_{mort}}}{wcol_p}}
\end{equation}


\chapter{土壤凋落物生物地球化学循环过程}\label{土壤凋落物生物地球化学循环过程}
%\addcontentsline{toc}{chapter}{土壤凋落物生物地球化学循环过程}

%\begin{土壤凋落物生物地球化学循环过程}
植被物候和自然死亡凋落的植被碳进入土壤凋落物库后进行进一步碳氮循环。在土壤中凋落物碳氮元素经过土壤分解,
垂直混合后。土壤碳通过土壤呼吸回到大气,土壤氮通过矿化作用进入土壤无机氮库,然后被植被吸收或进一步固定在土壤有机氮库中。

CoLM的土壤和凋落物存在垂直分层结构,碳氮动态变化根据平衡方程在每一层分别积分:
\begin{equation}
\begin{array}{r}\frac{\partial C_{i}(z)}{\partial t}=R_{i}(z)+\sum_{i \neq j}\left(1-r f_{j}\right) T_{j i} k_{j}(z) C_{j}(z)-k_{i}(z) C_{i}(z) \\ +\frac{\partial}{\partial z}\left(D(z) \frac{\partial C_{i}}{\partial z}\right)+\frac{\partial}{\partial z}\left(A(z) C_{i}(z)\right)\end{array}
\end{equation}
其中,$C_i\left(z\right)$是第z层的土壤碳密度($\rm gC m^{-3}$); $R_i\left(z\right)$是第$z$层土壤的凋落物碳输入,
$\sum_{i\neq j}{\left(1-{rf}_j\right)T_{ji}k_j\left(z\right)C_j\left(z\right)}$代表土壤碳库j到碳库i的碳转变;
$k_j\left(z\right)$是第$z$层碳库$j$的周转速率;${rf}_j$是$j$库在转换中呼吸掉的碳的比例;
$\left(1-{rf}_j\right)$是微生物对库$j$分解的碳利用效率,$T_{ji}$是碳转换途径$j$到$i$在所有离开$j$库碳通量
(不包括因呼吸作用离开$j$库)的比例;$k_i\left(z\right)C_i\left(z\right)$代表库i的碳分解;
$\frac{\partial}{\partial z}\left(D\left(z\right)\frac{\partial C_i}{\partial z}\right)+\frac{\partial}{\partial z}\left(A\left(z\right)C_i\left(z\right)\right)$代表碳垂直传输。

\section{土壤凋落物分解}\label{土壤凋落物分解}
土壤凋落物碳氮分解在每一层分别进行,分解速率受每一层不同的环境因子控制:
\begin{equation}
k_{i}(z)=k_{i,base} \cdot \xi_{tsoil}(z) \cdot \xi_{water}(z) \cdot \xi_{depth}(z)
\end{equation}
其中$k_{i,base}$是库$i$的基础周转速率,$\xi_{tsoil}$是温度环境因子,$\xi_{water}$是土壤水环境因子,$\xi_{depth}$是深度因子。

\section{环境限制因子}\label{环境限制因子}
CoLM生物地球化学模块,控制土壤分解的环境因子包括土壤温度因子,土壤水分因子和深度因子。其中土壤温度因子运用$Q_{10}$关系:
\begin{equation}
\xi_{t s o i l}(z)=Q_{10}^{\frac{T_{{soil }}(z)-273.15}{10}}
\end{equation}
其中,$\xi_{tsoil}\left(z\right)$是第$z$层的土壤温度因子;$T_{soil}\left(z\right)$是第$z$层的土壤温度 (K),$Q_{10}$是温度敏感性参数。


当土壤温度低于$\rm 0^{\circ}C$,土壤完全结冰,土壤分解速率迅速下降。土壤温度因子使用另一个关系:
\begin{equation}
\xi_{t s o i l}(z)=0.82 \cdot Q_{10}^{-2.5} \cdot 1.5^{\frac{T_{{soil }}(z)-273.15}{10}}
\end{equation}
土壤水分因子根据土壤水势的计算而得出,刻画土壤水分对土壤分解的影响:
\begin{equation}
\xi_{w a t e r}(z)=\left\{\begin{array}{ll}\frac{\log \left(\frac{\Psi_{\min }}{\Psi(z)}\right)}{\log \left(\frac{\Psi_{\min }}{\Psi_{\max }}\right)} 
    & \Psi(\mathrm{z})>\Psi_{\min } \\ 0 . & \Psi(z)<\Psi_{\min }\end{array}\right.
\end{equation}
其中,$\xi_{water}\left(z\right)$是第$z$层的土壤水分因子,$\Psi\left(z\right)$是第$z$层的土壤水势,
$\Psi_{min}$是最小土壤水势;$\Psi_{max}$是最大土壤水势。当土壤水势小于最小土壤水势时,土壤周转速率为0。


深度因子参数化了影响土壤分解的一些其他环境因素,比如,微生物过程、矿化表面或聚合稳定等过程 \citep{koven2013effect},
是拟合碳垂直浓度剖面的重要因素\citep{jenkinson2008turnover}:
\begin{equation}
\xi_{d e p t h}=e^{-\frac{z}{10}}
\end{equation}
其中z是土壤深度(m)。
\subsection{土壤分解的氮限制}\label{土壤分解的氮限制}
土壤分解速率同样会受到土壤中无机氮含量的限制,但量化氮限制对土壤分解速率的影响,需要先计算无氮限制条件下的潜在土壤分解,其中土壤/凋落物库i (第$z$层)的潜在土壤分解为:
\begin{equation}
C F_{pot, i}(z)=C_{i}(z) k_{i}(z)
\end{equation}
从库$i$到库$j$的碳传输(第$z$层)为:
\begin{equation}
C F_{pot, i \rightarrow j}=C_{i}(z) k_{i}(z) \cdot\left(1-r f_{i}\right) \cdot T_{i j}
\end{equation}
相应的从库$i$到库$j$的传输中,潜在的净无机氮增加等于矿化作用引起的无机氮增加减去有机氮固定引起的无机氮减少:
\begin{equation}
\begin{array}{l}N F_{pot_{m} i n, i \rightarrow j}=\frac{C_{i}(z) k_{i}(z) \cdot T_{i j}}{C N_{i}}-\frac{C F_{pot, i \rightarrow j}}{C N_{j}} \\
     =\frac{C_{i}(z) k_{i}(z) \cdot T_{i j} \cdot \frac{C N_{j}}{C N_{i}}-C_{i}(z) k_{i}(z) \cdot T_{i j} \cdot\left(1-r f_{i}\right)}{C N_{j}}=
     -\frac{C i_{i}(z) k_{i}(z) T_{i j}\left(1-r f_{i}-\frac{C N_{j}}{C N_{i}}\right)}{C N_{j}}\end{array}
\end{equation}
如图 11 2所示共有10个土壤/凋落物之间的碳氮传输,所有10个传输引起的潜在净无机氮增加可具体表示为:
\begin{equation}\label{NF_pot_minmet}
    N F_{{pot}_{min,met} \rightarrow {soil1,vr}}(z)=-\frac{C_{{met}}(z) k_{{met}}(z) T_{met, soil1}\left(1-r f_{met}-\frac{C N_{{soil1 }}}{C N_{met}}\right)}{C N_{{soil1 }}}
\end{equation}
\begin{equation}
N F_{pot_{min,cel} \rightarrow { soil1,vr }}(z)=-\frac{C_{cel}(z) k_{cel}(z) T_{cel, soil1}\left(1-r f_{cel}-\frac{C N_{{soil 1}}}{C N_{cel}}\right)}{C N_{{soil1}}}
\end{equation}
\begin{equation}
N F_{pot_{min,lig} \rightarrow \operatorname{soil2,vr}}(z)=-\frac{C_{lig}(z) k_{lig}(z) T_{lig, soil2}\left(1-r f_{lig}-\frac{C N_{{soil2 }}}{C N_{lig}}\right)}{C N_{{soil2 }}}
\end{equation}
\begin{equation}
NF_{pot_{min,soil1}\rightarrow soil2,vr}\left(z\right)=-\frac{C_{soil1}\left(z\right)k_{soil1}\left(z\right)T_{soil1,soil2}\left(1-rf_{soil1}-\frac{CN_{soil2}}{CN_{soil1}}\right)}{CN_{soil2}}
\end{equation}
\begin{equation}
    NF_{pot_{min,cwd}\rightarrow cel,vr}\left(z\right)=-\frac{C_{cwd}\left(z\right)k_{cwd}\left(z\right)T_{cwd,cel}\left(1-rf_{cwd}-\frac{CN_{cel}}{CN_{cwd}}\right)}{CN_{cel}}
\end{equation}
\begin{equation}
    NF_{pot_{min,cwd}\rightarrow lig,vr}\left(z\right)=-\frac{C_{cwd}\left(z\right)k_{cwd}\left(z\right)T_{cwd,lig}\left(1-rf_{cwd}-\frac{CN_{lig}}{CN_{cwd}}\right)}{CN_{lig}}
\end{equation}
\begin{equation}
    NF_{pot_{min,soil1}\rightarrow soil3,vr}\left(z\right)=-\frac{C_{soil1}\left(z\right)k_{soil1}\left(z\right)T_{soil1,soil3}\left(1-rf_{soil1}-\frac{CN_{soil3}}{CN_{soil1}}\right)}{CN_{soil3}}
\end{equation}
\begin{equation}
    NF_{pot_{min,soil}2\rightarrow soil1,vr}\left(z\right)=-\frac{C_{soil2}\left(z\right)k_{soil2}\left(z\right)T_{soil2,soil1}\left(1-rf_{soil2}-\frac{CN_{soil1}}{CN_{soil2}}\right)}{CN_{soil1}}
\end{equation}
\begin{equation}
    NF_{pot_{min,soil2}\rightarrow soil3,vr}\left(z\right)=-\frac{C_{soil2}\left(z\right)k_{soil2}\left(z\right)T_{soil2,soil3}\left(1-rf_{soil2}-\frac{CN_{soil3}}{CN_{soil2}}\right)}{CN_{soil3}}
\end{equation}
\begin{equation}\label{NF_pot_min_soil3}
    NF_{pot_{min,soil3}\rightarrow soil1,vr}\left(z\right)=-\frac{C_{soil3}\left(z\right)k_{soil3}\left(z\right)T_{soil3,soil1}\left(1-rf_{soil3}-\frac{CN_{soil1}}{CN_{soil3}}\right)}{CN_{soil1}}
\end{equation}
把公式(\ref{NF_pot_minmet})-(\ref{NF_pot_min_soil3})所有正项求和,得到总矿化作用生成的无机氮:
\begin{equation}\label{NF_immob_demand_vr}
    NF_{immob,demand,vr}\left(z\right)=-\sum_{k=1}^{10}\min{\left(NF_{pot_min,i\left(k\right)\rightarrow j\left(k\right)},0\right)}
\end{equation}
当总固化作用需要的无机氮($NF_{immob,demand}$)大于土壤中可以提供的无机氮时,某个传输量就将减小,土壤分解就将减慢。
\section{植被土壤的氮竞争}\label{植被土壤的氮竞争}
实际上,土壤分解可以利用的无机氮往往比土壤中存在的无机氮还要小,因为土壤微生物利用无机氮还需要和植被进行竞争。

植被氮需求($NF_{plant_{demand_{soil}}}$)在公式(\ref{NF_plant_demand_soil})中已经被完整计算,若植被氮需求和土壤微生物固氮需求($NF_{immob,demand}$)小于总土壤无机氮含量(${NS}_{sminn}$):
\begin{equation}
    \left(NF_{plant_{demand_{soil}}}+NF_{immob,demand}\right)\cdot\Delta t<NS_{sminn}
\end{equation}
则土壤无机氮含量充分,不存在氮限制,反之则需要计算氮限制。

CoLM将植被的氮需求($NF_{plant_{demand_{soil}}}$)按无机氮含量的垂直分布比例分摊给每层土壤:
\begin{equation}
    NF_{plant_{demand_{soil}},vr}\left(z\right)=NF_{plant_{demand_{soil}}}\cdot\frac{NS_{sminn,vr}\left(z\right)}{NS_{sminn}}
\end{equation}
结合公式(\label{NF_pot_minmet})-(\ref{NF_immob_demand_vr})计算土壤氮固化作用的需求,可以得出每层土壤无机氮的总需求
\begin{equation}
    NF_{total_{demand_{soil}},vr}\left(z\right)=NF_{plant_{demand_{soil}},vr}\left(z\right)+NF_{immob,demand,vr}\left(z\right)
\end{equation}
CoLM在每层土壤中分别考虑无机氮的供需关系及其和每层无机氮含量${NS}_{sminn,vr}\left(z\right)$的关系,
我们得到每层土壤分解的氮限制因子($f_{immob_{demand},vr}$)和植被氮吸收的限制因子($f_{plant_{demand},vr}$):
\begin{equation}
    f_{plant_{demand},vr}\left(z\right)=f_{immob_{demand},vr}\left(z\right)=\frac{NS_{sminn,vr}\left(z\right)}{NF_{total_{demand_{soil}},vr}\left(z\right)\cdot\Delta t}
\end{equation}
平均植被氮吸收的限制因子($f_{plant_{demand}}$):
\begin{equation}
    f_{plant_{demand}}=\frac{\sum_{z=1}^{n=10}{f_{plant_{demand},vr}\left(z\right)\cdot N\ F_{plant_{demand_{soil}},vr}\left(z\right)}}{NF_{plant_{demand_{soil}}}}
\end{equation}
实际氮吸收$(NF_{sminn_{alloc}}$)可以由公式(\label{NF_sminn_alloc})求出,至此,章节 \ref{植被碳氮分配} 的所有变量可解。
实际碳传输,也可根据氮限制因子($f_{immob_{demand},vr}\left(z\right))$进一步得出:
\begin{equation}
C F_{met \rightarrow { soil1,vr }}(z)=\left\{\begin{array}{ll}C F_{pot, met \rightarrow soil1, vr}(z) \cdot f_{{immob }_{d} { emand }, vr}(z) & N F_{pot_{m in }, { met } \rightarrow { soil } 1}<0 \\ C F_{pot, met \rightarrow { soil1,vr }}(z) & N F_{pot_{m in }, { met } \rightarrow { soil } 1} \geq 0\end{array}\right.
\end{equation}
\begin{equation}
C F_{cel \rightarrow { soil1,vr }}(z)=\left\{\begin{array}{ll}C F_{pot, cel \rightarrow { soil1,vr }}(z) \cdot f_{i m m o b_{d} e m a n d, vr}(z) & N F_{pot_{m} i n, cel \rightarrow { soil1 }}<0 \\ C F_{pot, cel \rightarrow { soil1,vr }}(z) & N F_{pot_{m in,cel } \rightarrow { soil1 }} \geq 0\end{array}\right.
\end{equation}
\begin{equation}
C F_{{lig } \rightarrow { soil2,vr }}(z)=\left\{\begin{array}{ll}C F_{pot, lig \rightarrow { soil2,vr }}(z) \cdot f_{{immob }_{d} { emand }, vr}(z) & N F_{pot_{m} i n, lig \rightarrow { soil2 }}<0 \\ C F_{pot, lig \rightarrow { soil2,vr }}(z) & N F_{pot_{m} i n, lig \rightarrow { soil } 2} \geq 0\end{array}\right.
\end{equation}
\begin{equation}
    C F_{{soil1 } \rightarrow { soil2,vr }}(z)=\left\{\begin{array}{ll}C F_{pot, soil1 \rightarrow { soil2,vr }}(z) \cdot f_{{immob }_{d} { emand }, vr}(z) & N F_{pot_{m} i n, soil1 \rightarrow { soil2 }}<0 \\ C F_{pot, soil1 \rightarrow { soil2,vr }}(z) & N F_{pot_{m} i n, soil1 \rightarrow { soil } 2} \geq 0\end{array}\right.
\end{equation}
\begin{equation}
C F_{c w d \rightarrow cel, vr}(z)=\left\{\begin{array}{ll}C F_{pot, c w d \rightarrow cel, vr}(z) \cdot f_{i m m o b_{d} e m a n d, vr}(z) & N F_{pot_{m} i n, c w d \rightarrow cel}<0 \\ C F_{pot, c w d \rightarrow cel, vr}(z) & N F_{pot_{m} i n, c w d \rightarrow cel} \geq 0\end{array}\right.
\end{equation}
\begin{equation}
C F_{c w d \rightarrow lig, vr}(z)=\left\{\begin{array}{ll}C F_{pot, c w d \rightarrow lig, vr}(z) \cdot f_{i m m o b_{d} e m a n d, vr}(z) & N F_{pot_{m} i n, c w d \rightarrow lig}<0 \\ C F_{pot, c w d \rightarrow lig, vr}(z) & N F_{pot_{m} i n, c w d \rightarrow lig} \geq 0\end{array}\right.
\end{equation}
\begin{equation}
    C F_{soil1 \rightarrow soil3, vr}(z)=\left\{\begin{array}{ll}C F_{pot, soil1 \rightarrow soil3, vr}(z) \cdot f_{i m m o b_{d} e m a n d, vr}(z) & N F_{pot_{m} i n, soil1 \rightarrow soil3}<0 \\ C F_{pot, soil1 \rightarrow soil3, vr}(z) & N F_{pot_{m} i n, soil1 \rightarrow soil3} \geq 0\end{array}\right.
\end{equation}
\begin{equation}
    C F_{soil2 \rightarrow soil1, vr}(z)=\left\{\begin{array}{ll}C F_{pot, soil2 \rightarrow soil1, vr}(z) \cdot f_{i m m o b_{d} e m a n d, vr}(z) & N F_{pot_{m} i n, soil2 \rightarrow soil1}<0 \\ C F_{pot, soil2 \rightarrow soil1, vr}(z) & N F_{pot_{m} i n, soil2 \rightarrow soil1} \geq 0\end{array}\right.
\end{equation}
\begin{equation}
    C F_{soil2 \rightarrow soil3, vr}(z)=\left\{\begin{array}{ll}C F_{pot, soil2 \rightarrow soil3, vr}(z) \cdot f_{i m m o b_{d} e m a n d, vr}(z) & N F_{pot_{m} i n, soil2 \rightarrow soil3}<0 \\ C F_{pot, soil2 \rightarrow soil3, vr}(z) & N F_{pot_{m} i n, soil2 \rightarrow soil3} \geq 0\end{array}\right.
\end{equation}
\begin{equation}
    C F_{soil3 \rightarrow soil1, vr}(z)=\left\{\begin{array}{ll}C F_{pot, soil3 \rightarrow soil1, vr}(z) \cdot f_{i m m o b_{d} e m a n d, vr}(z) & N F_{pot_{m} i n, soil3 \rightarrow soil1}<0 \\ C F_{pot, soil3 \rightarrow soil1, vr}(z) & N F_{pot_{m} i n, soil3 \rightarrow soil1} \geq 0\end{array}\right.
\end{equation}
每个土壤凋落物碳库在每层土壤的实际碳分解:
\begin{equation}
C F_{met, vr}(z)=\frac{C F_{met \rightarrow { soil1,vr }}(z)}{1-r f_{met}}
\end{equation}
\begin{equation}
C F_{cel, vr}(z)=\frac{C F_{cel \rightarrow soil1, vr}(z)}{1-r f_{cel}}
\end{equation}
\begin{equation}
C F_{lig, vr}(z)=\frac{C F_{lig \rightarrow soil2, vr}(z)}{1-r f_{lig}}
\end{equation}
\begin{equation}
C F_{c w d, vr}(z)=C F_{c w d \rightarrow cel, vr}(z)+C F_{c w d \rightarrow lig, vr}(z)
\end{equation}
\begin{equation}
C F_{{soil } 1, vr}(z)=\frac{C F_{{soil1 } \rightarrow { soil } 2, vr}(z)}{1-r f_{{soil }}}+\frac{C F_{{soil } 1 \rightarrow { soils,vr }}(z)}{1-r f_{{soil }}}
\end{equation}
\begin{equation}
C F_{{soil2,vr }}(z)=\frac{C F_{{soil2 } \rightarrow { soil1,vr }}(z)}{1-r f_{{soil2 }}}+\frac{C F_{{soil2 } \rightarrow { soill,vr }}(z)}{1-r f_{{soil2 }}}
\end{equation}
\begin{equation}
C F_{{soil3,vr }}(z)=\frac{C F_{{soil } 3 \rightarrow \operatorname{soil1,vr}}(z)}{1-r f_{{soil3 }}}
\end{equation}
其中每层土壤的异养呼吸$CF_{hr,vr}\left(z\right)$:
\begin{equation}
\begin{array}{l}C F_{h r, vr}(z)=C F_{met, vr}(z) \cdot r f_{met}+C F_{cel, vr}(z) \cdot r f_{cel}+C F_{lig, vr}(z) \cdot r f_{lig} \\ +C F_{{soill } 1, vr}(z) \cdot r f_{{soil1 }}+C F_{{soil2,vr }}(z) \cdot r f_{{soil2 }}+C F_{{soil3,vr }}(z) \cdot r f_{{soil3 }}\end{array}
\end{equation}
同样的,实际氮传输,也根据氮限制因子下降:
\begin{equation}
    C F_{met \rightarrow soil1, vr}(z)=\left\{\begin{array}{ll}C F_{pot, met \rightarrow soil1, vr}(z) \cdot f_{i m m o b_{d} e m a n d, vr}(z) & N F_{pot_{m} i n, met \rightarrow soil1}<0 \\ C F_{pot, met \rightarrow soil1, vr}(z) & N F_{pot_{m} i n, met \rightarrow soil1} \geq 0\end{array}\right.
\end{equation}
\begin{equation}
    C F_{cel \rightarrow soil1, vr}(z)=\left\{\begin{array}{ll}C F_{pot, cel \rightarrow soil1, vr}(z) \cdot f_{i m m o b_{d} e m a n d, vr}(z) & N F_{pot_{m} i n, cel \rightarrow soil1}<0 \\ C F_{pot, cel \rightarrow soil1, vr}(z) & N F_{pot_{m} i n, cel \rightarrow soil1} \geq 0\end{array}\right.
\end{equation}
\begin{equation}
    C F_{lig \rightarrow soil2, vr}(z)=\left\{\begin{array}{ll}C F_{pot, lig \rightarrow soil2, vr}(z) \cdot f_{i m m o b_{d} e m a n d, vr}(z) & N F_{pot_{m} i n, lig \rightarrow soil2}<0 \\ C F_{pot, lig \rightarrow soil2, vr}(z) & N F_{pot_{m} i n, lig \rightarrow soil2} \geq 0\end{array}\right.
\end{equation}
\begin{equation}
    C F_{soil1 \rightarrow soil2, vr}(z)=\left\{\begin{array}{ll}C F_{pot, soil1 \rightarrow soil2, vr}(z) \cdot f_{i m m o b_{d} e m a n d, vr}(z) & N F_{pot_{m} i n, soil1 \rightarrow soil2}<0 \\ C F_{pot, soil1 \rightarrow soil2, vr}(z) & N F_{pot_{m} i n, soil1 \rightarrow soil2} \geq 0\end{array}\right.
\end{equation}
\begin{equation}
    C F_{cwd \rightarrow cel, vr}(z)=\left\{\begin{array}{ll}C F_{pot, cwd \rightarrow cel, vr}(z) \cdot f_{i m m o b_{d} e m a n d, vr}(z) & N F_{pot_{m} i n, cwd \rightarrow cel}<0 \\ C F_{pot, cwd \rightarrow cel, vr}(z) & N F_{pot_{m} i n, cwd \rightarrow cel} \geq 0\end{array}\right.
\end{equation}
\begin{equation}
    C F_{cwd \rightarrow lig, vr}(z)=\left\{\begin{array}{ll}C F_{pot, cwd \rightarrow lig, vr}(z) \cdot f_{i m m o b_{d} e m a n d, vr}(z) & N F_{pot_{m} i n, cwd \rightarrow lig}<0 \\ C F_{pot, cwd \rightarrow lig, vr}(z) & N F_{pot_{m} i n, cwd \rightarrow lig} \geq 0\end{array}\right.
\end{equation}
\begin{equation}
    C F_{soil1 \rightarrow soil3, vr}(z)=\left\{\begin{array}{ll}C F_{pot, soil1 \rightarrow soil3, vr}(z) \cdot f_{i m m o b_{d} e m a n d, vr}(z) & N F_{pot_{m} i n, soil1 \rightarrow soil3}<0 \\ C F_{pot, soil1 \rightarrow soil3, vr}(z) & N F_{pot_{m} i n, soil1 \rightarrow soil3} \geq 0\end{array}\right.
\end{equation}
\begin{equation}
    C F_{soil2 \rightarrow soil1, vr}(z)=\left\{\begin{array}{ll}C F_{pot, soil2 \rightarrow soil1, vr}(z) \cdot f_{i m m o b_{d} e m a n d, vr}(z) & N F_{pot_{m} i n, soil2 \rightarrow soil1}<0 \\ C F_{pot, soil2 \rightarrow soil1, vr}(z) & N F_{pot_{m} i n, soil2 \rightarrow soil1} \geq 0\end{array}\right.
\end{equation}
\begin{equation}
    C F_{soil2 \rightarrow soil3, vr}(z)=\left\{\begin{array}{ll}C F_{pot, soil2 \rightarrow soil3, vr}(z) \cdot f_{i m m o b_{d} e m a n d, vr}(z) & N F_{pot_{m} i n, soil2 \rightarrow soil3}<0 \\ C F_{pot, soil2 \rightarrow soil3, vr}(z) & N F_{pot_{m} i n, soil2 \rightarrow soil3} \geq 0\end{array}\right.
\end{equation}
\begin{equation}
    C F_{soil3 \rightarrow soil1, vr}(z)=\left\{\begin{array}{ll}C F_{pot, soil3 \rightarrow soil1, vr}(z) \cdot f_{i m m o b_{d} e m a n d, vr}(z) & N F_{pot_{m} i n, soil3 \rightarrow soil1}<0 \\ C F_{pot, soil3 \rightarrow soil1, vr}(z) & N F_{pot_{m} i n, soil3 \rightarrow soil1} \geq 0\end{array}\right.
\end{equation}
实际土壤氮固化作用同样受氮限制因子的影响:
\begin{equation}
NF_{immob,vr}\left(z\right)=NF_{immob,demand,vr}\left(z\right)\cdot\ f_{immob_{demand},vr}\left(z\right)
\end{equation}
但因为氮矿化作用为土壤提供无机氮,实际的土壤氮矿化作用仍然运用公式(\ref{NF_immob_demand_vr})。
\section{土壤凋落物的垂直传输}\label{土壤凋落物的垂直传输}
土壤凋落物的碳氮循环的垂直结构在章节 \ref{垂直分层结构} 已经被大致介绍。地上凋落物和地下凋落物组成了地下碳循环的输入,存在显著的垂直变化。
此外,土壤碳氮分解存在深度因子的影响,也是影响土壤碳氮模拟垂直分布的重要因子。除此之外,碳氮的垂直混合也是影响土壤碳氮浓度垂直剖面曲线的关键。
CoLM考虑土壤碳氮传输仅存在扩散作用下,和求解土壤湿度的垂直分布解法类似。
详细土壤垂直扩散方程的参数化方案见\citep{koven2011permafrost,koven2009formation,koven2013effect,koven2015permafrost} 。
\chapter{火灾}\label{ch:火灾}
\begin{mymdframed}{代码}
  本章对应代码源文件位于\texttt{main/BGC/}目录下。
\end{mymdframed}

CoLM 里采用 Li 火灾参数化方案~\citep{LiF2012,LiF2013,LiF2017,LiF2019}。它包括四个部分:热带密闭林区由砍伐引起的火灾、农业火、泥炭火、以及农田和热带密闭林以外的非泥炭火。其中,前三部分是简单的经验统计方程;而第四部分是中等复杂程度的过程模式,是该火灾方案的主体。燃烧面积由天气气候、植被特征(组成和生物量)、人类活动决定。在燃烧面积计算完成后,我们估计火灾影响。

\section{农田和热带密闭林区以外的非泥炭火}
格点燃烧面积 $A_{\mathrm {b}} $ (\unit{km^2.s^{-1}})是着火数$N_{\mathrm {f}} $ (\unit{count.s^{-1}})以及每个火的平均蔓延面积$a$ (\unit{km^2.count^{-1}})的乘积:
\begin{equation}
  A_{\mathrm {b}}  = N_{\mathrm {f}}  a
\end{equation}

\subsection{火发生}
格点的着火数是格点点火数 $N_{\mathrm {i}} $ ($\text{count}~\text{s}^{-1}$),燃料供给率 $f_{\mathrm {b}} $, 燃料可燃性 $f_{\mathrm {m}} $,以及未被人为抑制的比例 $f_{\mathrm{se,o}}$ 的乘积:
\begin{equation}
  N_{\mathrm {f}}  = N_{\mathrm {i}}  f_{\mathrm {b}}  f_{\mathrm {m}}  f_{\mathrm{se,o}}
\end{equation}
%
公式中的点火数为:
\begin{equation}
  N_{\mathrm{i}}=\left(I_{\mathrm{n}}+I_{\mathrm{a}}\right) A_{\mathrm{g}}
\end{equation}
%
其中 $I_{\mathrm {n}} $ (\unit{count.s^{-1}.km^{-2}})和 $I_{\mathrm {a}} $ (\unit{count.s^{-1}.km^{-2}})为格点内的闪电和人为点火频率,$A_{\mathrm {g}} $ (\unit{km^2})为格点面积。

闪电引起的点火频率为:
\begin{equation}
  I_{\mathrm{n}}=\gamma \psi I_{\mathrm{l}}
\end{equation}
其中,$\gamma$ 是云-地闪电的点火效率;$\psi=1/\left[(5.16+2.16\cos(3\min(60,\lambda))\right]$ 是云-地闪电频率占总闪电频率的比率,随纬度 $\lambda$ 而变化, $I_{\mathrm{l}}$ 为闪电频率 (\unit{count.s^{-1}.km^{-2}})。

而人为点火频率为:
\begin{equation}
I_{\mathrm{a}}=\frac{\alpha D_{\mathrm{P}} k\left(D_{\mathrm{P}}\right)}{n}
\end{equation}
其中,$\alpha=0.01$ (\unit{count.person^{-1}.mon^{-1}})是每月人为潜在点火数, $D_{\mathrm {P}} $ ($\text{person}~\text{km}^{-2}$)是人口密度, $k\left(D_{\mathrm{P}}\right)=6.8D_{\mathrm {P}} ^{-0.6}$为人为点火比率,$n$ 是每月的秒数。

燃料供给率为:
\begin{equation}
  f_{\mathrm {b}}  = \begin{cases}
    0 & \text{if } B_{\mathrm{ag}} < B_{\text{low}}\\
    \frac{B_{\mathrm{ag}}-B_{\mathrm{low}}}{B_{\mathrm{up}}-B_{\mathrm{low}}} & \text{if } B_{\mathrm{low}} \leqslant B_{\mathrm{ag}} \leqslant B_{\mathrm{up}}\\
    1 & \text{if } B_{\mathrm{ag}} > B_{\mathrm{up}}
  \end{cases}
\end{equation}
其中, $B_{\mathrm{ag}}$ 为地上生物量, $B_{\mathrm{low}}$ 和 $B_{\mathrm{up}}$ 为阈值。

燃料可燃性为:
\begin{equation}
  f_{\mathrm {m}}  = f_{\mathrm{RH}} f_{\beta},\quad T_{17cm}>T_{\mathrm {frz}}
\end{equation}
其中, $f_{\mathrm{RH}}$ 和 $f_{\beta}$ 分别是相对湿度${\mathrm {RH}}$和根区土壤湿度因$\beta$的函数, $T_{17cm}$和$T_{\mathrm{frz}}$ 分别为表层17 cm的土壤温度和冰点温度。
%
\begin{equation}
  f_{\mathrm{RH}} = (1-\omega)l_{\mathrm{RH0}}+\omega l_{\mathrm{RH_{\rm 30d}}}
\end{equation}
其中的权重因子是地上生物量的权重 $$\omega=\max\left[0, \min\left(1, \frac{B_{\mathrm{ag}} - 2500}{2500}\right)\right]$$
$$l_{\mathrm{RH_0}}=1-\max\left[0, \min\left(1, \frac{RH_{\mathrm{0}} - 30}{80-30}\right)\right]$$
和
$$l_{\mathrm{RH_{\rm 30d}}}=1-\max\left[0.75, \min\left(1, \frac{RH_{\rm 30d}}{90}\right)\right]$$
分别是当前时步相对湿度和过去30天滑动平均相对湿度的函数。
\begin{equation}
  f_{\mathrm {b}}  = \begin{cases}
    1 & \text{if } \beta < \beta_{\mathrm{low}}\\
    \frac{\beta_{\mathrm{up}}-\beta}{\beta_{\mathrm{up}}-\beta_{\mathrm{low}}} & \text{if } \beta_{\mathrm{low}} \leqslant \beta \leqslant \beta_{\mathrm{up}}\\
    0 & \text{if } \beta \geqslant \beta_{\mathrm{up}}
  \end{cases}
\end{equation}
中的$\beta_{\mathrm{low}}=0.85$和$\beta_{\mathrm{up}}=0.98$。


对于人烟稀少的地区($P \leqslant 0.1$ person km$^{-2}$),我们假设人为火抑制可以忽略,即 $f_{\mathrm{se,o}} = 1.0$;对于其它地区,
\begin{equation}
  f_{\mathrm{se,o}} = f_{\mathrm {d}}  f_{\mathrm {e}}
\end{equation}
其中的 $f_{\mathrm {d}} $ 及 $f_{\mathrm {e}} $ 为社会经济条件对火发生的影响:
\begin{equation}
  f_{\mathrm {d}}  = 0.01 + 0.98\exp(-0.025 D_{\mathrm {P}} )
\end{equation}
对于草和灌木:
\begin{equation}
  f_{\mathrm {e}}  = 0.1 + 0.9\exp\left[-\pi \left(\frac{\text{GDP}}{8} \right)^{0.5} \right]
\end{equation}
%
对于树:
%
\begin{equation}
  f_{\mathrm {e}}  = \begin{cases}
    0.39 & \text{if GDP} > 20\\
    0.79 & \text{if } 8 < \text{GDP} \leqslant 20\\
    1.0 & \text{if GDP} \leqslant 8
  \end{cases}
\end{equation}
%
其中GDP为人均GDP (\unit{kUS\$(1995).capita^{-1}})。


\subsection{火蔓延}
每个火的平均蔓延面积为:
\begin{equation}
  a = a^* F_{\mathrm{se}}
\end{equation}
%
其中, $a^*$ 为每个火的潜在蔓延面积 ($\text{km}^2~\text{count}^{-1}$), $F_{\mathrm{se}}$ 为社会经济条件的影响。

每个火的潜在蔓延面积为:
%
\begin{equation}
  a^{*}=\pi \frac{l}{2} \frac{\omega}{2} \times 10^{-6}=\frac{\pi u_{\mathrm{p}}^{2} \tau^{2}}{4 L_{\mathrm{B}}}\left(1+\frac{1}{H_{\mathrm{B}}}\right)^{2} \times 10^{-6}
\end{equation}
%
其中,$t = 1$(d) $= 3600\times24$(s)为未考虑人为火抑制的潜在平均火持续时间; $L_{\mathrm {B}} $ 和 $H_{\mathrm {B}} $为火蔓延椭圆区域的长宽比和焦点距离长轴两端的距离比:
\begin{equation}
  L_{\mathrm{B}}=1.0+10.0\left[1-\exp (-0.06 W)\right],
\end{equation}
\begin{equation}
  H_{\mathrm{B}}=\frac{u_{\mathrm{p}}}{u_{\mathrm{b}}}=\frac{L_{\mathrm{B}}+\left(L_{\mathrm{B}}^{2}-1\right)^{0.5}}{L_{\mathrm{B}}-\left(L_{\mathrm{B}}^{2}-1\right)^{0.5}};
\end{equation}
$u_{\mathrm {p}} $ (km s$^{-1}$)为下风区的蔓延速率:
\begin{equation}
  u_{\mathrm{p}}=u_{\max } C_{\mathrm{m}} g(W)
\end{equation}
%
其中, $W$ 为风速, $u_{\mathrm{max}}$ 为下风区火最大蔓延速率, $C_{\mathrm{m}}=\sqrt{f_{\mathrm{m}}}$为燃料湿度对火蔓延的影响因子,风速对下风区火蔓延速率的影响$g(W)$为:
\begin{equation}
  g(W)=0.05 \times \frac{2 L_{\mathrm{B}}}{1+\frac{1}{H_{\mathrm{B}}}}。
\end{equation}
类似火发生部分,当 $D_{\mathrm {P}}  \leqslant 0.1$ 时,$F_{\mathrm{se}} = 1.0$;当 $D_{\mathrm {P}}  > 0.1$
%
\begin{equation}
  F_{\mathrm{se}} = F_{\mathrm {d}}  F_{\mathrm {e}}
\end{equation}
%
其中, $F_{\mathrm {d}} $ 和 $F_{\mathrm {e}} $ 分别为社会和经济因子对火蔓延的影响。对于草和灌木:
\begin{equation}
  F_{\mathrm{d}}=0.2+0.8 \times \exp \left[-\pi\left(\frac{D_{\mathrm{p}}}{450}\right)^{0.5}\right]
\end{equation}
和
\begin{equation}
  F_{\mathrm{e}}=0.2+0.8 \times \exp \left(-\pi \frac{\text{GDP}}{7}\right)。
\end{equation}
对于树:
\begin{equation}
  F_{\mathrm{d}}=0.4+0.6 \times \exp \left(-\pi \frac{D_{\mathrm{p}}}{125}\right)
\end{equation}
和
\begin{equation}
  F_{\mathrm{e}}=\begin{cases}
    0.62 & \text{GDP}>20 \\
    0.83 & 8< \text{GDP} \leqslant 20 \\
    1 & \text{GDP} \leqslant 8
  \end{cases}
\end{equation}


\subsection{农业火}

农田上发生的火灾燃烧面积为:
\begin{equation}
  A_{\mathrm{b}}=a_{1} f_{\mathrm{s e}} f_{\mathrm{t}} f_{\mathrm{c r o p}} A_{\mathrm{g}}
\end{equation}
%
其中,$a_{1} =4.4\times10^{-7}$ (\unit{s^{-1}})是根据 GFED3 全球农田发生燃烧面积占全球总燃烧面积比例率定的全球常熟; $f_{\mathrm{se}}$ 反映社会经济的影响; $f_{\mathrm {t}} $ 为描述的农业火发生时刻;$f_{\mathrm{crop}}$ 为格点里农田面积比例。


社会经济因子反映的是人口密度越高和经济越发达的地区,用秸秆焚烧的方法处理农业残余的比例越低:
\begin{equation}
  f_{\mathrm{se}} = f_{\mathrm {d}}  f_{\mathrm {e}}
\end{equation}
其中,
\begin{equation}
  f_{\mathrm{d}}=0.04+0.96 \times \exp \left[-\pi\left(\frac{D_{\mathrm{p}}}{350}\right)^{0.5}\right]
\end{equation}
和
\begin{equation}
  f_{\mathrm{e}}=0.01+0.99 \times \exp \left(-\pi \frac{\text{GDP}}{10}\right)
\end{equation}
分别为人口密度和人均 GDP 的函数。


\section{热带密闭林区由人类砍伐引起的火灾}
热带密闭林区(树覆盖率高于 60\%)由人类砍伐引起的火灾即包括发生在砍伐区的火灾(deforestation fires)也包括不受控制蔓延到砍伐区以外的(degradation fires)。其燃烧面积由下列公式计算:
\begin{equation}
  A_{\mathrm{b}}=b f_{\mathrm{l u}} f_{\mathrm{c l i, d}} f_{\mathrm{b}} A_{\mathrm{g}}
\end{equation}
%
其中,$b =3.8\times10^{-7}$ (\unit{s^{-1}})是根据 GFED3 亚马逊雨燃烧面积率定的全球常数, $f_{\mathrm{lu}}$ 为砍伐率影响因子, $f_{\mathrm{cli,d}}$ 为气候影响因子。

砍伐率影响因子:
\begin{equation}
  f_{\mathrm{lu}} = \max(0.0005, 0.19D - 0.001)
\end{equation}
是砍伐率 ($D$,\unit{yr^{-1}})的函数。砍伐率由模式输入的土地利用及土地覆盖变化数据计算得来。
气候影响因子
\begin{equation}
  \begin{aligned}
    f_{\mathrm{c l i, d}}=\max\left[0, \min \left(1, \frac{b_{2}-p_{\rm 60d}}{b_{2}}\right)\right]^{0.5}& \times\max\left[0, \min \left(1, \frac{b_{3}-p_{\rm 10d}}{b_{3}}\right)\right]^{0.5}\\
    & \times\max\left[0, \min \left(1, \frac{0.25-P}{0.25}\right)\right]^{0.5}
  \end{aligned}
\end{equation}
是当前时步降水量 $P$(\unit{mm.d^{-1}}) 和此前 60 天滑动平均降水$p_{\rm 60d}$ (\unit{mm.d^{-1}})及 10 天滑动平均降水$p_{\rm 60d}$ (\unit{mm.d^{-1}})的函数。 $b_2 = b_3$, 对于热带常绿树取 $4.0$ \unit{mm.d^{-1}},对于热带落叶树取 $1.8$ \unit{mm.d^{-1}}.


\section{泥炭火}

泥炭火燃烧面积为:
\begin{equation}
  A_{\mathrm{b}}=c f_{\mathrm{c l i, p}} f_{\mathrm{peat}}\left(1-f_{\mathrm{sat}}\right) A_{\mathrm{g}}
\end{equation}
其 中 , $c$ 是 全 球 常 数 , 对 于 热 带 泥 炭 火 $c=4.7\times10^{-7}$ (\unit{s^{-1}}) 及 对 于 寒 带 泥 炭 火 $c=2.5\times10^{-9}$ (\unit{s^{-1}}) 是基于 GFED3 的泥炭火燃烧面积估算而来; $f_{\mathrm{cli,p}}$ 为气候影响因子, $f_{\mathrm{peat}}$ 为格点内泥炭地面积比例; $f_{\mathrm{sat}}$ 为水位高于地表的面积比例。

对于热带泥炭火,气候影响因子为:
\begin{equation}
  f_{\mathrm{c l i, p}}=\max \left[0, \min \left(1, \frac{4-p_{\rm 60d}}{4}\right)\right]^{2}
\end{equation}
而对于寒带泥炭火,气候影响因子为:
\begin{equation}
  f_{\mathrm{c l i, p}}=\exp \left(-\pi \frac{\theta_{17 c m}}{0.3}\right) \times \max \left[0, \min \left(1, \frac{T_{17 c m}-T_{\mathrm{frz}}}{10}\right)\right]
\end{equation}
其中, $\theta_{\text{17cm}}$ 为表层 17 cm 土壤湿度。

\section{火灾影响}
\subsection{燃烧引起的碳排放}

对于过火区,第 $j$ 个 PFT 生物量燃烧引起的碳排放${\phi}_{j} $(\unit{g.C.s^{-1}.m^{-2}})为:
%
\begin{equation}
  \phi_{j}=\frac{A_{\mathrm{b},j}}{A_{\mathrm{g}}} C_{j} \times C C_{j}
\end{equation}
%
其中, $A_{\mathrm{b},j}/A_{\mathrm {g}} $ 为燃烧面积率(\unit{s^{-1}}); $
C_{j}=\left(C_{\mathrm{leaf}}, C_{\mathrm{stem}}, C_{\mathrm{root}}, C_{\mathrm{t s}}\right)
$是碳库(\unit{g.C.m^{-2}}),包括叶、茎、根、储存库的碳库; $
C C_{j}=\left(C C_{\mathrm{leaf}}, C C_{\mathrm{stem}}, C C_{\mathrm{root}}, C C_{\mathrm{t s}}\right)_{j}
$是燃烧完全因子 (表~\ref{tab:burning_factors})。 此外,我们假设 50\%和 28\%的 litter 和 coarse woody debris (CWD)的碳库被燃烧后排放到大气。


对于热带泥炭火,泥炭燃烧引起的碳排放(\unit{g.C.s^{-1}.m^{-2}})是泥炭火的燃烧面积率(\unit{s^{-1}})、土壤有机碳(\unit{g.C.m^{-2}})、0.18 的乘积;对于寒带泥炭火,泥炭燃烧引起的碳排放是泥炭火的燃烧面积率(\unit{s^{-1}})与2200 (\unit{g.C.m^{-2}})的乘积。以上常数由已有泥炭火研究数值结果推导而来。

对于热带密闭林砍伐引起的火灾,我们假设燃烧先发生在砍伐区,砍伐区燃烧2遍后才会蔓延到砍伐区以外。对于第一部分,我们按照比例 $\min \left(\frac{A_{\mathrm{b}}}{2 A_{\mathrm{g}}}, D\right)$ 从土地类型发生变化引起的碳排放里扣除。对于第二部分引起的燃烧面积 $\max(0, A_{\mathrm {b}} -2DA_{\mathrm {g}} )$,按照其它火灾去计算火灾的影响。


\subsection{火灾引起的植物组织死亡}

对于燃烧后剩余的碳库部分,即
\begin{equation}
  \begin{aligned}
    C_{j 1}^{\prime}=\big(& C_{\text {leaf}}\left(1-C C_{\text {leaf}}\right), C_{\text {livestem}}\left(1-C C_{\text {stem}}\right), \\
    &C_{\text {deadstem}}\left(1-C C_{\text {stem}}\right), C_{\text {root}}\left(1-C C_{\text {root}}\right), C_{\mathrm{t s}}\left(1-C C_{\mathrm{t s}}\right)\big)_{j}
  \end{aligned}
\end{equation}
%
考虑火灾引起植物组织死亡为:
\begin{equation}
  \Psi_{j 1}=\frac{A_{\mathrm{b},j}}{f_{j} A_{\mathrm{g}}} C_{j 1}^{\prime} \times M_{j 1}
\end{equation}
%
其中, $
M_{j 1}=\left(M_{\text{leaf}}, M_{\text{livestem}}, M_{\text{deadstem}}, M_{\text {root}}, M_{\text {ts}}\right)_{j}
$ 是植物组织死亡因子,相应的碳由植物碳库转移到 litter 或 CWD。此外,还有一部分 livestem 的碳库会转移到 deadstem:
\begin{equation}
  \Psi_{j 2}=\frac{A_{\mathrm{b},j}}{f_{j} A_{\mathrm{g}}} C_{\text {livestem }}\left(1-C C_{\text {stem }}\right) M_{\text {livestem}, 2}
\end{equation}
%
其中, $M_{\text{livestem},2}$ 为这部分 livestem 的死亡因子(表~\ref{tab:burning_factors})。


\subsection{火灾引起的痕量气体和气溶胶排放}

对于第$j$个PFT和第$x$种痕量气体和气溶胶,其排放量 (\unit{g.species.s^{-1}})为:
\begin{equation}
  E_{x,j}=E F_{x,j} \frac{\phi_{j}}{[C]}
\end{equation}
其中 $\phi_{j}$ (\unit{g.C.s^{-1}.m^{-2}})为第 $j$ 个 PFT 的火灾引起的碳排放,$E F_{x,j}$为排放因子~\citep{LiF2019},为单位转化常数。


针叶树、其它寒带和温带树木、热带地区树木、灌木和草,燃烧引起的火灾最高排放高度分别为:4.3、3、2.5、2、1 km。

\begin{landscape}
  \begin{table}[htbp]
    \caption{燃烧完全因子及火灾引起的植物组织死亡因子}
    \label{tab:burning_factors}
    \begin{tabular}{lcccccccccc}
      \toprule
      PFT                                   & $CC_{\mathrm{leaf}}$ & $CC_{\mathrm{stem}}$ & $CC_{\mathrm{root}}$ & $CC_{\mathrm{ts}}$ & $M_{\mathrm{leaf}}$ & $M_{\mathrm{livestem,1}}$ & $M_{\mathrm{deadstem}}$ & $M_{\mathrm{root}}$ & $M_{\mathrm{ts}}$ & $M_{\mathrm{livestem,2}}$ \\ \midrule
      \multicolumn{11}{l}{\textbf{Tree:}}  \\
      NET Temperate                         & 0.80                 & 0.27                 & 0.00                 & 0.45               & 0.80                & 0.13                      & 0.13                    & 0.13                & 0.45              & 0.32                      \\
      NET Boreal                            & 0.80                 & 0.30                 & 0.00                 & 0.50               & 0.80                & 0.15                      & 0.15                    & 0.15                & 0.50              & 0.35                      \\
      NDT Boreal                            & 0.80                 & 0.30                 & 0.00                 & 0.50               & 0.80                & 0.15                      & 0.15                    & 0.15                & 0.50              & 0.35                      \\
      BET Tropical                          & 0.80                 & 0.27                 & 0.00                 & 0.45               & 0.80                & 0.13                      & 0.13                    & 0.13                & 0.45              & 0.32                      \\
      BET Temperate                         & 0.80                 & 0.27                 & 0.00                 & 0.45               & 0.80                & 0.13                      & 0.13                    & 0.13                & 0.45              & 0.32                      \\
      BDT Tropical                          & 0.80                 & 0.27                 & 0.00                 & 0.45               & 0.80                & 0.10                      & 0.10                    & 0.10                & 0.35              & 0.25                      \\
      BDT Temperate                         & 0.80                 & 0.27                 & 0.00                 & 0.45               & 0.80                & 0.10                      & 0.10                    & 0.10                & 0.35              & 0.25                      \\
      BDT Boreal                            & 0.80                 & 0.27                 & 0.00                 & 0.45               & 0.80                & 0.13                      & 0.13                    & 0.13                & 0.45              & 0.32                      \\ \hline
      \multicolumn{11}{l}{\textbf{Shrub:}} \\
      BES Temperate                         & 0.80                 & 0.35                 & 0.00                 & 0.55               & 0.80                & 0.17                      & 0.17                    & 0.17                & 0.55              & 0.38                      \\
      BDS Temperate                         & 0.80                 & 0.35                 & 0.00                 & 0.55               & 0.80                & 0.17                      & 0.17                    & 0.17                & 0.55              & 0.38                      \\
      BDS Boreal                            & 0.80                 & 0.35                 & 0.00                 & 0.55               & 0.80                & 0.17                      & 0.17                    & 0.17                & 0.55              & 0.38                      \\ \hline
      \multicolumn{11}{l}{\textbf{Grass:}} \\
      C3  arctic grass                      & 0.80                 & -----                & 0.00                 & 0.80               & 0.80                & -----                     & -----                   & 0.20                & 0.80              & -----                     \\
      C3  grass                             & 0.80                 & -----                & 0.00                 & 0.80               & 0.80                & -----                     & -----                   & 0.20                & 0.80              & -----                     \\
      C4  grass                             & 0.80                 & -----                & 0.00                 & 0.80               & 0.80                & -----                     & -----                   & 0.20                & 0.80              & -----                     \\ \hline
      \multicolumn{11}{l}{\textbf{Crop:}}  \\
      Crop1                                 & 0.80                 & -----                & 0.00                 & 0.80               & 0.80                & -----                     & -----                   & 0.20                & 0.80              & -----                     \\
      Crop2                                 & 0.80                 & -----                & 0.00                 & 0.80               & 0.80                & -----                     & -----                   & 0.20                & 0.80              & -----                     \\ \bottomrule
    \end{tabular}
  \end{table}
\end{landscape}

\chapter{生物地球化学循环预热加速}\label{生物地球化学循环预热加速}
%\addcontentsline{toc}{chapter}{生物地球化学循环预热加速}
\begin{mymdframed}{代码}
  本节对应的代码文件为\texttt{MOD\_BGC\_CNSASU.F90}。
\end{mymdframed}

%\begin{生物地球化学循环预热加速}
生物地球化学循环预热是陆地生态系统碳氮循环模拟必不可少的初始化过程,是众多模式比较计划都采用的标准初始化流程。
生物地球化学循环预热通过重复使用同样的气候强迫场和大气二氧化碳浓度数据,使陆地生态系统碳氮储量达到平衡。
通常来说,不同模型间平衡态碳氮储量的差异往往大于历史时期由于$\rm CO_2$浓度上升引起的碳氮储量变化的不确定性。
因此,生物地球化学循环预热过程十分重要。但由于高纬度地区的低温气候造成土壤分解过慢,
碳氮预热达到平衡态需要上千年的模拟时间,如何加速生物地球化学循环预热过程十分关键。
CoLM生物地球化学循环模块运用半解析预热方法~\citep{xia2012semi}。从碳氮平衡方程入手,求解平衡态的碳氮库大小。


\section{植被生物地球化学循环预热加速}
CoLM植被碳氮循环可以表示为21个碳库和22个氮库的碳氮平衡方程组~\citep{lu2020full},
可以写成矩阵形式:
\begin{equation}
  \frac{{\mathrm {d}} C_{\mathrm{veg}}}{{\mathrm {d}} t}=B I_{\mathrm{Cin}}+\left(A_{\mathrm{p h c}}(t) K_{\mathrm{p h c}}+A_{\mathrm{gmc}}(t) K_{\mathrm{gmc}}\right) C_{\mathrm{veg}}(t)
\end{equation}
\begin{equation}
  \frac{{\mathrm {d}} N_{\text {veg}}}{{\mathrm {d}} t}=B I_{\mathrm{Nin}}+\left(A_{\mathrm{phn}}(t) K_{\mathrm{phn}}+A_{\mathrm{gmn}}(t) K_{\mathrm{gmn}}\right) N_{\text {veg }}(t)
\end{equation}
$C_{\mathrm{veg}}$ 和$N_{\mathrm{veg}}$ 是植被碳氮库的状态变量 (\unit{g.C.m^{-2}} 和 \unit{g.N.m^{-2}}),是长度分别为21和22的列向量,
具体内容见章节~\ref{植被碳氮库结构} 对植被碳氮库的详细介绍。$I_{\mathrm{Cin}}$ 和$I_{\mathrm{Nin}}$ 分别是植被的碳氮输入(\unit{g.C.m^{-2}.s^{-1}} 和 \unit{g.N.m^{-2}.s^{-1}}),
是标量。其中,碳输入来源于净第一性生产力,氮输入来源于生物固氮和植被被动氮吸收。
$B$是分配系数向量,代表植被碳氮输入分配到每个植被碳氮库的比例。
$K_{\mathrm{phc}}$和$K_{\mathrm{gmc}}$是21$\times$21的对角矩阵,分别代表植被每个库因为物候过程和自然死亡过程产生的碳周转速率(\unit{s^{-1}})。
$K_{\mathrm{phn}}$和$K_{\mathrm{gmn}}$是22$\times$22的对角矩阵,分别代表植被每个库因为物候过程和自然死亡过程产生的氮周转速率(\unit{s^{-1}}):
\begin{equation}
  K_{\mathrm{p h c}}=\left(\begin{array}{ccc}k_{\mathrm{p 1}} & \cdots & 0 \\ \vdots & \ddots & \vdots \\ 0 & \cdots & k_{\mathrm{p 21}}\end{array}\right)
\end{equation}
\begin{equation}
  K_{\mathrm{gmc}}=\left(\begin{array}{ccc}k_{\mathrm{g 1}} & \cdots & 0 \\ \vdots & \ddots & \vdots \\ 0 & \cdots & k_{\mathrm{n 21}}\end{array}\right)
\end{equation}
\begin{equation}
  K_{\mathrm{phn}}=\left(\begin{array}{ccc}k_{\mathrm{p 1}} & \cdots & 0 \\ \vdots & \ddots & \vdots \\ 0 & \cdots & k_{\mathrm{p 22}}\end{array}\right)
\end{equation}
\begin{equation}
  K_{\mathrm{gmn}}=\left(\begin{array}{ccc}k_{\mathrm{g 1}} & \cdots & 0 \\ \vdots & \ddots & \vdots \\ 0 & \cdots & k_{\mathrm{n 22}}\end{array}\right)
\end{equation}
$A_{\mathrm{phc}}$和$A_{\mathrm{phn}}$分别是碳氮传输系数矩阵,代表不同植被碳氮库之间的传输比例。

\afterpage{%\clearpage %
  \begin{landscape}
    \enlargethispage{10pt}
    \begin{equation}
      A_{\mathrm{phc}}=\left(\begin{array}{rcccccccccccccccccc}
          -1 & 0 & a_{1,3} & 0 & 0 & 0 & 0 & 0 & 0 & 0 & 0 & 0 & 0 & 0 & 0 & 0 & 0 & 0 & 0 \\
          0 & -1 & 0 & 0 & 0 & 0 & 0 & 0 & 0 & 0 & 0 & 0 & 0 & 0 & 0 & 0 & 0 & 0 & 0 \\
          0 & a_{3,2} & -1 & 0 & 0 & 0 & 0 & 0 & 0 & 0 & 0 & 0 & 0 & 0 & 0 & 0 & 0 & 0 & 0 \\
          0 & 0 & 0 & -1 & 0 & a_{4,6} & 0 & 0 & 0 & 0 & 0 & 0 & 0 & 0 & 0 & 0 & 0 & 0 & 0 \\
          0 & 0 & 0 & 0 & -1 & 0 & 0 & 0 & 0 & 0 & 0 & 0 & 0 & 0 & 0 & 0 & 0 & 0 & 0 \\
          0 & 0 & 0 & 0 & a_{6,5} & -1 & 0 & 0 & 0 & 0 & 0 & 0 & 0 & 0 & 0 & 0 & 0 & 0 & 0 \\
          0 & 0 & 0 & 0 & 0 & 0 & -1 & 0 & a_{7,9} & 0 & 0 & 0 & 0 & 0 & 0 & 0 & 0 & 0 & 0 \\
          0 & 0 & 0 & 0 & 0 & 0 & 0 & -1 & 0 & 0 & 0 & 0 & 0 & 0 & 0 & 0 & 0 & 0 & 0 \\
          0 & 0 & 0 & 0 & 0 & 0 & 0 & a_{9,8} & -1 & 0 & 0 & 0 & 0 & 0 & 0 & 0 & 0 & 0 & 0 \\
          0 & 0 & 0 & 0 & 0 & 0 & a_{10,7} & 0 & 0 & -1 & 0 & a_{10,12} & 0 & 0 & 0 & 0 & 0 & 0 & 0 \\
          0 & 0 & 0 & 0 & 0 & 0 & 0 & 0 & 0 & 0 & -1 & 0 & 0 & 0 & 0 & 0 & 0 & 0 &0 \\
          0 & 0 & 0 & 0 & 0 & 0 & 0 & 0 & 0 & 0 & a_{12,11}&-1 & 0 & 0 & 0 & 0 & 0 & 0 & 0\\
          0 & 0 & 0 & 0 & 0 & 0 & 0 & 0 & 0 & 0 & 0 & 0 & -1 & 0 & a_{13,15} & 0 & 0 & 0 & 0\\
          0 & 0 & 0 & 0 & 0 & 0 & 0 & 0 & 0 & 0 & 0 & 0 & 0 & -1 & 0 & 0 & 0 & 0 & 0 \\
          0 & 0 & 0 & 0 & 0 & 0 & 0 & 0 & 0 & 0 & 0 & 0 & 0 & a_{15,14}&-1 & 0 & 0 & 0 & 0\\
          0 & 0 & 0 & 0 & 0 & 0 & 0 & 0 & 0 & 0 & 0 & 0 & 0 & a_{16,13} & 0 & 0 & -1 & 0 & a_{16,18} \\
          0 & 0 & 0 & 0 & 0 & 0 & 0 & 0 & 0 & 0 & 0 & 0 & 0 & 0 & 0 & 0 & 0&-1 & 0\\
      0 & 0 & 0 & 0 & 0 & 0 & 0 & 0 & 0 & 0 & 0 & 0 & 0 & 0 & 0 & 0 & 0 &  a_{18,17} & -1\end{array}\right)
    \end{equation}
  \end{landscape}
}

%\afterpage{%\clearpage %
\begin{landscape}
  \enlargethispage{55pt}
  \begin{equation}
    A_{\mathrm{phn}}=\left(\begin{array}{rcccccccccccccccccc}
        -1 & 0 & a_{1,3} & 0 & 0 & 0 & 0 & 0 & 0 & 0 & 0 & 0 & 0 & 0 & 0 & 0 & 0 & 0 & a_{1,19} \\
        0 & -1 & 0 & 0 & 0 & 0 & 0 & 0 & 0 & 0 & 0 & 0 & 0 & 0 & 0 & 0 & 0 & 0 & a_{2,19} \\
        0 & a_{3,2} & -1 & 0 & 0 & 0 & 0 & 0 & 0 & 0 & 0 & 0 & 0 & 0 & 0 & 0 & 0 & 0 & a_{3,19} \\
        0 & 0 & 0 & -1 & 0 & a_{4,6} & 0 & 0 & 0 & 0 & 0 & 0 & 0 & 0 & 0 & 0 & 0 & 0 & a_{4,19} \\
        0 & 0 & 0 & 0 & -1 & 0 & 0 & 0 & 0 & 0 & 0 & 0 & 0 & 0 & 0 & 0 & 0 & 0 & a_{5,19} \\
        0 & 0 & 0 & 0 & a_{6,5} & -1 & 0 & 0 & 0 & 0 & 0 & 0 & 0 & 0 & 0 & 0 & 0 & 0 & a_{6,19} \\
        0 & 0 & 0 & 0 & 0 & 0 & -1 & 0 & a_{7,9} & 0 & 0 & 0 & 0 & 0 & 0 & 0 & 0 & 0 & a_{7,19} \\
        0 & 0 & 0 & 0 & 0 & 0 & 0 & -1 & 0 & 0 & 0 & 0 & 0 & 0 & 0 & 0 & 0 & 0 & a_{8,19} \\
        0 & 0 & 0 & 0 & 0 & 0 & 0 & a_{9,8} & -1 & 0 & 0 & 0 & 0 & 0 & 0 & 0 & 0 & 0 & a_{9,19} \\
        0 & 0 & 0 & 0 & 0 & 0 & a_{10,7} & 0 & 0 & -1 & 0 & a_{10,12} & 0 & 0 & 0 & 0 & 0 & 0 & a_{10,19} \\
        0 & 0 & 0 & 0 & 0 & 0 & 0 & 0 & 0 & 0 & -1 & 0 & 0 & 0 & 0 & 0 & 0 & 0 & a_{11,19} \\
        0 & 0 & 0 & 0 & 0 & 0 & 0 & 0 & 0 & 0 & a_{12,11}&-1 & 0 & 0 & 0 & 0 & 0 & 0 & a_{12,19} \\
        0 & 0 & 0 & 0 & 0 & 0 & 0 & 0 & 0 & 0 & 0 & 0 & -1 & 0 & a_{13,15} & 0 & 0 & 0 & a_{13,19} \\
        0 & 0 & 0 & 0 & 0 & 0 & 0 & 0 & 0 & 0 & 0 & 0 & 0 & -1 & 0 & 0 & 0 & 0 & a_{14,19} \\
        0 & 0 & 0 & 0 & 0 & 0 & 0 & 0 & 0 & 0 & 0 & 0 & 0 & a_{15,14}&-1 & 0 & 0 & 0 & a_{15,19} \\
        0 & 0 & 0 & 0 & 0 & 0 & 0 & 0 & 0 & 0 & 0 & 0 & a_{16,13} & 0 & 0 & -1 & 0 & a_{16,18}& a_{16,19} \\
        0 & 0 & 0 & 0 & 0 & 0 & 0 & 0 & 0 & 0 & 0 & 0 & 0 & 0 & 0 & 0 & -1 & 0 & a_{17,19} \\
        0 & 0 & 0 & 0 & 0 & 0 & 0 & 0 & 0 & 0 & 0 & 0 & 0 & 0 & 0 & 0 & a_{18,17}&-1 & a_{18,19} \\
        a_{19,1} & 0 & 0 & a_{19,4} & 0 & 0 & a_{19,7} & 0 & 0 & a_{19,10} & 0 & 0 & 0 & 0 & 0 & 0 & 0 & 0 & -1
    \end{array}\right)
  \end{equation}
\end{landscape}
%}
\begin{landscape}
  \enlargethispage{55pt}
  \begin{equation}
    A_{\mathrm{gmc}}=\left(\begin{array}{rrrrrrrrrrrrrrrrrrrrrrrrrrrrrr}
        -1 & 0 & 0 & 0 & 0 & 0 & 0 & 0 & 0 & 0 & 0 & 0 & 0 & 0 & 0 & 0 & 0 & 0 \\
        0 & -1 & 0 & 0 & 0 & 0 & 0 & 0 & 0 & 0 & 0 & 0 & 0 & 0 & 0 & 0 & 0 & 0 \\
        0 & 0 & -1 & 0 & 0 & 0 & 0 & 0 & 0 & 0 & 0 & 0 & 0 & 0 & 0 & 0 & 0 & 0 \\
        0 & 0 & 0 & -1 & 0 & 0 & 0 & 0 & 0 & 0 & 0 & 0 & 0 & 0 & 0 & 0 & 0 & 0 \\
        0 & 0 & 0 & 0 & -1 & 0 & 0 & 0 & 0 & 0 & 0 & 0 & 0 & 0 & 0 & 0 & 0 & 0 \\
        0 & 0 & 0 & 0 & 0 & -1 & 0 & 0 & 0 & 0 & 0 & 0 & 0 & 0 & 0 & 0 & 0 & 0 \\
        0 & 0 & 0 & 0 & 0 & 0 & -1 & 0 & 0 & 0 & 0 & 0 & 0 & 0 & 0 & 0 & 0 & 0 \\
        0 & 0 & 0 & 0 & 0 & 0 & 0 & -1 & 0 & 0 & 0 & 0 & 0 & 0 & 0 & 0 & 0 & 0 \\
        0 & 0 & 0 & 0 & 0 & 0 & 0 & 0 & -1 & 0 & 0 & 0 & 0 & 0 & 0 & 0 & 0 & 0 \\
        0 & 0 & 0 & 0 & 0 & 0 & 0 & 0 & 0 & -1 & 0 & 0 & 0 & 0 & 0 & 0 & 0 & 0 \\
        0 & 0 & 0 & 0 & 0 & 0 & 0 & 0 & 0 & 0 & -1 & 0 & 0 & 0 & 0 & 0 & 0 & 0 \\
        0 & 0 & 0 & 0 & 0 & 0 & 0 & 0 & 0 & 0 & 0 & -1 & 0 & 0 & 0 & 0 & 0 & 0 \\
        0 & 0 & 0 & 0 & 0 & 0 & 0 & 0 & 0 & 0 & 0 & 0 & -1 & 0 & 0 & 0 & 0 & 0 \\
        0 & 0 & 0 & 0 & 0 & 0 & 0 & 0 & 0 & 0 & 0 & 0 & 0 & -1 & 0 & 0 & 0 & 0 \\
        0 & 0 & 0 & 0 & 0 & 0 & 0 & 0 & 0 & 0 & 0 & 0 & 0 & 0 & -1 & 0 & 0 & 0 \\
        0 & 0 & 0 & 0 & 0 & 0 & 0 & 0 & 0 & 0 & 0 & 0 & 0 & 0 & 0 & -1 & 0 & 0 \\
        0 & 0 & 0 & 0 & 0 & 0 & 0 & 0 & 0 & 0 & 0 & 0 & 0 & 0 & 0 & 0 & -1 & 0 \\
        0 & 0 & 0 & 0 & 0 & 0 & 0 & 0 & 0 & 0 & 0 & 0 & 0 & 0 & 0 & 0 & 0  & -1
    \end{array}\right)
  \end{equation}
\end{landscape}

\begin{landscape}
  \enlargethispage{55pt}
  \begin{equation}
    A_{\mathrm{gmn}}=\left(\begin{array}{rrrrrrrrrrrrrrrrrrrrrrrrrrrrrr}
        -1 & 0 & 0 & 0 & 0 & 0 & 0 & 0 & 0 & 0 & 0 & 0 & 0 & 0 & 0 & 0 & 0 & 0 \\
        0 & -1 & 0 & 0 & 0 & 0 & 0 & 0 & 0 & 0 & 0 & 0 & 0 & 0 & 0 & 0 & 0 & 0 \\
        0 & 0 & -1 & 0 & 0 & 0 & 0 & 0 & 0 & 0 & 0 & 0 & 0 & 0 & 0 & 0 & 0 & 0 \\
        0 & 0 & 0 & -1 & 0 & 0 & 0 & 0 & 0 & 0 & 0 & 0 & 0 & 0 & 0 & 0 & 0 & 0 \\
        0 & 0 & 0 & 0 & -1 & 0 & 0 & 0 & 0 & 0 & 0 & 0 & 0 & 0 & 0 & 0 & 0 & 0 \\
        0 & 0 & 0 & 0 & 0 & -1 & 0 & 0 & 0 & 0 & 0 & 0 & 0 & 0 & 0 & 0 & 0 & 0 \\
        0 & 0 & 0 & 0 & 0 & 0 & -1 & 0 & 0 & 0 & 0 & 0 & 0 & 0 & 0 & 0 & 0 & 0 \\
        0 & 0 & 0 & 0 & 0 & 0 & 0 & -1 & 0 & 0 & 0 & 0 & 0 & 0 & 0 & 0 & 0 & 0 \\
        0 & 0 & 0 & 0 & 0 & 0 & 0 & 0 & -1 & 0 & 0 & 0 & 0 & 0 & 0 & 0 & 0 & 0 \\
        0 & 0 & 0 & 0 & 0 & 0 & 0 & 0 & 0 & -1 & 0 & 0 & 0 & 0 & 0 & 0 & 0 & 0 \\
        0 & 0 & 0 & 0 & 0 & 0 & 0 & 0 & 0 & 0 & -1 & 0 & 0 & 0 & 0 & 0 & 0 & 0 \\
        0 & 0 & 0 & 0 & 0 & 0 & 0 & 0 & 0 & 0 & 0 & -1 & 0 & 0 & 0 & 0 & 0 & 0 \\
        0 & 0 & 0 & 0 & 0 & 0 & 0 & 0 & 0 & 0 & 0 & 0 & -1 & 0 & 0 & 0 & 0 & 0 \\
        0 & 0 & 0 & 0 & 0 & 0 & 0 & 0 & 0 & 0 & 0 & 0 & 0 & -1 & 0 & 0 & 0 & 0 \\
        0 & 0 & 0 & 0 & 0 & 0 & 0 & 0 & 0 & 0 & 0 & 0 & 0 & 0 & -1 & 0 & 0 & 0 \\
        0 & 0 & 0 & 0 & 0 & 0 & 0 & 0 & 0 & 0 & 0 & 0 & 0 & 0 & 0 & -1 & 0 & 0 \\
        0 & 0 & 0 & 0 & 0 & 0 & 0 & 0 & 0 & 0 & 0 & 0 & 0 & 0 & 0 & 0 & -1 &0 \\
    0 & 0 & 0 & 0 & 0 & 0 & 0 & 0 & 0 & 0 & 0 & 0 & 0 & 0 & 0 & 0 & 0 & -1\end{array}\right)
  \end{equation}
\end{landscape}
其中,矩阵的非对角元素 $a_{(i,j)}$, 代表从$j$到$i$的碳氮传输在$j$库总周转掉的碳氮中的比例。


当碳氮平衡方程中$({\mathrm {d}} C_{\mathrm{veg}})/{\mathrm {d}} t=0$和 $({\mathrm {d}} N_{\mathrm{veg}})/{\mathrm {d}} t=0$,则碳氮处于平衡状态,所以可以解得植被碳氮的平衡状态($C_{\mathrm{(veg,ss)}}$, $N_{\mathrm{(veg,ss)}}$)是:
\begin{equation}
  C_{\mathrm{veg, s s}}=\left(A_{\mathrm{p h c}}(t) K_{\mathrm{p h c}}+A_{\mathrm{gmc}}(t) K_{\mathrm{gmc}}\right)^{-1} B I_{\mathrm{{Cin }}}
\end{equation}
\begin{equation}
  N_{\mathrm{veg, s s}}=\left(A_{\mathrm{phn}}(t) K_{\mathrm{phn}}+A_{\mathrm{gmn}}(t) K_{\mathrm{gmn}}\right)^{-1} B I_{\mathrm{Nin}}
\end{equation}
由于生物地球化学循环运用循环的气象强迫场数据,所以平衡状态需要累加计算一个气象循环的平均值。


\section{土壤凋落物生物地球化学循环预热加速}
CoLM土壤凋落物的碳氮循环可以表示为7个碳库和7个氮库的碳氮平衡方程组~\citep{lu2020full},其中每个库在土壤垂直方向分为10层,可以写成矩阵形式:
\begin{equation}
  \frac{{\mathrm {d}} C_{\text {soil }}}{{\mathrm {d}} t}=I_{\mathrm{C \text { soil }}}+\left[A_{\mathrm{h c}} \xi(t) K_{\mathrm{h}}+V(t)\right] C_{\text {soil }}(t)
\end{equation}
\begin{equation}
  \frac{{\mathrm {d}} N_{\text {soil }}}{{\mathrm {d}} t}=I_{\mathrm{N s o i l}}+\left[A_{\mathrm{h n}} \xi(t) K_{\mathrm{h}}+V(t)\right] N_{\text {soil }}(t)
\end{equation}
$C_{\mathrm{soil}}$和$N_{\mathrm{soil}}$是CoLM土壤凋落物有机碳氮库的状态变量(\unit{g.C.m^{-3}} 和 \unit{g.N.m^{-3}}), 是向量。
$I_{\mathrm{Csoil}}$和$I_{\mathrm{Nsoil}}$同样是长度为70的向量,代表进入土壤碳氮库中不同分层的凋落物输入。
$A_{\mathrm{hc}}$ 和$A_{\mathrm{hn}}$代表土壤碳氮在同一层内不同库之间的传输系数。$V$土壤碳氮在不同层之间的传输系数:
\begin{equation}
  A_{\mathrm{h c}} \text { or } A_{\mathrm{h n}}=\left(\begin{array}{ccccccc}
      A_{11} & 0 & 0 & 0 & 0 & 0 & 0 \\
      0 & A_{22} & 0 & 0 & 0 & 0 & 0 \\
      A_{31} & 0 & A_{33} & 0 & 0 & 0 & 0 \\
      A_{41} & 0 & 0 & A_{44} & 0 & 0 & 0 \\
      0 & A_{52} & A_{53} & 0 & A_{55} & A_{56} & A_{57} \\
      0 & 0 & 0 & A_{64} & A_{65} & A_{66} & 0 \\
  0 & 0 & 0 & 0 & A_{75} & A_{76} & A_{77}\end{array}\right)
\end{equation}
其中每个子矩阵 $A_{ij}$ 都是10$\times$10的对角矩阵:

\begin{equation}
  A_{i j}=\left\{\begin{array}{ccc}{\left[\begin{array}{ccc}-1 & \cdots & 0 \\ \vdots & \ddots & \vdots \\ 0 & \cdots & -1\end{array}\right]}   &i=j;\ C\ and\ N\ cycles \\
      {\left[\begin{array}{ccc}\left(1-r f_{j}\right) T_{i j} & \cdots & \\ \vdots & \ddots & \vdots \\ 0 & \cdots & \left(1-r f_{j}\right) T_{i j}\end{array}\right]}   &\ i\neq j;\ C\ cycle\ \left(A_{hc}\right)\  \\
  {\left[\begin{array}{cccc}\left(1-r f_{j}\right) T_{i j} \frac{C N_{j}(1)}{C N_{i}(1)} & \cdots & 0 \\ \vdots & \ddots & \vdots \\ 0 & & \cdots & \left(1-r f_{j}\right) T_{i j} \frac{C N_{j}(10)}{C N_{i}(10)}\end{array}\right]}& i\neq j;\ N\ cycle\left(A_{\mathrm{hn}}\right) \end{array}\right.
\end{equation}


对角线上的子矩阵 $A_{ii}$是负的单位矩阵。非对角子矩阵$A_{ij}$代表10层土壤中的传输系数的差异。
$r_{ij}$, $T_{ij}$,$CN_{j(z)}$以及环境控制因子$\xi$在第~\ref{土壤凋落物生物地球化学循环过程} 章都有过详细介绍。

周转速率矩阵$K_{\mathrm {h}} $是对角矩阵,代表土壤分解过程在不同土壤分层引起的碳氮周转:
\begin{equation}
  K_{\mathrm{h}}=\left(\begin{array}{ccc}k_{1} & \cdots & 0 \\ \vdots & \ddots & \vdots \\ 0 & \cdots & k_{\mathrm{n}}\end{array}\right)
\end{equation}
垂直传输系数矩阵V是由6个同样的三对角矩阵组成$v$:
\begin{equation}
  V(t)=\left(\begin{array}{ccccccc}v & 0 & 0 & 0 & 0 & 0 & 0 \\ 0 & v & 0 & 0 & 0 & 0 & 0 \\ 0 & 0 & v & 0 & 0 & 0 & 0 \\ 0 & 0 & 0 & 0 & 0 & 0 & 0 \\ 0 & 0 & 0 & 0 & v & 0 & 0 \\ 0 & 0 & 0 & 0 & 0 & v & 0 \\ 0 & 0 & 0 & 0 & 0 & 0 & v\end{array}\right)
\end{equation}
其中CoLM的粗木质残体库不存在土壤的垂直传输,所以其对应的矩阵$v$是0。
其余库对应的矩阵都是三对角矩阵,意味着碳氮的垂直传输只发生在相邻的两层土壤之间:
\begin{equation}
  \begin{array}{c}v=\operatorname{diag}\left({\mathrm {d}} z_{1},{\mathrm { d}} z_{2}, \ldots,{\mathrm {  d}} z_{20}\right)^{-1} \\
    \left(\begin{array}{ccccccc}g_{1} & -g_{1} & 0 & & 0 & 0 & 0 \\
        -h_{2} & h_{2}+g_{2} & -g_{2} & \ldots & 0 & 0 & 0 \\ 0 & -h_{3} & h_{3}+g_{3} & & 0 & 0 & 0 \\
        & \vdots & & \ddots & & \vdots & \\ 0 & 0 & 0 & & h_{18}+g_{18} & -g_{18} & 0 \\
  0 & 0 & 0 & \ldots & -h_{19} & h_{19}+g_{19} & -g_{19} \\ 0 & 0 & 0 & & 0 & -h_{10} & h_{20}\end{array}\right)\end{array}
\end{equation}
其中${\mathrm {d}} z$代表每层土壤的厚度,$g$和$h$是土壤碳氮的上行和下行垂直传输系数。

当碳氮平衡方程中$({\mathrm {d}} C_{\mathrm{soil}})/{\mathrm {d}} t=0$和 $({\mathrm {d}} N_{\mathrm{soil}})/{\mathrm {d}} t=0$,则碳氮处于平衡状态,
所以可以解得植被碳氮的平衡状态($C_{\mathrm{(soil,ss)}}$, $N_{\mathrm{(soil,ss)}}$)是:
%
\begin{equation}
  C_{\mathrm{soil, s s}}=\left[A_{\mathrm{h c}} \xi(t) K_{\mathrm{h}}+V(t)\right]^{-1} I_{\mathrm{C s o i l}}
\end{equation}
\begin{equation}
  N_{\mathrm{soil, s s}}=\left[A_{\mathrm{h n}} \xi(t) K_{\mathrm{h}}+V(t)\right]^{-1} I_{\mathrm{N s o i l}}
\end{equation}
同样的,由于生物地球化学循环运用循环的气象强迫场数据,
所以土壤碳氮的平衡状态也需要累加计算一个气象循环的平均值。 


\part{人类活动}{Human Activities}\label{part:human}
%\epart{Human Activities}
\chapter{城市模式}\label{城市模式}
%\addcontentsline{toc}{chapter}{城市模式}
CoLM城市模式不同于传统城市冠层街谷 (canyon) 假设,而是以三维城市建筑群落为基本假设,
在此基础上计算三维城市建筑群落短波、长波辐射传输及湍流交换过程。在辐射传输和湍流交换计算过程中,
添加对植被的模拟。第二个特点是高分辨率城市覆盖、植被、水体等数据应用。
另外,城市模式加入了人为热过程,包括建筑能耗、交通热、新陈代谢热的模拟。


\section{城市模式结构}
不同于传统的城市街谷假设模型,CoLM采用由单栋建筑物组成的建筑群落来表征城市,如图~\ref{fig:CoLM城市模式总体结构示意图} 所示。
建筑物随机分布,包括建筑物位置及朝向。城市几何参数包括建筑物覆盖度$f_b$,地面(透水、不透水面)覆盖度
$f_g$($f_{gimp}$,$f_{gper}$),建筑物高度$H$,建筑物高度与平均地面宽度比$H/W$($W$表示楼与楼之间平均距离),
植被冠层中心高度$H_v$,城市树木叶面积指数 LAI 及茎面积指数 SAI,植被(树)覆盖面积比例记为$f_v$。
其中$f_b+f_g=1$,$f_{gimp}+f_{gper}=1$。城市水体覆盖单独表征,其过程考虑为等效湖泊进行计算。
{
\begin{figure}[]
\centering
\includegraphics{Figures/城市模式/CoLM城市模式总体结构示意图.png}
\caption{CoLM城市模式总体结构示意图。}
\label{fig:CoLM城市模式总体结构示意图}
\end{figure}
}

建筑物底面考虑为正方形,边长为记为$L$。在已知$f_b$和$H/W$参数时,$L$可以计算为:
\begin{equation}
\frac{H}{L}=\frac{H}{W} \cdot \frac{1-\sqrt{f_{b}}}{\sqrt{f_{b}}}, \text { 即 } L=W \frac{\sqrt{f_{b}}}{1-\sqrt{f_{b}}}
\end{equation}
对于同一所在地城市覆盖,单栋建筑物几何参数一致,目前未考虑建筑物之间的几何差异,即以上参数代表为统计平均值。

在以下分模块公式推导中,下标约定如下:天空($s$),地面($g$,包括透水面$gper$和不透水面$gimp$),
墙面($w$,包括阳面墙$wsun$和阴面墙$wsha$),植被($v$)和屋顶$(r$)。$F$表示可视因子,例如$F_{gs}$表示从地面到天空的可视因子。
$S$表示阴影,$H/W$在推导过程中表示为$HW$,$H/L$表示为$HL$,总的直射入射和漫射入射太阳辐射能量通量采用单位能量1表示。

{
\begin{figure}[]
\centering
\includegraphics{Figures/城市模式/CoLM城市模式几何结构示意图.png}
\caption{CoLM城市模式几何结构示意图。}
\label{fig:CoLM城市模式几何结构示意图}
\end{figure}
}
\section{短波辐射传输}\label{短波辐射传输}
\subsection{无植被覆盖时短波辐射传输}\label{无植被覆盖时短波辐射传输}
太阳辐射直射照射时(太阳天顶角$\theta_s$),
建筑物群落在地面的阴影覆盖占比(被墙面遮挡的部分)计算为:
\begin{equation}\label{S_w}
S_{w}=1-\exp \left(-\frac{4}{\pi} \cdot \frac{f_{b}}{f_{g}} {HL} \cdot \tan \left(\theta_{s}\right)\right)
\end{equation}
阳面墙面积比例计算为:
\begin{equation}\label{f_wsun}
f_{wsun}=0.5 \cdot \frac{s_{w} f_{g}+f_{b}}{\frac{4}{\pi} f_{b} {HL} \tan \left(\theta_{s}\right)+f_{b}}
\end{equation}
阴面墙比例:
\begin{equation}\label{f_wsha}
f_{wsha}=1-f_{wsun}
\end{equation}
对于天空到墙面的可视因子$F_{sw}$,即天空漫射光源到达墙面的辐射部分,其计算类似与直射光在被墙面遮挡的面积比例:
\begin{equation}\label{f_Sw}
F_{S w}=1-\exp \left(-\frac{4}{\pi} \cdot \frac{f_{b}}{f_{g}} {HL} \cdot \tan \left(\theta_{s}^{*}\right)\right)
\end{equation}
其中$\theta_s^\ast$为漫射光情况下等效角度,可近似计算为:
\begin{equation}
\theta_{s}^{*}=\frac{53-10 \sqrt{\frac{f_{b}}{f_{g}} {HL}}}{180} \cdot \pi
\end{equation}
根据能量守恒,天空到达地面的可视因子$F_{sg}=1-F_{sw}$。
根据可视因子对称性,地面到达墙面和天空的可视因子$F_{gw}=F_{sw}$,$F_{gs}=F_{sg}$。
墙面到天空和地面的可视因子($F_{ws}$, $F_{wg}$)根据互易原理,满足如下关系:
\begin{equation}
F_{ws} \cdot 4 {HL} f_{b}=F_{s w} f_{g}
\end{equation}
\begin{equation}
F_{w g} \cdot 4 {HL} f_{b}=F_{s w} f_{g}
\end{equation}
由于太阳辐射在墙面垂直分布并不均匀,对上式进行简单调整,$F_{ws}$和$F_{wg}$计算为:
\begin{equation}
F_{ws}=0.75 F_{s w} \frac{f_{g}}{f_{b}} \frac{1}{2 {HL}}
\end{equation}
\begin{equation}
F_{w g}=0.25 F_{s w} \frac{f_{g}}{f_{b}} \frac{1}{2 {HL}}
\end{equation}
墙面到墙面的可视因子$F_{ww}=1-F_{ws}-F_{wg}$。

a. 直射入射太阳辐射

对于直射辐射太阳辐射,直射光到达阳面墙的辐射通量$E_{wsun}=S_w$,
到达阴面墙的辐射通量$E_{wsha}=0$,到达地面的辐射通量$E_g=1-E_{wsun}$,
其中$E_{gimp}=E_gf_{gimp}$,$E_{gper}=E_gf_{gper}$。
阳面墙、阴面墙、不透水面和透水面第一次散射辐射通量可分别计算为:
$E_{wsun}\alpha_w,E_{wsha}\alpha_w,E_{gimp}\alpha_{gimp},E_{gper}\alpha_{gper}$,
其中$\alpha$表示墙面、不透水面和透水面的反照率,不分区阳面和阴面,不区分漫射和直射照射,且反射后的辐射假设为漫射辐射。
设经过墙面地面之间多次散射,达到辐射平衡时的墙面和地面辐射出射通量为$M$,可建立如下辐射平衡方程:
\begin{landscape}
\begin{equation}
\begin{array}{l}
    M_{ {wsun }}=E_{wsun} \alpha_{w}+M_{wsun} F_{ww} f_{wsun} \alpha_{w}+M_{wsha} F_{ww} f_{wsun} \alpha_{w}+M_{gimp} F_{g w} f_{wsun} \alpha_{w}+M_{gper} F_{g w} f_{wsun} \alpha_{w} \\ 
    M_{wsha}=E_{wsha} \alpha_{w}+M_{wsha} F_{ww} f_{wsha} \alpha_{w}+M_{wsun} F_{ww} f_{wsha} \alpha_{w}+M_{gimp} F_{g w} f_{wsha} \alpha_{w}+M_{gper} F_{g w} f_{wsha} \alpha_{w} \\ 
    M_{gimp}=E_{gimp} \alpha_{gimp}+M_{wsun} F_{w g} f_{ gimp} \alpha_{gimp}+M_{wsha} F_{w g} f_{ gimp} \alpha_{gimp} \\ 
    M_{gper}=E_{gper} \alpha_{gper}+M_{wsun} F_{w g} f_{gper} \alpha_{gper}+M_{wsha} F_{w g} f_{gper} \alpha_{gper}
\end{array}
\end{equation}
通过整理得到:
\begin{equation}
\left(\begin{array}{cccc}1-F_{ww} f_{wsun} \alpha_{w} & -F_{ww} f_{wsun} \alpha_{w} & -F_{g w} f_{wsun} \alpha_{w} & -F_{g w} f_{wsun} \alpha_{w} \\
    -F_{ww} f_{wsha} \alpha_{w} & 1-F_{ww} f_{wsha} \alpha_{w} & -F_{g w} f_{wsha} \alpha_{w} & -F_{g w} f_{wsha} \alpha_{w} \\
    -F_{w g} f_{gimp} \alpha_{gimp} & -F_{w g} f_{gimp} \alpha_{gimp} & 1 & 0 \\ 
    -F_{w g} f_{gper} \alpha_{gper} & -F_{w g} f_{gper} \alpha_{gper} & 0 & 1\end{array}\right)
    \left(\begin{array}{l}M_{wsun} \\ M_{ {wsha }} \\ 
    M_{ gimp} \\ 
    M_{gper}\end{array}\right)=\left(\begin{array}{c}E_{wsun} \alpha_{w} \\
    E_{wsha} \alpha_{w} \\
    E_{gimp} \alpha_{ gimp} \\
    E_{gper} \alpha_{gper}\end{array}\right)
\end{equation}
\end{landscape}

简单表示为
\begin{equation}\label{mathbf_AX}
\mathbf{A X}=\mathbf{B}
\end{equation}
$\mathbf{A}$为辐射传输矩阵,$\mathbf{B}$为向量($E_{wsun}\alpha_w$,$E_{wsha}\alpha_w$,
$E_{gimp}\alpha_{gimp}$,$E_{gper}\alpha_{gper}$),
即墙面和地面第一次散射辐射通量。$\mathbf{X}$为求解变量组成的向量($M_{wsun}$,$M_{wsha}$,$M_{gimp}$,$M_{gper}$),
计算为:
\begin{equation}\label{mathbf_X}
\mathbf{X}=\mathbf{A}^{-1} \mathbf{B}
\end{equation}
矩阵$\mathbf{A}$是由已知变量组成的常数矩阵,可直接进行求逆计算,从而计算得到$\mathbf{X}$。墙面和地面的辐射吸收计算为:
\begin{equation}\label{s_wsun_wsha_gimp_gper_1}
\begin{array}{c}s_{wsun}=\frac{M_{wsun}}{\alpha_{w}}\left(1-\alpha_{w}\right) \\ 
    s_{wsha}=\frac{M_{wsha}}{\alpha_{w}}\left(1-\alpha_{w}\right) \\
    s_{gimp}=\frac{M_{gimp}}{\alpha_{gimp}}\left(1-\alpha_{gimp}\right) \\
    s_{gper}=\frac{M_{gper}}{\alpha_{gper}}\left(1-\alpha_{gper}\right)\end{array}
\end{equation}
注意,以上辐射吸收为总吸收量,其各自单面面积辐射吸收量修改为:
\begin{equation}\label{s_wsun_wsha_gimp_gper_2}
\begin{array}{c}s_{wsun}=\frac{f_{g}}{4 f_{wsun} \mathrm{HL} f_{b}} s_{wsun} \\ 
    s_{wsha}=\frac{f_{g}}{4 f_{wsha} \mathrm{HL} f_{b}} s_{wsha} \\ 
    s_{gimp}=\frac{1}{f_{gimp}} s_{gimp} \\ 
    s_{gper}=\frac{1}{f_{gper}} s_{gper}\end{array}
\end{equation}
不考虑屋顶时的城市反照率为:
\begin{equation}\label{alpha_u}
\alpha_{u}=M_{wsun} F_{ws}+M_{wsha} F_{ws}+M_{gimp} F_{gs}+M_{gper} F_{gs}
\end{equation}
屋顶的吸收计算为$s_{roof}=1-\alpha_{roof}$,考虑屋顶的城市反照率(\ref{alpha_u})修订为:
\begin{equation}\label{alpha_u2}
\alpha_{u}=\alpha_{{roof }} f_{b}+\alpha_{u} f_{g}
\end{equation}

b.漫射入射太阳辐射
对于漫射,计算过程与直射入射辐射过程基本类似,不同之处在于首次到达墙面和地面的辐射通量分别为:
$F_{sw}f_{wsun}$、$F_{sw}f_{wsha}$、$F_{sg}f_{gimp}$、$F_{sg}f_{gper}$,即公式(\ref{mathbf_AX})中的$\mathbf{B}$。
漫射入射辐射的传输矩阵同直射入射辐射情景,即A。其求逆后的结果可以直接用于漫射入射辐射,利用公式(\ref{mathbf_X})计算得到墙面和地面的辐射出射通量,
单位面积的辐射吸收和城市反照率同公式(\ref{alpha_u})和(\ref{alpha_u2}),在此不再列出。

\subsection{有植被覆盖时短波辐射传输}\label{有植被覆盖时短波辐射传输}
城市中考虑植被后的短波辐射传输过程是在无植被辐射传输的基础上,考虑植被在内的各组分之间的可视因子,
从而计算辐射平衡时的传输矩阵,采用类似的方法求解墙面、地面和植被的辐射吸收。
相比章节 \ref{无植被覆盖时短波辐射传输} 增加的部分为考虑植被在内的可视因子和阴影面积计算。

在植被树冠中心高度$h_v$处,墙面的阴影比例可根据公式(\ref{S_w})计算,利用$\frac{H-h_v}{H}$代替$HL$,记为$S_w^\prime$,即
\begin{equation}
S_{w}^{\prime}=1-\exp \left(-\frac{4}{\pi} \cdot \frac{f_{b}}{f_{g}} \frac{H-h_{v}}{H} \cdot \tan \left(\theta_{s}\right)\right)
\end{equation}
假设该阴影部分与植被覆盖$f_v$随机重叠,则植被被阴影遮挡的面积比例为$f_vS_w^\prime$,未覆盖的植被覆盖面积占比为$f_v^\prime=f_v-f_vS_w^\prime$。
该未被覆盖植被在整个城市覆盖区域的阴影覆盖$S_v$可根据公式(\ref{S_area})计算得到。将其覆盖比例转化为地面所占比例为:
\begin{equation}\label{S_v2}
S_{v}=S_{v} / f_{g}
\end{equation}

假设墙面在地面的阴影占比与植被的阴影随机重叠,则重叠部分占比$S_{vw}$计算为:
\begin{equation}
S_{v w}=S_{v}\left(S_{w}-S_{w}^{\prime}\right)
\end{equation}
该重叠部分假设全部为植被遮挡墙面部分。在地面覆盖区域,墙面的阴影占比修正为:
\begin{equation}\label{S_w2}
S_{w}=S_{w}-S_{v w}
\end{equation}
阴面墙和阳面墙的面积占比计算同公式(\ref{f_wsun})和(\ref{f_wsha})。

在不考虑植被存在时,可视因子$F_{sw}$, $F_{sg}$, $F_{gw}$, $F_{gs}$, $F_{ws}$, $F_{wg}$ 和 $F_{ww}$的
计算同章节 \ref{无植被覆盖时短波辐射传输},下面需要考虑植被的遮挡,对以上可视因子进行修改,并添加植被到各个面的可视因子。

同公式(\ref{S_v2})和(\ref{S_w2}),计算在漫射等效角度下的墙面和植被在地面的“阴影”占比(辐射遮挡) $S_w^\ast$和$S_v^\ast$,
利用公式(\ref{f_Sw})进行计算,则天空到达植被($F_{sv}$)、天空通过植被到墙面($F_{svw}$),以及天空通过植被到达地面的可视因子($F_{svg}$)分别计算为:
\begin{equation}\label{F_sv_svw_svg}
\begin{array}{c}F_{s v}=S_{v}^{*} \\ F_{s v w}=S_{v w}^{*} \\ F_{s v g}=F_{s v}-F_{s v w}\end{array}
\end{equation}
同理,地面到达植被($F_{gv}$)、地面通过植被到天空($F_{gvs}$),以及地面通过植被到达天空的可视因子($F_{gvs}$)可类似公式(\ref{F_sv_svw_svg})计算。

根据可视因子计算互易性原理,植被到天空、地面和墙面的可视因子计算为:
\begin{equation}
F_{vs}=F_{sv} \frac{f_g}{4 f_v}
\end{equation}
\begin{equation}
F_{vg}=F_{gv} \frac{f_g}{4 f_v}
\end{equation}
从而:
\begin{equation}
F_{v w}=1-F_{v s}-F_{v g}
\end{equation}
同理,根据互易性原理,$F_{wv}$计算为:
\begin{equation}
F_{w v}=F_{v w} \frac{f_{v}}{f_{b} \cdot {HL}}
\end{equation}
假设墙面通过植被到达墙面、天空和地面的可视因子与墙面到达相应各面的可视因子成正比,即:
\begin{equation}
F_{wvw}: F_{wvs}: F_{wvg}=F_{ww}: f_{1} F_{ws}: f_{2} F_{w g}
\end{equation}
其中$f_1=h_v/H$,$f_2=\left(H-h_v\right)/H$。根据能量守恒:
\begin{equation}
F_{wvw}+F_{wvs}+F_{wvg}=F_{w v}
\end{equation}
联合以上两式联立求解可得到:
\begin{equation}
F_{wvw}=\frac{F_{w v} F_{ww}}{F_{ww}+f_{1} F_{ws}+f_{2} F_{w g}}
\end{equation}
\begin{equation}
F_{wvs}=f_{1} \frac{F_{ws}}{F_{ww}} F_{wvw}
\end{equation}
\begin{equation}
F_{wvg}=f_{2} \frac{F_{w g}}{F_{ww}} F_{wvw}
\end{equation}
在此基础上,可以计算得到考虑植被影响的天空、墙面和地面之间的可视因子,
其中天空到墙面和地面的可视因子$F_{sw}^\prime$, $F_{sg}^\prime$计算为:
\begin{equation}
F_{s w}^{\prime}=F_{s w}-F_{s v w}+F_{s v w} T_{d}
\end{equation}
\begin{equation}
F_{s g}^{\prime}=F_{s g}-F_{s v g}+F_{s v g} T_{d}
\end{equation}
其中$T_d$为单棵球形树冠直射透射率,利用公式(\ref{T_ds_tau})计算。
地面到达墙面和天空的可视因子$F_{gw}^\prime$, $F_{gs}^\prime$计算为:
\begin{equation}
F_{g w}^{\prime}=F_{g w}-F_{g v w}+F_{g v w} T_{d}
\end{equation}
\begin{equation}
F_{gs}^{\prime}=F_{gs}-F_{g v s}+F_{g v s} T_{d}
\end{equation}
墙面到达地面、墙面和天空的可视因子$F_{wg}^\prime$, $F_{ww}^\prime$, $F_{ws}^\prime$计算为:
\begin{equation}
F_{w g}^{\prime}=F_{w g}-F_{wvg}+F_{wvg} T_{d}
\end{equation}
\begin{equation}
F_{ww}^{\prime}=F_{ww}-F_{wvw}+F_{wvw} T_{d}
\end{equation}
\begin{equation}
F_{ws}^{\prime}=F_{ww}-F_{wvs}+F_{wvs} T_{d}
\end{equation}

a. 直射入射太阳辐射\\
对于直射辐射太阳辐射,直射光到达阳面墙的辐射通量$E_{wsun}=S_w$,到达阴面墙的辐射通量$E_{wsha}=S_{vw}T_d$,
到达地面的辐射通量$E_g=1-S_w-S_v+\left(S_v-S_{vw}\right)T_d$,其中$E_{gimp}=E_gf_{gimp}$,$E_{gper}=E_gf_{gper}$,
$E_v=S_v$。阳面墙、阴面墙、不透水面、透水面和植被第一次散射辐射通量可分别计算为:$E_{wsun}\alpha_w$,$E_{wsha}\alpha_w$,
$E_{gimp}\alpha_{gimp}$,$E_{gper}\alpha_{gper}$,$E_v\alpha_v$,
其中$\alpha$表示反照率(对于植被表示所有向外散射辐射),不分区阳面和阴面,不区分漫射和直射照射,
且反射后的辐射假设为漫射辐射。假设经过墙面、地面和植被之间多次散射,达到辐射平衡时的墙面和地面辐射出射通量为$M$,
利用各组分之间的可视因子,可建立如下辐射平衡方程:
\begin{landscape}
\begin{equation}
\begin{array}{l}M_{wsun}=E_{wsun} \alpha_{w}+M_{wsun} F_{ww}^{\prime} f_{wsun} \alpha_{w}+M_{wsha} F_{ww}^{\prime} f_{wsun} \alpha_{w}+M_{gimp} F_{g w}^{\prime} f_{wsun} \alpha_{w}+M_{gper} F_{g w}^{\prime} f_{wsun} \alpha_{w}+M_{v} F_{v w} f_{wsun} \alpha_{w} \\ M_{wsha}=E_{wsha} \alpha_{w}+M_{wsha} F_{ww}^{\prime} f_{wsha} \alpha_{w}+M_{wsun} F_{ww}^{\prime} f_{wsha} \alpha_{w}+M_{gimp} F_{g w}^{\prime} f_{wsha} \alpha_{w}+M_{gper} F_{g w}^{\prime} f_{wsha} \alpha_{w}+M_{v} F_{v w} f_{wsha} \alpha_{w} \\ M_{gimp}=E_{gimp} \alpha_{gimp}+M_{wsun} F_{w g}^{\prime} f_{gimp} \alpha_{gimp}+M_{wsha} F_{w g}^{\prime} f_{gimp} \alpha_{gimp}+M_{v} F_{v g} f_{gimp} \alpha_{gimp} \\ M_{gper}=E_{gper} \alpha_{gper}+M_{wsun} F_{w g}^{\prime} f_{gper} \alpha_{gper}+M_{wsha} F_{w g}^{\prime} f_{gper} \alpha_{gper}+M_{v} F_{v g} f_{gper} \alpha_{gper} \\ M_{v}=E_{v} \alpha_{v}+M_{wsun} F_{w v} \alpha_{v}+M_{wsha} F_{w v} \alpha_{v}+M_{gimp} F_{g v} \alpha_{v}+M_{gper} F_{g v} \alpha_{v}\end{array}
\end{equation}
整理之后简化为:
\begin{equation}
\left(\begin{array}{ccccc}1-F_{ww}^{\prime} f_{wsun} \alpha_{w} & -F_{ww}^{\prime} f_{wsun} \alpha_{w} & -F_{g w}^{\prime} f_{wsun} \alpha_{w} & -F_{g w}^{\prime} f_{wsun} \alpha_{w} & -F_{v w} f_{wsun} \alpha_{w} \\ -F_{ww}^{\prime} f_{wsha} \alpha_{w} & 1-F_{ww}^{\prime} f_{wsha} \alpha_{w} & -F_{g w}^{\prime} f_{wsha} \alpha_{w} & -F_{g w}^{\prime} f_{wsha} \alpha_{w} & -F_{v w} f_{wsha} \alpha_{w} \\ -F_{w g}^{\prime} f_{gimp} \alpha_{gimp} & -F_{w g}^{\prime} f_{gimp} \alpha_{gimp} & 1 & 0 & -F_{v g} f_{gimp} \alpha_{gimp} \\ -F_{w g}^{\prime} f_{gper} \alpha_{gper} & -F_{w g}^{\prime} f_{gper} \alpha_{gper} & 0 & 1 & -F_{v g} f_{gper} \alpha_{gper} \\ -F_{w v} \alpha_{v} & -F_{w v} \alpha_{v} & -F_{g v} \alpha_{v} & -F_{g v} \alpha_{v} & 1\end{array}\right)\left(\begin{array}{c}M_{wsun} \\ M_{wsha} \\ M_{gimp} \\ M_{gper} \\ M_{v}\end{array}\right)=\left(\begin{array}{c}E_{wsun} \alpha_{w} \\ E_{wsha} \alpha_{w} \\ E_{gimp} \alpha_{gimp} \\ E_{gper} \alpha_{gper} \\ E_{v} \alpha_{v}\end{array}\right)
\end{equation}
\end{landscape}
方程求解同公式(\ref{mathbf_X}),阳面墙、阴面墙、不透水面和透水面的辐射总吸收及单位面积吸收同式(\ref{s_wsun_wsha_gimp_gper_1})和(\ref{s_wsun_wsha_gimp_gper_2})。植被的吸收计算类似,表达式为:
\begin{equation}
s_{v}=\frac{M_{v}}{a_{v}}\left(1-a_{v}-T_{d}\right)
\end{equation}
若考虑单位面积植被覆盖吸收辐射量,则$s_v$修订为:
\begin{equation}
s_{v}=\frac{f_{g}}{f_{v}} s_{v}
\end{equation}
不考虑屋顶时的城市反照率为:
\begin{equation}\label{alpha_u_a1}
\alpha_{u}=M_{wsun} F_{ws}^{\prime}+M_{wsha} F_{ws}^{\prime}+M_{gimp} F_{gs}^{\prime}+M_{gper} F_{gs}^{\prime}+M_{v} F_{v s}
\end{equation}
屋顶的吸收计算为$s_{roof}=1-\alpha_{roof}$,考虑屋顶的城市反照率(\ref{alpha_u_a1})修订为:
\begin{equation}
\alpha_{u}=\alpha_{roof} f_{b}+\alpha_{u} f_{g}
\end{equation}

b. 漫射入射太阳辐射\\
对于漫射,计算过程与直射入射辐射过程基本类似,不同之处在于首次到达墙面、地面和植被的辐射通量分别为:$F_{sw}^\prime f_{wsun}$、
$F_{sw}^\prime f_{wsha}$、$F_{sg}^\prime f_{gimp}$、$F_{sg}^\prime f_{gper}$和$F_{sv}$。
漫射入射辐射的传输矩阵同直射入射辐射情景。其求逆后的结果可以直接用于漫射入射辐射,各组分的辐射吸收和反照率计算过程在此不再赘述。 
\section{长波辐射传输}
\subsection{无植被覆盖时长波辐射传输}
无植被时的长波辐射传输类似与无植被时短波辐射传输时的漫射入射情景(长波辐射近似为漫射光源处理)。首次达到各组分表面的长波辐射通量为:
\begin{equation}
\begin{aligned} I_{wsun} &=L_{W} F_{s w} f_{wsun} \\ I_{wsha} &=L_{W} F_{s w} f_{wsha} \\ I_{gimp} &=L_{W} F_{s g} f_{gimp} \\ I_{gper} &=L_{W} F_{s g} f_{gper} \end{aligned}
\end{equation}
第一次反射的长波辐射通量依次分别为:
\begin{equation}
\begin{array}{c}I_{wsun}\left(1-\varepsilon_{w}\right) \\ I_{wsha}\left(1-\varepsilon_{w}\right) \\ I_{gimp}\left(1-\varepsilon_{gimp}\right) \\ I_{gper}\left(1-\varepsilon_{gper}\right)\end{array}
\end{equation}
上式中$\varepsilon$表示发射率,$\varepsilon_w$,$\varepsilon_{gimp}$,$\varepsilon_{gper}$为已知变量。各组分表面向外发射的长波辐射量为:
\begin{equation}
\begin{array}{c}4 f_{wsun} \mathrm{HL} * \frac{f_{b}}{f_{g}} * \sigma \varepsilon_{w} T_{wsun}^{4} \\ 4 f_{wsha} \mathrm{HL} * \frac{f_{b}}{f_{g}} * \sigma \varepsilon_{w} T_{wsha}^{4} \\ \sigma \varepsilon_{gimp} T_{gimp}^{4} f_{gimp} \\ \sigma \varepsilon_{gper} T_{gper}^{4} f_{gper}\end{array}
\end{equation}
$T$表示温度,$\sigma$为斯蒂芬-玻尔兹曼常数。类似与短波辐射传输,可以构建辐射平衡方程:
\begin{landscape}
\begin{equation}
    \begin{aligned}
        L_{wsun}^{\uparrow}=I_{wsun}\left(1-\varepsilon_{w}\right)+L_{wsun}^{\uparrow} F_{ww} f_{wsun}\left(1-\varepsilon_{w}\right)+L_{wsha}^{\uparrow} F_{ww} f_{wsun}\left(1-\varepsilon_{w}\right)+L_{gimp}^{\uparrow} F_{g w} f_{wsun}\left(1-\varepsilon_{w}\right)\\+L_{gper}^{\uparrow} F_{g w} f_{wsun}\left(1-\varepsilon_{w}\right)+4 f_{wsun} H L * \frac{f_{b}}{f_{g}} * \sigma \varepsilon_{w} T_{wsun}^{4}\\
        L_{wsha}^{\uparrow}=I_{wsha}\left(1-\varepsilon_{w}\right)+L_{wsha}^{\uparrow} F_{ww} f_{wsha}\left(1-\varepsilon_{w}\right)+L_{wsun}^{\uparrow} F_{ww} f_{wsha}\left(1-\varepsilon_{w}\right)+L_{gimp}^{\uparrow} F_{g w} f_{wsha}\left(1-\varepsilon_{w}\right)\\+L_{gper}^{\uparrow} F_{g w} f_{wsha}\left(1-\varepsilon_{w}\right)+4 f_{wsha} H L * \frac{f_{b}}{f_{g}} * \sigma \varepsilon_{w} T_{wsha}^{4}\\
        L_{gimp}^{\uparrow}=I_{gimp}\left(1-\varepsilon_{gimp}\right)+L_{wsun}^{\uparrow} F_{w g} f_{gimp}\left(1-\varepsilon_{gimp}\right)+L_{wsha}^{\uparrow} F_{w g} f_{gimp}\left(1-\varepsilon_{gimp}\right)+\sigma \varepsilon_{gimp} T_{gimp}^{4} f_{gimp}\\
        L_{gper}^{\uparrow}=I_{gper}\left(1-\varepsilon_{gper}\right)+L_{wsun}^{\uparrow} F_{w g} f_{gper}\left(1-\varepsilon_{gper}\right)+L_{wsha}^{\uparrow} F_{w g} f_{gper}\left(1-\varepsilon_{gper}\right)+\sigma \varepsilon_{gper} T_{gper}^{4} f_{gper}
    \end{aligned}
\end{equation}
其中$L^\uparrow$表示各组分表面总的向外长波辐射量(反射部分+自身发射部分)。经过整理,可得:
\begin{equation}
\begin{aligned}
\left(\begin{matrix}1-F_{ww}f_{wsun}\left(1-\varepsilon_w\right)&-F_{ww}f_{wsun}\left(1-\varepsilon_w\right)&-F_{gw}f_{wsun}\left(1-\varepsilon_w\right)&-F_{gw}f_{wsun}\left(1-\varepsilon_w\right)\\-F_{ww}f_{wsha}\left(1-\varepsilon_w\right)&1-F_{ww}f_{wsha}\left(1-\varepsilon_w\right)&-F_{gw}f_{wsha}\left(1-\varepsilon_w\right)&-F_{gw}f_{wsha}\left(1-\varepsilon_w\right)\\-F_{wg}f_{gimp}\left(1-\varepsilon_{gimp}\right)&-F_{wg}f_{gimp}\left(1-\varepsilon_{gimp}\right)&1&0\\-F_{wg}f_{gper}\left(1-\varepsilon_{gper}\right)&-F_{wg}f_{gper}\left(1-\varepsilon_{gper}\right)&0&1\\\end{matrix}\right)
\left(\begin{matrix}L_{wsun}^\uparrow\\L_{wsha}^\uparrow\\L_{gimp}^\uparrow\\L_{gper}^\uparrow\\\end{matrix}\right)\\
=\left(\begin{matrix}I_{wsun}\left(1-\varepsilon_w\right)+2f_{wsun}\frac{\sigma\varepsilon_wT_{wsun}^4H}{W}\\I_{wsha}\left(1-\varepsilon_w\right)+2f_{wsha}\frac{\sigma\varepsilon_wT_{wsha}^4H}{W}\\I_{gimp}\left(1-\varepsilon_{gimp}\right)+\sigma\varepsilon_{gimp}T_{gimp}^4f_{gimp}\\I_{gper}\left(1-\varepsilon_{gper}\right)+\sigma\varepsilon_{gper}T_{gper}^4f_{gper}\\\end{matrix}\right)
\end{aligned}
\end{equation}
\end{landscape}

以上方程也可以记为矩阵形式:
\begin{equation}
\mathbf{A X}=\mathbf{B}
\end{equation}
求解以上方程得到$\mathbf{X}$,各组分表面长波净辐射量为:
\begin{equation}\label{L_wsun_wsha_gimp_pger_1}
\begin{array}{c}L_{wsun}=\frac{\varepsilon L_{wsun}^{\uparrow}-4 f_{wsun} \mathrm{HL} * \frac{f_{b}}{f_{g}} * \sigma \varepsilon_{w} T_{wsun}^{4}}{1-\varepsilon_{w}} \\ L_{wsha}=\frac{\varepsilon L_{wsha}^{\uparrow}-4 f_{wsha} \mathrm{HL} * \frac{f_{b}}{f_{g}} * \sigma \varepsilon_{w} T_{wsha}^{4}}{1-\varepsilon_{w}} \\ L_{gimp}=\frac{\varepsilon L_{w gimp}^{\uparrow}-\sigma \varepsilon_{gimp} T_{gimp}^{4} f_{gimp}}{1-\varepsilon_{gimp}} \\ L_{p g e r}=\frac{\varepsilon L_{w gper}^{\uparrow}-\sigma \varepsilon_{gper} T_{gper}^{4} f_{gper}}{1-\varepsilon_{gper}}\end{array}
\end{equation}
转化为单位面积净辐射时上式修改为:
\begin{equation}\label{L_wsun_wsha_gimp_pger_2}
\begin{array}{c}L_{wsun}=\frac{f_{g}}{4 f_{wsun} \mathrm{HL} f_{b}} L_{wsun} \\ L_{wsha}=\frac{f_{g}}{4 f_{wsha} \mathrm{HL} f_{b}} L_{wsha} \\ L_{gimp}=\frac{1}{f_{gimp}} L_{gimp} \\ L_{gper}=\frac{1}{f_{gper}} L_{gper}\end{array}
\end{equation}
在不考虑屋顶时的向外长波辐射通量计算为:
\begin{equation}
L^{\uparrow}=L_{wsun}^{\uparrow} F_{ws}+L_{wsha}^{\uparrow} F_{ws}+L_{gimp}^{\uparrow} F_{gs}+L_{gper}^{\uparrow} F_{gs}
\end{equation}

\subsection{有植被覆盖时长波辐射传输}\label{有植被覆盖时长波辐射传输}
考虑植被覆盖时的长波辐射计算类似于有植被覆盖时的短波辐射传输,根据之前介绍的无植被时长波辐射传输平衡方程,可以写出有植被覆盖时: 
\begin{landscape}
\begin{equation}
    \begin{aligned}
    L_{{wsun }}^{\uparrow}=I_{{wsun }}\left(1-\varepsilon_{w}\right)+L_{wsun}^{\uparrow} F_{ww}^{\prime} f_{wsun}\left(1-\varepsilon_{w}\right)+L_{wsha}^{\uparrow} F_{ww}^{\prime} f_{wsun}\left(1-\varepsilon_{w}\right)+L_{gimp}^{\uparrow} F_{g w}^{\prime} f_{wsun}\left(1-\varepsilon_{w}\right)+L_{gper}^{\uparrow} F_{g w}^{\prime} f_{wsun}\left(1-\varepsilon_{w}\right)\\+L_{v}^{\uparrow} F_{v w} f_{wsun}\left(1-\varepsilon_{w}\right)+4 f_{wsun} H L * \frac{f_{b}}{f_{g}} * \sigma \varepsilon_{w} T_{wsun}^{4} \\
    L_{wsha}^{\uparrow}=I_{wsha}\left(1-\varepsilon_{w}\right)+L_{wsha}^{\uparrow} F_{ww}^{\prime} f_{wsha}\left(1-\varepsilon_{w}\right)+L_{wsun}^{\uparrow} F_{ww}^{\prime} f_{wsha}\left(1-\varepsilon_{w}\right)+L_{gimp}^{\uparrow} F_{g w}^{\prime} f_{wsha}\left(1-\varepsilon_{w}\right)+L_{gper}^{\uparrow} F_{g w}^{\prime} f_{wsha}\left(1-\varepsilon_{w}\right)\\+L_{v}^{\uparrow} F_{v w} f_{wsha}\left(1-\varepsilon_{w}\right)+4 f_{wsha} H L * \frac{f_{b}}{f_{g}} * \sigma \varepsilon_{w} T_{wsha}^{4}\\
    L_{gimp}^{\uparrow}=I_{gimp}\left(1-\varepsilon_{gimp}\right)+L_{wsun}^{\uparrow} F_{w g}^{\prime} f_{gimp}\left(1-\varepsilon_{gimp}\right)+L_{wsha}^{\uparrow} F_{w g}^{\prime} f_{gimp}\left(1-\varepsilon_{gimp}\right)+L_{v}^{\uparrow} F_{v g} f_{gimp}\left(1-\varepsilon_{gimp}\right)+\sigma \varepsilon_{gimp} T_{gimp}^{4} f_{gimp} \\
    L_{gper}^{\uparrow}=I_{gper}\left(1-\varepsilon_{gper}\right)+L_{wsun}^{\uparrow} F_{w g}^{\prime} f_{gper}\left(1-\varepsilon_{gper}\right)+L_{wsha}^{\uparrow} F_{w g}^{\prime} f_{gper}\left(1-\varepsilon_{gper}\right)+L_{v}^{\uparrow} F_{v g} f_{gper}\left(1-\varepsilon_{gper}\right)+\sigma \varepsilon_{{gpre }} T_{{gpre }}^{4} f_{{gpre }} \\
    L_{v}^{\uparrow}=4 f_{v} / f_{g} \sigma \varepsilon_{v} T_{v}^{4}
    \end{aligned}
\end{equation}
\end{landscape}

\begin{landscape}
上式中的$F$,$F^\prime$表示可视因子,同章节 \ref{有植被覆盖时短波辐射传输}。经过整理,可以得到:
\begin{equation}
    \begin{split}
    \left(\begin{matrix}1-\ F_{ww}^\prime f_{wsun}\left(1-\varepsilon_w\right)&-\ F_{ww}^\prime f_{wsun}\left(1-\varepsilon_w\right)&-F_{gw}^\prime f_{wsun}\left(1-\varepsilon_w\right)&-F_{gw}^\prime f_{wsun}\left(1-\varepsilon_w\right)&-F_{vw}f_{wsun}\left(1-\varepsilon_w\right)\\
    -\ F_{ww}^\prime f_{wsha}\left(1-\varepsilon_w\right)&1-\ F_{ww}^\prime f_{wsha}\left(1-\varepsilon_w\right)&-F_{gw}^\prime f_{wsha}\left(1-\varepsilon_w\right)&-F_{gw}^\prime f_{wsha}\left(1-\varepsilon_w\right)&-F_{vw}f_{wsha}\left(1-\varepsilon_w\right)\\
    -F_{wg}^\prime f_{gimp}\left(1-\varepsilon_{gimp}\right)&-F_{wg}^\prime f_{gimp}\left(1-\varepsilon_{gimp}\right)&1&0&-F_{vg}f_{gimp}\left(1-\varepsilon_{gimp}\right)\\
    -F_{wg}^\prime f_{gper}\left(1-\varepsilon_{gper}\right)&-F_{wg}^\prime f_{gper}\left(1-\varepsilon_{gper}\right)&0&1&-F_{vg}f_{gper}\left(1-\varepsilon_{gper}\right)\\
    0&0&0&0&1\\\end{matrix}\right)\cdot \\
    \left(\begin{matrix}L_{wsun}^\uparrow\\L_{wsha}^\uparrow\\L_{gimp}^\uparrow\\L_{gper}^\uparrow\\L_v^\uparrow\\\end{matrix}\right)
    =\left(\begin{matrix}I_{wsun}\left(1-\varepsilon_w\right)+4f_{wsun}HL\ast\frac{f_b}{f_g}\ast\sigma\varepsilon_wT_{wsun}^4\\
        I_{wsha}\left(1-\varepsilon_w\right)+4f_{wsha}HL\ast\frac{f_b}{f_g}\ast\sigma\varepsilon_wT_{wsha}^4\\
        I_{gimp}\left(1-\varepsilon_{gimp}\right)+\sigma\varepsilon_{gimp}T_{gimp}^4f_{gimp}\\
        I_{gper}\left(1-\varepsilon_{gper}\right)+\sigma\varepsilon_{gper}T_{gper}^4f_{gper}\\
        4f_v/f_g\sigma\varepsilon_vT_v^4\\\end{matrix}\right)
    \end{split}
\end{equation}
\end{landscape}

求解以上方程,墙面和地面长波净辐射及单位面积净辐射计算公式同(\ref{L_wsun_wsha_gimp_pger_1})和(\ref{L_wsun_wsha_gimp_pger_2})。植被净辐射吸收为:
\begin{equation}
L_{v}=\left(L_{wsun}^{\uparrow} F_{w v}+L_{wsha}^{\uparrow} F_{w v}+L_{gimp}^{\uparrow} F_{g v}+L_{gper}^{\uparrow} F_{g v}+L_{W} F_{s v}\right) \varepsilon_{v}-4 f_{v} / f_{g} \sigma \varepsilon_{v} T_{v}^{4}
\end{equation}
植被单位面积覆盖长波净辐射修改为:
\begin{equation}
L_{v}=\frac{f_{g}}{f_{v}} L_{v}
\end{equation}
不考虑屋顶时城市总的向外长波辐射计算为:
\begin{equation}
L^{\uparrow}=L_{wsun}^{\uparrow} F_{ws}^{\prime}+L_{wsha}^{\uparrow} F_{ws}^{\prime}+L_{gimp}^{\uparrow} F_{gs}^{\prime}+L_{gper}^{\uparrow} F_{gs}^{\prime}+L_{v}^{\uparrow} F_{v s}
\end{equation}
\section{湍流交换过程}
城市湍流交换过程与三维植被湍流交换过程类似,计算屋顶、墙面(阴、阳面)、地面和植被的感热、潜热交换量,
同样是基于相似性理论,建立每层(等效高度)通量守恒方程,联立求解。不同之处在于其粗糙度、迎风面积指数、
零平面位移高度、风速/湍流交换系数衰减系数、建筑表面边界层阻抗计算与植被有所不同。
下面就无植被覆盖和有植被覆盖时对不同于自然植被湍流交换过程的部分进行简要说明。
\subsection{无植被覆盖时湍流交换过程}
建筑物的零平面位移高度采用\citet{macdonald1998improved}方案:
\begin{equation}
\frac{d}{H}=1+A^{-f_{b}}\left(f_{b}-1\right)
\end{equation}
其中$A$取值为4.43。迎风面积指数可以计算为:
\begin{equation}
\lambda_{f}=\frac{2}{\pi} \cdot H L \cdot f_{b} \int_{0}^{\frac{\pi}{2}}(\cos \theta+\sin \theta) d \theta=\frac{4}{\pi} \cdot H L f_{b}
\end{equation}
粗糙度计算为:
\begin{equation}
\frac{z_{0}}{H}=\left(1-\frac{d}{H}\right) \exp \left[-\left(0.5 \frac{c_{D}}{\kappa^{2}}\left(1-\frac{d}{H}\right) \lambda_{f}\right)^{-0.5}\right]
\end{equation}
其中$C_D=1.2$,$\kappa=0.4$为卡曼常数。

城市冠层内的风速和湍流交换系数同样假设为指数衰减,但衰减系数与植被不同,
采用\citet{masson2000physically}方案,计算为0.5$HW$。建筑物(屋顶、墙面)边界层阻抗$r_b$计算为\citep{oleson2008urban}:
\begin{equation}
r_{b}=\frac{\rho_{a} c_{p}}{11.8+4.2 u_{e f f}}
\end{equation}
其中$u_{eff}$表示屋顶高度/墙面交换高度的等效风速值,$\rho_a$表示空气密度,$C_p$表示干空气定压比热容。

以上为主要区别于三维植被湍流交换过程计算的部分,整个正式湍流交换计算阻抗网络如图~\ref{fig:无植被覆盖时城市湍流交换阻抗示意图} 所示。
湍流交换计算过程与三维植被湍流交换方案类似,先通过相似性理论计算屋顶高度风速和湍流交换系数;
然后根据指数衰减假设,参考三维植被湍流交换方案~\citep{dai2019different},
根据公式(\ref{uz})和(\ref{Kz})计算风速和湍流交换系数在城市冠层内部剖面。屋顶的等效风速即为H高度时的风速,
墙面的等效风速为H到地面的平均风速,通过风速剖面积分计算得到。同理,屋顶与墙面等效交换高度之间的动力学交换阻抗,
以及墙面等效交换高度与地面之间的动力学阻抗$r_d$均是通过对交换系数剖面积分计算得到(公式(\ref{r_d1}))。
同多层植被湍流交换计算一样,对上图所示的屋顶和墙面等效交换高度建立感热和潜热通量守恒方程,
迭代求解其高度处的空气温度、湿度以及相似性理论交换长度(即空气不稳定程度)。
因为这部分过程与三维植被湍流交换过程完全一致,在此不再阐述。
{
\begin{figure}[]
\centering
\includegraphics{Figures/城市模式/无植被覆盖时城市湍流交换阻抗示意图.png}
\caption{无植被覆盖时城市湍流交换阻抗示意图。}
\label{fig:无植被覆盖时城市湍流交换阻抗示意图}
\end{figure}
}


\subsection{有植被覆盖式湍流交换过程}
有植被覆盖时的湍流交换网络如图~\ref{fig:有植被覆盖时城市湍流交换阻抗示意图} 所示,
其计算过程与无植被覆盖时类似,不同之处在于新增植被组分。因此,在建立通量守恒方程式时,
需要考虑植被的感热、潜热交换,涉及到植被光合作用(蒸腾)和长波辐射吸收计算。
光合作用过程同自然植被光合作用过程,长波辐射吸收计算在章节 \ref{有植被覆盖时长波辐射传输} 已给出。
在实际计算过程中,可根据植被与建筑物高度差异,视情况分为两层或三层等效交换(三层时,即植被单独考虑为一层)。
整个过程也是迭代求解,但此时需要新增迭代求解变量叶片温度,其他过程同无植被覆盖时湍流交换过程。
{
\begin{figure}[]
\centering
\includegraphics{Figures/城市模式/有植被覆盖时城市湍流交换阻抗示意图.png}
\caption{有植被覆盖时城市湍流交换阻抗示意图。}
\label{fig:有植被覆盖时城市湍流交换阻抗示意图}
\end{figure}
}


\section{城市温度传导计算}
城市内透水面、不透水面、墙面和屋顶的温度传导计算以自然界土壤温度传导计算为基础,基本过程一致,下面重点介绍城市的不同之处。

\subsection{透水面温度计算}
城市透水面即为土壤,与自然界土壤的热传递计算一致 (分层方案与土壤一致,
土壤热力参数也从全球数据中读取),只是上表面接收的短波、长波辐射及湍流交换通量(感热、潜热)由城市相应模块求解得到。

\subsection{不透水面温度计算}
不透水面的温度传导计算与透水面最大的不同之处在于需要将不透水面层的热力属性(导热率和热容)替换掉所在层的土壤热力属性。
另外在有雪、冰和水存在时,需要对第一层不透水面的热容进行订正。不透水面不考虑水分在内部的传递,相变过程只考虑第一层及以上积雪覆盖。

\subsection{墙面温度计算}
墙面(包括阴面墙、阳面墙)厚度从外部数据读取,同土壤一样,也分为10层,每层厚度一样,其热力参数从外部数据读取。
不同于透水面/不透水面,墙面不考虑积水、积雪覆盖,因此其热力属性完全由自身材料决定,同时也不考虑水传递、相变过程和潜热交换。

另一个不同之处在于最里(下)层的热量交换设置,对于土壤、不透水面,考虑最下一层无热量交换,但对于墙壁,
考虑室内墙壁表面温度与最里层墙壁的热量交换。除此之外,其他方面及其求解过程与土壤温度类似。


\subsection{屋顶温度计算}
屋顶的分层方案与墙壁一致。温度传递类似于墙面,但考虑屋顶积雪、积水覆盖时对屋顶第一层热力属性的影响,同不透水面。
同时最里层屋顶与室内屋顶表面温度考虑热量交换,相变过程只考虑第一层屋顶。


\section{城市水文过程}
对城市水文过程考虑主要分为三类:1) 透水面;2) 屋顶和不透水面;3) 城市水体。

对于透水面的处理同土壤水过程,计算产流和土壤水传输。对于城市水体,采用湖泊方案进行模拟。对于屋顶和不透水面,
考虑积雪和积水过程,类似与土壤积雪过程,但第一层为不透水面,液态水的最大承载量为不超过设定的值(max ponding),
超过的部分均当成地表产流处理。

由于目前城市水文过程比较简单,相应过程在土壤和湖泊模拟部分均有描述,在此不再赘述。


\section{城市人为热模拟}

\subsection{建筑能耗模型}
CoLM建筑物能耗模型如图~\ref{fig:建筑能耗模型示意图} 所示,同样是基于三维城市建筑群落结构假设,其过程类似于CLM5.0建筑能耗模型。
模型考虑墙体、屋顶的热传导过程,同时考虑室内空气与内墙壁、屋顶和室外的热量交换。根据室内设定的最高、
最低参考温度,来计算空调对应能耗,以感热的形式增加到湍流交换计算的源项中。
首先计算热交换通量并建立能量平衡方程,进而求解室内温度$T_{room}$,室内墙壁表面空气温度$T_{wsun,in}$,$T_{wsua,in}$和室内屋顶表面空气温度$T_{roof,in}$。
通量交换包括最里层墙壁/屋顶与屋内墙壁/屋顶表面空气热量交换,屋内墙壁/屋顶表面空气与室内空气热量交换以及室内空气与室外空气热量交换,通量平衡方程为: 
{
\begin{figure}[]
\centering
\includegraphics{Figures/城市模式/建筑能耗模型示意图.png}
\caption{建筑能耗模型示意图。}
\label{fig:建筑能耗模型示意图}
\end{figure}
}

\begin{landscape}
\begin{equation}
    \begin{array}{l}
        \begin{split}
        0.5 h_{{roof }, c v}\left(t_{{roof }, { in }}-t_{{room }}\right)+0.5 h_{{roof }, c v}\left(t_{{roof,in }}^{\prime}-t_{{room }}^{\prime}\right)=0.5 h_{{roof }, t k}\left(T_{{roof }, n}-T_{{roof }, { in }}\right)+0.5 h_{{roof }, t k}\left(T_{{roof }, n}-T_{{roof,in }}^{\prime}\right) \\
        0.5 h_{{wsun, }, v}\left(t_{{wsun,in }}-t_{{room }}\right)+0.5 h_{{wsun, cv }}\left(t_{wsun, i n}^{\prime}-t_{{room }}^{\prime}\right)=0.5 h_{{wsun,tk }}\left(T_{{wsun, } n}-T_{{wsun,in }}\right)+0.5 h_{{roof,tk }}\left(T_{{wsun, } n}-T_{{wsun,in }}^{\prime}\right)\\
        0.5 h_{wsha, c v}\left(t_{wsha, i n}-t_{{room }}\right)+0.5 h_{wsha, c v}\left(t_{wsha, i n}^{\prime}-t_{{room }}^{\prime}\right)=0.5 h_{wsha, t k}\left(T_{wsha, n}-T_{{roof }, i n}\right)+0.5 h_{wsha, t k}\left(T_{wsha, n}-T_{wsha, i n}^{\prime}\right)\\
        H \rho_{a} C_{p} \frac{T_{{room }}^{\prime}-T_{{room }}}{\Delta t}
        =\frac{ACH}{3600} H \rho_{a} C_{p}\left(T_{a f}-T_{{room }}^{\prime}\right)+0.5 h_{{roof }, c v}\left(t_{{roof }, { in }}-t_{{room }}\right)+0.5 h_{{roof }, c v}\left(t_{{roof }, { in }}^{\prime}-t_{{room }}^{\prime}\right)\\
        +0.5 h_{wsun, c v}\left(t_{wsun, i n}-t_{{room }}\right) f_{{wsun }}+0.5 h_{wsun, c v}\left(t_{wsun, i n}^{\prime}-t_{{room }}^{\prime}\right) f_{wsun} \\
        +0.5 h_{wsha, c v}\left(t_{wsha, i n}-t_{{room }}\right) f_{wsha}+0.5 h_{wsha, c v}\left(t_{wsha, i n}^{\prime}-t_{r o o m}^{\prime}\right) f_{wsha}
        \end{split}
    \end{array}
\end{equation}
以上方程组可写成矩阵形式:
\begin{equation}
    \begin{split}
\left(\begin{array}{cccc}0.5 h_{r o o f, c v}+0.5 h_{r o o f, t k} & 0 & 0 & -0.5 h_{r o o f, c v} \\ 0 & 0.5 h_{wsun, c v}+0.5 h_{wsun, t k} & 0 & -0.5 h_{wsun, c v} \\ 0 & 0 & 0.5 h_{wsha, c v}+0.5 h_{wsha, t k} & -0.5 h_{wsha, c v} \\ -0.5 h_{r o o f, c v} & -0.5 h_{wsun, c v} f_{wsun} & -0.5 h_{wsha, c v} f_{wsha} & \frac{H \rho_{a} C_{p}}{\Delta t}+\frac{{ ACH }}{3600} H \rho_{a} C_{p}\end{array}\right)
    \left(\begin{array}{c}T_{{roof,in }}^{\prime} \\ T_{wsun, i n}^{\prime} \\ T_{wsha, i n}^{\prime} \\ T_{{room }}^{\prime}\end{array}\right)=
    \\
    \left(\begin{array}{c}-0.5h_{roof,cv}\left(t_{roof,in}-t_{room}\right)+0.5h_{roof,tk}\left(T_{roof,n}-T_{roof,in}\right)+0.5h_{roof,tk}T_{roof,n}\\
        -0.5h_{wsun,cv}\left(t_{wsun,in}-t_{room}\right)+0.5h_{wsun,tk}\left(T_{wsun,n}-T_{wsun,in}\right)+0.5h_{wsun,tk}T_{wsun,n}\\
        -0.5h_{wsha,cv}\left(t_{wsha,in}-t_{room}\right)+0.5h_{wsun,tk}\left(T_{wsun,n}-T_{wsun,in}\right)+0.5h_{wsha,tk}T_{wsun,n}\\
        \frac{H \rho_{a} C_{p} T_{{room }}}{\Delta T}+\frac{ACH}{3600} H \rho_{a} C_{p} T_{a f}+0.5 h_{{roof,cv }}\left(t_{{roof,in }}-t_{{room }}\right)+0.5 h_{{wsun, cv }}\left(t_{{wsun,in }}-t_{{room }}\right) f_{{wsun }}+0.5 h_{{wsha,cv }}\left(t_{wsha, i n}-t_{{room }}\right) f_{{wsha }}  
    \end{array}\right)
    \end{split}
\end{equation}
\end{landscape}

上式中$T_{roof,in}^\prime$、$T_{wsun,in}^\prime$、$T_{wsha,in}^\prime$、$T_{room}^\prime$为需求解变量,其他变量均为已知量。
$T_{roof,in}$、$T_{wsun,in}$、$T_{wsha,in}$、$T_{room}$分别为上一时刻相应温度,
$T_{af}$为建筑室外空气温度。$h_{roof,cv}$、$h_{wsun,cv}$、$h_{wsha,cv}$
分别为室内屋顶表面空气、阳面、阴面墙表面空气与室内空气温度热交换系数,参考CLM5.0 \citep{oleson2020parameterization},$h_{roof,cv}=4.04$,
$h_{wsun,cv}=h_{wsha,cv}=3.076$。$ACH$为室内外空气热交换系数,设置为0.3。$T_{roof}$、$T_{wsun,n}$、$T_{wsha,n}$为最里层(用$n$表示)
屋顶、阴面墙、阳面墙的温度,为$h_{roof,tk}$、$h_{wsun,tk}$、$h_{wsha,tk}$为其与各自室内表面空气温度热交换系数,通过墙壁导热率和厚度计算得到。

以上方程组可通过矩阵求逆的方式进行求解,计算得到$T_{roof,in}^\prime$、$T_{wsun,in}^\prime$、
$T_{wsha,in}^\prime$、$T_{room}^\prime$。室内外热交换通量$F_{ach}$计算为:
\begin{equation}
F_{a c h}=\frac{ACH}{3600} H \rho_{a} C_{p}\left(T_{{room }}^{\prime}-T_{a f}\right)
\end{equation}
当$T_{room}^\prime>T_{room,max} $($T_{room,max}$为预设屋内最高温度,从外部数据读取)时,打开空调制冷,
使屋内温度维持在$T_{room,max}$,由此产生的空调热排放量$F_{hac}$为:
\begin{equation}
F_{{hac }}=H \rho_{a} C_{p} \frac{T_{{room }}^{\prime}-T_{{room,max }}}{\Delta t}
\end{equation}
当$T_{room}^\prime<T_{room,min}$ ($T_{room,min}$为预设屋内最高温度,从外部数据读取)时,
打开制暖设备,使屋内温度维持在$T_{room,min}$,由此产生的空调热排放量$F_{hac}$为:
\begin{equation}
F_{h a c}=H \rho_{a} C_{p} \frac{T_{{room }}^{\prime}-T_{{room,min }}}{\Delta t}
\end{equation}
在以上两种情况下,$T_{room}^\prime$更新为$T_{room,max}/T_{room,min}$。
由于制冷/制暖所浪费的热量排放分别计算为$0.6F_{hac}$和|$0.2F_{hac}$| (| |表示取绝对值)。
总的建筑热排放量为制冷/制暖热排放量加上其过程中浪费的热排放量。以上计算的热排放量均需乘以$f_b$来转化为单位城市面积通量,
并作为感热项 (源项) 加入到城市湍流交换平衡方程中,参考图~\ref{fig:建筑能耗模型示意图}。


\subsection{人体代谢热}
人体代谢热($Q_M$, \unit{W.m^{-2}})根据 \citet{allen2011} 的方法,按照能源清单法及 Top-Down 方法,利用人口密度数据,
将不同时刻的人体热排放映射到每个城市格点上,具体计算如下:
\begin{equation}
Q_{M}=P \cdot H_{M} \cdot 10^{-6}
\end{equation}
其中$P$为格点内人口密度(\unit{pop.km^{-2}}),目前使用的是 Gridded Population of the World (GPWv3) 数据,
分辨率为\ang{;2.5} (表~\ref{tab:人口密度数据}),$H_M$为不同时刻人体排放的热量。人体排放的热量遵循一个基本假设:
人在活动时每小时排放的热量为175 W (工作),非活动时排放量为75 W (休息),以及一个中间值125 W~\citep{sailor2004top};
除此之外,也没有考虑人口在不同格点间的移动。
% Please add the following required packages to your document preamble:
% \usepackage{booktabs}
\begin{table}[]
    \centering
    \caption{人口密度数据}
    \label{tab:人口密度数据}
    \begin{tabular}{@{}lll@{}}
    \toprule
    数据名称                            & 分辨率         & 来源                                                                                                     \\ \midrule
    Gridded Population of the World & 2.5 arc-min & \begin{tabular}[c]{@{}l@{}}https://sedac.ciesin.columbia.edu/\\   data/collection/gpw-v3\end{tabular} \\ \bottomrule
    \end{tabular}
\end{table}


\subsection{交通热}
交通热排放($Q_v$, \unit{W.m^{-2}})同样根据~\citet{allen2011} 的方法利用能源清单法进行计算,
并采用Top-Down的方法,利用人口密度以及各个国家的汽车拥有量计算出格点内的汽车数量,
然后根据汽车行驶排放的热量以及交通流量的日分布计算出不同时刻的交通热排放。
其数据来源如表~\ref{tab:交通热数据及来源},具体计算方法如下:
\begin{equation}
Q_{v}=\frac{\left(V_{c} E_{c}+V_{M} E_{M}+V_{F R} E_{F R}\right) \cdot F \cdot 24 \cdot P \cdot D \cdot H_{t r a c}}{3.6 \cdot 10^{12}}
\end{equation}
% Please add the following required packages to your document preamble:
% \usepackage{booktabs}
\begin{table}[]
    \centering
\caption{交通热数据及来源}
\label{tab:交通热数据及来源}
    \begin{tabular}{@{}lll@{}}
    \toprule
    数据名称                       & 来源                          & Spatial/administrative unit   \\ \midrule
    Vehicles density and types & World mapper                & All countries and territories \\
    Daily vehicle pattern      & \citet{Hallenbeck1997} & All countries and territories \\
    Vehicle heat emissions     & \citet{smith2009estimating}       & All countries and territories \\ \bottomrule
    \end{tabular}
    \end{table}
其中$V_c$、$V_M$、$V_{FR}$分别为每千人拥有的汽车、摩托车、货车(巴士)的数量,该数据来自于World mapper
(\url{http://www.worldmapper.org/textindex/texttransport.htm});
$E_c$、$E_M$、$E_{FR}$则为三种机动车的排放系数,与机动车行驶速度有关,目前使用的 \citet{smith2009estimating} 的系数设置。
$F$为改变汽车数量的系数,相当于汽车总量的一部分在行驶,目前为固定系数(0.8);
$P$为格点内人口密度 (\unit{pop.km^{-2}});$D$为行驶的距离,同样与速度有关;$H_{trac}$为时间步长内交通流量分布。
\chapter{作物模式}
%\addcontentsline{toc}{chapter}{作物模式}

%\begin{作物模式}
GPAM1可以模拟作物生长发育的关键过程及其对天气/气候、近地面大气$\rm CO_2$和$\rm O_3$浓度和氮沉降、
农田管理的响应和生物、化学、物理反馈(图~\ref{fig:作物模式GPAM1框图})。
图~\ref{fig:作物模式GPAM1框图}中蓝色高亮部分为作物区别与自然植被、需要特殊处理的过程。
{
\begin{figure}[]
\centering
\includegraphics{Figures/作物模式/作物模式GPAM1框图.png}
\caption{作物模式GPAM1框图。  }
\label{fig:作物模式GPAM1框图}
\end{figure}
}
\section{作物功能类型(CFT)}
GPAM1包含10个作物功能类型(Crop Functional Type, CFT),模拟4种作物类型: 小麦、水稻、玉米、大豆(图\ref{fig:作物功能类型覆盖率的空间分布})。
每种作物类型各分雨养(rainfed)和灌溉(irrigated) CFT,小麦又分冬小麦和春小麦。每个CFT占用一个水热独立的陆表单元(patch)。
{
\begin{figure}[]
\centering
\includegraphics{Figures/作物模式/作物功能类型覆盖率的空间分布.png}
\caption{GPAM1模式模拟的10个作物功能类型(CFT)覆盖率的空间分布。}
\label{fig:作物功能类型覆盖率的空间分布}
\end{figure}
}
\section{物候}
GPAM1的物候包括三个阶段: (1)播种->抽芽; (2)抽芽到开始灌浆(grain fill); (3)灌浆到成熟/收割。所涉及的参数取值见表 \ref{tab:作物物候方案相关参数}。\\
\begin{enumerate}
  \item 播种\\
  播种日期是给定的,采用全球播种日再分析数据(Jägermeyr et al., in prep.),并结合陆表数据中的作物分布制作而成,为GPAM1的输入场。
  \item 抽芽\\
  当热单元指数(Heat Unit Index, $HUI$):
  \begin{equation}\label{HUI}
  HUI=\frac{GDD_{base}}{GDD_{mat}}
  \end{equation}
  达到$f_{LE}$时,开始抽芽。$GDD_{base}$是基于陆表气温的积温($^{\circ}$C days):
  \begin{equation}
  GDD_{ {base }}=\sum_{planting}^{current\ day} \min \left(0, T_{sa}-T_{base}\right)
  \end{equation}
  公式中(\ref{HUI}),$T_{sa}$是日地表气温,$T_{base}$是底温(作物在该温度下停止生长)。\\
  \item 灌浆\\
  当$HUI$达到$f_{GF}$时,开始灌浆。
  \item 成熟\\
  当$HUI$达到1时或生长季达到最长生长季长度$GSL_{max}$时,作物成熟。其中,$GDD_{mat}$是$GDD_{base}$ 
  10年滑动平均的一元线性方程(表~\ref{tab:作物物候方案相关参数})。
  \begin{equation}
    GDD_{mat}=a GDD_{10yr}+b
  \end{equation}
  \item 收割\\
  假设达到成熟的时步收割。
\end{enumerate}
% Please add the following required packages to your document preamble:
% \usepackage{booktabs}
\begin{table}[]
  \centering
  \caption{作物物候方案相关参数}
  \label{tab:作物物候方案相关参数}
\begin{tabular}{@{}cccccc@{}}
\toprule
    & $T_{base}$ & $f_{LE}$  & $f_{GF}$  & $GDD_{mat}$          & $GSL_{max}$ \\ \midrule
春小麦 & 0     & 0.07 & 0.60 & 0.2$GDD_{10yr}$+1458 & 150    \\
冬小麦 & 0     & 0.03 & 0.67 & 0.38$GDD_{10yr}$+526 & 270    \\
水稻1 & 10    & 0.12 & 0.68 & 0.30$GDD_{10yr}$+695 & 150    \\
水稻2 & 10    & 0.35 & 0.75 & 0.27$GDD_{10yr}$+575 & 150    \\
玉米  & 8     & 0.11 & 0.64 & 0.26$GDD_{10yr}$r+907 & 150    \\
大豆  & 10    & 0.15 & 0.69 & 0.26$GDD_{10yr}$+802 & 150    \\ \bottomrule
\end{tabular}
\end{table}

\section{分配}
分配系数随物候阶段变化而变化。从出叶开始,按下列分配系数公式分配到叶库Cleaf,茎库Cleaf,根库Croot,粒库Cgrain。
不同物候期分配系数见表~\ref{tab:作物分配方案相关参数}。

(1)	物候期2 \\
\begin{equation}
\left\{\begin{array}{c}
  a_{grain}=0 \\ 
  a_{root}=a_{root\_i}-\left(a_{root\_i}-a_{root\_f}\right) HUI \\
  a_{leaf}=\left(1-a_{root}\right) a_{leaf\_i} \frac{{e}^{-{b}}-{e}^{-b \frac{HUI}{f_{GF}}}}{{e}^{-{b}}-1}   \\
  a_{stem}=1-a_{grain}-a_{root}-a_{leaf}
  \end{array}\right.
\end{equation}
其中,下标$i$和$f$表示该分配系数的初值和终值,$b=0.1$。

(2)	物候期3 \\
\begin{equation}
  \left\{\begin{array}{c}
    a_{leaf}=0.0 \\ 
    a_{root}=a_{root\_i}-\left(a_{root\_i}-a_{root\_f}\right) \min(1, HUI) \\
    a_{stem}=\max \left[a_{stem\_f}, a_{stem} \max \left(0, \frac{1-HUI}{1-f_{GF}}\right)^{d_{sem}}\right] \\
    a_{grain}=1-a_{root}-a_{leaf}-a_{stem}
  \end{array}\right.
\end{equation}
% Please add the following required packages to your document preamble:
% \usepackage{booktabs}
\begin{table}[]
  \centering
  \caption{作物分配方案相关参数}
  \label{tab:作物分配方案相关参数}
\begin{tabular}{@{}lcccc@{}}
  \toprule
参数       & 小麦   & 水稻   & 玉米   & 大豆   \\ \midrule
$\alpha_{leaf\_i}$ & 0.9  & 0.75 & 0.8  & 0.85 \\
$\alpha_{root\_i}$ & 0.1  & 0.1  & 0.4  & 0.2  \\
$\alpha_{root\_f}$ & 0    & 0    & 0.05 & 0.2  \\
$\alpha_{stem\_f}$ & 0.05 & 0.05 & 0.0  & 0.3  \\
$d_{stem}$    & 1    & 1    & 2    & 5   \\\bottomrule
\end{tabular}
\end{table}
\section{收割和农业残留物处理}
收割时粒库$C_{grain}$进入种子库$C_{seed}$和食品库$C_{food}$
\begin{equation}
{C}_{ {seed }}=\min \left({C}_{ {grain }}, 3\right)
\end{equation}
\begin{equation}
{C}_{ {food }}={C}_{ {grain }}-{C}_{ {seed }}
\end{equation}
其中,种子库$C_{seed}$在下一个生长季出叶期的第一时步转移到叶库$C_{leaf}$。叶和茎的碳库进入地上凋落物库。根的碳库进入地下凋落物库。


\section{氮循环}
作物的氮循环和自然植被类似,但碳氮比(CN)不同,且要考虑大豆生物固氮的特殊性和施肥。

(1) 碳氮比(CN)\\
表~\ref{tab:作物碳氮比}中,碳氮比$CN_{leaf}$,$ CN_{stem}$, $CN_{root}$用于灌浆前,其余碳氮比用于灌浆开始后。\\
% Please add the following required packages to your document preamble:
% \usepackage{booktabs}
\begin{table}[]
  \centering
  \caption{作物碳氮比}
  \label{tab:作物碳氮比}
\begin{tabular}{@{}lcccc@{}}
\toprule
参数         & 小麦  & 水稻  & 玉米  & 大豆  \\ \midrule
$CN_{leaf}$     & 20  & 20  & 25  & 20  \\
$CN_{stem}$     & 50  & 50  & 50  & 50  \\
$CN_{root}$     & 42  & 42  & 42  & 42  \\
$CN_{leaf\_f}$  & 65  & 65  & 65  & 65  \\
$CN_{stem\_f}$  & 100 & 100 & 120 & 130 \\
$CN_{root\_f}$  & 40  & 40  & 0   & 0   \\
$CN_{grain\_f}$ & 50  & 50  & 50  & 50  \\ \bottomrule
\end{tabular}
\end{table}

(2) 大豆共生固氮 (SNF)\\
考虑大豆根瘤中的根瘤菌与大豆共生固氮,大豆共生固氮能力设为自然植物的4倍,假设其余作物类型无生物共生固氮能力。生物共生固定的氮直接被植物吸收。\\

(3) 施肥\\
氮肥考虑了工业氮肥和粪肥。工业氮肥是输入场。粪肥设为2 g $\rm N/m^2/yr$。
施肥时直接将氮肥输入土壤$NH_4^{+}$库,从出叶开始,以匀速的方式施肥20天。


\section{光合作用和气孔导度}
大豆、小麦、水稻采用C3植物光合方案;玉米采用C4植物光合方案。
在基于Medlyn方案计算气孔导度时,玉米的参数取值$g_1=1.8$,低于自然植被,其余作物的参数取值$g_1=5.8$,高于自然植被。该参数越大,气孔导度越高。


\section{臭氧污染胁迫}
考虑臭氧对光合速率和气孔导度的影响因子 $F_{O3_A}$ 以及$F_{O3_{gs}}$:
\begin{equation}
F_{O3_{A}}=-0.058 \log _{10}\left(POD_{0.5}\right)+0.883
\end{equation}
\begin{equation}
F_{O3_{gs}}=-0.109 \tanh \left(POD_{0.5}\right)+0.951
\end{equation}
其中,$POD_{0.5}$ (Phytotoxic ozone does above a threshold of 0.5 $nmol m^{-2} s^{-1}$,  单位:mmol $\rm m^{-2}$)是植物吸收的$O_3$通量累积量:
\begin{equation}
POD_{0.5}=\sum_{t}\left(\left[{O}_{3}\right] \times g_{s} \times D-0.5\right)
\end{equation}
在公式(15.10)中,$\left[{O}_{3}\right]$为陆面大气的臭氧浓度,$g_{s}$为叶气孔导度,
$D=0.663$为臭氧和水蒸气的扩散比,$t$是作物暴露在臭氧污染的时间,从抽芽开始。
上面计算的$F_{{O3}_A}$ 以及$F_{O3_{gs}}$分别作用在Farquhar-Collatz方案计算的光合速率以及Medlyn或Ball-Berry方案计算出的气孔导度。


\section{冬小麦春化现象}
春化响应$V$(0$\sim$1)采用\citet{streck2003incorporating}方案:
\begin{equation}
V=\frac{D_{v}{ }^{5}}{22.5^{5}+D_{v}^{5}}
\end{equation}
其中,$D_v$是物候阶段2累积的日春化率:
\begin{equation}\label{D_v_a}
D_{v}=\sum R_{v}
\end{equation}
在公式(\ref{D_v_a})中,$R_{v}$是日地表温度的函数:
\begin{equation}
R_{v} = \begin{cases} 
\frac{\left[2\left(T-T_{min}\right)^{a}\left(T_{opt}-T_{min}\right)^{a} - \left(T-T_{min}\right)^{2a}\right]}{\left(T_{opt}-T_{min}\right)^{2a}}, &T_{min} \leq T \leq T_{max} \\
0,  &T<T_{min} \quad  \text{or} \quad T>T_{max}
\end{cases}
\end{equation}
其中,$T_{min}$= –1.3 \textcelsius, $T_{opt}$= 4.9 \textcelsius, $T_{max}$= 15.7 \textcelsius。$ R_v$ 的取值从 0 (当$ T\leq T_{min}$ or $ \geq  T_{max}$) 到 1 ( 当$T=T_{opt}$)。
\begin{equation}
a=\frac{\ln 2}{\ln \left[\left(T_{\max }-T_{\min }\right) /\left(T_{o p t}-T_{\min }\right)\right]}
\end{equation}
春化响应用于调整$GDD_{base}$和粒分配系数
\begin{equation}
G D D_{b a s e}=G D D_{b a s e,  { unadjusted }} \times V
\end{equation}
\begin{equation}
a_{ {grain }}=a_{ {grain,unadjusted }} \times V_{f}
\end{equation}


\section{热胁迫}
作物对热胁迫的敏感期大约是在开花前后及灌浆期,在GPAM1里设为$HUI $满足$0.5 \leq HUI \leq 0.8$的时段。在每个发生热胁迫的时步,采用场数损失率$l$:
\begin{equation}
l=\left\{\begin{array}{cc}1-\left(\frac{4.3}{10.4}\right)^{\frac{\Delta t}{3600\times 21}}, & T>T_{c} \\ 0, & T \leq T_{c}\end{array}\right.
\end{equation}
其中,$\Delta t$ 是时步长,$T$是该时步的温度;$T_c$是临界温度:$T_c$=34 \textcelsius (水稻,玉米,冬小麦),31 \textcelsius (春小麦),37 \textcelsius (大豆)。
每个时步考虑了热胁迫后的叶$C_{leaf}$库为:
\begin{equation}
C_{leaf}=C_{leaf,  {unadjusted}} \times l
\end{equation}


\section{秸秆焚烧及排放}
年燃烧面积百分率BAF来自GFED4s农田燃烧面积百分率观测数据,为输入场。火发生是根据GFED4s的3小时数据找到的燃烧顶峰时步中心点,为输入场。

在得到燃烧面积后,火灾引起的碳排放$CE$(g C $\rm m^{-2} s^{-1}$) 为:
\begin{equation}
CE=BAF \times C_{ab,litter} \times CC
\end{equation}
$C_{ab,litter}$是地上凋落物碳库, $CC=0.8$是燃烧完全因子。

此后,我们估算火灾引起的33种痕量气体和气溶胶排放。火灾引起的第$i$类痕量气体和气溶胶排放量$E_i$ ($\rm g species m^{-2} s^{-1}$):
\begin{equation}
E_{i}=EF_{i} \times CE /[{C}]
\end{equation}
$EF_i$(g species (kg dry matter (DM))$^{-1}$)是排放因子,见表\ref{tab:秸秆焚烧排放因子取值}。$[C]=0.5×10^3$ $\rm g C (kg DM)^{-1}$是单位转换因子。


% Please add the following required packages to your document preamble:
% \usepackage{booktabs}
\begin{table}[]
  \centering
  \caption{秸秆焚烧排放因子取值(g species $(kg DM)^{-1}$)}
  \label{tab:秸秆焚烧排放因子取值}
  \begin{tabular}{@{}lccc@{}}
  \toprule
  \multicolumn{1}{c}{种类} & 排放因子 & 种类                     & 排放因子 \\ \midrule
  CO2                    & 1421 & C2H4   (ethylene)      & 1.14 \\
  CO                     & 78   & C3H6   (propylene)     & 0.48 \\
  CH4                    & 5.9  & C5H8   (isoprene)      & 0.18 \\
  NMHC                   & 5.8  & C10H16   (terpenes)    & 0.03 \\
  H2                     & 2.65 & C7H8   (toluene)       & 0.18 \\
  NOx                    & 2.67 & C6H6   (benzene)       & 0.31 \\
  N2O                    & 0.09 & C8H10   (xylene)       & 0.09 \\
  PM2.5                  & 8.5  & CH2O   (formaldehyde)  & 1.80 \\
  TPM                    & 11.3 & C2H4O   (acetaldehyde) & 1.82 \\
  TPC                    & 5.5  & C3H6O   (acetone)      & 0.61 \\
  OC                     & 5.0  & C3H6O2(hydroxyacetone) & 1.74 \\
  BC                     & 0.43 & C6H5OH   (Phenol)      & 0.50 \\
  SO2                    & 0.81 & NH3 (ammonia)          & 1.04 \\
  C2H6   (ethane)        & 0.76 & HCN (hydrogen cyanide) & 0.43 \\
  CH3OH   (methanol)     & 2.63 & MEK/2-butanone         & 0.60 \\
  C3H8   (propane)       & 0.20 & CH3CN   (acetonitrile) & 0.25 \\
  C2H2   (acetylene)     & 0.32 &                        &      \\ \bottomrule
  \end{tabular}
  \end{table}


\section{灌溉}
灌溉水直接进入到达地表的水,不经过截流。灌溉包括两个方案:
\begin{enumerate}
  \item 灌溉量和灌溉时间为输入场;
  \item CLM5灌溉模型。
\end{enumerate}
\section{结构形态}
出叶后到收割前,叶面积为:
\begin{equation}
LAI=C_{leaf} \times SLA
\end{equation}
SLA为比叶面积 ($\rm m^2 \,leaf\, g^{-1}\, C$)(表\ref{tab:结构形态参数})。

茎叶面积为:
\begin{equation}
SAI=\left\{\begin{array}{lcc}0.1 LAI, & \text { for } & \text { maize } \\
   0.2 LAI, & \text { for } & \text {others}\end{array}\right.
\end{equation}

冠层顶高度$z_{top}$(米)和冠层底高度$z_{bot}$(米)为:
\begin{equation}
z_{top}=\max \left[0.05, z_{\max} \min \left(\frac{GDD_{base}}{f_{GE} G D D_{mat}}, 1.0\right)\right]
\end{equation}
\begin{equation}
z_{b o t}=0.02
\end{equation}

% Please add the following required packages to your document preamble:
% \usepackage{booktabs}
\begin{table}[]
  \centering
  \caption{结构形态参数}
  \label{tab:结构形态参数}
  \begin{tabular}{@{}ccccc@{}}
  \toprule
  参数   & 小麦   & 水稻    & 玉米   & 大豆    \\ \midrule
  $SLA$  & 0.05 & 0.035 & 0.05 & 0.035 \\
  $z_{max}$ & 1    & 1.5   & 3    & 1     \\ \bottomrule
  \end{tabular}
\end{table}

\chapter{水库模式}\label{水库模式}
\echapter{Reservoir Model}
%\addcontentsline{toc}{chapter}{陆地表面的水分循环}
\begin{mymdframed}{代码}
  本章对应的代码文件为\texttt{CaMa/src/cmf\_ctrl\_damout\_mod.F90}。
\end{mymdframed}

本模块在CaMa-Flood中实施水库运行方案,通过识别坝体所在的集水区单元,根据水库运行规则计算水库流出量,构建考虑大坝影响的河道汇流参数化方案,以刻画水库对陆面水文循环过程的影响。

\section{水库调度规则参数化方案}
\esection{Reservoir Regulation Parameterization}
在大尺度陆面模式中进行水库扰动模拟,需要对水库的调度规则进行一定的假设和概化,现有方案大致可分为三类:基于入流和需求的调度方案、基于目标库容的分段调度方案和基于离线优化的调度方案\citep{yassin2019representation},其中以前两种方法应用最为广泛。本版本已加入目前陆面模式中采用的主流水库调度参数化方案,如表~\ref{tab:水库模块可选水库调度方案} 所示,可通过相应的方案代码进行调用。

\begin{table}[htbp]
  \centering
  \caption{水库模块可选水库调度方案}
  \label{tab:水库模块可选水库调度方案}
  \begin{tabular}{cc>{\centering\arraybackslash}p{5.5cm}}
    \toprule
    方案类型                                     & 方案代码 & 代表模式                                    \\ \midrule
    \multirow[b]{2}{*}{基于入流和需求的调度方案} & H06      & H08, WaterGAP, WBMplus, MATSIRO, DBH, LPJmL \\ \cline{2-3}
                                                 & V13      & MOSART‐WM                                   \\
    \hline
    \multirow{2}{*}{基于目标库容的分段调度方案}  & LIS      & LISFLOOD, CWatM                             \\ \cline{2-3}
                                                 & H22      & CAMA-Flood                                  \\
    \bottomrule
  \end{tabular}
\end{table}

\subsection{H06方案}
\esubsection{H06 Scheme}
H06表示~\citet{hanasaki2006reservoir}提出的基于入流和需求的调度方案,该方案根据水库用途、水库初始库容、水库最大库容、模拟入流量和下游需水量等信息,分别为灌溉水库和非灌溉水库(除灌溉之外的用途均为非灌溉)设定调度规则,计算公式如下:
\begin{equation}
  Q_{\mathrm{out}}=\begin{cases}
    k_{\mathrm{rls}} \times Q_{\mathrm{out}}^{'}, & \quad c \geqslant 0.5 \\
    R \times k_{\mathrm{rls}} \times Q_{\mathrm{out}}^{'}+\left(1-R\right) \times Q_{\mathrm{in}}, & \quad c<0.5
  \end{cases}
\end{equation}
%
其中:
\begin{equation}
  \begin{cases}
    \text{灌溉水库,} Q_{\mathrm{out}}^{'}= \begin{cases}
      Q_{\mathrm{n}}+D-\overline{D}, &\quad {\rm DPI}<(1-M)\\
      Q_{\mathrm{n}} \times \left(M+\frac{\left(1-M\right)D}{\overline{D}}\right), &\quad {\rm DPI} \geqslant (1-M)
    \end{cases}\\
    \text{非灌溉水库,} Q_{\mathrm{out}}^{'}=Q_{\mathrm{n}}
  \end{cases}
\end{equation}

上式中变量和参数的含义如表~\ref{tab:H06方案变量参数表} 所示。

\begin{table}[htbp]
  \centering
  \caption{H06方案中的变量和参数的含义}
  \label{tab:H06方案变量参数表}
  \begin{threeparttable}
    \begin{tabular}{ccc}
      \toprule
      名称               & 含义/计算方法                                                            & 单位              \\ \midrule
      $Q_{\mathrm{out}}$ & 水库出流量[变量]                                                         & \unit{L^3.T^{-1}} \\
      $Q_{\mathrm{in}}$  & 水库入流量[变量]                                                         & \unit{L^3.T^{-1}} \\
      $D$                & 当日需水量[变量]                                                         & \unit{L^3.T^{-1}} \\
      $Q_{\mathrm {n}} $ & 水库多年平均正常流量[参数]                                               & \unit{L^3.T^{-1}} \\
      $\overline{D}$     & 多年平均日需水量[参数]                                                   & \unit{L^3.T^{-1}} \\
      $M$                & 最小需求满足系数[参数]                                                   & -                 \\
      DPI                & 年平均需求量与年平均水库入流量的比值[参数]                               & -                 \\
      $k_{\mathrm{rls}}$ & 释放系数[参数],$k_{\mathrm{rls}} = \frac{V_0}{\alpha V_{\mathrm {t}} }$ & -                 \\
      $R$                & 需求决定系数[参数],$R=\min(1, \alpha c)$                                & -                 \\
      $\alpha$           & 无量纲常数[参数]                                                         & -                 \\
      $c$                & 总库容量与年总水库入流量的比值[参数]                                     & -                 \\
      $V_0$              & 初始蓄水量[参数]                                                         & \unit{L^3}        \\
      $V_{\mathrm {t}} $ & 水库最大库容[参数]                                                       & \unit{L^3}        \\
      \bottomrule
    \end{tabular}
    \begin{tablenotes}
      \footnotesize
    \item[注:] 原 H06 方案中 $R=\min(1,4 c^2)$,\citet{Shin-etal_19} 将该参数的计算修改为 $R=\min(1,\alpha c)$,以修正原方案导致库容模拟不稳定的问题。
    \end{tablenotes}
  \end{threeparttable}
\end{table}

\subsection{V13方案}
\esubsection{V13 Scheme}
V13表示~\citet{voisin2013improved}提出的调度方案,在H06方案的基础上增加了防洪期以同时实现防洪和灌溉目标。该方案根据水库多年平均入库流量的季节变化,定义了三个特征时刻:NDFC表示防洪期结束月,为运营期开始前的湿润期的第一个月;STFC表示防洪期开始月,为运营期开始前的枯水期内流量最低的月份;STOP表示运营期的开始月,\citet{hanasaki2006reservoir}将运营期的开始定义为长期平均月流量低于长期平均年流量的第一个月。
在STFC到NDFC时段内,V13方案在H06方案的基础上增加额外的出流,从而使水库为汛期的到来留出更多库容,在原V13方案中该时段内每个时刻的额外出流:
\begin{equation}
  Q_{\rm drop} = \frac{\sum_{i = \text{STFC}}^{\text{NDFC} - 1}{{(Q}_{\text{in}}^{i} - Q_{\mathrm{n}})}}{\left( \text{NDFC} - \text{STFC} \right)}
\end{equation}
该方案为回溯性方案,如不选择回溯性方案,可将额外出流简化为:
\begin{equation}
  Q_{\rm drop} = \left| Q_{\text{in}}^{i} - Q_{\mathrm{n}} \right|
\end{equation}
在NDFC到STOP时段内,通过只释放年平均流量$Q_{\mathrm{n}}$来确保水库为灌溉季节再次蓄水。

\subsection{LIS方案}
\esubsection{LIS Scheme}
LIS表示LISFLOOD所采用的基于目标库容的分段调度方案(\url{https://ec-jrc.github.io/lisflood-model/3\_03\_optLISFLOOD\_reservoirs/}),计算公式如下:
\begin{equation}
  \begin{aligned}
    F > L_{\mathrm{f}}, \quad &Q_{\mathrm{out}}=\max\left[Q_{\mathrm{max}},\ (F-L_{\mathrm{f}}-0.01) \times \frac{V_{\mathrm {t}} }{\Delta t}\right] \\
    \quad &Q_{\mathrm{max}} = \min\left[Q_{\mathrm{f}},\ {\max(1.2 \times Q_{\mathrm{in}},Q_{\mathrm{adj,n}})}\right]
  \end{aligned}
\end{equation}
\begin{equation}
  L_{\mathrm{adj,f}} < F < L_{\mathrm{f}}, \quad Q_{\mathrm{out}}=Q_{\mathrm{adj,n}}+\left(Q_{\mathrm{f}}-Q_{\mathrm{adj,n}}\right) \times \frac{F-L_{\mathrm{adj,f}}}{L_{\mathrm{f}}-L_{\mathrm{adj,f}}}
\end{equation}
\begin{equation}
  L_{\mathrm{n}} < F < L_{\mathrm{adj,f}}, \quad Q_{\mathrm{out}}=Q_{\mathrm{adj,n}}
\end{equation}
\begin{equation}
  2L_{\mathrm{c}} < F < L_{\mathrm{n}}, \quad Q_{\mathrm{out}}=Q_{\mathrm{min}}+\left(Q_{\mathrm{adj,n}}-Q_{\mathrm{min}}\right) \times \frac{F-2L_{\mathrm{c}}}{L_{\mathrm{n}}-2L_{\mathrm{c}}}
\end{equation}
\begin{equation}
  F < 2L_{\mathrm{c}}, \quad Q_{\mathrm{out}}=\min\left[Q_{\mathrm{min}}, \frac{V_{\mathrm {t}} }{\Delta t}\right]
\end{equation}
如果出现$F<L_{\mathrm{f}}$且$Q_{\mathrm{out}}>\min(1.2 \times Q_{\mathrm{in}},\ Q_{\mathrm{adj,n}})$,则对水库出流结果进行以下修正:
\begin{equation}
  Q_{\mathrm{out}} = \min\left[Q_{\mathrm{out}}, \max(Q_{\mathrm{in}}, Q_{\mathrm{adj,n}})\right]
\end{equation}
上式中变量和参数的含义如表~\ref{tab:LIS方案变量参数表} 所示。

\begin{table}[htbp]
  \centering
  \caption{LIS方案中的变量和参数的含义}
  \label{tab:LIS方案变量参数表}
%    \begin{threeparttable}
  \begin{tabular}{ccc}
    \toprule
    名称                 & 含义/计算方法                                                               & 单位              \\ \midrule
    $Q_{\mathrm{out}}$   & 水库出流量[变量]                                                            & \unit{L^3.T^{-1}} \\
    $Q_{\mathrm{in}}$    & 水库入流量[变量]                                                            & \unit{L^3.T^{-1}} \\
    $V$                  & 水库实时蓄水量[变量]                                                        & \unit{L^3}        \\
    $F$                  & 实时蓄水系数[变量], $F=\frac{V}{V_{\mathrm{t}}}$                            & -                 \\
    $L_{\mathrm{f}}$     & 防洪库容系数[参数], $L_{\mathrm{f}}=\frac{V_{\mathrm{f}}}{V_{\mathrm{t}}}$  & -                 \\
    $L_{\mathrm{n}}$     & 正常库容系数[参数], $L_{\mathrm{n}}=\frac{V_{\mathrm{n}}}{V_{\mathrm{t}}}$  & -                 \\
    $L_{\mathrm{c}}$     & 保留库容系数[参数], $L_{\mathrm{c}}=\frac{V_{\mathrm{c}}}{V_{\mathrm{t}}}$  & -                 \\
    $L_{\mathrm{adj,f}}$ & \makecell{调整防洪库容系数[参数],                                          \\
    $L_{\mathrm{adj,f}} = L_{\mathrm{n}}+AdjL_{\mathrm{n}} \times \left(L_{\mathrm{f}}-L_{\mathrm{n}}\right)$}
                         & -                                                                           \\
    $AdjL_{\mathrm{n}}$  & 无量纲调整系数[参数],取值$\left(0,1\right)$                                & -                 \\
    $Q_{\mathrm{f}}$     & 防洪流量[参数]                                                              & \unit{L^3.T^{-1}} \\
    $Q_{\mathrm{n}}$     & 正常流量[参数]                                                              & \unit{L^3.T^{-1}} \\
    $Q_{\mathrm{min}}$   & 最小环境流量[参数]                                                          & \unit{L^3.T^{-1}} \\
    $Q_{\mathrm{adj,n}}$ & \makecell{调整正常流量[参数],                                              \\
    $Q_{\mathrm{adj,n}}=\max(Q_{\mathrm{min}},\,\min(AdjQ_{\mathrm{n}} \times Q_{\mathrm{n}},\,Q_{\mathrm{f}}))$}
                         & \unit{L^3}                                                                  \\
    $AdjQ_{\mathrm{n}}$  & 无量纲调整系数[参数],取值$\left(0.25,2\right)$                             & -                 \\
    \bottomrule
  \end{tabular}
%    \end{threeparttable}
\end{table}

\subsection{H22方案}
\esubsection{H22 Scheme}
H22表示~\citet{hanazaki2022development}提出的调度方案,根据水库入流量和实时库容大小,调节出库流量以削减汛期峰值流量,控制出库流量不超过防洪流量,且根据汛期发展时段调整出流系数,计算公式如下:
\begin{equation}
  V_{\mathrm{e}} < V < V_{\mathrm{t}}, \quad Q_{\mathrm{out}} = \max(Q_{\mathrm{f}},\, Q_{\text{in}})\
\end{equation}
\begin{equation}
  V_{\mathrm{f}}<V<V_{\mathrm{e}},\ Q_{\mathrm{out}} = \begin{cases}
    Q_{\mathrm{f}}+k \times \frac{V-V_{\mathrm{c}}}{V_{\mathrm{e}}-V_{\mathrm{c}}} \times \left[(Q_{\mathrm{in}}-Q_{\mathrm{f}})\right], & \quad Q_{\mathrm{in}} > Q_{\mathrm{f}}\\
    0.5Q_{\mathrm{n}}+k \times \left(\frac{V-V_{\mathrm{c}}}{V_{\mathrm{e}}-V_{\mathrm{c}}}\right)^2 \times (Q_{\mathrm{f}}-Q_{\mathrm{n}}), & \quad Q_{\mathrm{in}} < Q_{\mathrm{f}}
  \end{cases}
\end{equation}
\begin{equation}
  V_{\mathrm{c}}<V<V_{\mathrm{f}},\ Q_{\mathrm{out}} = \begin{cases}
    0.5Q_{\mathrm{n}}+k \times \frac{V-V_{\mathrm{c}}}{V_{\mathrm{f}}-V_{\mathrm{c}}} \times \left[(Q_{\mathrm{f}}-Q_{\mathrm{n}})\right], & \quad Q_{\mathrm{in}} > Q_{\mathrm{f}} \\
    0.5Q_{\mathrm{n}}+k \times \left(\frac{V-V_{\mathrm{c}}}{V_{\mathrm{e}}-V_{\mathrm{c}}}\right)^2 \times (Q_{\mathrm{f}}-Q_{\mathrm{n}}), & \quad Q_{\mathrm{in}} < Q_{\mathrm{f}}
  \end{cases}
\end{equation}
\begin{equation}
  V < V_{\mathrm{c}}, \quad Q_{\mathrm{out}} = Q_{\mathrm{n}} \times \frac{V}{V_{\mathrm{f}}}
\end{equation}
上式中变量和参数的含义如表~\ref{tab:H22方案变量参数表} 所示。

\begin{table}[htbp]
  \centering
  \caption{H22方案中的变量和参数的含义}
  \label{tab:H22方案变量参数表}
%    \begin{threeparttable}
  \begin{tabular}{ccc}
    \toprule
    名称                                                                & 含义/计算方法                   & 单位              \\ \midrule
    $Q_{\mathrm{out}}$                                                  & 水库出流量[变量]                & \unit{L^3.T^{-1}} \\
    $Q_{\mathrm{in}}$                                                   & 水库入流量[变量]                & \unit{L^3.T^{-1}} \\
    $V$                                                                 & 水库实时蓄水量[变量]            & \unit{L^3}        \\
    $V_{\mathrm{t}}$                                                    & 最大库容[参数]                  & \unit{L^3}        \\
    $V_{\mathrm{e}}$                                                    & 警戒库容[参数]                  & \unit{L^3}        \\
    $V_{\mathrm{f}}$                                                    & 防洪库容[参数]                  & \unit{L^3}        \\
    $V_{\mathrm{c}}$                                                    & 保留库容[参数]                  & \unit{L^3}        \\
    $Q_{\mathrm{f}}$                                                    & 防洪流量[参数]                  & \unit{L^3.T^{-1}} \\
    $Q_{\mathrm{n}}$                                                    & 正常流量[参数]                  & \unit{L^3.T^{-1}} \\
    $K$                                                                 & \makecell{出流调节系数[参数], \\
    取值$\max\left(1-\frac{V_{\mathrm{t}}-V_{\mathrm{f}}}{A},0\right)$} & -                              \\
    $A$                                                                 & 水库汇流面积[参数]              & \unit{L^2}        \\
    \bottomrule
  \end{tabular}
%    \end{threeparttable}
\end{table}

\section{考虑大坝影响的河道汇流参数化方案}
\esection{River Flow with Dam Influence}
相比自然河道的汇流方案,考虑大坝影响的汇流方案存在以下改动:当河道中存在大坝时,由于水的连续流动和河道的水力连结被切断,在与大坝网格紧连的下游网格处,圣维南动力方程中的局部加速项和静水压力项被忽略,在使用局部惯性方程(CaMa-Flood中使用的默认算法)进行一般计算后,使用运动波动方程重新计算流量。在下一步中,将来自上游河网的所有流出汇总为水库流入,并根据重新计算的流入和蓄水量确定水库流出。
\begin{equation}
  {\mathrm g} A \frac{\partial z}{\partial x}+\frac{{\mathrm g} n^{2}|Q| Q}{R_{\mathrm{H}}^{4 / 3} A}=0
\end{equation}
式中变量含义同章节~\ref{诊断洪泛状态} 所述。


此外,由于大坝对汇流过程的拦截作用,在计算大坝网格下游格点的河道蓄水和淹没时,质量连续方程中的汇流网格和汇流面积将不再包括大坝网格上游的部分,且与大坝网格紧连的下游网格其入流径流量替换为水库出流量。
\begin{equation}
  S_{i}^{t+\Delta t}=S_{i}^{t}+\sum_{k}^\text{Upstream-2} Q_{k}^{t} \Delta t-Q_{i}^{t} \Delta t+A c_{i} R_{i}^{t} \Delta t
\end{equation}
式中Upsteam-2表示在原上游网格的基础上除去大坝上游网格,其他变量含义同章节~\ref{诊断洪泛状态} 所述。


\chapter{土地利用与土地覆盖变化模拟}\label{土地利用与土地覆盖变化模拟}
%\addcontentsline{toc}{chapter}{土地利用与土地覆盖变化模拟}
\begin{mymdframed}{代码}
本章对应代码源文件位于\texttt{main/LULCC/}目录下。
\end{mymdframed}

\section{模拟方案概述}
土地利用与土地覆盖变化是陆面模拟中重要强迫项之一,对模拟结果具有不可忽视的影响,特别是在局地及区域尺度。
目前CoLM模式对土地利用与土地覆盖变化的模拟可以分为三种方式:
\begin{enumerate}
    \item 通过人为设定不同年份的地表输入数据,此为静态方案;
    \item 逐年更新,采用同次网格类型(包括IGBP、PFT、PC和Urban)状态变量进行赋值(hot-start),新增加类型进行初始化(cold-start),为同类赋值方案(Same Type Assignment, SAT);
    \item 在方案2的基础上,利用地表覆盖转移矩阵,对状态变量赋值时保持物质和能量守恒,即守恒方案(Mass
and Energy Conservation, MEC)。
\end{enumerate}

静态方案通过\texttt{namelist DEF\_LC\_YEAR}指定地表覆盖数据年份。目前提供2001-2022年地表输入数据(采用MODIS地表覆盖数据为基础制作而成),运行时指定模式地表数据年份即可,通过以上设置生成地表数据,进行初始化及主程序编译,生成的程序会读入指定年份的地表数据。该方式设置简单,适用于不同年份地表数据气候态结果分析和地表覆盖变化影响机理研究。

同类赋值和守恒方案则通过\texttt{namelist DEF\_LULCC\_SCHEME}进行设置,当
\texttt{DEF\_LULCC\allowbreak \_SCHEME = 1}时为同类赋值方案,\texttt{DEF\_LULCC\_SCHEME = 2}时为守恒方案。

对于同类赋值和守恒方案,有几点需要说明:

1) 对于所模拟的年份,需要提前制作好地表数据,可通过类似静态方案设置\texttt{namelist DEF\_LC\_YEAR}逐年生成(由于生成地表数据需要大量计算资源,在目前阶段暂不采用代码自动实现的方式来生成LULCC地表数据,而是根据用户需求手动设置生成);

2) 对于为模式发展自行新增的状态变量并不能自动实现LULCC,需要手动添加对应的变量代码;

3) LULCC目前暂不支持USGS地表分类、生物地球化学模块及单点运行;

4) LULCC方案打开后LAI将会根据模式运行时间自动更新,覆盖\texttt{namelist DEF\- \_LAI\_CHANGE\_YEARLY}设置。


\section{LULCC运行基本流程}
\begin{mymdframed}{代码}
本节对应代码文件为\texttt{LULCC/MOD\_Lulcc\_Driver.F90}。
\end{mymdframed}

{
\begin{figure}[htbp]
\centering
\includegraphics[width=0.85\columnwidth]{Figures/土地利用与土地覆盖变化模拟/LULCCDRIVER流程图.jpg}
\caption{LULCC主程序流程图 (\texttt{MOD\_LULCC\_Driver.F90})}
\label{fig:LULCC主程序流程图}
\end{figure}
}

对于同类赋值方案和守恒方案,目前模式提供2001-2022年地表输入数据,运行时指定模式初始运行年份地表覆盖数据,并在\texttt{define.h}文件中打开土地利用与土地覆盖变化宏(\texttt{\#define LULCC}),即可每年动态更新地表数据。通过以上设置产生多年地表数据,使用开始年地表数据完成模式初始化(restart run方式除外),主程序会相应读入模型运行年份的地表数据。地表数据的更新在每年最后一个时间步长结束后进行。

状态变量也需随地表数据的更新而更新,具体过程如下:

\begin{enumerate}
\item 当运行到每年的最后一个时刻,首先将模式运行相关变量内存释放,以确保patch数目在模式的后续运行中能与新一年的地表数据匹配。

\item 然后执行LULCC主程序,即对restart run所需的相关变量进行调整,主要包含以下步骤:1. 首先将最后一个时刻的状态变量和常量保存,用于后续对新一年状态变量的赋值或调整;2. 根据新一年的地表数据运行\texttt{mkinidata},对新一年的变量进行初始化。若变量在后续没有进行赋值或调整,则会保持该初始化的值,如叶面积指数(LAI);3. 根据\texttt{DEF\_LULCC\_SCHEME}选项设置的LULCC方案进行变量调整。

\item 最后将调整完的所有变量重新输出到restart文件,作为新一年模式运行的初始值,并根据新一年的patch重新初始化大气强迫与通量数据。
\end{enumerate}

如图~\ref{fig:LULCC主程序流程图}所示,同类赋值与守恒方案的不同仅在于对状态变量的调整,地表数据的更新方式以及调整前的准备步骤是完全相同的。

\section{同类赋值方案-SAT}
\begin{mymdframed}{代码}
本节对应代码文件为\texttt{MOD\_Lulcc\_Vars\_TimeVariables.F90}。
\end{mymdframed}

对于同类赋值方案,这里的同类包括植被次网格类型LCT、PFT和PC,同时也适用于打开城市模式后城市分类。下一年初始的状态变量由上一年最后一个时刻相同次网格类型上保存的状态变量赋值,若下一年存在新增类型,则其状态变量由初始化的数值代替。

该方式实现容易,简单考虑地表数据逐年替换,后续需要新增变量也可很快加入。同类赋值方案比较适合一些快变过程变量模拟,比如反照率、温度模拟等。但对于一些慢变过程(具有较长时间记忆)变量,如土壤水等,建议采用下面守恒方案。

\section{守恒方案-MEC}
\begin{mymdframed}{代码}
本节对应代码文件为\texttt{MOD\_Lulcc\_MassEnergyConserve.F90}。
\end{mymdframed}

{
\begin{figure}[htbp]
\centering
\includegraphics[width=\textwidth]{Figures/土地利用与土地覆盖变化模拟/LULCC流程图.png}
\caption{地表覆盖变化的物质能量守恒方案示意图}
\label{fig:LULCC流程图}
\end{figure}
}

在守恒方案中,变量的调整主要通过两个程序完成,即1) 制作相邻两年的地表覆盖追溯数据;2) 利用上一步生成的数据,对状态变量进行调整,以保持物质和能量守恒。

地表覆盖追溯转移矩阵的制作需首先在模式网格上获取今年的每一个patch所包含的pixel,以及pixel在去年的地表类型,对pixel的类型进行计数。然后计算每一种地表类型占网格面积(不考虑网格中的海洋面积)的比例,作为后续状态变量守恒调整的权重。该结果会作为诊断数据输出,可用于地表覆盖变化分析。

当获得每个patch的来源patch占比后,即可对restart文件中涉及变量的调整。守恒方案中的状态变量调整可以大致分为以下几种情况:


1) \textbf{物质守恒}。与物质相关的变量,如叶上的积水$L_{dew}$、土壤液态水含量$w_{liq}$、土壤固态水含量$w_{ice}$
、含水层水量$w_{a}$,新一年patch上的值采用来源patch的面积加权平均进行赋值。部分变量虽然不是守恒量,但也采用同样的方法进行调整,包括雪龄 (snow age) 、雪的有效粒径$snw\_rds$等。

2) \textbf{能量守恒}。与能量相关的变量,包括雪层和土壤层的温度$T$,是基于所有来源patch变化到新的温度之后,热量的变化总和为0的假设来计算:
\begin{equation}
T^{n} = \sum_{i = 1}^{\rm np}{\,\left[ T_{i}^{n - 1}\,\cdot\,\frac{C_{vi}^{n - 1} S_{i}^{n - 1}}{\sum_{i = 1}^{\rm np}{(C_{vi}^{n - 1} S_{i}^{n - 1})}}\right]}
\end{equation}
其中\(T_{i}^{n - 1}\)是去年来源patch的土壤温度,\(T^{n}\)是今年的patch的土壤温度,\(S_{i}^{n - 1}\)是来源patch的面积百分比,np为来源patch数量,\(C_{vi}^{n - 1}\)是土壤热容量(单位:\( \rm Jm^{-2}K^{-1}\),\(C_{vi}^{n - 1}\)的计算同土壤温度计算(\texttt{MOD\_GroundTemperature.F90})中的方案。因此土壤温度调整的权重为来源patch的面积与土壤热容量的乘积。

雪层的温度计算与土壤层略有不同,考虑了不同来源patch合并后可能发生的水相态变化。计算过程参考雪层的压实过程(\texttt{MOD\_SnowLayersCombineDivide.F90})中的方案,即新patch的焓与所有来源patch的焓的总和相等,依据焓值即可逐层计算新patch的雪层温度。

3) \textbf{通过物理过程计算调整}。经过以上物质和能量守恒调整后,一些变量会因物理过程的计算受到影响,此时需要根据物理过程部分的代码进行调整,包括雪层的厚度$\delta z$、雪层的深度$z$、雪水当量$W_{sno}$、积雪厚度$z_{sno}$、被积雪覆盖的地表面积比例$f_{sno}$、斑块中的有效植被比例$f_{sig}$等等。这些变量的调整需要首先确定雪的层数,雪的层数被设置为来源patch中面积占比最多的patch的雪层数,即以主要来源patch为准。其中每一层的含水量、含冰量以及雪的密度可通过来源patch对应层的数值做面积加权平均调整。当出现其他patch的雪层数多于主要patch的层数的情况时,则将多出来的部分采用同样的方式加到最上层。此时即可根据含水量、含冰量和雪的密度来计算雪层的厚度,同时也可计算出雪层的深度$z$、雪水当量$W_{sno}$和积雪厚度$z_{sno}$。被积雪覆盖的地表面积比例$f_{sno}$和斑块中的有效植被比例$f_{sig}$通过调用雪盖比例子程序\texttt{snowfraction}进行计算,同时将茎面积指数$SAI$更新。地表温度$T_{g}$采用调整后的最上层的温度(没有雪则采用土壤第一层的温度)。地下水位$z_{wt}$采用来源patch中水位最低的patch进行赋值。

4) \textbf{保持初始化或采用同类赋值}。叶面积指数$LAI$、茎面积指数$SAI$是在新一年初始化时从地表数据读取得到的。反照率$\alpha$、辐射吸收($ssun$, $ssha$)、消光系数($extkb$, $extkd$)、叶片温度$T_{leaf}$、植被覆盖度$f_{veg}$等根据新一年的植被状态进行初始化。与湖泊地表类型相关的变量如湖泊冻结比例$T$、湖泊温度$T_{lake}$等变量采用同类赋值方法。

当打开城市模型,每个存在的城市类别都单独成为一个patch,城市patch的变量调整采用同类赋值方案。但假如去年不存在该类城市patch,则会用patch所在网格中的其他城市类型patch的相关变量进行赋值。若去年的网格中也不存在城市patch,则采用初始化的值。下垫面与透水面相关的变量采用其来源patch中的土壤类型patch进行赋值。为与城市中的水量分量组成及计算保持一致,土壤液态水含量$w_{liq}$、土壤固态水含量$w_{ice}$和雪水当量$W_{sno}$都参考城市模型物理过程部分的代码重新进行调整。

目前的守恒调整仅考虑向土壤、城市、湿地、湖泊类型patch的转换,还未考虑向冰川类型的转换,冰川patch的变量采用同类赋值调整。当定义地表次网格方案为LULC\_PFT或LULC\_PC时,patch索引的变量调整与IGBP完全一致。由于所有PFT共享土壤,PFT/PC索引的变量采用同类赋值(例如$T_{leaf}\_p$、$L_{dew}\_p$、$f_{sig}\_p$),而$SAI_p$的计算与LULC\_IGBP类似,除了初始化之外,还考虑当前雪的覆盖情况,通过调用雪盖比例计算子程序\texttt{snowfraction\_pftwrap}进行调整。在patch尺度上,植被patch上的对应变量则按PFT/PC的百分比重新进行加权平均计算。

守恒方案较前两种物理上更为合理,具有同类赋值方案的特点,同时满足物质和能量守恒。

%\end{parbox}

\clearpage
\phantomsection % Correct the link in toc for Appendix
\addcontentsline{toc}{part}{附录}
\addcontentsline{toe}{part}{Appendix}
\part*{附录}{Appendix}
%\epart*{Appendix}

%\cleardoublepage
\appendix
\clearpage
%\addcontentsline{toc}{chapter}{附录}
\chapter{模式相关物理常数/参数}\label{模式相关物理常数参数}
%\addcontentsline{toc}{chapter}{附录}
%\begin{附录}
%\section{附录 A: 模式相关物理常数/参数}
% Please add the following required packages to your document preamble:
% \usepackage{booktabs}
\begin{table}[]
    \centering
    \caption{物理常数}
    \label{tab:物理常数}
    \begin{tabular}{@{}lc@{}}
    \toprule
    名称 {[}单位{]}                                      & 常数值         \\ \midrule
    固态水密度 {[}$\rm kg/m^3${]}                                & 917.        \\
    液态水密度 {[}$\rm kg/m^3${]}                                & 1000.       \\
    液态水比热容 {[}J/kg/K{]}                              & 4188.       \\
    固态水比热容 {[}J/kg/K{]}                              & 2117.27     \\
    干空气比热容 {[}J/kg/K{]}                              & 1004.64     \\
    固态水液化潜热 {[}J/kg{]}                               & $0.3336\times 10^6$  \\
    液态水蒸发潜热 {[}J/kg{]}                               & $2.5104\times 10^6$  \\
    固态水升华潜热 {[}J/kg{]}                               & $2.8440\times 10^6$  \\
    空气热力传导率 {[}W/m/K{]}                              & 0.023       \\
    固态水热力传导率 {[}W/m/K{]}                             & 2.290       \\
    液态水热力传导率 {[}W/m/K{]}                             & 0.6         \\
    液态水凝结温度 {[}K{]}                                  & 273.16      \\
    干空气气体常数 {[}J/kg/K{]}                             & 287.04      \\
    rw/g = (8.3144/0.018)/(9.80616)*1000. {[}mm/K{]} & $4.71047\times 10^4$ \\
    水汽气体常数{[}J/(kg K){]}                             & 461.296     \\
    重力加速度 {[}$\rm m/s^2${]}                                 & 9.80616     \\
    von Karman常数 {[}-{]}                             & 0.4         \\
    Stefan-Boltzmann常数 {[}$\rm W/m^2/K^4${]}                 & $5.67 \times 10^{-8}$   \\\bottomrule
    \end{tabular}
\end{table}

% Please add the following required packages to your document preamble:
% \usepackage{booktabs}
\begin{table}[]
\centering
\caption{其他全局参数设定值。}
\label{tab:其他全局参数设定值}
\begin{tabular}{@{}lc@{}}
\toprule
名称 {[}单位{]}                               & 参数值      \\ \midrule
土壤粗糙度 {[}m{]}                             & 0.01     \\
积雪覆盖地表粗糙度 {[}m{]}                         & 0.0024   \\
林下冠层与土壤热力/水汽交换系数 {[}-{]}                  & 0.004    \\
叶片最大载水量 {[}mm{]}                          & 0.1      \\
表层土壤水饱和区最大覆盖比例 (Niu et al., 2005) {[}-{]} & 0.38     \\
表层温度到表面温度转换系数 {[}-{]}                     & 0.34     \\
Crank Nicholson隐式格式权重因子 {[}-{]}           & 0.5      \\
积雪孔隙中的最大液态水饱和度 {[}-{]}                    & 0.033    \\
土壤可透水最小孔隙度 {[}-{]}                        & 0.05     \\
地面最大积水深度 {[}mm{]}                         & 10.0     \\
萎焉点土壤水势 {[}mm{]}                          & $-1.5\times 10^5$ \\
最小土壤水势 {[}mm{]}                           & $-1\times 10^8$   \\
最大蒸腾速率 {[}$\rm mm/s${]}                         & $2\times 10^{-4}$   \\
确定雨/雪临界温度 {[}\textcelsius {]}                         & 2.5      \\ \bottomrule
\end{tabular}
\end{table}


\chapter{USGS地表覆盖类型相关参数}\label{USGS地表覆盖类型相关参数}

% Please add the following required packages to your document preamble:
% \usepackage{booktabs}
\begin{table}[]
    \centering
    \caption{USGS地表覆盖植被顶部高度和底部高度值(当无植被高度数据读入时的默认高度)。}
    \label{tab:USGS地表覆盖植被顶部高度和底部高度值}
    \begin{tabular}{@{}lcc@{}}
    \toprule
    USGS地表覆盖类型     & 植被顶部高度 (m) & 植被底部高度 (m) \\ \midrule
    1 城市           & 1          & 0          \\
    2 干旱农田与牧场      & 0.5        & 0          \\
    3 灌溉农田与牧场      & 0.5        & 0          \\
    4 干旱/灌溉混合农田与牧场 & 0.5        & 0          \\
    5 农田草地过渡带      & 0.5        & 0          \\
    6 农田林地过渡带      & 0.5        & 0          \\
    7 草地           & 0.5        & 0          \\
    8 灌木地          & 0.5        & 0          \\
    9 草地灌木地混合带     & 0.5        & 0          \\
    10 稀疏草原        & 0.5        & 0          \\
    11 落叶阔叶林       & 20         & 1          \\
    12 落叶针叶林       & 17         & 1          \\
    13 常绿阔叶林       & 35         & 1          \\
    14 常绿针叶林       & 17         & 1          \\
    15 混合森林        & 20         & 1          \\
    16 内陆水体        & 0.5        & 0          \\
    17 草本湿地        & 0.5        & 0          \\
    18 森林湿地        & 17         & 1          \\
    19 贫瘠稀疏植被      & 0.5        & 0          \\
    20 草本苔原        & 0.5        & 0          \\
    21 森林苔原        & 0.5        & 0          \\
    22 混合苔原        & 0.5        & 0          \\
    23 裸土苔原        & 0.5        & 0          \\
    24 雪盖或冰川       & 0.5        & 0          \\ \bottomrule
    \end{tabular}
\end{table}

% Please add the following required packages to your document preamble:
% \usepackage{booktabs}
\begin{table}[]
\centering
\caption{USGS地表覆盖植被茎面积指数(当无植被茎面积指数数据读入时的默认值)。}
\label{tab:USGS地表覆盖植被茎面积指数}
\begin{tabular}{@{}lc@{}}
\toprule
USGS地表覆盖类型     & 茎面积指数 ($\rm m^2/m^2$ ) \\ \midrule
1 城市           & 0.2       \\
2 干旱农田与牧场      & 0.2       \\
3 灌溉农田与牧场      & 0.3       \\
4 干旱/灌溉混合农田与牧场 & 0.3       \\
5 农田草地过渡带      & 0.5       \\
6 农田林地过渡带      & 0.5       \\
7 草地           & 1         \\
8 灌木地          & 0.5       \\
9 草地灌木地混合带     & 1         \\
10 稀疏草原        & 0.5       \\
11 落叶阔叶林       & 2         \\
12 落叶针叶林       & 2         \\
13 常绿阔叶林       & 2         \\
14 常绿针叶林       & 2         \\
15 混合森林        & 2         \\
16 内陆水体        & 0         \\
17 草本湿地        & 0.2       \\
18 森林湿地        & 2         \\
19 贫瘠稀疏植被      & 0.2       \\
20 草本苔原        & 0.2       \\
21 森林苔原        & 0.2       \\
22 混合苔原        & 0.2       \\
23 裸土苔原        & 0         \\
24 雪盖或冰川       & 0         \\\bottomrule
\end{tabular}
\end{table}

% Please add the following required packages to your document preamble:
% \usepackage{booktabs}
\begin{table}[]
\centering
\caption{USGS地表覆盖粗糙度及零平面位移与植被高度比值(为CoLM2014默认方案,目前版本使用公式 \eqref{dOh}, \eqref{zOh} 和 \eqref{dooh})。}
\label{tab:USGS地表覆盖粗糙度及零平面位移与植被高度比值}
\begin{tabular}{@{}lcc@{}}
\toprule
USGS地表覆盖类型     & 粗糙度与植被高度比值 & 零平面位移与植被高度比值 \\ \midrule
1 城市           & 0.1        & 0.667        \\
2 干旱农田与牧场      & 0.1        & 0.667        \\
3 灌溉农田与牧场      & 0.1        & 0.667        \\
4 干旱/灌溉混合农田与牧场 & 0.1        & 0.667        \\
5 农田草地过渡带      & 0.1        & 0.667        \\
6 农田林地过渡带      & 0.1        & 0.667        \\
7 草地           & 0.1        & 0.667        \\
8 灌木地          & 0.1        & 0.667        \\
9 草地灌木地混合带     & 0.1        & 0.667        \\
10 稀疏草原        & 0.1        & 0.667        \\
11 落叶阔叶林       & 0.1        & 0.667        \\
12 落叶针叶林       & 0.1        & 0.667        \\
13 常绿阔叶林       & 0.1        & 0.667        \\
14 常绿针叶林       & 0.1        & 0.667        \\
15 混合森林        & 0.1        & 0.667        \\
16 内陆水体        & 0.1        & 0.667        \\
17 草本湿地        & 0.1        & 0.667        \\
18 森林湿地        & 0.1        & 0.667        \\
19 贫瘠稀疏植被      & 0.1        & 0.667        \\
20 草本苔原        & 0.1        & 0.667        \\
21 森林苔原        & 0.1        & 0.667        \\
22 混合苔原        & 0.1        & 0.667        \\
23 裸土苔原        & 0.1        & 0.667        \\
24 雪盖或冰川       & 0.1        & 0.667        \\ \bottomrule
\end{tabular}
\end{table}


% Please add the following required packages to your document preamble:
% \usepackage{booktabs}
\begin{sidewaystable}[]
\centering
\caption{USGS植被特征尺寸、叶倾角分布及叶片光学属性参数。$\chi_L$为叶倾角分布参数,$\rho$表示反射率,$\tau$表示透射率,下标$l$表示叶片,$s$表示茎,$vis$表示可见光波段,$nir$表示近红外波段。}
\label{tab:USGS植被特征尺寸叶倾角分布及叶片光学属性参数1}
    \begin{tabular}{@{}lcccccccccc@{}}
    \toprule
    USGS地表覆盖类型     & 叶片特征尺寸(cm) & $\chi_L$ &$\rho_{l, vis}$ & $\tau_{l, vis}$  &$\rho_{l, nir}$ &$\tau_{l, nir}$ & $\rho_{s, vis}$ &$\tau_{s, vis}$ &$\rho_{s, nir}$ &$\tau_{s,nir}$\\ \midrule
    1 城市           & 4.0        & -0.3                                                                         & 0.105                                                                                                           & 0.07                                                                                                            & 0.58                                                                                                            & 0.25                                                                                                            & 0.36                                                                                                            & 0.22                                                                                                            & 0.58                                                                                                            & 0.38                                                                                                            \\
    2 干旱农田与牧场      & 4.0        & -0.3                                                                         & 0.105                                                                                                           & 0.07                                                                                                            & 0.58                                                                                                            & 0.25                                                                                                            & 0.36                                                                                                            & 0.22                                                                                                            & 0.58                                                                                                            & 0.38                                                                                                            \\
    3 灌溉农田与牧场      & 4.0        & -0.3                                                                         & 0.105                                                                                                           & 0.07                                                                                                            & 0.58                                                                                                            & 0.25                                                                                                            & 0.36                                                                                                            & 0.22                                                                                                            & 0.58                                                                                                            & 0.38                                                                                                            \\
    4 干旱/灌溉混合农田与牧场 & 4.0        & -0.3                                                                         & 0.105                                                                                                           & 0.07                                                                                                            & 0.58                                                                                                            & 0.25                                                                                                            & 0.36                                                                                                            & 0.22                                                                                                            & 0.58                                                                                                            & 0.38                                                                                                            \\
    5 农田草地过渡带      & 4.0        & -0.3                                                                         & 0.105                                                                                                           & 0.07                                                                                                            & 0.58                                                                                                            & 0.25                                                                                                            & 0.36                                                                                                            & 0.22                                                                                                            & 0.58                                                                                                            & 0.38                                                                                                            \\
    6 农田林地过渡带      & 4.0        & -0.3                                                                         & 0.105                                                                                                           & 0.07                                                                                                            & 0.58                                                                                                            & 0.25                                                                                                            & 0.36                                                                                                            & 0.22                                                                                                            & 0.58                                                                                                            & 0.38                                                                                                            \\
    7 草地           & 4.0        & -0.3                                                                         & 0.105                                                                                                           & 0.07                                                                                                            & 0.58                                                                                                            & 0.25                                                                                                            & 0.36                                                                                                            & 0.22                                                                                                            & 0.58                                                                                                            & 0.38                                                                                                            \\
    8 灌木地          & 4.0        & 0.01                                                                         & 0.1                                                                                                             & 0.07                                                                                                            & 0.45                                                                                                            & 0.25                                                                                                            & 0.16                                                                                                            & 0.001                                                                                                           & 0.39                                                                                                            & 0.001                                                                                                           \\
    9 草地灌木地混合带     & 4.0        & 0.01                                                                         & 0.1                                                                                                             & 0.07                                                                                                            & 0.45                                                                                                            & 0.25                                                                                                            & 0.16                                                                                                            & 0.001                                                                                                           & 0.39                                                                                                            & 0.001                                                                                                           \\
    10 稀疏草原        & 4.0        & -0.3                                                                         & 0.105                                                                                                           & 0.07                                                                                                            & 0.58                                                                                                            & 0.25                                                                                                            & 0.36                                                                                                            & 0.22                                                                                                            & 0.58                                                                                                            & 0.38                                                                                                            \\
    11 落叶阔叶林       & 4.0        & 0.25                                                                         & 0.1                                                                                                             & 0.05                                                                                                            & 0.45                                                                                                            & 0.25                                                                                                            & 0.16                                                                                                            & 0.001                                                                                                           & 0.39                                                                                                            & 0.001                                                                                                           \\
    12 落叶针叶林       & 4.0        & 0.01                                                                         & 0.07                                                                                                            & 0.05                                                                                                            & 0.35                                                                                                            & 0.1                                                                                                             & 0.16                                                                                                            & 0.001                                                                                                           & 0.39                                                                                                            & 0.001                                                                                                           \\ %\bottomrule
%    \end{tabular}
%\end{sidewaystable}
%
%% Please add the following required packages to your document preamble:
%% \usepackage{booktabs}
%\begin{sidewaystable}[]
%    \centering
%    \caption{USGS植被特征尺寸、叶倾角分布及叶片光学属性参数 (续)。$\chi_L$为叶倾角分布参数,$\rho$表示反射率,$\tau$表示透射率,下标$l$表示叶片,$s$表示茎,$vis$表示可见光波段,$nir$表示近红外波段。}
%    \label{tab:USGS植被特征尺寸叶倾角分布及叶片光学属性参数2}
%        \begin{tabular}{@{}lcccccccccc@{}}
%        \toprule
%        USGS地表覆盖类型     & 叶片特征尺寸(cm) & $\chi_L$ &$\rho_{l, vis}$ & $\tau_{l, vis}$  &$\rho_{l, nir}$ &$\tau_{l, nir}$ & $\rho_{s, vis}$ &$\tau_{s, vis}$ &$\rho_{s, nir}$ &$\tau_{s, nir}$\\ \midrule
        13 常绿阔叶林   & 4.0        & 0.1                                                                          & 0.1                                                                                                             & 0.05                                                                                                            & 0.45                                                                                                            & 0.25                                                                                                            & 0.16                                                                                                            & 0.001                                                                                                           & 0.39                                                                                                            & 0.001                                                                                                           \\
        14 常绿针叶林   & 4.0        & 0.01                                                                         & 0.07                                                                                                            & 0.05                                                                                                            & 0.35                                                                                                            & 0.1                                                                                                             & 0.16                                                                                                            & 0.001                                                                                                           & 0.39                                                                                                            & 0.001                                                                                                           \\
        15 混合森林    & 4.0        & 0.125                                                                        & 0.07                                                                                                            & 0.05                                                                                                            & 0.4                                                                                                             & 0.15                                                                                                            & 0.16                                                                                                            & 0.001                                                                                                           & 0.39                                                                                                            & 0.001                                                                                                           \\
        16 内陆水体    & 4.0        & -0.3                                                                         & 0.105                                                                                                           & 0.07                                                                                                            & 0.58                                                                                                            & 0.25                                                                                                            & 0.36                                                                                                            & 0.22                                                                                                            & 0.58                                                                                                            & 0.38                                                                                                            \\
        17 草本湿地    & 4.0        & -0.3                                                                         & 0.1                                                                                                             & 0.07                                                                                                            & 0.58                                                                                                            & 0.25                                                                                                            & 0.36                                                                                                            & 0.22                                                                                                            & 0.58                                                                                                            & 0.38                                                                                                            \\
        18 森林湿地    & 4.0        & 0.1                                                                          & 0.105                                                                                                           & 0.05                                                                                                            & 0.45                                                                                                            & 0.25                                                                                                            & 0.16                                                                                                            & 0.001                                                                                                           & 0.39                                                                                                            & 0.001                                                                                                           \\
        19 贫瘠稀疏植被  & 4.0        & 0.01                                                                         & 0.1                                                                                                             & 0.07                                                                                                            & 0.45                                                                                                            & 0.25                                                                                                            & 0.16                                                                                                            & 0.001                                                                                                           & 0.39                                                                                                            & 0.001                                                                                                           \\
        20 草本苔原    & 4.0        & -0.3                                                                         & 0.07                                                                                                            & 0.07                                                                                                            & 0.58                                                                                                            & 0.25                                                                                                            & 0.36                                                                                                            & 0.22                                                                                                            & 0.58                                                                                                            & 0.38                                                                                                            \\
        21 森林苔原    & 4.0        & -0.3                                                                         & 0.1                                                                                                             & 0.07                                                                                                            & 0.58                                                                                                            & 0.25                                                                                                            & 0.36                                                                                                            & 0.22                                                                                                            & 0.58                                                                                                            & 0.38                                                                                                            \\
        22 混合苔原    & 4.0        & -0.3                                                                         & 0.07                                                                                                            & 0.07                                                                                                            & 0.58                                                                                                            & 0.25                                                                                                            & 0.36                                                                                                            & 0.22                                                                                                            & 0.58                                                                                                            & 0.38                                                                                                            \\
        23 裸土苔原    & 4.0        & -0.3                                                                         & 0.07                                                                                                            & 0.07                                                                                                            & 0.58                                                                                                            & 0.25                                                                                                            & 0.36                                                                                                            & 0.22                                                                                                            & 0.58                                                                                                            & 0.38                                                                                                            \\
        24 雪盖或冰川   & 4.0        & -0.3                                                                         & 0.105                                                                                                           & 0.07                                                                                                            & 0.58                                                                                                            & 0.25                                                                                                            & 0.36                                                                                                            & 0.22                                                                                                            & 0.58                                                                                                            & 0.38                                                                                                            \\\bottomrule
                \end{tabular}
\end{sidewaystable}



% Please add the following required packages to your document preamble:
% \usepackage{booktabs}
\begin{sidewaystable}[]
    \centering
    \caption{USGS植被光合作用参数。$V_{cmax}$表示植被冠层顶部 25~\textcelsius 时光合最大羧化速率($\rm mol\ m^{-2}\ s^{-1}$),$\alpha$为量子转化效率(0.05 $\rm mol\ CO_2 mol^{-1}$ photon),$m$为气孔导度经验拟合经验参数(无量纲),$b$为最小气孔导度($\rm mol\ CO_2\ m^{-2}s^{-1}$) ,
    $r_{base}$为叶基础呼吸速率系数(无量纲),$s_1$是高温抑制系数($\rm K^{-1}$),$s_2$是低温抑制系数($\rm K^{-1}$),$s_3$是叶呼吸高温抑制系数($\rm K^{-1}$),$T_{dm}$是叶呼吸高温抑制温度参数(K)。}
    \label{tab:USGS植被光合作用参数1}
    \begin{tabular}{@{}lccccccccccccccccccc@{}}
    \toprule
    USGS地表覆盖类型     &$ V_{cmax}$ & $\alpha$ & $m$& $b$ & $r_{base}$ & $s_1$ & $s_2$ & $s_3$ & $T_{dm}$ & $T_{op}$ & $T_{low}$ & $T_{high}$ & $K_n$  \\ \midrule
    1 城市           & 100                                                               & 0.08                                                                                                   & 9                                                                                  & 0.01                                                                               & 0.015                                                               & 0.3                                                       & 0.2                                                       & 1.3                                                       & 328                                                             & 298                                                             & 308                                                              & 281                                                               & 0.5                                                          \\
    2 干旱农田与牧场      & 57                                                                & 0.08                                                                                                   & 9                                                                                  & 0.01                                                                               & 0.015                                                               & 0.3                                                       & 0.2                                                       & 1.3                                                       & 328                                                             & 298                                                             & 308                                                              & 281                                                               & 0.5                                                          \\
    3 灌溉农田与牧场      & 57                                                                & 0.08                                                                                                   & 9                                                                                  & 0.01                                                                               & 0.015                                                               & 0.3                                                       & 0.2                                                       & 1.3                                                       & 328                                                             & 298                                                             & 308                                                              & 281                                                               & 0.5                                                          \\
    4 干旱/灌溉混合农田与牧场 & 57                                                                & 0.08                                                                                                   & 9                                                                                  & 0.01                                                                               & 0.015                                                               & 0.3                                                       & 0.2                                                       & 1.3                                                       & 328                                                             & 298                                                             & 308                                                              & 281                                                               & 0.5                                                          \\
    5 农田草地过渡带      & 52                                                                & 0.08                                                                                                   & 9                                                                                  & 0.01                                                                               & 0.015                                                               & 0.3                                                       & 0.2                                                       & 1.3                                                       & 328                                                             & 298                                                             & 308                                                              & 281                                                               & 0.5                                                          \\
    6 农田林地过渡带      & 52                                                                & 0.08                                                                                                   & 9                                                                                  & 0.01                                                                               & 0.015                                                               & 0.3                                                       & 0.2                                                       & 1.3                                                       & 328                                                             & 298                                                             & 308                                                              & 281                                                               & 0.5                                                          \\
    7 草地           & 52                                                                & 0.08                                                                                                   & 9                                                                                  & 0.01                                                                               & 0.015                                                               & 0.3                                                       & 0.2                                                       & 1.3                                                       & 328                                                             & 298                                                             & 308                                                              & 281                                                               & 0.5                                                          \\
    8 灌木地          & 52                                                                & 0.08                                                                                                   & 9                                                                                  & 0.01                                                                               & 0.015                                                               & 0.3                                                       & 0.2                                                       & 1.3                                                       & 328                                                             & 298                                                             & 313                                                              & 283                                                               & 0.5                                                          \\
    9 草地灌木地混合带     & 52                                                                & 0.08                                                                                                   & 9                                                                                  & 0.01                                                                               & 0.015                                                               & 0.3                                                       & 0.2                                                       & 1.3                                                       & 328                                                             & 298                                                             & 313                                                              & 283                                                               & 0.5                                                          \\
    10 稀疏草原        & 52                                                                & 0.08                                                                                                   & 9                                                                                  & 0.01                                                                               & 0.015                                                               & 0.3                                                       & 0.2                                                       & 1.3                                                       & 328                                                             & 298                                                             & 308                                                              & 281                                                               & 0.5                                                          \\
    11 落叶阔叶林       & 52                                                                & 0.08                                                                                                   & 9                                                                                  & 0.01                                                                               & 0.015                                                               & 0.3                                                       & 0.2                                                       & 1.3                                                       & 328                                                             & 298                                                             & 311                                                              & 283                                                               & 0.5                                                          \\
    12 落叶针叶林       & 57                                                                & 0.08                                                                                                   & 9                                                                                  & 0.01                                                                               & 0.015                                                               & 0.3                                                       & 0.2                                                       & 1.3                                                       & 328                                                             & 298                                                             & 303                                                              & 278                                                               & 0.5                                                          \\ %\bottomrule
%    \end{tabular}
%\end{sidewaystable}
%
%% Please add the following required packages to your document preamble:
%% \usepackage{booktabs}
%\begin{sidewaystable}[]
%    \centering
%    \caption{USGS植被光合作用参数 (续)。$V_{cmax}$表示植被冠层顶部 25\textcelsius 时光合最大羧化速率($\rm mol\ m^{-2}\ s^{-1}$),$\alpha$为量子转化效率(0.05 $\rm mol\ CO_2\ mol^{-1}$ photon),$m$为气孔导度经验拟合经验参数(无量纲),$b$为最小气孔导度($\rm mol\ CO_2\ m^{-2}s^{-1}$) ,
%    $r_{base}$为叶基础呼吸速率系数(unitless),$s_1$是高温抑制系数($\rm K^{-1}$),$s_2$是低温抑制系数($\rm K^{-1}$),$s_3$是叶呼吸高温抑制系数($\rm K^{-1}$)和$T_{dm}$是叶呼吸高温抑制温度参数(K)。}
%    \label{tab:USGS植被光合作用参数2}
%    \begin{tabular}{@{}lccccccccccccccccccc@{}}
%    \toprule
%    USGS地表覆盖类型     &$ V_{cmax}$ & $\alpha$ & $m$& $b$ & $r_{base}$ & $s_1$ & $s_2$ & $s_3$ & $T_dm$ & $T_{op}$ & $T_{low}$ & $T_{high}$ & $K_n$  \\ \midrule
    13 常绿阔叶林   & 72                                                                & 0.08                                                                                                   & 9                                                                                  & 0.01                                                                               & 0.015                                                               & 0.3                                                       & 0.2                                                       & 1.3                                                       & 328                                                             & 298                                                             & 313                                                              & 288                                                               & 0.5                                                          \\
    14 常绿针叶林   & 54                                                                & 0.08                                                                                                   & 9                                                                                  & 0.01                                                                               & 0.015                                                               & 0.3                                                       & 0.2                                                       & 1.3                                                       & 328                                                             & 298                                                             & 303                                                              & 278                                                               & 0.5                                                          \\
    15 混合森林    & 52                                                                & 0.08                                                                                                   & 9                                                                                  & 0.01                                                                               & 0.015                                                               & 0.3                                                       & 0.2                                                       & 1.3                                                       & 328                                                             & 298                                                             & 307                                                              & 281                                                               & 0.5                                                          \\
    16 内陆水体    & 57                                                                & 0.08                                                                                                   & 9                                                                                  & 0.01                                                                               & 0.015                                                               & 0.3                                                       & 0.2                                                       & 1.3                                                       & 328                                                             & 298                                                             & 308                                                              & 281                                                               & 0.5                                                          \\
    17 草本湿地    & 52                                                                & 0.08                                                                                                   & 9                                                                                  & 0.01                                                                               & 0.015                                                               & 0.3                                                       & 0.2                                                       & 1.3                                                       & 328                                                             & 298                                                             & 308                                                              & 281                                                               & 0.5                                                          \\
    18 森林湿地    & 52                                                                & 0.08                                                                                                   & 9                                                                                  & 0.01                                                                               & 0.015                                                               & 0.3                                                       & 0.2                                                       & 1.3                                                       & 328                                                             & 298                                                             & 313                                                              & 288                                                               & 0.5                                                          \\
    19 贫瘠稀疏植被  & 52                                                                & 0.08                                                                                                   & 9                                                                                  & 0.01                                                                               & 0.015                                                               & 0.3                                                       & 0.2                                                       & 1.3                                                       & 328                                                             & 298                                                             & 313                                                              & 283                                                               & 0.5                                                          \\
    20 草本苔原    & 52                                                                & 0.05                                                                                                   & 4                                                                                  & 0.04                                                                               & 0.025                                                               & 0.3                                                       & 0.2                                                       & 1.3                                                       & 328                                                             & 298                                                             & 313                                                              & 288                                                               & 0.5                                                          \\
    21 森林苔原    & 52                                                                & 0.05                                                                                                   & 4                                                                                  & 0.04                                                                               & 0.025                                                               & 0.3                                                       & 0.2                                                       & 1.3                                                       & 328                                                             & 298                                                             & 313                                                              & 288                                                               & 0.5                                                          \\
    22 混合苔原    & 52                                                                & 0.05                                                                                                   & 4                                                                                  & 0.04                                                                               & 0.025                                                               & 0.3                                                       & 0.2                                                       & 1.3                                                       & 328                                                             & 298                                                             & 313                                                              & 288                                                               & 0.5                                                          \\
    23 裸土苔原    & 52                                                                & 0.05                                                                                                   & 4                                                                                  & 0.04                                                                               & 0.025                                                               & 0.3                                                       & 0.2                                                       & 1.3                                                       & 328                                                             & 298                                                             & 313                                                              & 288                                                               & 0.5                                                          \\
    24 雪盖或冰川   & 52                                                                & 0.05                                                                                                   & 4                                                                                  & 0.04                                                                               & 0.025                                                               & 0.3                                                       & 0.2                                                       & 1.3                                                       & 328                                                             & 298                                                             & 308                                                              & 281                                                               & 0.5                                                          \\\bottomrule
    \end{tabular}
\end{sidewaystable}


% Please add the following required packages to your document preamble:
% \usepackage{booktabs}
\begin{table}[]
    \centering
    \caption{USGS (\citet{schenk2002rooting}方案,默认方案)。d50表示达到50\%根分布时土壤深度(cm),$\beta$为计算根分布时经验系数。}
    \label{tab:USGSSchenkANDJackson2002方案默认方案}
    \begin{tabular}{@{}lcc@{}}
    \toprule
    USGS地表覆盖类型     & d50  & $\beta$ \\ \midrule
    1 城市           & 23   & -1.757  \\
    2 干旱农田与牧场      & 21   & -1.835  \\
    3 灌溉农田与牧场      & 23   & -1.757  \\
    4 干旱/灌溉混合农田与牧场 & 22   & -1.796  \\
    5 农田草地过渡带      & 15.7 & -1.577  \\
    6 农田林地过渡带      & 19   & -1.738  \\
    7 草地           & 9.3  & -1.359  \\
    8 灌木地          & 47   & -3.245  \\
    9 草地灌木地混合带     & 28.2 & -2.302  \\
    10 稀疏草原        & 21.7 & -1.654  \\
    11 落叶阔叶林       & 16   & -1.681  \\
    12 落叶针叶林       & 16   & -1.681  \\
    13 常绿阔叶林       & 15   & -1.632  \\
    14 常绿针叶林       & 15   & -1.632  \\
    15 混合森林        & 15.5 & -1.656  \\
    16 内陆水体        & 1    & -1      \\
    17 草本湿地        & 9.3  & -1.359  \\
    18 森林湿地        & 15.5 & -1.656  \\
    19 贫瘠稀疏植被      & 27   & -2.051  \\
    20 草本苔原        & 9    & -2.621  \\
    21 森林苔原        & 9    & -2.621  \\
    22 混合苔原        & 9    & -2.621  \\
    23 裸土苔原        & 9    & -2.621  \\
    24 雪盖或冰川       & 1    & -1      \\ \bottomrule
\end{tabular}
\end{table}


% Please add the following required packages to your document preamble:
% \usepackage{booktabs}
\begin{table}[]
\centering
\caption{USGS植被根分布参数 (\citet{zeng2001global}方案),$a$、$b$为用于计算根分布经验系数。}
\label{tab:USGS植被根分布参数}
\begin{tabular}{@{}lcc@{}}
\toprule
USGS地表覆盖类型     & $a$ & $b$ \\ \midrule
1 城市           & 5.558      & 2.614      \\
2 干旱农田与牧场      & 5.558      & 2.614      \\
3 灌溉农田与牧场      & 5.558      & 2.614      \\
4 干旱/灌溉混合农田与牧场 & 5.558      & 2.614      \\
5 农田草地过渡带      & 8.149      & 2.611      \\
6 农田林地过渡带      & 5.558      & 2.614      \\
7 草地           & 10.74      & 2.608      \\
8 灌木地          & 7.022      & 1.415      \\
9 草地灌木地混合带     & 8.881      & 2.012      \\
10 稀疏草原        & 7.92       & 1.964      \\
11 落叶阔叶林       & 5.99       & 1.955      \\
12 落叶针叶林       & 7.066      & 1.953      \\
13 常绿阔叶林       & 7.344      & 1.303      \\
14 常绿针叶林       & 7.706      & 2.175      \\
15 混合森林        & 4.453      & 1.631      \\
16 内陆水体        & 10.74      & 2.608      \\
17 草本湿地        & 10.74      & 2.608      \\
18 森林湿地        & 4.453      & 1.631      \\
19 贫瘠稀疏植被      & 8.992      & 8.992      \\
20 草本苔原        & 8.992      & 8.992      \\
21 森林苔原        & 8.992      & 8.992      \\
22 混合苔原        & 8.992      & 8.992      \\
23 裸土苔原        & 4.372      & 0.978      \\
24 雪盖或冰川       & 10.74      & 2.608      \\ \bottomrule
\end{tabular}
\end{table}


\chapter{IGBP地表覆盖类型相关参数}\label{IGBP地表覆盖类型相关参数}

% Please add the following required packages to your document preamble:
% \usepackage{booktabs}
\begin{table}[]
    \centering
    \caption{IGBP地表覆盖植被顶部高度和底部高度值 (当无植被高度数据读入时的默认高度)。}
    \label{tab:IGBP地表覆盖植被顶部高度和底部高度值}
    \begin{tabular}{@{}lcc@{}}
    \toprule
    IGBP地表覆盖类型    & \text{植被顶部高度 (m)} & \text{植被底部高度 (m)} \\ \midrule
    0 水体          & 17                  & 8.5                 \\
    1 常绿针叶林       & 35                  & 1                   \\
    2 常绿阔叶林       & 17                  & 8.5                 \\
    3 落叶针叶林       & 20                  & 11.5                \\
    4 落叶阔叶林       & 20                  & 10                  \\
    5 混合林         & 0.5                 & 0.1                 \\
    6 郁闭灌丛        & 0.5                 & 0.1                 \\
    7 稀疏灌丛        & 1                   & 0.1                 \\
    8 稀疏大草原(木本为主) & 1                   & 0.1                 \\
    9 稀疏大草原       & 1                   & 0.01                \\
    10草地          & 1                   & 0.01                \\
    11 永久性湿地      & 1                   & 0.01                \\
    12 耕地         & 1                   & 0.3                 \\
    13 城市         & 1                   & 0.01                \\
    14 耕地和自然植被混合带 & 1                   & 0.01                \\
    15 积雪和冰川      & 1                   & 0.01                \\
    16 裸土或稀疏植被覆盖  & 1                   & 0.01               \\\bottomrule
    \end{tabular}
\end{table}


% Please add the following required packages to your document preamble:
% \usepackage{booktabs}
\begin{table}[]
    \centering
    \caption{IGBP地表覆盖植被茎面积指数 (当无植被茎面积指数数据读入时的默认值)。}
    \label{tab:IGBP地表覆盖植被茎面积指数}
\begin{tabular}{@{}lcc@{}}
\toprule
IGBP地表覆盖类型    & 茎面积指数 (m) & 植被底部高度 (m) \\ \midrule
0 水体          & 2                   & 8.5                 \\
1 常绿针叶林       & 2                   & 1                   \\
2 常绿阔叶林       & 2                   & 8.5                 \\
3 落叶针叶林       & 2                   & 11.5                \\
4 落叶阔叶林       & 2                   & 10                  \\
5 混合林         & 0.5                 & 0.1                 \\
6 郁闭灌丛        & 0.5                 & 0.1                 \\
7 稀疏灌丛        & 0.5                 & 0.1                 \\
8 稀疏大草原(木本为主) & 0.5                 & 0.1                 \\
9 稀疏大草原       & 0.2                 & 0.01                \\
10草地          & 0.2                 & 0.01                \\
11 永久性湿地      & 0.2                 & 0.01                \\
12 耕地         & 0.2                 & 0.3                 \\
13 城市         & 0.2                 & 0.01                \\
14 耕地和自然植被混合带 & 0                   & 0.01                \\
15 积雪和冰川      & 0                   & 0.01                \\
16 裸土或稀疏植被覆盖  & 0                   & 0.01                \\ \bottomrule
\end{tabular}
\end{table}

% Please add the following required packages to your document preamble:
% \usepackage{booktabs}
\begin{table}[]
\centering
\caption{IGBP地表覆盖粗糙度及零平面位移与植被高度比值(为 CoLM2014 默认方案,目前版本使用公式 \eqref{dOh}, \eqref{zOh} 和 \eqref{dooh})。}
\label{tab:IGBP地表覆盖粗糙度及零平面位移与植被高度比值}
\begin{tabular}{@{}lcc@{}}
\toprule
IGBP地表覆盖类型    & \text{粗糙度与植被高度比值} & \text{零平面位移与植被高度比值} \\ \midrule
0 水体          & 0.1                 & 0.667                 \\
1 常绿针叶林       & 0.1                 & 0.667                 \\
2 常绿阔叶林       & 0.1                 & 0.667                 \\
3 落叶针叶林       & 0.1                 & 0.667                 \\
4 落叶阔叶林       & 0.1                 & 0.667                 \\
5 混合林         & 0.1                 & 0.667                 \\
6 郁闭灌丛        & 0.1                 & 0.667                 \\
7 稀疏灌丛        & 0.1                 & 0.667                 \\
8 稀疏大草原(木本为主) & 0.1                 & 0.667                 \\
9 稀疏大草原       & 0.1                 & 0.667                 \\
10草地          & 0.1                 & 0.667                 \\
11 永久性湿地      & 0.1                 & 0.667                 \\
12 耕地         & 0.1                 & 0.667                 \\
13 城市         & 0.1                 & 0.667                 \\
14 耕地和自然植被混合带 & 0.1                 & 0.667                 \\
15 积雪和冰川      & 0.1                 & 0.667                 \\
16 裸土或稀疏植被覆盖  & 0.1                 & 0.667                 \\ \bottomrule
\end{tabular}
\end{table}


% Please add the following required packages to your document preamble:
% \usepackage{booktabs}
\begin{sidewaystable}[]
    \centering
    \caption{IGBP植被特征尺寸、叶倾角分布及叶片光学属性参数。$\chi_L$为叶倾角分布参数,$\rho$表示反射率,$\tau$表示透射率,下标$l$表示叶片,$s$表示茎,$vis$表示可见光波段,$nir$表示近红外波段。}
    \label{tab:IGBP植被特征尺寸叶倾角分布及叶片光学属性参数1}
        \begin{tabular}{@{}lcccccccccc@{}}
        \toprule
        IGBP地表覆盖类型     & 叶片特征尺寸(cm) & $\chi_L$ &$\rho_{l, vis}$ & $\tau_{l, vis}$  &$\rho_{l, nir}$ &$\tau_{l, nir}$ & $\rho_{s, vis}$ &$\tau_{s, vis}$ &$\rho_{s, nir}$ &$\tau_{s, nir}$\\ \midrule
        0 水体          & 4.0        & 0.01  & 0.07  & 0.05 & 0.35 & 0.1  & 0.16 & 0.001 & 0.39 & 0.001 \\
        1 常绿针叶林       & 4.0        & 0.1   & 0.1   & 0.05 & 0.45 & 0.25 & 0.16 & 0.001 & 0.39 & 0.001 \\
        2 常绿阔叶林       & 4.0        & 0.01  & 0.07  & 0.05 & 0.35 & 0.1  & 0.16 & 0.001 & 0.39 & 0.001 \\
        3 落叶针叶林       & 4.0        & 0.25  & 0.1   & 0.05 & 0.45 & 0.25 & 0.16 & 0.001 & 0.39 & 0.001 \\
        4 落叶阔叶林       & 4.0        & 0.125 & 0.07  & 0.05 & 0.4  & 0.15 & 0.16 & 0.001 & 0.39 & 0.001 \\
        5 混合林         & 4.0        & 0.01  & 0.105 & 0.05 & 0.45 & 0.25 & 0.16 & 0.001 & 0.39 & 0.001 \\
        6 郁闭灌丛        & 4.0        & 0.01  & 0.105 & 0.05 & 0.45 & 0.25 & 0.16 & 0.001 & 0.39 & 0.001 \\
        7 稀疏灌丛        & 4.0        & 0.01  & 0.105 & 0.05 & 0.58 & 0.25 & 0.16 & 0.001 & 0.39 & 0.001 \\
        8 稀疏大草原(木本为主) & 4.0        & 0.01  & 0.105 & 0.05 & 0.58 & 0.25 & 0.16 & 0.001 & 0.39 & 0.001 \\
        9 稀疏大草原       & 4.0        & -0.3  & 0.105 & 0.07 & 0.58 & 0.25 & 0.36 & 0.001 & 0.58 & 0.38  \\
        10草地          & 4.0        & 0.1   & 0.105 & 0.05 & 0.45 & 0.25 & 0.16 & 0.001 & 0.39 & 0.001 \\ %\bottomrule
%        \end{tabular}
%    \end{sidewaystable}
% 
    % Please add the following required packages to your document preamble:
    % \usepackage{booktabs}
%    \begin{sidewaystable}[]
%        \centering
%        \caption{IGBP植被特征尺寸、叶倾角分布及叶片光学属性参数 (续)。$\chi_L$为叶倾角分布参数,$\rho$表示反射率,$\tau$表示透射率,下标$l$表示叶片,$s$表示茎,$vis$表示可见光波段,$nir$表示近红外波段。}
%        \label{tab:IGBP植被特征尺寸叶倾角分布及叶片光学属性参数2}
%            \begin{tabular}{@{}lcccccccccc@{}}
%            \toprule
%            USGS地表覆盖类型     & 叶片特征尺寸(cm) & $\chi_L$ &$\rho_{l, vis}$ & $\tau_{l, vis}$  &$\rho_{l, nir}$ &$\tau_{l, nir}$ & $\rho_{s, vis}$ &$\tau_{s, vis}$ &$\rho_{s, nir}$ &$\tau_{s,nir}$\\ \midrule
            11 永久性湿地      & 4.0         & -0.3         & 0.105          & 0.07          & 0.58          & 0.25          & 0.36          & 0.001          & 0.58           & 0.38 \\
            12 耕地         & 4.0          & 0.01          & 0.105          & 0.05          & 0.45          & 0.25          & 0.16          & 0.001          & 0.39          & 0.001         \\
            13 城市         & 4.0          & -0.3          & 0.105          & 0.07          & 0.58          & 0.25          & 0.36          & 0.001          & 0.58          & 0.38          \\
            14 耕地和自然植被混合带 & 4.0          & 0.01          & 0.105          & 0.05          & 0.45          & 0.25          & 0.16          & 0.001          & 0.39          & 0.001         \\
            15 积雪和冰川      & 4.0          & 0.01          & 0.105          & 0.05          & 0.45          & 0.25          & 0.16          & 0.001          & 0.39          & 0.001         \\
            16 裸土或稀疏植被覆盖  & 4.0          & 0.01          & 0.105          & 0.05          & 0.58          & 0.25          & 0.16          & 0.001          & 0.58          & 0.001         \\\bottomrule
        \end{tabular}
    \end{sidewaystable}
    

    %Please add the following required packages to your document preamble:
    % \usepackage{booktabs}
    \begin{sidewaystable}[]
        \centering
        \caption{IGBP植被光合作用参数。$V_{cmax}$表示植被冠层顶部 25~\textcelsius 时光合最大羧化速率($\rm mol\ m^{-2}\ s{-1}$),
        $\alpha$为量子转化效率(0.05 $\rm mol CO_2 mol^{-1}$ photon),$m$为气孔导度经验拟合经验参数(无量纲),
        $b$为最小气孔导度($\rm mol\ CO_2\ m^{-2}s^{-1}$) ,
        $r_{base}$为叶基础呼吸速率系数(无量纲),$s_1$是高温抑制系数($\rm K^{-1}$),$s_2$是低温抑制系数($\rm K^{-1}$),
        $s_3$是叶呼吸高温抑制系数($\rm K^{-1}$),$T_{dm}$是叶呼吸高温抑制温度参数(K)。}
        \label{tab:IGBP植被光合作用参数1}
        \begin{tabular}{@{}lccccccccccccccccccc@{}}
        \toprule
        USGS地表覆盖类型     &$ V_{cmax}$ & $\alpha$ & $m$& $b$ & $r_{base}$ & $s_1$ & $s_2$ & $s_3$ & $T_{dm}$ & $T_{op}$ & $T_{low}$ & $T_{high}$ & $K_n$  \\ \midrule
        0 水体          & 54 & 0.08 & 9 & 0.01 & 0.015 & 0.3 & 0.2 & 1.3 & 328 & 298 & 303 & 278 & 0.5 \\
        1 常绿针叶林       & 72          & 0.08          & 9          & 0.01          & 0.015          & 0.3          & 0.2          & 1.3          & 328          & 298          & 313          & 288          & 0.5          \\
        2 常绿阔叶林       & 57          & 0.08          & 9          & 0.01          & 0.015          & 0.3          & 0.2          & 1.3          & 328          & 298          & 303          & 278          & 0.5          \\
        3 落叶针叶林       & 52          & 0.08          & 9          & 0.01          & 0.015          & 0.3          & 0.2          & 1.3          & 328          & 298          & 311          & 283          & 0.5          \\
        4 落叶阔叶林       & 52          & 0.08          & 9          & 0.01          & 0.015          & 0.3          & 0.2          & 1.3          & 328          & 298          & 307          & 281          & 0.5          \\
        5 混合林         & 52          & 0.08          & 9          & 0.01          & 0.015          & 0.3          & 0.2          & 1.3          & 328          & 298          & 308          & 281          & 0.5          \\
        6 郁闭灌丛        & 52          & 0.08          & 9          & 0.01          & 0.015          & 0.3          & 0.2          & 1.3          & 328          & 298          & 313          & 288          & 0.5          \\
        7 稀疏灌丛        & 52          & 0.08          & 9          & 0.01          & 0.015          & 0.3          & 0.2          & 1.3          & 328          & 298          & 313          & 288          & 0.5          \\
        8 稀疏大草原(木本为主) & 52          & 0.08          & 9          & 0.01          & 0.015          & 0.3          & 0.2          & 1.3          & 328          & 298          & 313          & 288          & 0.5          \\
        9 稀疏大草原       & 52          & 0.08          & 9          & 0.01          & 0.015          & 0.3          & 0.2          & 1.3          & 328          & 298          & 308          & 281          & 0.5          \\
        10草地          & 52          & 0.08          & 9          & 0.01          & 0.015          & 0.3          & 0.2          & 1.3          & 328          & 298          & 313          & 283          & 0.5          \\
        11 永久性湿地      & 57          & 0.08          & 9          & 0.01          & 0.015          & 0.3          & 0.2          & 1.3          & 328          & 298          & 308          & 281          & 0.5        \\ %\bottomrule
%                \end{tabular}
%    \end{sidewaystable}
%    
    % Please add the following required packages to your document preamble:
    % \usepackage{booktabs}
%    \begin{sidewaystable}[]
%        \centering
%        \caption{IGBP植被光合作用参数 (续)。$V_{cmax}$表示植被冠层顶部25$\deg$C时光合最大羧化速率($\rm mol\ m^{-2}\ s{-1}$),
%        $\alpha$为量子转化效率(0.05 $\rm mol CO_2 mol^{-1}$ photon),$m$为气孔导度经验拟合经验参数(无量纲),
%        $b$为最小气孔导度($\rm mol\ CO_2\ m^{-2}s^{-1}$) ,
%        $r_{base}$为叶基础呼吸速率系数(unitless),$s_1$是高温抑制系数($\rm K^{-1}$),$s_2$是低温抑制系数($\rm K^{-1}$),
%        $s_3$是叶呼吸高温抑制系数($\rm K^{-1}$)和$T_{dm}$是叶呼吸高温抑制温度参数(K)。}
%        \label{tab:IGBP植被光合作用参数2}
%        \begin{tabular}{@{}lccccccccccccccccccc@{}}
%        \toprule
%        USGS地表覆盖类型     &$ V_{cmax}$ & $\alpha$ & $m$& $b$ & $r_{base}$ & $s_1$ & $s_2$ & $s_3$ & $T_dm$ & $T_{op}$ & $T_{low}$ & $T_{high}$ & $K_n$  \\ \midrule
        12 耕地         & 100 & 0.08 & 9 & 0.01 & 0.015 & 0.3 & 0.2 & 1.3 & 328 & 298 & 308 & 281 & 0.5 \\
        13 城市         & 57  & 0.08 & 9 & 0.01 & 0.015 & 0.3 & 0.2 & 1.3 & 328 & 298 & 308 & 281 & 0.5 \\
        14 耕地和自然植被混合带 & 52  & 0.08 & 9 & 0.01 & 0.015 & 0.3 & 0.2 & 1.3 & 328 & 298 & 303 & 278 & 0.5 \\
        15 积雪和冰川      & 52  & 0.08 & 9 & 0.01 & 0.015 & 0.3 & 0.2 & 1.3 & 328 & 298 & 313 & 288 & 0.5 \\
        16 裸土或稀疏植被覆盖  & 52  & 0.08 & 9 & 0.01 & 0.015 & 0.3 & 0.2 & 1.3 & 328 & 298 & 308 & 281 & 0.5 \\\bottomrule
        \end{tabular}
    \end{sidewaystable}
    
% Please add the following required packages to your document preamble:
% \usepackage{booktabs}
\begin{table}[]
\centering
\caption{IGBP植被根分布参数 (\citet{zeng2001global}方案),$a$、$b$为用于计算根分布经验系数。}
\label{tab:IGBP植被根分布参数}
\begin{tabular}{@{}lcc@{}}
\toprule
USGS地表覆盖类型     & $a$ & $b$ \\ \midrule
1 城市           & 5.558      & 2.614      \\
2 干旱农田与牧场      & 5.558      & 2.614      \\
3 灌溉农田与牧场      & 5.558      & 2.614      \\
4 干旱/灌溉混合农田与牧场 & 5.558      & 2.614      \\
5 农田草地过渡带      & 8.149      & 2.611      \\
6 农田林地过渡带      & 5.558      & 2.614      \\
7 草地           & 10.74      & 2.608      \\
8 灌木地          & 7.022      & 1.415      \\
9 草地灌木地混合带     & 8.881      & 2.012      \\
10 稀疏草原        & 7.92       & 1.964      \\
11 落叶阔叶林       & 5.99       & 1.955      \\
12 落叶针叶林       & 7.066      & 1.953      \\
13 常绿阔叶林       & 7.344      & 1.303      \\
14 常绿针叶林       & 7.706      & 2.175      \\
15 混合森林        & 4.453      & 1.631      \\
16 内陆水体        & 10.74      & 2.608      \\
17 草本湿地        & 10.74      & 2.608      \\
18 森林湿地        & 4.453      & 1.631      \\
19 贫瘠稀疏植被      & 8.992      & 8.992      \\
20 草本苔原        & 8.992      & 8.992      \\
21 森林苔原        & 8.992      & 8.992      \\
22 混合苔原        & 8.992      & 8.992      \\
23 裸土苔原        & 4.372      & 0.978      \\
24 雪盖或冰川       & 10.74      & 2.608      \\ \bottomrule
\end{tabular}
\end{table}

% Please add the following required packages to your document preamble:
% \usepackage{booktabs}
\begin{table}[]
    \centering
    \caption{IGBP (\citet{schenk2002rooting}方案,默认方案)。d50表示达到50\%根分布时土壤深度(cm),$\beta$为计算根分布时经验系数。}
    \label{tab:IGBPSchenkANDJackson2002方案默认方案}
    \begin{tabular}{@{}lcc@{}}
    \toprule
    IGBP地表覆盖类型     & d50  & $\beta$ \\ \midrule
    0 水体          & 15   & -1.623 \\  
    1 常绿针叶林       & 15   & -1.623 \\
    2 常绿阔叶林       & 16   & -1.681 \\
    3 落叶针叶林       & 16   & -1.681 \\
    4 落叶阔叶林       & 15.5 & -1.652 \\
    5 混合林         & 19   & -1.336 \\
    6 郁闭灌丛        & 28   & -1.909 \\
    7 稀疏灌丛        & 18.5 & -1.582 \\
    8 稀疏大草原(木本为主) & 28   & -1.798 \\
    9 稀疏大草原       & 9    & -1.359 \\
    10草地          & 9    & -1.359 \\
    11 永久性湿地      & 22   & -1.796 \\
    12 耕地         & 23   & -1.757 \\
    13 城市         & 22   & -1.796 \\
    14 耕地和自然植被混合带 & 1    & -1     \\
    15 积雪和冰川      & 9    & -2.261 \\
    16 裸土或稀疏植被覆盖  & 1    & -1     \\ \bottomrule
\end{tabular}
\end{table}


% Please add the following required packages to your document preamble:
% \usepackage{booktabs}
\begin{table}[]
\centering
\caption{IGBP植被根分布参数 (\citet{zeng2001global}方案),$a$、$b$为用于计算根分布经验系数。}
\label{tab:IGBP植被根分布参数zeng方案}
\begin{tabular}{@{}lcc@{}}
\toprule
IGBP地表覆盖类型     & $a$ & $b$ \\ \midrule
0 水体            & 6.706           & 2.175 \\ \midrule
1 常绿针叶林       & 7.344          & 1.303          \\
2 常绿阔叶林       & 7.066          & 1.953          \\
3 落叶针叶林       & 5.99           & 1.955          \\
4 落叶阔叶林       & 4.453          & 1.631          \\
5 混合林         & 6.326          & 1.567          \\
6 郁闭灌丛        & 7.718          & 1.262          \\
7 稀疏灌丛        & 7.604          & 2.3            \\
8 稀疏大草原(木本为主) & 8.235          & 1.627          \\
9 稀疏大草原       & 10.74          & 2.608          \\
10草地          & 10.74          & 2.608          \\
11 永久性湿地      & 5.558          & 2.614          \\
12 耕地         & 5.558          & 2.614          \\
13 城市         & 5.558          & 2.614          \\
14 耕地和自然植被混合带 & 10.74          & 2.608          \\
15 积雪和冰川      & 4.372          & 0.978          \\
16 裸土或稀疏植被覆盖  & 10.74          & 2.608          \\ \bottomrule
\end{tabular}
\end{table}


\chapter{植被功能型(PFT)相关参数}\label{植被功能型PFT相关参数}

% Please add the following required packages to your document preamble:
% \usepackage{booktabs}
\begin{table}[]
    \centering
    \caption{PFT分类植被顶部高度和底部高度值(当无植被高度数据读入时的默认高度)。}
    \label{tab:PFT分类植被顶部高度和底部高度值}
    \begin{tabular}{@{}lcc@{}}
    \toprule
    PFT地表覆盖类型     & 植被顶部高度 (m) & 植被底部高度 (m) \\ \midrule
 % Please add the following required packages to your document preamble:
% \usepackage{booktabs}
    0 裸土           & 0.5          & 0          \\ \midrule
    1 温带常绿针叶树   & 17           & 1          \\
    2 北方常绿针叶树   & 17           & 1          \\
    3 北方落叶针叶树   & 14           & 1          \\
    4 热带常绿阔叶树   & 35           & 1          \\
    5 温带常绿阔叶树   & 35           & 1          \\
    6 热带落叶阔叶树   & 18           & 1          \\
    7 温带落叶阔叶林   & 20           & 1          \\
    8 北方落叶阔叶林   & 20           & 1          \\
    9 常绿阔叶灌木    & 0.5          & 0           \\
    10 温带落叶阔叶灌木 & 0.5          & 0          \\
    11 北方落叶阔叶灌木 & 0.5          & 0          \\
    12 极地C3草    & 0.5          & 0          \\
    13 C3草      & 0.5          & 0          \\
    14 C4草      & 0.5          & 0          \\
    15 C3作物     & 0.5          & 0          \\ \bottomrule

    \end{tabular}
\end{table}

% Please add the following required packages to your document preamble:
% \usepackage{booktabs}
\begin{table}[]
    \centering
    \caption{PFT茎面积指数 (当无植被茎面积指数数据读入时的默认值)。}
    \label{tab:PFT茎面积指数}
    \begin{tabular}{@{}lc@{}}
    \toprule
    PFT类型       & 茎面积指数 ( $\rm m^2/m^2$) \\ \midrule
    0 裸土        & 0              \\
    1 温带常绿针叶树   & 2              \\
    2 北方常绿针叶树   & 2              \\
    3 北方落叶针叶树   & 2              \\
    4 热带常绿阔叶树   & 2              \\
    5 温带常绿阔叶树   & 2              \\
    6 热带落叶阔叶树   & 2              \\
    7 温带落叶阔叶林   & 2              \\
    8 北方落叶阔叶林   & 2              \\
    9 常绿阔叶灌木    & 0.5            \\
    10 温带落叶阔叶灌木 & 0.5            \\
    11 北方落叶阔叶灌木 & 0.5            \\
    12 极地C3草    & 0.2            \\
    13 C3草      & 0.2            \\
    14 C4草      & 0.2            \\
    15 C3作物     & 0.2            \\ \bottomrule
    \end{tabular}
\end{table}


% Please add the following required packages to your document preamble:
% \usepackage{booktabs}
\begin{table}[]
    \centering
    \caption{PFT粗糙度及零平面位移与植被高度比值(为CoLM2014默认方案,目前版本使用公式\eqref{dOh}, \eqref{zOh} 和 \eqref{dooh})。}
    \label{tab:PFT粗糙度及零平面位移与植被高度比值}
    \begin{tabular}{@{}lcc@{}}
    \toprule
    PFT类型       & 粗糙度与植被高度比值 & 零平面位移与植被高度比值 \\ \midrule
    0 裸土        & 0.1        & 0.667        \\
    1 温带常绿针叶树   & 0.1        & 0.667        \\
    2 北方常绿针叶树   & 0.1        & 0.667        \\
    3 北方落叶针叶树   & 0.1        & 0.667        \\
    4 热带常绿阔叶树   & 0.1        & 0.667        \\
    5 温带常绿阔叶树   & 0.1        & 0.667        \\
    6 热带落叶阔叶树   & 0.1        & 0.667        \\
    7 温带落叶阔叶林   & 0.1        & 0.667        \\
    8 北方落叶阔叶林   & 0.1        & 0.667        \\
    9 常绿阔叶灌木    & 0.1        & 0.667        \\
    10 温带落叶阔叶灌木 & 0.1        & 0.667        \\
    11 北方落叶阔叶灌木 & 0.1        & 0.667        \\
    12 极地C3草    & 0.1        & 0.667        \\
    13 C3草      & 0.1        & 0.667        \\
    14 C4草      & 0.1        & 0.667        \\
    15 C3作物     & 0.1        & 0.667        \\ \bottomrule
    \end{tabular}
\end{table}



% Please add the following required packages to your document preamble:
% \usepackage{booktabs}
\begin{sidewaystable}[]
    \centering
    \caption{PFT植被特征尺寸、叶倾角分布及叶片光学属性参数。$\chi_L$为叶倾角分布参数,$\rho$表示反射率,$\tau$表示透射率,下标$l$表示叶片,$s$表示茎,$vis$表示可见光波段,$nir$表示近红外波段。}
    \label{tab:PFT植被特征尺寸叶倾角分布及叶片光学属性参数1}
        \begin{tabular}{@{}lcccccccccc@{}}
        \toprule
        IGBP地表覆盖类型   & 叶片特征尺寸(cm) & $\chi_L$ &$\rho_{l, vis}$ & $\tau_{l, vis}$  &$\rho_{l, nir}$ &$\tau_{l, nir}$ & $\rho_{s, vis}$ &$\tau_{s, vis}$ &$\rho_{s, nir}$ &$\tau_{s,nir}$\\ \midrule
        0 裸土           & 4.0        & -0.3 & 0.11 & 0.05 & 0.35 & 0.34 & 0.31 & 0.12  & 0.53 & 0.25  \\
        1 温带常绿针叶树   & 4.0        & 0.01 & 0.07 & 0.05 & 0.35 & 0.1  & 0.16 & 0.001 & 0.39 & 0.001 \\
        2 北方常绿针叶树   & 4.0        & 0.01 & 0.07 & 0.05 & 0.35 & 0.1  & 0.16 & 0.001 & 0.39 & 0.001 \\
        3 北方落叶针叶树   & 4.0        & 0.01 & 0.07 & 0.05 & 0.35 & 0.1  & 0.16 & 0.001 & 0.39 & 0.001 \\
        4 热带常绿阔叶树   & 4.0        & 0.1  & 0.1  & 0.05 & 0.45 & 0.25 & 0.16 & 0.001 & 0.39 & 0.001 \\
        5 温带常绿阔叶树   & 4.0        & 0.1  & 0.1  & 0.05 & 0.45 & 0.25 & 0.16 & 0.001 & 0.39 & 0.001 \\
        6 热带落叶阔叶树   & 4.0        & 0.01 & 0.1  & 0.05 & 0.45 & 0.25 & 0.16 & 0.001 & 0.39 & 0.001 \\
        7 温带落叶阔叶林   & 4.0        & 0.25 & 0.1  & 0.05 & 0.45 & 0.25 & 0.16 & 0.001 & 0.39 & 0.001 \\
        8 北方落叶阔叶林   & 4.0        & 0.25 & 0.1  & 0.05 & 0.45 & 0.25 & 0.16 & 0.001 & 0.39 & 0.001 \\
        9 常绿阔叶灌木    & 4.0        & 0.01 & 0.07 & 0.05 & 0.35 & 0.1  & 0.16 & 0.001 & 0.39 & 0.001 \\
        10 温带落叶阔叶灌木 & 4.0        & 0.25 & 0.1  & 0.05 & 0.45 & 0.25 & 0.16 & 0.001 & 0.39 & 0.001 \\
        11 北方落叶阔叶灌木 & 4.0        & 0.25 & 0.1  & 0.05 & 0.45 & 0.25 & 0.16 & 0.001 & 0.39 & 0.001 \\ %\bottomrule
%        \end{tabular}
%    \end{sidewaystable}
% 
%
%    % Please add the following required packages to your document preamble:
%    % \usepackage{booktabs}
%    \begin{sidewaystable}[]
%        \centering
%        \caption{PFT植被特征尺寸、叶倾角分布及叶片光学属性参数 (续)。$\chi_L$为叶倾角分布参数,$\rho$表示反射率,$\tau$表示透射率,下标$l$表示叶片,$s$表示茎,$vis$表示可见光波段,$nir$表示近红外波段。}
%        \label{tab:PFT植被特征尺寸叶倾角分布及叶片光学属性参数2}
%            \begin{tabular}{@{}lcccccccccc@{}}
%            \toprule
%            PFT地表覆盖类型     & 叶片特征尺寸(cm) & $\chi_L$ &$\rho_{l, vis}$ & $\tau_{l, vis}$  &$\rho_{l, nir}$ &$\tau_{l, nir}$ & $\rho_{s, vis}$ &$\tau_{s, vis}$ &$\rho_{s, nir}$ &$\tau_{s, nir}$\\ \midrule
            12 极地C3草 & 4.0 & -0.3 & 0.11 & 0.05 & 0.35 & 0.34 & 0.31 & 0.12 & 0.53 & 0.25 \\
            13 C3草   & 4.0 & -0.3 & 0.11 & 0.05 & 0.35 & 0.34 & 0.31 & 0.12 & 0.53 & 0.25   \\
            14 C4草   & 4.0 & -0.3 & 0.11 & 0.05 & 0.35 & 0.34 & 0.31 & 0.12 & 0.53 & 0.25   \\
            15 C3作物  & 4.0 & -0.3 & 0.11 & 0.05 & 0.35 & 0.34 & 0.31 & 0.12 & 0.53 & 0.25  \\\bottomrule
        \end{tabular}
    \end{sidewaystable}



    % Please add the following required packages to your document preamble:
% \usepackage{booktabs}
\begin{table}[]
    \centering
    \caption{PFT植被根分布参数(\citet{schenk2002rooting}方案,默认方案)。d50表示达到50\%根分布时土壤深度(cm),$\beta$为计算根分布时经验系数。}
    \label{tab:PFTSchenkANDJackson2002方案默认方案}
   % \begin{tabular}{@{}lll@{}}
   \begin{tabular}{@{}lcc@{}}
    \toprule
    PFT地表覆盖类型     & d50  & $\beta$ \\ \midrule
    0 裸土        & 27   & -2.051 \\
    1 温带常绿针叶树   & 21   & -1.835 \\
    2 北方常绿针叶树   & 12   & -1.88  \\
    3 北方落叶针叶树   & 12   & -1.88  \\
    4 热带常绿阔叶树   & 15   & -1.632 \\
    5 温带常绿阔叶树   & 23   & -1.757 \\
    6 热带落叶阔叶树   & 16   & -1.681 \\
    7 温带落叶阔叶林   & 23   & -1.757 \\
    8 北方落叶阔叶林   & 12   & -1.88  \\
    9 常绿阔叶灌木    & 23.5 & -1.623 \\
    10 温带落叶阔叶灌木 & 23.5 & -1.623 \\
    11 北方落叶阔叶灌木 & 23.5 & -1.623 \\
    12 极地C3草    & 9    & -2.621 \\
    13 C3草      & 7    & -1.176 \\
    14 C4草      & 16   & -1.452 \\
    15 C3作物     & 22   & -1.796 \\ \bottomrule
\end{tabular}
\end{table}


% Please add the following required packages to your document preamble:
% \usepackage{booktabs}
\begin{table}[]
\centering
\caption{PFT植被根分布参数 (\citet{zeng2001global}方案),$a$、$b$为用于计算根分布经验系数。}
\label{tab:PFT植被根分布参数}
\begin{tabular}{@{}lcc@{}}
\toprule
PFT地表覆盖类型     & $a$ & $b$ \\ \midrule
 0 裸土        & 0  & 0   \\
 1 温带常绿针叶树   & 7  & 2   \\
 2 北方常绿针叶树   & 7  & 2   \\
 3 北方落叶针叶树   & 7  & 2   \\
 4 热带常绿阔叶树   & 7  & 1   \\
 5 温带常绿阔叶树   & 7  & 1   \\
 6 热带落叶阔叶树   & 6  & 2   \\
 7 温带落叶阔叶林   & 6  & 2   \\
 8 北方落叶阔叶林   & 6  & 2   \\
 9 常绿阔叶灌木    & 7  & 1.5 \\
10 温带落叶阔叶灌木 & 7  & 1.5 \\
11 北方落叶阔叶灌木 & 7  & 1.5 \\
12 极地C3草    & 11 & 2   \\
13 C3草      & 11 & 2   \\
14 C4草      & 11 & 2   \\
15 C3作物     & 6  & 3  \\ \bottomrule
\end{tabular}
\end{table}


\chapter{植物群落(PC)次网格PFT相关参数}\label{植物群落PC次网格PFT相关参数}
% Please add the following required packages to your document preamble:
% \usepackage{booktabs}

\begin{table}[]
	\centering
	\caption{植物群落(PC)次网格叶倾角分布和叶片光学属性参数。$\chi_L$为叶倾角分布参数,$\rho$表示反射率,$\tau$表示透射率,下标$l$表示叶片,$vis$表示可见光波段,$nir$表示近红外波段。}
	\label{tab:PC叶倾角分布和叶片光学属性参数}
    \begin{tabular}{@{}lccccc@{}}
    \toprule
    PFT类型       & $\chi_L$ & $\rho_{l,vis}$ & $\tau_{l,vis}$ & $\rho_{l,nir}$ & $\tau_{l,nir}$ \\ \midrule
    0 裸土        & -0.3                   & 0.11                           & 0.05                           & 0.35                           & 0.34                            \\
    1 温带常绿针叶树   & 0.01                   & 0.07                           & 0.05                           & 0.36                           & 0.28                            \\
    2 北方常绿针叶树   & 0.01                   & 0.07                           & 0.05                           & 0.37                           & 0.29                            \\
    3 北方落叶针叶树   & 0.01                   & 0.07                           & 0.05                           & 0.36                           & 0.38                            \\
    4 热带常绿阔叶树   & 0.32                   & 0.1                            & 0.05                           & 0.45                           & 0.25                            \\
    5 温带常绿阔叶树   & 0.32                   & 0.11                           & 0.06                           & 0.46                           & 0.33                            \\
    6 热带落叶阔叶树   & 0.2                    & 0.1                            & 0.05                           & 0.45                           & 0.25                            \\
    7 温带落叶阔叶林   & 0.59                   & 0.1                            & 0.06                           & 0.42                           & 0.43                            \\
    8 北方落叶阔叶林   & 0.59                   & 0.1                            & 0.05                           & 0.4                            & 0.42                            \\
    9 常绿阔叶灌木    & 0.32                   & 0.07                           & 0.05                           & 0.45                           & 0.1                             \\
    10 温带落叶阔叶灌木 & 0.59                   & 0.1                            & 0.05                           & 0.45                           & 0.25                            \\
    11 北方落叶阔叶灌木 & 0.59                   & 0.1                            & 0.05                           & 0.45                           & 0.25                            \\
    12 极地C3草    & -0.23                  & 0.11                           & 0.05                           & 0.45                           & 0.34                            \\
    13 C3草      & -0.23                  & 0.11                           & 0.05                           & 0.45                           & 0.34                            \\
    14 C4草      & -0.23                  & 0.11                           & 0.05                           & 0.45                           & 0.34                            \\
    15 C3作物     & 0.25                   & 0.11                           & 0.05                           & 0.45                           & 0.34                            \\ \bottomrule
    \end{tabular}
    \end{table}

\bookmarksetup{startatroot}
\addtocontents{toc}{\bigskip}

\clearpage
%\renewcommand{\bibname}{参考文献}
\addtocontents{toe}{\bigskip}
%\addcontentsline{toe}{chapter}{Bibliography}
\bibliographystyle{agufull08-mod}
\bibliography{References/References}

%\clearpage
%\chapter *{致谢 (Acknowledgments)}

作者感谢自2004年以来对CoLM的发展做出重大贡献的人员。他们是:

曾庆存、Robert E. Dickinson、陈海山、梁信忠、曾旭斌、周黎明、牛国跃, …

纪多颖、张倩、朱司光、张香香,以及其他所有参与人员。
%\addcontentsline{toc}{chapter}{致谢}

\end{document}
