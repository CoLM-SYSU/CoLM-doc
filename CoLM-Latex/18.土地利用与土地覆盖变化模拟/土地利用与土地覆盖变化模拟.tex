\chapter{土地利用与土地覆盖变化模拟}\label{土地利用与土地覆盖变化模拟}
%\addcontentsline{toc}{chapter}{土地利用与土地覆盖变化模拟}

土地利用与土地覆盖变化是陆面模拟中重要强迫项之一,对模拟结果具有不可忽视的影响,特别是在局地及区域尺度。
目前CoLM模式对土地利用与土地覆盖变化进行较为简单的考虑,可以分为三种方式:
\begin{enumerate}
    \item 通过设定不同年份的地表输入数据,此为静态方案;
    \item 逐年更新,采用同类型(IGBP, PFT, PC, Urban)状态变量赋值(hot-start),新增加类型初始化(cold-start),记为同类赋值;
    \item 在2的基础上,利用地表覆盖转移矩阵,对状态变量赋值时保持物质和能量守恒,即守恒方案。
\end{enumerate}

对于静态方案,目前模式提供2001-2020年地表输入数据(以MODIS地表覆盖数据为基础制作而成),运行时指定模式地表数据年份即可,
通过以上设置生成地表数据和初始化及主程序,程序会相应读入或生成指定年的地表数据。
该方式设置简单,基本无需代码实现,适用于不同年地表数据气候态结果分析和地表覆盖变化影响机理研究。

对于同类覆盖,这里的同类包括地表覆盖类型LCT,或者PFT/PC。目前模式提供2001-2020年地表输入数据,
运行时指定模式初始运行年地表数据年份,并打开土地利用与土地覆盖宏(LULCC),即可每年动态更新地表数据。
通过以上设置运行mksrfdata和mkinidata及主程序,程序启动后相应读入模型运行年的地表数据。
地表数据的更新在每年最后一个时间步长结束后进行。该方式实现容易,简单考虑地表数据逐年替换,
适合年际/年代际或更长时间结果对比研究。

对于守恒方案,需要提供年与年之间的地表覆盖转移矩阵,运行设置同方案2。该方案较前两种物理上更为合理,
具有方案2的特点,同时满足物质和能量守恒。目前代码版本仅提供接口,
在提供地表变化转移矩阵和确定守恒变量后,代码实现并不复杂。
