\chapter{模式构架}\label{模式构架}
%\addcontentsline{toc}{chapter}{模式结构}
\section{网格构建}\label{网格构建}

通用陆面模式(CoLM,Common Land Model)首先对陆地表面进行剖分,将模拟区域按一定规则划分成单元(Element),模拟分辨率决定了单元平均面积的大小。CoLM中包含三种单元划分规则:1)经纬度网格;2)非结构网格(三角形/六边形);3)流域单元(catchment)。

经纬度网格通过规定经度和纬度分割线的位置来建立,单元的边界由东西两段经线和南北两段纬线组成。经纬度网格的分辨率通常使用经纬度分割线的间距来表达,例如,分辨率为0.5度表示相邻两条纬度分割线和经度分割线的距离均为0.5度。


包括三角形网格和六边形网格在内的非结构网格是一种无规则拓扑关系的网格。与经纬度网格相比,非结构网格的分辨率取决于指定一个或者多个目标的分布特征(例如高程、坡度、土地利用、植被类型等),在目标变化梯度较大的地区采用高分辨率,而在变化梯度较小的地区采用低分辨率。因此非结构化基于网格的多分辨率模拟保留了全局模型的整体结构,同时支持局部区域的高分辨率模拟。非结构网格的构建是基于 Delaunay 三角网等值线生成算法,首先将经纬度网格数据转化插值,生成铺盖整个球面的三角形网格数据;接着根据所选取的目标进行一次或者多次细化;再通过连接具有相同顶点的六个三角形三条边的重心生成蜂窝六边形(使用六边形网格的情况下),最后生成全球或者区域的三角形或六边形可变分辨率网格。

流域单元网格通过对河网进行分段来建立,一个单元定义为一段河道的集水区域,流域单元之间具有明确的上下游关系。流域单元网格的分辨率使用单元的面积阈值(即河道集水区域的面积阈值)来表达,自然连通的湖泊设置为独立的单元,不受面积阈值的限制。
\subsection{经纬度网格}\label{经纬度网格}

\subsection{流域单元网格}\label{流域单元网格}

\subsection{非结构网格}\label{非结构网格}

\section{次网格}\label{次网格}

通用陆面模式(CoLM,Common Land Model)计算基本结构单元(次网格)为patch (斑块),由patch可以组成按经纬度划分或按流域划分的网格,
在patch尺度计算得到的通量按其所在网格覆盖比例进行面积加权平均,作为地表通量模拟结果,与大气进行交换。
地表状态/预报变量亦是如此,在patch尺度计算得到的结果按照面积加权平均后作为模式结果输出。


Patch的类型可以分为植被(含裸土)、城市、湿地、冰川和水体五大类。其中植被和城市可根据不同类型进一步细分。


植被patch可以分为自然植被和作物。自然植被patch采用三种植被结构进行表征:地表覆盖类型-LCT(Land Cover Type)、
植被功能型-PFT (Plant Function Type) 和植物群落-PC (plant community)。以上三种方式,模式运行时只能选其中1种。LCT方案为CoLM原有方案,
把某一地表覆盖类型中的所有植被当成混合植被进行模拟,
其设置的相关参数可视为等效参数,如热带大草原,虽然可能包含树和草,但被当成一种植被看待,
因此相应的参数只有1套。PFT表征方式为目前陆面模式(如CLM和JULES)较多采用的方式,是本版CoLM新添加的方式。
PFT方案是将每一个细网格地表覆盖类型进行拆解,得到其PFT的组成种类和各自面积占比,分类聚合到模式网格。
本版CoLM采用CLM的PFT方案,模式格点中的所有PFT作为1个patch,即共享土壤水热环境,但PFT的辐射和通量等过程计算相对独立。
PC方案保留LCT方案模拟对象,即地表覆盖类型,同时也对LCT中的地表覆盖植被类型进行细分。
不同于LCT方案把某一类型地表覆盖的所有植被当成混合植被,PC方案对其所组成的PFT进行表达计算,所有PFT共享土壤水热环境,
辐射和通量计算等过程也是同时求解、相互影响,类似植物群落概念,故命名为PC (plant community)方案。


LCT方案所依赖的地表覆盖类型数据可以直接由USGS或MODIS-IGBP地表覆盖数据获取(章节 \ref{USGS地表覆盖数据} 和 \ref{IGBP地表覆盖数据}) 
。PFT和PC方案所需要的植被结构及属性数据由MODIS-IGBP地表覆盖数据加以其他辅助数据制作而成(章节 \ref{PFTPC数据及其依赖数据})。


对于作物patch,当作物模式未打开时,作物被当成一种特殊的自然植被进行模拟;当打开作物模式时,
每个模式格点根据包含的作物分类及组成比例(外部数据读取)分别建立相互独立的patch进行模拟计算。
城市patch与作物类似,当城市模式未打开时,城市被当成一种特殊的地表覆盖进行模拟;当城市模式打开时,
每个模式格点根据包含的城市类型和组成比例分别建立相互独立的patch进行模拟计算。目前城市分类有两种方案:
\begin{enumerate}
    \item 根据城市密度分为高建筑、高密度和中密度3类(基于NCAR CLMU城市模式);
    \item 根据城市局地气候区(LCZ,Local Climate Zone)分为10类。
\end{enumerate}
城市分类及其属性数据从外部数据读取(章节 \ref{城市数据})。

\section{计算框架}\label{计算框架}



\section{陆气耦合}\label{陆气耦合}
\subsection{陆面模式与大气模式的信息交换}
陆面模式的运行需要积分过程中各时间步长内的大气状态变量进行驱动。当离线运行陆面模式时(offline),大气状态可由观测数据、再分析数据或大气模式模拟结果数据直接提供;当在耦合模式中运行陆面模式时(online),大气状态可由耦合模式中的大气模式模拟结果提供,通过耦合器或程序调用接口传递给陆面模式。陆面模式接收到当前时刻的大气状态后,首先结合上一时刻的植被、雪盖或土壤的状态和地表特征(如地表反照率、空气动力学阻抗等)计算当前时刻的陆气湍流交换通量、辐射通量、进入地表的能量和水分通量等,然后根据这些通量计算当前时刻的植被、雪盖或土壤的状态变量和地表特征。在耦合模式中,这些通量和地表状态作为陆面下边界条件被返回给大气模式,供大气模式进行下一时刻大气状态的计算。陆面模式所需的大气状态变量和大气模式所需的陆面下边界条件变量分别见表~\ref{tab:陆面模式所需的大气状态变量} 和表~\ref{tab:大气模式所需的陆面模式输出变量} 。


\begin{table}[]
\centering
\caption{陆面模式所需的大气状态变量}
\label{tab:陆面模式所需的大气状态变量}
\begin{threeparttable}
\begin{tabular}{lcc}
\toprule
大气状态变量               & 变量名           & 单位           \\  \midrule
大气风速参考高度             & $z_{atm,m}$   & m            \\
大气温度参考高度             & $z_{atm,h}$   & m            \\
大气比湿参考高度             & $z_{atm,w}$   & m            \\
位于$z_{atm,m}$高度的纬向风速 & $u_{atm}$     & m/s          \\
位于$z_{atm,m}$高度的经向风速 & $v_{atm}$     & m/s          \\
位于$z_{atm,h}$高度的大气温度 & $T_{atm}$     & K            \\
位于$z_{atm,w}$高度的大气比湿 & $q_{atm}$     & kg/kg        \\
近地面气压                & $P_{atm}$     & Pa           \\
近地面下行长波辐射            & $L\downarrow$ & W/m2         \\
近地面下行短波辐射            & $S\downarrow$ & W/m2         \\
降水                   & $p$           & mm/s         \\
二氧化碳浓度               & $c_a$         & ppmv         \\
臭氧浓度                 & $c_o$         & mol/mol      \\
大气气溶胶沉降速率            & $D_{sp}$      & kg/m2/s      \\
氮沉降速率                & $N_{dep}$     & g(N)/m2/yr   \\
闪电频率                 & $I_l$         & flash/km2/hr \\ \bottomrule
         
\end{tabular}
\begin{tablenotes}
\footnotesize
\item[1] 根据气溶胶种类和亲水性,气溶胶沉降速率可按照14种不同的气溶胶给出,其中沙尘气溶胶可分为8种(视为4种不同气溶胶颗粒大小的干气溶胶或湿气溶胶),黑碳气溶胶分为3种(干亲水性气溶胶、湿亲水性气溶胶、干疏水性气溶胶),有机碳气溶胶分为3种(干亲水性气溶胶、湿亲水性气溶胶、干疏水性气溶胶)。气溶胶沉降主要用于积雪、冰盖和气溶胶辐射模型(SNICAR),影响积雪反照率和积雪内部辐射传输过程的计算。
\item[2] 氮沉降速率用于生物地球化学循环模型(BGC),表征无机氮(主要由氮氧化物NOy和氨化物NHx组成)在陆地表面的沉降通量。
\item[3] 臭氧浓度用于作物模式,闪电频率用于火灾模式。
\end{tablenotes}
\end{threeparttable}
\end{table}

\begin{table}[]
\centering
\caption{大气模式所需的陆面模式输出变量}
\label{tab:大气模式所需的陆面模式输出变量}
\begin{threeparttable}
\begin{tabular}{lcc}
\toprule
陆面模式输出变量    & 变量名                            & 单位      \\ \midrule
潜热通量        & $\lambda_vE_p$+$\lambda\ E_g$+ & W/m2    \\
感热通量        & $H_p$+$H_g$                    & W/m2    \\
水汽通量        & $E_p$+$E_g$                    & mm/s    \\
纬向动量通量      & $\tau_{p,x}$+$\tau_{g,x}$      & kg/m/s2 \\
经向动量通量      & $\tau_{p,y}$+$\tau_{g,y}$      & kg/m/s2 \\
地表出射长波辐射通量  & $L_p\uparrow$+$L_g\uparrow$    & W/m2    \\
直射光可见光波段反照率 & $\alpha_{vis,dir}$             & -       \\
直射光近红外波段反照率 & $\alpha_{nir,dir}$             & -       \\
漫射光可见光波段反照率 & $\alpha_{vis,dif}$             & -       \\
漫射光近红外波段反照率 & $\alpha_{nir,dif}$             & -       \\
地表辐射温度      & $T_{rad}$                      & K       \\
近地面2m温度     & $T_{2m}$                       & K       \\
近地面2m比湿     & $q_{2m}$                       & kg/kg   \\
近地面10m风速    & $u_{10m}$                      & m/s     \\
雪水当量        & $W_{sno}$                      & mm      \\
空气动力学阻抗     & $r_{am}$                       & s/m     \\
摩擦速度        & $u_\ast$                       & m/s     \\
净生态系统碳交换通量  &                                &         \\
$NEE$       & gC/m2/s                        &         \\
氧化亚氮浓度      & $\mathrm{N_2O}$                         & gN/m2/s\\\bottomrule
         
\end{tabular}
\begin{tablenotes}
\footnotesize
\item[1] 下标$p$和$g$分别代表网格单元内植被覆盖部分和无植被覆盖部分的计算结果
\item[2] $\lambda_v$表示蒸发潜热(J/kg),$\lambda$根据地表水份是否冻结取为蒸发潜热$\lambda_v$或升华潜热$\lambda_s$。
\end{tablenotes}
\end{threeparttable}
\end{table}

\subsection{离线运行CoLM可使用的大气驱动数据集}
CoLM支持多套大气驱动数据集的使用,包括全球格点数据和单点数据,可用数据集列表见表~\ref{tab:可用于驱动CoLM离线运行的大气驱动数据集} 。用户也可使用其他数据,只需参照任意一种目前支持的数据集制作成对应的数据格式,并参照该数据对应的namelist提供相应的信息即可使用。特别地,当运行区域或单点模式时,大气驱动数据可选择任意一套覆盖该区域或该点的数据,模式可自动从大气驱动数据中提取出该区域或该点的数据信息来驱动CoLM。

%\begin{landscape}
%\begin{longtable}
%\end{longtable}
%\end{landscape}

% Please add the following required packages to your document preamble:
% \usepackage{booktabs}
\begin{landscape}
%\begin{threeparttable}
\begin{center}
\begin{longtable}{p{3cm}p{3cm}p{2cm}p{2cm}p{4cm}p{6cm}<{\centering}}
\caption{可用于驱动 CoLM 离线运行的大气驱动数据集}
\label{tab:可用于驱动CoLM离线运行的大气驱动数据集}
\\
\hline 
\textbf{驱动名} & \textbf{分辨率}& \textbf{时间范围} & \textbf{空间范围} & \textbf{参考文献} & \textbf{附加说明} \\ 
\hline 
\endfirsthead

\multicolumn{6}{c}%
{{\bfseries \tablename\ \thetable{} -- \kaishu 续表}} \\
\hline
\textbf{驱动名} & \textbf{分辨率} & \textbf{时间范围} & \textbf{空间范围} & \textbf{参考文献} & \textbf{附加说明} \\ 
\hline 
\endhead

\hline 
\multicolumn{6}{r}{{\kaishu 接下一页表格}} \\ 
\hline
\endfoot

\hline
\endlastfoot

\textbf{QIAN}              & 1.875\textdegree/6-hourly  & 1948-2004             & Global                              & Qian et al. (2006)                                                                                                                                                                    & 长波辐射由空气温度和水汽压计算得到                                                                      \\\midrule 
\textbf{CRU-NCEP\_V4}      & 0.5\textdegree/6-hourly    & 1980-2014             & Global                              & Viovy (2011)                                                                                                             & –                                                                                      \\\midrule 
\textbf{CRU-NCEP\_V7}      & 0.5\textdegree/6-hourly    & 1901-2016             & Global                              & Viovy (2018)                                                                                                                                                                                                   & –                                                                                      \\\midrule 
\textbf{CRUJRA2.3}         & 0.5\textdegree/6-hourly    & 1901-2021             & Global                              & Harris et al. (2019)                                                                                                                                                             & 降水与短波辐射在模式中自动由6小时累计值转换为6小时平均值                                                          \\\midrule 
\textbf{Princeton}         & 0.5\textdegree/6-hourly    & 1901-2012             & Global                              & Sheffield et al. (2006)                                                                                                                       & 垂直维度需去除                                                                                \\\midrule 
\textbf{GSWP3}             & 0.5\textdegree/3-hourly    & 1901-2014             & Global                              & Kim (2017)                                                                                                                                                                                                  & 数据在海洋、陆地水体和南极洲的缺省值或异常值已由QIAN数据填充                                                       \\\midrule 
\textbf{GDAS\_GPCP(GLDAS)} & 0.5\textdegree/3-hourly    & 2002-2021             & Global (land only)                  & Rodell et al. (2004); Kumar et al. (2006)                                                                                                                                                                        & 数据文件需按月为存储单位进行合并                                                                       \\\midrule 
\textbf{WFDEI}             & 0.5\textdegree/3-hourly    & 1979-2016             & Global (land only)                  & Weedon et al. (2011); Weedon et al. (2014)                                                                                                                                                               & –                                                                                      \\\midrule 
\textbf{ERA5}              & 0.25\textdegree/hourly     & 1979-2021             & Global                              & Hersbach et al. (2020); Bell et al. (2021)                                                                                                                            & 比湿需根据近地面空气温度和露点温度计算得到                                                                  \\\midrule 
\textbf{ERA5LAND}          & 0.1\textdegree/hourly      & 1951-2021             & Global (land only)                  & Muñoz-Sabater et al. (2021)                                                                                                                               & 比湿需根据近地面空气温度和露点温度计算得到,降水需按照m/hr读入                                                      \\\midrule 
\textbf{MSWX\_V100}        & 0.1\textdegree/3-hourly    & 1979-2021             & Global                              & Beck et al. (2022)                                                                                                                                                        & 降水在模式中自动由3小时累计值转换为3小时平均值;温度在模式中自动转换为单位开尔文;数据文件需按月为存储单位进行合并                             \\\midrule 
\textbf{JRA55}             & 0.5625\textdegree/3-hourly & 1979-2022             & Global                              & Kobayashi et al. (2015)                                                                                                                                                                                & 降水在模式中自动由每天累计值转换为每天平均值;垂直维度需去除                                                         \\\midrule 
\textbf{WFDE5}             & 0.5\textdegree/hourly      & 1979-2019             & Global (land only)                  & Cucchi et al. (2020)                                                                                                           \\\midrule 
\textbf{CLDAS}             & 0.0625\textdegree/hourly   & 2008-2020             & 东亚区域; 60\textdegree $\sim$ 160\textdegree E, 0\textdegree $\sim$ 65\textdegree N   & 国家气象信息中心                                                                                                                                                                              & 降水在模式中自动由逐小时累计值转换为逐小时平均值;异常或缺省值需补充                                                     \\\midrule 
\textbf{CMFD}              & 0.1\textdegree/3-hourly    & 1979-2018             & 中国区域;  70\textdegree $\sim$ 140 \textdegree E, 15\textdegree $\sim$55 \textdegree N & Yang and He (2019); He et al. (2020)                                                                                                                                          & 降水在模式中自动由逐小时累计值转换为逐小时平均值                                                               \\\midrule 
\textbf{TPMFD}             & 0.01\textdegree/hourly     & 1979-2020             & 青藏高原 (25\textdegree $\sim$ 106\textdegree E, 41\textdegree $\sim$ 61\textdegree N) & Yang et al. (2023)                                                                                 & 降水在模式中自动由逐小时累计值转换为逐小时平均值;气压在模式中自动转换为单位帕                                                \\
\textbf{CMIP6/SSPs}        & 1\textdegree /3-hourly      & 1850-2100             & Global                              & Stockhause et al. (2021)                                                                                                                                                                               & 数据文件需按月为存储单位进行合并;需采用异常值叠加方式产生未来情景下的大气驱动数据(见下文说明)                                       \\
\textbf{PLUMBER2}          & 单点/hourly         &1990-2019         & Global                              & Ukkola et al. (2022)                                                                                                                                                & 需制作 surface data 给定站点的USGS/IGBP分类和经纬度等信息                                                 \\\midrule 
\textbf{大气$\mathrm{CO_2}$ 浓度}           & 全球统一值/yearly     & 1849-2100             & Global                              & Büchner and Reyer (2022); Keeling et al. (2005)                                     & 1849-1957 \& 2023-2100 $\mathrm{CO_2}$ 浓度来自ISIMIP3b 大气组份浓度输入数据集;1958-2022 $\mathrm{CO_2}$ 浓度来自夏威夷 Mauna Loa 观测站监测数据 \\\midrule 
\textbf{臭氧浓度}              & 0.1\textdegree /daily       & 2013-2022             & 中国区域 (73\textdegree $\sim$135\textdegree E,18\textdegree $\sim$ 54\textdegree N)                                                                                  & -                                                                                                                                                                     & 来自中科院大气所李芳研究员                                                                        \\\midrule 
\textbf{大气气溶胶沉降速率}         & 0.9\textdegree $\times$ 1.25\textdegree/monthly & 1849-2001 / climatology & Global                              & CLM5 Documentation \& Lamarque et al. (2010)     & 来自NCAR-CAM模式模拟结果                                                                       \\\midrule 
\textbf{氮沉降速率}             & 1.9\textdegree $\times$ 2.5\textdegree/yearly   & 1849-2006             & Global                              & CLM5 Documentation                                                                                                                                                                                                             & 来自NCAR-WACCM模式模拟结果                                                                     \\\midrule 
\textbf{闪电频率}              & 2\textdegree/3-hourly      & climatology           & Global                              & CLM5 Documentation                                                                                                                                                                                                       & 由1995-2011年 NASA LIS/OTD 格点数据产品 v2.2 版通过双线性插值得到        \\

\end{longtable}
\end{center}
\begin{tablenotes}
\footnotesize
%\item 注:
\item[1] 1. 模式中针对以上数据会自动对比湿协调性做检查,当比湿数据大于通过近地面空气温度和气压数据计算得到的饱和比湿时,比湿将被更新为饱和比湿的计算结果。

\item[2] 2. 凡原始数据带有$offset$ 和 $scale\_{factor}$ 属性的均需转换为无 $offset$ 和$scale\_{factor}$ 属性的数据。

\item[3] 3. 大气 $\mathrm{O_2}$ 分压由公式 0.209 $P_{atm}$ 计算得到,其中 $P_{atm}$ 为大气压强。

\item[4] 4. 默认大气风速参考高度为100 米,大气温度和比湿参考高度为50米。

\item[5] 5. 当驱动仅给出总风速时,纬向风速和经向风速平均分配:$u_{atm}=v_{atm}=W_{atm}/\sqrt2$

\end{tablenotes}
%\rule{\textheight}{1pt}
\hrule height 1pt
%%\end{threeparttable}
\end{landscape}





CoLM默认的积分时间步长为1800秒。当大气驱动数据的时间分辨率不满足这一积分步长时,相邻数据之间将进行插值以保持大气驱动数据的时间分辨率与积分步长相同。其中,温度、气压、比湿、风速、长波辐射通量采用线性插值获得,降水速率采用临近时刻插值获得,短波辐射通量采用基于太阳高度角的余弦值加权平均获得。短波辐射通量的插值具体操作为:对于插值时刻$t_M$,短波辐射通量S($t_M$)计算为
\begin{equation}\label{t_M}
S\left(t_{M}\right)=\left\{\begin{array}{ll}\frac{\frac{\Delta t_{FD}}{\Delta t_{M}} S\left(t_{F D}\right) \mu\left(t_{M}\right)}{\sum_{i=1}^{\frac{\Delta t_{FD}}{\Delta t_{M}}} \mu\left(t_{M_{i}}\right)} & \text { 当 } \mu\left(t_{M}\right)>0.001 \\ 0 & \text { 当 } \mu\left(t_{M}\right) \leq 0.001\end{array}\right.
\end{equation}
其中,$\Delta t_{FD}$代表驱动数据的时间间隔(例如,对于3-hourly数据$\Delta t_{FD}$=10800秒),$\Delta t_{M}$代表模式积分步长,$S(t_{FD})$)代表比$t_M$时刻早的最接近$t_M$时刻的短波辐射驱动数据,$\mu\left(t_M\right)$代表$t_M$时刻的太阳高度角的余弦值,$
\sum_{i=1}^{\frac{\Delta t_{F D}}{\Delta t_{M}}} \mu\left(t_{M_{i}}\right)$
代表落在驱动数据时间间隔内的所有模式积分时刻的太阳高度角余弦值的总和。

短波太阳辐射在模式中被分为可见光与近红外两个波段,分配系数$\alpha$=0.5。对于任意波段,太阳辐射被进一步分为直射光与漫射光。直射光在可见光波段的比例系数为
\begin{equation}\label{R_vis}
R_{vis}=a_{0}+a_{1} \alpha S+a_{2}(\alpha S)^{2}+a_{3}(\alpha S)^{3} \quad 0.01 \leq R_{vis} \leq 0.99
\end{equation}
直射光在近红外波段的比例系数为
\begin{equation}
R_{nir}=b_{0}+b_{1}(1-\alpha) S+b_{2}((1-\alpha) S)^{2}+b_{3}((1-\alpha) S)^{3} \quad 0.01 \leq R_{nir} \leq 0.99
\end{equation}
其中$a_0$=0.17639, $a_1$=0.0038, $a_2=-9.0039\times{10}^{-8}$, $a_3=8.1351\times10^{-9}$, $b_0=0.29548$, $b_1=0.00504$, $b_2=-1.4957\times10^{-5}$, $b_3=1.4881\times10^{-8}$。这些系数来自对 NCAR-CAM 模式模拟结果的拟合。

\begin{mymdframed}{需补充内容}
\textbf{\color{red}降水拆分雨雪方案由谭奥博同学补充…}
\end{mymdframed}

当大气驱动数据未提供下行长波辐射通量时,模式将根据大气温度$T_{atm}$、比湿$q_{atm}$和水汽压$e_{atm}$计算获得(Idso 1981),计算方案为
\begin{equation}\label{L_downarrow}
L \downarrow=\left[0.70+5.95 \times 10^{-5} \times 0.01 e_{a t m} \exp \left(\frac{1500}{T_{a t m}}\right)\right] \sigma T_{a t m}^{4}
\end{equation}
其中大气水汽压计算为$e_{a t m}=\frac{P_{a t m} q_{a t m}}{0.622+0.378 q_{a t m}}$,$\sigma$为Stefan-Boltzmann常数。

未来情景下的大气驱动数据可采用CMIP6气候模式的模拟输出结果。当历史时期与未来情景共同模拟时,未来时期CMIP6气候模式的模拟结果往往不能直接使用,因为当模拟历史时期使用的大气驱动数据来自观测或再分析数据时,其均值、时空变率、数据拼接时刻的取值与CMIP6气候模式的模拟结果往往存在较大差异。这里可使用异常值叠加方式来融合历史与未来的大气驱动数据。举例说明:当模拟1850-2099年的陆面状态时,1850-2014年的历史时期模拟可采用GSWP3大气驱动数据,2015-2099年的未来时期模拟可采用CMIP6气候模式的模拟结果,两套数据的融合方式如下:首先,将2015-2100年各驱动变量的逐月数据去除气候态的季节循环,得到带有某种社会经济排放情景下的各驱动变量的异常值序列,此序列可表征未来气候变化趋势的影响;其次,将历史时期有限时间段的各大气驱动变量反复叠加到未来时期各驱动变量的异常值序列(例如将2010-2014年的大气驱动数据反复叠加到2015-2019、2020-2024、…、2095-2099年的异常值序列),得到可使用的未来时期的大气驱动数据。此种数据融合方式既得到了未来时期的气候变化趋势,又保留了历史时期大气驱动数据的年际变率,缩小了历史时期与未来时期数据拼接时刻的数据阶跃程度。对于不同的大气驱动变量,历史时期数据与未来时期异常值序列的叠加方式不同:对于温度、气压、比湿、风速等大气状态变量,叠加方式采用加法叠加($S_{new}=S_{old}+k_{anomaly}$),对于降水、下行短波辐射通量、下行长波辐射通量,叠加方式采用乘法叠加($F_{new}=F_{old}\times k_{anomaly}$)。


最后,针对单点模拟,CoLM已经收集整理测试的站点信息见表~\ref{tab:CoLM离线运行已经测试的站点列表} 。

\begin{center}
\begin{longtable}{lcccccc}
\caption{CoLM离线运行已经测试的站点列表}
\label{tab:CoLM离线运行已经测试的站点列表}
\\
\hline 
\textbf{站点名称} & \textbf{起始年份} & \textbf{终止年份} & \textbf{数据来源} & \textbf{IGBP分类} & \textbf{纬度坐标} & \textbf{经度坐标}\\ \hline 
\endfirsthead

\multicolumn{7}{c}%
{{\bfseries \tablename\ \thetable{} -- \kaishu 续表}} \\
\hline \multicolumn{1}{c}{\textbf{站点名称}} & \multicolumn{1}{c}{\textbf{起始年份}} & \multicolumn{1}{c}{\textbf{终止年份}} & \multicolumn{1}{c}{\textbf{数据来源}} & \multicolumn{1}{c}{\textbf{IGBP分类}} & \multicolumn{1}{c}{\textbf{纬度坐标}} & \multicolumn{1}{c}{\textbf{经度坐标}}\\ \hline 
\endhead

\hline \multicolumn{7}{r}{{\kaishu 接下一页表格}} \\ \hline
\endfoot

\hline
\endlastfoot


AR-SLu & 2010 & 2010 & FLUXNET2015 & 5      & -66.45980072 & -33.46480179 \\
AT-Neu & 2002 & 2012 & FLUXNET2015 & 10     & 11.31750011  & 47.1166687   \\
AU-ASM & 2011 & 2017 & OzFlux      & 1      & 133.2489929  & -22.28300095 \\
AU-Cow & 2010 & 2015 & OzFlux      & 2      & 145.4271545  & -16.2381897  \\
AU-Cpr & 2011 & 2017 & OzFlux      & 9      & 140.5891266  & -34.00205994 \\
AU-Ctr & 2010 & 2017 & OzFlux      & 2      & 145.4468536  & -16.10327911 \\
AU-Cum & 2013 & 2018 & OzFlux      & 2      & 150.7224731  & -33.61329651 \\
AU-DaP & 2009 & 2012 & OzFlux      & 10     & 131.3181     & -14.06330013 \\
AU-DaS & 2010 & 2017 & OzFlux      & 9      & 131.3880005  & -14.15928268 \\
AU-Dry & 2011 & 2015 & OzFlux      & 9      & 132.3706055  & -15.25879955 \\
AU-Emr & 2012 & 2013 & OzFlux      & 10     & 148.4745941  & -23.8586998  \\
AU-GWW & 2013 & 2017 & OzFlux      & 9      & 120.6540985  & -30.19129944 \\
AU-Gin & 2012 & 2017 & OzFlux      & 8      & 115.6500015  & -31.375      \\
AU-How & 2003 & 2017 & OzFlux      & 8      & 131.1499939  & -12.49520016 \\
AU-Lit & 2016 & 2017 & OzFlux      & 8      & 130.7944946  & -13.17903996 \\
AU-Otw & 2009 & 2010 & OzFlux      & 10     & 142.816803   & -38.532341   \\
AU-Rig & 2011 & 2016 & OzFlux      & 10     & 145.5758972  & -36.64989853 \\
AU-Rob & 2014 & 2017 & OzFlux      & 2      & 145.6300964  & -17.11750031 \\
AU-Sam & 2011 & 2017 & OzFlux      & 10     & 152.8778076  & -27.38809967 \\
AU-Stp & 2010 & 2017 & OzFlux      & 10     & 133.3502045  & -17.15069962 \\
AU-TTE & 2013 & 2017 & OzFlux      & 7      & 133.6399994  & -22.28700066 \\
AU-Tum & 2002 & 2017 & OzFlux      & 2      & 148.1517029  & -35.65660095 \\
AU-Whr & 2015 & 2016 & OzFlux      & 2      & 145.0294037  & -36.6731987  \\
AU-Wrr & 2016 & 2017 & OzFlux      & 2      & 146.6544952  & -43.09502029 \\
AU-Ync & 2011 & 2017 & OzFlux      & 10     & 146.2906952  & -34.98929977 \\
BE-Bra & 2004 & 2014 & FLUXNET2015 & 5      & 4.520559788  & 51.30916595  \\
BE-Lon & 2005 & 2014 & FLUXNET2015 & 12     & 4.74612999   & 50.55158997  \\
BE-Vie & 1997 & 2014 & FLUXNET2015 & 5      & 5.998050213  & 50.30506897  \\
BR-Sa3 & 2001 & 2003 & FLUXNET2015 & 2      & -54.97143555 & -3.018029213 \\
BW-Ma1 & 2000 & 2000 & LaThuile    & 9      & 23.56032944  & -19.91650009 \\
CA-NS1 & 2003 & 2003 & FLUXNET2015 & 1      & -98.48390198 & 55.87919998  \\
CA-NS2 & 2002 & 2004 & FLUXNET2015 & 1      & -98.52469635 & 55.90579987  \\
CA-NS4 & 2003 & 2004 & FLUXNET2015 & 1      & -98.38059998 & 55.91439819  \\
CA-NS5 & 2003 & 2004 & FLUXNET2015 & 1      & -98.48500061 & 55.86309814  \\
CA-NS6 & 2002 & 2004 & FLUXNET2015 & 7      & -98.96440125 & 55.91669846  \\
CA-NS7 & 2003 & 2004 & FLUXNET2015 & 7      & -99.94830322 & 56.63579941  \\
CA-Qcu & 2002 & 2006 & LaThuile    & 1      & -74.03652954 & 49.2670784   \\
CA-Qfo & 2004 & 2010 & FLUXNET2015 & 1      & -74.34210205 & 49.69250107  \\
CA-SF1 & 2004 & 2006 & FLUXNET2015 & 1      & -105.8175964 & 54.48500061  \\
CA-SF2 & 2003 & 2005 & FLUXNET2015 & 1      & -105.8775024 & 54.25389862  \\
CA-SF3 & 2003 & 2005 & FLUXNET2015 & 7      & -106.0053024 & 54.09159851  \\
CH-Cha & 2006 & 2014 & FLUXNET2015 & 10     & 8.410440445  & 47.21022034  \\
CH-Dav & 1997 & 2014 & FLUXNET2015 & 1      & 9.855919838  & 46.81533432  \\
CH-Fru & 2007 & 2014 & FLUXNET2015 & 10     & 8.537779808  & 47.11583328  \\
CH-Oe1 & 2002 & 2008 & FLUXNET2015 & 10     & 7.731939793  & 47.28583145  \\
CN-Cha & 2003 & 2005 & FLUXNET2015 & 5      & 128.0957947  & 42.40250015  \\
CN-Cng & 2008 & 2009 & FLUXNET2015 & 10     & 123.509201   & 44.59339905  \\
CN-Dan & 2004 & 2005 & FLUXNET2015 & 10     & 91.06639862  & 30.49780083  \\
CN-Din & 2003 & 2005 & FLUXNET2015 & 2      & 112.5361023  & 23.17329979  \\
CN-Du2 & 2007 & 2008 & FLUXNET2015 & 10     & 116.2835999  & 42.04669952  \\
CN-HaM & 2002 & 2003 & FLUXNET2015 & 10     & 101.1800003  & 37.36999893  \\
CN-Qia & 2003 & 2005 & FLUXNET2015 & 1      & 115.0580978  & 26.74139977  \\
CZ-wet & 2007 & 2014 & FLUXNET2015 & 11     & 14.77035046  & 49.02465057  \\
DE-Bay & 1997 & 1999 & LaThuile    & 1      & 11.86693954  & 50.14194107  \\
DE-Geb & 2001 & 2014 & FLUXNET2015 & 12     & 10.91429996  & 51.10010147  \\
DE-Gri & 2004 & 2014 & FLUXNET2015 & 10     & 13.51253033  & 50.94947052  \\
DE-Hai & 2000 & 2012 & FLUXNET2015 & 4      & 10.45300007  & 51.07916641  \\
DE-Kli & 2005 & 2014 & FLUXNET2015 & 12     & 13.52237988  & 50.89305878  \\
DE-Meh & 2004 & 2006 & LaThuile    & 5      & 10.65546989  & 51.27531052  \\
DE-Obe & 2008 & 2014 & FLUXNET2015 & 1      & 13.72128963  & 50.78665924  \\
DE-Seh & 2008 & 2010 & FLUXNET2015 & 12     & 6.449649811  & 50.87062454  \\
DE-SfN & 2013 & 2014 & FLUXNET2015 & 11     & 11.32750034  & 47.80638885  \\
DE-Tha & 1998 & 2014 & FLUXNET2015 & 1      & 13.56694031  & 50.9636116   \\
DE-Wet & 2002 & 2006 & LaThuile    & 1      & 11.45753002  & 50.45349884  \\
DK-Fou & 2005 & 2005 & FLUXNET2015 & 12     & 9.587220192  & 56.48419952  \\
DK-Lva & 2005 & 2006 & LaThuile    & 10     & 12.08329964  & 55.68330002  \\
DK-Ris & 2004 & 2005 & LaThuile    & 12     & 12.09722042  & 55.53028107  \\
DK-Sor & 1997 & 2014 & FLUXNET2015 & 4      & 11.64463997  & 55.48587036  \\
DK-ZaH & 2000 & 2013 & FLUXNET2015 & 10     & -20.5503006  & 74.47319794  \\
ES-ES1 & 1999 & 2006 & LaThuile    & 1      & -0.318809986 & 39.34597015  \\
ES-ES2 & 2005 & 2006 & LaThuile    & 12     & -0.31527999  & 39.27555847  \\
ES-LMa & 2004 & 2006 & LaThuile    & 9      & -5.773359776 & 39.9426918   \\
ES-LgS & 2007 & 2007 & FLUXNET2015 & 7      & -2.965830088 & 37.09793472  \\
ES-VDA & 2004 & 2004 & LaThuile    & 10     & 1.448500037  & 42.15217972  \\
FI-Hyy & 1996 & 2014 & FLUXNET2015 & 1      & 24.29500008  & 61.84749985  \\
FI-Kaa & 2000 & 2002 & LaThuile    & 11     & 27.29503059  & 69.14069366  \\
FI-Lom & 2007 & 2009 & FLUXNET2015 & 11     & 24.20918083  & 67.99720001  \\
FI-Sod & 2008 & 2014 & FLUXNET2015 & 1      & 26.63783073  & 67.36186218  \\
FR-Fon & 2005 & 2013 & FLUXNET2015 & 4      & 2.780139923  & 48.47639847  \\
FR-Gri & 2005 & 2013 & FLUXNET2015 & 12     & 1.951910019  & 48.84421921  \\
FR-Hes & 1997 & 2006 & LaThuile    & 4      & 7.065559864  & 48.67416     \\
FR-LBr & 2003 & 2008 & FLUXNET2015 & 1      & -0.769299984 & 44.71710968  \\
FR-Lq1 & 2004 & 2006 & LaThuile    & 10     & 2.735830069  & 45.64305878  \\
FR-Lq2 & 2004 & 2006 & LaThuile    & 10     & 2.737030029  & 45.63919067  \\
FR-Pue & 2000 & 2014 & FLUXNET2015 & 2      & 3.595829964  & 43.74139023  \\
GF-Guy & 2004 & 2014 & FLUXNET2015 & 2      & -52.92485809 & 5.278771877  \\
HU-Bug & 2003 & 2006 & LaThuile    & 10     & 19.60129929  & 46.69110107  \\
ID-Pag & 2002 & 2003 & LaThuile    & 2      & 113.9000015  & -2.319999933 \\
IE-Ca1 & 2004 & 2006 & LaThuile    & 12     & -6.918139935 & 52.85879135  \\
IE-Dri & 2003 & 2005 & LaThuile    & 10     & -8.751810074 & 51.98669052  \\
IT-Amp & 2003 & 2006 & LaThuile    & 10     & 13.60515976  & 41.90409851  \\
IT-BCi & 2005 & 2010 & FLUXNET2015 & 12     & 14.95744038  & 40.5237999   \\
IT-CA1 & 2012 & 2013 & FLUXNET2015 & 4      & 12.02655983  & 42.38040924  \\
IT-CA2 & 2012 & 2013 & FLUXNET2015 & 12     & 12.02604008  & 42.37722015  \\
IT-CA3 & 2012 & 2013 & FLUXNET2015 & 4      & 12.02219963  & 42.38000107  \\
IT-Col & 2007 & 2014 & FLUXNET2015 & 4      & 13.58813953  & 41.84936142  \\
IT-Cpz & 2001 & 2008 & FLUXNET2015 & 2      & 12.37611008  & 41.70524979  \\
IT-Isp & 2013 & 2014 & FLUXNET2015 & 4      & 8.633580208  & 45.81264114  \\
IT-LMa & 2003 & 2004 & LaThuile    & 4      & 7.582590103  & 45.15258026  \\
IT-Lav & 2005 & 2014 & FLUXNET2015 & 1      & 11.28131962  & 45.95619965  \\
IT-MBo & 2003 & 2012 & FLUXNET2015 & 10     & 11.04582977  & 46.01467896  \\
IT-Mal & 2003 & 2003 & LaThuile    & 10     & 11.70333958  & 46.1140213   \\
IT-Noe & 2004 & 2014 & FLUXNET2015 & 6      & 8.151689529  & 40.60617828  \\
IT-Non & 2002 & 2002 & LaThuile    & 4      & 11.0910902   & 44.69018936  \\
IT-PT1 & 2003 & 2004 & FLUXNET2015 & 4      & 9.061039925  & 45.20087051  \\
IT-Ren & 2010 & 2013 & FLUXNET2015 & 1      & 11.43369007  & 46.58686066  \\
IT-Ro1 & 2002 & 2006 & FLUXNET2015 & 4      & 11.93000984  & 42.4081192   \\
IT-Ro2 & 2002 & 2008 & FLUXNET2015 & 4      & 11.92092991  & 42.39025879  \\
IT-SR2 & 2013 & 2014 & FLUXNET2015 & 1      & 10.29090977  & 43.73202896  \\
IT-SRo & 2003 & 2012 & FLUXNET2015 & 1      & 10.28444004  & 43.7278595   \\
JP-SMF & 2003 & 2006 & FLUXNET2015 & 5      & 137.0787964  & 35.26169968  \\
NL-Ca1 & 2003 & 2006 & LaThuile    & 10     & 4.927000046  & 51.97100067  \\
NL-Hor & 2008 & 2011 & FLUXNET2015 & 10     & 5.07130003   & 52.24034882  \\
NL-Loo & 1997 & 2013 & FLUXNET2015 & 1      & 5.743559837  & 52.1665802   \\
PL-wet & 2004 & 2005 & LaThuile    & 11     & 16.30940056  & 52.7621994   \\
PT-Esp & 2002 & 2004 & LaThuile    & 2      & -8.601799965 & 38.63940048  \\
PT-Mi1 & 2005 & 2005 & LaThuile    & 2      & -8.000060081 & 38.54064178  \\
PT-Mi2 & 2005 & 2006 & LaThuile    & 10     & -8.024550438 & 38.47650146  \\
RU-Che & 2003 & 2004 & FLUXNET2015 & 11     & 161.3414307  & 68.61303711  \\
RU-Fyo & 2003 & 2014 & FLUXNET2015 & 1      & 32.92208099  & 56.46152878  \\
RU-Zot & 2003 & 2003 & LaThuile    & 8      & 89.35079956  & 60.80080032  \\
SD-Dem & 2005 & 2009 & FLUXNET2015 & 9      & 30.47830009  & 13.28289986  \\
SE-Deg & 2002 & 2005 & LaThuile    & 10     & 19.55653954  & 64.18196869  \\
UK-Gri & 2000 & 2001 & LaThuile    & 1      & -3.79805994  & 56.6072197   \\
UK-Ham & 2004 & 2004 & LaThuile    & 4      & -0.858299971 & 51.15353012  \\
UK-PL3 & 2005 & 2006 & LaThuile    & 5      & -1.266669989 & 51.45000076  \\
US-AR1 & 2010 & 2012 & FLUXNET2015 & 10     & -99.41999817 & 36.42670059  \\
US-AR2 & 2010 & 2011 & FLUXNET2015 & 10     & -99.59750366 & 36.63579941  \\
US-ARM & 2003 & 2012 & FLUXNET2015 & 12     & -97.48880005 & 36.60580063  \\
US-Aud & 2003 & 2005 & LaThuile    & 10     & -110.509201  & 31.59070015  \\
US-Bar & 2005 & 2005 & LaThuile    & 5      & -71.28807831 & 44.06464005  \\
US-Bkg & 2005 & 2006 & LaThuile    & 10     & -96.83616638 & 44.34529114  \\
US-Blo & 2000 & 2006 & FLUXNET2015 & 1      & -120.6327515 & 38.89530182  \\
US-Bo1 & 1997 & 2006 & LaThuile    & 12     & -88.29039764 & 40.00619888  \\
US-Cop & 2002 & 2003 & FLUXNET2015 & 10     & -109.3899994 & 38.09000015  \\
US-FPe & 2000 & 2006 & LaThuile    & 10     & -105.1018982 & 48.30770111  \\
US-GLE & 2009 & 2014 & FLUXNET2015 & 1      & -106.2398987 & 41.36650085  \\
US-Goo & 2004 & 2006 & LaThuile    & 10     & -89.87349701 & 34.25469971  \\
US-Ha1 & 1992 & 2012 & FLUXNET2015 & 4      & -72.17150116 & 42.53779984  \\
US-Ho1 & 1996 & 2004 & LaThuile    & 1      & -68.74019623 & 45.20410156  \\
US-KS2 & 2003 & 2006 & FLUXNET2015 & 6      & -80.67153168 & 28.60857773  \\
US-Los & 2000 & 2008 & FLUXNET2015 & 11     & -89.97920227 & 46.08269882  \\
US-MMS & 1999 & 2014 & FLUXNET2015 & 4      & -86.4131012  & 39.32320023  \\
US-MOz & 2005 & 2006 & LaThuile    & 4      & -92.20001221 & 38.74411011  \\
US-Me2 & 2002 & 2014 & FLUXNET2015 & 1      & -121.5574036 & 44.45230103  \\
US-Me4 & 1996 & 2000 & LaThuile    & 1      & -121.6223984 & 44.49919891  \\
US-Me6 & 2011 & 2014 & FLUXNET2015 & 1      & -121.6078033 & 44.32329941  \\
US-Myb & 2011 & 2014 & FLUXNET2015 & 11     & -121.7650986 & 38.04980087  \\
US-NR1 & 1999 & 2014 & FLUXNET2015 & 1      & -105.546402  & 40.03290176  \\
US-Ne1 & 2002 & 2012 & FLUXNET2015 & 12     & -96.47660065 & 41.1651001   \\
US-Ne2 & 2002 & 2012 & FLUXNET2015 & 12     & -96.4701004  & 41.16490173  \\
US-Ne3 & 2002 & 2012 & FLUXNET2015 & 12     & -96.43969727 & 41.17969894  \\
US-PFa & 1995 & 2014 & FLUXNET2015 & 5      & -90.27230072 & 45.94589996  \\
US-Prr & 2011 & 2013 & FLUXNET2015 & 1      & -147.4875946 & 65.123703    \\
US-SP1 & 2005 & 2005 & LaThuile    & 1      & -82.21877289 & 29.73806953  \\
US-SP2 & 2000 & 2004 & LaThuile    & 1      & -82.24481964 & 29.76479912  \\
US-SP3 & 1999 & 2004 & LaThuile    & 2      & -82.16327667 & 29.75477028  \\
US-SRG & 2009 & 2014 & FLUXNET2015 & 10     & -110.8276978 & 31.7894001   \\
US-SRM & 2004 & 2014 & FLUXNET2015 & 8      & -110.8659973 & 31.82139969  \\
US-Syv & 2002 & 2008 & FLUXNET2015 & 5      & -89.34770203 & 46.24200058  \\
US-Ton & 2001 & 2014 & FLUXNET2015 & 8      & -120.9660034 & 38.43159866  \\
US-Tw4 & 2014 & 2014 & FLUXNET2015 & 11     & -121.6413269 & 38.10274506  \\
US-Twt & 2010 & 2014 & FLUXNET2015 & 12     & -121.6530991 & 38.1086998   \\
US-UMB & 2000 & 2014 & FLUXNET2015 & 4      & -84.71379852 & 45.55979919  \\
US-Var & 2001 & 2014 & FLUXNET2015 & 10     & -120.9507599 & 38.41329956  \\
US-WCr & 1999 & 2006 & FLUXNET2015 & 4      & -90.07990265 & 45.80590057  \\
US-Whs & 2008 & 2014 & FLUXNET2015 & 7      & -110.0522003 & 31.74379921  \\
US-Wkg & 2005 & 2014 & FLUXNET2015 & 10     & -109.9419022 & 31.73649979  \\
ZA-Kru & 2000 & 2002 & FLUXNET2015 & 9      & 31.49690056  & -25.0196991  \\
ZM-Mon & 2008 & 2008 & FLUXNET2015 & 4      & 23.25278091  & -15.43777847 \\ 
\end{longtable}
\end{center}


\subsection{在地球系统模式中通过耦合器实现CoLM与大气模式的耦合模拟}
\subsubsection{通用耦合框架(或耦合器介绍}
\subsubsection{CAS-ESM/CWRF耦合实例}
CPL7是NCAR研发的耦合器,它被广泛地应用于地球系统模型中以调度不同模块的运行并实现模块之间的变量交换。CoLM202X支持CPL7,可通过CPL7中的模型耦合工具(MCT,The Model Coupling Toolkit)作为陆面分量与其他支持CPL7的模型进行耦合。本节将介绍CoLM202X与中国科学院地球系统模式(CAS-ESM)和区域气候模式(CWRF)的耦合实例。