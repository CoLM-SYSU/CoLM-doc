\chapter{光合作用和气孔导度}
%\addcontentsline{toc}{chapter}{光合作用和气孔导度}

%\begin{光合作用和气孔导度}

\section{植物的光合作用}\label{植物的光合作用}
CoLM光合作用模型是建立在冠层尺度,双大叶冠层结构基础上的。即冠层尺度的光合同化速率($A_{n}$)等于阳叶冠层的光合同化速率($A_{n,sun}$)加上阴叶冠层的光合同化速率($A_{n,sun}$):
\begin{equation}\label{Ansun_Ansha}
A_{n}=A_{n,sun}+A_{n,sha}
\end{equation}

阳叶和阴叶冠层尺度的光合同化速率模拟是通过对叶片尺度的光合作用模型进行升尺度而得到。C3植物叶片尺度的光合作用模拟是基于Farquhar光合作用模型~\citep{farquhar1980biochemical},
C4植物则是基于~\citet{collatz1992} 的光合作用改进模型。卡尔文循环是高等植物光合作用中的重要途径之一,CO$_2$和1,5-二磷酸核酮糖(RuBP)在1,5-二磷酸核酮糖羧化酶(Rubisco)的催化下产生羧化反应,生成3-磷酸甘油酸(PGA),并被还原为3-磷酸甘油醛(PGAL),PGA经过一系列转变再次形成RuBP,这被称为RuBP的更新阶段,完成卡尔文循环。当RuBP充足,卡尔文循环稳定时,多余的PGAL才会合成蔗糖和淀粉作为光合作用的产物存储在植物内。Faquhar模型认为卡尔文循环中的羧化速率($A_{c}$)和RuBP的再生速率是限制($A_{j}$)是光合作用总速率的两个关键限制因子,CoLM光合作用模块在此基础上增加了蔗糖和淀粉产物合成的速率限制($A_{p}$)。另外,叶片净光合同化速率 ($A_{n}$) 等于总光合同化速率减去叶呼吸速率 ($R_d$),在阴叶和阳叶冠层分别有:
\begin{equation}\label{An1sun}
A_{n,sun}=\min \left(A_{c,sun}, A_{j,sun}, A_{p,sun}\right)-R_{d,sun}
\end{equation}
\begin{equation}\label{An1sha}
A_{n,sha}=\min \left(A_{c,sha}, A_{j,sha}, A_{p,sha}\right)-R_{d,sha}
\end{equation}
其中,下标$sun$和$sha$分别代表在阳叶和阴叶冠层尺度上的光合同化速率和呼吸速率。


RuBP羧化酶限制主要指当羧化酶活性较低时,光合作用羧化速率受限,也称为Rubisco限制。由于Rubisco同时催化RuBP的羧化反应和氧化反应,羧化反应需要CO$_2$而氧化反应需要O$_2$,羧化反应和氧化反应同时进行。因此,C3植物的胞间CO$_2$分压$c_i$和O$_2$分压$o_i$的比例极大影响了Rubisco作为羧化酶的效率。基于这个原因,C3植物的Rubisco限制下的光合同化速率($A_c$)可表达为Michaelis-Menten关于$c_i$和$o_i$的函数。此外,C4植物Rubisco限制下的光合羧化速率不再受胞间CO$_2$分压和O$_2$分压的影响,这是由于C4植物的光合作用途径的最初步骤是PEP的羧化,PEP羧化酶的活性很高,所以转运到维管束鞘细胞中的CO$_2$的浓度很高,大约是空气中的十倍。C4植物的卡尔文循环在在维管束鞘细胞进行,CO$_2$分压和O$_2$分压的变化对羧化速率的影响较小,因此,Rubisco限制下的光合同化速率在阳叶和阴叶冠层尺度($A_{c,sun}$和$A_{c,sha}$)下可表示为CO$_2$分压($c_{i,sun}$和$c_{i,sha}$),O$_2$分压($o_{i,sun}$和$o_{i,sha}$)和最大羧化速率的函数($V_{c \max,sun}$和$V_{c \max,sha}$):
\begin{equation}\label{A_C1sun}
A_{c,sun}=\begin{cases}
\frac{V_{c \max,sun}\left(c_{i,sun}-\Gamma_{sun}\right)}{c_{i,sun}+K_{c}\left(1+\frac{o_{i}}{K_{o}}\right)}
     & \text{for C3 plants} \\ 
V_{c \max,sun } & \text{for C4 plants}
\end{cases}
\end{equation}
\begin{equation}\label{A_C1sha}
A_{c,sha}=\begin{cases}
\frac{V_{c \max,sha}\left(c_{i,sha}-\Gamma_{sha}\right)}{c_{i,sha}+K_{c}\left(1+\frac{o_{i}}{K_{o}}\right)}
     & \text{for C3 plants} \\ 
V_{c \max,sha } & \text{for C4 plants}
\end{cases}
\end{equation}

其中,下标$sun$和$sha$分别代表在阳叶和阴叶冠层尺度上的相应变量,$\Gamma$是CO$_2$补偿点,$K_c$和$K_o$分别是对于CO$_2$和O$_2$的Michaelis--Menten常数(Pa),$V_{c \max,sun}$和$V_{c \max,sha}$代表阳叶和阴叶冠层尺度的最大羧化速率,它是根据叶面积指数从叶片尺度上升到冠层尺度的光合参数。\\
C3植物:\\
\begin{equation}\label{V_cmaxsun_a}
V_{c \max,sun }=\frac{V_{c \max 25} \cdot 2.1^{\frac{T_{{leaf }}-T_{o p}}{10}}}{1+e^{s_{1}\left(T_{{leaf }}-T_{{high }}\right)}} \cdot \beta_{sun} \cdot L_{v,sun}
\end{equation}
\begin{equation}\label{V_cmaxsha_a}
V_{c \max,sha }=\frac{V_{c \max 25} \cdot 2.1^{\frac{T_{{leaf }}-T_{o p}}{10}}}{1+e^{s_{1}\left(T_{{leaf }}-T_{{high }}\right)}} \cdot \beta_{sha} \cdot L_{v,sha}
\end{equation}
C4植物:\\
\begin{equation}\label{V_cmaxsun_b}
V_{c \max,sun }= \frac{V_{c \max 25} \cdot 2.1^{\frac{T_{{leaf }}-T_{o p}}{10}}}{\left(1+e^{s_{2}\left(T_{{low }}
 - T_{{leaf }}\right)}\right)\left(1+e^{s_{1}\left(T_{{leaf }}-T_{h i g h}\right)}\right)} \cdot \beta_{sun} \cdot L_{v,sun}
\end{equation}
\begin{equation}\label{V_cmaxsha_b}
V_{c \max,sha }= \frac{V_{c \max 25} \cdot 2.1^{\frac{T_{{leaf }}-T_{o p}}{10}}}{\left(1+e^{s_{2}\left(T_{{low }}
 - T_{{leaf }}\right)}\right)\left(1+e^{s_{1}\left(T_{{leaf }}-T_{h i g h}\right)}\right)} \cdot \beta_{sha} \cdot L_{v,sha}
\end{equation}
不同植被的羧化能力存在差异,我们用25 \textcelsius 下的最大羧化速率 $V_{c \max 25}$ 来刻画植被的光合羧化能力,单位: \unit{mol.m^{-2}.s^{-1}};Rubisco羧化酶的活性随温度显著变化,CoLM光合作用模块针对C3和C4植物制定了两套温度响应函数。$T_{op}$是参考温度298 K;$s_1$和$s_2$分别为高温和低温的温度敏感性参数;$T_{low}$和$T_{high}$分别为羧化速率的低温和高温响应参数,
取值根据植被类型而变化,范围分别为: 278$\sim$288 K,303$\sim$313 K;$T_{leaf}$是叶片温度,通过对叶片能量平衡方程进行牛顿迭代方法而求解得到,详见章节~\ref{植被叶片温度计算},$\beta_{sun}$和$\beta_{sha}$是阳叶和阴叶的水分胁迫因子,水分胁迫因子的取值范围0$\sim$1,详见章节~\ref{气孔导度的水分胁迫}。当植被水力模式开启时,需要最大气孔导度作为植被水力模式输入,在水分胁迫因子为1的条件下,光合气孔耦合模型可得到最大气孔导度。另外,$L$代表叶片到冠层的尺度转换系数,下标sun和sha分别代表阳叶和阴叶,v代表参数$V_{c \max}$的转换系数,它们均是叶面积指数的函数:

\begin{equation}\label{L_vsun}
L_{v,sun}=\frac{1-e^{-\left(0.11+K_{b}\right) \cdot LAI}}{0.11+K_{b}}
\end{equation}
\begin{equation}\label{L_vsha}
L_{v,sha}=\frac{1-e^{-0.11\cdot LAI}}{0.11} - L_{v,sun}
\end{equation}
$K_{b}$代表直射光的消光系数。

当光照不足时,RuBP再生速率下降,成为制约光合作用开尔文循环的最主要因素。因此,RuBP再生速率限制下的羧化速率在阳叶和阴叶冠层尺度下 ($A_{j,sun}$和$A_{j,sha}$) 可表达为阳叶和阴叶的有效光合辐射 ($PAR_{sun}$和$PAR_{sha}$) 的函数:
\begin{equation}\label{A_J1sun}
A_{j,sun}=\begin{cases}\frac{J_{x,sun}\cdot\left(c_{i,sun}-\Gamma\right)}{4c_{i,sun}+8\Gamma}
     & \text{for C3 plants} \\
\alpha\left(4.6 \times 10^{-6} \cdot PAR_{sun}\right) & \text{for C4 plants}
\end{cases}
\end{equation}
\begin{equation}\label{A_J1sha}
A_{j,sha}=\begin{cases}\frac{J_{x,sha}\cdot\left(c_{i,sha}-\Gamma\right)}{4c_{i,sha}+8\Gamma}
     & \text{for C3 plants} \\
\alpha\left(4.6 \times 10^{-6} \cdot PAR_{sha}\right) & \text{for C4 plants}
\end{cases}
\end{equation}

$J_{x,sun}$和$J_{x,sha}$是阳叶和阴叶冠层尺度的电子传输速率,是有效光合辐射($PAR_{sun}$和$PAR_{sha}$)的函数,并受叶片温度 ($T_{leaf}$)和水分胁迫因子($\beta_{sun}$和$\beta_{sha}$)的调节:
\begin{equation}
\begin{aligned}
J_{x,sun}=\min \left(\begin{array}{c} \alpha\left(4.6 \times 10^{-6} \cdot PAR_{sun}\right) \\
  J_{x25} \cdot \exp\left(\frac{37000\left(T_{{leaf }}-T_{o p}\right)\left[1+\exp\left(\frac{710 \cdot T_{o p}-220000}
 {R \cdot T_{o p}}\right)\right]R \cdot T_{{leaf}}}{T_{o p} \cdot T_{{leaf }} \cdot R 
 \cdot \left(710 \cdot T_{{leaf }}-220000\right)}\right)  \end{array} \right) \beta_{sun} \cdot L_{j,sun} 
\end{aligned}
\end{equation}
\begin{equation}
\begin{aligned}
J_{x,sha}=\min \left(\begin{array}{c} \alpha\left(4.6 \times 10^{-6} \cdot PAR_{sha}\right) \\
  J_{x25} \cdot \exp\left(\frac{37000\left(T_{{leaf }}-T_{o p}\right)\left[1+\exp\left(\frac{710 \cdot T_{o p}-220000}
 {R \cdot T_{o p}}\right)\right]R \cdot T_{{leaf}}}{T_{o p} \cdot T_{{leaf }} \cdot R 
 \cdot \left(710 \cdot T_{{leaf }}-220000\right)}\right)  \end{array} \right) \beta_{sha} \cdot L_{j,sha} 
\end{aligned}
\end{equation}
其中$\alpha$是量子效率 (\qty{0.05}{mol.CO_2.mol^{-1}.photon});$PAR$是有效光合辐射,单位: \unit{W.m^{-2}},详细计算见章节~\ref{短波吸收辐射通量};
\num{4.6e-6} 代表单位从 \unit{W.m^{-2}} 转换到 \unit{mol.photon.m^{-2}} 的转换系数;
$J_{x25}$是25 \textcelsius 下的最大电子传输速率,单位: \unit{mol.m^{-2}.s^{-1}},$J_{x25}=1.97 \cdot V_{c \max 25}$; 
$R$是通用气体常数,$R=$ \qty{8.3145}{mol.m^{-2}.s^{-1}}。$L_{j,sun}$和$L_{j,sha}$代表电子传输速率的尺度转换系数(叶片到冠层):

\begin{equation}\label{L_jsun}
L_{j,sun}=\frac{1-e^{-\left(K_{d}+K_{b}\right) \cdot LAI}}{K_{d}+K_{b}}
\end{equation}
\begin{equation}\label{L_jsha}
L_{j,sha}=\frac{1-e^{-K_{d}\cdot LAI}}{K_{d}} - L_{j,sun}
\end{equation}
$K_{d}$代表散射光的消光系数。

第三,光合作用速率还受到卡尔文循环中蔗糖和淀粉的合成速率限制($A_{p,sun}$和$A_{p,sha}$),在阳叶和阴叶冠层尺度上,用冠层25 \textcelsius 最大羧化速率 ($V_{c \max25}$) 进行参数化,并且描述其受叶温和水分胁迫因子的调控:\\
C3植物:\\
\begin{equation}\label{A_e_a_sun}
A_{p,sun}=\frac{V_{c \max 25}}{2} \cdot \frac{1.8^{\frac{T_{{leaf }}-T_{o p}}{10}}}{1+e^{s_{2}\left(T_{{leaf }}-T_{{low }}\right)}} \cdot \beta_{sun} \cdot L_{sun,v}
\end{equation}
\begin{equation}\label{A_e_a_sha}
A_{p,sha}=\frac{V_{c \max 25}}{2} \cdot \frac{1.8^{\frac{T_{{leaf }}-T_{o p}}{10}}}{1+e^{s_{2}\left(T_{{leaf }}-T_{{low }}\right)}} \cdot \beta_{sha} \cdot L_{sha,v}
\end{equation}
C4植物:\\
\begin{equation}\label{A_e_b_sun}
A_{p,sun}=\frac{V_{c \max 25}}{5} \cdot 1.8^{\frac{T_{{leaf }}-298.16}{10}} \cdot \beta_{sun} \cdot L_{sun,v}
\end{equation}
\begin{equation}\label{A_e_b_sha}
A_{p,sha}=\frac{V_{c \max 25}}{5} \cdot 1.8^{\frac{T_{{leaf }}-298.16}{10}} \cdot \beta_{sha} \cdot L_{sha,v}
\end{equation}
三方面限制下的光合同化速率共用同一套温度响应常数 ($T_{op}$,$T_{low}$和$T_{high}$)以及温度敏感性参数($s_1$和$s_2$)


阳叶和阴叶冠层尺度的呼吸速率($R_{d,sun}$和$R_{d,sha}$)对温度的响应曲线可表示为$V_{c \max25}$的函数:
\begin{equation}\label{R_d1_sun}
R_{d,sun}=r_{{base }} \cdot V_{cmax 25} \cdot \frac{2.0^{\frac{T_{leaf}-T_{op}}{10}}}{1+e^{1.3 \cdot\left(T_{leaf}-T_{d m}\right)}} \cdot \beta_{sun} \cdot L_{sun,v}
\end{equation}
\begin{equation}\label{R_d1_sha}
R_{d,sha}=\tau_{{base }} \cdot V_{cmax 25} \cdot \frac{2.0^{\frac{T_{leaf}-T_{op}}{10}}}{1+e^{1.3 \cdot\left(T_{leaf}-T_{d m}\right)}} \cdot \beta_{sha} \cdot L_{sha,v}
\end{equation}
其中$T_{dm}$是叶呼吸的高温抑制温度常数,单位 K; $\tau_{base}$是基础呼吸速率系数,无量纲。

在方程\eqref{An1sun}和\eqref{An1sha}求解最小值的计算中,我们将三值最小问题的求解拆分为两个二值最小问题的求解:
\begin{equation}\label{min_Ac_Aj_Ae}
\min \left(A_{c}, A_{j}, A_{e}\right)=\min \left(\min \left(A_{c}, A_{j}\right), A_{e}\right)
\end{equation}
引入形状参数$\theta$,构造一元二次方程,将求解最小值问题转换成求一元二次方程较小根的问题,以此避免模拟中不同光合限制之间过渡转换时的光合同化速率突变 \citep{collatz1991,collatz1992}:
\begin{equation}\label{theta_cj}
\theta_{c j} \cdot A_{i1}^{2}-\left(A_{c}+A_{j}\right) A_{i1}+A_{c} A_{j}=0
\end{equation}
\begin{equation}\label{theta_cje}
\theta_{c j e} \cdot A_{i2}^{2}-\left(A_{i1}+A_{e}\right) A_{i2}+A_{i1} A_{e}=0
\end{equation}
其中形状参数$\theta_{cj}=0.877$,$\theta_{cje}=0.95$。$A_{i1}$为方程\eqref{theta_cj}的较小根,代表$A_c$和$A_j$的最小值。$A_{i2}$为方程\eqref{theta_cje}的较小根,代表$A_{i1}$和$A_e$的最小值。


\section{气体扩散方程和气孔导度模型}\label{气体扩散方程和气孔导度模型}
气孔是植被和大气相互作用的最重要通道,陆地生态系统碳水耦合很大程度取决于气孔的行为。CoLM在的光合同化速率和蒸腾速率计算中,考虑植被气孔的行为,结合气体扩散方程,刻画了碳水通量与浓度梯度之间的重要关系~\eqref{A_n2_sun} -~\eqref{ea_ei_sha}。CoLM将CO$_2$和水汽的气体扩散问题类比于电路问题进行建模,利用叶绿体细胞间、叶表和冠层大气的$\rm CO_2$浓度和水汽浓度梯度,来刻画环境对植被碳水通量的驱动力,如图~\ref{fig:叶片气孔光合作用导度模型示意图}。因此,在阳叶和阴叶冠层尺度上,分别存在关于$\rm CO_2$和水汽的气体扩散方程:

{
\begin{figure}[htbp]
\centering
\includegraphics{Figures/气孔导度和光合作用/叶片气孔光合作用导度模型示意图.png}
\caption{叶片气孔光合作用导度模型示意图。}
\label{fig:叶片气孔光合作用导度模型示意图}
\end{figure}
}

\begin{equation}\label{A_n2_sun}
A_{n,sun}=\left(c_{a}-c_{s,sun}\right) /\left(\frac{1.37}{G_{b}} p_{s}\right)=\left(c_{s,sun}-c_{i,sun}\right) /\left(\frac{1.6}{G_{s,sun}} p_{s}\right)
\end{equation}
\begin{equation}\label{A_n2_sha}
A_{n,sha}=\left(c_{a}-c_{s,sha}\right) /\left(\frac{1.37}{G_{b}} p_{s}\right)=\left(c_{s,sha}-c_{i,sha}\right) /\left(\frac{1.6}{G_{s,sha}} p_{s}\right)
\end{equation}
\begin{equation}\label{ea_ei_sun}
\left(e_{a}-e_{i}\right) /\left(\frac{1}{G_{b}}+\frac{1}{G_{s,sun}}\right)=\left(e_{s,sun}-e_{i}\right) / \frac{1}{G_{s,sun}}
\end{equation}
\begin{equation}\label{ea_ei_sha}
\left(e_{a}-e_{i}\right) /\left(\frac{1}{G_{b}}+\frac{1}{G_{s,sha}}\right)=\left(e_{s,sha}-e_{i}\right) / \frac{1}{G_{s,sha}}
\end{equation}
其中,$c_{i,sun}$和$c_{i,sha}$是阳叶和阴叶的胞间CO$_2$分压,$c_{s,sun}$和$c_{s,sha}$是阳叶和阴叶的叶表CO$_2$分压,$c_a$是冠层大气$\rm CO_2$分压,叶表水汽分压$e_{s,sun}$和$e_{s,sha}$,$e_a$是冠层大气水汽分压,$e_i$是胞间水汽分压,单位: Pa,通常假定为饱和水汽压,$G_b$是冠层尺度的叶边界层导度,$G_{s,sun}$和$G_{s,sha}$是阳叶和阴叶的冠层尺度叶气孔导度,单位: \unit{mol.m^{-2}.s^{-1}};单位: Pa。

气孔导度理论认为植被根据自身光合同化速率、环境大气水分亏缺以及土壤水分胁迫等因素,主动调控气孔导度。CoLM的气孔导度计算沿用Ball-Berry模型,但将尺度从叶片扩展到冠层,Ball-Berry模型根据冠层净光合同化速率 
($A_n$,单位: \unit{mol.CO_2.m^{-2}.s^{-1}})、叶表水汽压 ($e_{s,sun}$和$e_{s,sha}$,单位: Pa)、叶表二氧化碳分压 ($c_{s,sha}$和$c_{s,sha}$,单位: Pa) 
基于气孔导度的观测回归经验关系,在阳叶和阴叶上分别计算冠层尺度的气孔导度 ($G_{s,sun}$和$G_{s,sha}$, 单位: \unit{mol.CO_2.m^{-2}.s^{-1}}): 
\begin{equation}\label{rs_a1sun}
\frac{1}{r_{s,sun}}=G_{s,sun}=m \frac{A_{n,sun}}{c_{s,sun}} \frac{e_{s,sun}}{e_{i}} p_{s}+b\beta_{sun}
\end{equation}
\begin{equation}\label{rs_a1sha}
\frac{1}{r_{s,sha}}=G_{s,sha}=m \frac{A_{n,sha}}{c_{s,sha}} \frac{e_{s,sha}}{e_{i}} p_{s}+b\beta_{sha}
\end{equation}
$r_{s,sun}$和$r_{s,sha}$代表阳叶和阴叶冠层尺度的气孔阻抗,单位: \unit{s.m^{-1}},是气孔导度$G_{s,sun}$和$G_{s,sha}$的倒数;$m$是无量纲经验参数;$b$是最小气孔导度,
单位: \unit{mol.CO_2.m^{-2}.s^{-1}},$m$和$b$是观测拟合的经验系数;$e_i$是饱和水蒸气压,
是气温的函数,单位: Pa;$p_s$是大气压强,单位: Pa,$\beta_{sun}$和$\beta_{sha}$是阳叶和阴叶的水分胁迫因子,取值范围0$\sim$1。


因为气孔控制植被大气气体交换的同时受植被自身的光合能力和土壤水分水分亏缺的影响,需要联立光合作用模块方程、气体扩散方程,气孔导度模型和土壤水分胁迫方案,在阳叶和阴叶上分别求解冠层尺度气孔导度。
由方程~\eqref{A_n2_sun}和~\eqref{A_n2_sha} 可知:
\begin{equation}\label{cs_a1sun}
c_{s,sun}=c_{a}-\frac{1.37 A_{n,sun}}{G_{b}} p_{s}
\end{equation}
\begin{equation}\label{cs_a1sha}
c_{s,sha}=c_{a}-\frac{1.37 A_{n,sha}}{G_{b}} p_{s}
\end{equation}
由方程~\eqref{ea_ei_sun}和~\eqref{ea_ei_sha}可知:
\begin{equation}\label{e_s1sun}
e_{s,sun}=\left(\frac{e_{a}}{G_{s,sun}}+\frac{e_{i}}{G_{b}}\right) /\left(\frac{1}{G_{b}}+\frac{1}{G_{s,sun}}\right)
\end{equation}
\begin{equation}\label{e_s1sha}
e_{s,sha}=\left(\frac{e_{a}}{G_{s,sha}}+\frac{e_{i}}{G_{b}}\right) /\left(\frac{1}{G_{b}}+\frac{1}{G_{s,sha}}\right)
\end{equation}
将方程~\eqref{e_s1sun}和~\eqref{e_s1sha}分别代入方程~\eqref{rs_a1sun}和~\eqref{rs_a1sha}中,得到关于$G_{s,sun}$和$G_{s,sha}$的两个一元二次方程:
\begin{equation}\label{ei_cssun}
\frac{e_{i} c_{s,sun}}{m A_{n,sun} p_{s}} G_{s,sun}^{2}+\left(\frac{e_{i} c_{s,sun}\left(G_{b} -b \beta_{sun}\right)}{m A_{n,sun} p_{s}}-e_{i}\right) G_{s,sun}
-e_{a} G_{b}+\frac{b \beta_{sun} G_{b} e_{i} c_{s,sun}}{m A_{n,sun} p_{s}}=0
\end{equation}
\begin{equation}\label{ei_cssha}
\frac{e_{i} c_{s,sha}}{m A_{n,sha} p_{s}} G_{s,sha}^{2}+\left(\frac{e_{i} c_{s,sha}\left(G_{b} -b \beta_{sha}\right)}{m A_{n,sha} p_{s}}-e_{i}\right) G_{s,sha}
-e_{a} G_{b}+\frac{b \beta_{sha} G_{b} e_{i} c_{s,sha}}{m A_{n,sha} p_{s}}=0
\end{equation}
冠层尺度的气孔导度 ($G_{s,sun}$和$G_{s,sha}$) 的解即为一元二次方程\eqref{ei_cssun}和\eqref{ei_cssha}的正根,其中叶片表层$\mathrm{CO_2}$分压 ($c_{s,sun}$和$c_{s,sha}$) 由方程~\eqref{cs_a1sun}和~\eqref{cs_a1sha} 得出,$A_{n,sun}$和$A_{n,sha}$由光合作用模块公式~\eqref{An1sun}和~\eqref{An1sha} 得出,
但仍然包含未知变量阳叶和阴叶的胞间 $\mathrm{CO_2}$ 分压 ($c_{i,sun}$和$c_{i,sha}$),完整求解光合气孔模式还需根据$\mathrm{CO_2}$气体扩散方程~\eqref{A_n2_sun}和~\eqref{A_n2_sha} 得出:
\begin{equation}\label{ci_1sun}
c_{i,sun}=c_{s,sun}-\frac{1.6 A_{n,sun} p_{s}}{G_{s,sun}}
\end{equation}
\begin{equation}\label{ci_1sha}
c_{i,sha}=c_{s,sha}-\frac{1.6 A_{n,sha} p_{s}}{G_{s,sha}}
\end{equation}
气孔导度问题可以被抽象为求解方程组问题,即联立方程~\eqref{An1sun}, ~\eqref{An1sha}, \eqref{cs_a1sun}, \eqref{cs_a1sha}, \eqref{ei_cssun}, \eqref{ei_cssha}, \eqref{ci_1sun}和\eqref{ci_1sha},求解未知变量 $G_{s,sun}$, $G_{s,sha}$, $c_{i,sun}$, $c_{i,sha}$, $c_{s,sun}$, $c_{s,sha}$, $A_{n,sun}$和 $A_{n,sha}$。详细求解方案见(章节~\ref{气孔导度数值计算方案})。

\section{气孔导度数值计算方案}\label{气孔导度数值计算方案}
气孔导度问题的求解,由于光合作用Farquhar模型的非线性,CoLM采用牛顿迭代法对该问题进行数值求解。